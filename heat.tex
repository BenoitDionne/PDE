\chapter{Heat Equation} \label{ChapHeatEqu}

Let $\Omega$ be an open subset of $\displaystyle \RR^n$.  The
{\bfseries heat equation}\index{Heat Equation}
is the partial differential equation
\[
L(\VEC{x},t,D) u = \pdydx{u}{t} - c^2 \Delta u = 0
\]
on $\Omega$.  The constant $\displaystyle c^2$ represent the heat
conductivity in the region $\Omega$.  In this chapter $\Delta$ stands for
$\Delta_{\VEC{x}}$ unless otherwise specified.

The problem is to find $u$ that satisfies the heat equation, some
boundary conditions, and an initial condition.  The most frequent
boundary conditions are
\begin{itemize}
\item $u(\VEC{x},t)=f(\VEC{x})$ for $\VEC{x} \in \partial \Omega$ and
$t\geq 0$.  This means that the temperature is maintained to
$f(\VEC{x})$ at $\VEC{x} \in \partial \Omega$ for all time.
\item $\displaystyle \pdydx{u}{\VEC{\nu}}(\VEC{x},t) = g(\VEC{x})$ for
$\VEC{x} \in \partial \Omega$ and $t\geq 0$, where $\VEC{\nu}(\VEC{x})$
is the outward unit normal to the surface $\partial \Omega$ at
$\VEC{x}\in \partial \Omega$.  This means that the flow of heat
through $\VEC{x} \in \partial \Omega$ is $g(\VEC{x})$ for all time.
The most frequent situation is when $g=0$ on $\partial \Omega$;
namely, the boundary of $\Omega$ is insulated and no heat flow in or
out of $\Omega$.
\item $u(\VEC{x},t)=f(\VEC{x})$ on some parts of $\partial \Omega$ and
$\displaystyle \pdydx{u}{\VEC{\nu}}(\VEC{x},t) = g(\VEC{x})$ on the
other parts of $\partial \Omega$ for all $t\geq 0$.
\item $\displaystyle \pdydx{u}{\VEC{\nu}}(\VEC{x},t) + \alpha \left(
u(\VEC{x},t) - u_o(\VEC{x},t)\right) = 0$ for
$\VEC{x} \in \partial \Omega$ and $t\geq 0$,
where $\alpha$ is a constant.  This means that the heat flow
across $\partial \Omega$ at all time is proportional to the difference
between the temperature $u$ inside $\Omega$ and the temperature $u_o$
outside $\Omega$.  This is
{\bfseries Newton's Law}\index{Newton's Law of heating} of heating.
\end{itemize}
The initial condition is normally $u(\VEC{x},0) = v(\VEC{x})$ for
$\VEC{x} \in \Omega$ and represent the heat distribution in $\Omega$
initially.

\section{Cauchy Problem}  \label{SectHeatEqCauchyPr}

\subsection{One Dimensional Heat (Diffusion) Equation}

We consider
\begin{align}
\pdydx{u}{t} - c^2 \pdydxn{u}{x}{2} &= 0 \quad , \quad x\in \RR
\ \text{and} \ t > 0 \ , \label{heatODimE} \\
u(x,0) &= f(x) \quad , \quad x \in \RR \ . \label{heatODimIC}
\end{align}
We assume that $f$ is continuous and integrable on $\RR$.  This
implies in particular that $\displaystyle \lim_{x\to \pm \infty} f(x) = 0$.

The solution of the heat equation satisfies the following properties
which will guide us in the search of this solution.

\begin{enumerate}
\item Spatial translation of a solution $u$ of (\ref{heatODimE})
is still a solution of (\ref{heatODimE}); namely, $v(x,t) = u(x-y,t)$
satisfies (\ref{heatODimE}).
\item Any partial derivative of a solution $u$ of (\ref{heatODimE})
is still a solution of (\ref{heatODimE}).
\item Linear combination of solutions of (\ref{heatODimE}) is a
solution of (\ref{heatODimE}).
\item If $K$ is a solution of (\ref{heatODimE}), then
\[
v(x,t) = \int_{-\infty}^\infty K(x-y,t) f(y) \dx{y}
\]
is also a solution of (\ref{heatODimE}) if proper convergence
conditions are assumed.
\item If $a>0$, then $v(x,t) = u(\sqrt{a}\,x, at)$ is a solution of
(\ref{heatODimE}) if $u$ is a solution of (\ref{heatODimE}).  The
solution of (\ref{heatODimE}) is invariant under the transformation
$(x,t) \mapsto (\sqrt{a}\,x, at)$.
\end{enumerate}

To solve the heat equation, we first find the solution of the
following problem.
\begin{align}
\pdydx{Q}{t} - c^2 \pdydxn{Q}{x}{2} &= 0 \quad , \quad x\in \RR
\ \text{and} \ t>0 \ , \label{heatODimQE} \\
Q(x,0) &= \begin{cases}
1 & \quad \text{if} \ x > 0 \\
0 & \quad \text{if} \ x < 0
\end{cases} \label{heatODimQC}
\end{align}
The solution that we will find has a singularity at the origin.

Because the solution of (\ref{heatODimQE}) is invariant under the
transformation $(x,t) \mapsto (\sqrt{a}\,x, at)$, we seek a solution
of the form
$\displaystyle Q(x,t) = q\left(\frac{x}{2c\sqrt{t}}\right)$.
We introduce the factor $\displaystyle \frac{1}{2c}$ for aesthetic
reasons as it will become apparent later.  Hence, (\ref{heatODimQE}) becomes
\begin{equation} \label{heatODimqE}
q''(s) + 2 s q'(s) = 0 \ ,
\end{equation}
where $s= x/(2c\sqrt{t})$.  The general solution of (\ref{heatODimqE})
is
\[
q(s) = C_1 \int_0^s e^{-r^2} \dx{r} + C_2
\]
for some constants $C_1$ and $C_2$.  Hence, we get
\begin{equation} \label{heatODimQSone}
Q(x,t) = C_1 \int_0^{x/(2c\sqrt{t})} e^{-r^2}\dx{r} + C_2
\end{equation}
To determine $C_1$ and $C_2$, we use the initial condition
(\ref{heatODimQC}).  We require that
\[
1 = \lim_{t\to 0^+} Q(x,t) = C_1 \int_0^{\infty} e^{-r^2} \dx{r} + C_2
= C_1 \frac{\sqrt{\pi}}{2} + C_2
\]
for $x>0$ and
\[
0 = \lim_{t\to 0^+} Q(x,t) = C_1 \int_0^{-\infty} e^{-r^2} \dx{r} + C_2
= - C_1 \frac{\sqrt{\pi}}{2} + C_2
\]
for $x<0$.  We get $C_1 = 1/\sqrt{\pi}$ and $C_2 = 1/2$.  Thus
\[
Q(x,t) = \begin{cases} \displaystyle
\frac{1}{2} + \frac{1}{\sqrt{\pi}} \int_0^{x/(2c\sqrt{t})}
e^{-r^2}\dx{r} & \quad \text{if} \ x\in \RR, t>0 \\
1 & \quad \text{if} \ x>0 , t = 0\\
1/2 & \quad \text{if} \ x = 0 \\
0 & \quad \text{if} \ x< 0 , t = 0
\end{cases}
\]

Let
\[
K(x,t) = \pdydx{Q}{x}(x,t) = \frac{1}{2c\sqrt{\pi t}} e^{-x^2/(4c^2t)}
\]
for $x\in \RR$ and $t > 0$.  We show that
\begin{equation} \label{heatODimSolH}
u(x,t) = \begin{cases}
\displaystyle \int_{-\infty}^{\infty} K(x-y,t) f(y) \dx{y}
& \quad \text{if} \ x\in \RR, t>0 \\
f(x) & \quad \text{if} \ x \in \RR, t = 0 
\end{cases}
\end{equation}
is the solution of (\ref{heatODimE}) and (\ref{heatODimIC}).  The
second and fourth properties for the solution of (\ref{heatODimE})
mentioned above implies that $u$ satisfies (\ref{heatODimE}).  For the
fourth properties, we note that for every $T>0$ there exists a constant
$C$ such that
$\displaystyle |K(x-y,t) f(y)| \leq C |f(y)|$,
$\displaystyle \left| \pdydx{K}{t}(x-y,t) f(y)\right| \leq C |f(y)|$,
$\displaystyle \left| \pdfdx{K}{x}(x-y,t) f(y)\right| \leq C |f(y)|$
and
$\displaystyle \left| \pdfdxn{K}{x}{2}(x-y,t) f(y)\right| \leq C |f(y)|$
for $\VEC{x} \in \RR$ and $t \geq T > 0$.  Since
$\displaystyle f \in L^1(\RR)$, we
may interchange the integral in the definition of $u$ with the
derivatives of $u$ with respect to $t$ and $x$ (see \cite{Flem}).

To prove that $u$ is a continuous function for $x\in \RR$ and $t\geq 0$,
we only have to prove that $u$ is continuous at $(x,0)$ for $x \in \RR$
because $u$ is continuous for $\RR \times ]0,\infty[$ follows from
$(x,t) \mapsto K(x,t)f(x)$ being continuous on $\RR \times ]0,\infty[$.
We first note that
$\displaystyle \int_{-\infty}^{\infty} K(y,t) \dx{y} = 1$ for all $t>0$.
Let $\displaystyle M = \sup_{x \in \RR} |f(x)|$.  We have that
$M < \infty$ because $f$ is a continuous function in $\displaystyle L^1(\RR)$.
Given $x \in \RR$ and $\epsilon>0$, choose $\delta > 0$ such that
$|f(y) - f(x)| < \epsilon/2$ for $|y-x|< \delta$.
Then, choose $\delta$ smaller if needed such that
\[
\int_{|x|>\delta} K(x,t) \dx{x}
= \frac{1}{c \sqrt{\pi t}} \int_{\delta}^\infty e^{-x^2/(4c^2t)} \dx{x}
= \frac{2}{\sqrt{\pi}} \int_{\delta/(2c\sqrt{t})}^\infty e^{-s^2} \dx{s}
< \frac{\epsilon}{4M}
\]
for $0<t<\delta$.  Hence,
\begin{align*}
&|u(z,t) - u(x,0)|
= \left| \int_{-\infty}^{\infty} K(z-y,t) (f(y) - f(x)) \dx{y} \right| \\
&\quad \leq \left| \int_{|z-y|<\delta} K(z-y,t) (f(y) - f(x)) \dx{y} \right|
+ \left| \int_{|z-y|\geq \delta} K(z-y,t) (f(y) - f(x)) \dx{y} \right| \\
&\quad \leq \int_{|z-y|<\delta} |K(z-y,t)|\, |f(y) - f(x)| \dx{y} 
+ \int_{|z-y| \geq \delta} |K(z-y,t)|\,|f(y) - f(x)| \dx{y} \\
&\quad \leq \frac{\epsilon}{2}
\underbrace{\int_{|z-y|<\delta} |K(z-y,t)| \dx{y}}_{\leq 1}
+ 2 M \underbrace{\int_{|z-y|>\delta} |K(z-y,t)| \dx{y}}_{\epsilon/(4M)}
\leq \epsilon
\end{align*}
for $|z-x| < \delta$ and $0<t<\delta$.

\begin{rmk}
If $f$ is differentiable and $f' \in L^1(\RR)$, there is an easy to prove that
$\displaystyle \lim_{t\to 0^+} u(x,t) = f(x)$.  Note that this is not
enough to prove the continuity of $u$ at $(x,0)$ as a function from
$\RR \times [0,\infty[$ to $\RR$ because we approach $(x,0)$ in only
one direction.

Using integration by parts, we have for $t>0$ that
\begin{align*}
u(x,t) &= \int_{-\infty}^{\infty} \pdydx{Q}{x}(x-y,t) f(y) \dx{y}
= -\int_{-\infty}^{\infty} \left(\pdfdx{Q(x-y,t)}{y}\right) f(y) \dx{y} \\
&= \int_{-\infty}^{\infty} Q(x-y,t) f'(y)\dx{y}
- Q(x-y,t) f(y) \bigg|_{-\infty}^{\infty}
\end{align*}
Since $\displaystyle \lim_{x\to \pm\infty}f(x) = 0$, we get
\[
u(x,t) = \int_{-\infty}^{\infty} Q(x-y,t) f'(y)\dx{y} \ .
\]
Hence,
\[
\lim_{t\to 0^+} u(x,t)
= \int_{-\infty}^{\infty} Q(x-y,0) f'(y)\dx{y} \ .
= \int_{-\infty}^x f'(y) \dx{y} = f(y)\bigg|_{-\infty}^x = f(x) \ ,
\]
where we have again used $\displaystyle \lim_{x\to -\infty}f(x) = 0$.
We should note that $|Q(x-y,t) f'(y)| < |f'(y)|$ for all $x$ and
$t\geq 0$.  Since $\displaystyle f' \in L'(\RR)$, we may use the
Lebesgue Dominated Convergence Theorem to justify the first equality
in the previous expression.
\end{rmk}

Explicitly, (\ref{heatODimSolH}) is
\[
u(x,t) = \frac{1}{2c\sqrt{\pi t}} \int_{-\infty}^{\infty}
e^{-(x-y)^2/(4c^2t)} f(y) \dx{y}
\]
for $t>0$.

The function $(x,t) \mapsto K(x,t)$ is know under several names:
{\bfseries source function}\index{Source Function},
{\bfseries Green's function}\index{Green's Function},
{\bfseries fundamental solution}\index{Fundamental Solution}, etc.  As
$t$ converges toward $0$, the function $x \mapsto K(x,t)$ behaves like
the Dirac delta function at the origin.  From the solution of the heat
equation, we observe that if we have a hot spot near the origin at
$t=0$, then the heat will spread evenly as $t$ increases.

The heat equation can be solve ``in some sense'' even when the
initial condition is not a continuously function.  This
will be the subject of a section later on.  First, we consider the
non-homogeneous wave equation.

\subsection{Non-homogeneous Heat Equation, Diffusion with a Source}

We consider
\begin{align}
\pdydx{u}{t} - c^2 \pdydxn{u}{x}{2} &= g(x,t) \quad , \quad
x\in \RR \ \text{and} \ t > 0 \ , \label{heatODimNHE} \\
u(x,0) &= f(x) \quad , \quad x \in \RR \ . \label{heatODimNHIC}
\end{align}
We assume that $f$ is continuous and integrable on $\RR$.  In
particular, this implies that $\displaystyle \lim_{x\to \pm \infty} f(x) = 0$.
Moreover, we assume that $g$ is also continuously differentiable (at
least with respect to $x$) on $\RR \times ]0,\infty[$ and integrable on
$\RR \times [0,T]$ for all $T>0$.
Again, this implies that $\displaystyle \lim_{x\to \pm \infty} g(x,t) = 0$
for all $t$.   With these assumptions, all the
following integrals involving $g$ and $\displaystyle \pdydx{g}{x}$
converge and we can interchange limits and integrals using the
Lebesgue Dominated Convergence Theorem.

We show that the solution of (\ref{heatODimNHE}) and
(\ref{heatODimNHIC}) is
\[
u(x,t) = \int_{-\infty}^{\infty} K(x-y,t) f(y) \dx{y}
+ \int_0^t \int_{-\infty}^{\infty} K(x-y,t-s) g(y,s) \dx{y}\dx{s}
\]
for $t>0$.  The first integral is the solution of the homogeneous heat
equation (\ref{heatODimE}) and (\ref{heatODimIC}), and the second
integral is a solution of the non-homogeneous heat equation
\begin{align}
\pdydx{u}{t} - c^2 \pdydxn{u}{x}{2} &= g(x,t) \quad , \quad
x\in \RR \ \text{and} \ t > 0 \ , \label{heatODimNHEG}\\
u(x,0) &= 0 \quad ,\quad x \in \RR \ . \label{heatODimNHICG}
\end{align}

We only have to prove that
\[
v(x,t) = \int_0^t \int_{-\infty}^{\infty} K(x-y,t-s) g(y,s) \dx{y}\dx{s}
\]
satisfies (\ref{heatODimNHEG}) because (\ref{heatODimNHICG}) is
obviously satisfied by $v$.

Since
\[
\pdydx{K}{t}(x-y,t-s) = c^2 \pdydxn{K}{x}{2}(x-y,t-s) \ ,
\]
we have
\begin{align*}
\pdydx{v}{t} &= \int_0^t \int_{-\infty}^{\infty}
\pdydx{K}{t}(x-y, t-s) g(y,s) \dx{y}\dx{s} 
+ \lim_{s\to t^-} \int_{-\infty}^{\infty} K(x-y,t-s)g(y,s) \dx{y} \\
& = \int_0^t \int_{-\infty}^{\infty} c^2
\pdydxn{K}{x}{2}(x-y, t-s) g(y,s) \dx{y}\dx{s} + g(x,t)
= c^2 \pdydxn{v}{x}{2}(x,t) + g(x,t)
\end{align*}
for $t>0$ and $x\in \RR$ because
\begin{align*}
& \lim_{s\to t^-} \int_{-\infty}^{\infty} K(x-y,t-s)g(y,s) \dx{y}
= \lim_{\epsilon \to 0^+} \int_{-\infty}^{\infty}
K(x-y,\epsilon)g(y,t-\epsilon) \dx{y} \\
& \qquad = -\lim_{\epsilon \to 0^+} \int_{-\infty}^{\infty}
\left(\pdfdx{Q(x-y,\epsilon)}{y}\right) g(y,t-\epsilon) \dx{y} \\
& \qquad = \lim_{\epsilon \to 0^+} \left( \int_{-\infty}^{\infty} Q(x-y,\epsilon)
\pdydx{g}{y}(y,t-\epsilon) \dx{y}
- Q(x-y,\epsilon) g(y,t-\epsilon) \bigg|_{y=-\infty}^{\infty} \right) \\
& \qquad = \int_{-\infty}^{\infty} Q(x-y,0) \pdydx{g}{y}(y,t) \dx{y}
= \int_{-\infty}^x \pdydx{g}{y}(y,t) \dx{y}
= g(y,t) \bigg|_{y=-\infty}^x = g(x,t) \ ,
\end{align*}
where we have used several times that
$\displaystyle \lim_{x\to \pm\infty} g(x,t) = 0$ for all $t>0$.

\subsection{Higher Dimensional Heat Equation}

Consider the Cauchy problem
\begin{align}
\pdydx{u}{t}(\VEC{x},t) - \Delta u(\VEC{x},t) &= 0 \quad , \quad
\VEC{x} \in \RR^n \ \text{and} \ t> 0 \ , \label{heat_cauchyA} \\
u(\VEC{x},0) &= f(\VEC{x}) \quad , \quad \VEC{x} \in \RR^n \ ,
\label{heat_cauchyB} 
\end{align}
where $f:\RR^n\rightarrow \RR$ is a function to be specified later.

\begin{rmk}
The hyperplane $t=0$ is a characteristic surface because the outward
unit normal to this hyperplane is $\VEC{\nu}(\VEC{x},0) = (\VEC{0},-1)$ for
all $\displaystyle \VEC{x} \in \RR^n$, where $\VEC{0}$ is the origin
in $\displaystyle \RR^n$, and 
the principal symbol for the heat equation is
\[
Q((\VEC{x},t), (\VEC{y},s)) = \sum_{j=1}^n y_j^2 \ .
\]
Hence, $\displaystyle Q((\VEC{x},0), \VEC{\nu}(\VEC{x},0)) = 0$ for
all $\displaystyle \VEC{x} \in \RR^n$.
\end{rmk}

Let us assume that $\displaystyle f \in \SS(\RR^n)$.  Moreover, in addition of
$u$ being of class $\displaystyle C^2$ on
$\displaystyle \RR^n \times ]0,\infty[$ and
continuous on $\displaystyle \RR^n \times [0,\infty[$, let us assume that
$\VEC{x} \mapsto u(\VEC{x},t)$ is and
$\displaystyle \VEC{x} \mapsto \pdydx{u}{t}(\VEC{x},t)$ are absolutely
integrable on $\displaystyle \RR^n$ for all $t \geq 0$.  We may then use Fourier
transform with respect to $\VEC{x}$ to transform (\ref{heat_cauchyA})
and (\ref{heat_cauchyB}) into
\begin{align*}
\pdydx{\hat{u}}{t}(\VEC{y},t) + \|\VEC{y}\|^2\, \hat{u}(\VEC{y},t)
&= 0 \quad , \quad \VEC{y} \in \RR^n \ \text{and} \ t \geq 0 \ , \\
\hat{u}(\VEC{y},0) &= \hat{f}(\VEC{y}) \quad , \quad \VEC{y} \in \RR^n \ .
\end{align*}
The solution of this initial value problem is
\[
\hat{u}(\VEC{y},t) = \hat{f}(\VEC{y})\, e ^{-\|\VEC{y}\|^2 t}
\]
for $\displaystyle \VEC{y} \in \RR^n$ and $t \geq 0$.
Using the Inverse Fourier transform, we get from
Proposition~\ref{FTLofPs} that
\begin{align} \label{heat_sol_cauch1}
u(\VEC{x},t) = (f \ast K_t)(\VEC{x})
\end{align}
for $\displaystyle \VEC{x} \in \RR^n$ and $t \geq 0$, 
where the convolution is with respect to $\VEC{x}$ only and
$K_t$ is the inverse Fourier transform of
$\displaystyle \VEC{y} \mapsto (2\pi)^{-n/2} e ^{-\|\VEC{y}\|^2 t}$
\footnote{Recall that the integral used in
the definition of convolution in Section~\ref{sectionFT} was
with respect to the Lebesgue measure $m$ divided by
$(2\pi)^{n/2}$.}.  If we use (\ref{FTexpx2eq1}) in
Example~\ref{FTexpx2} with $a = 1/(4t)$, we find that
\[
K_t(\VEC{x}) = (4\pi t)^{-n/2} e^{-\|\VEC{x}\|^2/(4t)}
\]
for $\displaystyle \VEC{x} \in \RR^n$ and $t > 0$.  Proceeding as we
did for the one dimensional heat equation, we can prove that
\[
u(\VEC{x},t) = (f \ast K_t)(\VEC{x})
= (4\pi t)^{-n/2} \int_{\RR^n} e^{-\|\VEC{x}-\VEC{y}\|^2/(4t)}
f(\VEC{y}) \dx{\VEC{y}}
\]
is a classical solution of the heat equation given by (\ref{heat_cauchyA})
and (\ref{heat_cauchyB}).

This discussion motivates the introduction of the function
\[
K(\VEC{x},t) =
\begin{cases}
\displaystyle (4\pi t)^{-n/2} e^{-\|\VEC{x}\|^2/(4t)} & \quad 
\text{if} \quad t>0 \\
0 &  \quad \text{if} \quad t\leq 0
\end{cases}
\]
$K(\VEC{x},t)$ has many names; it is called
{\bfseries Gaussian kernel}\index{Heat Equation!Gaussian Kernel},
{\bfseries Gauss-Weierstrass kernel}
\index{Heat Equation!Gauss-Weierstrass Kernel} or
{\bfseries heat kernel}\index{Heat Equation!Heat Kernel}.

Can we used (\ref{heat_sol_cauch1}) even when
$\displaystyle f\not\in \SS(\RR^n)$?
We answer this question in the next theorem.

\begin{theorem} \label{heat_solution_hs}
Assume that $\displaystyle f \in L^p(\RR^n)$, where $1\leq p \leq \infty$.  Then
\begin{align} \label{heat_sol_cauch}
u(\VEC{x},t) = (f \ast K_t)(\VEC{x}) \quad , \quad
(\VEC{x},t) \in \RR^n \times [0,\infty[ \  , 
\end{align}
is a solution of (\ref{heat_cauchyA}).

If $f$ is bounded and continuous, then $u$ can be extended
continuously to $\RR^n \times [0,\infty[$ with
$u(\VEC{x},0) = f(\VEC{x})$ for $\VEC{x} \in \RR^n$.

If $f \in L^p(\RR^n)$, where $1\leq p < \infty$, then
$v_t(\VEC{x}) = u(\VEC{x},t)$ for $\VEC{x} \in \RR^n$ satisfies
$v_t \rightarrow f$ in $L^p(\RR^n)$ as $t\rightarrow 0$.
\end{theorem}

\begin{proof}
The proof is similar to the proof of Theorem~\ref{laplace_half_plane_proof}.

\stage{i} First, we prove that $u$ is well defined.
We have that $\displaystyle K_t \in L^q(\RR^n)$ for
$1\leq q \leq \infty$ and $t>0$.
Hence,  $\displaystyle \VEC{y} \mapsto f(\VEC{y})\,K_t(\VEC{x}-\VEC{y})$
for $\displaystyle \VEC{y} \in \RR^n$ is in
$\displaystyle L^1(\RR^n)$.  For instance, if
$1<p<\infty$ we have
\[
\int_{\RR^n} \left| f(\VEC{y})\, K_t\left(\VEC{x} - \VEC{y}\right) \right|
\dx{\VEC{y}}
\leq
\left( \int_{\RR^n} \left| f(\VEC{y}) \right|^p \dx{\VEC{y}} \right)^{1/p} \,
\left( \int_{\RR^n} \left|K_t\left(\VEC{x} - \VEC{y}\right) \right|^q
\dx{\VEC{y}} \right)^{1/q} < \infty
\]
for $\displaystyle \frac{1}{p} + \frac{1}{q} = 1$.
Thus $\displaystyle u:\RR^n\times ]0,\infty[ \rightarrow \RR$ is well defined.

\stage{ii} We prove that $u$ satisfies the heat equation.
Proceeding as before, we can show that
$\displaystyle \VEC{y} \mapsto f(\VEC{y})\, \left(
D^\alpha_{\VEC{x}} \pdfdxn{K_t\left(\VEC{x} - \VEC{y}\right)}{t}{k} \right)$
for $\displaystyle \VEC{y} \in \RR^n$ is in
$\displaystyle L^1(\RR^n)$ for $t>0$ whatever the
choice of $k\geq 0$ and the multi-index
$\displaystyle \alpha \in \NN^n$.  Hence, we
may interchange the integral with respect to $\VEC{y}$ with the partial
derivatives of any order with respect to the $x_i$'s and $t$.

Therefore,
\[
\left(\pdydx{}{t} - \Delta_{\VEC{x}} \right)u(\VEC{x},t)
= \int_{\RR^n} f(\VEC{y})\,\left(\pdydx{}{t} - \Delta_{\VEC{x}}\right)
K_t\left(\VEC{x} - \VEC{y}\right) \dx{\VEC{y}} = 0 \quad , \quad
(\VEC{x},t) \in \RR^n\times ]0.\infty[ \  ,
\]
because
$\displaystyle \left(\pdydx{}{t} - \Delta_{\VEC{x}}\right) K_t\left(\VEC{x}
- \VEC{y}\right) = 0$ for all $\VEC{x}\in \RR^n$ and $t>0$.

\stage{iii} For the boundary conditions, we first note that
\begin{align}
\int_{\RR^n} K_t(\VEC{y}) \dx{\VEC{y}}
&= (2\pi)^{n/2} \hat{K_t}(\VEC{0}) = e^{-\|\VEC{y}\|^2 t}\big|_{\VEC{y}=\VEC{0}}
= 1
\label{heat_convol1}
\intertext{and}
K_t(\VEC{x}) &= t^{-n/2} K_1\left(t^{-1/2} \VEC{x}\right)
\label{heat_convol2}
\end{align}

If $\displaystyle f\in L^p(\RR^n)$ with $1\leq p < \infty$, it follows from
Proposition~\ref{distr_limit_convol} that
$f \ast K_t \rightarrow f$ in $\displaystyle L^p(\RR^n)$ as $t \rightarrow 0$.

If $f$ is continuous and bounded on $\displaystyle \RR^n$, then $f$ is uniformly
continuous on any bounded open set $V$ because $\overline{V}$ is a
compact subset of $\displaystyle \RR^n$.  It follows from
Proposition~\ref{distr_limit_convol} that
$f \ast K_t \rightarrow f$ uniformly on compact subsets of
$\displaystyle \RR^n$ as $t \rightarrow 0$.  In particular,
$u(\VEC{x},0) = f(\VEC{x})$ for $\displaystyle \VEC{x} \in \RR^n$ defines a
continuous extension of $u$ to $\displaystyle \RR^n \times [0,\infty[$.
\end{proof}

\begin{prop}
Assume that $\displaystyle f \in L^p(\RR^n)$, where $1\leq p \leq \infty$.  The
solution $u$ of (\ref{heat_cauchyA}) given in (\ref{heat_sol_cauch}) is
of class $\displaystyle C^\infty(\RR^n\times ]0,\infty[)$.
\end{prop}

\begin{proof}
We have seen in the proof of the previous theorem that we may
interchange integration with respect to $\VEC{y}$ in
\[
u(\VEC{x},t) = 
\int_{\RR^n} f(\VEC{y})\,K_t\left(\VEC{x} - \VEC{y}\right) \dx{\VEC{y}}
\]
with partial derivatives of any order with respect to the $x_i$'s
and $t$.  Since $\displaystyle K \in C^\infty(\RR^n \times ]0,\infty[)$,
we get that $\displaystyle u \in C^\infty(\RR^n \times ]0,\infty[)$.
\end{proof}

\begin{rmkList} \label{heat_smooth_sol}
\begin{enumerate}
\item The solution of (\ref{heat_cauchyA}) and (\ref{heat_cauchyB}) given by
(\ref{heat_sol_cauch}) smooths out any initial condition in
$\displaystyle L^p(\RR^n)$.
\item We may relax the condition on $f$ in the previous theorem if we assume
that $\displaystyle |f(\VEC{x})| \leq C e^{\|\VEC{x}\|^2/(4T)}$ for
$\displaystyle \VEC{x} \in \RR^n$, and some positive constants $C$ and
$T$.  One can show that the function
\begin{align*}
u: \RR^n \times ]0,T[ &\rightarrow \RR \\
(\VEC{x}, t) &\mapsto (f\ast K_t)(\VEC{x})
\end{align*}
is a solution of the heat equation on $\displaystyle \RR^n \times ]0,T[$.  In
addition, if $f$ is also continuous, then
$u(\cdot,t) \rightarrow f(\cdot)$ uniformly on compact subsets of
$\displaystyle \RR^n$ as $t\rightarrow 0$.
\item If no conditions are imposed on the growth at infinity of the
solution $u$ of (\ref{heat_cauchyA}) and (\ref{heat_cauchyB}), then
the solution is not unique.  Consider
\[
g(t) =
\begin{cases}
e^{-1/t^2} & \quad \text{if} \ t\neq 0 \\
0 & \quad \text{if} \ t=0
\end{cases}
\]
The function $g$ is in $C^\infty(\RR)$.  The function
\begin{equation} \label{heat_not_uno}
u(x,t) = \sum_{k=0}^\infty \frac{g^{(k)}(t)}{(2k)!}\,x^{2k}
\end{equation}
for $x\in \RR$ and  $t\in \RR$ is a solution of the heat equation
$\displaystyle \pdydx{u}{t} - \pdydxn{u}{x}{2} = 0$.  This follows
from the fact that the series (\ref{heat_not_uno}),
$\displaystyle \sum_{k=0}^\infty \frac{g^{(k+1)}(t)}{(2k)!}\,x^{2k}$
and $\displaystyle \sum_{k=1}^\infty \frac{g^{(k)}(t)}{(2k-2)!}\,x^{2k-2}$
converge uniformly on compact sets.  Hence, we may interchange
summation and partial derivatives in (\ref{heat_not_uno}). 
Since $g^{(k)}(0)=0$ for all $k$, $u(x,0)=0$ for all
$x\in \RR$ but $u$ is not the trivial solution of the heat equation.
\end{enumerate}
\end{rmkList}

The following theorem gives conditions on the growth at infinity of
the solution $u$ of (\ref{heat_cauchyA}) and (\ref{heat_cauchyB}) to
provide unicity of solutions.

\begin{prop}
Suppose that
$\displaystyle u \in C(\RR^n \times [0,\infty[) \cap
C^2(\RR^n \times ]0,\infty[)$
is a solution of
$\displaystyle \pdydx{u}{t} - \Delta u = 0$ on $\RR^n \times ]0,\infty[$
and $u(\VEC{x},0)=0$ for $\displaystyle \VEC{x} \in \RR^n$.  If, for every
$\epsilon >0$, there exists $C>0$ such that
$\displaystyle |u(\VEC{x},t)| \leq C e^{\epsilon \|\VEC{x}\|^2}$
and
$\displaystyle \|\graD_{\VEC{x}} u(\VEC{x},t)\| \leq C e^{\epsilon \|\VEC{x}\|^2}$
for all $\VEC{x}$ and $t$, then $u \equiv 0$.
\end{prop}

\begin{proof}
Given $\displaystyle (\VEC{x}_0, t_0) \in \RR^n \times ]0,\infty[$, we
prove that $u(\VEC{x}_0,t_0) = 0$.

We defined two functions, $\displaystyle f:\RR^n\times ]0,\infty[ \to \RR$ and
$\displaystyle g:\RR^n\times ]-\infty,t_0[ \to \RR$, as follows.
$f(\VEC{x},t) = u(\VEC{x},t)$ for
$\displaystyle (\VEC{x},t)\in \RR^n\times ]0,\infty[$
and $g(\VEC{x},t) = K(\VEC{x}-\VEC{x}_0, t_0-t)$ for
$\displaystyle (\VEC{x},t)\in \RR^n\times ]-\infty, t_0[$.

We have that $\displaystyle \pdydx{f}{t} - \Delta f = 0$
on $\displaystyle \RR^n \times ]0,\infty[$ and
$\displaystyle \pdydx{g}{t} + \Delta g = 0$ on $\RR^n \times ]-\infty,t_0[$.

Given $0< a < b < t_0$ and $r>0$, let
$\displaystyle \Omega =
\left\{ (\VEC{x},t) : \|\VEC{x}\| < r, \ a<t<b \right\}$.
Moreover, let $F:\RR^n \times ]0,t_0[ \to \RR^{n+1}$ be the vector
field defined by
\[
F(\VEC{x},t) = \sum_{j=1}^n
\left( f(\VEC{x},t) \pdydx{g}{x_j}(\VEC{x},t) - g(\VEC{x},t)
\pdydx{f}{x_j}(\VEC{x},t) \right) \VEC{e}_j
+ f(\VEC{x},t)g(\VEC{x},t) \VEC{e}_{n+1} \ .
\]
Since
\begin{align*}
\diV_{\VEC{x},t} F(\VEC{x},t)
&= \sum_{j=1}^n \pdfdx{
\left( f(\VEC{x},t) \pdydx{g}{x_j}(\VEC{x},t) - g(\VEC{x},t)
\pdydx{f}{x_j}(\VEC{x},t) \right) }{x_j}
+ \pdfdx{ \left(f(\VEC{x},t)g(\VEC{x},t)\right) }{t} \\
&= 
f(\VEC{x},t) \left( \pdydx{g}{t}(\VEC{x},t) + \Delta g(\VEC{x},t) \right)
+ g(\VEC{x},t) \left( \pdydx{f}{t}(\VEC{x},t) - \Delta f(\VEC{x},t) \right)
= 0
\end{align*}
in $\displaystyle \RR^n \times ]0,t_0[$, we get from the divergence theorem that
\begin{equation} \label{heat_unicA}
\begin{split}
0 &= \int_{\Omega} \diV_{\VEC{x},t}\cdot F(\VEC{x},t) \dx{\VEC{x}}\dx{t} 
= \int_{\partial \Omega} F \cdot \nu \dx{S} \\
&= \int_{\|\VEC{x}\|\leq r} f(\VEC{x},b)g(\VEC{x},b) \dx{\VEC{x}}
- \int_{\|\VEC{x}\|\leq r} f(\VEC{x},a)g(\VEC{x},a) \dx{\VEC{x}} \\
&\quad + \int_{\|\VEC{x}\|=r} \int_a^b
\sum_{j=1}^n \left( f(\VEC{x},t) \pdydx{g}{x_j}(\VEC{x},t) - g(\VEC{x},t)
\pdydx{f}{x_j}(\VEC{x},t) \right) \frac{x_j}{r} \dx{t}\dx{S_{\VEC{x}}} \ .
\end{split}
\end{equation}
The last integral in (\ref{heat_unicA}) converges to $0$ as
$r\to \infty$ because, given $0 < \epsilon < 1/(4(t_0-a))$, there
exist $C>0$ such that
\begin{align*}
&\int_{\|\VEC{x}\|=r} \int_a^b
\left| f(\VEC{x},t) \pdydx{g}{x_j}(\VEC{x},t) \right|
\underbrace{\left|\frac{x_j}{r}\right|}_{\leq 1} \dx{t}\dx{S_{\VEC{x}}} \\
& \quad
\leq C e^{\epsilon r^2}\, \frac{2\pi}{(4\pi (t_0-b))^{1+n/2}}
\, \left(r+\|\VEC{x}_0\|\right)
\, e^{-(r-\|\VEC{x}_0\|)^2/(4(t_0-a))} (b-a) r^{n-1} \omega_n \to 0 \
\text{as} \ r \to \infty
\end{align*}
and
\begin{align*}
&\int_{\|\VEC{x}\|=r} \int_a^b
\left| g(\VEC{x},t) \pdydx{f}{x_j}(\VEC{x},t) \right|
\underbrace{\left|\frac{x_j}{r}\right|}_{\leq 1} \dx{t}\dx{S_{\VEC{x}}} \\
& \quad
\leq C e^{\epsilon r^2}\, \frac{1}{(4\pi (t_0-b))^{n/2}}
\, e^{-(r-\|\VEC{x}_0\|)^2/(4(t_0-a))} (b-a) r^{n-1} \omega_n \to 0 \
\text{as} \ r \to \infty \ .
\end{align*}
Hence,
$r \to \infty$ in (\ref{heat_unicA}) yields
\begin{equation} \label{heat_unicB}
\big(u(\cdot, b) \ast K_{t_0-b}(\cdot)\big)(\VEC{x}_0)
- \big(u(\cdot, a) \ast K_{t_0-a}(\cdot)\big)(\VEC{x}_0) = 0
\end{equation}
because
$g(\VEC{x},t) = K(\VEC{x}-\VEC{x}_0, t_0-t) = K(\VEC{x}_0-\VEC{x}, t_0-t)$.

\stage{i} We claim that
$\displaystyle \big(u(\cdot, a) \ast K_{t_0-a}(\cdot)\big)(\VEC{x}_0) \to 0$
as $a \to 0^+$.  We have that
\[
\lim_{a\to 0^+} u(\VEC{x},a) K(\VEC{x}-\VEC{x}_0, t_0 - a) =
u(\VEC{x},0) K(\VEC{x}-\VEC{x}_0, t_0) = 0
\]
for all $\displaystyle \VEC{x} \in \RR^n$ because $u(\VEC{x},0) = 0$
for all $\VEC{x} \in \RR^n$.  Moreover, given
$\epsilon < 1/(4t_0)$, there exist $C>0$ such that
\[
\left| u(\VEC{x},a) K(\VEC{x}-\VEC{x}_0, t_0 - a) \right|
< C e^{\epsilon \|\VEC{x}\|^2}
\ \frac{e^{-\|\VEC{x}-\VEC{x}_0\|^2/(4t_0)}}{(\pi t_0)^{n/2}} \in L^1(\RR^n)
\]
for $0<a< t_0/2$.  We then get from the
Lebesgue Dominated Convergence Theorem that
\begin{equation} \label{heat_unicC}
\big(u(\cdot, a) \ast K_{t_0-a}(\cdot)\big)(\VEC{x}_0)
\to 0 \quad \text{as}\quad a \to 0^+ \ .
\end{equation}

\stage{ii} Our second claim is that
$\displaystyle \big(u(\cdot, b) \ast K_{t_0-b}(\cdot)\big)(\VEC{x}_0) \to
u(\VEC{x}_0,t_0)$ as $b \to t_0^-$.

We have that
\begin{equation} \label{heat_unicD}
\begin{split}
\left| u(\cdot, b) \ast K_{t_0-b}(\cdot)\big)(\VEC{x}_0)
- u(\VEC{x}_0,t_0) \right|
&\leq \int_{\RR^n} \left|u(\VEC{x},b) - u(\VEC{x}_0,b) \right|
K_{t_0-b}(\VEC{x}-\VEC{x}_0) \dx{\VEC{x}} \\
&\quad + \int_{\RR^n} \left|u(\VEC{x}_0,b) - u(\VEC{x}_0,t_0) \right|
K_{t_0-b}(\VEC{x}-\VEC{x}_0) \dx{\VEC{x}} \ .
\end{split}
\end{equation}
Given $\displaystyle \VEC{x} \in \RR^n$, there exists
$\theta \in ]0,1[$ such that 
\[
\left|u(\VEC{x},b) - u(\VEC{x}_0,b) \right|
\leq \| \nabla_{\VEC{x}} u( \VEC{x}_0+\theta(\VEC{x}-\VEC{x}_0),b)\|
\, \|\VEC{x} - \VEC{x}_0 \|
\]
according to the Mean Value Theorem for functions of several variables.
Therefore, given $\epsilon>0$, there exists $C>0$ such that
\[
\left|u(\VEC{x},b) - u(\VEC{x}_0,b) \right|
\leq C e^{\epsilon \|\VEC{x}_0 + \theta (\VEC{x}-\VEC{x}_0)\|^2}\,
\|\VEC{x}-\VEC{x}_0\|
\]
by hypothesis.  Choose $0 < c < t_0$ and $\epsilon < 1/(8(t_0-c))$.  Then,
\begin{align*}
&\left|u(\VEC{x},b) - u(\VEC{x}_0,b) \right| K_{t_0-b}(\VEC{x}-\VEC{x}_0)
\leq \frac{C}{(4\pi (t_0-b))^{n/2}}
e^{\epsilon \|\VEC{x}_0 + \theta (\VEC{x}-\VEC{x}_0)\|^2}\, \|\VEC{x}-\VEC{x}_0\|
e^{-\|\VEC{x}-\VEC{x}_0||^2/(4(t_0-b))} \\
&\leq \frac{C}{(4\pi)^{n/2}}
\underbrace{\left(\frac{1}{(t_0-b)^{n/2}} e^{-\|\VEC{x}-\VEC{x}_0||^2/(8(t_0-b))}
\right)}_{\leq (4n)^{n/2} e^{-n/2} /\|\VEC{x}-\VEC{x}_0\|^n}
e^{\epsilon \|\VEC{x}_0 + \theta (\VEC{x}-\VEC{x}_0)\|^2}\, \|\VEC{x}-\VEC{x}_0\|
e^{-\|\VEC{x}-\VEC{x}_0||^2/(8(t_0-c))} \\
&\leq \frac{C n^{n/2} e^{-n/2}}{\pi^{n/2}}
\|\VEC{x}-\VEC{x}_0\|^{n-1} e^{\epsilon \max \{\|\VEC{x}_0\|^2,\|\VEC{x}\|^2\}}\,
e^{-\|\VEC{x}-\VEC{x}_0||^2/(8(t_0-c))}  \in L^1(\RR^n)
\end{align*}
for all $b \in ]c,t_0[$.  Since
$\displaystyle
\left|u(\VEC{x},b) - u(\VEC{x}_0,b) \right| K_{t_0-b}(\VEC{x}-\VEC{x}_0)
\to 0$ as $b\to t_0^-$ for all $\VEC{x} \neq \VEC{x}_0$, we get from
the Lebesgue Dominate Convergence Theorem that
\begin{equation} \label{heat_unicE}
\int_{\RR^n} \left|u(\VEC{x},b) - u(\VEC{x}_0,b) \right|
K_{t_0-b}(\VEC{x}-\VEC{x}_0) \dx{\VEC{x}} \to 0
\quad \text{as} \quad b \to t_0^- \ .
\end{equation}

Moreover,
\begin{align}
\int_{\RR^n} \left|u(\VEC{x}_0,b) - u(\VEC{x}_0,t_0) \right|
K_{t_0-b}(\VEC{x}-\VEC{x}_0) \dx{\VEC{x}}
& = \left|u(\VEC{x}_0,b) - u(\VEC{x}_0,t_0) \right| \int_{\RR^n}  
K_{t_0-b}(\VEC{x}-\VEC{x}_0) \dx{\VEC{x}} \nonumber \\
& = \left|u(\VEC{x}_0,b) - u(\VEC{x}_0,t_0) \right| \to 0
\quad \text{as} \quad b \to t_0^- \ . \label{heat_unicF}
\end{align}

It follows from (\ref{heat_unicD}), (\ref{heat_unicE}) and
(\ref{heat_unicF}) that
\begin{equation} \label{heat_unicG}
\big(u(\cdot, b) \ast K_{t_0-b}(\cdot)\big)(\VEC{x}_0)
\to u(\VEC{x}_0,t_0)
\quad \text{as}\quad b \to t_0^- \ .
\end{equation}
This complete the prove of the second claim.

\stage{iii} If we let $a\to 0^+$ and $b\to t_0^-$ in (\ref{heat_unicB}), we get
from (\ref{heat_unicC}) and (\ref{heat_unicG}) that
$u(\VEC{x}_0, t_0) = 0$.
\end{proof}

\section{Fundamental Solution} \label{SectHeatFundSol}

As before, let $\displaystyle K:\RR^n \times \RR \rightarrow \RR$ be
the function defined by
\[
K(\VEC{x},t) =
\begin{cases}
\displaystyle (4\pi t)^{-n/2} e^{-\|\VEC{x}\|^2/(4t)} & \quad 
\text{if} \quad t>0 \\
0 &  \quad \text{if} \quad t\leq 0
\end{cases}
\]
We now generalize the discussion of the previous section to the
solution of the heat equation in the sense of distributions.

\begin{theorem}
$K$ is a fundamental solution of the heat equation
$\displaystyle \pdydx{u}{t} - \Delta u = 0$ on $\displaystyle \RR^n\times \RR$.
\end{theorem}

\begin{proof}
\stage{i} We prove that $\displaystyle K\in \DD'(\RR^n\times \RR)$ by
proving that $\displaystyle K \in L^1_{loc}(\RR^n \times \RR)$.  We
prove in fact a stronger result.  Since
$\displaystyle \int_{\RR^n} \left| K(\VEC{x},t) \right| \dx{\VEC{x}} = 1$
for all $t>0$, we get
\[
\int_I \int_{\RR^n} \left| K(\VEC{x},t) \right| \dx{\VEC{x}} \dx{t}
\leq \int_I \dx{t} < \infty
\]
for any bounded interval $I \subset \RR$.  Thus $K$ defines a proper
distribution on $\displaystyle \RR^n \times \RR$.

\stage{ii} We show that
$\displaystyle \pdydx{K}{t} - \Delta K = \delta$.  For $\epsilon>0$,
let
\[
K^{[\epsilon]}(\VEC{x},t) =
\begin{cases}
K(\VEC{x}, t) & \quad \text{if} \quad t > \epsilon, \VEC{x} \in \RR^n \\
0 & \quad \text{if} \quad t \leq \epsilon, \VEC{x} \in \RR^n \\
\end{cases}
\]
It is easy to show, as we have done in (i), that
$\displaystyle K^{[\epsilon]}\in \DD'(\RR^n\times \RR)$.

Given $\displaystyle \phi \in \DD(\RR^n \times \RR)$,
we have that $\displaystyle K^{[\epsilon]}(\VEC{x},t) \phi(\VEC{x},t)
\rightarrow K(\VEC{x},t) \phi(\VEC{x},t)$ as
$\epsilon \rightarrow 0^+$ for all $(\VEC{x},t) \in \RR^n\times \RR$.
Moreover, since $\phi$ has a compact support,
$\displaystyle M = \sup_{(\VEC{x},t) \in \RR^n \times \RR} |\phi(\VEC{x},t)|
< \infty$ and there exists $I \subset \RR$ such that
$\displaystyle \supp \phi \subset \RR^n \times I$. 
Since
\[
\left| K^{[\epsilon]}(\VEC{x},t) \phi(\VEC{x},t) \right|
\leq M \Chi_{\RR^n \times I}(\VEC{x},t) K(\VEC{x},t)
\]
for all $(\VEC{x},t) \in \RR^n\times \RR$ and
$\displaystyle M \Chi_{\RR^n \times I} K \in L^1(\RR^n \times \RR)$,
we get from the Lebesgue Dominate Convergence Theorem that
\[
\int_{\RR} \int_{\RR^n} K^{[\epsilon]}(\VEC{x},t) \phi(\VEC{x},t)
\dx{\VEC{x}}\dx{t} \to
\int_{\RR} \int_{\RR^n} K(\VEC{x},t) \phi(\VEC{x},t)
\dx{\VEC{x}}\dx{t} \quad \text{as} \quad \epsilon \to 0^+ \ .
\]
Since $\displaystyle \phi \in \DD(\RR^n \times \RR)$ is arbitrary,
we have that $\displaystyle K^{[\epsilon]} \rightarrow K$ in
$\displaystyle \DD'(\RR^n\times \RR)$ as $\epsilon \rightarrow 0^+$.

To complete the proof, we only have to show that
$\displaystyle \pdfdx{K^{[\epsilon]}}{t} - \Delta_{\VEC{x}} K^{[\epsilon]}
\rightarrow \delta$ in $\displaystyle \DD'(\RR^n\times \RR)$ as
$\epsilon \to 0^+$.
Namely, for $\displaystyle \phi \in \DD(\RR^n\times \RR)$,
we have to show that
\[
\ps{K^{[\epsilon]}}{-\pdfdx{\phi}{t} - \Delta_{\VEC{x}} \phi}
\rightarrow \phi(\VEC{0},0) \quad \text{as} \quad \epsilon \to 0^+ \ .
\]
Integration by parts yields
\begin{align*}
\ps{K^{[\epsilon]}}{-\pdfdx{\phi}{t} - \Delta_{\VEC{x}} \phi}
&= \int_{\epsilon}^\infty \int_{\RR^n} K(\VEC{x},t) \left(
-\pdydx{\phi}{t}(\VEC{x},t) - \Delta_{\VEC{x}} \phi(\VEC{x},t)\right)
\dx{\VEC{x}} \dx{t} \\
&= \int_{\epsilon}^\infty \int_{\RR^n} \left( \pdydx{K}{t}(\VEC{x},t) 
-\Delta_{\VEC{x}} K(\VEC{x},t)\right) \phi(\VEC{x},t) \dx{\VEC{x}} \dx{t}
+ \int_{\RR^n} K(\VEC{x},\epsilon) \phi(\VEC{x},\epsilon) \dx{\VEC{x}}
\end{align*}
because $\phi$ has a compact support in $\displaystyle \RR^n\times \RR$.  Since
$\displaystyle \pdydx{K}{t}(\VEC{x},t) -\Delta_{\VEC{x}} K(\VEC{x},t) = 0$
and $K(\VEC{x},t) = K(-\VEC{x},t)$ for
$\displaystyle \VEC{x}\in \RR^n$ and $t>0$, we have
that
\[
\ps{K^{[\epsilon]}}{-\pdfdx{\phi}{t} - \Delta_{\VEC{x}} \phi}
= \int_{\RR^n} K(-\VEC{x},\epsilon) \phi(\VEC{x},\epsilon)
\dx{\VEC{x}}
= \left( K(\cdot,\epsilon) \ast \phi(\cdot, \epsilon)\right)(\VEC{0}) \ .
\]

Finally, we prove that
\begin{align*}
\big( K(\cdot,\epsilon) \ast \phi(\cdot, \epsilon)\big)(\VEC{0})
&= \big( K(\cdot,\epsilon) \ast \phi(\cdot, 0)\big)(\VEC{0}) +
\big( K(\cdot,\epsilon) \ast ( \phi(\cdot, \epsilon) -
\phi(\cdot, 0) ) \big)(\VEC{0}) \\
& \rightarrow \phi(\VEC{0},0) \quad \text{as} \quad \epsilon \to 0^+ \ .
\end{align*}
According to (\ref{heat_convol1}), (\ref{heat_convol2}) and
Proposition~\ref{distr_limit_convol}, we have that
\[
K(\cdot,\epsilon) \ast \phi(\cdot, 0) \rightarrow \phi(\cdot, 0)
\quad \text{as} \quad \epsilon \to 0^+ \ .
\]
Thus
\[
(K(\cdot,\epsilon) \ast \phi(\cdot, 0))(\VEC{0}) \rightarrow \phi(\VEC{0},0)
\quad \text{as} \quad \epsilon \to 0^+ \  .
\]
Since $\displaystyle \phi:\RR^n\times \RR$ is a continuous function with
compact support, $\phi$ is uniformly continuous.  In particular,
$\phi(\cdot, \epsilon) - \phi(\cdot, 0) \rightarrow 0$ 
uniformly on $\displaystyle \RR^n$.  Since $K(\VEC{x},\epsilon) >0$ for all
$\VEC{x}\in \RR^n$, it follows from (\ref{heat_convol1}) that
$\|K(\cdot,\epsilon)\|_1 = 1$ for $\epsilon >0$.  Hence
\begin{align*}
\left| \big( K(\cdot,\epsilon) \ast ( \phi(\cdot, \epsilon) -
\phi(\cdot, 0) )\big)(\VEC{0})  \right| &\leq
\| K(\cdot, \epsilon) \|_1 \,
\|\phi(\cdot, \epsilon) - \phi(\cdot, 0)\|_\infty \\
&= \|\phi(\cdot, \epsilon) - \phi(\cdot, 0)\|_\infty
\rightarrow 0 \quad \text{as} \quad \epsilon \to 0^+ \ . \qedhere
\end{align*}
\end{proof}

\begin{cor}
The heat operator
$\displaystyle L(\VEC{x},y,D) = \pdydx{}{t} - \Delta_{\VEC{x}}$ is hypoelliptic.
\end{cor}

\begin{proof}
This follows from Theorem~\ref{distr_hypoTH} because $\displaystyle K \in
C^\infty\left((\RR^n\times \RR) \setminus \{ (\VEC{0},0) \}\right)$.
\end{proof}

If $\displaystyle f \in \EE'(\RR^n\times \RR)$, then $u = K\ast f$ is
a solution (in the sense of distributions) of the non-homogeneous heat
equation $\displaystyle \pdydx{u}{t} - \Delta_{\VEC{x}} u = f$ on
$\displaystyle \RR^n\times \RR$.
The conclusion of the next theorem states that $K\ast f$ can define a
function almost everywhere if $f$ is nice enough.

\begin{theorem}
If $\displaystyle f \in L^1(\RR^n\times \RR)$, then $u = f\ast K$ is
well defined almost everywhere and
$\displaystyle \pdydx{u}{t} - \Delta_{\VEC{x}} u = f$ in the sense of
distribution.
\end{theorem}

\begin{proof}
Since $\displaystyle f\in L^1(\RR^n\times \RR)$, we have by Fubini's
Theorem that $\displaystyle f(\cdot, s) \in L^1(\RR^n)$ for almost all
$s\in \RR$.  We have also seen that $\| K(\cdot, t-s) \|_1 = 1$ for
all $s<t$.  Hence, from 
Young's Inequality, Theorem~\ref{distr_sp_young}, we have
\[
\| |f|(\cdot,s) \ast K(\cdot, t-s) \|_1 \leq \|f(\cdot, s)\|_1\,
\|K(\cdot, s-t)\|_1 = \|f(\cdot, s)\|_1 < \infty
\]
for almost all $s$ such that $s<t$.  Thus
$\displaystyle \VEC{x} \mapsto \left( |f|(\cdot,s) \ast K(\cdot, t-s)
\right)(\VEC{x})$ for $\displaystyle \VEC{x} \in \RR^n$ is in
$\displaystyle L^1(\RR^n)$ for almost all $s$ such that $s<t$.  The
convolution is obviously on $\displaystyle \RR^n$.  We then have
\begin{align*}
\left| u(\VEC{x},t)\right| &\leq
\int_{-\infty}^\infty \int_{\RR^n} |f|(\VEC{y},s)
K(\VEC{x}-\VEC{y}, t-s) \dx{\VEC{y}} \dx{s}
= \int_{-\infty}^t \int_{\RR^n} |f|(\VEC{y},s)
K(\VEC{x}-\VEC{y}, t-s) \dx{\VEC{y}} \dx{s} \\
&= \int_{-\infty}^t \left( |f|(\cdot,s) \ast K(\cdot, t-s)
\right)(\VEC{x}) \dx{s}
\end{align*}
because $K(\VEC{z}, t-s) = 0$ for $\displaystyle \VEC{z}\in \RR^n$
and $s\geq t$.  Hence,
\begin{align*}
\int_{\RR^n} \left| u(\VEC{x},t) \right| \dx{\VEC{x}} 
&\leq \int_{-\infty}^t \int_{\RR^n} \left( |f|(\cdot,s) \ast K(\cdot, t-s)
\right)(\VEC{x}) \dx{\VEC{x}} \dx{s}
= \int_{-\infty}^t \| |f|(\cdot,s) \ast K(\cdot, t-s) \|_1 \dx{s} \\
&\leq \int_{-\infty}^t \|f(\cdot, s)\|_1 \dx{s}
= \int_{-\infty}^t \int_{\RR^n} \left| f(\VEC{x},s)\right| \dx{\VEC{x}}\dx{s}
\leq \|f\|_1 < \infty \  ,
\end{align*}
where the last $\displaystyle L^1$-norm is on
$\displaystyle L^1(\RR^n\times \RR)$.  It follows
that $u(\VEC{x},t)$ exists for all $t \in \RR$ and for almost every
$\displaystyle \VEC{x} \in \RR^n$.

The rest of the proof is similar to the proof of
Theorem~\ref{laplace_not_EEp} with $N$ and $\Delta$ in that proof
replaced by $K$ and $\displaystyle \pdydx{}{t} - \Delta_{\VEC{x}}$
respectively.
\end{proof}

\section{Maximum Principle}

\begin{theorem} [Weak Maximum Principle]
Let $\Omega$ be a bounded open subset of $\RR^n$.  Suppose that
$u \in C(\overline{\Omega} \times [0,T])$ is a solution of the heat
equation $\displaystyle \pdydx{u}{t} - \Delta u = 0$ in
$\Omega \times ]0,T[$.  Then, $u$ reaches its maximum on
$(\Omega \times \{0\}) \cup ( \partial \Omega \times [0,T])$.
\end{theorem}

\begin{proof}
Since $\displaystyle \pdydx{}{t}-\Delta_{\VEC{x}}$ is hypoelliptic, we
have that $\displaystyle u \in C^\infty(\Omega\times ]0,T[)$.

Consider $\epsilon>0$ and $T'$ such that $0<T'<T$.  let
$\displaystyle v(\VEC{x},t) = u(\VEC{x},t) + \epsilon \|\VEC{x}\|^2$
for $(\VEC{x},t) \in \overline{\Omega} \times [0,T]$.  We have
\begin{equation} \label{heat_max}
\pdydx{v}{t} - \Delta_{\VEC{x}} v = -2n\epsilon < 0 \quad , \quad
(\VEC{x},t) \in \Omega \times ]0,T[ \  .
\end{equation}

If the maximum of $v$ on $\overline{\Omega}\times [0,T']$ is
$(\VEC{x}_0,t_0) \in \Omega\times ]0,T'[$, then $(\VEC{x}_0,t_0)$ is a
critical point and
$\displaystyle \graD_{\VEC{x},t} v(\VEC{x}_0,t_0) = \VEC{0} \in \RR^n\times \RR$.
Moreover, $\displaystyle \pdydxn{v}{x_j}{2}(\VEC{x}_0,t_0) \leq 0$ for
$1\leq j\leq n$ because, in each $x_j$-direction, $v$ has a local
maximum or $v$ is constant near $(\VEC{x}_0,t_0)$.
Hence, $\displaystyle \pdydx{v}{t}(\VEC{x}_0,t_0) =0$ and
$\Delta_{\VEC{x}} v(\VEC{x}_0,t_0) \leq 0$ implies that
$\displaystyle \pdydx{v}{t} - \Delta_{\VEC{x}} v \geq 0$.  This
contradicts (\ref{heat_max}).  We need to add the term
$\displaystyle \epsilon \|\VEC{x}\|^2$ to $u$ to get the strict inequality in
(\ref{heat_max}).

If the maximum of $v$ on $\overline{\Omega}\times [0,T']$ is
$(\VEC{x}_0,T')$ with $\VEC{x}_0 \in \Omega$, we have that
$\displaystyle \pdydx{v}{t}(\VEC{x}_0,T') \geq 0$ because of the
result in the previous paragraph; namely, $v$ is
constant or increasing in the direction of $t$.  We still have
$\displaystyle \pdydxn{v}{x_j}{2}(\VEC{x}_0,T') \leq 0$ for
all $j$ for the same reason than before.  We need to introduce $T'<T$
because there is no guarantee that the second order partial
derivatives $\displaystyle \pdydxn{u}{x_j}{2}(\VEC{x},t)$ exist for
$t=T$.  Hence, $\displaystyle \pdydx{v}{t}(\VEC{x}_0,T') \geq 0$ and
$\Delta_{\VEC{x}} v(\VEC{x}_0,T') \leq 0$ implies that
$\displaystyle \pdydx{v}{t} - \Delta_{\VEC{x}} v \geq 0$.  This again
contradicts (\ref{heat_max}).

We have found that
\begin{align*}
&\max \{ u(\VEC{x},t) : (\VEC{x},t)\in \overline{\Omega}\times [0,T'] \}
\leq \max \{ v(\VEC{x},t) : (\VEC{x},t)\in \overline{\Omega}\times
[0,T'] \} \\
&\qquad = \max \{ v(\VEC{x},t) : (\VEC{x},t)\in (\Omega\times \{0\}) \cup
(\partial \Omega \times [0,T'])\,\} \\
&\qquad \leq \max \{ u(\VEC{x},t) : (\VEC{x},t)\in (\Omega\times \{0\}) \cup
(\partial \Omega \times [0,T'])\,\} + \epsilon
\max \{ \|\VEC{x}\|^2 : \VEC{x}\in \overline{\Omega} \} \  .
\end{align*}
If $\epsilon \rightarrow 0$, we get
\[
\max \{ u(\VEC{x},t) : (\VEC{x},t)\in \overline{\Omega}\times [0,T'] \}
\leq \max \{ u(\VEC{x},t) : (\VEC{x},t)\in (\Omega\times \{0\}) \cup
(\partial \Omega \times [0,T'])\,\}
\]
for all $T'<T$.  Thus
\[
\max \{ u(\VEC{x},t) : (\VEC{x},t)\in \overline{\Omega}\times [0,T] \}
\leq \max \{ u(\VEC{x},t) : (\VEC{x},t)\in (\Omega\times \{0\}) \cup
(\partial \Omega \times [0,T])\,\} \ . \qedhere
\]
\end{proof}

\begin{cor}
There is only one solution
$u \in C(\overline{\Omega}\times [0.T])$ (if any) of
\begin{equation} \label{heat_wm_un}
\begin{split}
\pdydx{u}{t}(\VEC{x},t) - \Delta_{\VEC{x}}(\VEC{x},t) u
&=0 \quad , \quad (\VEC{x},t) \in \Omega \times ]0,T[ \\
u(\VEC{x},0) &= f(\VEC{x}) \quad , \quad \VEC{x} \in \Omega \\
u(\VEC{x},t) &= g(\VEC{x},t) \quad , \quad (\VEC{x},t) \in
\partial \Omega \times ]0,T[
\end{split}
\end{equation}
\end{cor}

\begin{proof}
If $u_1$ and $u_2$ are two solutions of (\ref{heat_wm_un}) in
$C(\overline{\Omega}\times [0.T])$, we may apply the weak maximum principle
above to $u_1-u_2$ and $u_2-u_1$ to conclude that
$u_1 - u_2 \leq 0$ and $u_2 - u_1 \leq 0$ on $\Omega \times [0,T]$
respectively.  Thus $u_1 =u_2$ on $\Omega \times [0,T]$.
\end{proof}

\section{Separation of Variables}

\subsection{Cartesian Coordinates}

\begin{egg}
Find the solution of the heat equation
\begin{equation} \label{heat_spv_he}
\pdydx{u}{t} = c^2 \pdydxn{u}{x}{2} \quad ,
\quad 0 < x < \pi \ \text{and} \ t > 0 ,
\end{equation}
with $c^2 = 4$, the boundary conditions
$u(0,t) = u(\pi,t) = 0$ for $t>0$ and the initial condition
\begin{equation} \label{heat_series_init_cond}
u(x,0) = f(x) = \begin{cases}
x & \quad 0 < x \leq \pi/2 \\
\pi - x & \quad \pi/2 < x < \pi
\end{cases}
\end{equation}

\subI{Separation of Variables}
If we substitute $u(x,t) = F(x)G(t)$ in (\ref{heat_spv_he}), we get
\[
F(x)\,\dydx{G}{t}(t) = 4 \dydxn{F}{x}{2}(x)\, G(t)
\quad , \quad t>0 \ \text{and} \ 0<x<\pi \ .
\]
Thus, after dividing both sides by $4\,F(x)G(t)$, we get
\[
\frac{1}{4\,G(t)} \, \dydx{G}{t}(t) = \frac{1}{F(x)}\, \dydxn{F}{x}{2}(x)
\quad , \quad t>0 \ \text{and} \ 0<x<\pi \ .
\]
Since the right hand side is independent of $t$ and the left hand side
is independent of $x$, we get
\[
\frac{1}{4\,G(t)} \, \dydx{G}{t}(t) = \frac{1}{F(x)}\,
\dydxn{F}{x}{2}(x) = k \quad , \quad t>0 \ \text{and} \ 0<x<\pi \ ,
\]
for some constant $k$.  We end up with two ordinary differential equations.
\begin{equation} \label{heat_spv_eqs}
\dydxn{F}{x}{2}(x) -k F(x) = 0 \quad \text{and}
\quad \dydx{G}{t}(t) - 4\,k\,G(t) = 0 \ .
\end{equation}

Two boundary conditions are associated to the first ordinary
differential equation in (\ref{heat_spv_eqs}).   From $u(0,t) = 0$, we
get $F(0)G(t)=0$.  Since we assume that $G$ is not null, we get
$F(0)=0$.  Similarly, from $u(\pi,t) = 0$, we get $F(\pi)G(t)=0$.
Again, since we assume that $G$ is not null, we get $F(\pi)=0$.
The boundary conditions for the first ordinary differential equation
in (\ref{heat_spv_eqs}) are $F(0)=F(\pi)=0$.

\subI{Simple Functions}
We first consider the boundary value problem
\begin{equation} \label{heat_spv_bvp}
\dydxn{F}{x}{2}(x) -k F(x) = 0 \quad , \quad 0 \leq x \leq \pi \ ,
\end{equation}
with $F(0)=F(\pi)=0$.  The form of the general solution of this
ordinary differential equation is determined by the roots of the
characteristic equation $\lambda^2-k=0$.

If $k>0$, the roots of the characteristic equation are $\pm \sqrt{k}$.
Since the roots are real, the solution of the ordinary differential
equation is of the form $\displaystyle
F(x) = A e^{\sqrt{k}\,x} + B e^{-\sqrt{k}\, x}$.
However, $F(0)=0$ implies that $A+B=0$ and $F(\pi)=0$ implies
$\displaystyle A e^{\pi\sqrt{k}} + B e^{-\pi\sqrt{k}} = 0$.  The only
solution of the system
\[
\begin{pmatrix}
1 & 1 \\ e^{\pi\sqrt{k}} & e^{-\pi\sqrt{k}}
\end{pmatrix}
\begin{pmatrix}
A \\ B
\end{pmatrix}
=
\begin{pmatrix}
0 \\ 0 
\end{pmatrix}
\]
is $A=B=0$ because
\[
\det
\begin{pmatrix}
1 & 1 \\ e^{\pi\sqrt{k}} & e^{-\pi\sqrt{k}}
\end{pmatrix}
= e^{-\pi\sqrt{k}} - e^{\pi\sqrt{k}} 
= e^{-\pi\sqrt{k}}\left(1 - e^{2\pi\sqrt{k}}\right) \neq 0
\]
for $k\neq 0$.  Therefore, the null solution is the only solution
of the boundary value problem (\ref{heat_spv_bvp}) for $k>0$.

If $k=0$, the solution of the ordinary differential equation is $F(x)=Ax+B$.
However, $F(0)=0$ implies that $B=0$.  Hence, $F(\pi)=0$ implies
that $A\pi = 0$.  Thus $A=B=0$ and again the null solution is the
only solution of the boundary value problem (\ref{heat_spv_bvp}) for $k=0$.

If $k<0$, the roots of the characteristic equation are $\pm i \sqrt{-k}$.
Since the roots are complex, the solution of the ordinary differential
equation is of the form $\displaystyle
F(x) = A \cos\left(\sqrt{-k}\,x\right) + B \sin\left(\sqrt{-k}\,x\right)$.
The condition $F(0)=0$ implies that $A=0$.  Hence, $F(\pi)=0$ implies that
$B \sin\left(\pi\sqrt{-k}\right) = 0$.  If we take
$B=0$, we get the null solution.  We must therefore have
$\sin\left(\pi\sqrt{-k}\right) = 0$.  This implies that
$k = k_n \equiv -n^2$ for $n>0$.  The boundary value problem
(\ref{heat_spv_bvp}) has non-null solutions only for $k=k_n=-n^2<0$
with $n$ a positive integer, and the solutions associated to $k_n$ are of
the form $F(x)=F_n(x) \equiv B_n \sin(n\,x)$.

The second ordinary differential equation in (\ref{heat_spv_eqs}) is
therefore
\[
\dydx{G}{t}(t) + 4n^2\,G(t) = 0 \  ,
\]
where $n$ is a positive integer.  For each $n$, the general solution
of this first order separable ordinary differential equation is
$\displaystyle G(t) = G_n(t) \equiv C_n e^{-4n^2t}$.

We have found the simple functions
\[
u_n(x,t) \equiv F_n(x)G_n(t) = b_n\,\sin(n\,x)\, e^{-4n^2t} \quad , \quad
n>0 \ .
\]
The constant $b_n$ comes from the product of the constants $B_n$ and
$C_n$.

\subI{Initial condition}
We seek a solution of the form
\[
u(x,t) = \sum_{n=1}^\infty u_n(x,t)
= \sum_{n=1}^\infty b_n\,\sin(n\,x)\, e^{-4n^2t} \  .
\]

From $u(x,0) = f(x)$, we get
$\displaystyle f(x) = \sum_{n=1}^\infty b_n\,\sin(n\,x)$
for $0<x<\pi$.  This is the Fourier sine series of $f$.  The coefficients of
this series are given by
\[
b_n = \frac{2}{\pi} \int_0^\pi f(x) \sin(nx)\dx{x}
= \frac{2}{\pi} \int_0^{\pi/2} x\sin(nx) \dx{x}
+ \frac{2}{\pi} \int_{\pi/2}^\pi (\pi-x) \sin(nx)\dx{x} \ .
\]
The first integral can be evaluated using integration by parts.
% With $u(x)=x$ and $v'(x) = \sin(nx)$, we get $u'(x) =1$,
% $\displaystyle v(x) = -\frac{1}{n}\cos(nx)$.
We get
\begin{align*}
\int_0^{\pi/2} x\sin(nx) \dx{x}
% &= u(x)v(x)\bigg|_0^{\pi/2}-\int_0^{\pi/2} v(x) u'(x) \dx{x} \\
&= -\frac{1}{n}\,x\cos(nx)\bigg|_0^{\pi/2}
+ \frac{1}{n} \int_0^{\pi/2} \cos(nx) \dx{x}
= -\frac{\pi}{2n}\cos\left(\frac{n\pi}{2}\right)
+ \frac{1}{n^2} \sin(nx)\bigg|_0^{\pi/2} \\
&= -\frac{\pi}{2n}\cos\left(\frac{n\pi}{2}\right) + \frac{1}{n^2}
\sin\left(\frac{n\pi}{2}\right)
\end{align*}
Similarly, the second integral can be evaluated using integration by
parts
% With $u(x)=\pi-x$ and $v'(x) = \sin(nx)$, we get $u'(x) = -1$,
% $\displaystyle v(x) = -\frac{1}{n}\cos(nx)$.
\begin{align*}
\int_{\pi/2}^\pi (\pi-x) \sin(nx) \dx{x}
% &= u(x)v(x)\bigg|_{\pi/2}^\pi-\int_{\pi/2}^\pi v(x) u'(x) \dx{x} \\
&= -\frac{1}{n}\,(\pi-x)\cos(nx)\bigg|_{\pi/2}^\pi
- \frac{1}{n} \int_{\pi/2}^\pi \cos(nx) \dx{x} \\
&= \frac{\pi}{2n}\cos\left(\frac{n\pi}{2}\right)
- \frac{1}{n^2} \sin(nx)\bigg|_{\pi/2}^\pi
= \frac{\pi}{2n}\cos\left(\frac{n\pi}{2}\right) + \frac{1}{n^2}
\sin\left(\frac{n\pi}{2}\right)
\end{align*}
Thus
\[
b_n = \frac{4}{n^2 \pi} \sin\left(\frac{n\pi}{2}\right) \\
=
\begin{cases}
0 & \quad \text{if $n$ is even} \\
\displaystyle (-1)^{(n+3)/2}\frac{4}{n^2 \pi} & \quad \text{if $n$ is odd}
\end{cases}
\]

The solution of the heat equation is
\begin{equation} \label{heat_series_sol}
u(x,t) = \sum_{m=1}^\infty (-1)^{m+1}\frac{4}{(2m-1)^2 \pi}\,
\sin((2m-1)\,x)\, e^{-4(2m-1)^2t} \ .
\end{equation}

Since
\[
\left| (-1)^{m+1}\frac{4}{(2m-1)^2 \pi}\, \sin((2m-1)\,x)\, e^{-4(2m-1)^2t}
\right| \leq \frac{4}{\pi(2m-1)^2}
\]
and the series
\[
\sum_{m=1}^\infty \frac{4}{\pi(2m-1)^2}
\]
converges, we have that the series solution converge absolutely and
uniformly for $(x,t) \in [0,\pi] \times [0,\infty[$.  Since each term
of the series are continuous functions on $[0,\pi] \times [0,\infty[$,
it follow from the uniform convergence of the series that $u$ is
continuous on $[0,\pi] \times [0,\infty[$.  Moreover, $u$
satisfies the boundary and initial conditions.

To prove that the method of separation of variables is valid in the
classical sense, it remains to prove that the differential equation is
satisfied by $u$ when standard derivatives of order two in $x$ and of
order one in $t$ are used.  Since each term of the series solution of
$u$ satisfies the differential equation in the classical sense, it is
enough to prove that the series (\ref{heat_series_sol}),
\begin{equation} \label{heat_spv_der_series}
\begin{split}
\pdydx{u}{x} &\sim \sum_{m=1}^\infty  \left( (-1)^{m+1}\frac{4}{(2m-1) \pi}\,
\cos((2m-1)\,x)\, e^{-4(2m-1)^2t} \right) \ , \\
\pdydxn{u}{x}{2} &\sim \sum_{m=1}^\infty \left( (-1)^m\frac{4}{\pi}\,
\sin((2m-1)\,x)\, e^{-4(2m-1)^2t} \right) \ , \\
\pdydx{u}{t} &\sim \sum_{m=1}^\infty \left( (-1)^m\frac{16}{\pi}\,
\sin((2m-1)\,x)\, e^{-4(2m-1)^2t} \right)
\end{split}
\end{equation}
converge absolutely and uniformly for
$(x,t)\in [0,\pi]\times ]\epsilon,\infty[$
whatever $\epsilon>0$ arbitrary but fixed.  To prove the absolute and uniform
convergence of these series, it suffices to prove that the series
\begin{equation}\label{heat_series_solB}
\sum_{m=1}^\infty (2m-1)^{k} e^{-4(2m-1)^2 t}
\end{equation}
converges uniformly for $t\geq \epsilon$ whatever the value of
$k\geq -2$.  Since
$\displaystyle e^{-4(2m-1)^2\,t} \leq e^{-4(2m-1)^2\epsilon}$
for $t\geq \epsilon$,  we get as a consequence of the comparison test that
(\ref{heat_series_solB}) converges uniformly because
\[
\sum_{m=1}^\infty (2m-1)^{k} e^{-4(2m-1)^2\epsilon}
\]
converges whatever the value of $k$.  That this series converges can
be deduced from the ratio test since
\[
\lim_{m\rightarrow \infty}
\frac{ \displaystyle (2m+1)^{k} e^{-4(2m+1)^2\epsilon}}
{ \displaystyle (2m-1)^{k} e^{-4(2m-1)^2\epsilon}}
= \lim_{m\rightarrow \infty} \left(\frac{2m+1}{2m-1}\right)^{k}
e^{-32 m \epsilon} = 0 < 1 \  .
\]
\end{egg}

\begin{rmk}
In the previous example, the proof of the absolute and uniform
convergence of the series like those in (\ref{heat_spv_der_series}) is
valid for any series obtained from the series solutions of $u$ by deriving the
terms of this series.  This shows that the solution $u$ is in fact
infinitely differentiable with respect to $x$ and $t$ in
$]0,\pi[\times ]0,\infty[$.  That was expected since
$\displaystyle \pdydx{}{t} - \Delta_{\VEC{x}}$ is hypoelliptic.

Moreover, the previous discussion is valid whatever the choice of $f$
and is not limited to the function $f$ given in (\ref{heat_series_init_cond}).

Even if the initial condition was not differentiable everywhere on
the interval $[0,\pi]$, the solution is smooth for $t>0$.  This is
typical of the parabolic equation.  For the heat equation, we have
seen in Remark~\ref{heat_smooth_sol} that this property is not limited
to one-dimension heat equations.
\end{rmk}

From now on, we will not verify that the series solution and its
derivatives converge absolutely and (locally) uniformly on the domain,
and that the series solution satisfy the partial differential
equation.  In fact, the series solution is a strong solution of the
partial differential equation.

\begin{egg}
Find the solution of the heat equation
\begin{equation} \label{heat_spv_diffa}
\pdydx{u}{t} = c^2 \pdydxn{u}{x}{2} \quad , \quad 0 < x < 5
\ \text{and} \ t > 0 \ ,
\end{equation}
with $c^2 = 4$, the boundary conditions
$u(0,t) = 0$ and $\displaystyle \pdydx{u}{x}(5,t) = 0$ for $t>0$, and
the following initial condition.
\begin{center}
\begin{tabular}{l@{\hspace{0.5em}}l@{\hspace{5em}}l@{\hspace{0.5em}}l}
\subQ{a} & $\displaystyle u(x,0) = 8 \sin\left(\frac{13\pi x}{10}\right)$ &
\subQ{b} & $u(x,0) = x(5-x)^2$
\end{tabular}
\end{center}

\subI{Separation of Variables}
If we substitute $u(x,t) = F(x)G(t)$ in (\ref{heat_spv_diffa}), we get
\[
F(x)\,\dydx{G}{t}(t) = 4 \dydxn{F}{x}{2}(x)\, G(t)
\quad , \quad t>0 \ \text{and} \ 0<x<5 \ . 
\]
Thus, after dividing both sides by $4\,F(x)G(t)$, we get
\[
\frac{1}{4\,G(t)} \, \dydx{G}{t}(t) = \frac{1}{F(x)}\, \dydxn{F}{x}{2}(x)
\quad , \quad t>0 \ \text{and} \ 0<x<5 \ .
\]
Since the right hand side is independent of $t$ and the left hand side
is independent of $x$, we get
\[
\frac{1}{4\,G(t)} \, \dydx{G}{t}(t) = \frac{1}{F(x)}\,
\dydxn{F}{x}{2}(x) = k \quad , \quad t>0 \ \text{and} \ 0<x<5 \ ,
\]
for some constant $k$.  We end up with two ordinary differential equations.
\begin{equation} \label{heat_spv_diffb}
\dydxn{F}{x}{2}(x) - k F(x) = 0 \quad \text{and}
\quad \dydx{G}{t}(t) - 4\,k\,G(t) = 0 \ .
\end{equation}

Two boundary conditions are associated to the first ordinary
differential equation in (\ref{heat_spv_diffb}).   From $u(0,t) = 0$,
we get $F(0)G(t)=0$.  Since we assume that $G$ is not null, we get
$F(0)=0$.  Similarly, from $\displaystyle \pdydx{u}{x}(5,t) = 0$, we
get $F'(5)G(t)=0$. Again, since we assume that $G$ is not null, we
get $F'(5)=0$.  The boundary conditions for the first ordinary
differential equation in (\ref{heat_spv_diffb}) are $F(0)=F'(5)=0$.

\subI{Simple Functions}
We first consider the boundary value problem
\begin{equation} \label{heat_spv_diffc}
\dydxn{F}{x}{2}(x) -k F(x) = 0 \quad , \quad 0 \leq x \leq 5 \ ,
\end{equation}
with $F(0)=F'(5)=0$.  The form of the general solution of this
ordinary differential equation is determined by the 
roots of the characteristic equation $\lambda^2-k=0$.

If $k>0$, the roots of the characteristic equation are $\pm \sqrt{k}$.
Since the roots are real, the solution of the ordinary differential
equation is of the form
$\displaystyle F(x) = A e^{\sqrt{k}\,x} + B e^{-\sqrt{k}\, x}$.
However, $F(0)=0$ implies that $A+B=0$ and $F'(5)=0$ implies
$\displaystyle A \sqrt{k} e^{5\sqrt{k}} - B \sqrt{k} e^{-5\sqrt{k}} = 0$.
The only solution of the system
\[
\begin{pmatrix}
1 & 1 \\ \sqrt{k} e^{5\sqrt{k}} &-\sqrt{k} e^{-5\sqrt{k}}
\end{pmatrix}
\begin{pmatrix}
A \\ B
\end{pmatrix}
=
\begin{pmatrix}
0 \\ 0 
\end{pmatrix}
\]
is $A=B=0$ because
\[
\det
\begin{pmatrix}
1 & 1 \\ \sqrt{k} e^{5\sqrt{k}} & -\sqrt{k}e^{-5\sqrt{k}}
\end{pmatrix}
= -\sqrt{k} \left(e^{-5\sqrt{k}} + e^{5\sqrt{k}}\right) 
< 0
\]
for $k>0$.  Therefore, the null solution is the only solution
of the boundary value problem (\ref{heat_spv_diffc}) for $k>0$.

If $k=0$, the solution of the ordinary differential equation is $F(x)=Ax+B$.
However, $F(0)=0$ yields $B=0$ and $F'(5)=0$ yields $A = 0$.  Thus
$A=B=0$ and again the null solution is the only solution of the
boundary value problem (\ref{heat_spv_diffc}) for $k=0$.

If $k<0$, the roots of the characteristic equation are $\pm i \sqrt{-k}$.
Since the roots are complex, the solution of the ordinary differential equation is of the form
$\displaystyle
F(x) = A \cos\left(\sqrt{-k}\,x\right) + B \sin\left(\sqrt{-k}\,x\right)$.
The condition $F(0)=0$ implies that $A=0$.  Hence, $F'(5)=0$ implies that
$B \sqrt{-k} \cos\left(5\sqrt{-k}\right) = 0$.  If we take
$B=0$, we get the null solution.  We must therefore have
$\cos\left(5\sqrt{-k}\right) = 0$.  This implies that
$\displaystyle k = k_n \equiv -\left((2n-1)\pi/10\right)^2$ for
$n>0$.  The boundary value problem (\ref{heat_spv_diffc}) has
non-null solutions only for
$\displaystyle k = k_n = -\left((2n-1)\pi/10\right)^2$ with
$n$ a positive integer, and the solutions associated to $k_n$ are of the
form $\displaystyle
F(x)=F_n(x) \equiv B_n \sin\left((2n-1)\pi x/10\right)$.

The second ordinary differential equation in (\ref{heat_spv_diffb}) is then
\[
\dydx{G}{t}(t) + 4\left(\frac{(2n-1)\pi}{10}\right)^2\,G(t) = 0 \ ,
\]
where $n$ is a positive integer.  For each $n$, the general solution
of this first order separable ordinary differential equation is
$\displaystyle G(t) = G_n(t) \equiv C_n e^{-t ((2n-1)\pi/5)^2}$.

We have found the simple functions
\[
u_n(x,t) \equiv F_n(x)G_n(t) = b_n\,\sin\left(\frac{(2n-1)\pi x}{10}\right)
\, e^{-t ((2n-1)\pi/5)^2} \quad , \quad n > 0 \ .
\]
The constant $b_n$ comes from the product of the constants $B_n$ and
$C_n$.

\subI{Initial condition}
We seek a solution of the form
\[
u(x,t) = \sum_{n=1}^\infty u_n(x,t)
= \sum_{n=1}^\infty b_n\,\sin\left(\frac{(2n-1)\pi x}{10}\right)
\, e^{-t ((2n-1)\pi/5)^2} \  .
\]

\subQ{a} From
$\displaystyle u(x,0) = 8 \sin\left(\frac{13\pi x}{10}\right)$,
we get
\[
8 \sin\left(\frac{13\pi x}{10}\right)
= \sum_{n=1}^\infty b_n\,\sin\left(\frac{(2n-1)\pi x}{10}\right) \  .
\]
Thus,
\[
b_n =
\begin{cases}
8 & \quad \text{if $n=7$} \\
0 & \quad \text{if $n\neq 7$}
\end{cases}
\]
The solution is
\[
u(x,t) = 8 \sin\left(\frac{13 \pi x}{10}\right)
\, e^{-t (13\pi/5)^2} \  .
\]

\subQ{b} From $u(x,0) = x(5-x)^2$, we get
\[
x(5-x)^2 = \sum_{n=1}^\infty b_n\,\sin\left(\frac{(2n-1)\pi x}{10}\right)
\]
for $0\leq x\leq 5$.  Expanding $f(x) = x(5-x)^2$ on the interval
$[0,5]$ to a odd function of period $20$ as suggested in
Section~\ref{period4L}, we find that
\[
b_n = \frac{2}{5} \int_0^5 x(5-x)^2
\sin\left(\frac{(2n-1)\pi x}{10}\right) \dx{x} \quad , \quad n>0 \ .
\]
A long computation using integration by parts yields
\[
b_n = 4000\left( \frac{4n\pi - 2\pi + (-1)^n 6}
{\pi^4(16n^4 -32 n^3 +24 n^2 -8n +1)}\right) \quad , \quad n > 0 \ .
\]
The solution of the heat equation is
\[
u(x,t) = 
4000\, \sum_{n=1}^\infty \frac{4n\pi - 2\pi + (-1)^n 6}
{\pi^4(16n^4 -32 n^3 +24 n^2 -8n +1)} \,
\sin\left(\frac{(2n-1)\pi x}{10}\right)\, e^{-t ((2n-1)\pi/5)^2} \  .
\]
\end{egg}

\begin{rmk}
By separation of variables, we find that the solutions of the heat
equation for a finite interval with null boundary 
conditions are linear combinations of the following simple functions.
\[
u(x,t) = e^{-c^2 k^2 t}(A \cos (kx) + B \sin(kx)) \ .
\]
The solutions of the heat equation for a rectangular domain with
null boundary conditions are linear combinations of the following
simple functions.
\[
u(x,y,t) = e^{-c^2 k^2 t}(A \cos (\mu x) +B \sin (\mu x))
(C \cos (\eta y) + D \sin (\eta y))
\]
with $k^2 = \mu^2 + \eta^2$.
\end{rmk}

\begin{egg}
Solve the heat equation
\[
\pdydx{u}{t} = c^2 \Delta u = c^2 \left( \pdydxn{u}{x}{2} +
\pdydxn{u}{y}{2} \right) \quad ,
\quad 0<x , y<5 \ \text{and} \ t>0
\]
with $c=3$, the boundary conditions
\begin{align*}
u(0,y,t) &= u(5,y,t) = 0 \quad , \quad 0<y<5 \quad \text{and}
\quad t>0 \ , \\
\pdydx{u}{y}(x,0,t) &= \pdydx{u}{y}(x,5,t) = 0 \quad , \quad
0<x<5 \quad \text{and} \quad t>0 \ ,
\end{align*}
and the initial conditions
\[
u(x,y,0) = 11 \sin\left(\frac{3\pi x}{5}\right) + 17 \sin\left(2\pi x\right)
\cos\left(\frac{7\pi y }{5}\right) \quad , \quad
0\leq x, y \leq 5 \ .
\]

The simple functions are of the form
\[
u(x,y,t) = e^{-c^2 k^2 t}(A \cos (\mu x) + B \sin (\mu x))
(C \cos (\eta y) + D \sin (\eta y)) \ ,
\]
where $k^2 = \mu^2 + \eta^2$.
From $u(0,y,t) = 0$, we get
$\displaystyle A e^{-c^2 k^2 t} (C \cos (\eta y) + D \sin (\eta y)) = 0$.
Since the expression between parentheses cannot be zero
for all $0 \leq y \leq 5$ (i.e.\ we cannot have $C=D=0$), we must have
$A=0$.  From $u(5,y,t) = 0$, we then get
$\displaystyle
B \sin(5\mu) e^{-c^2 k^2 t} (C \cos (\eta y) + D \sin (\eta y)) =0$.
Again, since the expression between parentheses cannot be zero for all
$0 \leq y \leq 5$ and $B$ cannot be $0$, we must have $\sin(5\mu) = 0$.  This
yields $\mu = m\pi/5$ for $m >0$.

Since 
$\displaystyle \pdydx{u}{y} = B \sin\left(\frac{m\pi x}{5}\right) e^{-c^2 k^2 t}
(-C \eta \sin (\eta y) + D \eta \cos (\eta y))$,
we get from $\displaystyle \pdydx{u}{y}(x,0,t) = 0$ that
$\displaystyle B D \eta \sin\left(\frac{m\pi x}{5}\right) e^{-c^2 k^2 t} = 0$
for all $0 \leq x \leq 5$ and $t>0$, where $B\neq 0$.  Thus $D=0$.  From
$\displaystyle \pdydx{u}{y}(x,5,t) = 0$, we then get
$\displaystyle
-B C \eta \sin(5\eta)\sin\left(\frac{m\pi x}{5}\right) e^{-c^2 k^2 t} = 0$
for all $0 \leq x \leq 5$ and $t>0$, where $B\neq 0$.  Since we cannot have
$C=0$, we must have $\sin(5\eta) = 0$.  Thus $\eta = n \pi/5$ for
$n \geq 0$.

We have found the simple functions
\[
u_{n,m}(x,y,t) \equiv A_{n,m} e^{-9 k_{n,m}^2 t} \sin\left(\frac{m\pi x}{5}\right)
\cos\left(\frac{n\pi y}{5}\right) \ ,
\]
where $m$ is a positive integer, $n$ is a non-negative integer (so
$n=0$ is possible), and
\[
k_{n.m}^2 = \left(\frac{m \pi}{5}\right)^2 + \left(\frac{m \pi}{5}\right)^2
= \frac{\pi^2}{5^2} \left(m^2+n^2\right) \ .
\]

We seek a solution of the form
\[
u(x,y,t) = \sum_{m=1}^\infty \sum_{n=0}^\infty u_{n,m}(x,y,t)
= \sum_{m=1}^\infty \sum_{n=0}^\infty
A_{n,m} e^{-9 k_{n,m}^2 t} \sin\left(\frac{m\pi x}{5}\right)
\cos\left(\frac{n\pi y}{5}\right) \ .
\]

From the initial condition
\[
u(x,y,0) = 11 \sin\left(\frac{3\pi x}{5}\right) +
17 \sin\left(\frac{10\pi x}{5}\right)
\cos\left(\frac{7\pi y }{5}\right) \ ,
\]
we get
\[
\sum_{m=1}^\infty \sum_{n=0}^\infty
A_{n,m} \sin\left(\frac{m\pi x}{5}\right) \cos\left(\frac{n\pi y}{5}\right) =
11 \sin\left(\frac{3\pi x}{5}\right) + 17 \sin\left(\frac{10\pi x}{5}\right)
\cos\left(\frac{7\pi y }{5}\right) \ .
\]
Thus,
\[
A_{n,m} =
\begin{cases}
11 & \quad \text{if} \quad m=3\ , \ n=0 \\
17 & \quad \text{if} \quad m=10\ , \ n=7 \\
0 & \quad \text{otherwise}
\end{cases}
\]
Since $k_{n.m} = 9 \pi^2/25$ for $m=3$ and $n=0$, and
$k_{n.m} = 149 \pi^2/25$ for $m=10$ and $n=7$, we get the solution
\[
u(x,y,t) = 11 e^{-81 \pi^2 t/25} \sin\left(\frac{3\pi x}{5}\right)
+ 17 e^{-1341 \pi^2 t/25} \sin\left(\frac{10\pi x}{5}\right)
\cos\left(\frac{7\pi y}{5}\right) \ .
\]
\end{egg}

\begin{egg}
Find the solution of the heat equation
\begin{equation} \label{heat_he_egg_nhc}
\pdydx{u}{t} - c^2 \pdydxn{u}{x}{2} = 0 \quad , \quad 0 < x < L
  \ \text{and} \ t > 0 \ ,
\end{equation}
with the boundary conditions $u(0,t) = t^2$ and $u(L,t) = 1$ for
$t>0$, and the initial condition $\displaystyle u(x,0) = x/L$
for $0 \leq x \leq L$.

We cannot readily use our formulae for the simple functions because
the boundary conditions are not null.  We have first to transform the
problem into one where the formulae can be used.

Let $\phi(x,t) = (1-t^2)x/L + t^2 = (x+(L-x)t^2)/L$.
This is the equation of the line in the $x,u$-plane that contains the
points $(0,t^2)$ and $(L,1)$.  The substitution
$u(x,t) = w(x,t) + \phi(x,t)$ in (\ref{heat_he_egg_nhc}) yields
\begin{equation} \label{heat_he_idx3}
\pdydx{w}{t} - c^2 \pdydxn{w}{x}{2} = \frac{2(x-L)t}{L} \quad ,
\quad 0 < x < L \ \text{and} \ t>0 \ ,
\end{equation}
with the boundary conditions
$w(0,t) = 0$ and $u(L,t) = 0$ for $t>0$, and the initial condition
$w(x,0) = 0$ for $0 \leq x \leq L$.

The simple functions of the heat equation
(i.e.\ when the right hand side of (\ref{heat_he_idx3}) is $0$) are of
the form
$\displaystyle B_n e^{-c^2n^2\pi^2t/L^2} \sin\left(\frac{n\pi x}{L}\right)$
due to the boundary conditions.  So, we seek a solution of the form
\begin{equation} \label{heat_he_idx1}
w(x,t) = \sum_{n=1}^\infty T_n(t) \sin\left(\frac{n\pi x}{L}\right) \ .
\end{equation}
The boundary conditions are satisfied by each term of the series.

We need the Fourier sine series of
$\displaystyle h(x,t) = \frac{2(x-L)t}{L}$
with respect to $x$.  Using integration by parts, we find that the
coefficients of the series are
\begin{align*}
b_n &= \frac{2}{L}\int_0^n \frac{2(x-L)t}{L}
\sin\left(\frac{n\pi x}{L}\right) \dx{x}
= \left( \frac{4t(L-x)}{n\pi L} 
\cos\left(\frac{n\pi x}{L}\right)\right)\bigg|_0^L 
+ \frac{4t}{n\pi L} \int_0^L \cos\left(\frac{n\pi x}{L}\right) \dx{x} \\
&= -\frac{4t}{n\pi} + \frac{4t}{n^2\pi^2} \,
\sin\left(\frac{n\pi x}{L}\right)\bigg|_0^L
= -\frac{4t}{n\pi}  \ .
\end{align*}
Thus,
\begin{equation} \label{heat_he_idx2}
h(x,t) = -\sum_{n=1}^\infty \frac{4t}{n\pi} \sin\left(\frac{n\pi x}{L}\right) \ .
\end{equation}
If we substitute (\ref{heat_he_idx1}) and (\ref{heat_he_idx2}) into
(\ref{heat_he_idx3}), we get
\[
\sum_{n=1}^\infty \left( T_n'(t) + \left(\frac{c n\pi}{L}\right)^2
T_n(t) \right) \sin\left(\frac{n\pi x}{L}\right) =
-\sum_{n=1}^\infty \frac{4t}{n\pi} \sin\left(\frac{n\pi x}{L}\right) \ .
\]
Each $T_n$ must satisfy the ordinary differential equation.
\[
T_n'(t) + \left(\frac{c n\pi}{L}\right)^2 T_n(t) = - \frac{4t}{n\pi}
\quad , \quad n>0 \ .
\]
Those are linear first order ordinary differential equations; namely,
ordinary differential equations of the form 
$\displaystyle \dydx{T}{t} + P(t)T = Q(t)$.
The general solution of this ordinary differential equation is
$\displaystyle T(t) = e^{-\int P(t)\dx{t}}
\left(\int Q(t) e^{\int P(t)\dx{x}} \dx{t} + C \right)$.
Hence,
\[
T_n(t) = e^{-c^2n^2\pi^2 t/L^2} \left( - \int \frac{4t}{n\pi}
e^{c^2n^2\pi^2 t/L^2} \dx{t} + C_n \right)
\]
Using integration by parts to evaluate the integral in the
expression of $T_n$ above, we obtain
\begin{align*}
\int \frac{4t}{n\pi} e^{c^2n^2\pi^2 t/L^2} \dx{t} &=
\left(\frac{4t}{n\pi}\right)\left(\frac{L^2}{c^2n^2\pi^2}\right)
e^{c^2n^2\pi^2 t/L^2}
- \left(\frac{4}{n\pi}\right)\left(\frac{L^2}{c^2n^2\pi^2}\right)
\int e^{c^2n^2\pi^2 t/L^2} \dx{t} \\
&= \left(\frac{4L^2\,t}{c^2n^3\pi^3}\right) e^{c^2n^2\pi^2 t/L^2}
- \left(\frac{4L^4}{c^4n^5\pi^5}\right) e^{c^2n^2\pi^2 t/L^2} \ ,
\end{align*}
where we may ignored the constant of integration since we already have
$C_n$.  Hence,
\[
T_n(t) = -\left(\frac{4L^2\,t}{c^2n^3\pi^3}\right)
+ \left(\frac{4L^4}{c^4n^5\pi^5}\right) + C_n e^{-c^2n^2\pi^2 t/L^2}
\quad , \quad n > 0 \ .
\]

We have found that
\[
w(x,t) = \sum_{n=1}^\infty \left(
-\left(\frac{4L^2\,t}{c^2n^3\pi^3}\right)
+ \left(\frac{4L^4}{c^4n^5\pi^5}\right) + C_n e^{-c^2n^2\pi^2 t/L^2}
\right) \sin\left(\frac{n\pi x}{L}\right) \  .
\]
To determine the $C_n$, we use the initial condition $w(x,0)=0$ to
get
\[
\left(-\left(\frac{4L^2\,t}{c^2n^3\pi^3}\right)
+ \left(\frac{4L^4}{c^4n^5\pi^5}\right) + C_n e^{-c^2n^2\pi^2 t/L^2}
\right)\bigg|_{t=0} = \left(\frac{4L^4}{c^4n^5\pi^5}\right) +
C_n = 0
\]
for all $n$.  Thus
$\displaystyle C_n = -\left(\frac{4L^4}{c^4n^5\pi^5}\right)$ for $n>0$
and
\[
w(x,t) = \sum_{n=1}^\infty \left(\frac{4L^4}{c^4n^5\pi^5}\right)
\left( -\frac{c^2n^2\pi^2}{L^2}\, t + 1 - e^{-c^2n^2\pi^2 t/L^2}\right)
\sin\left(\frac{n\pi x}{L}\right) \  .
\]

The solution of the problem is
\[
u(x.t) = \frac{(1-t^2)x}{L} + t^2 +
\sum_{n=1}^\infty \left(\frac{4L^4}{c^4n^5\pi^5}\right)
\left( -\frac{c^2n^2\pi^2}{L^2}\, t + 1 - e^{-c^2n^2\pi^2 t/L^2}\right)
\sin\left(\frac{n\pi x}{L}\right) \  .
\]
\end{egg}

\subsection{Cylindrical and Spherical Coordinates}

\begin{rmk}
The solutions of the heat equation in a cylindrical domain with null
boundary conditions (this includes the planar circular domain if the
solution is independent of the $z$-coordinate) are linear combinations
of the following simple functions.  If the problem is independent of
$\theta$ and $z$,
\[
u(r,t) = A e^{-c^2 k^2 t} J_0(k r) \ .
\]
If the problem is independent of $z$ only
\[
u(r,\theta,t) = e^{-c^2 k^2 t}
\left(A \cos(n \theta) + B \sin (n \theta)\right) J_n(k r) \quad ,
\quad n \geq 0 \ .
\]
If the problem is independent of $\theta$ only
\[
u(r,z,t) = e^{-c^2 k^2 t}
\left(A \cos (\eta z) + B \sin (\eta z)\right)J_0(\mu r)\
\]
with $k^2 = \mu^2 + \eta^2$.
\end{rmk}

\begin{rmk}
The solutions of the heat equation on a spherical domain with null
boundary conditions are linear combinations of the following simple
functions.  If the problem is independent of $\theta$ and $\phi$,
\[
u(r,t) = e^{-c^2 k^2 t} \left(\frac{A}{k r}
\cos(k r) + \frac{B}{k r} \sin (k r)\right) \ .
\]
\end{rmk}

\section{Addendum}

The heat equation is the archetypal example of a parabolic equation.
It is also used as an example for an evolution or dissipating partial
differential equation.  We are talking of irreversible time-dependent
partial differential equations.  Namely, the partial differential equation
is not invariant under the substitution $t \to -t$.  We cannot go back
in time.

There are many techniques used to study parabolic partial differential
equations.  Some of the classical techniques are energy methods, and
Laplace transforms.  There is also a more modern approach using
semi-group theory.  One of the fundamental results of this approach is
the Hille-Yosida theorem used to determine if a semi-group is strongly
continuous which is an important property to solve parabolic partial
differential equations.  Unfortunately, all these techniques will not
be presented in this book.  A good reference for the semi-group
approach to parabolic partial differential equations is \cite{He}.

For the study of nonlinear (parabolic) partial differential
equations, we recommend \cite{Smo} and, in particular, \cite{He}.

\section{Exercises}

Suggested exercises:

\begin{itemize}
\item In \cite{J}: numbers 4 to 6, 8, 11 and 12 in Section 7.1(a);
number 1 in Section 7.1(c).
\item In \cite{McO}: numbers 6 and 7 in Sections 5.1; numbers 2, 3, 5 to
8, and 11 in Section 5.2.
\item In \cite{PinRub}: numbers 5.1, 5.5 to 5.10, 5.12, 5.13 and 5.20
  in Section 5.7; numbers 6.13 to 6.17, 6.21 to 6.23, and 6.25 in Section 6.7. 
\item In \cite{Str}: all the numbers in Section 2.3; numbers 1 to 4, 8,
  11. 16 to 19 in Section 2.4; numbers 1 to 3 in Section 3.1; all
  the numbers in Section 3.3; numbers 1 and 4 in Section 4.2; numbers 1
  and 8 in Section 5.6.
\end{itemize}

%%% Local Variables: 
%%% mode: latex
%%% TeX-master: "notes"
%%% End: 
