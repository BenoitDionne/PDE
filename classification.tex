\chapter{Classification of Partial Differential
Equations} \label{ChapClassifPDE}

\section{Linear Second Order Partial Differential Equation in
the Plane} \label{classif_reduction_lsoPDE}

Let $\Gamma$ be a curve in $\displaystyle \RR^2$, and let
$f:\Gamma\rightarrow \RR$ and $g:\Gamma\rightarrow \RR$ be two
analytic functions.  We consider the Cauchy Problem of the form
\begin{equation} \label{classif_PDE1}
\begin{split}
& L(x,y,\diff) = a_{2,0}(x,y) \pdydxn{u}{x}{2}(x,y) + 2a_{1,1}(x,y)
\pdydxnm{u}{x}{y}{2}{}{}(x,y) a_{0,2}(x,y) \pdydxn{u}{y}{2}(x,y) \\
&\qquad  + a_{1,0}(x,y) \pdydx{u}{x}(x,y)
+ a_{0,1}(x,y) \pdydx{u}{y}(x,y) + a_{0,0}(x,y)u(x,y) + b(x,y) = 0
\end{split}
\end{equation}
with the conditions
\begin{equation} \label{classif_PDE1_cond}
u\big|_\Gamma = f \quad \text{and} \quad \pdydx{u}{\nu}\bigg|_\Gamma = g \ ,
\end{equation}
where $\displaystyle \nu\left(\VEC{p}\right)$ for $\VEC{p} \in \Gamma$
is a normal unit vector to $\Gamma$ at $\VEC{p}$.
Equation (\ref{classif_PDE1}) is a
{\bfseries linear second order partial differential equation}%
\index{Linear Second Order Partial Differential Equation}. 

We first reduce the Cauchy Problem (\ref{classif_PDE1}) and
(\ref{classif_PDE1_cond}) to one where $\Gamma$ is the $x$ axis.  Let
$\VEC{p}$ be a point on $\Gamma$,  In a neighbourhood of $\VEC{p}$, we
may define two orthogonal families of curves $\eta(x,y)=h$ and
$\xi(x,y)=k$ such that $\Gamma$ is the curve $\xi(x,y)=0$
(Figure~\ref{class_fig1}).  We may also assume that $\VEC{p}$ is the
intersection of the curves $\xi(x,y)=0$ and $\eta(x,y)=0$.

\pdfF{classification/class_fig1}{Change of coordinates to reduce a
linear second order partial differential equation to its standard
form}{The orthogonal families of curves $\eta(x,y)=h$ and $\xi(x,y)=k$
provide a change of coordinates to reduce a linear second order
partial differential equation to a standard form}{class_fig1}

The condition for two families of curves to be transverse in a
neighbourhood of $\VEC{p}$ is that
\begin{equation} \label{classif_PDE2}
\frac{\partial (\eta, \xi)}{\partial (x,y)} =
\begin{pmatrix}
\displaystyle \pdydx{\eta}{x} & \displaystyle \pdydx{\eta}{y} \\[0.7em]
\displaystyle \pdydx{\xi}{x} & \displaystyle \pdydx{\xi}{y}
\end{pmatrix} \neq 0
\end{equation}
in the neighbourhood of $\VEC{p}$.  Effectively, for $\VEC{q}$ near
$\VEC{p}$,
\begin{align*}
\frac{\partial (\eta, \xi)}{\partial (x,y)}\bigg|_{\VEC{q}} = 0
& \Leftrightarrow 
\left(\pdydx{\eta}{x}(\VEC{q}),\pdydx{\eta}{y}(\VEC{q})\right)
\cdot \left(-\pdydx{\xi}{y}(\VEC{q}), \pdydx{\xi}{x}(\VEC{q})\right) = 0 \\
& \Leftrightarrow
\left(-\pdydx{\xi}{y}(\VEC{q}), \pdydx{\xi}{x}(\VEC{q})\right) \ 
\text{is orthogonal to} \ 
\left(\pdydx{\eta}{x}(\VEC{q}),\pdydx{\eta}{y}(\VEC{q})\right) \\
& \Leftrightarrow
\left(\pdydx{\eta}{x}(\VEC{q}),\pdydx{\eta}{y}(\VEC{q})\right) \ 
\text{is parallel to} \ 
\left(\pdydx{\xi}{x}(\VEC{q}),\pdydx{\xi}{y}(\VEC{q})\right)
\end{align*}
because
$\displaystyle \left(-\pdydx{\xi}{y}(\VEC{q}), \pdydx{\xi}{x}(\VEC{q})\right)$
is also orthogonal to
$\displaystyle \left(\pdydx{\xi}{x}(\VEC{q}),\pdydx{\xi}{y}(\VEC{q})\right)$,
This shows that the curves $\xi(x,y)=k$ and $\eta(x,y)=h$ have the
same perpendicular line at $\VEC{q}$, therefore the two curves are
tangent at $\VEC{q}$.

Because of (\ref{classif_PDE2}), we may use the Inverse Function
Theorem to express $x$ and $y$ in a neighbourhood of $\VEC{p}$ as
functions of $\xi$ and $\eta$ in a neighbourhood of the origin.  We
will use this change of variables to reduce (\ref{classif_PDE1}).

If we substitute
\begin{align*}
u(\eta,\xi) & = u(x,y) \ , \quad
\ \pdydx{u}{x} = \pdydx{u}{\xi}\pdydx{\xi}{x} +
\pdydx{u}{\eta}\pdydx{\eta}{x} \ , \quad
\ \pdydx{u}{y} = \pdydx{u}{\xi}\pdydx{\xi}{y} +
\pdydx{u}{\eta}\pdydx{\eta}{y} \ ,\\
\pdydxn{u}{x}{2} &= \pdydxn{u}{\xi}{2} \left(\pdydx{\xi}{x}\right)^2 +
2\pdydxnm{u}{\eta}{\xi}{2}{}{}\pdydx{\eta}{x}\pdydx{\xi}{x}
+ \pdydxn{u}{\eta}{2}\left(\pdydx{\eta}{x}\right)^2 +
\pdydx{u}{\xi} \pdydxn{\xi}{x}{2} + \pdydx{u}{\eta}\pdydxn{\eta}{x}{2}  \ ,\\
\pdydxn{u}{y}{2} &= \pdydxn{u}{\xi}{2} \left(\pdydx{\xi}{y}\right)^2 +
2\pdydxnm{u}{\eta}{\xi}{2}{}{}\pdydx{\eta}{y}\pdydx{\xi}{y}
+ \pdydxn{u}{\eta}{2}\left(\pdydx{\eta}{y}\right)^2 +
\pdydx{u}{\xi} \pdydxn{\xi}{y}{2} + \pdydx{u}{\eta}\pdydxn{\eta}{y}{2}
\intertext{and}
\pdydxnm{u}{x}{y}{2}{}{} &= \pdydxn{u}{\xi}{2}
\pdydx{\xi}{x}\pdydx{\xi}{y} +
\pdydxnm{u}{\eta}{\xi}{2}{}{}\pdydx{\eta}{y}\pdydx{\xi}{x}
+\pdydxnm{u}{\eta}{\xi}{2}{}{}\pdydx{\eta}{x}\pdydx{\xi}{y}
+\pdydxn{u}{\eta}{2}\pdydx{\eta}{x}\pdydx{\eta}{y} +
\pdydx{u}{\xi} \pdydxnm{\xi}{y}{x}{2}{}{}
+ \pdydx{u}{\eta}\pdydxnm{\eta}{y}{x}{2}{}{}
\end{align*}
in (\ref{classif_PDE1}), we get
\begin{equation} \label{classif_PDE3}
A_{2,0} \pdydxn{u}{\eta}{2}
+ 2A_{1,1}\pdydxnm{u}{\eta}{\xi}{2}{}{}
+ A_{0,2}\pdydxn{u}{\xi}{2}
+ A_{1,0} \pdydx{u}{\eta}
+ A_{0,1} \pdydx{u}{\xi} + A_{0,0} u + B = 0 \ ,
\end{equation}
where
\begin{align*}
A_{2,0}(\eta,\xi) &= a_{2,0}\, \left(\pdydx{\eta}{x}\right)^2 
+2a_{1,1}\,\pdydx{\eta}{x}\pdydx{\eta}{y}
+ a_{0,2}\, \left(\pdydx{\eta}{y}\right)^2 \ , \\
A_{1,1}(\eta,\xi) &= a_{2,0}\, \pdydx{\eta}{x}\pdydx{\xi}{x} + a_{1,1}\, 
\left(\pdydx{\eta}{y}\pdydx{\xi}{x}+\pdydx{\eta}{x}\pdydx{\xi}{y}\right)
+ a_{0,2}\, \pdydx{\eta}{y}\pdydx{\xi}{y} \ , \\
A_{0,2}(\eta,\xi) &= a_{2,0}\,\left(\pdydx{\xi}{x}\right)^2
+ 2a_{1,1}\, \pdydx{\xi}{x}\pdydx{\xi}{y}
+ a_{0,2}\,\left(\pdydx{\xi}{y}\right)^2 \ , \\
A_{1,0}(\eta,\xi) &= a_{2,0}\,\pdydxn{\eta}{x}{2} +
2a_{1,1}\,\pdydxnm{\eta}{x}{y}{2}{}{} + a_{0,2}\, \pdydxn{\eta}{y}{2}
+a_{1,0} \, \pdydx{\eta}{x} + a_{0,1}\,\pdydx{\eta}{y} \ , \\
A_{0,1}(\eta,\xi) &= a_{2,0}\,\pdydxn{\xi}{x}{2} +
2a_{1,1}\,\pdydxnm{\xi}{x}{y}{2}{}{} + a_{0,2}\, \pdydxn{\xi}{y}{2}
+a_{1,0} \, \pdydx{\xi}{x} + a_{0,1}\,\pdydx{\xi}{y} \ , \\
A_{0,0}(\eta,\xi) &= a_{0,0} \quad \text{and} \quad
B(\eta,\xi) = b(x,y)
\end{align*}
with the right hand side of each expression evaluated at
$(x,y) = \left(x(\eta,\xi),y(\eta,\xi)\right)$.

The initial condition given in (\ref{classif_PDE1_cond}) becomes
\begin{equation} \label{classif_PDE3_cond}
u(\eta,0) = f\left(x(\eta,0),y(\eta,0)\right) \quad \text{and} \quad
\pdydx{u}{\xi}(\eta,0) = g\left(x(\eta,0),y(\eta,0)\right) \ .
\end{equation}

As we will see in Chapter~\ref{ChapterCauchyP}, to find the series
expansion of the solution $u$ of (\ref{classif_PDE3}) and
(\ref{classif_PDE3_cond}) in a neighbourhood of the origin in the
$\eta$,$\xi$ plane (so near $\VEC{p}$ in the $x$,$y$ plane), we need to compute
$\displaystyle \pdydxn{u}{\xi}{j}$ for $j\geq 2$ at the origin.  For 
that, we assume that (\ref{classif_PDE3}) can be solved for
$\displaystyle \pdydxn{u}{\xi}{2}$ at $(\eta,\xi)=(0,0)$.  This is
possible only if
\begin{equation} \label{classif_PDE9}
A_{0,2}(0,0) = a_{2,0}(\VEC{p})\, \left(\pdydx{\xi}{x}(\VEC{p})\right)^2
+2a_{1,1}(\VEC{p})\,\pdydx{\xi}{x}(\VEC{p})\,\pdydx{\xi}{y}(\VEC{p})
+ a_{0,2}(\VEC{p})\, \left(\pdydx{\xi}{y}(\VEC{p})\right)^2
\neq 0 \ .
\end{equation}
This condition is fundamental and is the subject of the next section.

Before moving on to the next section, we note that
\begin{equation} \label{classif_PDE8}
A_{1,1}^2 - A_{2,0}\,A_{0,2} = \left(a_{1,1}^2 - a_{2,0}a_{0,2}\right)
\left( \pdydx{\eta}{x}\pdydx{\xi}{y} - \pdydx{\eta}{y}\pdydx{\xi}{x}
\right)^2
\end{equation}
with the left hand side evaluated at $(\eta,\xi)$ and the right hand
side evaluated at $(x,y) = \left(x(\eta,\xi),y(\eta,\xi)\right)$.
This identity will be used later for the classification of the linear second
order partial differential equations.  The reader should verify this identity.

\section{Characteristic curves}

As we will see later, not all initial curves $\Gamma$ will yield a
well posed Cauchy problem formed of (\ref{classif_PDE1}) and
(\ref{classif_PDE1_cond}); namely, one with a unique solution. 

\begin{defn}
A {\bfseries characteristic curve}\index{Characteristic Curve} for the
Cauchy problem (\ref{classif_PDE1}) and (\ref{classif_PDE1_cond}) is a
curve $\Gamma$ such that it is not possible from
(\ref{classif_PDE1}) and (\ref{classif_PDE1_cond}) to 
determine all the second order (mixed or not) derivatives of $u$ (in
the neighbourhood of $\Gamma$).
\end{defn}

The previous definition is a little bit vague and does not provide precise
conditions to determine if a curve is a characteristic curve or not.
We now find such conditions.  Suppose that
\begin{equation} \label{classif_GammaEq1}
\Gamma = \left\{ (x_0(s), y_0(s)) : s \in I \right\}
\end{equation}
for two differentiable functions $x_0:I\rightarrow \RR$ and
$y_0:I\rightarrow \RR$ defined on an open interval $I$.
Then, the derivative of the initial condition
$u(x_0(s),y_0(s)) = f(x_0(s),y_0(s))$ yields
\begin{equation} \label{classif_class1}
\dfdx{f(x_0(s),y_0(s))}{s} =
\pdydx{u}{x}(x_0(s),y_0(s))\,x_0'(s) + \pdydx{u}{y}(x_0(s),y_0(s))\,
y_0'(s) \ .
\end{equation}
Moreover,
\begin{align}
&g(x_0(s),y_0(s)) = \pdydx{u}{\nu}(x_0(s),y_0(s))
= \graD u(x_0(s),y_0(s)) \cdot
\nu(x_0(s),y_0(s)) \nonumber \\
&\quad = \pdydx{u}{x}(x_0(s),y_0(s))\,\nu_1(x_0(s),y_0(s)) +
\pdydx{u}{y}(x_0(s),y_0(s))\,\nu_2(x_0(s),y_0(s)) \ .
\label{classif_class2}
\end{align}
The system of equations (\ref{classif_class1}) and
(\ref{classif_class2}) determine $\displaystyle \pdydx{u}{x}$ and 
$\displaystyle \pdydx{u}{y}$ uniquely along $\Gamma$ because
\begin{align*}
&\det
\begin{pmatrix}
x_0'(s) & y_0'(s) \\
\nu_1(x_0(s),y_0(s)) & \nu_2(x_0(s),y_0(s))
\end{pmatrix} \\
&\quad = \big(-y_0'(s),x_0'(s)\big)\cdot \big(\nu_1(x_0(s),y_0(s)),
\nu_2(x_0(s),y_0(s))\big) = \epsilon \left\|(y_0'(s),x_0'(s))\right\|
\neq  0  \ ,
\end{align*}
where $\epsilon = 1$ or $-1$ according to the directions of
$(-y_0'(s),x_0'(s))$ and $\nu(x_0(s),y_0(s))$.  We have that
$(-y_0'(s),x_0'(s))$ is perpendicular to the curve $\Gamma$ at
$(x_0(s),y_0(s))$ and so parallel to the unit vector
$\nu(x_0(s),y_0(s))$.

We may therefore assume that, instead of
$\displaystyle \pdydx{u}{\nu}\bigg|_\Gamma = g$, we are given
\begin{equation} \label{classif_class3}
\pdydx{u}{x}\bigg|_\Gamma = g_1 \quad \text{and} \quad 
\pdydx{u}{y}\bigg|_\Gamma = g_2 \ .
\end{equation}
For consistency, we have from (\ref{classif_class1}) that
\[
\dfdx{f(x_0(s),y_0(s))}{s} =
g_1((x_0(s),y_0(s)))\,x_0'(s) + g_2(x_0(s),y_0(s))\, y_0'(s) \ .
\]

The derivatives $\displaystyle \pdydxn{u}{x}{2}$,
$\displaystyle \pdydxnm{u}{x}{y}{2}{}{}$ and
$\displaystyle \pdydxn{u}{y}{2}$ along $\Gamma$ are the solutions (if
there are solutions) of the linear system of equation formed of
(\ref{classif_PDE1}) and the derivatives with respect to $s$ of the 
equations in (\ref{classif_class3}); namely,
\begin{align*}
& a_{2,0}(x_0(s),y_0(s)) \pdydxn{u}{x}{2}(x_0(s),y_0(s)) +
2a_{1,1}(x_0(s),y_0(s)) \pdydxnm{u}{x}{y}{2}{}{}(x_0(s),y_0(s)) \\
& \quad + a_{0,2}(x_0(s),y_0(s)) \pdydxn{u}{y}{2}(x_0(s),y_0(s)) = -
a_{1,0}(x_0(s),y_0(s)) \pdydx{u}{x}(x_0(s),y_0(s)) \\
& \quad - a_{0,1}(x_0(s),y_0(s)) \pdydx{u}{y}(x_0(s),y_0(s))
- a_{0,0}(x_0(s),y_0(s))u(x_0(s),y_0(s)) - b(x_0(s),y_0(s)) \ , \\
&\pdydxn{u}{x}{2}(x_0(s),y_0(s))\, x_0'(s) +
\pdydxnm{u}{x}{y}{2}{}{}(x_0(s),y_0(s))\, y_0'(s) = \dfdx{g_1}{s}(s)
\intertext{and}
&\pdydxnm{u}{x}{y}{2}{}{}(x_0(s),y_0(s))\, x_0'(s) +
\pdydxn{u}{y}{2}(x_0(s),y_0(s))\, y_0'(s) = \dfdx{g_2}{s}(s) \ .
\end{align*}

This system has a (unique) solution if and only if
\begin{align}
&\det
\begin{pmatrix}
a_{2,0}(x_0(s),y_0(s)) & 2a_{1,1}(x_0(s),y_0(s)) & a_{0,2}(x_0(s),y_0(s)) \\
x_0'(s) & y_0'(s) & 0 \\
0 & x_0'(s) & y_0'(s)
\end{pmatrix} \nonumber \\
& \qquad = a_{2,0}(x_0(s),y_0(s)) \left(y_0'(s)\right)^2
-2 a_{1,1}(x_0(s),y_0(s)) x_0'(s) y_0'(s) \nonumber \\
& \qquad\qquad + a_{0,2}(x_0(s),y_0(s)) \left(x_0'(s)\right)^2 \neq 0
\ . \label{classif_se_pds}
\end{align}

This motivates the following definition.

\begin{defn}
The {\bfseries principal symbol}\index{Principal Symbol}
of (\ref{classif_PDE1}) is the quadratic form
\begin{align*}
Q(x,y,\zeta_1,\zeta_2) &= a_{2,0}(x,y) \zeta_1^2 +
2a_{1,1}(x,y)\zeta_1\zeta_2 + a_{0,2}(x,y) \zeta_2^2 \\
&=
\begin{pmatrix}
\zeta_1 & \zeta_2
\end{pmatrix}
\begin{pmatrix}
a_{2,0}(x,y) & a_{1,1}(x,y) \\
a_{1,1}(x,y) & a_{0,2}(x,y)  
\end{pmatrix}
\begin{pmatrix}
\zeta_1 \\ \zeta_2
\end{pmatrix} \ .
\end{align*}
for $(x,y)$ and $(\zeta_1,\zeta_2)$ in $\displaystyle \RR^2$.
\end{defn}

We have proved the following proposition.

\begin{prop}
A curve $\Gamma$ given by (\ref{classif_GammaEq1}) is a characteristic
curve if and only if
$\displaystyle Q(x_0(s),y_0(s), -y_0'(s) , x_0'(s)) = 0$
for all $s \in I$.
\end{prop}

Since $\displaystyle \left(-y_0'(s) , x_0'(s)\right)$ is orthogonal
to the curve $\Gamma$ at $\displaystyle \left(x_0(s), y_0(s)\right)$.
The condition above can be restated as follows.
A curve $\Gamma$ is a characteristic curve if and only if
at each point $(x,y)$ of $\Gamma$ we have that
$Q(x,y, \zeta_1 , \zeta_2) = 0$ for all orthogonal vectors
$\VEC{\zeta}$ to the curve $\Gamma$ at $(x,y)$.

If $\Gamma$ is given implicitly as $\mu(x,y) = c$ for some constant
$c$, the condition above becomes the following.
A curve $\Gamma$ is a characteristic curve if and only if
\[
Q\left(x,y, \pdydx{\mu}{x}(x,y), \pdydx{\mu}{y}(x,y) \right) = 0
\]
for all $(x,y) \in \Gamma$.

In Section~\ref{classif_reduction_lsoPDE}, we have found that
(\ref{classif_PDE9}) was necessary and sufficient to reduce
the Cauchy problem (\ref{classif_PDE1}) and (\ref{classif_PDE1_cond}) to
one where the curve $\Gamma$ is replaced by an axis and where we can
compute all the partial derivatives of $u$.  The condition in
(\ref{classif_PDE9}) can be expressed with the principal symbol for
(\ref{classif_PDE1}); namely, it is the condition
\[
Q\left(p_1,p_2, \pdydx{\xi}{x}(p_1,p_2), \pdydx{\xi}{y}(p_1,p_2) \right)
\neq 0 \ .
\]
This simply says that $\Gamma$ is not a characteristic curve near
$\VEC{p} = (p_1,p_2)$.

If the curve $\Gamma$ for the conditions (\ref{classif_PDE1_cond}) of
(\ref{classif_PDE1}) is a characteristic curve, the Cauchy Problem will
not be well posed.  We may end up with no possible solution or many
solutions.

\section{Classification of Linear Second Order Partial
Differential Equation in the Plane} \label{classif_class_plane}

The second order linear partial differential equations can be classify
(at each point) into three categories.  The existence of
characteristic curves differs from one category to the other.

\begin{defn} \label{classif_EHP}
Let
\[
C(x,y) = \det \begin{pmatrix}
a_{2,0}(x,y) & a_{1,1}(x,y) \\
a_{1,1}(x,y) & a_{0,2}(x,y)  
\end{pmatrix} = a_{2,0}(x,y)a_{0,2}(x,y) - a_{1,1}^2(x,y) \  .
\]
\begin{enumerate}
\item The partial differential equation in (\ref{classif_PDE1}) is
{\bfseries elliptic}\index{Elliptic Partial Differential
Equation!Second Order} at $(x,y)$ if $C(x,y)>0$.
\item The partial differential equation in (\ref{classif_PDE1}) is
{\bfseries hyperbolic}%
\index{Hyperbolic Partial Differential Equation!Second Order}
at $(x,y)$ if $C(x,y)<0$.
\item The partial differential equation in (\ref{classif_PDE1}) is
{\bfseries parabolic}\index{Parabolic Partial Differential
Equation!Second Order} at $(x,y)$ if $C(x,y)=0$. 
\end{enumerate}
\end{defn}

Since the principal symbol depends on $x$ and $y$, a partial
differential equation may be elliptic in one region of the plane,
parabolic in a second region of the plane, and hyperbolic in a third
region of the plane.

The two families of curves $\xi(x,y)=k$ and $\eta(x,y)=h$, used in
Section~\ref{classif_reduction_lsoPDE} to reduce the linear second
order partial differential equation in (\ref{classif_PDE1}) to a
simpler form, do not have to be
orthogonal.  Moreover, $\Gamma$ does not have to be a member of one of
the families.  The only requirement is that the curves of one family
intersect the curves of the other family transversely.  Obviously, we
lose control of the transformations on $\Gamma$ and the conditions
on $\Gamma$.  In particular,
the condition $\displaystyle \pdydx{u}{\nu}\bigg|_\Gamma = g$ may not
be reduced to the simple expression
$\displaystyle \pdydx{u}{\xi}(\eta,0) = g\left(x(\eta,0),y(\eta,0)\right)$.
The derivative with respect to $\xi$ may no longer correspond to a
directional derivative perpendicular to $\Gamma$.

From (\ref{classif_PDE8}), we have that the change of coordinates
provided by the transversal families of curves $\xi(x,y)=k$ and
$\eta(x,y)=h$ in Section~\ref{classif_reduction_lsoPDE} does not
change the category of the partial differential equation.

To study the properties of the categories of partial differential
equation, we will only use families of curves that satisfy the
transversal condition when they intersect.  In some cases, we will
even chose families of characteristic curves.

\subsection{Elliptic Equations}

If (\ref{classif_PDE1}) is elliptic on an open set
$\displaystyle U \subset \RR^2$, then we have
\[
\det \begin{pmatrix}
a_{2,0}(x,y) & a_{1,1}(x,y) \\
a_{1,1}(x,y) & a_{0,2}(x,y)  
\end{pmatrix}
>0
\]
for $(x,y) \in U$.  Thus, the principal symbol is either
{\bfseries positive definite}\index{Positive Definite}
for all $(x,y) \in U$ or
{\bfseries negative definite}\index{Negative Definite}
for all $(x,y) \in U$; namely, $Q(x,y,\zeta_1,\zeta_2) > 0$ for all
$(\zeta_1,\zeta_2) \in \RR^2 \setminus \{(0,0)\}$ or
$Q(x,y,\zeta_1,\zeta_2) < 0$ for all
$(\zeta_1,\zeta_2) \in \RR^2 \setminus \{ (0,0)\}$ respectively.  There is no
point other than the origin where the principal symbol is
null. Therefore, there are not characteristic curves and, in
particular, any curve $\Gamma$ can be used for the conditions
(\ref{classif_PDE1_cond}).

We can reduce (\ref{classif_PDE1}) to a simpler form.  Consider the
polynomial
\begin{equation} \label{classif_PDE4}
P(x,y,\phi) = a_{2,0}(x,y) \phi^2(x,y) - 2a_{1,1}(x,y) \phi(x,y)
+ a_{0,2}(x,y) \ .
\end{equation}
This polynomial is suggested by the condition (\ref{classif_se_pds})
used to motivate the principal symbol.  There is no real
root for this polynomial but there are two distinct complex roots:
\[
\phi_{\pm}(x,y) = \frac{a_{1,1}(x,y) \pm i
\sqrt{ a_{2,0}(x,y)a_{0,2}(x,y)-a_{1,1}^2(x,y)}}{a_{2,0}(x,y)} \ .
\]
If we solve $\displaystyle \dydx{y}{x} = \phi_{+}(x,y)$, we get a family
of curves $\displaystyle \psi(x,y) = c \in \CC$.
Since
\[
\dydx{y}{x} = \phi_{+}(x,y) = -\pdydx{\psi}{x} \bigg/ \pdydx{\psi}{y} \ ,
\]
we get from (\ref{classif_PDE4}) that
\begin{equation} \label{classif_PDE5}
a_{2,0}(x,y) \left(\pdydx{\psi}{x}\right)^2 +
2a_{1,1}(x,y) \pdydx{\psi}{x}\pdydx{\psi}{y} + a_{0,2}(x,y)
\left(\pdydx{\psi}{y}\right)^2 = 0 \ .
\end{equation}

Let $\displaystyle
\eta(x,y) = \RE \psi(x,y) = \frac{ \psi(x,y)+ \overline{\psi(x,y)}}{2}$
and
$\displaystyle \xi(x,y) = \IM \psi(x,y)
= \frac{ \psi(x,y) -\overline{\psi(x,y)}}{2i}$.
We consider the families of curves
\[
\eta(x,y) = h \quad \text{and} \quad \xi(x,y) = k \ ,
\]
where $h$ and $k$ are real constants.  Since
\[
\pdydx{\psi}{x} \bigg/ \pdydx{\psi}{y} = -\phi_{+}(x,y) \neq -\phi_{-}(x,y)
= \overline{ \left(\pdydx{\psi}{x} \bigg/ \pdydx{\psi}{y}\right)}
= \pdydx{\overline{\psi}}{x} \bigg/ \pdydx{\overline{\psi}}{y}
\]
because $\IM \phi_{\pm} \neq 0$, we have
\begin{align*}
\frac{\partial(\eta,\xi)}{\partial(x,y)}
&= \pdydx{\eta}{x}\pdydx{\xi}{y} - \pdydx{\eta}{y}\pdydx{\xi}{x}
=\frac{1}{4i}\left(\pdydx{\psi}{x}+ \overline{\pdydx{\psi}{x}}\right)
\left(\pdydx{\psi}{y} - \overline{\pdydx{\psi}{y}}\right)
- \frac{1}{4i}\left(\pdydx{\psi}{y}+ \overline{\pdydx{\psi}{y}}\right)
\left(\pdydx{\psi}{x} - \overline{\pdydx{\psi}{x}}\right) \\
&= - \frac{1}{2i} \left( \pdydx{\psi}{x}\overline{\pdydx{\psi}{y}}
- \pdydx{\psi}{y}\overline{\pdydx{\psi}{x}}\right)
\neq 0
\end{align*}
and so (\ref{classif_PDE2}) is satisfied.

We get from (\ref{classif_PDE5}) that
\begin{align*}
0 &= a_{2,0}(x,y) \left(\pdydx{\eta}{x}+i\pdydx{\xi}{x}\right)^2 \! +
2 a_{1,1}(x,y) \left(\pdydx{\eta}{x}+i\pdydx{\xi}{x}\right)
\!\! \left(\pdydx{\eta}{y}+i\pdydx{\xi}{y}\right) \!
+ a_{0,2}(x,y)\left(\pdydx{\eta}{y}+i\pdydx{\xi}{y}\right)^2 \\
&= \left( a_{2,0}(x,y)\,\left(\pdydx{\eta}{x}\right)^2
+ 2a_{1,1}(x,y)\, \pdydx{\eta}{x}\pdydx{\eta}{y}
+ a_{0,2}(x,y)\, \left(\pdydx{\eta}{y}\right)^2 \right) \\
&\qquad - \left( a_{2,0}(x,y)\,\left(\pdydx{\xi}{x}\right)^2
+ 2a_{1,1}(x,y)\, \pdydx{\xi}{x}\pdydx{\xi}{y}
+ a_{0,2}(x,y)\, \left(\pdydx{\xi}{y}\right)^2 \right) \\
&\qquad + 2i\left( a_{2,0}(x,y)\,\pdydx{\eta}{x}\pdydx{\xi}{x}
+ a_{1,1}(x,y)\, \left( \pdydx{\eta}{y}\pdydx{\xi}{x}
+\pdydx{\eta}{x}\pdydx{\xi}{y}\right)
+a_{0,2}(x,y)\,\pdydx{\eta}{y}\pdydx{\xi}{y}\right) \ .
\end{align*}
The real and imaginary parts of this expression yield
\begin{align*}
& a_{2,0}(x,y)\,\left(\pdydx{\eta}{x}\right)^2
+ 2a_{1,1}(x,y)\, \pdydx{\eta}{x}\pdydx{\eta}{y}
+ a_{0,2}(x,y)\, \left(\pdydx{\eta}{y}\right)^2 \\
& \qquad = a_{2,0}(x,y)\,\left(\pdydx{\xi}{x}\right)^2
+ 2a_{1,1}(x,y)\, \pdydx{\xi}{x}\pdydx{\xi}{y}
+ a_{0,2}(x,y)\, \left(\pdydx{\xi}{y}\right)^2
\end{align*}
and
\[
a_{2,0}(x,y)\,\pdydx{\eta}{x}\pdydx{\xi}{x}
+ a_{1,1}(x,y)\, \left( \pdydx{\eta}{y}\pdydx{\xi}{x}
+\pdydx{\eta}{x}\pdydx{\xi}{y}\right)
+a_{0,2}(x,y)\,\pdydx{\eta}{y}\pdydx{\xi}{y} = 0 \ .
\]
With this change of coordinates, (\ref{classif_PDE1}) becomes
(\ref{classif_PDE3}) with $A_{2,0} = A_{0,2} \neq 0$ (since $Q$ is
positively defined) and $A_{1,1} = 0$.  If we divide by $A_{2,0}$, we
get an equation of the form
\begin{equation} \label{classif_StandEll}
\pdydxn{u}{\eta}{2} + \pdydxn{u}{\xi}{2}
+ \alpha_{1,0}(\eta,\xi) \pdydx{u}{\eta}
+ \alpha_{0,1}(\eta,\xi) \pdydx{u}{\xi} + \alpha_{0,0}(\eta,\xi)u +
\beta (\eta,\xi) = 0 \ , 
\end{equation}
where $\displaystyle \alpha_{1,0} = A_{1,0}/A_{2,0}$, etc.
(\ref{classif_StandEll}) is the
{\bfseries standard form}\index{Elliptic Partial Differential
Equation!Standard Form} of an elliptic equation.

\begin{egg}
The Laplace Equation
\[
\pdydxn{u}{x}{2} + \pdydxn{u}{y}{2} = 0
\]
is an example of an elliptic equation on all $\displaystyle \RR^2$.
\end{egg}

\subsection{Hyperbolic Equations} \label{classif_class_planeHyper}

Suppose that (\ref{classif_PDE1}) is hyperbolic on an open set of
$\displaystyle \RR^2$, then we have
\[
\det \begin{pmatrix}
a_{2,0}(x,y) & a_{1,1}(x,y) \\
a_{1,1}(x,y) & a_{0,2}(x,y)  
\end{pmatrix}
<0
\]
for $(x,y)$ in this set.  Thus, the principal symbol represents
a saddle point in $(\zeta_1,\zeta_2)$.  There are two lines in
$\displaystyle \RR^2$ along which the principal symbol is null.
Therefore, there are characteristic curves and not all curves $\Gamma$
can be used for the conditions (\ref{classif_PDE1_cond}).

We can reduce (\ref{classif_PDE1}) to a simpler form.  Consider again
the polynomial
\begin{equation} \label{classif_PDE6}
P(x,y,\phi) = a_{2,0}(x,y) \phi^2(x,y) - 2a_{1,1}(x,y) \phi(x,y)
+ a_{0,2}(x,y) \ .
\end{equation}
The two real roots of this polynomial are
\[
\phi_{\pm}(x,y) = \frac{a_{1,1}(x,y) \pm
\sqrt{a_{1,1}^2(x,y)-a_{2,0}(x,y)a_{0,2}(x,y)}}{a_{2,0}(x,y)} \ .
\]

Consider the two families of curves $\eta(x,y)=h$ and $\xi(x,y)=k$
where the first family of curve is given by the implicit solution of
$\displaystyle \dydx{y}{x} = \phi_{+}(x,y)$ and the second family of
curves is given by the implicit solution of
$\displaystyle \dydx{y}{x} = \phi_{-}(x,y)$.

The relation in (\ref{classif_PDE2}) is satisfied because
\[
\pdydx{\eta}{x}\bigg/\pdydx{\eta}{y}
= - \phi_{+}(x,y) \neq -\phi_{-}(x,y) = \pdydx{\xi}{x}\bigg/ \pdydx{\xi}{y} \ ,
\]

If we substitute $\phi_{\pm}$ in (\ref{classif_PDE6}), we get
\begin{align*}
a_{2,0}(x,y) \,\left(\pdydx{\eta}{x}\right)^2 + 2a_{1,1}(x,y)\,
\pdydx{\eta}{x} \pdydx{\eta}{y}  +
a_{0,2}(x,y)\,\left(\pdydx{\eta}{y}\right)^2 &= 0
\intertext{and}
a_{2,0}(x,y) \,\left(\pdydx{\xi}{x}\right)^2 + 2a_{1,1}(x,y)\,
\pdydx{\xi}{x} \pdydx{\xi}{y}  +
a_{0,2}(x,y)\,\left(\pdydx{\xi}{y}\right)^2 &= 0 \ .
\end{align*}
These two equations are
\[
Q\left(x,y,\pdydx{\eta}{x}(x,y),\pdydx{\eta}{y}(x,y)\right) = 0 \quad
\text{and} \quad
Q\left(x,y,\pdydx{\xi}{x}(x,y),\pdydx{\xi}{y}(x,y)\right) = 0
\]
respectively.  Hence, the curves of the two families are
characteristic curves.

With the change of coordinates $(x,y)=(x(\eta,\xi),y(\eta,\xi))$,
equation (\ref{classif_PDE1}) is reduced to
(\ref{classif_PDE3}) with $A_{2,0} = A_{0,2} = 0$ and $A_{1,1} \neq 0$.
If we divide by $A_{1,1}$, we get an equation of the form
\begin{equation} \label{classif_StandHyp}
\pdydxnm{u}{\eta}{\xi}{2}{}{}
+ \alpha_{1,0}(\eta,\xi) \pdydx{u}{\eta}
+ \alpha_{0,1}(\eta,\xi) \pdydx{u}{\xi} + \alpha_{0,0}(\eta,\xi)u +
\beta (\eta,\xi) = 0 \  , 
\end{equation}
where $\displaystyle \alpha_{1,0} = A_{1,0}/A_{1,1}$, etc.
(\ref{classif_StandHyp}) is the
{\bfseries standard form}\index{Hyperbolic Partial Differential
Equation!Standard Form} of an hyperbolic equation.

\begin{egg}
The wave equation
\[
\pdydxn{u}{t}{2} = c^2 \pdydxn{u}{x}{2}
\]
is an example of an hyperbolic equation on all $\displaystyle \RR^2$.
We will study it extensively later.
\end{egg}

\begin{egg}
Consider the partial differential equation
\begin{equation}\label{classif_k3}
\pdydxnm{u}{x}{y}{2}{}{} + 2 \pdydx{u}{x} = 0 \ .
\end{equation}
This is an hyperbolic equation already in its standard form.  With
$\displaystyle p = \pdydx{u}{x}$, (\ref{classif_k3}) becomes
\[
\pdydx{u}{y} + 2 p = 0 \ .
\]
For $x$ fixed, this is a separable ordinary differential equation
whose solution is $\displaystyle p = A(x) e^{-2y}$, where $A$ is an
arbitrary function.  We now consider
\[
\pdydx{u}{x} = A(x) e^{-2y} \ .
\]
We can integrate both sides with respect to $x$ to get
\[
u = \underbrace{\left(\int A(x) \dx{x}\right)}_{\equiv B(x)} e^{-2y} + C(y)
= B(x) e^{-2y} +C(y) \ ,
\]
where $C$ is an arbitrary functions.
\end{egg}

\begin{egg}
The partial differential equation
\begin{equation} \label{classif_egg_hypk1}
\pdydxnm{u}{x}{y}{2}{}{} - \pdydxn{u}{y}{2} = 0
\end{equation}
is of the form (\ref{classif_PDE1}) with
$a_{2,0}(x,y) = 0$, $a_{1,1}(x,y) = 1/2$ and $a_{0,2}(x,y) = -1$.
Hence
\[
\det \begin{pmatrix}
a_{2,0}(x,y) & a_{1,1}(x,y) \\
a_{1,1}(x,y) & a_{0,2}(x,y)  
\end{pmatrix}
=
\det \begin{pmatrix}
0 & 1/2 \\
1/2 & -1  
\end{pmatrix}
= -1/4 < 0
\]
and (\ref{classif_egg_hypk1}) is hyperbolic.  To get the families of
curves used to reduce this hyperbolic partial differential equation to
its standard form, we need to interchange the role of $x$ and $y$ in
the previous discussion.  We use
\[
\phi_{\pm}(x,y) = \frac{a_{1,1}(x,y) \pm
\sqrt{a_{1,1}^2(x,y)-a_{2,0}(x,y)a_{0,2}(x,y)}}{a_{0,2}(x,y)} = -1/2 \mp 1/2 \ .
\]
We get a family of curves $\eta(x,y) = h$ from the implicit
solution of $\displaystyle \dydx{x}{y} = -1$ and another family of curves
$\xi(x,y) = k$ from the implicit solution of
$\displaystyle \dydx{x}{y} = 0$.  We respectively get
$x = -y + \eta$ and $x=\xi$.  Hence $\eta(x,y)=x+y$ and $\xi(x,y) = x$.

Since $\displaystyle \pdydx{\eta}{x} = 1$, $\displaystyle \pdydx{\eta}{y} = 1$,
$\displaystyle \pdydx{\xi}{x} = 1$ and $\displaystyle \pdydx{\xi}{y} = 0$,
we get
\begin{align*}
\pdydx{u}{y} &= \pdydx{u}{\eta} \pdydx{\eta}{y} +
\pdydx{u}{\xi}\pdydx{\xi}{y} = \pdydx{u}{\eta} \quad , \quad
\pdydxn{u}{y}{2} = \pdfdx{ \left(\pdydx{u}{y}\right) }{y}
= \pdydxn{u}{\eta}{2} \pdydx{\eta}{y}
+ \pdydxnm{u}{\xi}{\eta}{2}{}{} \pdydx{\xi}{y}
= \pdydxn{u}{\eta}{2}
\intertext{and}
\pdydxnm{u}{x}{y}{2}{}{} &= \pdfdx{ \left(\pdydx{u}{y}\right) }{x}
=  \pdydxn{u}{\eta}{2}\pdydx{\eta}{x}
+ \pdydxnm{u}{\xi}{\eta}{2}{}{} \pdydx{\xi}{x}
= \pdydxn{u}{\eta}{2} + \pdydxnm{u}{\xi}{\eta}{2}{}{} \  .
\end{align*}
If we substitute these expressions for the derivatives in
(\ref{classif_egg_hypk1}), we get
$\displaystyle \pdydxnm{u}{\xi}{\eta}{2}{}{} = 0$.
We integrate with respect to $\xi$ to get
$\displaystyle \pdydx{u}{\eta} = H(\eta)$ for some function
$H$, and with respect to $\eta$ to get
\[
u = \underbrace{\int H(\eta) \dx{\eta}}_{\equiv F(\eta)} + G(\xi)
= F(\eta) + G(\xi)
\]
for some function $G$.  The final solution is
\[
u(x,y) = u(\xi,\eta) = F(\eta) + G(\xi) = F(x+y) + G(x) \ .
\]
\end{egg}

\begin{egg}
The partial differential equation
\begin{equation} \label{classif_egg_hypk2}
x \pdydxn{u}{x}{2} - y \pdydxnm{u}{x}{y}{2}{}{} + \pdydx{u}{x} = 0
\end{equation}
is of the form (\ref{classif_PDE1}) with
$a_{2,0}(x,y) = x$, $a_{1,1}(x,y) = -y/2$ and $a_{0,2}(x,y) = 0$.
Hence
\[
\det \begin{pmatrix}
a_{2,0}(x,y) & a_{1,1}(x,y) \\
a_{1,1}(x,y) & a_{0,2}(x,y)  
\end{pmatrix}
=
\det \begin{pmatrix}
x & -y/2 \\
-y/2 & 0  
\end{pmatrix}
= -y^2/4 \ .
\]
The partial differential equation in (\ref{classif_egg_hypk2}) is
hyperbolic for $y\neq 0$ and parabolic for $y=0$.  We will consider
only the region $y > 0$.  We have
\[
\phi_{\pm}(x,y) = \frac{a_{1,1}(x,y) \pm
\sqrt{a_{1,1}^2(x,y)-a_{2,0}(x,y)a_{0,2}(x,y)}}{a_{2,0}(x,y)}
= (-1/2 \pm 1/2)\frac{y}{x} \ .
\]
We get a family of curves $\eta(x,y) = h$ from the implicit
solution of $\displaystyle \dydx{y}{x} = -\frac{y}{x}$ and another family
of curves $\xi(x,y) = k$ from the implicit solution of
$\displaystyle \dydx{y}{x} = 0$.  From
$\displaystyle \dydx{y}{x} = -\frac{y}{x}$, we get $\eta = xy$.  We get
$\xi = y$ from $\displaystyle \dydx{y}{x} = 0$.
Since $\displaystyle \pdydx{\eta}{x} = y = \xi$,
$\displaystyle \pdydx{\eta}{y} = x = \frac{\eta}{\xi}$,
$\displaystyle \pdydx{\xi}{x} = 0$ and $\displaystyle \pdydx{\xi}{y} = 1$,
we get
\begin{align*}
\pdydx{u}{x} &= \pdydx{u}{\eta} \pdydx{\eta}{x} +
\pdydx{u}{\xi}\pdydx{\xi}{x} = \xi \pdydx{u}{\eta} \  , \\ 
\pdydxn{u}{x}{2} &= \pdfdx{ \left(\pdydx{u}{x}\right) }{x}
= \pdfdx{ \left(\xi \pdydx{u}{\eta}\right) }{\eta}\pdydx{\eta}{x}
+ \pdfdx{ \left(\xi \pdydx{u}{\eta}\right) }{\xi}\pdydx{\xi}{x}
= \xi^2 \pdydxn{u}{\eta}{2}
\intertext{and}
\pdydxnm{u}{x}{y}{2}{}{} &= \pdfdx{ \left(\pdydx{u}{x}\right) }{y}
= \pdfdx{ \left(\xi \pdydx{u}{\eta}\right) }{\eta}\pdydx{\eta}{y}
+ \pdfdx{ \left(\xi \pdydx{u}{\eta}\right) }{\xi}\pdydx{\xi}{y}
= \eta \pdydxn{u}{\eta}{2}
+ \pdydx{u}{\eta} + \xi \pdydxnm{u}{\xi}{\eta}{2}{}{} \  .
\end{align*}
If we substitute these expressions for the derivatives in
(\ref{classif_egg_hypk2}), we get
\[
0 = \frac{\eta}{\xi} \left( \xi^2 \pdydxn{u}{\eta}{2} \right)
- \xi \left( \eta \pdydxn{u}{\eta}{2}
+ \pdydx{u}{\eta} + \xi \pdydxnm{u}{\xi}{\eta}{2}{}{} \right)
+ \xi \pdydx{u}{\eta}
= - \xi^2 \pdydxnm{u}{\xi}{\eta}{2}{}{} \ .
\]
Since $\xi = y \neq 0$, we get
$\displaystyle \pdydxnm{u}{\xi}{\eta}{2}{}{} = 0$.
As we have seen before, its solution is
$u(\xi,\eta) = F(\eta) + G(\xi)$ for some functions $F$ and $G$.
Therefore, the final solution is
\[
u(x,y) = u(\xi,\eta) = F(\eta) + G(\xi) = F(xy) + G(y) \ .
\]
\end{egg}

\begin{egg}
Consider the partial differential equation          \label{classif_egg_hyperb1}
\begin{equation} \label{classif_egg_hypb}
x \pdydxn{u}{x}{2} + 2 x^2 \pdydxnm{u}{x}{y}{2}{}{} - \pdydx{u}{x} - 1
= 0
\end{equation}
It is a partial differential equation of the form (\ref{classif_PDE1}) with
$a_{2,0}(x,y) = x$, $\displaystyle a_{1,1}(x,y) = x^2$ and $a_{0,2}(x,y) = 0$.
Hence
\[
\det \begin{pmatrix}
a_{2,0}(x,y) & a_{1,1}(x,y) \\
a_{1,1}(x,y) & a_{0,2}(x,y)  
\end{pmatrix}
=
\det \begin{pmatrix}
x & x^2 \\
x^2 & 0  
\end{pmatrix}
=-x^4
\]
and (\ref{classif_egg_hypb}) is parabolic on the $y$ axis and
hyperbolic everywhere else.  The families of curves used to reduce this
hyperbolic partial differential equation to its standard form will
work for $x<0$ and $x>0$.  Since
\[
\phi_{\pm}(x,y) = \frac{a_{1,1}(x,y) \pm
\sqrt{a_{1,1}^2(x,y)-a_{2,0}(x,y)a_{0,2}(x,y)}}{a_{2,0}(x,y)}
= \frac{x^2 \pm x^2}{x} = x \pm x \ ,
\]
we get a family of curves $\eta(x,y) = h$ from the implicit
solution of $\displaystyle \dydx{y}{x} = 2x$ and another family of curves
$\xi(x,y) = k$ from the implicit solution of
$\displaystyle \dydx{y}{x} = 0$.  The solution of the first equation
is $\displaystyle y = x^2 + \eta$ and the solution of the second is
$y=\xi$.  Hence, $\displaystyle \eta(x,y)=y-x^2$ and $\xi(x,y) = y$.

To complete the example, we assumed that $x>0$ and thus
$x=\sqrt{\xi-\eta}$.  The solution below does not change if we assume
that $x<0$ and $x= -\sqrt{\xi-\eta}$.

Since $\displaystyle \pdydx{\eta}{x} = -2x = -2\sqrt{\xi-\eta}$,
$\displaystyle \pdydx{\eta}{y} = 1$,
$\displaystyle \pdydx{\xi}{x} = 0$ and $\displaystyle \pdydx{\xi}{y} = 1$,
we get
\begin{align*}
\pdydx{u}{x} &= \pdydx{u}{\eta} \pdydx{\eta}{x} +
\pdydx{u}{\xi}\pdydx{\xi}{x} = -2\sqrt{\xi-\eta}\,\pdydx{u}{\eta}\ , \\ 
\pdydxn{u}{x}{2} &= \pdfdx{ \left(\pdydx{u}{x}\right) }{x}
= \pdfdx{ \left( -2\sqrt{\xi-\eta}\,\pdydx{u}{\eta} \right) }{\eta}
\pdydx{\eta}{x}
+ \pdfdx{ \left( -2\sqrt{\xi-\eta}\,\pdydx{u}{\eta} \right) }{\xi}
\pdydx{\xi}{x}\\
&= \left( \frac{1}{\sqrt{\xi-\eta}}\,\pdydx{u}{\eta}
-2 \sqrt{\xi-\eta}\,\pdydxn{u}{\eta}{2} \right)
\left(-2\sqrt{\xi-\eta}\right)
=-2\,\pdydx{u}{\eta} +4 (\xi -\eta)\,\pdydxn{u}{\eta}{2}
\intertext{and}
\pdydxnm{u}{x}{y}{2}{}{} &= \pdfdx{ \left(\pdydx{u}{x}\right) }{y}
= \pdfdx{ \left( -2\sqrt{\xi-\eta}\,\pdydx{u}{\eta} \right) }{\eta}
\pdydx{\eta}{y}
+ \pdfdx{ \left( -2\sqrt{\xi-\eta}\,\pdydx{u}{\eta} \right) }{\xi}
\pdydx{\xi}{y}\\
&= \left( \frac{1}{\sqrt{\xi-\eta}}\,\pdydx{u}{\eta}
-2 \sqrt{\xi-\eta}\,\pdydxn{u}{\eta}{2} \right)
+ \left( -\frac{1}{\sqrt{\xi-\eta}}\,\pdydx{u}{\eta}
-2 \sqrt{\xi-\eta}\,\pdydxnm{u}{\eta}{\xi}{2}{}{} \right)\\
&= -2 \sqrt{\xi-\eta}\, \left( \pdydxn{u}{\eta}{2} +
\pdydxnm{u}{\eta}{\xi}{2}{}{} \right) \ .
\end{align*}
If we substitute these expressions for the derivatives in
(\ref{classif_egg_hypb}), we get
\begin{align*}
0 &= \sqrt{\xi-\eta} \left(-2\,\pdydx{u}{\eta}
+ 4 (\xi -\eta)\,\pdydxn{u}{\eta}{2} \right)
+2 (\xi-\eta) \left( -2 \sqrt{\xi-\eta}\, \left( \pdydxn{u}{\eta}{2} +
    \pdydxnm{u}{\eta}{\xi}{2}{}{} \right) \right) \\
&\qquad + 2\sqrt{\xi-\eta}\,\pdydx{u}{\eta} - 1
= - 4 (\xi -\eta)^{3/2} \pdydxnm{u}{\eta}{\xi}{2}{}{} - 1 \ .
\end{align*}
Thus,
\[
\pdydxnm{u}{\xi}{\eta}{2}{}{} = -\frac{1}{4} \left(\xi-\eta\right)^{-3/2} \ .
\]
An integration with respect to $\xi$ followed by and integration with
respect to $\eta$ yields
\[
u(\eta,\xi) = - \sqrt{\xi-\eta} + P(\xi) + Q(\eta)
\]
for arbitrary functions $P$ and $Q$.  Finally,
\[
u(x,y) = -x + P(y) + Q(y-x^2) \  .
\]
\end{egg}

\subsection{Parabolic Equations}

Suppose that (\ref{classif_PDE1}) is parabolic on an open set of
$\RR^2$, then we have
\begin{equation}\label{pe_cond}
\det \begin{pmatrix}
a_{2,0}(x,y) & a_{1,1}(x,y) \\
a_{1,1}(x,y) & a_{0,2}(x,y)  
\end{pmatrix}
=0
\end{equation}
for $(x,y)$ in this set.  Hence, the principal symbol is
degenerated.  There are characteristic curves and not all curves
$\Gamma$ can be used for the conditions
(\ref{classif_PDE1_cond}).

Without loss of generality, we may assume that $a_{2,0}(x,y) \neq 0$
and $a_{0,2}(x,y) \neq 0$.  If $a_{2,0}(x,y)=0$ or 
$a_{0,2}(x,y) =0$, then (\ref{pe_cond}) gives $a_{1,1}(x,y)=0$ and
we already have a standard form for (\ref{classif_PDE1}).  If
$a_{2,0}(x,y) \neq 0$ and $a_{0,2}(x,y) \neq 0$, it follows from
(\ref{pe_cond}) that $a_{1,1}(x,y) \neq 0$.

We can reduce (\ref{classif_PDE1}) to a simpler form.  Consider again
the polynomial
\begin{equation} \label{classif_PDE7}
P(x,y,\phi) = a_{2,0}(x,y) \phi^2(x,y) - 2a_{1,1}(x,y) \phi(x,y)
+ a_{0,2}(x,y) \ .
\end{equation}
There is only one non-zero root of this polynomial, it is
\[
\phi(x,y) = \frac{a_{1,1}(x,y)}{a_{2,0}(x,y)} \ .
\]

Consider a family of curve $\eta(x,y)=h$ given by the
implicit solution of $\displaystyle \dydx{y}{x} = \phi(x,y)$.
Choose any other family of curves $\xi(x,y)=k$ such that
(\ref{classif_PDE2}) is satisfied.  As we have shown before, it is
always possible to find such a family at least locally near a point
$\VEC{p}$ of the curve $\Gamma$.

Since
\[
\dydx{y}{x} = \phi(x,y) = -\pdydx{\eta}{x}\bigg/ \pdydx{\eta}{y} \ ,
\]
we get after substituting $\phi$ in
(\ref{classif_PDE7}) that
\begin{equation} \label{classif_ParEq1}
a_{2,0}(x,y) \,\left(\pdydx{\eta}{x}\right)^2 + 2a_{1,1}(x,y)\,
\pdydx{\eta}{x} \pdydx{\eta}{y}  +
a_{0,2}(x,y)\,\left(\pdydx{\eta}{y}\right)^2 = 0 \ .
\end{equation}
This equation is
$\displaystyle Q\left(x,y,\pdydx{\eta}{x}(x,y),\pdydx{\eta}{y}(x,y)\right) = 0$.
Hence, the curves of this family are characteristic curves.

It follows (\ref{classif_ParEq1}) that $A_{2,0}=0$ in
(\ref{classif_PDE8}).  Since the right hand side of
(\ref{classif_PDE8}) is null by hypothesis, we must also have that
$A_{1,1}=0$.
Therefore, with the change of coordinates $(x,y)=(x(\eta,\xi),y(\eta,\xi))$,
equation (\ref{classif_PDE1}) is reduced to
(\ref{classif_PDE3}) with $A_{2,0} = A_{1,1} = 0$ and $A_{0,2} \neq 0$.
If we divide by $A_{0,2}$, we get an equation of the form
\begin{equation} \label{classif_StandPar}
\pdydxn{u}{\xi}{2}
+ \alpha_{1,0}(\eta,\xi) \pdydx{u}{\eta}
+ \alpha_{0,1}(\eta,\xi) \pdydx{u}{\xi} + \alpha_{0,0}(\eta,\xi)u +
\beta (\eta,\xi) = 0 \  , 
\end{equation}
where $\displaystyle \alpha_{1,0} = A_{1,0}/A_{0,2}$, etc.
(\ref{classif_StandPar}) is the
{\bfseries standard form}\index{Parabolic Partial Differential
Equation!Standard Form} of a parabolic equation.

\begin{egg}
The heat equation
\[
\pdydx{u}{t} = c^2 \pdydxn{u}{x}{2}
\]
is an example of a parabolic equation on all $\RR^2$.
\end{egg}

\section{More on the Classification of Partial Differential
Equations}

There are several direction to generalize the classification of
partial differential equations.  We present some of them below.

\subsection{Classification of Second Order Semi-Linear Partial
Differential Equations in the Plane}

We consider the Cauchy problem
\begin{align}
& L(x,y,u,\diff) = a_{2,0}(x,y,u) \pdydxn{u}{x}{2}(x,y)
+ 2a_{1,1}(x,y,u) \pdydxnm{u}{x}{y}{2}{}{}(x,y)
+ a_{0,2}(x,y,u) \pdydxn{u}{y}{2}(x,y) \nonumber \\
&\qquad + a_{1,0}(x,y,u) \pdydx{u}{x}(x,y)
+ a_{0,1}(x,y,u) \pdydx{u}{y}(x,y) + a_{0,0}(x,y,u) = 0
\label{classif_PDE1semi}
\end{align}
with the conditions
\begin{equation} \label{classif_PDE1semi_cond}
u\big|_\Gamma = f \quad \text{and} \quad \pdydx{u}{\nu}\bigg|_\Gamma = g \  ,
\end{equation}
where $\nu(x,y)$ for $(x,y)\in \Gamma$ is a normal unit vector to
$\Gamma$ at $(x,y)$.  Equation (\ref{classif_PDE1semi}) is a
{\bfseries second order semi-linear Partial Differential Equation}%
\index{Second Order Semi-Linear Partial Differential Equation}.
The coefficients $a_{i,j}$ may depend on $x$, $y$ as well as on $u$.

The classification of the linear second order partial differential
equation given in the previous section is still valid for the
semi-linear second order partial differential equation.  The only
difference is that the {\bfseries principal symbol}\index{Principal Symbol}
is now defined by
\begin{align*}
Q(x,y,u,\zeta_1,\zeta_2) &= a_{2,0}(x,y,u) \zeta_1^2 +
2a_{1,1}(x,y,u)\zeta_1\zeta_2 + a_{0,2}(x,y,u) \zeta_2^2 \\
&=
\begin{pmatrix}
\zeta_1 & \zeta_2
\end{pmatrix}
\begin{pmatrix}
a_{2,0}(x,y,u) & a_{1,1}(x,y,u) \\
a_{1,1}(x,y,u) & a_{0,2}(x,y,u)  
\end{pmatrix}
\begin{pmatrix}
\zeta_1 \\ \zeta_2
\end{pmatrix} \ .
\end{align*}
Now, the principal symbol depends of $x$, $y$ and $u$.  Unfortunately,
there is not any nice method to reduce a second order semi-linear
partial differential equation to a standard form.

\subsection{System of First Order Partial Differential Equations}

Like for ordinary differential equations, we may reduce (a system of)
partial differential equations of order greater
than one to a system of partial differential equations of order one.

\begin{egg}
The partial differential equation
\[
x \pdydxn{u}{x}{2} + 2 x^2 \pdydxnm{u}{x}{y}{2}{}{} - \pdydx{u}{x} - 1
= 0
\]
of Example~\ref{classif_egg_hyperb1} can be reduced to a system of
first order partial differential equations.

Let $\displaystyle v_1 = \pdydx{u}{x}$ and
$\displaystyle v_2 = \pdydx{u}{y}$.  Then,
\[
\pdydx{v_1}{y} = \pdydxnm{u}{x}{y}{2}{}{} = \pdydx{v_2}{x}
\]
and
\[
x \pdydx{v_1}{x} + 2 x^2 \pdydx{v_1}{y} - v_1 - 1 = 0 \  .
\]
We get the system
\[
\begin{pmatrix}
0 & -1 \\
x & 0
\end{pmatrix}
\pdydx{\VEC{v}}{x}
+
\begin{pmatrix}
1 & 0 \\
2x^2 & 0
\end{pmatrix}
\pdydx{\VEC{v}}{y}
=
\begin{pmatrix}
0 \\ v_1+1
\end{pmatrix} \quad , \quad
\VEC{v} =
\begin{pmatrix}
v_1 \\ v_2
\end{pmatrix} \ .
\]
Once we have found $v_1$ and $v_2$, then $u$ is
given by integrating $v_1$ with respect of $x$ or $v_2$ with
respect to $y$.
\end{egg}

Let $\displaystyle \LL(\RR^n)$ be the space of \nn matrices.  A
{\bfseries system of linear first order Partial Differential Equations}%
\index{System of Linear First Order Partial Differential Equations} is
a system of the form
\begin{equation} \label{classif_SPDE1}
A(x,y) \pdydx{\VEC{v}}{x}(x,y) + B(x,y) \pdydx{\VEC{v}}{y}(x,y) =
\VEC{c}\left(x,y,\VEC{v}(x,y)\right) \ ,
\end{equation}
where $\displaystyle A:\RR^2\rightarrow \LL(\RR^n)$,
$\displaystyle B:\RR^2\rightarrow \LL(\RR^n)$ and
$\displaystyle \VEC{c}:\RR^{2+n}\rightarrow \RR^n$ are continuous functions.
If we combine the condition
\begin{equation} \label{classif_SPDE1_cond}
\VEC{v}\big|_\Gamma = \VEC{f}
\end{equation}
for some curve $\Gamma$ in the plane and some function
$\displaystyle \VEC{f}:\Gamma \rightarrow \RR^n$ with
(\ref{classif_SPDE1}), then we get the general form of the
{\bfseries Cauchy Problem}\index{Cauchy Problem} for systems of first
order partial differential equation. 

As for single partial differential equation, we need to define
characteristic curves.

\begin{defn}
A {\bfseries characteristic curve}\index{Characteristic Curve} for the
Cauchy problem (\ref{classif_SPDE1}) and (\ref{classif_SPDE1_cond}) is a
curve $\Gamma$ such that (\ref{classif_SPDE1}) and
(\ref{classif_SPDE1_cond}) are not enough to determine all
the first order partial derivatives of $\VEC{v}$ in a neighbourhood of
$\Gamma$.
\end{defn}

As for single partial differential equation, we need to give a precise
condition to determine if a curve is or is not a characteristic curve.
We limit the discussion to $n=2$. Suppose that
\begin{equation} \label{classif_GammaEq2}
\Gamma = \left\{ (x_0(s), y_0(s)) : s \in I \right\} \ ,
\end{equation}
where $x_0:I\rightarrow \RR$ and $y_0:I\rightarrow \RR$ are
differentiable functions, and $I$ is an open interval.  The condition
on $\Gamma$ that $\VEC{v}$ must satisfy is of the form
\begin{equation} \label{classif_SPDE1_c1}
\VEC{v}(x_0(s),y_0(s)) = \VEC{f}(s)
\end{equation}
for $s\in I$.

There are two differential (systems of) equations that must be
satisfied on $\Gamma$.
\begin{equation} \label{classif_SPDE1_c2}
\begin{split}
&A(x_0(s),y_0(s)) \pdydx{\VEC{v}}{x}(x_0(s),y_0(s)) + B(x_0(s),y_0(s))
\pdydx{\VEC{v}}{y}(x_0(s),y_0(s)) \\
&\qquad =\VEC{c}\left(x_0(s),y_0(s),\VEC{v}(x_0(s),y_0(s))\right)
\end{split}
\end{equation}
and
\begin{equation} \label{classif_SPDE1_c3}
x_0'(s)\pdydx{\VEC{v}}{x}(x_0(s),y_0(s)) +
y_0'(s)\pdydx{\VEC{v}}{y}(x_0(s),y_0(s)) = \dfdx{\VEC{f}}{s}(s) \ .
\end{equation}
The second equation being the derivative of (\ref{classif_SPDE1_c1})
with respect to $s$.

(\ref{classif_SPDE1_c2}) multiplied by $y_0'(s)$ minus
(\ref{classif_SPDE1_c3}) multiplied by $B(x_0(s),y_0(s))$ (from the
left) yields
\begin{align*}
& \left( y_0'(s)A(x_0(s),y_0(s)) - x_0'(s)B(x_0(s),y_0(s)) \right)
\pdydx{\VEC{v}}{x}(x_0(s),y_0(s)) \\
&\qquad = \left( y_0'(s) \VEC{c}(x_0(s),y_0(s),\VEC{v}(x_0(s),y_0(s)))
- B(x_0(s),y_0(s))\,\dfdx{\VEC{f}}{s}(s)\right)
\end{align*}
and (\ref{classif_SPDE1_c2}) multiplied by $x_0'(s)$ minus
(\ref{classif_SPDE1_c3}) multiplied by $A(x_0(s),y_0(s))$ (from the
left) yields
\begin{align*}
&\left( x_0'(s)B(x_0(s),y_0(s)) - y_0'(s)A(x_0(s),y_0(s)) \right)
\pdydx{\VEC{v}}{y}(x_0(s),y_0(s)) \\
&\qquad = \left( x_0'(s) \VEC{c}(x_0(s),y_0(s),\VEC{v}(x_0(s),y_0(s)))
- A(x_0(s),y_0(s))\,\dfdx{\VEC{f}}{s}(s)\right) \ .
\end{align*}
The first system can be solved for
$\displaystyle \pdydx{\VEC{v}}{x}(x_0(s),y_0(s))$ and the second for
$\displaystyle \pdydx{\VEC{v}}{y}(x_0(s),y_0(s))$ if the matrix
\begin{equation} \label{classif_SPDE1_c4}
\left( x_0'(s)B(x_0(s),y_0(s)) - y_0'(s)A(x_0(s),y_0(s)) \right)
\end{equation}
is invertible.

\begin{defn}
The {\bfseries principal symbol}\index{Principal Symbol} for
(\ref{classif_SPDE1}) is defined by
$\displaystyle
Q\left(x,y,\xi_1, \xi_2\right) = \xi_1 A(x,y) + \xi_2 B(x,y)$
for $(x,y)$ and $\xi_1,\xi_2)$ in $\RR^2$.
\end{defn}

The following proposition is a consequence of (\ref{classif_SPDE1_c4}).

\begin{prop}
A curve $\Gamma$ given by (\ref{classif_GammaEq2}) is a characteristic
curve for the Cauchy Problem (\ref{classif_SPDE1}) and
(\ref{classif_SPDE1_cond}) if and only if
$Q(x_0(s),y_0(s), -y_0'(s) , x_0'(s))$
is singular for all $s \in I$.
\end{prop}

Since $(-y_0'(s) , x_0'(s))$ is perpendicular to the curve $\Gamma$ at
$(x_0(s), y_0(s))$.  The condition above can be restated as follows.
A curve $\Gamma$ is a characteristic curve if and only if
at each point $(x,y)$ of $\Gamma$ we have that
$Q(x,y, \zeta_1 , \zeta_2)$ is singular for all orthogonal vectors
$\VEC{\zeta}$ to the curve $\Gamma$ at $(x,y)$.

If $\Gamma$ is given implicitly as $\mu(x,y) = c$ for some constant
$c$, the condition above becomes the following.
A curve $\Gamma$ is a characteristic curve if and only if
\[
Q\left(x,y, \pdydx{\mu}{x}(x,y), \pdydx{\mu}{y}(x,y) \right)
\]
is singular for all $(x,y) \in \Gamma$.

\subsection{Classification of Linear Partial Differential Equations in
$\displaystyle \RR^n$}

Though it is possible to reduce a linear partial differential equation
of order greater than one to a system of linear first order partial
differential equations, it is not necessarily
advantageous to do so.  Moreover, since the reduced linear system of
first order partial differential equations may not have the same
numbers of equations and unknown, it is not trivial to define the
principal symbol for such a system (see \cite{RenRog}).  For this
reason, we present only the
classification of general linear partial differential equations of the form
\begin{equation} \label{classif_gen_class}
L(\VEC{x}, \diff)u = \sum_{|\alpha|\leq m} a_\alpha(\VEC{x})
\diff^\alpha u = f \ ,
\end{equation}
where $\displaystyle u:\RR^n \rightarrow \RR$ is at least of class
$\displaystyle C^m$, $\displaystyle f:\RR^n \rightarrow \RR$ is continuous and
$\displaystyle a_\alpha:\RR^n \rightarrow \RR$ is of class
$\displaystyle C^\infty$ for all multi-indices involved in the sum.
We will relax the conditions on $f$
and $a_\alpha$ starting in Chapter~\ref{elliptic_PDEs}.

\begin{defn}
A subset $S$ of $\displaystyle \RR^n$ is an
{\bfseries hyper-surface}\index{Hyper-Surface} 
if $S$ is an orientable manifold of dimension $n-1$.  We will say that
$S$ is an hyper-surface of class $\displaystyle C^k$ (resp.\ an analytic
hyper-surface) if $S$ is a manifold of class $\displaystyle C^k$
(resp. an analytic manifold).
\end{defn}

If $S$ be a hyper-surface, then we may express $S$ locally as the
solution of an equation $\phi(\VEC{x}) = 0$ for some function
$\displaystyle \phi:\RR^n \to \RR$.  If $S$ is of class
$\displaystyle C^k$ (resp. analytic), we may assume that
$\phi$ is of class $\displaystyle C^k$ (resp. analytic).  We will
assume from now on that the hyper-surfaces that we consider are
sufficiently smooth.

As for second order partial differential equation in the plane, we
have the following definitions.

\begin{defn} \label{char_surf_hd_def}
Consider the Cauchy problem given by (\ref{classif_gen_class}) with
conditions on $u$ and its (directional) derivatives of order $\alpha$
with $|\alpha|<m$ on a hyper-surface $S$.  Them $S$ is a
{\bfseries characteristic surface}\index{Characteristic Surface}
for this Cauchy problem if (\ref{classif_gen_class}) and the
conditions on $u$ and its (directional) derivatives are not
enough to determine all the partial derivatives (mixed or not) of
order $m$ of $u$ in the neighbourhood of $S$.
\end{defn}

\begin{defn}
The {\bfseries principal symbol}\index{Principal Symbol} (also called the
{\bfseries characteristic form}\index{Characteristic Form})
associated to the linear partial differential operator
$\displaystyle L(\VEC{x}, \diff)$ defined in
(\ref{classif_gen_class}) is
\begin{equation} \label{classif_gen_princ_symb}
Q(\VEC{x}, \VEC{\xi}) = \sum_{|\alpha|=m} a_\alpha(\VEC{x}) \VEC{\xi}^\alpha
\end{equation}
for $\VEC{x}$ and $\VEC{\xi}$ in $\displaystyle \RR^n$.
\end{defn}

We have the following result.

\begin{prop}
An hyper-surface $S$ is a characteristic surface if and only if
$\displaystyle Q\left(\VEC{x}, \VEC{\nu}(\VEC{x}) \right) = 0$ for all
$\VEC{x} \in S$,
where $\displaystyle \VEC{\nu}(\VEC{x})$ is any orthogonal vector to $S$
at $\VEC{x}$.
\end{prop}

Before giving a classification of the linear partial differential
equations of order greater than $2$, we expand the classification of
the linear partial differential equations of order 
$2$ given in Section~\ref{classif_class_plane} to
$\displaystyle \RR^n$ with $n>2$.

Consider the linear second order partial differential equation
\begin{equation} \label{classif_class_n}
L(\VEC{x}, \diff)u = \sum_{i,j=1}^n a_{i,j}(\VEC{x})
\pdydxnm{u}{x_i}{x_j}{2}{}{} + \sum_{i=1}^n b_i(\VEC{x})
\pdydx{u}{x_i} + c(\VEC{x})u = 0 \ ,
\end{equation}
where $\displaystyle u:\RR^n \rightarrow \RR$,
$\displaystyle a_{i,j} \in C^2(\RR^n)$,
$\displaystyle b_i \in C^1(\RR^n)$ and $\displaystyle c \in C(\RR^n)$.
Its principal symbol takes the form
\begin{equation} \label{classif_princ_symb_n}
Q(\VEC{x}, \VEC{\xi}) = \sum_{i,j=1}^n a_{i,j}(\VEC{x})\xi_i \xi_j
= \xi^\top\ A(\VEC{x})\ \xi \ ,
\end{equation}
where $A(\VEC{x})$ is the \nn matrix with entries $a_{i,j}(\VEC{x})$.

\begin{defn}\label{category2}
Consider the linear partial differential operator $L(\VEC{x}, \diff)$ in
(\ref{classif_class_n}) with its principal symbol $Q$ given in 
(\ref{classif_princ_symb_n}).
\begin{enumerate}
\item $L(\VEC{x}, \diff)$ is
  {\bfseries elliptic}\index{Linear Partial Differential Operator!Elliptic}
  at $\VEC{x}$ if all eigenvalues of $A(\VEC{x})$ have the same sign. 
\item $L(\VEC{x}, \diff)$ is
  {\bfseries hyperbolic}\index{Linear Partial Differential Operator!Hyperbolic}
  at $\VEC{x}$ if all eigenvalues of $A(\VEC{x})$ are non-null and all
  but one have the same sign.
\item $L(\VEC{x}, \diff)$ is 
{\bfseries ultrahyperbolic}
\index{Linear Partial Differential Operator!Ultrahyperbolic} at
$\VEC{x}$ if all eigenvalues of $A(\VEC{x})$ are non-null and there
are more than one of each sign.
\item $L(\VEC{x}, \diff)$ is {\bfseries parabolic}
\index{Linear Partial Differential Operator!Parabolic} at $\VEC{x}$ if
$A(\VEC{x})$ is singular; namely, $A(\VEC{x})$ has at least one zero
eigenvalue.
\end{enumerate}
\end{defn}

In $\displaystyle \RR^2$, this definition is equivalent to
Definition~\ref{classif_EHP}.  Recall that all the eigenvalues of a
symmetric matrix are real.

We will not attempt to provide a full classification of partial
differential equations of order greater than $2$.   In fact, we will
consider only elliptic partial differential equations and 
prove one nice result about them.

\begin{defn}\label{categoryM}
Let $L(\VEC{x}, \diff)$ be the linear partial differential operator
defined in (\ref{classif_gen_class}) and $Q$ be the associated principal
symbol defined in (\ref{classif_gen_princ_symb}).
$L(\VEC{x}, \diff)$ is
{\bfseries elliptic}\index{Linear Partial Differential Operator!Elliptic}
at $\VEC{x}$ if $\displaystyle Q(\VEC{x}, \VEC{\xi}) \neq 0$ for all
$\displaystyle \VEC{\xi} \in \RR^n\setminus \{\VEC{0}\}$.
\end{defn}

In the case where $m=2$ in (\ref{classif_gen_class}), the definition of
elliptic partial differential equation given in
Definition~\ref{category2} is equivalent to
the definition given in Definition~\ref{categoryM}.

The following proposition shows that the order of an elliptic linear
partial differential operator must be even.

\begin{prop} \label{classif_ell_even}
Let $L(\VEC{x}, \diff)$ be an elliptic linear partial differential
operator of order $m$ at $\displaystyle \VEC{x} \in \RR^n$, $n>1$.
\begin{enumerate}
\item If $\displaystyle a_\alpha:\RR^n \rightarrow \RR$ for all
multi-indices $\alpha$ such that $|\alpha|=m$, then $m$ is even.
Moreover, the associated principal symbol $Q(\VEC{x}, \VEC{\xi})$ is
either positive for all $\VEC{\xi} \neq \VEC{0}$ or negative for all
$\VEC{\xi} \neq \VEC{0}$.
\item If $\displaystyle a_\alpha:\RR^n \rightarrow \CC$ for some multi-indices
$\alpha$ such that $|\alpha|=m$ and $n>2$, then $m$ is even.
\end{enumerate}
\end{prop}

\begin{proof}
\stage{i} Suppose that $\displaystyle a_\alpha:\RR^n\rightarrow \RR$ for all
multi-indices $\alpha$ such that $|\alpha|=m$.

Moreover, suppose that their exist $\VEC{\xi}_1$ and $\VEC{\xi}_2$ in
$\displaystyle \RR^n$ such that
$\displaystyle Q\left(\VEC{x}, \VEC{\xi}_1\right) < 0 <
Q\left(\VEC{x}, \VEC{\xi}_2\right)$.  Since $Q$ is continuous, we may
assume that the line
$\displaystyle \left\{ (1-s)\VEC{\xi}_1 + s\VEC{\xi}_2 : s \in [0,1] \right\}$
does not contain the origin.  If the origin is on this line, we
replace $\VEC{\xi}_1$ by another point $\VEC{\xi}_1'$ near
$\VEC{\xi}_1$ where we still have
$Q\left(\VEC{x}, \VEC{\xi}_1'\right) < 0$.  Consider the function
\begin{align*}
g:[0,1] &\rightarrow \RR \\
s &\mapsto Q\left(\VEC{x}, (1-s)\VEC{\xi}_1 + s\VEC{\xi}_2\right)
\end{align*}
Since $Q$ is continuous, $g$ is a continuous function such that
$g(0)<0<g(1)$.  By the Intermediate Value Theorem, there exists
$s_0 \in ]0,1[$ such that $g(s_0)=0$.  Thus
$\displaystyle
Q\left(\VEC{x}, (1-s_0)\VEC{\xi}_1 + s_0\VEC{\xi}_2\right) = 0$
with $\displaystyle (1-s_0)\VEC{\xi}_1 + s_0\VEC{\xi}_2 \neq \VEC{0}$.
This is a contradiction that $L(\VEC{x}, \diff)$ is elliptic.

Therefore, for
$\displaystyle \VEC{\xi} \in \RR^n \setminus \{\VEC{0}\}$, we have
that $\displaystyle Q(\VEC{x}, -\VEC{\xi}) = (-1)^mQ(\VEC{x}, \VEC{\xi})$
and $\displaystyle Q(\VEC{x}, \VEC{\xi})$ are of the same sign.  This
is possible only if $m$ is even.

\stage{ii} Suppose that $\displaystyle a_\alpha:\RR^n\rightarrow \CC$ for some
multi-indices $\alpha$ such that $|\alpha|=m$ and $n>2$.

For each $\displaystyle \VEC{y} \in \RR^{n-1}$, let
\[
p_{\VEC{y}}(z) = Q(\VEC{x}, \,(\VEC{y},z)\,) \quad , \quad z \in \CC \ .
\]
Let $n_+(\VEC{y})$ be the number of roots (counted with multiplicity)
of $p_{\VEC{y}}$ with a positive imaginary part, and let $n_-(\VEC{y})$ be
the number of roots (counted with multiplicity) of $p_{\VEC{y}}$ with
a negative imaginary part.

Since $p_{\VEC{y}}(z) \neq 0$ for all $z\in \RR$ and
$\displaystyle \VEC{y} \in \RR^{n-1}\setminus \{\VEC{0}\}$ because
$L(\VEC{x},\diff)$ is elliptic, we have $n_+(\VEC{y})+n_-(\VEC{y})=m$
for $\displaystyle \VEC{y} \in \RR^{n-1}\setminus \{\VEC{0}\}$.

Moreover, it follows from Rounch\'e's theorem (see \cite{Co}) that
$n_+$ and $n_-$ are locally constant.  So, they are constant on the
largest connected component of $\displaystyle \RR^{n-1}$ that does not
contain the origin (the only value of $\VEC{y}$ for which
$p_{\VEC{y}}$ may have real roots).  Since
$\displaystyle \RR^{n-1}\setminus\{\VEC{0}\}$ is connected for
$n>2$,  $n_+$ and $n_-$ are constant on
$\displaystyle \RR^{n-1}\setminus\{\VEC{0}\}$.

Finally, $n_+(-\VEC{y}) = n_-(\VEC{y})$ for
$\displaystyle \RR^{n-1}\setminus\{\VEC{0}\}$ because
$\displaystyle p_{-\VEC{y}}(-z) = (-1)^m p_{\VEC{y}}(z)$.  Since $n_+$
and $n_-$ are constant on $\displaystyle \RR^{n-1}\setminus\{\VEC{0}\}$, we get
$n_-(\VEC{y}) = n_+(-\VEC{y}) = n_+(\VEC{y})$ for
$\displaystyle \VEC{y} \in \RR^{n-1}\setminus\{\VEC{0}\}$.

Hence, $m = n_+(\VEC{y})+n_-(\VEC{y}) = 2n_+(\VEC{y})$ and it follows
that $m$ is even.
\end{proof}

To be able to use variational methods to solve elliptic partial
differential equations in Chapter~\ref{elliptic_PDEs}, elliptic
operator in general is not strong enough.  We need to request a little
more from our elliptic operators.
Though we may not use this full generality later, we assume that the
coefficients $a_\alpha$ in the linear partial differential operator
$L(\VEC{x}, \diff)$ defined in (\ref{classif_gen_class}) are complex
valued functions.

Since we are interested in elliptic linear partial differential
equations, it follows from Proposition~\ref{classif_ell_even} that we
may assume that the order $m$ of the linear partial differential operator
$L(\VEC{x}, \diff)$ defined in (\ref{classif_gen_class}) is even.

\begin{defn} \label{strong_ell_form}
Let $L(\VEC{x}, \diff)$ be the linear partial differential operator of
even order defined in (\ref{classif_gen_class}) and $Q$ be the
associated principal symbol defined in (\ref{classif_gen_princ_symb}).
The operator $L(\VEC{x}, \diff)$ is
{\bfseries strongly or uniformly elliptic}%
\index{Linear Partial Differential Operator!Uniformly Elliptic}%
\index{Linear Partial Differential Operator!Strongly Elliptic} 
on an open subset $\displaystyle \Omega \subset \RR^n$ if there exists
a constant $C>0$ such that
\begin{equation} \label{strong_ell_formEq1}
(-1)^{m/2} \RE Q(\VEC{x}, \VEC{\xi}) \geq C \|\VEC{\xi}\|^m
\end{equation}
for all $\displaystyle \VEC{\xi} \in \RR^n$ and $\VEC{x}\in \Omega$.
\end{defn}

Note that original partial differential equation may have to be
multiplied by $-1$ to get (\ref{strong_ell_formEq1}).  We assume that
this is always done if necessary.

\begin{rmk}
If the linear partial differential operator $L(\VEC{x}, \diff)$
is elliptic at $\VEC{x}$, then         \label{ell_implies_strong_ell}
\[
0 < C_{\VEC{x}}
= \inf_{\|\VEC{\xi}\|_2 = 1} \left| Q(\VEC{x}, \VEC{\xi}) \right| < \infty
\]
because $\VEC{\xi} \to |Q(\VEC{x},\VEC{\xi})|$ is continuous on the
compact set $\displaystyle B= \{\VEC{\xi}\in \RR^n : \|\VEC{\xi}\| = 1\}$
and so reaches its minimum at a point of $B$.  If furthermore we
assume that the linear partial differential equation is paused on a
bounded open set $\Omega$ with coefficients continuous on
$\overline{\Omega}$,  then we may find $C>0$ such that $C \leq
C_{\VEC{x}}$ for all $\VEC{x} \in \Omega$ because $\VEC{x} \to
C_{\VEC{x}}$ is continuous on the compact set $\overline{\Omega}$.  We
therefore have that
\[
  \inf_{\|\VEC{\xi}\|_2 = 1} \left| Q(\VEC{x}, \VEC{\xi}) \right| \geq  C
\]
for all $\VEC{x} \in \Omega$.  It follows that
\[
  \left| Q(\VEC{x}, \VEC{\xi}) \right| \geq  C \|\VEC{\xi}\|^m
\]
for all $\displaystyle \VEC{\xi} \in \RR^n$ and $\VEC{x} \in \Omega$.
If we also assume that all the coefficients $a_\alpha$ in
$L(\VEC{x},\diff)$ are functions from $\displaystyle \RR^n$ to $\RR$, then
$\RE\, Q(\VEC{x}, \VEC{\xi}) = Q(\VEC{x}, \VEC{\xi})$ and thus
$L(\VEC{x}, \diff)$ is strongly elliptic on $\Omega$
because we can always multiply the $a_{\alpha}$ by $1$ or $-1$ to get
the right sign for the definition of strongly elliptic.
\end{rmk}

\begin{rmk}
The definition of a strongly elliptic linear differential operator
$L(\VEC{x}, \diff)$ as defined in (\ref{classif_gen_class}) is
sometime stated as follows.  A linear differential operator
$L(\VEC{x}, \diff)$ is strongly elliptic on a open
set $\displaystyle \Omega \subset \RR^n$ if there exist a complex
valued function
$\displaystyle \gamma \in C^\infty(\overline{\Omega})$ and a constant $C>0$ such
that $|\gamma(\VEC{x})|=1$ for all $\VEC{x} \in \overline{\Omega}$ and
\begin{equation} \label{strong_ell_formEq2}
\RE\left(\gamma(\VEC{x})\, Q(\VEC{x},\VEC{\xi})\right) \geq C
\|\VEC{\xi}\|^m
\end{equation}
for all $\displaystyle \VEC{\xi} \in \RR^n$ and
$\VEC{x} \in \overline{\Omega}$.
This is not a more general definition than Definition~\ref{strong_ell_form}
because we can always multiply the linear partial differential
equation by $\displaystyle (-1)^{m/2}/\gamma$ to obtain a linear partial
differential equation equivalent to the original one.  The
coefficients $a_\alpha$ of the original linear partial differential
equation are replaced by the coefficients
$\displaystyle (-1)^{m/2} a_\alpha/\gamma$ in the new linear partial
differential equation and thus
(\ref{strong_ell_formEq2}) is reduced to (\ref{strong_ell_formEq1}).
\end{rmk}

\section{Exercises}

Suggested exercises:

\begin{itemize}
\item In \cite{McO}: numbers 1 to 5 in Section 2.2.
\item In \cite{PinRub}: all the numbers in Sections 3.6. 
\item In \cite{Str}: all the numbers in Section 1.6.
\end{itemize}

%%% Local Variables: 
%%% mode: latex
%%% TeX-master: "notes"
%%% End: 
