\chapter{Cauchy Problem} \label{ChapterCauchyP}

\section{Cauchy-Kovalevski's Theorem}

Let $S$ be an hyper-surface in $\displaystyle \RR^n$ and
$g_j:S\rightarrow \RR$ for $0\leq j \leq m-1$ be continuous functions.
The {\bfseries Cauchy Problem}\index{Cauchy Problem} is to find a function
$\displaystyle u: \RR^n \rightarrow \RR$ such that
\begin{equation}\label{cp_GPDE}
G\left(\VEC{x},
\left\{ \diff^\alpha u\left(\VEC{x}\right) : \alpha \in \NN^n \ , \
  |\alpha|\leq m \right\} \right) = 0
\end{equation}
for $\displaystyle \VEC{x} \in \RR^n$,
\begin{equation}\label{cp_GPDE_cond}
u\big|_S = g_0 \ , \quad \pdydx{u}{\nu}\bigg|_S = g_1 \ ,
\ \ldots\ ,\quad \pdydxn{u}{\nu}{m-2}\bigg|_S = g_{m-2} \ 
\text{and} \ \pdydxn{u}{\nu}{m-1}\bigg|_S = g_{m-1} \ ,
\end{equation}
where $\displaystyle \nu:S \rightarrow \RR^n$ is a differentiable
function such that $\nu\left(\VEC{x}\right)$ is a unit normal vector
to the hyper-surface $S$ at $\VEC{x} \in S$.  Recall that
\[
\pdydx{u}{\nu}\left(\VEC{x}\right) =
\lim_{s\rightarrow 0} \frac{u\left(\VEC{x}+s\nu\left(\VEC{x}\right)\right) -
u\left(\VEC{x}\right) }{s} = \graD u(\VEC{x}) \cdot \nu\left(\VEC{x}\right)
\]
is the directional derivative of $u$ at $\VEC{x}$ in the direction
$\nu\left(\VEC{x}\right)$.

As we have seen for the one-dimensional partial differential
equations, not all hyper-surfaces $S$ yields a Cauchy problem that has a
unique solution.  We will give later conditions on $S$ that ensure that a
Cauchy Problem is well posed.

Using a local changed of coordinates near a point $\VEC{p}$, we may
assume that the hyper-surface $S$ is defined by
\[
S = \left\{ \VEC{x} \in \RR^n : x_n = 0 \right\} \ .
\]
Hence the Cauchy Problem becomes
\begin{equation} \label{cauchy_GPDE}
F\left(\VEC{x},
\left\{ \diff^\alpha u\left(\VEC{x}\right) : \alpha \in \NN^n \ , \ 
|\alpha| \leq m \right\} \right) = 0
\end{equation}
for $\displaystyle \VEC{x} \in \RR^n$,
\[
u\big|_S = f_0 \ , \quad \pdydx{u}{x_n}\bigg|_S = f_1 \ , \ 
\ldots \ , \quad \pdydxn{u}{x_n}{m-2}\bigg|_S = f_{m-2} \ \text{and}
\ \pdydxn{u}{x_n}{m-1}\bigg|_S = f_{m-1} \ .
\]

Let us assume that $F$, $f_0$, $f_1$, \ldots , $f_{m-1}$ are analytic
functions.  Moreover, let us assume that $u$ is also analytic.
From $u(x_1,x_2, \ldots, x_{n-1},0) = f_0(x_1,x_2, \ldots, x_{n-1})$,
we can compute
\[
\diff^\alpha u(x_1, x_2, \ldots, x_{n-1}, 0) =
\diff^\alpha f_0(x_1, x_2, \ldots, x_{n-1})
\]
for all $\displaystyle \alpha \in \NN^n$ with $\alpha_n = 0$.
From $\displaystyle \pdydx{u}{x_n}(x_1,x_2, \ldots, x_{n-1},0)
= f_1(x_1,x_2, \ldots, x_{n-1})$,
we can compute
\[
\diff^\alpha u(x_1, x_2, \ldots, x_{n-1}, 0) =
\diff^\alpha f_1(x_1, x_2, \ldots, x_{n-1})
\]
for all $\displaystyle \alpha \in \NN^n$ with $\alpha_n = 1$.
Proceeding like this for the derivative of $u$ up to order $m-1$, we
can compute $\displaystyle \diff^\alpha u(x_1, x_2, \ldots, x_{n-1}, 0)$ for
all $\displaystyle \alpha \in \NN^n$ with $\alpha_n < m$.

The problem is to compute
$\displaystyle \pdydxn{u}{x_n}{j}(x_1,x_2,\ldots,x_{n-1},0)$ for $j\geq m$.
For this, we assume that (\ref{cauchy_GPDE}) can be solved for
$\displaystyle \pdydxn{u}{x_n}{m}$ for $\VEC{x}$ near the hyper-surface $S$.
Namely, we assume that we can write
\begin{equation} \label{cauchy_CK_cond}
\displaystyle \pdydxn{u}{x_n}{m} = H\left(\VEC{x},
\left\{ \diff^\alpha u\left(\VEC{x}\right) : \alpha \in \NN^n \ , \ 
|\alpha| \leq m \ , \, \alpha \neq m \VEC{e}_n \right\} \right)
\end{equation}
for $\displaystyle \VEC{x} \in \RR^n$ near $S$, where, as usual,
$\displaystyle \left\{ \VEC{e}_1, \VEC{e}_2, \ldots, \VEC{e}_n\right\}$ 
is the canonical basis of $\displaystyle \RR^n$.  Not all
hyper-surfaces $S$ will meet this requirement.   This is the basis for
the classification of partial differential equations given in
Chapter~\ref{ChapClassifPDE}.

Using (\ref{cauchy_CK_cond}), we may compute derivatives of $u$ of
order greater or equal to $m$ with respect to $x_n$.  Thus, all the
mixed derivatives of $u$ at a point
$\VEC{a} = (a_1,a_2, \ldots, a_{n-1}, 0)$
can be computed.  Since we assume that $u$ is analytic, we can write
for $\VEC{x}$ near $\VEC{p}$ that
\[
u\left(\VEC{x}\right) =
\sum_{\alpha \in \NN^n} \frac{1}{\alpha!}
\diff^\alpha u\left(\VEC{a}\right) \left( \VEC{x}- \VEC{p} \right)^\alpha \ ,
\]
where
\[
\alpha! = \alpha_1!\,\alpha_2!\ldots \alpha_n!
\]
and
\[
\left( \VEC{x}- \VEC{p} \right)^\alpha =
\left(x_1-p_1\right)^{\alpha_1}\,\left(x_2-p_2\right)^{\alpha_2}\ldots
\left(x_n-p_n\right)^{\alpha_n} \ .
\]

This is the essence of the following theorem that we will not prove.

\begin{theorem}[Cauchy-Kovalevski] \label{cauchyKovalTh}
Let $\displaystyle S \subset \RR^n$ be a analytic hyper-surface.  Assume that
$G$ is an analytic function in a neighbourhood of $S$, and
$g_0:S\rightarrow \RR$, $g_1:S\rightarrow \RR$, \ldots, 
$g_{m-1}:S\rightarrow \RR$ are analytic functions on $S$.
If $S$ is a non-characteristic hyper-surface
(See Definition~\ref{char_surf_hd_def}),
then there exists a unique analytic solution $u$ of (\ref{cp_GPDE}) and
(\ref{cp_GPDE_cond}) in a neighbourhood of $S$.
\end{theorem}

A proof of this theorem can be found in \cite{J,Tr} for instance.

The uniqueness in Cauchy-Kovalevski theorem is among the analytic
functions.  It does not say anything about less smooth solutions.
The next theorem roughly states that, for linear partial differential
equations, the uniqueness is true among all functions.  This theorem
will also not be proved.  A proof can be found in \cite{Smo} for
instance.

\begin{theorem}[Holmgren] \label{HolmgrenTheo}
Consider the m$^{th}$ order linear partial differential equation
\[
L(\VEC{x}, \diff)u = \sum_{|\alpha|\leq m} a_\alpha(\VEC{x}) \diff^\alpha u = 0
\]
on $\displaystyle \RR^n$, where the functions
$\displaystyle a_\alpha:\RR^n \rightarrow \RR$
are real analytic. Suppose that $S$ is an analytic non-characteristic
hyper-surface.  Then, the unique solution of class $\displaystyle C^m(\RR^n)$
which satisfy the Cauchy conditions $\displaystyle \diff^\alpha u = 0$
on $S$ for all multi-indices $\alpha$ with $|\alpha|<m$ is $u=0$ on $R$.
\end{theorem}

\cite{J} proves a little bit more general version of Holmgren's theorem.
The proof of Holmgren's theorem is not easy.
Before stating this version, we need a little definition.

\begin{defn}
Let $\displaystyle \Omega \subset \RR^n$ be an open set.  Suppose that
\begin{enumerate}
\item there exists a real analytic bijection
$F:B \times ]a,b[ \rightarrow \Omega$, where $B$ is an open ball in
$\displaystyle \RR^{n-1}$ and $a<b$,
\item the Jacobian of $F$ is different from $0$ everywhere on
$B\times ]a,b[$, and
\item $\displaystyle \Omega = \bigcup_{\lambda \in ]a,b[} S_\lambda$, 
where $\displaystyle S_\lambda = \left\{ \VEC{x} : \VEC{x} = F(\VEC{y},\lambda)
\ \text{for} \ \VEC{y} \in B \right\}$.
\end{enumerate}
Then the family of hyper-surfaces
$\displaystyle \left\{ S_\lambda \right\}_{\lambda \in ]a,b[}$ is an
{\bfseries analytic field}\index{Analytic Field} for $\Omega$.
\end{defn}

\begin{theorem}[Holmgren]
Consider the m$^{th}$ order linear partial differential equation
\[
L(\VEC{x}, \diff)u = \sum_{|\alpha|\leq m} a_\alpha(\VEC{x}) \diff^\alpha u = 0
\]
on an open set $\Omega$ of $\displaystyle \RR^n$, where the functions
$a_\alpha:\Omega\rightarrow \RR$ are real analytic. Suppose that
$\displaystyle \left\{ S_\lambda \right\}_{\lambda \in ]a,b[}$ is an
analytic field for $\Omega$.\\
Let
$\displaystyle
R = \left\{ \VEC{x} : \VEC{x} \in \Omega \ \text{with} \ x_n\geq 0 \right\}$,
$\displaystyle
Z = \left\{ \VEC{x} : \VEC{x} \in \Omega \ \text{with} \ x_n = 0 \right\}$
and
$\displaystyle \Omega_\mu = \bigcup_{a<\lambda \leq \mu} S_\lambda$.

If $Z \neq \emptyset$, $Z$ and the $S_\lambda$ are not characteristic
hyper-surfaces, and
$\Omega_\mu \cap R$ is a closed subset of $\Omega$ (with
the induced topology on $\Omega$) for all $\mu \in ]a,b[$, then the
unique solution of class $\displaystyle C^m(\overline{R})$ which
satisfy the Cauchy conditions $\displaystyle \diff^\alpha u = 0$ on
$Z$ for all multi-indices $\alpha$ with $|\alpha|<m$ is $u=0$ on $R$.
\end{theorem}

If $\displaystyle R\subset \RR^n$, recall that
$\displaystyle f \in C^m(\overline{R})$ means that
$\displaystyle f\in C(\overline{R}) \cap C^m(R)$ and
$\displaystyle \diff^\alpha f$ can be extended
continuously to $\overline{R}$ for all multi-indices $\alpha$ with
$|\alpha|\leq m$.

The proof of Theorem~\ref{HolmgrenTheo} consists in fact in
constructing a specific analytic field.

\section{Lewy Example}

The next theorem provide an example of a partial differential equation
that cannot have a classical solution  if analyticity is dropped from
the assumption on the partial differential equation.  This theorem is
due to Hans Lewy.  For this theorem, we consider complex valued
functions.

\begin{theorem}[Lewy] \label{Lewy}
Let
\[
L(x,y,\diff) = \pdydx{}{x} + i \pdydx{}{y} - 2i(x+iy)\pdydx{}{t}
\]
and $f:\RR\to \RR$ be a continuous function.  If
$\displaystyle u:\RR^3\to \CC$ is a
solution of $L(x,y,\diff)u = f$ of class $\displaystyle C^1$ in the
neighbourhood of the origin, then $f$ is analytic at $t=0$.
\end{theorem}

\begin{proof}
Let
\[
v(r,\theta,t) = - e^{i\theta} \sqrt{r}
\, u(\sqrt{r}\cos(\theta) , \sqrt{r}\sin(\theta), t)
\]
for
$(r,\theta,t) \in \Omega_1 = ]0,k[\times \RR \times ]-k,k[$,
where $k$ is small enough to have $L(x,y,\diff)u = f$ in $\Omega_1$.
Then $\displaystyle v \in C^1(\Omega_1)$ and
\[
L(x,y,\diff)u = -2 \pdydx{v}{r} - \frac{i}{r} \pdydx{v}{\theta}
+ 2i \pdydx{v}{t}
= f
\]
on $\Omega_1$ if we use the substitution 
$(x,y) = (\sqrt{r} \cos(\theta), \sqrt{r} \sin(\theta))$.  Note that
$L(x,y,\diff)$ in the $r,\theta$ coordinates given by
$(x,y) = (\sqrt{r} \cos(\theta), \sqrt{r} \sin(\theta))$ is
\[
  L(r,\theta,\diff) = 2 \sqrt{r}\, e^{i\theta} \pdydx{}{r}
  + \frac{i}{\sqrt{r}} \, e^{i\theta} \pdydx{}{\theta} 
  - 2 i \sqrt{r} \, e^{i\theta} \pdydx{}{t} \ .
\]

Let
\[
V(t,r) = \int_0^{2\pi} v(r,\theta, t) \dx{\theta}
\]
for $(t,r) \in \Omega_2 = ]-k,k[\times ]0,k[$.  Then
$\displaystyle V\in C^1(\Omega_2)$ and
\begin{align*}
\pdydx{V}{t}(t,r) + i \pdydx{V}{r}(t,r) 
&= \int_0^{2\pi} \left( i\pdydx{v}{r}(r,\theta,t)
- \frac{1}{2r} \pdydx{v}{\theta}(r,\theta, t)
+ \pdydx{v}{t}(r,\theta,t)  \right) \dx{\theta} \\
&= \frac{1}{2i} \int_0^{2\pi} f(t) \dx{\theta} = -\pi i f(t) \ ,
\end{align*}
where we have use the fact that
$\displaystyle \int_0^{2\pi} \pdydx{v}{\theta}(r,\theta,t) \dx{\theta}
= v(r,\theta,t) \Big|_0^{2\pi} = 0 $ because $v$ is $2\pi$-periodic.

Let
$\displaystyle F(t) = \int_0^t f(s)\dx{s}$ and $W(t,r) = V(t,r) + \pi\, i F(t)$
for $(t,r) \in \Omega_2$.  The function $W$ satisfies the
Cauchy-Riemann Equation
\[
\pdydx{W}{t} + i \pdydx{W}{r} = 0
\]
on $\Omega_2$.  Thus $W$ is an holomorphic function of $w = t+ r i$
for $(t,r) \in \Omega_2$.

The continuity of $u$ near the origin implies the continuity of $v$ in
$\Omega_1$.  Moreover, $v(0,\theta,t)=0$ for
$(\theta,t) \in \RR \times ]-k,k[$.  Hence, $V$ is continuous on
$\Omega_2$ and $V(t,0) = 0$ for $t \in ]-k,k[$.  Thus,
$W$ is continuous on $\Omega_2$ and $W(t,0) = \pi\, i F(t)$ for
$t\in]-k,k[$; namely, $\RE W(t,0) = 0$ for $t\in]-k,k[$.

By Schwarz reflection principle applied to $iW$, the function
\[
\tilde{W}(t,r) = \begin{cases}
i W(t,r) & \quad \text{if} \ r \geq 0 \\[0.5em]
- i \overline{W(t,-r)} & \quad \text{if} \ r < 0
\end{cases}
\]
defines an holomorphic extension on $]-k,k[\times ]-k,k[$
of $i W$.  Thus $\tilde{W}(t,0) = - \pi F(t)$ is real analytic at $t = 0$.
It follows that $\displaystyle f = \dydx{F}{t}$ is real
analytic at $t=0$.
\end{proof}

\begin{rmk}
Given $\displaystyle (x_0,y_0,t_0) \in \RR^3$, we define a translation in $\RR^3$ by
\begin{align*}
T:\RR^3 &\to \RR^3 \\
 (x,y,t) &\mapsto (x+x_0,y+y_0, t+t_0 +2y_0x - 2x_0 y)
\end{align*}
If $L(x,y,\diff)$ is the Lewy operator defined in Theorem~\ref{Lewy}, we have
$\displaystyle L(x,y,\diff)(u \circ T) = \left(L(x,y,\diff)u\right) \circ T$
for any function $\displaystyle u \in C^1(\RR^3)$.
Thus, solving $L(x,y,\diff)u = f(t)$ near $(x_0,y_0,t_0)$ is
equivalent to solving $L(s,y,\diff)(u\circ T) = f(t+t_0+2y_0x - 2x_0y)$ near the
origin.  It follows from Lewy's Theorem that there is no solution of
class $C^1$ of $L(x,y,\diff)u = f(t)$ near $(x_0,y_0,t_0)$ unless $f$ is
analytic at $t=-t_0-2y_0x + 2x_0y$.

H. Lewy has also shown that there are functions
$\displaystyle f \in C^{\infty}(\RR^3)$
such that $L(x,y,\diff)u = f$ has no solution $u$ of class
$\displaystyle C^{1,\mu}(\RR^n)$ for $\mu > 0$ in a neighborhood of
the origin and by extension anywhere in $\displaystyle \RR^3$.  The spaces
$\displaystyle C^{k,\mu}(\RR^n)$ of Hölder continuous functions are
defined later in Definition~\ref{defnHolderContFunct}.  A proof of
this result can be found in \cite{Smo}.
\end{rmk}

A generalization of Theorem~\ref{Lewy} is given later in
Theorem~\ref{Hormander}.

\section{Exercises}

Suggested exercises:

\begin{itemize}
\item In \cite{McO}: 1, 2, 4 and 7 in Section 2.1.
\end{itemize}


%%% Local Variables: 
%%% mode: latex
%%% TeX-master: "notes"
%%% End: 
