\chapter{Wave Equation in One-Dimension}\label{ChapWaveEqu1D}

This chapter is devoted to the study of the
{\bfseries wave equation}\index{Wave Equation}
\[
\pdydxn{u}{t}{2} - c^2 \pdydxn{u}{x}{2} = 0 \ .
\]
The constant $c$ is called the
{\bfseries (propagation) speed}\index{Wave Equation!Propagation Speed}.  We
will justify shortly why $c$ is given this name.
We will also assume that the solution of the wave equation satisfies
some initial and boundary conditions according to the domain of the
wave equation.

\section{D'Alembert Method} \label{wave_sec_oneD}

\subsection{Over the Real Line}

Consider the wave equation
\begin{equation} \label{wave_oneD}
\pdydxn{u}{t}{2} - c^2 \pdydxn{u}{x}{2} = 0 \quad , \quad t >0
\text{ and } x \in \RR \ ,
\end{equation}
with the initial conditions
\begin{equation} \label{wave_oneD_init}
u(x,0)= f(x) \quad \text{and} \quad \pdydx{u}{t}(x,0) = g(x) \quad ,
\quad x \in \RR \ ,
\end{equation}
for two functions $f:\RR\rightarrow \RR$ and $g:\RR \rightarrow \RR$.
The smoothness of the functions $f$ and $g$ will be specified later.

As we have seen in Section~\ref{classif_class_planeHyper}, the wave equation
is an hyperbolic partial differential equation.  We may used a change
of variables to reduce the wave equation to the standard form for an
hyperbolic equation.  We now repeat this reduction process.  The wave
equation is of the form
\[
L(x,t,\diff)u = a_{2,0} \pdydxn{u}{t}{2} + 2 a_{1,1} \pdydxnm{u}{x}{t}{2}{}{} +
a_{0,2} \pdydxn{u}{x}{2} = 0
\]
with $a_{2,0} = 1$, $a_{1,1} = 0$ and $\displaystyle a_{0,2}=-c^2$.
The principal symbol is
\[
p(x,t, \xi_1, \xi_2) =
\begin{pmatrix}
\xi_1 & \xi_2
\end{pmatrix}
\begin{pmatrix}
a_{2,0} & a_{1,1} \\
a_{1,1} & a_{0,2}
\end{pmatrix}
\begin{pmatrix}
\xi_1 \\ \xi_2
\end{pmatrix} \ .
\]
The wave equation is hyperbolic on $\displaystyle \RR^2$ because
$\displaystyle \det
\begin{pmatrix}
a_{2,0} & a_{1,1} \\
a_{1,1} & a_{0,2}
\end{pmatrix} = -c^2 < 0$.
The solutions of the differential equations
\[
\dydx{x}{t} = \frac{a_{1,1} + \sqrt{a_{1,1}^2 - a_{2,0} a_{0,2}}}{a_{2,0}} = c
\quad \text{and} \quad
\dydx{x}{t} = \frac{a_{1,1} - \sqrt{a_{1,1}^2 - a_{2,0} a_{0,2}}}{a_{2,0}} =-c
\]
are respectively $x=ct+\eta$ and $x=-ct+\xi$.  The level curves of
$\eta(x,t) = x-ct$ and $\xi(x,t) = x +ct$ are characteristic curves.  We use
the change of coordinates $\eta = x-ct$ and $\xi = x +ct$ to reduce the
wave equation.

Since
\[
\pdydx{u}{x} = \pdydx{u}{\eta}\pdydx{\eta}{x} + \pdydx{u}{\xi}\pdydx{\xi}{x}
= \pdydx{u}{\eta} + \pdydx{u}{\xi}
\quad \text{and} \quad
\pdydx{u}{t} = \pdydx{u}{\eta}\pdydx{\eta}{t} + \pdydx{u}{\xi}\pdydx{\xi}{t}
= -c\, \pdydx{u}{\eta} +c\, \pdydx{u}{\xi} \ ,
\]
we get
\begin{align*}
\pdydxn{u}{x}{2} &= \pdfdx{\left(\pdydx{u}{x}\right)}{x}
= \pdfdx{\left(\pdydx{u}{\eta} + \pdydx{u}{\xi}\right)}{\eta} \pdydx{\eta}{x}
+ \pdfdx{\left(\pdydx{u}{\eta} + \pdydx{u}{\xi}\right)}{\xi} \pdydx{\xi}{x}\\
&= \pdydxn{u}{\eta}{2}\pdydx{\eta}{x} +
\pdydxnm{u}{\eta}{\xi}{2}{}{}\pdydx{\eta}{x} +
\pdydxnm{u}{\xi}{\eta}{2}{}{}\pdydx{\xi}{x} +
\pdydxn{u}{\xi}{2}\pdydx{\xi}{x}
= \pdydxn{u}{\eta}{2} + 2\pdydxnm{u}{\eta}{\xi}{2}{}{} +
\pdydxn{u}{\xi}{2}
\end{align*}
and
\begin{align*}
\pdydxn{u}{t}{2} &= \pdfdx{\left(\pdydx{u}{t}\right)}{t}
= \pdfdx{\left(-c\pdydx{u}{\eta} + c\pdydx{u}{\xi}\right)}{\eta}\pdydx{\eta}{t}
+ \pdfdx{\left(-c\pdydx{u}{\eta} + c\pdydx{u}{\xi}\right)}{\xi}\pdydx{\xi}{t}\\
&= -c\,\left( -c\, \pdydxn{u}{\eta}{2}
   + c\,\pdydxnm{u}{\eta}{\xi}{2}{}{}\right) +
c\, \left( -c \,\pdydxnm{u}{\xi}{\eta}{2}{}{} + c\,\pdydxn{u}{\xi}{2} 
\right)
= c^2\,\left(\pdydxn{u}{\eta}{2} - 2\pdydxnm{u}{\eta}{\xi}{2}{}{}
+ \pdydxn{u}{\xi}{2} \right) \ .
\end{align*}
Hence, (\ref{wave_oneD}) becomes
\[
0 = \pdydxn{u}{t}{2} - c^2 \pdydxn{u}{x}{2} =
c^2\,\left(\pdydxn{u}{\eta}{2} - 2\pdydxnm{u}{\eta}{\xi}{2}{}{}
+ \pdydxn{u}{\xi}{2} \right) - c^2 \left(
\pdydxn{u}{\eta}{2} + 2\pdydxnm{u}{\eta}{\xi}{2}{}{} +
\pdydxnm{u}{\eta}{\xi}{2}{}{} \right)
= -4c^2 \pdydxnm{u}{\eta}{\xi}{2}{}{} \ .
\]
We have to solve the partial differential equation
\begin{equation} \label{wave_oneD1}
\pdydxnm{u}{\eta}{\xi}{2}{}{} = 0 \ .
\end{equation}
Integrating (\ref{wave_oneD1}) with
respect to $\xi$ yields $\displaystyle \pdydx{u}{\eta} = H(\eta)$ for
some function $H:\RR\rightarrow \RR$.  Integrating
$\displaystyle \pdydx{u}{\eta} = H(\eta)$ with respect to $\eta$ yields
$\displaystyle u(\eta,\xi) = \int H(\eta) \dx{\eta} + G(\xi)$ for some
function $G:\RR\rightarrow \RR$.  If we introduce the function $F$
defined by
$\displaystyle F(\eta) = \int H(\eta) \dx{\eta}$, we get the solution
\[
u(\eta,\xi) = F(\eta) + G(\xi)
\]
for (\ref{wave_oneD1}).  In terms of $x$ and $t$, we get the solution
\begin{equation} \label{wave_oneD1uFG}
u(x,t) = F(x-ct) + G(x+ct)
\end{equation}
for (\ref{wave_oneD}).

We now consider the initial conditions in (\ref{wave_oneD_init}).
The functions $F$ and $G$ satisfy the equations
\[
f(x) = F(x) + G(x) \quad \text{and} \quad g(x) = -c F'(x) + c G'(x) \ .
\]
If we assume that $g$ is locally integrable, we may write
\[
\int_0^x g(s)\dx{s} = -c F(x) + c G(x) + c F(0) - c G(0) \ .
\]
We end up with two linearly independent equations for $F$ and $G$.
\begin{align}
F(x) + G(x) &= f(x) \label{wave_oneD2}
\intertext{and}
F(x) - G(x) &= F(0) - G(0) -\frac{1}{c} \int_0^x g(s)\dx{s} \ .
\label{wave_oneD3}
\end{align}
Adding (\ref{wave_oneD2}) and (\ref{wave_oneD3}), and dividing by $2$
yield
\[
F(x) = \frac{1}{2} f(x) - \frac{1}{2c} \int_0^x g(s)\dx{s} +
\frac{1}{2} \left(F(0) - G(0)\right) \ .
\]
Subtracting (\ref{wave_oneD3}) from (\ref{wave_oneD2}), and dividing by $2$
yield
\[
G(x) = \frac{1}{2} f(x) + \frac{1}{2c} \int_0^x g(s)\dx{s} -
\frac{1}{2} \left(F(0) - G(0)\right) \ .
\]
We finally get
\begin{equation} \label{wave_oneD_sol}
u(x,t) = F(x-ct) + G(x+ct)
= \frac{1}{2} \left(f(x-ct) + f(x+ct)\right)
+ \frac{1}{2c} \int_{x-ct}^{x+ct} g(s)\dx{s} \  .
\end{equation}

If $\displaystyle f \in C^2(\RR)$ and $\displaystyle g \in C^1(\RR)$, then
(\ref{wave_oneD_sol}) provides a solution of class $\displaystyle C^2(\RR)$ for
(\ref{wave_oneD}) with the initial conditions (\ref{wave_oneD_init}).

If $g = 0$, then the solution of the wave equation is
$\displaystyle u(x,t) = \frac{1}{2} \left(f(x-ct) + f(x+ct)\right)$.
We have illustrated in Figure~\ref{wave_FIG1} the behaviour of this
solution with respect to time.

\pdfF{wave_equ/wave_fig1}{The wave equation}{There are two waves
emanating from the initial deformation and moving at speed $c$ in
opposite directions.}{wave_FIG1}

The value of $u$ at a fixed point $x$ and time $t$ is determined by
the values of $f$ at $x\pm ct$ and the values of $g$ between $x-ct$
and $x+ct$.  For this reason, the interval $[x-ct,x+ct]$ is called
the {\bfseries domain of determinacy}\index{Wave Equation!Domain of
Determinacy} for $u$ at $(x,t)$.  Conversely, 
the values of $f$ and $g$ on the interval $[a,b]$ influence only the
values of $u(x,t)$ for $a-ct \leq x \leq b+ct$ and $t>0$.  The region 
$\left\{ (x,t) : t>0 \ \text{and} \ a-ct \leq x \leq b+ct \right\}$ is
called the {\bfseries range of influence}\index{Wave Equation!Range of
Influence} of the interval $[a,b]$. 
These definition are illustrated in Figure~\ref{wave_FIG2}.
A discontinuity of the initial condition on $u$ at $(x_0,0)$ is
instantaneously propagated along the characteristic lines emanating
from $(x_0,0)$

\pdfF{wave_equ/wave_fig2}{Domain of determinacy and range of
influence}{Domain of determinacy and range of influence for the wave
equation.}{wave_FIG2}

\begin{egg}
Sketch the solution $u$ of the wave equation $\displaystyle u_{tt} = c^2 u_{xx}$
with the initial conditions
\[
u_t(x,0) = g(x) = 0 \qquad \text{and} \qquad
u(x,0) = f(x) =
\begin{cases}
h & \quad \text{if} \quad |x| \leq a \\
0 & \quad \text{if} \quad |x| > a
\end{cases}
\]
for $x \in \RR$.  We assume that $a$, $c$ and $h$ are arbitrary
positive constants.

We have from (\ref{wave_oneD_sol}) that $u(x,t) = f(x-ct) +
f(x+ct)$.  Hence, we get the following sketch for the solution.
\begin{center}
\pdfbox{wave_equ/wave_fig10}
\end{center}
\end{egg}

\subsection{Over a Semi-Finite Interval}

In this section, we consider the wave equation
\[
\pdydxn{u}{t}{2} - c^2 \pdydxn{u}{x}{2} = 0 \quad ,
\quad t >0 \text{ and } x > 0 \ ,
\]
with the boundary condition
\begin{equation}\label{wave_oneDHLBC}
u(0,t) = 0  \quad , \quad t>0 \ ,
\end{equation}
and the initial conditions
\[
u(x,0) = f(x) \quad \text{and} \quad \pdydx{u}{t}(x,0) = g(x) \quad ,
\quad x > 0 \ .
\]

As in the previous section, we may use the change of coordinates $\eta
= x-ct$ and $\xi = x +ct$ to reduce the wave equation to
(\ref{wave_oneD1}).  Integrating this equation yields
\begin{equation}\label{wave_oneDHL1}
u(\eta,\xi) = F(\eta) + G(\xi)
\end{equation}
for some functions $F$ and $G$.  In terms of $x$ and $t$, we get the
solution
\[
u(x,t) = F(x-ct) + G(x+ct)
\]
as long as $x-ct>0$.  With the help of the initial conditions as in
the previous section, we get
\begin{equation}\label{wave_oneDHLsolp}
u(x,t) = \frac{1}{2} \left(f(x-ct) + f(x+ct)\right)
+ \frac{1}{2c} \int_{x-ct}^{x+ct} g(s)\dx{s}
\end{equation}
for $x-ct > 0$.

We now consider what happens when $x-ct <0$.  We use the boundary
condition (\ref{wave_oneDHLBC}) to obtain $0 = F(-ct) + G(ct)$ for
$t>0$.  Thus, $F(w) = -G(-w)$ for $w<0$.  We can now proceed as we did
for the case $x-ct>0$.
\begin{align}
u(x,t) &= F(x-ct) + G(x+ct) = -G(ct-x) + G(x+ct) \nonumber \\
&= -\left( \frac{1}{2} f(ct-x) + \frac{1}{2c} \int_0^{ct-x} g(s)\dx{s} -
\frac{1}{2} \left(F(0) - G(0)\right) \right) \nonumber \\
&\qquad + \left( \frac{1}{2} f(x+ct) + \frac{1}{2c} \int_0^{x+ct} g(s)\dx{s}
 - \frac{1}{2} \left(F(0) - G(0)\right) \right) \nonumber \\
&= \frac{1}{2} (f(x+ct) - f(ct-x)) + \frac{1}{2c}
\int_{ct-x}^{ct+x} g(s) \dx{s}
\label{wave_oneDHLsoln}
\end{align}
for $x-ct < 0$.

For $x-ct>0$, The value of $u$ at a fixed point $x$ and time $t$ is
determined by the values of $f$ at $x\pm ct$ and the values of $g$
between $x-ct$ and $x+ct$.  For $x-ct<0$, The value of $u$ at a fixed
point $x$ and time $t$ is determined by the values of $f$ at $ct\pm x$
and the values of $g$ between $ct-x$ and $ct+x$.  The interval
$[ct-x,ct+x]$ is the {\bfseries domain of determinacy}%
\index{Wave Equation!Domain of Determinacy} for $u$ at
$(x,t)$ when $x-ct <0$.

\begin{egg}
Sketch the solution of the wave equation $\displaystyle u_{tt} = c^2 u_{xx}$ for
$x\geq 0$ and $t \in \RR$ with the initial conditions
\[
u(x.0) = \begin{cases}
0 & \quad \text{if} \ 0<x<a \text{ or } x >b \\
h & \quad \text{if} \ a\leq x \leq b
\end{cases}
\]
and $u_t(x.0) = 0$ for $x \geq 0$, and the boundary condition
$u(0,t) = 0$ for $t \geq 0$.

Using (\ref{wave_oneDHLsolp}) and (\ref{wave_oneDHLsoln}), we get the
following sketch for the solution.
\begin{center}
\pdfbox{wave_equ/wave_fig11}
\end{center}
\end{egg}

\subsection{Over a Finite Interval}

We first deduce a graphical method to evaluate the solution $u$
of the wave equation
\[
\pdydxn{u}{t}{2} - c^2 \pdydxn{u}{x}{2} = 0 \quad , \quad t>0
\text{ and } 0<x<L \ ,
\]
with the boundary conditions
\[
u(0,t) = A(t) \quad \text{and} \quad u(L,t) = B(t) \quad , \quad t>0 \ ,
\]
and the initial conditions
\[
u(x,0) = f(x) \quad \text{and} \quad \pdydx{u}{t}(x,0) = g(x) \quad ,
\quad 0 < x < L \ .
\]
The function $A$, $B$, $f$ and $g$ are given.

\pdfF{wave_equ/wave_fig3}{Graphical Method to evaluate the solution
of the wave equation}{Graphical Method to evaluate the solution
of the wave equation in one dimension.}{wave_FIG3}

The equation (\ref{wave_oneD1uFG}) is still true for
$0 \leq x-ct < x+ct \leq L$.  Consider Figure~\ref{wave_FIG3}.
We have that $F$ is constant along
the lines $x=ct+x_1$ and $x=ct+x_2$, and $G$ is constant along the
lines $x=-ct+x_3$ and $x=-ct+x_4$.  Hence,
\begin{align*}
u(\VEC{a}) + u(\VEC{c}) &= F(x_1) + G(x_4) + F(x_2) + G(x_3) \\
&= F(x_1) + G(x_3) + F(x_2) + G(x_4) = u(\VEC{d}) + u(\VEC{b}) \ .
\end{align*}
Thus,
\begin{equation} \label{wave_geo1}
u(\VEC{a}) = u(\VEC{d}) + u(\VEC{b}) - u(\VEC{c})  \ .
\end{equation}

\begin{egg}
Consider Figure~\ref{wave_FIG4}.  The values of $u$ inside the region
$R_1$ are given by (\ref{wave_oneD_sol}).  To find the values of $u$
at $\VEC{a}$ in $R_3$, we use (\ref{wave_geo1}).  $u(\VEC{b})$ is
given by the boundary condition on $x=L$,  $u(\VEC{c})$ and
$u(\VEC{d})$ are given by the values of $u$ on the boundary of $R_1$.
Proceeding recursively, we may evaluate $u$ in any region $R_i$.
\label{wave_egg1}

\pdfF{wave_equ/wave_fig4}{D'Alembert method to solve the wave
equation with boundary and initial conditions}{D'Alembert method to
solve the wave equation with boundary and initial conditions}{wave_FIG4}

Is the solution that we have defined continuous?  Inside the region
$R_i$, $u$ will always be continuous if we assume that $f\in C(]0,L[)$
and $\displaystyle g\in L^1(]0,L[)$.  We only have to consider $u$ along the
characteristic lines.  Referring to Figure~\ref{wave_FIG4},
we see that
\begin{equation} \label{wave_geo2}
u(\VEC{a}) \rightarrow u(\VEC{d}) + B(0) - f(L)
\end{equation}
as $\VEC{a}$ approaches the line $x=-ct+L$.  If $B(0)=f(L)$, then $u$
will be continuous along $x=-ct+L$ delimiting $R_3$.  Similarly, for
$u$ to be continuous along the line $x=ct$ delimiting the region $R_1$,
we need $A(0)=f(0)$.  We have that
$A,B \in C([0,\infty[)$, $\displaystyle g \in L^1(]0,L[)$ and $f\in C([0,L])$
with $B(0)=f(L)$ and $A(0) = f(0)$ are sufficient to guarantee that
$u$ is continuous on the domain
$\{ (x,t) : t\geq 0, \ 0 \leq x \leq L \}$.

The solution $u$ will be of class $\displaystyle C^1$ inside the $R_i$'s if
$\displaystyle f\in C^1(]0,L[)$ and $g \in C(]0,L[)$.  The solution
$u$ will even be of class $\displaystyle C^2$ inside the $R_i$'s if
$\displaystyle f\in C^2(]0,L[)$ and $\displaystyle g \in C^1(]0,L[)$.
The issue with the smoothness of the solution is along the lines $x= ct$ and
$x=-ct+L$, and their reflections on the boundary lines $x=0$ and $x=L$.

If $\displaystyle A,B \in C^2(]0,\infty[)\cap C([0,\infty[)$,
$\displaystyle g\in C^1(]0,L[)\cap C([0,L])$,
$\displaystyle f\in C^2(]0,L[)\cap C([0,L])$,
$A(0)=f(0)$, $A'(0)=g(0)$, $\displaystyle A''(0)= c^2 f''(0)$, $B(0)=f(L)$,
$B'(0)=g(L)$ and $\displaystyle B''(0)= c^2 f''(L)$, then
the solution $u$ will be of class $\displaystyle C^2$ on all the domain.
The conditions on the second order derivatives are necessary to
satisfies the wave equation on the characteristic lines.
The conditions above are the {\bfseries consistency
conditions}\index{Wave Equation!Consistency Conditions} to get a
classical solution of the wave equation.
\end{egg}

\section{Non-Homogeneous Wave Equation}

Consider the {\bfseries non-homogeneous wave equation}%
\index{Wave Equation!Non-Homogeneous}
\begin{equation} \label{nh_wave_oneD}
\pdydxn{u}{t}{2} - c^2 \pdydxn{u}{x}{2} = F(x,t) \quad , \quad
t >0 \ \text{and} \ x \in \RR\ ,  
\end{equation}
with the initial conditions
\begin{equation} \label{nh_wave_oneD_init}
u(x,0)= f(x) \quad \text{and} \quad \pdydx{u}{t}(x,0) = g(x) \quad ,
\quad x \in \RR \ .
\end{equation}
From now on, we assume that $F$, $f$ and $g$ are sufficiently
differentiable.

To find a solution of this problem, we use Green's Integral Formula
\begin{equation}\label{nh_GIF}
\iint_K \left( \pdydx{Q}{x} - \pdydx{P}{t} \right) \dx{x}\dx{t}
= \int_{\partial K} P \dx{x} + Q \dx{t} \ ,
\end{equation}
where $P$ and $Q$ are continuous differentiable functions in an open
set containing $\displaystyle K \in \RR^2$, and $\partial K$ is positively
oriented.  The region $K  = K_{x_0,t_0}$ that we consider is
drawn in Figure~\ref{nh_wave_domain}.

\pdfF{wave_equ/nh_wave_fig1}{Domain of determinacy}
{Domain of determinacy used to find the solution of the
non-homogeneous wave equation (\ref{nh_wave_oneD}) with the initial
conditions (\ref{nh_wave_oneD_init})}{nh_wave_domain}

From (\ref{nh_wave_oneD}), we have
\[
-\iint_{K_{x_0,t_0}} F(x,t) \dx{x}\dx{t} = \iint_{K_{x_0,t_0}} \left(
c^2 \pdydxn{u}{x}{2} - \pdydxn{u}{t}{2} \right) \dx{x}\dx{t}
= \int_{\Gamma} \pdydx{u}{t} \dx{x} + c^2 \pdydx{u}{x} \dx{t} \ ,
\]
where we have used Green's Integral Formula (\ref{nh_GIF}) with
$\displaystyle P = \pdydx{u}{t}$ and
$\displaystyle Q = c^2 \pdydx{u}{x}$, and
$\Gamma = \gamma_1 + \gamma_2 + \gamma_3$ is the positively oriented
boundary of $K_{x_0,t_0}$ as drawn in Figure~\ref{nh_wave_domain}.

On $\gamma_1$, we use the parametric representation $(x,t) = (x,0)$ for
$x_0-ct_0 \leq x \leq x_0+ct_0$ to get
\[
\int_{\gamma_1} \pdydx{u}{t} \dx{x} + c^2 \pdydx{u}{x} \dx{t}
= \int_{x_0-ct_0}^{x_0+ct_0} g(x) \dx{x}
\]
because $\displaystyle \pdydx{u}{t}(x,0) = g(x)$ for $x\in \RR$.
On $\gamma_2$, we use the parametric representation
$(x,t) = (-ct+x_0+ct_0,t)$ for $0 \leq t \leq t_0$ to get
\begin{align*}
\int_{\gamma_2} \pdydx{u}{t} \dx{x} + c^2 \pdydx{u}{x} \dx{t}
& = -c \int_0^{t_0} \left( \pdydx{u}{t} - c \pdydx{u}{x} \right) \dx{t}
= -c \int_0^{t_0} \dfdx{u(-ct+x_0+ct_0,t)}{t} \dx{t} \\
& = -c \left( u(x_0,t_0) - u(x_0+ct_0,0) \right)
=  -c \left( u(x_0,t_0) - f(x_0+ct_0) \right)
\end{align*}
because $\displaystyle u(x,0) = f(x)$ for $x \in \RR$.
On $\gamma_3$, we use the parametric representation
$(x,t) = (c(t_0-t)+x_0-ct_0,t_0-t)$ for 
$0 \leq t \leq t_0$ to get
\begin{align*}
\int_{\gamma_3} \pdydx{u}{t} \dx{x} + c^2 \pdydx{u}{x} \dx{t}
& = -c \int_0^{t_0} \left( \pdydx{u}{t} + c \pdydx{u}{x} \right) \dx{t}
= c \int_0^{t_0} \dfdx{u(c(t_0-t)+x_0-ct_0,t_0-t)}{t} \dx{t} \\
& = c \left( u(x_0-ct_0,0) - u(x_0,t_0) \right)
= c \left( f(x_0-ct_0) -u(x_0,t_0) \right)
\end{align*}
because $\displaystyle u(x,0) = f(x)$ for $x \in \RR$.

Hence,
\begin{align*}
-\iint_{K_{x_0,t_0}} F(x,t) \dx{x}\dx{t} &
= \int_{\gamma_1} \pdydx{u}{t} \dx{x} + c^2 \pdydx{u}{x} \dx{t}
+ \int_{\gamma_2} \pdydx{u}{t} \dx{x} + c^2 \pdydx{u}{x} \dx{t}
+ \int_{\gamma_3} \pdydx{u}{t} \dx{x} + c^2 \pdydx{u}{x} \dx{t} \\
&= \int_{x_0-ct_0}^{x_0+ct_0} g(x) \dx{x} 
+ c \left( f(x_0-ct_0) +  f(x_0+ct_0) \right) -2c u(x_0,t_0) \ .
\end{align*}
If we solved for $u(x_0,t_0)$, we get
\[
u(x_0,t_0) = \frac{1}{2} \left(f(x_0-ct_0) + f(x_0+ct_0)\right)
+ \frac{1}{2c} \int_{x_0-ct_0}^{x_0+ct_0} g(x)\dx{x}
+ \frac{1}{2c} \iint_{K_{x_0,t_0}} F(x,t) \dx{x}\dx{t}
\]

This suggest the solution $u = u_1 + u_2$ for
(\ref{nh_wave_oneD}) and (\ref{nh_wave_oneD_init}), where
\[
u_1(x,t) = \frac{1}{2} \left(f(x-ct) + f(x+ct)\right)
+ \frac{1}{2c} \int_{x-ct}^{x+ct} g(s)\dx{s} 
\]
is a solution of the homogeneous wave equation (\ref{wave_oneD}) and
(\ref{wave_oneD_init}), and
\[
u_2(x,t) = \frac{1}{2c} \iint_{K_{x,t}} F(z,s) \dx{z}\dx{s}
= \frac{1}{2c} \int_0^t \int_{x-c(t-s)}^{x+c(t-s)} F(z,s)\dx{z}\dx{s} \ .
\]

To prove that $u$ is effectively the solution of our non-homogeneous
wave equation, we need to prove that $u_2$ satisfies
(\ref{nh_wave_oneD}) with the initial condition
$\displaystyle u(x,0)= 0$ and $\displaystyle \pdydx{u}{t}(x,0) = 0$
for $x \in \RR$.  To prove this, we need to use the Fundamental
Theorem of Calculus several times.  We obviously have $u_2(x,0) =0$.
We also have that
\begin{align*}
\pdydx{u_2}{t}(x,t) &= \frac{1}{2c} \int_{x}^{x} F(z,t)\dx{z}
+ \frac{1}{2} \int_0^t \big( F(x+c(t-s),s) + F(x-c(t-s),s) \big)
\dx{s} \\
&= \frac{1}{2} \int_0^t \big( F(x+c(t-s),s) + F(x-c(t-s),s) \big) \dx{s} \ .
\end{align*}
Thus, $\displaystyle \pdydx{u_2}{t}(x,0) = 0$ as desired.
Furthermore,
\begin{equation} \label{nh_u2tt}
\pdydxn{u_2}{t}{2}(x,t) = F(x,s)
+ \frac{c}{2} \int_0^t \left( \pdydx{F}{x}(x+c(t-s),s)
- \pdydx{F}{x}(x-c(t-s),s) \right) \dx{s} \ .
\end{equation}
Similarly,
\[
\pdydx{u_2}{x}(x,t) = \frac{1}{2c} \int_0^t
\left( F(x+c(t-s),s) - F(x-c(t-s),s) \right) \dx{s}
\]
and
\begin{equation} \label{nh_u2xx}
\pdydxn{u_2}{x}{2}(x,t) = 
\frac{1}{2c} \int_0^t \left( \pdydx{F}{x}(x+c(t-s),s)
- \pdydx{F}{x}(x-c(t-s),s) \right) \dx{s} \ .
\end{equation}
(\ref{nh_wave_oneD}) follows from (\ref{nh_u2tt}) and (\ref{nh_u2xx}).

\begin{rmk}
We will introduce in Section~\ref{sectDuhamel} another method, the
Duhamel principle, to solve non-homogeneous wave equations.
\end{rmk}

\begin{egg}
Solve the non-homogeneous wave equation
\[
\pdydxn{u}{t}{2} - 4 \pdydxn{u}{x}{2} = e^x + \sin(t) \quad , \quad
x\in \RR\ \text{and} \ t>0 \ ,
\]
with the initial conditions
$\displaystyle u(x,0) = 0$ and
$\displaystyle \pdydx{u}{t}(x,0) = \frac{1}{1+x^2}$ for $x \in \RR$.

The solution is given by
\[
u(x,t) = \frac{1}{2} \left(f(x-ct) + f(x+ct)\right)
+ \frac{1}{2c} \int_{x-ct}^{x+ct} g(s)\dx{s} 
+ \frac{1}{2c} \int_0^t \int_{x-c(t-s)}^{x+c(t-s)} F(z,s)\dx{z}\dx{s} \ ,
\]
where $c=2$, $f(x)=0$ for all $x \in \RR$,
$\displaystyle g(x) = 1/(1+x^2)$ for all
$x\in \RR$, and $\displaystyle F(x,t) = e^x + \sin(t)$.  We get
\begin{align*}
u(x,t) &= \frac{1}{4} \int_{x-2t}^{x+2t} \frac{1}{1+s^2} \dx{s} 
+ \frac{1}{4} \int_0^t \int_{x-2(t-s)}^{x+2(t-s)} (e^z + \sin(s))
\dx{z}\dx{s} \\
&= \frac{1}{4} \arctan(s) \bigg|_{x-2t}^{x-2t}
+ \frac{1}{4} \int_0^t \left(e^{x+2(t-s)} - e^{x-2(t-s)} +
  4(t-s)\sin(s) \right) \dx{s} \\
&= \frac{1}{4} \left( \arctan(x-2t) - \arctan(x-2t) \right)
+ \frac{1}{8} \left( e^{x+2t} + e^{x-2t} \right) - \frac{e^x}{4}
+ t - \sin(t) \ .
\end{align*}
\end{egg}

\begin{rmk}
For the previous example, there is a more direct approach to solve the
non-homogeneous wave equation.

It is easy to see that $u_1(x,t)=-\sin(t)$ is a solution of
\[
\pdydxn{u}{t}{2} - 4 \pdydxn{u}{x}{2} =  \sin(t) \quad , \quad
x\in \RR \text{ and } t>0 \ ,
\]
and $\displaystyle u_2(x,t)= - e^x/4$ is a solution of
\[
\pdydxn{u}{t}{2} - s^4 \pdydxn{u}{x}{2} =  e^x \quad , \quad
x\in \RR\ \text{and} \ t>0 \ .
\]
The solution of the non-homogeneous problem of the previous example is
therefore $u = u_1 + u_2 + u_3$, where $u_3$ is a solution of the
homogeneous wave equation
\[
\pdydxn{u}{t}{2} - 4 \pdydxn{u}{x}{2} = 0 \quad , \quad
x\in \RR\ \text{and} \ t>0 \ ,
\]
with the initial conditions
\begin{align*}
u(x,0) &= 0 - u_1(x,0) - u_2(x,0) = \frac{e^x}{4}
\intertext{and}
\pdydx{u}{t}(x,0) &= \frac{1}{1+x^2} - \pdydx{u_1}{t}(x,0) -
\pdydx{u_2}{t}(x,0) = \frac{1}{1+x^2} + 1  \ .
\end{align*}
We find
\begin{align*}
u_3(x,t) &= \frac{1}{2}\left( \frac{e^{x-2t}}{4} + \frac{e^{x+2t}}{4} \right)
+ \frac{1}{4} \int_{x-2t}^{x+2t} \left( \frac{1}{1+s^2} + 1
\right)\dx{s} \\
&= \frac{e^{x-2t}}{8} + \frac{e^{x+2t}}{8}
+ \frac{1}{4} \left(\arctan(s) + s\right)\bigg|_{x-2t}^{x+2t} \\
&= \frac{e^{x-2t}}{8} + \frac{e^{x+2t}}{8}
+ \frac{1}{4} \left(\arctan(x+2t) - \arctan(x-2t)\right) + t \ .
\end{align*}
\end{rmk}

\section{Strong Solution of the Wave equation} \label{SectWaveEq1DStSol}

Consider the hyperbolic linear partial differential equation of order two
\begin{equation}  \label{we_classic_sol}
L(\eta,\xi,\diff)u = \pdydxnm{u}{\eta}{\xi}{2}{}{} + A_{1,0}
\pdydx{u}{\eta} + A_{0,1} \pdydx{u}{\xi} + A_{0,0} u = 0
\end{equation}
which is the general form of a linear homogeneous hyperbolic partial
differential equation of order two after reduction using a change of
variables provided by level curves which are characteristic curves.

The adjoint of $L(\eta,\xi,\diff)$ is defined by
\[
L^\ast(\eta,\xi,\diff)\phi = \pdydxnm{\phi}{\eta}{\xi}{2}{}{} -
\pdfdx{\left(A_{1,0}\phi\right)}{\eta} -
\pdfdx{\left(A_{0,1}\phi\right)}{\xi} + A_{0,0} \phi
\]
for $\displaystyle \phi \in \DD(\RR^2)$. 

$u:\Omega \to \RR$ is a strong solution of $L(\eta,\xi,\diff)u=0$
over an open set $\displaystyle \Omega \subset \RR^2$ if
\begin{equation} \label{we_strong_sol}
\int_\Omega u(\eta,\xi)\, L^\ast(\eta,\xi,\diff)\phi(\eta,\xi)
\dx{\eta}\dx{\xi} = 0
\end{equation}
for all $\phi \in \DD(\Omega)$.  This is satisfied if
$\displaystyle u \in C^2(\Omega)$ is a solution of
(\ref{we_classic_sol}) as can be shown using integration by parts.

Suppose that $\Omega$ is traversed by a characteristic curve $\Gamma$
along which the smoothness of $u$ is broken.  Since the characteristic
curves are the level curves of the change of variables used to get the
reduced form of the hyperbolic equation, we may assume that the
characteristic curve $\Gamma$ traversing $\Omega$ is the line $\xi=0$
as in Figure~\ref{wave_FIG5}.

\pdfF{wave_equ/wave_fig5}{The Transmission condition for a strong
solution}{The Transmission condition for a strong
solution across a characteristic line represented by the $\eta$ axis.}
{wave_FIG5}

We have $\Omega = \Omega_+ \cup \Omega_0 \cup \Omega_-$, where
$\displaystyle \Omega_+ = \left\{ (\eta,\xi) \in \Omega : \xi>0 \right\}$,
$\displaystyle \Omega_- = \left\{ (\eta,\xi) \in \Omega : \xi<0 \right\}$
and $\Omega_0 = \left\{ (\eta,\xi) \in \Omega : \xi=0 \right\}$.

Suppose that $u:\Omega \to \RR$ is a strong solution which also satisfies
(\ref{we_classic_sol}) in the classical sense on $\Omega_+$ and
$\Omega_-$.  The fact that $u$ is a strong solution imposes some extra
conditions on $u$ along $\Omega_0$.

We may assume that $u(\eta,\xi) = 0$ for $(\eta,\xi) \not\in \Omega$.
We may also assume that $\phi \in \DD(\Omega)$ is in fact
in $\displaystyle \DD(\RR^2)$ and has a compact support in $\Omega$.  This is
something that we will often do to simplify the discussion when we
have to work on an open subset of $\displaystyle \RR^n$.

Using integration by parts, we get for $\phi \in \DD(\Omega)$ that
\begin{align*}
& \int_\Omega u(\eta,\xi)\, L^\ast(\eta,\xi,\diff)\phi(\eta,\xi)
\dx{\eta}\dx{\xi} \\
&= \int_{\Omega_+} u(\eta,\xi)\, L^\ast(\eta,\xi,\diff)\phi(\eta,\xi)
\dx{\eta}\dx{\xi}
+\int_{\Omega_-} u(\eta,\xi)\, L^\ast(\eta,\xi,\diff)\phi(\eta,\xi)
\dx{\eta}\dx{\xi} \\
&= \int_0^{+\infty} \int_{-\infty}^{+\infty} u \left(
\pdydxnm{\phi}{\eta}{\xi}{2}{}{} -\pdfdx{\left(A_{1,0}\phi\right)}{\eta} -
\pdfdx{\left(A_{0,1}\phi\right)}{\xi}
+ A_{0,0} \phi \right) \dx{\eta}\dx{\xi} \\
& \quad +  \int_{-\infty}^0 \int_{-\infty}^{+\infty} u \left(
\pdydxnm{\phi}{\eta}{\xi}{2}{}{}-\pdfdx{\left(A_{1,0}\phi\right)}{\eta} -
\pdfdx{\left(A_{0,1}\phi\right)}{\xi} + A_{0,0}\phi \right) \dx{\eta}\dx{\xi} \\
&= \int_0^{+\infty} \left( u
\pdydx{\phi}{\xi}\bigg|_{-\infty}^{+\infty} - \int_{-\infty}^{+\infty}
\pdydx{u}{\eta}\pdydx{\phi}{\xi} \dx{\eta} \right) \dx{\xi}
- \int_0^{+\infty} \left( u A_{1,0} \phi\bigg|_{-\infty}^{+\infty}
- \int_{-\infty}^{+\infty} \pdydx{u}{\eta} A_{1,0} \phi
\dx{\eta}\right) \dx{\xi} \\
& \quad + \int_0^{+\infty} \int_{-\infty}^{+\infty}
u \left(-\pdfdx{\left(A_{0,1}\phi\right)}{\xi}
+ A_{0,0} \phi \right) \dx{\eta}\dx{\xi} \\
& \quad + \int_{-\infty}^0 \left( u
\pdydx{\phi}{\xi}\bigg|_{-\infty}^{+\infty} - \int_{-\infty}^{+\infty}
\pdydx{u}{\eta}\pdydx{\phi}{\xi} \dx{\eta} \right) \dx{\xi}
- \int_{-\infty}^0 \left( u A_{1,0} \phi\bigg|_{-\infty}^{+\infty}
- \int_{-\infty}^{+\infty} \pdydx{u}{\eta} A_{1,0} \phi
\dx{\eta}\right) \dx{\xi} \\
& \quad + \int_{-\infty}^0 \int_{-\infty}^{+\infty}
u \left(-\pdfdx{\left(A_{0,1}\phi\right)}{\xi}
+ A_{0,0} \phi \right) \dx{\eta}\dx{\xi} \\
&= \int_{-\infty}^{+\infty} \int_0^{+\infty} \left(
- \pdydx{u}{\eta}\pdydx{\phi}{\xi} + \pdydx{u}{\eta} A_{1,0} \phi
- u \pdfdx{\left(A_{0,1}\phi\right)}{\xi}
+ u A_{0,0} \phi \right) \dx{\xi}\dx{\eta} \\
&\quad + \int_{-\infty}^{+\infty} \int_{-\infty}^0 \left(
-\pdydx{u}{\eta}\pdydx{\phi}{\xi} + \pdydx{u}{\eta} A_{1,0} \phi
- u \pdfdx{\left(A_{0,1}\phi\right)}{\xi} + u A_{0,0} \phi \right)
\dx{\xi}\dx{\eta} \\
&= \int_{-\infty}^{+\infty} \left( -\pdydx{u}{\eta}
\phi\bigg|_0^{+\infty} + \int_0^{+\infty}
\pdydxnm{u}{\eta}{\xi}{2}{}{} \phi \dx{\xi} \right) \dx{\eta}
- \int_{-\infty}^{+\infty} \left( u A_{0,1} \phi\bigg|_0^{+\infty}
- \int_0^{+\infty} \pdydx{u}{\xi} A_{0,1} \phi \dx{\xi}\right) \dx{\eta} \\
&\quad + \int_0^{+\infty} \int_{-\infty}^{+\infty}
\left(\pdydx{u}{\eta} A_{1,0}\phi + u A_{0,0} \phi \right) \dx{\eta}\dx{\xi} \\
&\quad + \int_{-\infty}^{+\infty} \left( -\pdydx{u}{\eta}
\phi\bigg|_{-\infty}^0 + \int_{-\infty}^0
\pdydxnm{u}{\eta}{\xi}{2}{}{} \phi \dx{\xi} \right) \dx{\eta}
- \int_{-\infty}^{+\infty} \left( u A_{0,1} \phi\bigg|_{-\infty}^0
- \int_{-\infty}^0 \pdydx{u}{\xi} A_{0,1} \phi
\dx{\xi}\right) \dx{\eta} \\
&\quad + \int_{-\infty}^0 \int_{-\infty}^{+\infty}
\left(\pdydx{u}{\eta} A_{1,0}\phi + u A_{0,0} \phi \right) \dx{\eta}\dx{\xi} \\
&= \int_{\Omega_+} \left( L(\eta,\xi,\diff) u(\eta,\xi)\right)\phi(\eta,\xi)
\dx{\eta}\dx{\xi}
+ \int_{\Omega_-} \left(L(\eta,\xi,\diff) u(\eta,\xi)\right)\phi(\eta,\xi)
\dx{x}\dx{t} \\
&\quad +\int_{\Omega_0} \left( u_+(\eta,0) A_{0,1}(\eta,0) \phi(\eta,0) +
\pdydx{u_+}{\eta}(\eta,0)\phi(\eta,0) \right) \dx{\eta} \\
&\quad - \int_{\Omega_0} \left( u_-(\eta,0) A_{0,1}(\eta,0) \phi(\eta,0) +
\pdydx{u_-}{\eta}(\eta,0)\phi(\eta,0) \right) \dx{\eta} \\
&=\int_{\Omega_0} \left(u_+(\eta,0) -u_-(\eta,0)\right)
A_{0,1}(\eta,0) \phi(\eta,0)\dx{\eta}
+\int_{\Omega_0} \left( \pdydx{u_+}{\eta}(\eta,0)
- \pdydx{u_-}{\eta}(\eta,0) \right)\phi(\eta,0) \dx{\eta}
\end{align*}
where $\displaystyle u_\pm(\eta,0) = \lim_{\xi\rightarrow 0^\pm} u(\eta,\xi)$
and $\displaystyle \pdydx{u_\pm}{\eta}(\eta,0) =
\lim_{\xi\rightarrow 0^\pm} \pdydx{u}{\eta}(\eta,\xi)$.  We assume
that these limits exist.

To get $\displaystyle \int_R u(\eta,\xi)\, L^\ast(\eta,\xi,\diff)\phi(\eta,\xi)
\dx{\eta}\dx{\xi} =0$ for all $\phi \in \DD(\Omega)$, we must have
\[
  \int_{\Omega_0} \big(u_+(\eta,0) - u_-(\eta,0)\big) A_{0,1}(\eta,0)
\phi(\eta,0) \dx{\eta} +\int_{\Omega_0} \left( \pdydx{u_+}{\eta}(\eta,0)
- \pdydx{u_-}{\eta}(\eta,0)\right) \phi(\eta,0) \dx{\eta} = 0
\]
for all $\phi \in \DD(\Omega)$.  This is the
{\bfseries transmission condition}\index{Wave Equation!Transmission Condition}
to get a strong solution.

If we consider the wave equation
$\displaystyle L(\eta,\xi,\diff)u = \pdydxnm{u}{\eta}{\xi}{2}{}{} = 0$, the
transmission condition for $u$ to be a strong solution is
\begin{equation} \label{wave_egg4}
\int_{\Omega_0} \left( \pdydx{u_+}{\eta}(\eta,0)
- \pdydx{u_-}{\eta}(\eta,0)\right) \phi(\eta,0) \dx{\eta} = 0
\end{equation}
for all $\displaystyle \phi\in \DD(\Omega)$.
Since this is true for all test functions $\phi \in \DD(\Omega)$, we must
have
\[
\pdydx{u_+}{\eta}(\eta,0) - \pdydx{u_-}{\eta}(\eta,0) = 0
\]
for almost every $\eta \in \RR$.  If $u$ and $\displaystyle
\pdydx{u}{\eta}$ can be continuously extended to the $\xi$ axis, then
this shows that $\displaystyle u_+(\eta,0) - u_-(\eta,0)$ is constant
for $\eta \in \RR$.

\begin{egg}
In Example~\ref{wave_egg1}, if the functions $f$, $g$, $A$ and $B$ are
$\displaystyle C^\infty$. then the solution $u$ constructed in the example is a
strong solution because the transmission condition is satisfied.
We have a constant jump discontinuity along the characteristic 
lines $x = ct$ and $x = -ct +L$, and their reflections through the
boundary $x=0$ and $x=L$.  For instance, it follows from
(\ref{wave_geo2}) that 
$\displaystyle \lim_{\substack{a \to d\\ a \in E_3}} u(a) = u(d)+ B(0) - f(L)$
where the region $E_3$ is defined in Figure~\ref{wave_FIG4}.
\end{egg}

\section{Separation of variables}

In this section, we use the method of separation of variables to solve
the wave equation with some initial and boundary conditions.

The method of separation of variables can generally be used to find
strong solutions for the wave equation.  We will come back on this
subject later on.  The next theorem states that if the initial
conditions are smooth enough, then the method of separation of
variables yields a classical solution.

\begin{theorem} \label{wave_spv_conditions}
Suppose that $\displaystyle f \in C^3([0,L])$,
$\displaystyle g\in C^2([0,L])$, $f(0)=f(L)=0$,
$f''(0)=f''(L)=0$ and $g(0)=g(L)=0$.  Let
\begin{equation} \label{wave_class_sol1}
u(x,t) = \sum_{n=1}^\infty \left(a_n\cos\left(\frac{n\pi t}{L}\right)
+ b_n \sin\left(\frac{n\pi t}{L}\right)\right)
\sin\left(\frac{n\pi x}{L}\right)
\end{equation}
with
\[
a_n = \frac{2}{L} \int_0^L f(x) \,\sin\left(\frac{n\pi x}{L}\right)
\dx{x} \quad \text{and} \quad
b_n = \frac{2}{n\pi} \int_0^L g(x)\, \sin\left(\frac{n\pi x}{L}\right)
\dx{x} \ .
\]
Then $u$ is a classical solution of the wave equation
\begin{equation} \label{wave_class_sol2}
\pdydxn{u}{t}{2} - \pdydxn{u}{x}{2} = 0 \quad , \quad t>0 \text{ and }
0<x<L \ ,
\end{equation}
with the boundary condition
\[
u(0,t) = 0 \quad \text{and} \quad u(L,t) = 0 \quad , \quad t>0 \  ,
\]
and the initial condition
\[
u(x,0) = f(x) \quad \text{and} \quad \pdydx{u}{t}(x,0) = g(x) \quad ,
\quad 0 < x < L \ .
\]
\end{theorem}

\begin{proof}
To prove the theorem, it is enough to show that
\begin{align*}
u(x,t) &=
\sum_{n=1}^\infty \left(a_n\cos\left(\frac{n\pi t}{L}\right)
+ b_n \sin\left(\frac{n\pi t}{L}\right)\right)
\sin\left(\frac{n\pi x}{L}\right) \ , \\
\pdydx{u}{t}(x,t) &=
\sum_{n=1}^\infty \frac{n\pi}{L} \left(-a_n \sin\left(\frac{n\pi t}{L}\right)
+ b_n \cos\left(\frac{n\pi t}{L}\right)\right)
\sin\left(\frac{n\pi x}{L}\right) \ , \\
\pdydxn{u}{t}{2}(x,t) &=
-\sum_{n=1}^\infty \frac{n^2\pi^2}{L^2} \left( a_n
\cos\left(\frac{n\pi t}{L}\right)
+ b_n \sin\left(\frac{n\pi t}{L}\right)\right)
\sin\left(\frac{n\pi x}{L}\right) \ , \\
\pdydx{u}{x}(x,t) &=
\sum_{n=1}^\infty \frac{n\pi}{L} \left(a_n\cos\left(\frac{n\pi t}{L}\right)
+ b_n \sin\left(\frac{n\pi t}{L}\right)\right)
\cos\left(\frac{n\pi x}{L}\right) \ , \\
\pdydxn{u}{x}{2}(x,t) &= -\sum_{n=1}^\infty \frac{n^2\pi^2}{L^2}
\left(a_n\cos\left(\frac{n\pi t}{L}\right)
+ b_n \sin\left(\frac{n\pi t}{L}\right)\right)
\sin\left(\frac{n\pi x}{L}\right)
\intertext{and}
\pdydxnm{u}{t}{x}{2}{}{}(x,t) &=
\sum_{n=1}^\infty \frac{n^2\pi^2}{L^2}
\left(-a_n \sin\left(\frac{n\pi t}{L}\right)
+ b_n \cos\left(\frac{n\pi t}{L}\right)\right)
\cos\left(\frac{n\pi x}{L}\right)
\end{align*}
converge uniformly on $[0,L]\times[0,T]$ whatever $T>0$.

Assuming the uniform convergence of the series above, we have that $u$
is continuous on $[0,L]\times [0,\infty[$ because it is continuous on
each set $[0,L]\times [0,T]$ being the uniform limit in
(\ref{wave_class_sol1}) of continuous functions.  We can also
interchange the summation in (\ref{wave_class_sol1}) and the
derivatives $\displaystyle \pdydx{}{t}$, $\displaystyle \pdydxn{}{t}{2}$,
$\displaystyle \pdydx{}{x}$, $\displaystyle \pdydxn{}{x}{2}$ and
$\displaystyle \pdydxnm{}{x}{t}{2}{}{}$.  Since each term of the
series in (\ref{wave_class_sol1}) is twice differentiable, it follows
that the same is true for $u$.  Thus
$\displaystyle u\in C^2(]0,L[\times]0,\infty[)$.
Moreover, $u$ satisfies (\ref{wave_class_sol2}) because each term of
the series in (\ref{wave_class_sol1}) satisfies
(\ref{wave_class_sol2}).  The boundary conditions are also satisfied
by each term of the series in (\ref{wave_class_sol1}).  The initial
conditions are satisfied because the $a_n$ and $n\pi b_n/L$ are the
coefficients for the Fourier sine series of $f$ and $g$ respectively
on the interval $[0,L]$.

To prove the convergence of all the series above, it is enough to
prove that
$\displaystyle \sum_{n=1}^\infty n^2|a_n|$ and
$\displaystyle \sum_{n=1}^\infty n^2|b_n|$ converge.

Using integration by parts, we find
\[
a_n = \frac{2}{L} \int_0^L f(x) \sin\left(\frac{n\pi x}{L}\right) \dx{x}
= -\frac{2}{L} \left( \frac{L}{n\pi}\right)^3 
\int_0^L \dydxn{f}{x}{3}(x) \cos\left(\frac{n\pi x}{L}\right) \dx{x} \  .
\]
Thus, for $N>0$, we have
\begin{align*}
0 &\leq \frac{2}{L} \int_0^L \left( \dydxn{f}{x}{3}(x) +
\left(\frac{\pi}{L}\right)^3 \sum_{n=1}^N n^3 a_n
\cos\left(\frac{n\pi x}{L}\right) \right)^2 \dx{x} \\
&= \frac{2}{L} \int_0^L \left( \dydxn{f}{x}{3}(x) \right)^2 \dx{x}
+ \frac{4}{L} \left(\frac{\pi}{L}\right)^3
\sum_{n=1}^N n^3 a_n \underbrace{\int_0^L \dydxn{f}{x}{3}(x)
\cos\left(\frac{n\pi x}{L}\right)
\dx{x}}_{=-(a_n L/2)(n \pi/L)^3} \\
&\qquad \qquad + \left(\frac{\pi}{L}\right)^6 \sum_{n=1}^N \sum_{m=1}^N
n^3 a_n m^3 a_m \underbrace{\int_0^L \cos\left(\frac{n\pi x}{L}\right)
\cos\left(\frac{m\pi x}{L}\right) \dx{x}}_{=
\begin{cases} 0 &\text{ if } n\neq m\\ 1/2 & \text{ if } n = m
\end{cases}} \\
&= \frac{2}{L} \int_0^L \left( \dydxn{f}{x}{3}(x) \right)^2 \dx{x}
- 2 \left(\frac{\pi}{L}\right)^6 \sum_{n=1}^N n^6 a_n^2
+ \left(\frac{\pi}{L}\right)^6 \sum_{n=1}^N n^6 a_n^2 \ .
\end{align*}
It follows that
\[
S_N \equiv \sum_{n=1}^N n^6 a_n^2 \leq
\frac{2}{L} \left(\frac{L}{\pi}\right)^6
\int_0^L \left( \dydxn{f}{x}{3}(x) \right)^2 \dx{x} <
\infty
\]
for all $N>0$.  Since
$\displaystyle \left\{ S_N \right\}_{N=1}^\infty$ is a bounded increasing 
sequence, it converges.  Namely, $\displaystyle \sum_{n=1}^\infty n^6 a_n^2$
is a convergent series of positive terms.

Using integration by parts, we find
\[
b_n = \frac{2}{n\pi} \int_0^L g(x) \, \sin\left(\frac{n\pi x}{L}\right) \dx{x}
= -\frac{2}{L} \left( \frac{L}{n\pi}\right)^3 \int_0^L
\dydxn{g}{x}{2}(x) \, \sin\left( \frac{n\pi x}{L}\right) \dx{x} \  .
\]
A reasoning similar to the previous one shows that
$\displaystyle \sum_{n=1}^\infty n^6 b_n^2$
is a convergent series of positive terms.

Since
\[
n^2 |a_n| = \frac{1}{2} \left( \frac{1}{n^2} + n^6 a_n^2\right) -
\frac{1}{2} \left( \frac{1}{n} - n^3 |a_n| \right)^2 \leq \frac{1}{2}
\left( \frac{1}{n^2} + n^6 a_n^2\right)
\]
for all $n>0$, we can use the comparison test for series of positive
terms to conclude that $\displaystyle \sum_{n=1}^\infty n^2 |a_n|$
converges.  Similarly, we have that
$\displaystyle n^2 |b_n| \leq \frac{1}{2} \left( \frac{1}{n^2} + n^6
  b_n^2\right)$ and $\displaystyle \sum_{n=1}^\infty n^2 |b_n|$ converges.
\end{proof}

Theorems like the previous one can be given for different boundary
conditions.  We will not provide an encyclopedic list of the results.

\begin{rmk}
The function
$\displaystyle u_n(x,t) = \left(a_n\cos\left(\frac{n\pi t}{L}\right)
+ b_n \sin\left(\frac{n\pi t}{L}\right)\right)
\sin\left(\frac{n\pi x}{L}\right)$ is called the
{\bfseries $\mathbf{n^{th}}$-normal mode}\index{Wave Equation!$n^{th}$-Normal
Mode} or {\bfseries $\mathbf{n^{th}}$-harmonic}\index{Wave
Equation!$n^{th}$-Harmonic}.  The first-harmonic is called the
{\bfseries fundamental harmonic}\index{Wave Equation!Fundamental Harmonic}.
The frequency of the fundamental harmonic is $1/(2L)$.  All the other
harmonics have frequencies which are integer multiples of $1/(2L)$.  This is
the reason why a well tuned instrument can produce good quality tones.
\end{rmk}

\begin{egg}
For our first example, we find the deflection $u(x,t)$ of a vibrating
string of length $L=\pi$ attached at both ends.  We assume that the
tension $T$ and density $\rho$ of the string are such that
$c^2 = T/\rho = 1$.  The initial deflection is
$f(x) = 0.1 \sin(x)$ and the initial velocity is $g(x) = -0.2 \sin(x)$.

The solution is given by (\ref{wave_class_sol1}) with $L=\pi$,
\begin{equation} \label{wesvegg1a}
a_n = \frac{2}{\pi} \int_0^\pi f(x) \sin(n x) \dx{x} =
\begin{cases}
0.1 & \qquad \text{if} \quad n=1 \\
0 & \qquad \text{if} \quad n>1 \\
\end{cases}
\end{equation}
and
\begin{equation} \label{wesvegg1b}
b_n = \frac{2}{n\pi} \int_0^\pi g(x) \sin(n x) \dx{x} =
\begin{cases}
-0.2 & \qquad \text{if} \quad n=1 \\
0 & \qquad \text{if} \quad n>1 \\
\end{cases}
\end{equation}
Thus,
\[
u(x,t) = \big(0,1 \cos(t) - 0.2 \sin(t) \big) \sin(x) \ .
\]
There was no need to compute the integrals to find the values of $a_n$
and $b_n$.  From $f(x) = u(x,0)$, we have
\[
0.1 \sin(x) = \sum_{n=1}^\infty a_n \sin(n x) \ .
\]
By uniqueness of the Fourier series, we get (\ref{wesvegg1a}).  Likewise,
from $\displaystyle g(x) = \pdydx{u}{t}(x,0)$, we have
\[
-0.2 \sin(x) = \sum_{n=1}^\infty n b_n \sin(n x) \ .
\]
This time, by uniqueness of the Fourier series, we get
(\ref{wesvegg1b}).
\end{egg}

In the following examples, we will find the solutions starting from
separation of variables.  This allows us to deal with boundary
conditions different than those in Theorem~\ref{wave_spv_conditions}
if needed.  The presentation will be a little bit formal.  We will not
study the convergence of the series solutions.

\begin{egg}
We again consider a special case of Example~\ref{wave_egg1}.
\begin{equation} \label{wave_spv_egg1_e}
\pdydxn{u}{t}{2} - c^2 \pdydxn{u}{x}{2} = 0 \quad , \quad t>0
\ \text{and} \ 0 < x <L \ ,
\end{equation}
with the boundary conditions $u(0,t) = 0$ and $u(L,t) = 0$ for $t>0$,
and the initial conditions
$u(x,0) = f(x)$ and $\displaystyle \pdydx{u}{t}(x,0) = g(x)$ for
$0 < x < L$.              \label{wave_spv_egg1}

\subI{Separation of Variables}
If we substitute $u(x,t) = F(x)G(t)$ in (\ref{wave_spv_egg1_e}), we get
\[
F(x)\,\dydxn{G}{t}{2}(t) = c^2 \dydxn{F}{x}{2}(x)\, G(t) \ .
\]
Thus, after dividing both sides by $\displaystyle c^2\,F(x)G(t)$, we get
\[
\frac{1}{c^2\,G(t)} \, \dydxn{G}{t}{2}(t) =
\frac{1}{F(x)}\, \dydxn{F}{x}{2}(x)
\]
for $t>0$ and $0<x<L$.  Since the right hand side is independent of
$t$ and the left hand side is independent of $x$, we get
\[
\frac{1}{c^2\,G(t)} \, \dydxn{G}{t}{2}(t) = \frac{1}{F(x)}\,
\dydxn{F}{x}{2}(x) = k
\]
for $t>0$ and $0<x<L$, and some constant $k$.  We end up with two
ordinary differential equations.
\begin{equation} \label{wave_spv_egg1_ode}
\dydxn{F}{x}{2}(x) -k F(x) = 0 \quad \text{and}
\quad \dydxn{G}{t}{2}(t) - c^2\,k\,G(t)) = 0 \ .
\end{equation}

The first ordinary differential equation in (\ref{wave_spv_egg1_ode})
satisfies two boundary conditions.   From $\displaystyle u(0,t) = 0$, we get
$F(0)G(t)=0$.  Since we assume that $G$ is non-null, we get
$F(0)=0$.  Similarly, from $\displaystyle u(L,t) = 0$, we
get $F(L)G(t)=0$.  Again, since we assume that $G$ is non-null, we
get $F(L)=0$.  The boundary conditions for 
the first ordinary differential equation in (\ref{wave_spv_egg1_ode})
are $F(0)=F(L)=0$.

\subI{Simple Functions}
We first consider the boundary value problem
\begin{equation} \label{wave_spv_egg1_bvp}
\dydxn{F}{x}{2}(x) -k F(x) = 0 \quad \text{with} \quad F(0)=F(L)=0  \ .
\end{equation}
The form of the general solution of the boundary value problem above
is determined by the roots of the characteristic equation
$\displaystyle \lambda^2-k=0$.

If $k>0$, the roots of the characteristic equation are $\pm \sqrt{k}$.
Since the roots are real, the solution of the ordinary differential
equation is of the form
$\displaystyle F(x) = A e^{\sqrt{k}\,x} + B e^{-\sqrt{k}\, x}$.
However, $F(0)=0$ implies that $A + B=0$ and $F(L)=0$ implies
$\displaystyle A e^{L\sqrt{k}} + B e^{-L\sqrt{k}} = 0$.
The only solution of the system
\[
\begin{pmatrix}
1 & 1 \\ e^{L\sqrt{k}} & e^{-L\sqrt{k}}
\end{pmatrix}
\begin{pmatrix}
A \\ B
\end{pmatrix}
=
\begin{pmatrix}
0 \\ 0
\end{pmatrix}
\]
is $A=B=0$ because
\[
\det
\begin{pmatrix}
1 & 1 \\  e^{L\sqrt{k}} &  e^{-L\sqrt{k}}
\end{pmatrix}
= e^{-L\sqrt{k}} - e^{L\sqrt{k}} =
e^{-L\sqrt{k}}\left(1-e^{2L\sqrt{k}}\right) \neq 0
\]
for $k\neq 0$.  Therefore, the null solution is the only solution
of the boundary value problem (\ref{wave_spv_egg1_bvp}) for $k>0$.

If $k=0$, the solution of the ordinary differential equation is
$F(x)=B x+ A$.  However, $F(0)=0$ implies that $A=0$ and $F(L)=0$
implies that $B L + A = 0$.  Thus, $A = B =0$.  The boundary value problem
(\ref{wave_spv_egg1_bvp}) has no non-null solutions for $k =0$.

If $k<0$, the roots of the characteristic equation are $\pm i \sqrt{-k}$.
Since the roots are complex, the general solution of the ordinary
differential equation is of the form $\displaystyle
F(x) = A \cos\left(\sqrt{-k}\,x\right) + B \sin\left(\sqrt{-k}\,x\right)$.
Hence, $F(0)=0$ implies that $A =0$ and $F(L)=0$ implies that
$B \sin\left(L\sqrt{-k}\right) = 0$.  If we take $B=0$, we get the
null solution.  We must therefore have
$\sin\left(L\sqrt{-k}\right) = 0$ with $k< 0$.  This implies
that $\displaystyle k = k_n = -\left(n\pi/L\right)^2$ for $n>0$.  The
boundary value problem (\ref{wave_spv_egg1_bvp}) has non-null solutions only for
$\displaystyle k=k_n\equiv -\left(n\pi/L\right)^2<0$ with $n>0$, and
the solutions associated to $k_n$ is of the form
\[
F(x)=F_n(x) \equiv B_n \sin\left(\frac{n\pi\,x}{L}\right) \ .
\]

For $\displaystyle k= k_n=-\left(n\pi/L\right)^2$, the second ordinary
differential equation in (\ref{wave_spv_egg1_ode}) becomes
\[
\dydxn{G}{t}{2}(t) + \left(\frac{c n\pi}{L}\right)^2\,G(t)) = 0
\]
for $n>0$.  For each $n$, this is a second order ordinary differential
equation with the characteristic equation
$\displaystyle \lambda^2 +\left(n c\pi/L\right)^2=0$.  The two
roots of this equation are the complex numbers
$\displaystyle \lambda_{\pm} = \pm (c n\pi/L)i$.
The general solution of this ordinary differential equation is therefore
\[
G(t) = G_n(t) \equiv C_n \cos\left(\frac{cn\pi t}{L}\right)
+ D_n \sin\left(\frac{cn\pi t}{L}\right) \  .
\]

We have found the simple functions
\[
u_n(x,t) \equiv F_n(x)G_n(t) =
\left( a_n \cos\left(\frac{cn\pi t}{L}\right)
+ b_n \sin\left(\frac{cn\pi t}{L}\right) \right)
\sin\left(\frac{n\pi\,x}{L}\right)
\]
for $n>0$ that satisfy the partial differential equation in
(\ref{wave_spv_egg1_e}) and the boundary conditions.  The constant
$a_n$ is the product of the constants $B_n$ and $C_n$, and $b_n$ is
the product of $B_n$ and $D_n$.  We now need to determine the values
of $a_n$ and $b_n$ to satisfy the initial conditions.

\subI{Initial condition}
We seek a solution of the form
\[
u(x,t) = \sum_{n=1}^\infty u_n(x,t)
= \sum_{n=1}^\infty
\left( a_n \cos\left(\frac{cn\pi t}{L}\right)
+ b_n \sin\left(\frac{cn\pi t}{L}\right) \right)
\sin\left(\frac{n\pi\,x}{L}\right) \ .
\]

From $u(x,0) = f(x)$, we get
\[
f(x) = \sum_{n=1}^\infty a_n\,\sin\left(\frac{n\pi\,x}{L}\right)
\]
for $0<x<L$.  This is the Fourier sine series of $f$.  The coefficients of
this series are given by
\[
a_n = \frac{2}{L} \int_0^L f(x)
\sin\left(\frac{n\pi\,x}{L}\right) \dx{x} \  .
\]

From $\displaystyle \pdydx{u}{t}(x,0) = g(x)$, we get
\[
g(x) =\sum_{n=1}^\infty \frac{cn\pi b_n}{L}\,
\sin\left(\frac{n\pi\,x}{L}\right)
\]
for $0<x<L$.  This is the Fourier sine series of $g$.  The formula to
compute the coefficients of this Fourier series yields
\[
b_n = \frac{2}{cn\pi} \int_0^L g(x)
\cos\left(\frac{n\pi\,x}{L}\right) \dx{x} \  .
\]

To show that we effectively get a solution of the wave equation, we
should prove that the series converges as we have done for
Theorem~\ref{wave_spv_conditions} if we seek a classical solution.
This is left to the reader.
\end{egg}

\begin{egg}
Solve the wave equation
\begin{equation} \label{wave_spv_egg2_e}
\pdydxn{u}{t}{2} = c^2 \pdydxn{u}{x}{2}  \quad ,
\quad 0 < x < L \text{ and } t>0 \ ,
\end{equation}
with the boundary conditions
$\displaystyle \pdydx{u}{x}(0,t) = \pdydx{u}{x}(L,t) = 0$ for
$t>0$, and the initial conditions
$u(x,0) = f(x)$ and $\displaystyle \pdydx{u}{t}(x,0) = g(x)$
for $0<x<L$.           \label{wave_spv_egg2}

\subI{Separation of Variables}
If we substitute $u(x,t) = F(x)G(t)$ in (\ref{wave_spv_egg2_e}), we get
\[
F(x)\,\dydxn{G}{t}{2}(t) = c^2 \dydxn{F}{x}{2}(x)\, G(t) \ .
\]
After dividing both sides by $\displaystyle c^2\,F(x)G(t)$, we get
\[
\frac{1}{c^2\,G(t)} \, \dydxn{G}{t}{2}(t) =
\frac{1}{F(x)}\, \dydxn{F}{x}{2}(x)
\]
for $t>0$ and $0<x<L$.
Since the right hand side is independent of $t$ and the left hand side
is independent of $x$, we get
\[
\frac{1}{c^2\,G(t)} \, \dydxn{G}{t}{2}(t) = \frac{1}{F(x)}\,
\dydxn{F}{x}{2}(x) = k
\]
for $t>0$ and $0<x<L$, and some constant $k$.  We end up with two
ordinary differential equations.
\begin{equation} \label{wave_spv_egg2_ode}
\dydxn{F}{x}{2}(x) -k F(x) = 0 \quad \text{and}
\quad \dydxn{G}{t}{2}(t) - c^2\,k\,G(t) = 0 \ .
\end{equation}

The first ordinary differential equation in (\ref{wave_spv_egg2_ode})
satisfies two boundary conditions.   From
$\displaystyle \pdydx{u}{x}(0,t) = 0$, we get
$F'(0)G(t)=0$.  Since we assume that $G$ is non-null, we get
$F'(0)=0$.  Similarly, from $\displaystyle \pdydx{u}{x}(L,t) = 0$, we
get $F'(L)G(t)=0$.  Again, since we assume that $G$ is non-null, we
get $F'(L)=0$.  The boundary conditions for 
the first ordinary differential equation in
(\ref{wave_spv_egg2_ode}) are $F'(0)=F'(L)=0$.

\subI{Simple Functions}
We first consider the boundary value problem
\begin{equation} \label{wave_spv_egg2_bvp}
\dydxn{F}{x}{2}(x) -k F(x) = 0 \quad \text{with} \quad F'(0)=F'(\pi)=0  \ .
\end{equation}
The form of the general solution of the ordinary differential equation
above is determined by the roots of the characteristic equation
$\displaystyle \lambda^2-k=0$.

If $k>0$, the roots of the characteristic equation are $\pm \sqrt{k}$.
Since the roots are real, the solution of the ordinary differential
equation is of the form
$\displaystyle F(x) = A e^{\sqrt{k}\,x} + B e^{-\sqrt{k}\, x}$.
However, $F'(0)=0$ implies that $A\sqrt{k} - B\sqrt{k}=0$ and
$F'(L)=0$ implies
$\displaystyle A \sqrt{k}\, e^{L\sqrt{k}} - B \sqrt{k}\, e^{-L\sqrt{k}} = 0$.
The only solution of the system
\[
\begin{pmatrix}
\sqrt{k} & -\sqrt{k} \\ \sqrt{k}\, e^{L\sqrt{k}} & -\sqrt{k}\, e^{-L\sqrt{k}}
\end{pmatrix}
\begin{pmatrix}
A \\ B
\end{pmatrix}
=
\begin{pmatrix}
0 \\ 0
\end{pmatrix}
\]
is $A=B=0$ because
\[
\det
\begin{pmatrix}
\sqrt{k} & -\sqrt{k} \\ \sqrt{k}\, e^{L\sqrt{k}} & -\sqrt{k}\, e^{-L\sqrt{k}}
\end{pmatrix}
= -k \left( e^{-L\sqrt{k}} - e^{L\sqrt{k}}\right) \neq 0
\]
for $k\neq 0$.  Therefore, the null solution is the only solution
of the boundary value problem (\ref{wave_spv_egg2_bvp}) for $k>0$.

If $k=0$, the solution of the ordinary differential equation is
$F(x)=B_0 x+ A_0$.  However, $F'(0)=0$ implies that $B_0=0$.
$F'(L)=0$ does not give more information.  Thus $B_0=0$ and $A_0$ is
free.  The boundary value problem
(\ref{wave_spv_egg2_bvp}) has a non-null solution for $k=k_0 \equiv 0$
which is $F(x) = F_0(x) \equiv A_0$ for all $x\in [0,L]$, where $A_0$ can
be any constant.

If $k<0$, the roots of the characteristic equation are $\pm i \sqrt{-k}$.
Since the roots are complex, the solution of the ordinary differential
equation is of the form $\displaystyle
F(x) = A \cos\left(\sqrt{-k}\,x\right) + B \sin\left(\sqrt{-k}\,x\right)$.
However, $F'(0)=0$ implies that $B \sqrt{-k} =0$.  So $B=0$ because
$k\neq 0$.  Then $F'(L)=0$ implies that
$A \sqrt{-k} \sin\left(L\sqrt{-k}\right) = 0$.  If we take
$A=0$, we get the null solution.  We must therefore have
$\sin\left(L\sqrt{-k}\right) = 0$ because $k< 0$.  This implies
that $\displaystyle k = k_n = -\left(n\pi/L\right)^2$ for $n>0$.  The
boundary value problem (\ref{wave_spv_egg2_bvp}) has non-null
solutions only for
$\displaystyle k=k_n\equiv -\left(n\pi/L\right)^2<0$ with $n>0$, and
the solution associated to each $k_n$ is of the form
\[
F(x)=F_n(x) \equiv A_n \cos\left(\frac{n\pi x}{L}\right) \ .
\]

For $k=0$, the second ordinary differential equation in
(\ref{wave_spv_egg2_ode}) becomes $g''(t) = 0$.  This equation has the
general solution $G(t) = G_0(t) \equiv D_0 t + C_0$.

For $\displaystyle k=k_n=-\left(n\pi/L\right)^2$, the second ordinary
differential equation in (\ref{wave_spv_egg2_ode}) becomes
\[
\dydxn{G}{t}{2}(t) + \left(\frac{c n\pi}{L}\right)^2\,G(t) = 0
\]
for $n>0$.  For each $n$, this is a second order ordinary differential
equation with the characteristic equation
$\displaystyle \lambda^2 +\left(c n\pi/L\right)^2=0$.  The two
roots of this equation are the complex numbers
$\displaystyle \lambda_{\pm} = \pm (c n\pi/L)i$.
The general solution of this ordinary differential equation is therefore
\[
G(t) = G_n(t) \equiv C_n \cos\left(\frac{cn\pi t}{L}\right)
+ D_n \sin\left(\frac{cn\pi t}{L}\right) \ .
\]

We have found the simple functions
\begin{align*}
u_0(x,t) &\equiv F_0(x)G_0(t) = b_0 t + a_0
\intertext{and}
u_n(x,t) &\equiv F_n(x)G_n(t) =
\left( a_n \cos\left(\frac{cn\pi t}{L}\right)
+ b_n \sin\left(\frac{cn\pi t}{L}\right) \right)
\cos\left(\frac{n\pi x}{L}\right)
\end{align*}
for $n>0$.  The constant $a_n$ is the product of the constants $A_n$ and
$C_n$, and $b_n$ is the product of $A_n$ and $D_n$.

\subI{Initial condition}
We seek a solution of the form
\[
u(x,t) = \sum_{n=0}^\infty u_n(x,t)
= b_0 t + a_0 + \sum_{n=1}^\infty
\left( a_n \cos\left(\frac{cn\pi t}{L}\right)
+ b_n \sin\left(\frac{cn\pi t}{L}\right) \right)
\cos\left(\frac{n\pi x}{L}\right) \  .
\]

From $u(x,0) = f(x)$, we get
\[
f(x) = a_0 + \sum_{n=1}^\infty a_n\,\cos\left(\frac{n\pi x}{L}\right)
\]
for $0<x<L$.  This is the Fourier cosine series of $f$.  The coefficients of
this series are given by
\[
a_0 = \frac{1}{L} \int_0^L f(x) \dx{x} \quad \text{and} \quad
a_n = \frac{2}{L} \int_0^L f(x) \cos\left(\frac{n\pi x}{L}\right) \dx{x}
\]
for $n>0$.

From $\displaystyle \pdydx{u}{t}(x,0) = g(x)$, we get
\[
g(x) = b_0 + \sum_{n=1}^\infty \frac{cn\pi b_n}{L}\,
\cos\left(\frac{n\pi x}{L}\right)
\]
for $0<x<L$.  This is the Fourier cosine series of $g$.  The coefficients of
this series are given by
\[
b_0 = \frac{1}{L} \int_0^L g(x) \dx{x} \quad \text{and} \quad
b_n = \frac{2}{cn\pi} \int_0^L g(x)
\cos\left(\frac{n\pi x}{L}\right) \dx{x}
\]
for $n>0$.
\end{egg}

\begin{egg}
Solve the wave equation
\begin{equation} \label{wave_spv_egg3_a}
\pdydxn{u}{t}{2} = \pdydxn{u}{x}{2} \quad , \quad t > 0 \
\text{and} \ 0 < x < 5 \ ,
\end{equation}
with the boundary conditions
$u(0,t) = 0$ and $\displaystyle \pdydx{u}{x}(5,t) = 0$ for $t > 0$,
and the initial conditions
$\displaystyle u(x,0) = 2 \sin\left(\frac{7\pi x}{10}\right)$
and
$\displaystyle \pdydx{u}{t}(x,0) = 5 \sin\left(\frac{11 \pi x}{10}\right)$
for $0 < x < 5$.

\subI{Separation of Variables}
If we substitute $u(x,t) = F(x)G(t)$ in (\ref{wave_spv_egg3_a}), we get
\[
F(x)\,\dydxn{G}{t}{2}(t) = \dydxn{F}{x}{2}(x)\, G(t) \ .
\]
Thus, after dividing both sides by $F(x)G(t)$, we get
\[
\frac{1}{G(t)} \, \dydxn{G}{t}{2}(t) = \frac{1}{F(x)}\, \dydxn{F}{x}{2}(x)
\quad , \quad t>0 \ \text{and} \ 0<x<5 \ .
\]
Since the right hand side is independent of $t$ and the left hand side
is independent of $x$, we get
\[
\frac{1}{G(t)} \, \dydxn{G}{t}{2}(t) = \frac{1}{F(x)}\,
\dydxn{F}{x}{2}(x) = k
\]
for $t>0$ and $0<x<5$, and some constant $k$.  We end up with two
ordinary differential equations.
\begin{equation} \label{wave_spv_egg3_p_ode}
\dydxn{F}{x}{2}(x) -k F(x) = 0 \quad \text{and}
\quad \dydxn{G}{t}{2}(t) - k\,G(t) = 0 \ .
\end{equation}

The first ordinary differential equation in
(\ref{wave_spv_egg3_p_ode}) satisfies two boundary
conditions.   From $u(0,t) = 0$, we get $F(0)G(t)=0$.  Since we assume
that $G$ is non-null, we get $F(0)=0$.  Similarly, from
$\displaystyle \pdydx{u}{x}(5,t) = 0$, we get
$\displaystyle \dydx{F}{x}(5)G(t)=0$.
Again, since we assume that $G$ is non-null, we get
$\displaystyle \dydx{F}{x}(5)=0$.  The
boundary conditions for the first ordinary differential equation in
(\ref{wave_spv_egg3_p_ode}) are $\displaystyle F(0)=\dydx{F}{x}(5)=0$. 

\subI{Simple Functions}
We first consider the boundary value problem
\begin{equation} \label{wave_spv_egg3_p_bvp}
\dydxn{F}{x}{2}(x) -k F(x) = 0 \quad \text{with} \quad
F(0)=\dydx{F}{x}(5)=0 \ .
\end{equation}
The form of the general solution of the ordinary differential equation
above is determined by the roots of the characteristic equation
$\displaystyle \lambda^2-k=0$.

If $k>0$, the roots of the characteristic equation are $\pm \sqrt{k}$.
Since the roots are real, the solution of the ordinary differential
equation is of the form
$\displaystyle F(x) = A e^{\sqrt{k}\,x} + B e^{-\sqrt{k}\, x}$.
However, $F(0)=0$ implies that $A+B=0$ and
$\displaystyle \dydx{F}{x}(5)=0$ implies
$\displaystyle A\sqrt{k} e^{5\sqrt{k}} - B\sqrt{k} e^{-5\sqrt{k}} = 0$.
The only solution of the system
\[
\begin{pmatrix}
1 & 1 \\ \sqrt{k} e^{5\sqrt{k}} & -\sqrt{k} e^{-5\sqrt{k}}
\end{pmatrix}
\begin{pmatrix}
A \\ B
\end{pmatrix}
=
\begin{pmatrix}
0 \\ 0 
\end{pmatrix}
\]
is $A=B=0$ because
\[
\det
\begin{pmatrix}
1 & 1 \\ \sqrt{k} e^{5\sqrt{k}} & -\sqrt{k} e^{-5\sqrt{k}}
\end{pmatrix}
= -\sqrt{k} \left(e^{-5\sqrt{k}} + e^{5\sqrt{k}}\right) 
\neq 0
\]
for $k\neq 0$.  Therefore, the null solution is the only solution
of the boundary value problem (\ref{wave_spv_egg3_p_bvp}) for $k>0$.

If $k=0$, the solution of the ordinary differential equation is $F(x)=Ax+B$.
However, $F(0)=0$ implies that $B=0$ and
$\displaystyle \dydx{F}{x}(5)=0$ implies that $A = 0$.
Thus $A=B=0$ and again the null solution is the
only solution of the boundary value problem (\ref{wave_spv_egg3_p_bvp})
for $k=0$.

If $k<0$, the roots of the characteristic equation are $\pm i \sqrt{-k}$.
Since the roots are complex, the solution of the ordinary differential
equation is of the form $\displaystyle
F(x) = A \cos\left(\sqrt{-k}\,x\right) + B \sin\left(\sqrt{-k}\,x\right)$.
Hence, $F(0)=0$ implies that $A=0$ and
$\displaystyle \dydx{F}{x}(5)=0$ implies that\\
$B \sqrt{-k} \cos(5\sqrt{-k}) = 0$.  If we take
$B=0$, we get the null solution.  We must therefore have
$\cos(5\sqrt{-k}) = 0$.  This implies that
$\displaystyle 5\sqrt{-k} = (2n-1)\pi/2$ for $n >0$. Thus
$\displaystyle k = k_n \equiv -\left((2n-1)\pi/10\right)^2$
for $n>0$.
The boundary value problem (\ref{wave_spv_egg3_p_bvp}) has non-null
solutions only for these values of $k$, and the solution associated
to each $k_n$ is of the form
\[
F(x)=F_n(x) \equiv B_n \sin\left(\frac{(2n-1)\pi x}{10}\right) \  .
\]

The second ordinary differential equation in (\ref{wave_spv_egg3_p_ode}) with
$\displaystyle k = k_n = -\left((2n-1)\pi/10\right)^2$
becomes
\[
\dydxn{G}{t}{2}(t) + \left(\frac{(2n-1)\pi}{10}\right)^2\,G(t) = 0
\]
for $n>0$.  For each $n$, the general solution
of this second order ordinary differential equation is
\[
G(t) = G_n(t) \equiv C_n \cos\left(\frac{(2n-1)\pi t}{10}\right)
+ D_n \sin\left(\frac{(2n-1)\pi t}{10}\right) \ .
\]

We have found the simple functions
\[
u_n(x,t) \equiv F_n(x)G_n(t) = \left( a_n\,
\cos\left(\frac{(2n-1)\pi t}{10}\right) + b_n
\sin\left(\frac{(2n-1)\pi t}{10}\right) \right)
\sin\left(\frac{(2n-1)\pi x}{10}\right)
\]
for $n>0$.  The constant $a_n$ comes from the product of the constants
$B_n$ and $C_n$, and $b_n$ comes from the product of $B_n$ and $D_n$.

\subI{Initial condition}
We seek a solution of the form
\[
u(x,t) = \sum_{n=1}^\infty u_n(x,t)
= \sum_{n=1}^\infty \left( a_n\, \cos\left(\frac{(2n-1)\pi t}{10}\right)
 + b_n \sin\left(\frac{(2n-1)\pi t}{10}\right) \right)
\sin\left(\frac{(2n-1)\pi x}{10}\right) \ .
\]

From $\displaystyle u(x,0) = 2 \sin\left(\frac{7 \pi x}{10}\right)$, we get
\[
2 \sin\left(\frac{7 \pi x}{10}\right)
= \sum_{n=1}^\infty a_n\,\sin\left(\frac{(2n-1)\pi x}{10}\right)
\]
for $0<x<5$.  Thus
\[
a_n =
\begin{cases}
2 & \quad \text{if} \ n=4 \\
0 & \quad \text{otherwise}
\end{cases}
\]
From $\displaystyle \pdydx{u}{t}(x,0)
= 5 \sin\left(\frac{11\pi x}{10}\right)$, we get
\begin{align*}
5 \sin\left(\frac{11\pi x}{10}\right)
& = \sum_{n=1}^\infty \left(\frac{(2n-1)\pi}{10}\right) b_n\,
\sin\left(\frac{(2n-1)\pi x}{10}\right)
\end{align*}
for $0<x<5$.  Thus
\[
b_n =
\begin{cases}
\displaystyle \frac{50}{11\pi} & \quad \text{if} \ n=6 \\
0 & \quad \text{otherwise}
\end{cases}
\]

The solution is
\[
u(x,t) = 2\, \cos\left(\frac{7\pi t}{10}\right)
\sin\left(\frac{7\pi x}{10}\right)
+\frac{50}{11\pi} \, \sin
\left(\frac{11\pi t}{10}\right) \sin\left(\frac{11\pi x}{10}\right) \ .
\]
\end{egg}

\begin{egg}
Solve the wave equation
\begin{equation} \label{wave_spv_egg3_e}
\pdydxn{u}{t}{2} = 9 \pdydxn{u}{x}{2} \quad ,
\quad t > 0 \ \text{and} \ 0 < x < \pi/2 \ , 
\end{equation}
with the boundary conditions
$u(0,t) = 0$ and $\displaystyle \pdydx{u}{x}\left(\frac{\pi}{2},t\right) = 0$
for $t > 0$, and the initial conditions
$\displaystyle u(x,0) = 3 \sin\left(7 x\right) + 5 \sin\left(13 x\right)$
and
$\displaystyle \pdydx{u}{t}(x,0) = 3 \sin\left(7 x\right)$
for $0 < x < \pi/2$.

\subI{Separation of Variables}
If we substitute $u(x,t) = F(x)G(t)$ in (\ref{wave_spv_egg3_e}), we get
\[
F(x)\,\dydxn{G}{t}{2}(t) = 9\, \dydxn{F}{x}{2}(x)\, G(t) \ .
\]
After dividing both sides by $9\,F(x)G(t)$, we get
\[
\frac{1}{9\,G(t)} \, \dydxn{G}{t}{2}(t) = \frac{1}{F(x)}\, \dydxn{F}{x}{2}(x)
\]
for $t>0$ and $0<x<\pi/2$.
Since the right hand side is independent of $t$ and the left hand side
is independent of $x$, we get
\[
\frac{1}{9\,G(t)} \, \dydxn{G}{t}{2}(t) = \frac{1}{F(x)}\,
\dydxn{F}{x}{2}(x) = k
\]
for $t>0$ and $0<x<\pi/2$, and some constant $k$.  We end up with two
ordinary differential equations.
\begin{equation} \label{wave_spv_egg3_ode}
\dydxn{F}{x}{2}(x) -k F(x) = 0 \quad \text{and}
\quad \dydxn{G}{t}{2}(t) - 9\,k\,G(t) = 0 \ .
\end{equation}

The first ordinary differential equation in (\ref{wave_spv_egg3_ode})
satisfies two boundary conditions.   From $u(0,t) = 0$, we get
$F(0)G(t)=0$.  Since we assume that $G$ is non-null, we get
$F(0)=0$.  Similarly, from
$\displaystyle \pdydx{u}{x}\left(\frac{\pi}{2},t\right) = 0$, we get
$\displaystyle \dydx{F}{x}\left(\frac{\pi}{2}\right)G(t)=0$.
Again, since we assume that $G$ is non-null, we get
$\displaystyle \dydx{F}{x}\left(\frac{\pi}{2}\right)=0$.  The
boundary conditions for the first ordinary differential equation in
(\ref{wave_spv_egg3_ode}) are
$\displaystyle F(0)=\dydx{F}{x}\left(\frac{\pi}{2}\right)=0$. 

\subI{Simple Functions}
We first consider the boundary value problem
\begin{equation} \label{wave_spv_egg3_bvp}
\dydxn{F}{x}{2}(x) -k F(x) = 0 \quad \text{with} \quad
F(0)=\dydx{F}{x}\left(\frac{\pi}{2}\right)=0 \  .
\end{equation}
The form of the general solution of the ordinary differential equation
above is determined by the roots of the characteristic equation
$\displaystyle \lambda^2-k=0$.

If $k>0$, the roots of the characteristic equation are $\pm \sqrt{k}$.
Since the roots are real, the solution of the ordinary differential
equation is of the form
$\displaystyle F(x) = A e^{\sqrt{k}\,x} + B e^{-\sqrt{k}\, x}$.
However, $F(0)=0$ implies that $A+B=0$ and
$\displaystyle \dydx{F}{x}\left(\frac{\pi}{2}\right)=0$ implies
$\displaystyle A\sqrt{k} e^{\pi\sqrt{k}/2} - B\sqrt{k} e^{-\pi\sqrt{k}/2} = 0$.
The only solution of the system
\[
\begin{pmatrix}
1 & 1 \\ \sqrt{k} e^{\pi\sqrt{k}/2} & -\sqrt{k} e^{-\pi\sqrt{k}/2}
\end{pmatrix}
\begin{pmatrix}
A \\ B
\end{pmatrix}
=
\begin{pmatrix}
0 \\ 0 
\end{pmatrix}
\]
is $A=B=0$ because
\[
\det
\begin{pmatrix}
1 & 1 \\ \sqrt{k} e^{\pi\sqrt{k}/2} & -\sqrt{k} e^{-\pi\sqrt{k}/2}
\end{pmatrix}
= -\sqrt{k} \left(e^{-\pi\sqrt{k}/2} + e^{\pi\sqrt{k}/2}\right) 
\neq 0
\]
for $k\neq 0$.  Therefore, the null solution is the only solution
of the boundary value problem (\ref{wave_spv_egg3_bvp}) for $k>0$.

If $k=0$, the solution of the ordinary differential equation is $F(x)=Ax+B$.
However, $F(0)=0$ implies that $B=0$ and
$\displaystyle \dydx{F}{x}\left(\frac{\pi}{2}\right)=0$ implies
that $A = 0$.  Thus $A=B=0$ and again the null solution is the
only solution of the boundary value problem (\ref{wave_spv_egg3_bvp})
for $k=0$.

If $k<0$, the roots of the characteristic equation are $\pm i \sqrt{-k}$.
Since the roots are complex, the solution of the ordinary differential
equation is of the form $\displaystyle
F(x) = A \cos\left(\sqrt{-k}\,x\right) + B \sin\left(\sqrt{-k}\,x\right)$.
Moreover, $F(0)=0$ implies that $A=0$ and
$\displaystyle \dydx{F}{x}\left(\frac{\pi}{2}\right)=0$ implies that
$B \sqrt{-k} \cos\left(\pi\sqrt{-k}/2\right) = 0$.  If we take
$B=0$, we get the null solution.  We must therefore have
$\cos\left(\pi\sqrt{-k}/2\right) = 0$.  This implies that
$\displaystyle \pi \sqrt{-k}/2 = (2n-1)\pi/2$ for
$n\in \NN$. Thus $\displaystyle k = k_n \equiv -(2n-1)^2$
for $n>0$  The boundary value problem (\ref{wave_spv_egg3_bvp}) has
non-null solutions only for these values of $k$, and the solution
associated to each $k_n$ is of the form 
\[
F(x)=F_n(x) \equiv B_n \sin\left((2n-1)x\right) \ .
\]

The second ordinary differential equation in (\ref{wave_spv_egg3_ode}) with
$\displaystyle k = k_n = -(2n-1)^2$ becomes
\[
\dydxn{G}{t}{2}(t) + (3(2n-1))^2\,G(t) = 0
\]
for $n>0$.  For each $n$, the general solution of this second order
ordinary differential equation is
\[
G(t) = G_n(t) \equiv C_n \cos\left(3(2n-1)t\right)
+ D_n \sin\left(3(2n-1)t\right) \ .
\]

We have found the simple functions
\[
u_n(x,t) \equiv F_n(x)G_n(t) = \left( a_n\,
\cos\left(3(2n-1)t\right) + b_n
\sin\left(3(2n-1)t\right) \right) \sin\left((2n-1)x\right)
\]
for $n>0$.  The constant $a_n$ comes from the product of the constants
$B_n$ and $C_n$, and $b_n$ comes from the product of $B_n$ and $D_n$.

\subI{Initial condition}
We seek a solution of the form
\[
u(x,t) = \sum_{n=1}^\infty u_n(x,t)
= \sum_{n=1}^\infty \left( a_n\, \cos\left(3(2n-1)t\right) + b_n
\sin\left(3(2n-1)t\right) \right) \sin\left((2n-1)x\right) \ .
\]

From $u(x,0) = 3 \sin\left(7 x\right) + 5 \sin\left(13 x\right)$, we get
\[
3 \sin\left(7 x\right) + 5 \sin\left(13 x\right)
= \sum_{n=1}^\infty a_n\,\sin((2n-1)\,x)
\]
for $0<x<\pi/2$.  Thus
\[
a_n =
\begin{cases}
3 & \quad \text{if} \ n=4 \\
5 & \quad \text{if} \ n=7 \\
0 & \quad \text{otherwise}
\end{cases}
\]
From $\displaystyle \pdydx{u}{t}(x,0) = 3 \sin\left(7 x\right)$, we get
\[
3 \sin\left(7 x\right)
 = \sum_{n=1}^\infty 3(2n-1)b_n\,\sin((2n-1)\,x)
\]
for $0<x<\pi/2$.  Thus
\[
b_n =
\begin{cases}
1/7 & \quad \text{if} \ n=4 \\
0 & \quad \text{otherwise}
\end{cases}
\]

The solution is
\[
u(x,t) = \left( 3\, \cos\left(21\,t\right) + \frac{1}{7}
\sin\left(21\,t\right) \right) \sin\left(7\,x\right)
+5\, \cos\left(39\,t\right) \sin\left(13\,x\right) \ .
\]
\end{egg}

In the next example, we consider a non-homogeneous wave equation.

\begin{egg}
Solve the non-homogeneous wave equation
\begin{equation} \label{wave_spv_egg4_e}
\pdydxn{u}{t}{2} = c^2 \pdydxn{u}{x}{2} = h(x,t) \equiv 2Lx-x^2 \quad ,
\quad 0 < x < L \text{ and } t>0 \ ,
\end{equation}
with the boundary conditions
$\displaystyle \pdydx{u}{x}(0,t) = \pdydx{u}{x}(L,t) = 0$ for
$t>0$, and the initial conditions
$\displaystyle u(x,0) = x^2(3L-2x)$ and
$\displaystyle \pdydx{u}{t}(x,0) = \sin^2\left(\pi x/L\right)$ for
$0<x<L$.

The solution of the homogeneous wave equation in
Example~\ref{wave_spv_egg2} suggest to seek a solution of the form
\begin{equation} \label{wave_spv_egg4_su}
u(x,t) = T_0(t) + \sum_{n=1}^\infty T_n(t)
\cos\left(\frac{n\pi\,x}{L}\right) \ .
\end{equation}

Suppose that the Fourier cosine series of $h$ with respect to the
variable $x$ is
\begin{equation} \label{wave_spv_egg4_sh}
h(x,t) = h_0(t) +  \sum_{n=1}^\infty h_n(t) \cos\left(\frac{n\pi x}{L}\right) \ .
\end{equation}
If we substitute (\ref{wave_spv_egg4_su}) and (\ref{wave_spv_egg4_sh}) in
(\ref{wave_spv_egg4_e}), assuming that we may interchange derivatives and
infinite summations, we get
\[
T_0''(t) + \sum_{n=1}^\infty \left( T_n''(t) +
\left(\frac{c n\pi}{L}\right)^2 T_n(t) \right)
\cos\left(\frac{n\pi\,x}{L}\right) =
h_0(t) +  \sum_{n=1}^\infty h_n(t) \cos\left(\frac{n\pi x}{L}\right)
\]
for $0<x<L$.  Thus,
\[
T_0''(t)  = h_0(t) \quad \text{and} \quad
T_n''(t) + \left(\frac{c n\pi}{L}\right)^2 T_n(t) = h_n(t)
\]
for $t>0$ and $n>0$.

We first have to find the Fourier cosine series of
$\displaystyle h(x,t) = 2Lx-x^2$ for $0<x<L$.  We have
\[
a_0 = \frac{1}{L}\int_0^L \left( 2Lx-x^2 \right) \dx{x}
= \frac{1}{L} \left( Lx^2 - \frac{x^3}{3} \right)\bigg|_0^L = \frac{2L^2}{3}
\]
and, using integration by parts, we have
\begin{align*}
a_n &= \frac{2}{L} \int_0^L \left(2Lx-x^2\right)
\cos\left(\frac{n\pi x}{L}\right)\dx{x} \\
&= \left( \frac{2}{n\pi} \left(2Lx-x^2\right)
\sin\left(\frac{n\pi x}{L}\right)\right)\bigg|_0^L
- \frac{4}{n\pi} \int_0^L \left(L-x\right)
\sin\left(\frac{n\pi x}{L}\right) \dx{x} \\
&= \left( \frac{4L}{n^2\pi^2} \left(L-x\right)
\cos\left(\frac{n\pi x}{L}\right) \right)\bigg|_0^L 
+ \frac{4L}{n^2\pi^2} \int_0^L \cos\left(\frac{n\pi x}{L}\right)
\dx{x} \\
&= -\frac{4L^2}{n^2\pi^2}
+ \frac{4L^2}{n^3\pi^3} \sin\left(\frac{n\pi x}{L}\right)\bigg|_0^L
= -\frac{4L^2}{n^2\pi^2} \ .
\end{align*}
Thus
\[
h(x,t) = \frac{2L^2}{3} - \sum_{n=1}^\infty \frac{4L^2}{n^2\pi^2}
\cos\left(\frac{n\pi x}{L}\right) \ .
\]

To determine the value of $T_n$ for $n\geq 0$, we have to solve
\[
T_0''(t) = \frac{2L^2}{3} \quad \text{and} \quad
T_n''(t) + \left(\frac{c n\pi}{L}\right)^2 T_n(t) = -\frac{4L^2}{n^2\pi^2}
\]
for $t>0$ and $n>0$.  We find
\[
T_0(t) = \frac{L^2}{3}\,t^2 + B_0t + A_0 \quad \text{and} \quad
T_n(t) = A_n\cos\left(\frac{c n \pi t}{L}\right)
+ B_n \sin\left(\frac{c n \pi t}{L}\right) -
\frac{4 L^4}{c^2n^4\pi^4}
\]
for $t>0$ and $n>0$.

The solution is
\[
u(x,t) = \left(\frac{L^2}{3}\,t^2 + B_0t + A_0\right)
+ \sum_{n=1}^\infty 
\left( A_n\cos\left(\frac{c n \pi t}{L}\right)
+ B_n \sin\left(\frac{c n \pi t}{L}\right) -
\frac{4 L^4}{c^2n^4\pi^4} \right)\cos\left(\frac{n\pi x}{L}\right)
\]
for $0<x<L$ and $t>0$.

From $\displaystyle u(x,0) = x^2(3L-2x)$, we get
\[
u(x,0) = A_0
+ \sum_{n=1}^\infty 
\left( A_n - \frac{4 L^4}{c^2n^4\pi^4} \right)\cos\left(\frac{n\pi x}{L}\right)
= x^2(3L-2x) \ .
\]
This is the Fourier cosine series of $\displaystyle x^2(3L-2x)$.  Thus,
\[
A_0 = \frac{1}{L}\int_0^L x^2(3L-2x) \dx{x} =
\frac{1}{L}\left( Lx^3 - \frac{1}{2} x^4\right)\bigg|_0^L
= \frac{L^3}{2}
\]
and, using integration by parts,
\begin{align*}
A_n - \frac{4 L^4}{c^2n^4\pi^4} &= \frac{2}{L}\int_0^L x^2(3L-2x)
\cos\left(\frac{n\pi x}{L}\right) \dx{x} \\
&= \left(\frac{2}{n\pi} x^2(3L-2x)
\sin\left(\frac{n\pi x}{L}\right)\right)\bigg|_0^L
- \frac{12}{n\pi} \int_0^L \left(Lx-x^2\right)
\sin\left(\frac{n\pi x}{L}\right)\dx{x} \\
&= \left( \frac{12L}{n^2\pi^2} \left(Lx-x^2\right)
\cos\left(\frac{n\pi x}{L}\right)\right)\bigg|_0^L
- \frac{12L}{n^2\pi^2} \int_0^L \left(L-2x\right)
\cos\left(\frac{n\pi x}{L}\right) \dx{x} \\
&= - \left( \frac{12L^2}{n^3\pi^3} \left(L-2x\right)
\sin\left(\frac{n\pi x}{L}\right) \right)\bigg|_0^L
- \frac{24L^2}{n^3\pi^3} \int_0^L
\sin\left(\frac{n\pi x}{L}\right) \dx{x} \\
&= \frac{24L^3}{n^4\pi^4}
\cos\left(\frac{n\pi x}{L}\right)\bigg|_0^L 
= \frac{24L^3}{n^4\pi^4} \left( (-1)^n -1 \right) \ .
\end{align*}

From
\[
\pdydx{u}{t}(x,0) = \sin^2\left(\frac{\pi x}{L}\right)
= \frac{1}{2} - \frac{1}{2}\cos\left(\frac{2 \pi x}{L}\right) \ ,
\]
we get
\[
B_0
+ \sum_{n=1}^\infty B_n\frac{cn\pi}{L}\cos\left(\frac{n\pi x}{L}\right)
= \frac{1}{2} - \frac{1}{2}\cos\left(\frac{2 \pi x}{L}\right) \ .
\]
This is the Fourier cosine series of $\displaystyle
\frac{1}{2} - \frac{1}{2}\cos\left(\frac{2 \pi x}{L}\right)$.  By
uniqueness of the Fourier series, we have
\[
B_n =
\begin{cases}
\displaystyle \frac{1}{2} & \quad \text{if} \ n=0 \\[0.7em]
\displaystyle -\frac{1}{2} & \quad \text{if} \ n=2 \\[0.7em]
0 & \quad \text{otherwise}
\end{cases}
\]

The solution of the non-homogeneous wave equation is
\begin{align*}
u(x,t) &= \left(\frac{L^2}{3}\,t^2 + \frac{t}{2} + \frac{L^3}{2} \right)
+ \sum_{\substack{n=1\\n\neq 2}}^\infty 
\left(\frac{24L^3}{n^4\pi^4} \left( (-1)^n -1 \right)
- \frac{4 L^4}{c^2n^4\pi^4} \right)
\cos\left(\frac{n\pi x}{L}\right) \\
&\qquad  - \left(\frac{1}{2}\sin\left(\frac{2c\pi t}{L}\right)
+ \frac{L^4}{4c^2\pi^4} \right)\cos\left(\frac{2\pi x}{L}\right) \ .
\end{align*} 

We could show that $u$ and its derivatives of order up to $2$
converges absolutely and locally uniformly for $0<x<L$ and $t>0$.
\end{egg}

\begin{rmk}
Using separation of variables, we find that the solutions of the wave
equation in (\ref{wave_class_sol2}) are linear combinations of simple
functions of the form
\[
u(x,t) = (A \cos (ckt) + B \sin (ckt))(C \cos (kx) + D \sin(kx)) \ .
\]
\end{rmk}

\section{Exercises}

Suggested exercises:

\begin{itemize}
\item In \cite{J}: all the numbers in Section 2.4.
\item In \cite{McO}: all the numbers in Sections 3.1.
\item In \cite{PinRub}: all the numbers in Sections 4.6; numbers 6.20,
  6.24 and 6.26 in Section 6.7. 
\item In \cite{Str}: all numbers in Sections 2.1, 3.2 and 3.4;
numbers 1, 2, 4 and 5 in Section 4.1; number 2 in Section 4.2;
number 13 in Section 4.3; numbers 4 to 6, 9 and 13 in Section 5.6. 
\end{itemize}


%%% Local Variables: 
%%% mode: latex
%%% TeX-master: "notes"
%%% End: 
