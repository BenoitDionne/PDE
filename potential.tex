\section{Layer Potentials} \label{potential_sect}

We follows closely the content of \cite{FoPDE} on
the subject of layer potentials.

In this chapter, $\Omega$ is a bounded open subset of $\displaystyle \RR^n$.  We
assume that $\displaystyle \Omega = \bigcup_{i=1}^{m} \Omega_i$, where the
$\Omega_i$ are the connected components of $\Omega$.  In particular,
$\Omega_i \cap \Omega_j = \emptyset$ for $i\neq j$.  Moreover, we
assume that $\displaystyle \Omega^\prime = \RR^n \setminus \overline{\Omega}
= \bigcup_{i=0}^{m^{\prime}} \Omega_i^{\prime}$,
where the $\displaystyle \Omega_i^{\prime}$ are the connected components of
$\displaystyle \Omega^{\prime}$ with
$\displaystyle \Omega_0^{\prime}$ being the unbounded component.
We assume that $\partial \Omega$ is $\displaystyle C^2$ and observe that
$\displaystyle \partial \Omega = \partial \Omega^{\prime}$.

As usual, let $\nu(\VEC{x})$ denote the outward unit normal to
$\partial \Omega$ at $\VEC{x} \in \partial \Omega$.  The
{\bfseries interior normal derivative}\index{Interior Normal Derivative} of
$\displaystyle u \in C^1(\Omega) \cap C(\overline{\Omega})$ at
$\VEC{x} \in \partial \Omega$ is
\[
\pdydx{u}{\nu^-}(\VEC{x})
= \lim_{t\rightarrow 0^-} \frac{1}{t}
\left( u(\VEC{x} + t \nu(\VEC{x})) - u(\VEC{x})\right)
\]
for $\VEC{x} \in \partial \Omega$ if the limit exists.  Similarly, the
{\bfseries exterior normal derivative}\index{Exterior Normal Derivative} of
$\displaystyle u \in C^1(\Omega^{\prime}) \cap C(\overline{\Omega^{\prime}})$ at
$\displaystyle \VEC{x} \in \partial \Omega^{\prime}$ is
\[
\pdydx{u}{\nu^+}(\VEC{x})
= \lim_{t\rightarrow 0^+} \frac{1}{t}
\left( u(\VEC{x} + t \nu(\VEC{x})) - u(\VEC{x}) \right)
\]
for $\VEC{x} \in \partial \Omega$ if the limit exists.

The following spaces of functions will be used for the studies of
Neumann problems. $C_{\nu}(\Omega)$ is the space of functions
$\displaystyle u \in C^1(\Omega) \cap C(\overline{\Omega})$ such that the
interior normal derivative exists at all points
$\VEC{x} \in \partial \Omega$ and
\begin{equation} \label{pot_unif_ddn}
\max_{\VEC{x}\in \partial \Omega}
\left| \graD u(\VEC{x} + t \nu(\VEC{x})) \cdot \nu(\VEC{x})
- \pdydx{u}{\nu^-}(\VEC{x}) \right| \rightarrow 0 \quad
\text{as} \quad t \rightarrow 0^- \ .
\end{equation}
Similarly, $\displaystyle C_{\nu}(\Omega^{\prime})$ is the space
of functions
$\displaystyle u \in C^1(\Omega^{\prime}) \cap C(\overline{\Omega^{\prime}})$
such that the exterior normal derivative exists at all points
$\displaystyle \VEC{x} \in \partial \Omega^{\prime}$ and
\begin{equation} \label{pot_ddn_unif}
\max_{\VEC{x}\in \partial \Omega^{\prime}}
\left| \graD u(\VEC{x} + t \nu(\VEC{x})) \cdot \nu(\VEC{x})
- \pdydx{u}{\nu^+}(\VEC{x}) \right| \rightarrow 0 \quad
\text{as} \quad t \rightarrow 0^+ \ .
\end{equation}

The main goal of this chapter is to solve the following four problems.
\begin{description}
\item[D1 - Interior Dirichlet Problem:] Given
$f\in C(\partial \Omega)$, find $u \in C(\overline{\Omega})$ 
such that $\Delta u = 0$ in $\Omega$ (in the sense of distributions)
and $\displaystyle u\big|_{\partial \Omega} = f$.
\item[D2 - Exterior Dirichlet Problem:] Given
$\displaystyle f\in C(\partial \Omega^{\prime})$, find
$\displaystyle u \in C(\overline{\Omega^{\prime}})$
such that $\Delta u = 0$ in
$\displaystyle \Omega^{\prime} \cup \{\infty\}$ (in the sense of
distributions) and $\displaystyle u\big|_{\partial \Omega} = f$.
\item[N1 - Interior Neumann Problem:] Given
$f\in C(\partial \Omega)$, find $u \in C_{\nu}(\Omega)$ 
such that $\Delta u = 0$ in $\Omega$ (in the sense of distributions)
and $\displaystyle \pdydx{u}{\nu^-} = f$ on $\partial \Omega$.
\item[N2 - Exterior Dirichlet Problem:] Given
$\displaystyle f\in C(\partial \Omega^{\prime})$, find
$\displaystyle u \in C_{\nu}(\Omega^{\prime})$
such that $\Delta u = 0$ in
$\displaystyle \Omega^{\prime} \cup \{\infty\}$ (in the sense of
distributions) and $\displaystyle \pdydx{u}{\nu^+} = f$ on
$\partial \Omega$.
\end{description}

We need to explain what it means for a function to be harmonic at
infinity in (D2) and (N2).

Suppose that $g: V\rightarrow W$ is a $\displaystyle C^\infty$ diffeomorphism
between two open subsets $V$ and $W$ of $\displaystyle \RR^n$.
Let $\displaystyle G(\VEC{x}) = \big(\diff g(\VEC{x})\big)
\big(\diff g(\VEC{x})\big)^\top$ for
$\VEC{x} \in V$.  The components of $G(\VEC{x})$ are
\[
G_{i,j}(\VEC{x}) = \sum_{k=1}^n \pdydx{g_i}{x_k}(\VEC{x})\,
\pdydx{g_j}{x_k}(\VEC{x}) \quad , \quad
1 \leq i,j \leq n \ .
\]
Let $h = g^{-1}$ and
$\displaystyle H(\VEC{y}) = \big(\diff h(\VEC{y})\big)^\top
\big(\diff h(\VEC{y})\big)$ for $\VEC{y} \in W$.  The components of
$H(\VEC{y})$ are
\[
H_{i,j}(\VEC{y}) = \sum_{k=1}^n \pdydx{h_k}{y_i}(\VEC{y})\,
\pdydx{h_k}{y_j}(\VEC{y}) \quad , \quad
1 \leq i,j \leq n \ .
\]
We have that
$\displaystyle G(h(\VEC{y})) = (H(\VEC{y}))^{-1}$ for all $\VEC{y} \in W$
because
\[
\Id = \diff \big( g(h(\VEC{y})) \big) = \diff g(h(\VEC{y}))
\,\big(\diff h(\VEC{y}) \big)
\]
for all $\VEC{y} \in W$ implies that
$\displaystyle \diff g(h(\VEC{y})) = (\diff h(\VEC{y}))^{-1}$
for all $\VEC{y} \in W$.  Thus,
\[
G(h(\VEC{y})) = \big(\diff g(h(\VEC{y}))\big)
\big(\diff g(h(\VEC{y}))\big)^\top
= \big(\diff h(\VEC{y})\big)^{-1}
\left( \big(\diff h(\VEC{y}) \big)^{-1}\right)^\top
= (H(\VEC{y}))^{-1}
\]
for all $\VEC{y} \in W$

Let $d_h(\VEC{y}) = \left|\det \big(\diff h(\VEC{y})\big)\right|$
for $\VEC{y} \in W$.  We have that
\[
d_h(g(\VEC{x})) = \left|\det \big(\diff h(g(\VEC{x}))\big)\right|
= \left|\det \big(\diff g(\VEC{x})\big)^{-1} \right|
=\left|\det \big(\diff g(\VEC{x})\big) \right|^{-1}
\]
for all $\VEC{x} \in V$.

Suppose that $\displaystyle u\in C^2(V)$ and let
$\breve{u}(\VEC{y}) = u(h(\VEC{y}))$ for all $\VEC{y} \in W$.
We have the following result.

\begin{prop}
Suppose that $u$, $g$, $\displaystyle h = g^{-1}$ and $G$ are as
defined above.  Then
\begin{equation} \label{pot_chain_rule}
\Delta u (\VEC{x}) = \left(
\frac{1}{d_h(\VEC{y})} \sum_{i,j=1}^n
\pdfdx{ \left( d_h(\VEC{y}) G_{i,j}(h(\VEC{y}))
\pdydx{\breve{u}}{y_i}(\VEC{y})\right)}{y_j} \right)\bigg|_{\VEC{y} = g(\VEC{x})}
\end{equation}
in the sense of distributions for $\VEC{x} \in V$.
\end{prop}

\begin{proof}
Given $\phi \in \DD(V)$, let $\breve{\phi}(\VEC{y}) = \phi(h(\VEC{y}))$
for $\VEC{y} \in W$.  We have
\begin{align*}
&\int_V (\Delta u) \phi \dx{\VEC{x}}
= \sum_{k=1}^n \int_V \pdydxn{u}{x_k}{2}\,\phi \dx{\VEC{x}}
= -\sum_{k=1}^n \int_V \pdydx{u}{x_k}\,\pdydx{\phi}{x_k} \dx{\VEC{x}} \\
&\quad = -\sum_{k=1}^n \int_W \left( \sum_{i=1}^n \pdydx{\breve{u}}{y_i}(\VEC{y})
\pdydx{g_i}{x_k}(h(\VEC{y})) \right)\,
\left( \sum_{j=1}^n \pdydx{\breve{\phi}}{y_j}(\VEC{y})
\pdydx{g_j}{x_k}(h(\VEC{y})) \right) d_h(\VEC{y}) \dx{\VEC{y}} \\
&\quad = - \sum_{j,i=1}^n \int_W \pdydx{\breve{u}}{y_i}(\VEC{y})\,
\pdydx{\breve{\phi}}{y_j}(\VEC{y})
\left( \sum_{k=1}^n \pdydx{g_i}{x_k}(h(\VEC{y}))\,\pdydx{g_j}{x_k}(h(\VEC{y}))
\right) d_h(\VEC{y}) \dx{\VEC{y}} \\
&\quad =- \sum_{j,i=1}^n \int_W \pdydx{\breve{u}}{y_i}(\VEC{y})
\, \pdydx{\breve{\phi}}{y_j}(\VEC{y})
\, G_{i,j}(h(\VEC{y}))\, d_h(\VEC{y}) \dx{\VEC{y}}\\
&=\quad  - \sum_{j,i=1}^n \int_W \left( \pdydx{\breve{u}}{y_i}(\VEC{y})
\, G_{i,j}(h(\VEC{y}))\, d_h(\VEC{y}) \right)
\pdydx{\breve{\phi}}{y_j}(\VEC{y}) \dx{\VEC{y}} \\ 
&\quad = \sum_{j,i=1}^n \int_W \pdfdx{\left( \pdydx{\breve{u}}{y_i}(\VEC{y})
\, G_{i,j}(h(\VEC{y})) d_h(\VEC{y}) \right)}{y_j} \breve{\phi}(\VEC{y})
\dx{\VEC{y}} \\
&\quad = \int_V \left( \sum_{j,i=1}^n \pdfdx{\left(
\pdydx{\breve{u}}{y_i}(\VEC{y}) \, G_{i,j}(h(\VEC{y})
d_h(\VEC{y}) \right)}{y_j} \right)\bigg|_{\VEC{y}=g(\VEC{x})}
\phi(\VEC{x})\, |\det (\diff g(\VEC{x}))| \dx{\VEC{x}} \\
&\quad = \int_V \left(\frac{1}{d_h(\VEC{y})}  \sum_{j,i=1}^n
\pdfdx{\left( \pdydx{\breve{u}}{y_i}(\VEC{y})
\, G_{i,j}(h(\VEC{y})) d_h(\VEC{y}) \right)}{y_j}
\right)\bigg|_{\VEC{y}=g(\VEC{x})} \, \phi(\VEC{x}) \dx{\VEC{x}} \ .  \qedhere
\end{align*}
\end{proof}

Consider the inversion with respect to the unit ball in $\displaystyle \RR^n$,
\begin{align*}
T:\RR^n \setminus \{\VEC{0}\} & \rightarrow \RR^n \setminus \{\VEC{0}\} \\
\VEC{x} & \mapsto \frac{1}{\|\VEC{x}\|_2^2} \, \VEC{x}
\end{align*}
We have that $\displaystyle T=T^{-1}$.  Without loss of generality, we
may assume that $\VEC{0} \in \Omega$.  The
{\bfseries Kelvin transform}\index{Kelvin Transform} of a
function $\displaystyle u\in C(\Omega^{\prime})$, is the function
\begin{align*}
\tilde{u}: T(\Omega^{\prime}) & \rightarrow \RR \\
\VEC{y} & \mapsto \|\VEC{y}\|_2^{2-n} u(T(\VEC{y}))
\end{align*}
The function $T$ is well defined on $\displaystyle T(\Omega^{\prime})$ because
$\displaystyle \VEC{0} \not\in \Omega^{\prime}$.

Let $h(\VEC{y}) = T(\VEC{y})$ for all
$\displaystyle \VEC{y} \in T(\Omega^{\prime})$.
So, the set $W$ in the previous discussion is now
$\displaystyle T(\Omega^{\prime})$ and
the set $V$ is now $\displaystyle \Omega^{\prime}$.  As before,
$\breve{u}(\VEC{y}) = u(h(\VEC{y})) = u(T(\VEC{u}))$.

We have that
\[
\pdydx{h_k}{y_i} = \pdfdx{\left( y_k \left( \sum_{j=1}^n y_j^2
\right)^{-1} \right)}{y_i}
= -\frac{2 y_k y_i}{\|\VEC{y}\|_2^4} + \frac{\delta_{i,k}}{\|\VEC{y}\|_2^2}
\]
for $1 \leq i, k \leq n$.  Hence,
\begin{align*}
H_{i,j}(\VEC{y}) &= \sum_{k=1}^n \left( -\frac{2 y_i y_k}{\|\VEC{y}\|_2^4} +
  \frac{\delta_{i,k}}{\|\VEC{y}\|_2^2} \right)
\left( -\frac{2 y_j y_k}{\|\VEC{y}\|_2^4} +
  \frac{\delta_{j,k}}{\|\VEC{y}\|_2^2} \right) \\
&= \frac{4 y_iy_j}{\|\VEC{y}\|_2^8} \sum_{k=1}^n y_k^2
- \sum_{k=1}^n \frac{2 y_i y_k \delta_{j,k}}{\|\VEC{y}\|_2^6}
- \sum_{k=1}^n \frac{2 y_j y_k \delta_{i,k}}{\|\VEC{y}\|_2^6}
+ \sum_{k=1}^n \frac{ \delta_{i,k}\delta_{j,k}}{\|\VEC{y}\|_2^4} \\
&= \frac{4 y_iy_j}{\|\VEC{y}\|_2^6}
- \frac{2 y_i y_j}{\|\VEC{y}\|_2^6}
- \frac{2 y_j y_i}{\|\VEC{y}\|_2^6}
+ \frac{ \delta_{i,j}}{\|\VEC{y}\|_2^4}
= \frac{ \delta_{i,j}}{\|\VEC{y}\|_2^4}
\end{align*}
for $1 \leq i,j \leq n$.  It follows that
\[
d_h(\VEC{y}) = \left| \det (\diff h)(\VEC{y}) \right|
= \left| \det H(\VEC{y}) \right|^{1/2}
= \frac{1}{\|\VEC{y}\|_2^{2n}} \ .
\]
We also have that
$\displaystyle G_{i,j}(h(\VEC{y})) = \delta_{i,j} \|\VEC{y}\|_2^4$
because $\displaystyle G(h(\VEC{y})) = (H(\VEC{y}))^{-1}$ for all
$\displaystyle \VEC{y} \in T(\Omega^{\prime})$.

If follows from (\ref{pot_chain_rule}) with $h=T$ and
$\displaystyle u\in C^2(\Omega^{\prime})$ that
\begin{align*}
\Delta u (\VEC{x})
&= \|\VEC{y}\|_2^{2n}  \sum_{i,j=1}^n
\pdfdx{ \left( \frac{ \delta_{i,j}}{\|\VEC{y}\|_2^{2n-4}}
\, \pdydx{\breve{u}}{y_i}(\VEC{y})\right)}{y_j}
= \|\VEC{y}\|_2^{2n}  \sum_{i=1}^n
\pdfdx{ \left( \frac{1}{\|\VEC{y}\|_2^{2n-4}}
\, \pdydx{\breve{u}}{y_i}(\VEC{y})\right)}{y_i} \\
&= \|\VEC{y}\|_2^{2n}  \sum_{i=1}^n
\left( (2-n) \frac{2y_i}{\|\VEC{y}\|_2^{2n-2}}\,
\pdydx{\breve{u}}{y_i}(\VEC{y})
+ \frac{1}{\|\VEC{y}\|_2^{2n-4}} \pdydxn{\breve{u}}{y_i}{2} (\VEC{y})
\right) \\
&= \|\VEC{y}\|_2^{n+2}  \sum_{i=1}^n
\left( (2-n) \frac{2y_i}{\|\VEC{y}\|_2^n}\,
\pdydx{\breve{u}}{y_i}(\VEC{y})
+ \frac{1}{\|\VEC{y}\|_2^{n-2}} \pdydxn{\breve{u}}{y_i}{2} (\VEC{y})
\right) \\
&= \|\VEC{y}\|_2^{n+2}  \sum_{i=1}^n
\left( \pdfdxn{ \left( \frac{1}{\|\VEC{y}\|_2^{n-2}} \right)}{y_i}{2}
  \, \breve{u}(\VEC{y})
+ 2\pdfdx{ \left( \frac{1}{\|\VEC{y}\|_2^{n-2}} \right)}{y_i}
\pdydx{\breve{u}}{y_i}(\VEC{y})
+ \frac{1}{\|\VEC{y}\|_2^{n-2}} \pdydxn{\breve{u}}{y_i}{2} (\VEC{y})
\right) \\
&= \|\VEC{y}\|_2^{n+2}  \sum_{i=1}^n
\pdfdxn{ \left( \frac{1}{\|\VEC{y}\|_2^{n-2}}
\, \breve{u}(\VEC{y}) \right)}{y_i}{2}
\end{align*}
for $\VEC{y} = T(\VEC{x})$, we here we have used
$\displaystyle \Delta \|\VEC{y}\|_2^{2-n} = 0$ for
$\VEC{y} \neq \VEC{0}$ for the
second to last equality.

We have proved that
\[
\Delta u (\VEC{x}) =
\bigg( \|\VEC{y}\|^{n+2} \Delta_{\VEC{y}} \tilde{u}(\VEC{y}) \bigg)
\bigg|_{\VEC{y} = T(\VEC{x})}
\]
for $\VEC{x} \in \Omega^{\prime}$.
The following result follows from this relation.

\begin{prop} \label{pot_u_ut_harm}
If $\displaystyle u\in C^2(\Omega^{\prime})$, then $u$ is harmonic on
$\displaystyle \Omega^{\prime}$
if and only if $\tilde{u}$ is harmonic on $\displaystyle T(\Omega^{\prime})$.
\end{prop}

\begin{defn}
We say that $\displaystyle u\in C(\Omega^{\prime})$ is harmonic at
infinity if $\tilde{u}$ has a removable singularity at $\VEC{0}$.
Namely, $\tilde{u}$ can be extended to $\VEC{x}=\VEC{0}$ in such a way
that $\tilde{u}$ is harmonic on a neighbourhood of $\VEC{0}$.
\end{defn}

We will need conditions later to determine when a function is harmonic
at infinity.  To obtain this conditions, we first need the following
lemma.

\begin{lemma} \label{lem_PotInftyU}
Suppose that $\displaystyle U \subset \RR^n$ is an open set and that
$\displaystyle \VEC{p} \in \RR^n$.
If $u:U \to \RR$ is an harmonic function on $U \setminus \{\VEC{p}\}$
such that
$\displaystyle \left|u\left(\VEC{x}\right)\right|
= o\left(\|\VEC{x} - \VEC{p}\|^{2-n}\right)$ as
$\|\VEC{x} - \VEC{p}\| \to 0$ when $n>2$ or
$\displaystyle \left|u\left(\VEC{x}\right)\right|
= o\left(\ln\left(\left\|\VEC{x} - \VEC{p}\right\|^{-1}\right)\right)$
as $\|\VEC{x} - \VEC{p}\| \to 0$ when $n=2$, then $u$ has a
removable singularity at $\VEC{p}$; namely, we may define $u$ at
$\VEC{p}$ such that $u$ is harmonic on $U$.
\end{lemma}

\begin{proof}
Since $U$ is open, there exists $R>0$ such that
$\overline{B_R(\VEC{p})} \subset U$.
Let $\displaystyle V = R^{-1} \left( U - \{\VEC{p}\} \right)$.
The lemma will be proved if we prove that
$\displaystyle v: V  \to \RR$ defined by
$\displaystyle v(\VEC{x}) = u\left(R(\VEC{x} + \VEC{p})\right)$
has a removable singularity at $\VEC{0}$.  We may therefore assume
that $\displaystyle \overline{B_1(\VEC{0})} \subset U$.

Since $v$ is continuous on $\partial B_1(\VEC{0})$, it follows
from Theorem~\ref{laplace_exist_ball} that there exists a continuous
function $w: \overline{B_1(\VEC{0})} \to \RR$ such that
$w$ is harmonic on $B_1(\VEC{0})$ and $w(\VEC{x}) = v(\VEC{x})$
for $\VEC{x} \in \partial B_1(\VEC{0})$.

We now prove that $v=w$ on
$\overline{B_1(\VEC{0})}\setminus\{\VEC{0}\}$.  We will therefore be
able to remove the singularity at the origin by setting
$v(\VEC{0}) = w(\VEC{0})$. 

Given $\epsilon >0$ and $0<\delta<1$, let
$f: \overline{B_1(\VEC{0})} \setminus B_\delta(\VEC{0}) \to \RR$
be the function defined by
\[
f(\VEC{x}) =  
\begin{cases}
v(\VEC{x}) - w(\VEC{x}) - \epsilon \left( \|\VEC{x}\|^{2-n} - 1 \right)
& \quad \text{if} \ n > 2 \\
v(\VEC{x}) - w(\VEC{x}) - \epsilon \ln(\|\VEC{x}\|^{-1})
& \quad \text{if} \ n = 2 \\
\end{cases}
\]
We have that $f$ is continuous on
$\displaystyle \overline{B_1(\VEC{0})} \setminus B_\delta(\VEC{0})$
and harmonic on
$\displaystyle B_1(\VEC{0}) \setminus \overline{B_\delta(\VEC{0})}$
because $v$, $w$, $\displaystyle \|\VEC{x}\|^{2-n} - 1$ when $n>2$,
and $\displaystyle \ln\left(\|\VEC{x}\|)^{-1}\right)$ when $n=2$ are
continuous on
$\displaystyle \overline{B_1(\VEC{0})} \setminus B_\delta(\VEC{0})$
and harmonic on
$\displaystyle B_1(\VEC{0}) \setminus \overline{B_\delta(\VEC{0})}$.
Corollary~\ref{laplace_sharm} is used to justify the previous
statement.

We have that $f(\VEC{x}) = 0$ for $\VEC{x} \in \partial B_1(\VEC{0})$.
Let $\displaystyle m = \min_{\VEC{x} \in \overline{B_1(\VEC{0})}} |w(\VEC{x})|$.
Such a minimum exists because $w$ is continuous on the compact set
$\overline{B_1(\VEC{0})}$.  By taking $\delta$ small enough, it
follows from the assumption of the lemma that
\[
f(\VEC{x}) \leq
\begin{cases}
\displaystyle \underbrace{\left(
\frac{v(\VEC{x})}{\|\VEC{x}\|^{2-n}} - \epsilon \right)}_{<0}
\underbrace{\|\VEC{x}\|^{2-n}}_{\to +\infty} - m + \epsilon < 0 \quad
& \ \text{if}\ n > 2 \\[2.5em]
\displaystyle \underbrace{\left(
\frac{v(\VEC{x})}{\ln\left(\|\VEC{x}\|^{-1}\right)} - \epsilon \right)}_{<0}
\underbrace{\ln\left(\|\VEC{x}\|^{-1}\right)}_{\to +\infty} - m < 0
\quad & \ \text{if}\ n = 2
\end{cases}
\]
for all $\VEC{x} \in B_\delta(\VEC{0})$.
Thus $f(\VEC{x}) \leq 0$ on $\partial B_\delta(\VEC{0})$.

It follows from the maximum principle, Theorem~\ref{laplace_HMP},
applied to $f$ on
$\displaystyle \overline{B_1(\VEC{0})} \setminus B_\delta(\VEC{0})$
that $f(\VEC{x}) \leq 0$ for
$\displaystyle \VEC{x} \in \overline{B_1(\VEC{0})} \setminus B_\delta(\VEC{0})$.
Since this is true for all $\delta$ small enough, we get that
$f(\VEC{x}) \leq 0$ for
$\displaystyle \VEC{x} \in \overline{B_1(\VEC{0})} \setminus \{\VEC{0}\}$.
Letting $\epsilon$ converges to $0$, we get that
$\displaystyle v(\VEC{x}) - w(\VEC{x}) \leq 0$ for
$\displaystyle \VEC{x} \in \overline{B_1(\VEC{0})} \setminus \{\VEC{0}\}$.

Proceeding similarly with $-f$ instead of $f$, we get that
$\displaystyle w(\VEC{x}) - v(\VEC{x}) \geq 0$ for
$\displaystyle \VEC{x} \in \overline{B_1(\VEC{0})} \setminus \{\VEC{0}\}$.
Completing the proof that
$\displaystyle w(\VEC{x}) - v(\VEC{x}) = 0$ for
$\displaystyle \VEC{x} \in \overline{B_1(\VEC{0})} \setminus \{\VEC{0}\}$.
\end{proof}

\begin{prop} \label{pot_infty_u}
Suppose that $\displaystyle u\in C^2(\Omega^{\prime})$ is harmonic on
$\displaystyle \Omega^{\prime}$.
The following statement are equivalent.
\begin{enumerate}
\item $u$ is harmonic at infinity.
\item $\displaystyle u(\VEC{x}) = O(\|\VEC{x}\|^{2-n})$ as
$\|\VEC{x}\| \rightarrow \infty$.
\item $|u(\VEC{x})| \to 0 $ as $\|\VEC{x}\| \rightarrow \infty$
when $n>2$, or
$|u(\VEC{x})| = o(\ln(\|\VEC{x}\|))$ as $\|\VEC{x}\| \rightarrow \infty$
when $n=2$.
\end{enumerate}
\end{prop}

\begin{proof}
\stage{1$\mathbf{\Rightarrow}$2}
If $u$ is harmonic at infinity, then the Kelvin transform
$\tilde{u}$ has a removable singularity at the origin.  Thus, there
exist $\delta>0$ such that
$\displaystyle B_{\delta}(\VEC{0}) \subset T\left(\Omega^{\prime}\right)$
and a harmonic function $w: B_{\delta}(\VEC{0}) \to \RR$
such that 
\[
w(\VEC{x}) = \tilde{u}\left(\VEC{x}\right)
= \|\VEC{x}\|_2^{2-n} u(T(\VEC{x}))
\]
for all $\VEC{x} \in B_{\delta}(\VEC{0}) \setminus \{\VEC{0}\}$.
Let
$\displaystyle M = \max_{\VEC{x} \in \overline{B_{\delta/2}(\VEC{0})}}
|w(\VEC{x})|$.

We get that
\[
u(\VEC{x}) = \|\VEC{x}\|_2^{2-n} w\left(T(\VEC{x})\right)
\]
for $\displaystyle \VEC{x} \in T\left(B_{\delta}(\VEC{0})\right)
= \left\{\VEC{x} : \| \VEC{x} \| > 1/\delta \right\} \subset \Omega^{\prime}$.
Hence, $\displaystyle | u(\VEC{x}) | \leq M \|\VEC{x}\|^{2-n}$ for
$\|\VEC{x}\| > 2/\delta$.

\stage{2$\mathbf{\Rightarrow}$3}
If $\displaystyle u(\VEC{x}) = O(\|\VEC{x}\|^{2-n})$ as
$\|\VEC{x}\| \rightarrow \infty$, then there exist $C, R >0$ such that
$\displaystyle |u(\VEC{x})| \leq C \|\VEC{x}\|^{2-n}$ for $\|\VEC{x}\| > R$.
If $n>0$, then it automatically follows that
$|u(\VEC{x})| \to 0$ as $\|\VEC{x}\| \to \infty$.
If $n=2$, then $\displaystyle |u(\VEC{x})| \leq C $ for $\|\VEC{x}\| > R$.
Hence,
\[
  \left| \frac{u(\VEC{x})}{\ln\left(\|\VEC{x}\|\right)} \right|
  \leq \frac{C}{\ln\left(\|\VEC{x}\|\right)} \to 0
\]
as $\|\VEC{x}\|\to \infty$.

\stage{3$\mathbf{\Rightarrow}$1}
It follows from the definition of the Kelvin transform $\tilde{u}$ of
$u$ that
\[
\left| \frac{\tilde{u}\left(\VEC{x}\right)}{\|\VEC{x}\|_2^{2-n}} \right|
= \left| u(T(\VEC{x})) \right|
= \left| u\left( \|\VEC{x}\|^{-2}\VEC{x}\right) \right| \to 0
\]
as $\|\VEC{x}\| \to 0$ when $n>2$, and
\[
\left| \frac{\tilde{u}\left(\VEC{x}\right)}{\ln\left(\|\VEC{x}\|^{-1}\right)}
\right|
= \left| \frac{u(T(\VEC{x}))}{\ln\left(\|\VEC{x}\|^{-1}\right)} \right|
= \left| \frac{u\left( \|\VEC{x}\|^{-2}\VEC{x}\right)}
{\ln\left(\|\VEC{x}\|^{-1}\right)} \right| \to 0
\]
as $\|\VEC{x}\| \to 0$ when $n=2$ by assumption.
Hence, according to Lemma~\ref{lem_PotInftyU}, $\tilde{u}$ has a
removable singularity at the origin in both cases.
\end{proof}

To prove the next proposition, we will use some of the results about the
homogeneous polynomials of degree $k$ in $\RR^n$ presented in
Section~\ref{SectHramPoly}.

\begin{prop} \label{pot_infty_ddu}
Suppose that $\displaystyle u\in C^2(\Omega^{\prime})$ is harmonic on
$\displaystyle \Omega^{\prime} \cup \{\infty\}$.  Let
$\nu(\VEC{x})$ be the outward
unit normal to $\displaystyle \partial B_r(\VEC{0}) \subset \Omega^{\prime}$ at
$\VEC{x} \in \partial B_r(\VEC{0})$.
If $n>2$, then
$\displaystyle \left| \pdydx{u}{\nu}(\VEC{x})\right| =
O(\|\VEC{x}\|^{1-n})$ as $\|\VEC{x}\|\rightarrow \infty$.  If $n=2$, then
$\displaystyle \left| \pdydx{u}{\nu}(\VEC{x})\right| =
O(\|\VEC{x}\|^{-2})$ as $\|\VEC{x}\|\rightarrow \infty$.
\end{prop}

\begin{proof}
We may assume that there exists $0<R<1$ such that
$\displaystyle \left\{ \VEC{x} : \|\VEC{x}\| > R \right\} \subset
\Omega^{\prime}$.
If it is not so, choose $q > 0$ such that 
$\displaystyle \left\{ \VEC{x} : \|\VEC{x}\| > q/2 \right\} \subset
\Omega^{\prime}$ and set $\displaystyle W = q^{-1} \Omega^{\prime}$.
The function $w: W \to \RR$ defined by
$w(\VEC{x}) = u(q \VEC{x})$ for $\VEC{x} \in W$
is of class $\displaystyle C^2$ and is harmonic on $W \cup \{\infty\}$.
Since $\displaystyle \left\{ \VEC{x} : \|\VEC{x}\| > 1/2 \right\} \subset W$,
we may choose $0<R<1$ such that
$\displaystyle \left\{ \VEC{x} : \|\VEC{x}\| > R \right\} \subset W$.
It is then enough to prove the proposition for $w: W \to \RR$.

For the rest of this proof, we will refer to the notation introduced
in Section~\ref{SectHramPoly}.

We consider the Kelvin function $\tilde{u}$ of $u$.  Since $u$ is
harmonic at infinity, we may assume that
$\tilde{u}$ is harmonic on
$\displaystyle \overline{B_1(\VEC{0})} \subset T\left(B_R(\VEC{0})\right)$.
Let $S = \partial B_1(\VEC{0})$ and consider the functions $f: S \to \RR$
and $\displaystyle \phi_k \in H_k\big|_S$ defined by
$f(\VEC{x}) = \tilde{u}(\VEC{x})$ and
$\phi_k(\VEC{x}) = \ps{f}{p_{\VEC{x}}^{[k]}}$ for all $\VEC{x} \in S$
respectively.  Recall that $\displaystyle p_{\VEC{x}}^{[k]}$ is the
homogeneous polynomial of degree $k$ defined in Section~\ref{SectHramPoly}.
It follows from Corollary~\ref{L2AbsConvUrxCor} that
\[
\tilde{u}(\VEC{x}) = \sum_{k=0}^\infty \|\VEC{x}\|^k
\phi_k\left(\|\VEC{x}\|^{-1}\VEC{x}\right)
\]
for $\VEC{x} \in B_1(\VEC{0})$.  We also have that the
series converges absolutely and uniformly on $B_R(\VEC{0})$ for every
$0 \leq R <1$.   We get that
\[
u(\VEC{x}) = \|T(\VEC{x})\|^{n-2} \tilde{u}(T(\VEC{x}))  
= \sum_{k=0}^\infty \|\VEC{x}\|^{2-n-k}
\phi_k\left(\|\VEC{x}\|^{-1} \VEC{x}\right)
\]
for $\|\VEC{x}\| > 1$.  If
$\VEC{x} = r \VEC{y}$ with $r > 1$ and $\|\VEC{y}\|=1$, we have that
\[
u(r\VEC{y}) = \sum_{k=0}^\infty r^{2-n-k} \phi_k\left(\VEC{y}\right) \ ,
\]
where the convergence of the series is uniform for $r>1/R$ and
$\VEC{y} \in S$ whatever $0 < R < 1$.  We may therefore interchange
the derivative with respect to $r$ with the summation to get
\begin{equation} \label{dudnurxO}
\pdydx{u}{\nu}(r \VEC{y})
= \sum_{k=0}^\infty (2-n-k) r^{1-n-k} \phi_k\left(\VEC{y}\right)
= r^{1-n} \sum_{k=0}^\infty (2-n-k) r^{-k} \phi_k\left(\VEC{y}\right)
\end{equation}
for $r>1$.

\stage{i} We consider first the case $n>2$.  For $r>3$ and $p$ large
enough such that $\displaystyle (n+p-2) < 2^p$, we have that
\[
\left| \sum_{k=p}^{\infty} (2-n-k) r^{-k} \phi_k\left(\VEC{y}\right) \right|
\leq \sum_{k=p}^{\infty} (n+k-2) 3^{-k}
\left| \phi_k\left(\VEC{y}\right) \right| \\
\leq \sum_{k=p}^{\infty} \left(\frac{2}{3}\right)^k
\left| \phi_k\left(\VEC{y}\right) \right|
\]
for all $\VEC{y} \in S$.  Since
$\displaystyle \sum_{k=p}^{\infty} (2/3)^k |\phi_k|$
converges uniformly on $S$ to a continuous function, there exists a
constant $C>0$ such that
$\displaystyle \sum_{k=p}^{\infty} (2/3)^k \left|
  \phi_k\left(\VEC{y}\right) \right| < C$
for all $\VEC{y} \in S$.  It then follows
from (\ref{dudnurxO}) that
$\displaystyle \left| \pdydx{u}{\nu}(r \VEC{y}) \right| = O(r^{1-n})$
for $n>2$ as $r \to \infty$.

\stage{ii} If $n=2$, the expression in (\ref{dudnurxO}) becomes
\[
\pdydx{u}{\nu}(r \VEC{y})  
= r^{-2} \sum_{k=1}^\infty (-k) r^{-k-1} \phi_k\left(\VEC{y}\right) \ .
\]
For $r>3$ and $p$ large enough such that $\displaystyle p < 2^{p+1}$,
we have that
\[
\left| \sum_{k=p}^{\infty} (-k) r^{-k-1} p_k\left(\VEC{y}\right) \right|
\leq \sum_{k=p}^{\infty} k\, 3^{-k-1} \left| p_k\left(\VEC{y}\right) \right| \\
\leq \sum_{k=p}^{\infty} \left(\frac{2}{3}\right)^{k+1}
\left| p_k\left(\VEC{y}\right) \right|
\]
for all $\VEC{y} \in S$.  As before, since
$\displaystyle \sum_{k=p}^{\infty} (2/3)^k |\phi_k|$
converges uniformly on $S$ to a continuous function, there exists a
constant $C>0$ such that
$\displaystyle \sum_{k=p}^{\infty} (2/3)^k \left|
  \phi_k\left(\VEC{y}\right) \right| < C$
for all $\VEC{y} \in S$.  It then follows
from (\ref{dudnurxO}) that
$\displaystyle \left| \pdydx{u}{\nu}(r \VEC{y}) \right| = O(r^{-2})$
for $n=2$ as $r \to \infty$.
\end{proof}

\begin{rmk}
The method used in this chapter to solve (D1), (D2), (N1) and (N2) is called
{\bfseries Layer Potentials}\index{Layer Potentials}.
In Section~\ref{ell_GreenF}, we use the results of the present chapter
to prove Theorem~\ref{laplace_dirichlet2}.
\end{rmk}

\subsection{Uniqueness of solutions} \label{pot_sect_unique}

\subsubsection{Problem D1}
The uniqueness of the solution of (D1) is given by
Theorem~\ref{laplace_uniqu_th}.

\subsubsection{Problem D2}
To prove the uniqueness of the solution of (D2), suppose
that $\displaystyle u \in C(\overline{\Omega^{\prime}})$ is a solution
of (D2) with $u = f = 0$ on $\displaystyle \partial \Omega^{\prime}$.
According to Proposition~\ref{pot_u_ut_harm}, The Kelvin transform
$\tilde{u}$ of $u$ is harmonic in $\displaystyle T(\Omega^{\prime})$
and satisfies $\tilde{u} = \tilde{f} = 0$ on
$\displaystyle T(\partial \Omega^{\prime}) = \partial\,T(\Omega^{\prime})$.
Because $\displaystyle T(\Omega^{\prime})$
is a bounded open subset of $\displaystyle \RR^n$, we have that
$\tilde{u} = 0$ on $\displaystyle T(\Omega^{\prime})$
by uniqueness of solution for the problem (D1). 
This implies that $u =0$ on $\displaystyle \Omega^{\prime}$.

\subsubsection{Problem N1}
To prove the uniqueness of the solution of (N1) modulo a
function which is constant on each connected component of $\Omega$,
suppose that $u \in C_{\nu}(\Omega)$ is a solution of
(N1) with $\displaystyle \pdydx{u}{\nu^-} = f = 0$ on
$\partial \Omega$.

We will need the following famous result.

\begin{theorem}[Tubular Neighbourhood Theorem]\label{pot_TBN}
Suppose that $\displaystyle S \subset \RR^n$ is a $(n-1)$-dimensional
compact oriented manifold of class $\displaystyle C^k$ with $k>1$.
There exists $\mu >0$ such that
\[
O_\mu = \left\{ \VEC{x} + t\, \nu(\VEC{x}) : \VEC{x} \in S \ ,
\ |t|<\mu \right\}
\]
is an open neighbourhood of $S$.  As usual, $\nu(\VEC{x})$ is
the outward unit normal to $S$ at $\VEC{x} \in M$.  Moreover, the
mapping
\begin{align*}
F: S\times ]-\mu,\mu[ &\rightarrow \RR^n \\
(\VEC{x}, t) &\mapsto \VEC{x} + t\, \nu(\VEC{x})
\end{align*}
is a diffeomorphism of class $\displaystyle C^{k-1}$ onto $O_\mu$.
\end{theorem}

\begin{proof}
The proof can be found in any good textbook on differential geometry
like \cite{Ab,Sp}.
\end{proof}

Let $\Omega_\epsilon \subset \Omega$ be the set bounded by
$\partial \Omega_\epsilon = \{ \VEC{x} + \epsilon \nu(\VEC{x}) : \VEC{x}
\in \partial \Omega \}$ for $\epsilon<0$, where is $|\epsilon|$ small
enough so that
$\displaystyle
O_{|\epsilon|} = \left\{ \VEC{x} + t\, \nu(\VEC{x}) : \VEC{x} \in
  \partial \Omega \ , \ |t|<|\epsilon| \right\}$ is a tubular
neighbourhood of $\partial \Omega$.

From the Green's identity (\ref{laplace_green1}) with $v=u$, we have
\begin{equation} \label{potN1Eq1}
\int_{\Omega_\epsilon} \| \graD u \|^2 \dx{\VEC{x}} =
\int_{\partial \Omega_\epsilon} u\,\pdydx{u}{\nu} \dss{S}{x}
- \int_{\Omega_\epsilon} u \Delta u \dx{\VEC{x}}  =
\int_{\partial \Omega_\epsilon} u\,\pdydx{u}{\nu} \dss{S}{x}
\end{equation}
because $\Delta u = 0$ on $\Omega$.  From the Lebesgue's Monotone
Convergence Theorem, we may conclude that
\begin{equation} \label{potN1Eq2}
\int_{\Omega_\epsilon} \| \graD u \|_2^2 \dx{\VEC{x}}
= \int_\Omega \X_{\Omega_\epsilon} \| \graD u \|_2^2 \dx{\VEC{x}}
\rightarrow \int_\Omega \| \graD u \|^2 \dx{\VEC{x}}
\end{equation}
as $\epsilon \rightarrow 0^-$,  where
\[
\X_{\Omega_\epsilon}(\VEC{x}) =
\begin{cases}
1 & \quad \text{if} \ \VEC{x} \in \Omega_\epsilon \\
0 & \quad \text{if} \ \VEC{x} \not\in \Omega_\epsilon
\end{cases}
\]
Moreover, we have that
\begin{equation} \label{potN1Eq3}
\begin{split}
\int_{\partial \Omega_\epsilon} u\,\pdydx{u}{\nu} \dss{S}{x}
&= \int_{\partial \Omega} u\big(\VEC{x} + \epsilon \nu(\VEC{x})\big)
\graD u(\VEC{x} + \epsilon \nu(\VEC{x})) \cdot \nu(\VEC{x}) 
\dss{S}{x} \\
&\rightarrow \int_{\partial \Omega} u(\VEC{x}) \pdydx{u}{\nu^-}(\VEC{x})
\dss{S}{x} = 0
\end{split}
\end{equation}
as $\epsilon \rightarrow 0^-$ because
$\displaystyle
\graD u(\VEC{x} + \epsilon \nu(\VEC{x})) \cdot \nu(\VEC{x})
\to \pdydx{u}{\nu^-}(\VEC{x})$ uniformly on $\partial \Omega$ as
$\epsilon \to 0^-$ since $u \in C_{\nu}(\Omega)$, and
$\displaystyle u\big(\VEC{x} + \epsilon \nu(\VEC{x})\big) \to
u(\VEC{x})$ uniformly on $\partial \Omega$ as $\epsilon \to 0^-$
since $u$ is uniformly continuous on the compact set $\overline{\Omega}$.
We have also used the fact that the outward unit normal to
$\partial \Omega_\epsilon$ at
$\VEC{x} + \epsilon \nu(\VEC{x})$ is $\nu(\VEC{x})$
because of our tubular neighbourhood.

It follows from (\ref{potN1Eq1}), (\ref{potN1Eq2}) and (\ref{potN1Eq3})
that $\displaystyle \int_\Omega \| \graD u \|^2 \dx{\VEC{x}} = 0$.
Since $\displaystyle u \in C^1(\Omega)$, this implies that $\graD u = \VEC{0}$
everywhere on $\Omega$.  Thus $u$ is constant on each connected
component of $\Omega$.

\subsubsection{Problem N2}
We prove the uniqueness of the solution of (N2) modulo a
function which is constant on each connected component of $\Omega$.
Moreover, we prove that $u=0$ on the unbounded component if $n>2$.

Suppose that $\displaystyle u \in C_{\nu}(\Omega^{\prime})$ is a
solution of (N2) with $\displaystyle \pdydx{u}{\nu^+} = f = 0$ on
$\partial \Omega$.

Let $\displaystyle \Omega^{\prime}_\epsilon \subset \Omega^{\prime}$
be the set bounded by $\partial \Omega^{\prime}_\epsilon
= \{ \VEC{x} + \epsilon \nu(\VEC{x}) : \VEC{x}
\in \partial \Omega \}$ for $\epsilon>0$ small enough
so that
$\displaystyle
O_{\epsilon} = \left\{ \VEC{x} + t\, \nu(\VEC{x}) : \VEC{x} \in
  \partial \Omega \ , \ |t|<\epsilon \right\}$ is a tubular
neighbourhood of $\displaystyle \partial \Omega = \partial \Omega^{\prime}$.

Choose $r>0$ such that
$\displaystyle \overline{\RR^n \setminus \Omega^{\prime}_\epsilon}
\subset B_r(\VEC{0})$.
From the Green's identity (\ref{laplace_green1}) with $v=u$, we have
\begin{align}
\int_{B_r(\VEC{0})\cap\Omega^{\prime}_\epsilon} \| \graD u \|^2 \dx{\VEC{x}} &=
\int_{\partial B_r(\VEC{0})} u\,\pdydx{u}{\nu} \dss{S}{x}
- \int_{\partial \Omega^{\prime}_\epsilon} u\,\pdydx{u}{\nu} \dss{S}{x}
- \int_{B_r(\VEC{0})\cap\Omega^{\prime}_\epsilon} u \Delta u \dx{\VEC{x}}
\nonumber \\
&= \int_{\partial B_r(\VEC{0})} u\,\pdydx{u}{\nu} \dss{S}{x}
- \int_{\partial \Omega^{\prime}_\epsilon} u\,\pdydx{u}{\nu} \dss{S}{x}
\label{potN2Eq1}
\end{align}
because $\Delta u =0$ in $\displaystyle \Omega^{\prime}$.

Using the Lebesgue's Monotone Convergence Theorem as we did to prove
the uniqueness of the solution for the problem (N1), we have that
\begin{equation} \label{potN2Eq2}
\int_{B_r(\VEC{0})\cap\Omega^{\prime}_\epsilon} \| \graD u \|^2 \dx{\VEC{x}}
\rightarrow \int_{B_r(\VEC{0})\cap\Omega^{\prime}} \| \graD u \|^2 \dx{\VEC{x}}
\end{equation}
as $\epsilon \rightarrow 0^+$.
Moreover, it follows from (\ref{pot_ddn_unif}) that
\begin{equation} \label{potN2Eq3}
\begin{split}
\int_{\partial \Omega^{\prime}_\epsilon} u\,\pdydx{u}{\nu} \dss{S}{x}
&= \int_{\partial \Omega} u\big(\VEC{x} + \epsilon \nu(\VEC{x})\big)
\graD u(\VEC{x} + \epsilon \nu(\VEC{x})) \cdot
\nu(\VEC{x}) \dss{S}{x} \\
&\rightarrow \int_{\partial \Omega} u(\VEC{x}) \pdydx{u}{\nu^+}(\VEC{x})
\dss{S}{x} = 0
\end{split}
\end{equation}
as $\epsilon \rightarrow 0^+$ because
$\displaystyle
\graD u(\VEC{x} + \epsilon \nu(\VEC{x})) \cdot \nu(\VEC{x})
\to \pdydx{u}{\nu^+}(\VEC{x})$ uniformly on $\partial \Omega$ as
$\epsilon \to 0^+$ since
$\displaystyle u \in C_{\nu}(\Omega^{\prime})$, and
$\displaystyle u\big(\VEC{x} + \epsilon \nu(\VEC{x})\big) \to
u(\VEC{x})$ uniformly on $\partial \Omega$ as $\epsilon \to 0^-$
since $u$ is uniformly continuous on the compact set
$\displaystyle\overline{B_r(\VEC{0})\cap\Omega^{\prime}}$.
It follows from (\ref{potN2Eq1}), (\ref{potN2Eq2}) and
(\ref{potN2Eq3}) that 
\begin{equation} \label{pot_N2_pequ}
\int_{B_r(\VEC{0})\cap\Omega^{\prime}} \| \graD u \|^2 \dx{\VEC{x}} =
\int_{\partial B_r(\VEC{0})} u\,\pdydx{u}{\nu} \dss{S}{x} \  .
\end{equation}

\subI{$\mathbf{n>2}$} From Proposition~\ref{pot_infty_u} and
Proposition~\ref{pot_infty_ddu}, we have that
$\displaystyle |u(\VEC{x})| = O(\|\VEC{x}\|^{2-n})$ and
$\displaystyle \left| \pdydx{u}{\nu}(\VEC{x})\right|
= O(\|\VEC{x}\|^{1-n})$ as $\|\VEC{x}\|\rightarrow \infty$.  Hence,
there exists a constant $C$ such that
\[
\left| \int_{\partial B_r(\VEC{0})} u \pdydx{u}{\nu} \dss{S}{x}
\right| \leq
\int_{\partial B_r(\VEC{0})} \left| u \right|\,\left| \pdydx{u}{\nu} \right|
\dss{S}{x}
\leq C r^{2-n} r^{1-n}
\underbrace{\int_{\partial B_r(\VEC{0})} \dss{S}{x}}_{=\omega_n \,r^{n-1}}
= C \omega_n r^{2-n} \rightarrow 0 
\]
as $r \rightarrow \infty$.
It follows from (\ref{pot_N2_pequ}) and the Lebesgue's Monotone
Convergence Theorem that
\[
\int_{B_r(\VEC{0})\cap\Omega^{\prime}} \| \graD u \|_2^2 \dx{\VEC{x}} \rightarrow
\int_{\Omega^{\prime}} \| \graD u \|_2^2 \dx{\VEC{x}} = 0
\]
as $r \rightarrow \infty$.
Since $\displaystyle u \in C^1(\Omega^{\prime})$, we find that
$\graD u = \VEC{0}$ everywhere on $\displaystyle \Omega^{\prime}$.
Thus $u$ is constant on each connected component of $\Omega^{\prime}$.

Moreover, $u=0$ on $\displaystyle \Omega_0^{\prime}$, the unbounded component
of $\displaystyle \Omega^{\prime}$, because $u(\VEC{x}) \rightarrow 0$ as
$\|\VEC{x}\|\rightarrow \infty$ according to
Proposition~\ref{pot_infty_u}.  This is possible for a constant
function $u$ on $\displaystyle \Omega_0^{\prime}$ only if $u=0$
on $\displaystyle \Omega_0^{\prime}$.

\subI{$\mathbf{n=2}$} From Proposition~\ref{pot_infty_u} and
Proposition~\ref{pot_infty_ddu}, we have that
$|u(\VEC{x})| = o(\ln(\|\VEC{x}\|))$ and
$\displaystyle \left| \pdydx{u}{\nu}(\VEC{x})\right|
= O(\|\VEC{x}\|^{-2})$ as $\|\VEC{x}\|\rightarrow \infty$.  Hence,
there exists a constant $C$ such that
\[
\left| \int_{\partial B_r(\VEC{0})} u \pdydx{u}{\nu} \dss{S}{x}
\right| \leq
\int_{\partial B_r(\VEC{0})} \left| u \right|\,\left| \pdydx{u}{\nu} \right|
\dss{S}{x}
\leq C r^{-2} \ln(r)
\underbrace{\int_{\partial B_r(\VEC{0})} \dss{S}{x}}_{=2\pi r}
= 2\pi C \frac{\ln(r)}{r} \rightarrow 0
\]
as $r \rightarrow \infty$.
As before, it follows from (\ref{pot_N2_pequ}) and the Lebesgue
Monotone Convergence Theorem that
\[
\int_{B_r(\VEC{0})\cap\Omega^{\prime}} \| \graD u \|_2^2 \dx{\VEC{x}} \rightarrow
\int_{\Omega^{\prime}} \| \graD u \|^2 \dx{\VEC{x}} = 0
\]
as $r \rightarrow \infty$.
Since $\displaystyle u \in C^1(\Omega^{\prime})$, we find that
$\graD u = \VEC{0}$ everywhere on $\displaystyle \Omega^{\prime}$.
Thus $u$ is constant on each connected component of
$\displaystyle \Omega^{\prime}$.

\subsection{Necessary condition for the Neumann Problem}
\label{pot_sect_necessary}

In this section, we give necessary conditions for the existence of the
solutions of the Interior and Exterior Neumann problem.  The first
proposition has been proved in the previous section.

\begin{prop} \label{pot_nec_Nprobl0}
If $u$ is a solution of (N2) with $f=0$ on
$\displaystyle \partial \Omega^{\prime}$, the following statements are true.
\begin{enumerate}
\item For $n=2$, $u$ is constant on each connected component
$\displaystyle \Omega^{\prime}_j$,
$\displaystyle 0\leq j \leq m^{\prime}$, of $\displaystyle \Omega^{\prime}$.
\item For $n>2$, $u$ is constant on each bounded connected component
$\displaystyle \Omega^{\prime}_j$,
$\displaystyle 1\leq j \leq m^{\prime}$, of $\displaystyle \Omega^{\prime}$
and $u=0$ on the unbounded connected component
$\displaystyle \Omega^{\prime}_0$.
\end{enumerate}
\end{prop}

\begin{prop} \label{pot_nec_Nprobl}
\begin{enumerate}
\item If $u$ is a solution of (N1), then
$\displaystyle \int_{\partial \Omega_i} f =0$ for $1 \leq i \leq m$.
\item If $u$ is a solution of (N2), then
$\displaystyle \int_{\partial \Omega^{\prime}_i} f =0$ for
$\displaystyle 1 \leq i \leq m^{\prime}$.
Moreover, $\displaystyle \int_{\partial \Omega^{\prime}_0} f =0$ for $n=2$.
\end{enumerate}
\end{prop}

\begin{proof}
\stage{1}
For $1\leq i \leq m$, let $\Omega_{i,\epsilon} \subset \Omega_i$ be
the set bounded by
$\displaystyle \partial \Omega_{i,\epsilon}
= \{\VEC{x} + \epsilon \nu(\VEC{x}) : \VEC{x}
\in \partial \Omega_i \}$ for $\epsilon<0$ and $|\epsilon|$ small
enough so that
$\displaystyle
O_{|\epsilon|} = \left\{ \VEC{x} + t\, \nu(\VEC{x}) : \VEC{x} \in
  \partial \Omega_i \ , \ |t|<|\epsilon| \right\}$ is a tubular
neighbourhood of $\partial \Omega_i$.

From the Green's identity (\ref{laplace_green1}) with $v=1$, we have
\begin{equation} \label{potNecessEq1}
\int_{\partial \Omega_{i,\epsilon}} \pdydx{u}{\nu} \dss{S}{x}
= \int_{\Omega_{i,\epsilon}} \Delta u \dx{\VEC{x}} = 0
\end{equation}
because $\Delta u = 0$ on $\Omega_i$.  However, we have that
\[
\int_{\partial \Omega_{i,\epsilon}} \pdydx{u}{\nu} \dss{S}{x}
= \int_{\partial \Omega_i} \graD u(\VEC{x} + \epsilon \nu(\VEC{x}))
\cdot \nu(\VEC{x}) \dss{S}{x}
\rightarrow \int_{\partial \Omega_i} \pdydx{u}{\nu^-} \dss{S}{x}
= \int_{\partial \Omega_i} f \dss{S}{x}
\]
as $\epsilon \rightarrow 0^-$ because
$\displaystyle
\graD u(\VEC{x} + \epsilon \nu(\VEC{x})) \cdot \nu(\VEC{x})
\to \pdydx{u}{\nu^-}(\VEC{x})$ uniformly on $\partial \Omega_i$ as
$\epsilon \to 0^-$ since $u \in C_{\nu}(\Omega)$.  It follows
from (\ref{potNecessEq1}) that
$\displaystyle \int_{\partial \Omega_i} f \dss{S}{x} = 0$.

\stage{2} The prove that
$\displaystyle \int_{\partial \Omega^{\prime}_i} f \dss{S}{x} = 0$ for
$\displaystyle 1 \leq i \leq m^{\prime}$ is identical to the proof given in
(i) if $\Omega_i$ is replaced by $\displaystyle \Omega_i^{\prime}$ and
$\epsilon >0$ is used.

If $n=2$, let
$\displaystyle \Omega^{\prime}_{0,\epsilon} \subset \Omega^{\prime}_0$ be
the set bounded by
$\displaystyle \partial \Omega^{\prime}_{0,\epsilon}
= \{ \VEC{x} + \epsilon \nu(\VEC{x}) :
\VEC{x} \in \partial \Omega^{\prime}_0 \}$ for $\epsilon>0$ small
enough so that
$\displaystyle
O_{\epsilon} = \left\{ \VEC{x} + t\, \nu(\VEC{x}) : \VEC{x} \in
  \partial \Omega_0^{\prime} \ , \ |t|<\epsilon \right\}$ is a tubular
neighbourhood of $\displaystyle \partial \Omega = \partial \Omega^{\prime}_0$.

Choose $r>0$ large enough such that
$\displaystyle \partial \Omega^{\prime}_{0,\epsilon} \subset B_r(\VEC{0})$.

From the Green's identity (\ref{laplace_green1}) with $v=1$, we have
\begin{equation} \label{pot_ness_idx1}
\int_{\partial \Omega^{\prime}_{0,\epsilon}} \pdydx{u}{\nu} \dss{S}{x}
= \int_{\partial B_r(\VEC{0})} \pdydx{u}{\nu} \dss{S}{x}
- \int_{B_r(\VEC{0}) \cap \Omega^{\prime}_{0,\epsilon}} \Delta u \dx{\VEC{x}} =
\int_{\partial B_r(\VEC{0})} \pdydx{u}{\nu} \dss{S}{x}
\end{equation}
because $\displaystyle \Delta u = 0$ on $\Omega_0^{\prime}$.  We have that
\begin{equation} \label{pot_ness_idx2}
\int_{\partial \Omega^{\prime}_{0,\epsilon}} \pdydx{u}{\nu} \dss{S}{x}
= \int_{\partial \Omega^{\prime}_0} \graD u(\VEC{x}
+ \epsilon \nu(\VEC{x}))
\cdot \nu(\VEC{x}) \dss{S}{x}
\rightarrow \int_{\partial \Omega^{\prime}_0}
\pdydx{u}{\nu^+} \dss{S}{x}
= \int_{\partial \Omega^{\prime}_0} f \dss{S}{x}
\end{equation}
as $\epsilon \rightarrow 0^+$ because
$\displaystyle
\graD u(\VEC{x} + \epsilon \nu(\VEC{x})) \cdot \nu(\VEC{x})
\to \pdydx{u}{\nu^+}(\VEC{x})$ uniformly on
$\partial \Omega^{\prime}_0$ as
$\epsilon \to 0^+$ since $u \in C_{\nu}(\Omega^{\prime})$.
Moreover, from Proposition~\ref{pot_infty_ddu}, we have that 
$\displaystyle \left| \pdydx{u}{\nu}(\VEC{x})\right| =
O(\|\VEC{x}\|^{-2})$ as $\|\VEC{x}\|\rightarrow \infty$.  Therefore,
there exists a constant $C$ such that
\begin{equation} \label{pot_ness_idx3}
\left| \int_{\partial B_r(\VEC{0})} \pdydx{u}{\nu} 
\dss{S}{x} \right| \leq
\int_{\partial B_r(\VEC{0})} \left| \pdydx{u}{\nu} \right|
\dss{S}{x} \leq C r^{-2} \int_{\partial B_r(\VEC{0})}
\dss{S}{x}
= 2\pi C r^{-1} \rightarrow 0
\end{equation}
as $r \rightarrow \infty$.
It follows from (\ref{pot_ness_idx1}), (\ref{pot_ness_idx2}) and
(\ref{pot_ness_idx3}), that
$\displaystyle \int_{\partial \Omega^{\prime}_0} f \dss{S}{x} = 0$.
\end{proof}

\subsection{Note on the Integration on Manifolds} \label{pot_int_man}

Before diving into the study of the integral operators which are of
interest to us, we need to review integration on manifolds.  Good
references on the subject are \cite{Ab,Sp}.

\begin{defn} \label{pot_part_unity}
Let $\Omega$ be an open subset of $\displaystyle \RR^n$ and $S$ be a subset of
$\Omega$.  Let $\displaystyle \{V_i\}_{i=1}^\infty$ be a collection of
open subsets of $\Omega$ such that
$\displaystyle S \subset \bigcup_{i=1}^\infty V_i$.
A {\bfseries partition of unity}\index{Partition of Unity}
for $S$ subordinated to $\displaystyle \{V_i\}_{i=1}^\infty$ is a set
of functions $\displaystyle \{\phi_i\}_{i=1}^\infty$ such that:
\begin{enumerate}
\item The $\phi_i$ are $\displaystyle C^\infty$ functions on $\Omega$.
\item $0 \leq \phi_i(\VEC{x}) \leq 1$ for all $i$ and all $\VEC{x} \in \Omega$.
\item $\supp \phi_i \subset V_i$ for all $i$.
\item For any compact set $K \subset S$, we have that
$\supp \phi_i \cap K = \emptyset$ for all but a finite number of indices.
\item $\displaystyle \sum_{i=1}^\infty \phi_i(\VEC{x}) = 1$ for all
$\VEC{x}\in S$.
\end{enumerate}
The definition of partition of unity subordinated to a finite
collection of open subsets $\displaystyle \{V_i\}_{i=1}^N$ is identical.
Note that the support of $\phi_i$ may not be compact if the open set
$V_i$ is not bounded.
\end{defn}

The summation in the previous definition is finite for every
$\VEC{x} \in S$ because, given any compact set $K\subset S$,
$K\cap \supp \phi_i = \emptyset$ for all but a finite number of indices.

\begin{prop}
Let $\Omega$ be an open subset of $\displaystyle \RR^n$ and $S$ be a subset of
$\Omega$.  If $\displaystyle \{ V_i\}_{i=1}^\infty$ is a collection
of open sets in $\Omega$ such that
$\displaystyle S \subset \bigcup_{i=1}^\infty V_i$.
Then, there exists a partition of unity for $S$
subordinate to $\displaystyle \{ V_i\}_{i=1}^\infty$.
\end{prop}

\begin{proof}
This proposition is proved in any good functional analysis or
differential geometry textbook.  For instance, see \cite{Ad}. 
\end{proof}

There is another definition of partition of unity that will be useful
in Section~\ref{WequalH} and that we introduce right now.

\begin{defn} \label{pot_part_unity_V2}
Let $\Omega$ be an open subset of $\displaystyle \RR^n$ and
$\displaystyle \{V_i\}_{i=1}^\infty$
be a collection of open subsets of $\displaystyle \RR^n$ such that
$\displaystyle \Omega = \bigcup_{i=1}^\infty V_i$.
A {\bfseries partition of unity}\index{Partition of Unity}
for $\Omega$ subordinated to $\displaystyle \{V_i\}_{i=1}^\infty$ is a
set of functions $\displaystyle \{\phi_i\}_{i=1}^\infty$ such that:
\begin{enumerate}
\item The $\phi_i$ are $\displaystyle C^\infty$ functions on $\Omega$.
\item $0 \leq \phi_i(\VEC{x}) \leq 1$ for all $i$ and all $\VEC{x} \in \Omega$.
\item $\supp \phi_i \subset V_i$ for all $i$.
\item For any compact set $K \subset \Omega$, there exists an open set
$W \supset K$ such that $\supp \phi_i \cap W = \emptyset$ for all but
a finite number of indices.
\item $\displaystyle \sum_{i=1}^\infty \phi_i(\VEC{x}) = 1$ for all
$\VEC{x}\in \Omega$.
\end{enumerate}
Note that the support of $\phi_i$ may not be compact if the open set
$V_i$ is not bounded.
\end{defn}

As for the previous definition of partition of unity,
the summation in the previous definition is finite for every
$\VEC{x} \in \Omega$ because, given any compact set $K\subset \Omega$,
there exist an open set $W \supset K$ such that
$W \cap \supp \phi_i = \emptyset$ for all but a finite number of indices.

Note that Definition~\ref{pot_part_unity_V2} implies
Definition~\ref{pot_part_unity}.

\begin{prop}
Let $\Omega$ be an open subset of $\displaystyle \RR^n$ and
$\displaystyle \{ V_i\}_{i=1}^\infty$ is be collection
of open sets in $\Omega$ such that
$\displaystyle S \subset \bigcup_{i=1}^\infty V_i$.
Then, there exists a partition of unity for $\Omega$
subordinate to $\displaystyle \{ V_i\}_{i=1}^\infty$.
\end{prop}

\begin{proof}
See \cite{RuFA}. 
\end{proof}

Suppose that $S$ is an oriented differentiable manifold and
$f:S \rightarrow \RR$ is a
continuous function with compact support.  Let
$\displaystyle \BB = \{ (U_j, \psi_j) \}_{j\in J}$ be an atlas for
$S$, where $J \subset \NN$ is an index set.
Moreover, let $\displaystyle \{ \phi_j \}_{j\in J}$ be a partition of unity
for $S$ subordinated to $\BB$; namely,
$\displaystyle \{ \phi_j \}_{j\in J}$ is a partition of unity for $S$
subordinated to $\displaystyle \{ V_j \}_{j\in J}$,
where $U_j = V_j \cap S$ and $V_j$ is an open subset of
$\displaystyle \RR^n$.  Note 
that we assume that the topology on $S$ is the induced topology from
$\displaystyle \RR^n$.

The integral $\displaystyle \int_S f(\VEC{x}) \dx{S}$ is defined by
\begin{align*}
\int_S f(\VEC{x}) \dx{S} &= \int_S
\sum_{j\in J} \phi_j(\VEC{x}) f(\VEC{x}) \dx{S} =
\sum_{j\in J} \int_{U_j} \phi_j(\VEC{x}) f(\VEC{x}) \dx{S} \\
&= \sum_{j\in J} \int_{\psi_j(U_j)} \phi_j(\psi_j^{-1}(\VEC{w}))
f(\psi_j^{-1}(\VEC{w}))
\left|\det \diff \psi_j^{-1} (\VEC{w}) \right| \dx{\VEC{w}} \ .
\end{align*}
The integral is independent of the chosen atlas and partition of
unity.  Moreover, since $f$ has a compact support, the summation is
finite.  The sum is over all $j$ such that
$U_j \cap \supp f \neq \emptyset$.

The manifold that we consider in this chapter is $S = \partial \Omega$
where $\displaystyle \Omega \subset \RR^n$ is a bounded open set.  We also
assume $S$ is of class at least $\displaystyle C^2$.  The orientation on $S$ is
determined by the outward unit normal $\nu(\VEC{x})$ to $S$ at
$\VEC{x} \in S$.

To simplify the computations later, we select a particular atlas of
$S$ as it follows.  At each point $\VEC{x} \in M$, we select a
chart $\displaystyle \left(U_{\VEC{x}},\psi_{\VEC{x}}\right)$ such
that:
\begin{itemize}
\item $\displaystyle U_{\VEC{x}}$ is a connected set containing
$\VEC{x}$ and $\displaystyle U_{\VEC{x}} = S \cap V_{\VEC{x}}$ for
some bounded open set $\displaystyle V_{\VEC{x}} \subset \RR^n$.  This is
possible because $S$ is the boundary of an open set.
\item $\displaystyle \psi_{\VEC{x}}$ is the projection of
$\displaystyle U_{\VEC{x}}$ onto
$\displaystyle W_{\VEC{x}} \subset T_{\VEC{x}}$ along the vector
$\nu(\VEC{x})$, where $\displaystyle T_{\VEC{x}}$ is the
tangent space to $\displaystyle U_{\VEC{x}}$ at the point
$\VEC{x}$.  We have that
$\displaystyle \psi_{\VEC{x}}^{-1}: W_{\VEC{x}} \to \RR^n$ is of
class $\displaystyle C^2$.  Using translation and rotation, we may consider
that $\displaystyle U_{\VEC{x}}$ is the graph
of a function of class $\displaystyle C^2$ from $\displaystyle \RR^{n-1}$
to $\RR$
(Figure~\ref{MAN_PARAM_V2}).
\item By shrinking $\displaystyle V_{\VEC{x}}$ and $\displaystyle U_{\VEC{x}}$
if necessary, we may assume that $\displaystyle \psi_{\VEC{x}}^{-1}$ is of class
$\displaystyle C^2$ on the compact set $\displaystyle \overline{W_{\VEC{x}}}$.
\item It is also convenient to assume that the diameter of
$V_{\VEC{x}}$, and so of $W_{\VEC{x}}$, is lest than $1$.
\end{itemize}

\pdfF{potential/man_param_v2}{Chart of a manifold}
{Chart $\displaystyle (U_{\VEC{x}},\psi_{\VEC{x}})$ of the manifold $S$ in
Section~\ref{pot_int_man}.}{MAN_PARAM_V2}

Since $S$ is compact, there exists a finite cover
$\displaystyle \left\{ V_{\VEC{x}_j} \right\}_{j\in J}$ of $S$
and so a finite atlas
$\displaystyle \BB
= \left\{ \left(U_{\VEC{x}_j}, \psi_{\VEC{x}_j}\right) \right\}_{j\in J}$
of $S$.

Let $\displaystyle Q_{\VEC{x}_j} = \max_{\VEC{w} \in \overline{W_{\VEC{x}_j}}}
= \left|\det \diff \psi_{\VEC{x}_j}^{-1}(\VEC{w})\right|$.  We have that
$\displaystyle Q_{\VEC{x}_j} < \infty$ because we assume that
$\displaystyle \phi_{\VEC{x}_j}^{-1}$ is of class $\displaystyle C^2$
on the compact set $\displaystyle \overline{W_{\VEC{x}_j}}$.
If $\displaystyle Q = \max_{j \in J} Q_{\VEC{x}_j}$, then
\begin{equation} \label{defnQdetpsim1}
\left|\det \diff \psi_{\VEC{x}_j}^{-1}(\VEC{w})\right| < Q
\end{equation}
for all $\displaystyle \VEC{w} \in W_{\VEC{x}_j}$ and all $j \in J$.

Given $\VEC{x} \in S$ and $\eta < 1$,
let $\displaystyle S_{\VEC{x},\eta} = S \cap B_{\eta}(\VEC{x})$.
In the next sections, we will need an upper bound for the integrals
\[
\int_{S_{\VEC{x},\eta}} \|\VEC{x}-\VEC{y}\|^{-\alpha} \dss{S}{y} 
\quad \text{and} \quad
\int_{S_{\VEC{x},\eta}} \left| \ln(\|\VEC{x}-\VEC{y}\|) \right| \dss{S}{y} 
\]
which depends only on $\eta$ but not on $\VEC{x} \in S$, where
$0 \leq \alpha < n - 1$.

If $\eta/2$ is smaller than the Lebesgue number associated to the
open covering $\displaystyle \{ V_{\VEC{x}_j}\}_{j\in J}$ of
$S$~\footnote{\cite{Mu} is a good reference on the subject of the Lebesgue
number.}, then
$\displaystyle \overline{B_\eta(\VEC{x})} \subset V_{\VEC{x}_j}$
for some $j \in J$.  Thus
$\displaystyle \overline{S_{\VEC{x},\eta}} \subset U_{\VEC{x}_j}$
(Figure~\ref{MAN_PARAM_V3}).

\pdfF{potential/man_param_v3}{Small region on a manifold}
{Region $\displaystyle S_{\VEC{x},\eta}$ of the manifold $S$ in
Section~\ref{pot_int_man}.}{MAN_PARAM_V3}

\stage{$\mathbf{n>2}$}
Since $\displaystyle \|\VEC{x}-\VEC{y}\| 
\geq \|\psi_{\VEC{x}_j}(\VEC{x}) - \psi_{\VEC{x}_j}(\VEC{y})\|$ for all
$\displaystyle \VEC{x},\VEC{y} \in U_{\VEC{x}_j}$ and
$\displaystyle \psi_{\VEC{x}_j}(S_{\VEC{x},\eta}) \subset
B_{\eta}(\psi_{\VEC{x}_j}(\VEC{x}))$, we get that
\begin{align*}
&\int_{S_{\VEC{x},\eta}} \|\VEC{x}-\VEC{y}\|^{-\alpha} \dss{S}{y}
= \int_{\psi_{\VEC{x}_j}(S_{\VEC{x},\eta})}
\left\|\VEC{x}-\psi_{\VEC{x}_j}^{-1}(\VEC{w})\right\|^{-\alpha}
\left| \det \diff \psi_{\VEC{x}_j}^{-1}(\VEC{w})\right| \dx{\VEC{w}} \\
& \qquad
\leq \int_{\psi_{\VEC{x}_j}(S_{\VEC{x},\eta})}
\left\|\psi_{\VEC{x}_j}(\VEC{x})-\VEC{w}\right\|^{-\alpha}
\left| \det \diff \psi_{\VEC{x}_j}^{-1}(\VEC{w})\right| \dx{\VEC{w}}
\leq Q \int_{B_{\eta}(\psi_{\VEC{x}_j}(\VEC{x}))}
\left\|\psi_{\VEC{x}_j}(\VEC{x})-\VEC{w}\right\|^{-\alpha} \dx{\VEC{w}} \\
&\qquad
= Q \int_{\partial B_1(\VEC{0})} \int_0^\eta r^{-\alpha}
r^{n-2} \dx{r} \dx{S}
= \frac{ Q \omega_{n-1} \eta^{n-\alpha-1}}{n-\alpha -1} \ ,
\end{align*}
where we have use the change of variables
$\VEC{w} = \psi_{\VEC{x}_j}(\VEC{x}) + r \VEC{z}$ for
$0 \leq r < \eta$ and $\VEC{z} \in \partial B_1(\VEC{0}) \subset \RR^{n-1}$.

Similarly, we get
\begin{align*}
&\int_{S_{\VEC{x},\eta}} \left|\ln(\|\VEC{x}-\VEC{y}\|)\right| \dss{S}{y}
= \int_{\psi_{\VEC{x}_j}(S_{\VEC{x},\eta})}
\left| \ln\left(\left\|\VEC{x}
-\psi_{\VEC{x}_j}^{-1}(\VEC{w})\right\|\right)\right| \,
\left| \det \diff \psi_{\VEC{x}_j}^{-1}(\VEC{w})\right| \dx{\VEC{w}} \\
& \quad
\leq \int_{\psi_{\VEC{x}_j}(S_{\VEC{x},\eta})}
\left| \ln\left( \left\|\psi_{\VEC{x}_j}(\VEC{x})-\VEC{w}\right\|\right)\right|
\, \left| \det \diff \psi_{\VEC{x}_j}^{-1}(\VEC{w})\right| \dx{\VEC{w}}
\leq Q \int_{B_{\eta}(\psi_{\VEC{x}_j}(\VEC{x}))}
\left|\ln\left( \left\|\psi_{\VEC{x}_j}(\VEC{x})-\VEC{w}\right\|\right)\right|
\dx{\VEC{w}} \\
&\quad
= Q \int_{\partial B_1(\VEC{0})} \int_0^\eta |\ln(r)|
  r^{n-2} \dx{r} \dx{S}
\leq Q \omega_{n-1} \left( \frac{|\ln(\eta)|\, \eta^{n-1}}{n-1}  +
  \frac{\eta^{n-1}}{(n-1)^2}  \right) \ ,
\end{align*}
where again we have use the change of variables 
$\VEC{w} = \psi_{\VEC{x}_j}(\VEC{x}) + r \VEC{z}$ for
$0 \leq r < \eta <1$ and
$\VEC{z} \in \partial B_1(\VEC{0}) \subset \RR^{n-1}$.

\stage{$\mathbf{n=2}$}
The computations are very similar to those of the case $n>2$ with a
single difference as we will see.  Since $n = 2$, we have that
$\alpha = 0$.  Hence, we get
\[
\int_{S_{\VEC{x},\eta}} \|\VEC{x}-\VEC{y}\|^{-\alpha} \dss{S}{y}
= \int_{S_{\VEC{x},\eta}} \dss{S}{y}
\leq Q \int_{B_{\eta}(\psi_{\VEC{x}_j}(\VEC{x})) \subset \RR} \dx{w}
= Q \int_{\psi_{\VEC{x}_j}(\VEC{x})-\eta}^{\psi_{\VEC{x}_j}(\VEC{x})+\eta}
  \dx{w} = 2 Q \eta  \ .
\]

Similarly, we get
\begin{align*}
&\int_{S_{\VEC{x},\eta}} \left|\ln(\|\VEC{x}-\VEC{y}\|)\right| \dss{S}{y}
\leq Q \int_{B_{\eta}(\psi_{\VEC{x}_j}(\VEC{x}))\subset \RR}
\left|\ln\left( \left\|\psi_{\VEC{x}_j}(\VEC{x})-w\right\|\right)\right|
\dx{w} \\
&\qquad = 2Q \int_{\psi_{\VEC{x}_j}(\VEC{x})}^{\psi_{\VEC{x}_j}(\VEC{x})+\eta}
\left| \ln\left( \left\|\psi_{\VEC{x}_j}(\VEC{x})-w\right\|\right) \right|
\dx{w}
= 2Q \int_0^{\eta} |\ln(r)| \dx{r}
\leq 2 Q \left( \eta |\ln(\eta)| + \eta \right) \ .
\end{align*}

In conclusion,
\begin{equation} \label{Sxepsint1}
\int_{S_{\VEC{x},\eta}} \|\VEC{x}-\VEC{y}\|^{-\alpha} \dss{S}{y}
\leq \frac{ Q \omega_{n-1} \eta^{n-\alpha-1}}{n-\alpha -1}
\end{equation}
and
\begin{equation} \label{Sxepsint2}
\int_{S_{\VEC{x},\eta}} \left|\ln(\|\VEC{x}-\VEC{y}\|)\right| \dss{S}{y}
\leq Q \omega_{n-1} \left( \frac{|\ln(\eta)|\, \eta^{n-1}}{n-1}  +
  \frac{\eta^{n-1}}{(n-1)^2} \right)
\end{equation}
for $0 \leq \alpha < n-1$ and $n\geq 2$ if we set
$\omega_1 = 2$.

\subsection{Integral Operators}

In this section, we set $S = \partial \Omega$.  Recall that $S$ is a compact
$\displaystyle C^2$-manifold of dimension $n-1$ in $\displaystyle \RR^n$.

\begin{defn} \label{pot_kerOk}
Let $K:S \times S \rightarrow \RR$ be a measurable function.  $K$ is a
{\bfseries kernel of order $\alpha$}\index{Kernel of Order $\alpha$}
with $0< \alpha < n-1$ if
\[
K(\VEC{x},\VEC{y}) = A(\VEC{x}, \VEC{y}) \|\VEC{x}-\VEC{y}\|^{-\alpha} \ ,
\]
where $A:S\times S \rightarrow \RR$ is a bounded measurable function.
$K$ is a
{\bfseries kernel of order $0$}\index{Kernel of Order $0$} if
\[
K(\VEC{x},\VEC{y}) = A(\VEC{x}, \VEC{y}) \ln(\|\VEC{x}-\VEC{y}\|)
+B(\VEC{x},\VEC{y}) \ ,
\]
where $A, B:S\times S \rightarrow \RR$ are bounded measurable
functions.  Moreover, if
\[
K: \{ (\VEC{x},\VEC{y}) \in S \times S : \VEC{x} \neq \VEC{y} \}
\rightarrow \RR
\]
is also continuous, we say that $K$ is a
{\bfseries continuous kernel of order $\alpha$}\index{Continuous
Kernel of Order $\alpha$}
\end{defn}

\begin{prop} \label{pot_lin_op_K}
Suppose that $K:S\times S \rightarrow \RR$ is a kernel of order $\alpha$
with $0\leq \alpha < n-1$, and 
$\supp K \subset \{(\VEC{x}, \VEC{y}) \in S \times S :
\|\VEC{x}-\VEC{y}\| <\epsilon \}$ for $\epsilon > 0$.  Then
\[
(T_K f)(\VEC{x}) = \int_S K(\VEC{x},\VEC{y}) f(\VEC{y}) \dss{S}{y}
\quad , \quad \VEC{x} \in S \ ,
\]
defines a bounded linear operator
$\displaystyle T_K : L^p(S) \rightarrow L^p(S)$
for $1\leq p \leq \infty$.
If $\alpha>0$, then there exists a constant $C$ such that
\begin{equation} \label{pot_TK_Bound1}
\| T_K f \|_p \leq C \epsilon^{n-\alpha-1} \|A\|_{\infty} \|f\|_p
\end{equation}
for $\displaystyle f \in L^p(S)$,
where $A$ is given in Definition~\ref{pot_kerOk}.  If $\alpha = 0$,
then there exists a constant $C$ such that
\begin{equation} \label{pot_TK_Bound2}
\| T_K f \|_p \leq C \epsilon^{n-1} \left( \|A\|_{\infty}
(1 + |\ln(\epsilon)|) + \|B\|_\infty \right) \|f\|_p
\end{equation}
for $\displaystyle f \in L^p(S)$, where $A$ and $B$ are given in
Definition~\ref{pot_kerOk}.
\end{prop}

\begin{proof}
That $\displaystyle T_K : L^p(S) \rightarrow L^p(S)$ defines a bounded linear
operator follows from (\ref{pot_TK_Bound1}) and (\ref{pot_TK_Bound2}).

Let $\displaystyle \BB
= \left\{ \left(U_{\VEC{x}_j}, \psi_{\VEC{x}_j}\right) \right\}_{j\in J}$
be the finite atlas for $S$ defined in Section~\ref{pot_int_man}, and
let $\displaystyle \{ \phi_{\VEC{x}_j} \}_{j\in J}$ be a partition of unity
for $S$ subordinated to $\BB$.

\stage{$\mathbf{\alpha>0}$}
We have that
\begin{align*}
&\int_S \left| K(\VEC{x},\VEC{y}) \right| \dss{S}{y}
= \int_{S\cap B_\epsilon(\VEC{x})} \left| A(\VEC{x},\VEC{y}) \right| \,
\|\VEC{x}-\VEC{y}\|^{-\alpha} \dss{S}{y}
\leq \|A\|_\infty
\int_{S\cap B_\epsilon(\VEC{x})} \|\VEC{x}-\VEC{y}\|^{-\alpha} \dss{S}{y} \\ 
&\qquad = \|A\|_\infty \sum_{j\in J}
\int_{U_{\VEC{x}_j}\cap B_\epsilon(\VEC{x})} \phi_{\VEC{x}_j}(\VEC{y})
\left\|\VEC{x} - \VEC{y} \right\|^{-\alpha} \dss{S}{y}
\leq \|A\|_\infty \sum_{j\in J} \int_{U_{\VEC{x}_j}\cap B_\epsilon(\VEC{x})}
\left\|\VEC{x} - \VEC{y} \right\|^{-\alpha} \dss{S}{y} \\
&\qquad = \|A\|_\infty \sum_{j\in J}
\int_{\psi_{\VEC{x}_j}^{-1}(U_{\VEC{x}_j}\cap B_\epsilon(\VEC{x}))}
\left\| \VEC{x} - \psi_{\VEC{x}_j}^{-1}(\VEC{w}) \right\|^{-\alpha}
\, \left|\det \diff \psi_{\VEC{x}_j}^{-1} (\VEC{w}) \right| \dx{\VEC{w}} \\
&\qquad \leq \|A\|_\infty Q \sum_{j\in J}
\int_{\psi_{\VEC{x}_j}^{-1}(U_{\VEC{x}_j}\cap B_\epsilon(\VEC{x}))}
\left\| \VEC{x} - \psi_{\VEC{x}_j}^{-1}(\VEC{w}) \right\|^{-\alpha} \dx{\VEC{w}}
\\
&\qquad \leq \|A\|_\infty Q \sum_{j\in J}
\int_{\psi_{\VEC{x}_j}^{-1}(U_{\VEC{x}_j}\cap B_\epsilon(\VEC{x}))}
\left\| \psi_{\VEC{x}_j}(\VEC{x}) - \VEC{w} \right\|^{-\alpha} \dx{\VEC{w}} \\
&\qquad \leq \|A\|_\infty Q \sum_{j\in J} \int_{\partial B_1(\VEC{0})}
\int_0^{\epsilon} r^{-\alpha}\,r^{n-2} \dx{r} \dx{S}
= \frac{\|A\|_\infty Q\, \omega_{n-1} |J|\, \epsilon^{n-\alpha-1}}{n-\alpha-1}
< \infty
\end{align*}
for all $\VEC{x} \in S$, where $|J|$ is the number of elements in $J$.
Some of the inequalities in the relation above will benefit from
some explanations.  The second inequality is a consequence of
$0 \leq \phi_{\VEC{x}_j}(\VEC{y}) \leq 1$ for all
$\VEC{y}$ and all functions in the partition of unity.
The third inequality comes from (\ref{defnQdetpsim1}).  The fourth
inequality comes
$\displaystyle \|\VEC{x}-\VEC{y}\|
\geq \|\psi_{\VEC{x}_j}(\VEC{x}) - \psi_{\VEC{x}_j}(\VEC{y})\|$ for all
$\displaystyle \VEC{x},\VEC{y} \in U_{\VEC{x}_j}$.
The last inequality is true because
$\displaystyle \psi_{\VEC{x}_j}^{-1}(U_{\VEC{x}_j}\cap B_\epsilon(\VEC{x}))
\subset B_\epsilon(\psi_{\VEC{x}_j}^{-1}(\VEC{x})) \subset \RR^{n-1}$.
We have used the change of
variables $\VEC{w} = \psi_{\VEC{x}_j}(\VEC{x}) + r \VEC{z}$ with
$0 \leq r < \epsilon$ and
$\displaystyle \VEC{z} \in \partial B_1(\VEC{0}) \subset \RR^{n-1}$.

Similarly,
\[
\int_S | K(\VEC{x},\VEC{y}) | \dss{S}{x} \leq
\frac{\|A\|_\infty Q\, \omega_{n-1} |J|\, \epsilon^{n-\alpha-1}}{n-\alpha-1}
\]
for all $\VEC{y} \in S$.

Thus, (\ref{pot_TK_Bound1}) with
$\displaystyle C = \frac{Q\, \omega_{n-1} |J|}{n-\alpha-1}$
follows from the generalized Young's inequality,
Theorem~\ref{distr_GyoungI}.

\stage{$\alpha=0$}  Proceeding as we did for $\alpha > 0$, we find
that
\begin{align*}
&\int_S \left| K(\VEC{x},\VEC{y}) \right| \dss{S}{y}
= \int_{S \cap B_\epsilon(\VEC{x})}
\left| A(\VEC{x},\VEC{y}) \,\ln(\|\VEC{x}-\VEC{y}\|) +
B(\VEC{x},\VEC{y}) \right| \dss{S}{y} \\
&\qquad \leq \|A\|_\infty \int_{S \cap B_\epsilon(\VEC{x})}
\left|\ln(\|\VEC{x}-\VEC{y}\|) \right| \dss{S}{y} +
\|B\|_\infty \int_{S \cap B_\epsilon(\VEC{x})}\dss{S}{y} \\
&\qquad = \|A\|_\infty\sum_{j\in J}
\int_{U_{\VEC{x}_j}\cap B_\epsilon(\VEC{x})}
\phi_{\VEC{x}_j}(\VEC{y})
\left| \ln\left( \left\| \VEC{x} - \VEC{y} \right\|\right) \right|
\dss{S}{\VEC{y}} + \|B\|_\infty \sum_{j\in J}
\int_{U_{\VEC{x}_j}\cap B_\epsilon(\VEC{x})} \phi_{\VEC{x}_j}(\VEC{y})
\dss{S}{\VEC{y}} \\
&\qquad \leq \|A\|_\infty\sum_{j\in J}
\int_{U_{\VEC{x}_j}\cap B_\epsilon(\VEC{x})}
\left| \ln\left( \left\| \VEC{x} - \VEC{y} \right\|\right) \right|
\dss{S}{\VEC{y}} + \|B\|_\infty \sum_{j\in J}
\int_{U_{\VEC{x}_j}\cap B_\epsilon(\VEC{x})}\dss{S}{\VEC{y}} \\
&\qquad = \|A\|_\infty\sum_{j\in J}
\int_{\psi_{\VEC{x}_j}^{-1}(U_{\VEC{x}_j}\cap B_\epsilon(\VEC{x}))}
\left| \ln\left( \left\| \VEC{x} - \psi_{\VEC{x}_j}^{-1}(\VEC{w}) 
\right\|\right) \right|\, \left|\det \diff \psi_{\VEC{x}_j}^{-1}(\VEC{w}) \right|
\dx{\VEC{w}} \\
&\qquad \qquad + \|B\|_\infty \sum_{j\in J}
\int_{\psi_{\VEC{x}_j}^{-1}(U_{\VEC{x}_j}\cap B_\epsilon(\VEC{x}))}
\left|\det \diff \psi_{\VEC{x}_j}^{-1}(\VEC{w}) \right| \dx{\VEC{w}} \\
&\qquad \leq \|A\|_\infty Q \sum_{j\in J}
\int_{\psi_{\VEC{x}_j}^{-1}(U_{\VEC{x}_j}\cap B_\epsilon(\VEC{x}))}
\left| \ln\left( \left\| \VEC{x} - \psi_{\VEC{x}_j}^{-1}(\VEC{w}) 
\right\|\right) \right| \dx{\VEC{w}}
+ \|B\|_\infty Q \sum_{j\in J}
\int_{\psi_{\VEC{x}_j}^{-1}(U_{\VEC{x}_j}\cap B_\epsilon(\VEC{x}))} \dx{\VEC{w}} \\
&\qquad \leq \|A\|_\infty Q \sum_{j\in J}
\int_{\psi_{\VEC{x}_j}^{-1}(U_{\VEC{x}_j}\cap B_\epsilon(\VEC{x}))}
\left| \ln\left(
\left\| \psi_{\VEC{x}_j}(\VEC{x}) - \VEC{w} \right\|\right) \right|
\dx{\VEC{w}}
+ \|B\|_\infty Q \sum_{j\in J}
\int_{\psi_{\VEC{x}_j}^{-1}(U_{\VEC{x}_j}\cap B_\epsilon(\VEC{x}))} \dx{\VEC{w}} \\
&\qquad \leq \|A\|_\infty Q \sum_{j\in J} \int_{\partial B_1(\VEC{0})}
\int_0^{\epsilon} |\ln(r)| \,r^{n-2} \dx{r} \dx{S}
+ \|B\|_\infty Q \sum_{j\in J} \int_{\partial B_1(\VEC{0})}
\int_0^{\epsilon} r^{n-2} \dx{r} \dx{S} \\
&\qquad \leq \frac{\|A\|_\infty Q\, \omega_{n-1}
|J|\,\left( |\ln(\epsilon)| + 1 \right) \epsilon^{n-1}}{n-1}
+ \frac{\|B\|_\infty Q \omega_{n-1} |J|\, \epsilon^{n-1}}{n-1}
< \infty
\end{align*}
for all $\VEC{x} \in S$.  To obtain the fourth inequality, we have
the fact that the diameter of $V_{\VEC{x}_j}$ is less than $1$ to get
$\displaystyle
\ln\left(\left\| \VEC{x} - \psi_{\VEC{x}_j}^{-1}(\VEC{w}) \right\|\right)
\leq \ln\left(\left\| \psi_{\VEC{x}_j}(\VEC{x}) - \VEC{w} \right\|\right)$
for
$\displaystyle 
\left\| \psi_{\VEC{x}_j}(\VEC{x}) - \VEC{w} \right\|
\leq \left\| \VEC{x} - \psi_{\VEC{x}_j}^{-1}(\VEC{w}) \right\| \leq  1$.

Similarly,
\[
\int_S | K(\VEC{x},\VEC{y}) | \dss{S}{x}
\leq \frac{\|A\|_\infty Q\, \omega_{n-1}
|J|\,\left( |\ln(\epsilon)| + 1 \right) \epsilon^{n-1}}{n-1}
+ \frac{\|B\|_\infty Q \omega_{n-1} |J|\, \epsilon^{n-1}}{n-1}
< \infty
\]
for all $\VEC{y} \in S$.

Thus, (\ref{pot_TK_Bound2}) with $C= Q\, \omega_{n-1} |J|/(n-1)$
follows from the generalized Young's inequality,
Theorem~\ref{distr_GyoungI}.

Note that
\[
\int_0^{\epsilon} |\ln(r)| \,r^{n-2} \dx{r} 
= \frac{|\ln(\epsilon)|\epsilon^{n-1}}{n-1} +
\frac{\epsilon^{n-1}}{(n-1)^2}
\leq \frac{(|\ln(\epsilon)| + 1)\epsilon^{n-1}}{n-1}
\]
for $0 < \epsilon < 1$, and
\[
\int_0^{\epsilon} |\ln(r)| \,r^{n-2} \dx{r} 
= \frac{|\ln(\epsilon)|\epsilon^{n-1}}{n-1} -
\frac{\epsilon^{n-1}}{(n-1)^2} + \frac{2}{(n-1)^2}
\leq \frac{|\ln(\epsilon)|\epsilon^{n-1}}{n-1}
+ \frac{\epsilon^{n-1}}{(n-1)^2}
\leq \frac{(|\ln(\epsilon)| + 1)\epsilon^{n-1}}{n-1}
\]
for $\epsilon >1$.
\end{proof}

The proof of the previous proposition contains in some way the
following results.

\begin{prop} \label{pot_prop_KL1}
If $K$ is a kernel of order $0 \leq \alpha < n-1$, then there exists a
constant $C$ such that
$\displaystyle \int_S |K(\VEC{x},\VEC{y})| \dss{S}{y} < C$
for all $\VEC{x} \in S$.  In particular,  the function defined by
$\VEC{y} \mapsto K(\VEC{x},\VEC{y})$ for $\VEC{y} \in S$ 
is in $\displaystyle L^1(S)$ for all $\VEC{x}\in S$.
\end{prop}

\begin{proof}
We may assume that $0<\alpha < n$.  If $K$ is a kernel of order
$0$, we have
\begin{align*}
|K(\VEC{x},\VEC{y})| &\leq \|A\|_\infty \,
|\ln(\|\VEC{x}-\VEC{y}\|)| + \|B\|_\infty \\
&= \left( \|A\|_\infty \,\|\VEC{x}-\VEC{y}\|^\alpha
|\ln(\|\VEC{x}-\VEC{y}\|)| + \|B\|_\infty \|\VEC{x}-\VEC{y}\|^\alpha \right)
\|\VEC{x}-\VEC{y}\|^{-\alpha}
\leq D \|\VEC{x}-\VEC{y}\|^{-\alpha}
\end{align*}
for $\VEC{x},\VEC{y} \in S$ and $\alpha >0$, where
\[
D = \max_{\VEC{x},\VEC{y}\in S} \left( \|A\|_\infty
\,\|\VEC{x}-\VEC{y}\|^\alpha |\ln(\|\VEC{x}-\VEC{y}\|)| +
\|B\|_\infty \|\VEC{x}-\VEC{y}\|^\alpha \right) < \infty
\]
because $S$ is compact and
$\displaystyle \lim_{r\rightarrow 0} r^\alpha \ln(r)=0$ whatever $\alpha>0$.
Thus $K$ is a kernel of order $\alpha>0$.

Choose $\epsilon >0$ such that $\supp K \subset 
\{ (\VEC{x},\VEC{y}) \in S \times S : \|\VEC{x} - \VEC{y} < \epsilon \}$.
Note that we may not have $\epsilon <1$ but this is not necessary for
$\alpha >0$.

We can proceed exactly as in the case $\alpha >0$ of the proof of
Proposition~\ref{pot_lin_op_K} to show that
\[
\int_S \left| K(\VEC{x},\VEC{y}) \right| \dss{S}{y}
\leq C \equiv
\frac{\|A\|_\infty Q\, \omega_{n-1} |J|\, \epsilon^{n-\alpha-1}}{n-\alpha-1}
\]
for all $\VEC{x} \in S$.
\end{proof}

\begin{prop} \label{pot_K_LtwoComp}
If $K:S\times S \rightarrow \RR$ is a kernel of order $\alpha$ with
$0\leq \alpha < n -1$, then $\displaystyle T_K: L^2(S) \rightarrow L^2(S)$ is
compact.
\end{prop}

\begin{proof}
Given $0 < \epsilon <1$, defined
$\displaystyle K_\epsilon^{[o]}:S\times S \rightarrow \RR$
by
\[
K_\epsilon^{[o]}(\VEC{x},\VEC{y}) =
\begin{cases}
K(\VEC{x},\VEC{y}) & \quad \text{if} \quad 
\|\VEC{x}-\VEC{y}\| \geq \epsilon \\
0 & \quad \text{otherwise}
\end{cases}
\]
and $\displaystyle K_\epsilon^{[i]}:S\times S \rightarrow \RR$ by
$\displaystyle K_\epsilon^{[i]} = K - K_\epsilon^{[o]}$.  Choose $r>0$
such that $S \subset \overline{B_r(\VEC{0})}$.

\stage{i} $\displaystyle K_\epsilon^{[o]} \in L^2(S \times S)$ because
$\displaystyle K_\epsilon^{[o]}$ is bounded on the compact set $S \times S$.
More precisely, for $\alpha>0$, we have
$\displaystyle
\left|K_\epsilon^{[o]}(\VEC{x},\VEC{y})\right| =
\left| A(\VEC{x},\VEC{y}) \|\VEC{x} - \VEC{y} \|^{-\alpha} \right|
\leq \|A\|_{\infty} \epsilon^{-\alpha}$
for $\|\VEC{x}-\VEC{y}\| \geq \epsilon$ and
$\displaystyle \left|K_\epsilon^{[o]}(\VEC{x},\VEC{y})\right| = 0$
for $\|\VEC{x}-\VEC{y}\| < \epsilon$.
For $\alpha = 0$, we have
\[
\left|K_\epsilon^{[o]}(\VEC{x},\VEC{y})\right|
\leq
\left| A(\VEC{x},\VEC{y}) \ln(\|\VEC{x} - \VEC{y} \|) + B(\VEC{x},\VEC{y})
\right|
\leq \|A\|_{\infty} \max \{ |\ln(\epsilon)|, |\ln(2r)| \} +
\|B\|_\infty
\]
for $\|\VEC{x}-\VEC{y}\| \geq \epsilon$ and
$\displaystyle
\left|K_\epsilon^{[o]}(\VEC{x},\VEC{y})\right| = 0$
for $\|\VEC{x}-\VEC{y}\| < \epsilon$.

Thus $\displaystyle K_\epsilon^{[o]}$ is an Hilbert-Schmidt kernel according to
Theorem~\ref{fu_an_HSKern}.  This implies in
particular that $T_{K_\epsilon^{[o]}}:L^2(S) \rightarrow L^2(S)$ is compact.

\stage{ii} Since $\displaystyle K_\epsilon^{[i]}$ is a kernel of order $\alpha$,
it follows from Proposition~\ref{pot_lin_op_K} that
$\displaystyle
\|T_K - T_{K_\epsilon^{[o]}}\|_2 =
\|T_{K_\epsilon^{[i]}}\|_2 \leq C \|A\|_\infty \epsilon^{n-\alpha-1}$
for a constant $C$ when $\alpha>0$, and
$\displaystyle
\|T_K - T_{K_\epsilon^{[o]}}\|_2 =
\|T_{K_\epsilon^{[i]}}\|_2 \leq C \left( \|A\|_\infty
  (1+|\ln(\epsilon)|) + \|B\|_\infty \right) e^{n}$
for a constant $C$ when $\alpha = 0$.

Hence $\displaystyle T_{K_\epsilon^{[o]}} \rightarrow T$ as
$\epsilon \rightarrow 0$.
Since the operators $\displaystyle T_{K_\epsilon^{[o]}}$ are compact,
it follows from Proposition~\ref{fu_an_Kclose} that $T_K$ is compact.
\end{proof}

\begin{prop} \label{pot_Kc_fb}
If $K:S\times S \rightarrow \RR$ is a continuous kernel of order
$\alpha$ with $0\leq \alpha < n-1$ and $f:S\rightarrow \RR$ is
a bounded function, then $T_K f \in C(S)$.
\end{prop}

\begin{proof}
As we did at the beginning of the proof of Proposition~\ref{pot_prop_KL1},
we may assume that $0<\alpha < n$; namely, that $K$ is a kernel of
order $\alpha>0$.

Choose $\VEC{x}\in S$ and $\epsilon>0$.  We prove that $T_K f$ is
continuous at $\VEC{x}$.  For $r>0$ and $\VEC{w} \in S$, let
\[
S_{\VEC{w},r} = \{ \VEC{z}\in S : \|\VEC{w}-\VEC{z}\| < r \}
= S \cap B_r(\VEC{w}) \ .
\]
Given $\delta > 0$ and $\VEC{y} \in S_{\VEC{x},\delta}$, we have that
\begin{align*}
\left| (T_K f)(\VEC{x}) - (T_K f)(\VEC{y}) \right|
&= \left| \int_S \left(K(\VEC{x},\VEC{z}) - K(\VEC{y},\VEC{z})\right)
  f(\VEC{z}) \dss{S}{z} \right| \\
&\leq \int_S \left| K(\VEC{x},\VEC{z}) - K(\VEC{y},\VEC{z})\right|\,
 \left| f(\VEC{z}) \right| \dss{S}{z}
\leq I_1(\VEC{x},\VEC{y}) + I_2(\VEC{x},\VEC{y}) \ ,
\end{align*}
where
\begin{align*}
I_1(\VEC{x},\VEC{y}) &=
\int_{S_{\VEC{x},2\delta}} \big( \left| K(\VEC{x},\VEC{z})\right| +
\left|K(\VEC{y},\VEC{z})\right| \big)\, \left| f(\VEC{z})\right| \dss{S}{z}
\intertext{and}
I_2(\VEC{x},\VEC{y}) &= \int_{S\setminus S_{\VEC{x},2\delta}}
\left| K(\VEC{x},\VEC{z}) - K(\VEC{y},\VEC{z})\right|\,
\left| f(\VEC{z}) \right|\dss{S}{z} \ .
\end{align*}

\stage{i}
We have that
\[
I_1(\VEC{x},\VEC{y})
\leq \|A\|_\infty \|f\|_\infty \int_{S_{2\delta}} \left(
\|\VEC{x}-\VEC{z}\|^{-\alpha} + \|\VEC{y}-\VEC{z}\|^{-\alpha}\right)
\dss{S}{z} \ .
\]

If we assume that $3\delta/2$ is smaller than the Lebesgue number
associated to the atlas that we have introduced in
Section~\ref{pot_int_man}, then we get from (\ref{Sxepsint1}) that
\[
\int_{S_{\VEC{x},2\delta}} \|\VEC{x}-\VEC{z}\|^{-\alpha} \dss{S}{z}
\leq
\frac{ Q \omega_{n-1} (2\delta)^{n-\alpha-1}}{n-\alpha -1} \ .
\]
Moreover, since $S_{\VEC{x},2\delta} \subset S_{\VEC{y},3\delta}$, we
get that
\[
\int_{S_{\VEC{x},2\delta}} \|\VEC{y}-\VEC{z}\|^{-\alpha} \dss{S}{z}
\leq \int_{S_{\VEC{y},3\delta}}\|\VEC{y}-\VEC{z}\|^{-\alpha} \dss{S}{z}
\leq \frac{ Q \omega_{n-1} (3\delta)^{n-\alpha-1}}{n-\alpha -1} \ .
\]
Thus
\[
I_1(\VEC{x},\VEC{y}) \leq 
\|A\|_\infty \|f\|_\infty 
Q \omega_{n-1}\, \frac{2^{n-\alpha} + 3^{n-\alpha} }{n-\alpha-1} \,
\delta^{n-\alpha-1} \rightarrow 0 \quad \text{as} \quad
\delta\rightarrow 0 \ .
\]
Choose $\delta$ smaller if needed to get
$I_1(\VEC{x},\VEC{y})< \epsilon/2$ for $\VEC{y} \in S_{\VEC{x},\delta}$.

\stage{ii}
$K$ is continuous on the compact set
$\overline{S_{\VEC{x},\delta}} \times (S\setminus S_{\VEC{x},2\delta})$ because
$(\VEC{z},\VEC{z}) \not\in \overline{S_{\VEC{x},\delta}}
\times (S\setminus S_{\VEC{x},2\delta})$
for all $\VEC{z}\in S$.  Hence $K$ is uniformly continuous on
$\overline{S_{\VEC{x},\delta}} \times (S\setminus S_{\VEC{x},2\delta})$.
Choose a positive number $\delta_0 < \delta$ such that
\[
|K(\VEC{x},\VEC{z}) - K(\VEC{y},\VEC{z})| <
\frac{\epsilon}{2 \|f\|_\infty \,\int_{S\setminus S_{2\delta}} \dx{S}}
\]
for $\VEC{y} \in S_{\VEC{x},\delta_0}$ and
$\VEC{z} \in S\setminus S_{\VEC{x},2\delta}$.  Then,
\[
I_2(\VEC{x},\VEC{y}) = \int_{S\setminus S_{\VEC{x},2\delta}}
\left| K(\VEC{x},\VEC{z}) - K(\VEC{y},\VEC{z})\right|\,
\left| f(\VEC{z}) \right|\dss{S}{z}
\leq \|f\|_{\infty}
\left( \frac{\epsilon}{2 \|f\|_\infty \,\int_{S\setminus S_{2\delta}} \dx{S}}
\right) \int_{S\setminus S_{2\delta}}  \dx{S}
< \frac{\epsilon}{2}
\]
for $\VEC{y} \in S_{\VEC{x},\delta_0}$.

\stage{iii} Hence,
$\displaystyle \left| (T_K f)(\VEC{x}) - (T_K f)(\VEC{y}) \right| < \epsilon$
for $\VEC{y} \in S_{\VEC{x},\delta_0}$.  This proves that $T_K f$ is continuous
at an arbitrary point $\VEC{x}\in S$.
\end{proof}

\begin{prop} \label{pot_compt_cont_K}
Let $K:S\times S \rightarrow \RR$ be a continuous kernel of order
$\alpha$ with $0\leq \alpha < n-1$.  Given $\lambda \in \RR$, if
$\displaystyle f \in L^2(S)$ and $f + \lambda T_K f \in C(S)$,
 then $f \in C(S)$.
\end{prop}

\begin{proof}
Given $0 < \epsilon < 1$, choose $\psi \in \DD(\RR)$ such that
$\psi(x)=1$ for $x \in ]-\epsilon/2, \epsilon/2[$,
$\supp \psi \subset ]-\epsilon, \epsilon[$ and $0\leq \psi(x)\leq 1$
for all $x\in \RR$.  Let
$\phi(\VEC{x},\VEC{y}) = \psi\big(\|\VEC{x}-\VEC{y}\|\big)$ for
$\VEC{x}, \VEC{y} \in S$.  Moreover, let $K_1 = \phi K$ and
$K_2 = (1-\phi)K$.

\stage{i} We prove that $T_{K_2} f \in C(S)$.  Choose $\VEC{x}\in S$.
From Schwarz inequality, we have
\begin{align*}
\left| (T_{K_2} f)(\VEC{x}) - (T_{K_2} f)(\VEC{y})\right|
&\leq \int_S \left| K_2(\VEC{x},\VEC{z}) - K_2(\VEC{y},\VEC{z}) \right|\,
\left| f(\VEC{z}) \right| \dss{S}{z} \\
&\leq \|f\|_2 \, \left( \int_S \left|K_2(\VEC{x},\VEC{z}) -
K_2(\VEC{y},\VEC{z}) \right|^2 \dss{S}{z} \right)^{1/2}
\end{align*}
for $\VEC{y} \in S$.  Moreover, since $K_2 =0$ in an neighbourhood of
the diagonal $\{(\VEC{z},\VEC{z}) : \VEC{z} \in S\}$, we have the
$K_2$ is continuous on $S\times S$.

Let
$\displaystyle h_{\VEC{y}}(\VEC{z}) = \left|K_2(\VEC{x},\VEC{z}) -
K_2(\VEC{y},\VEC{z})\right|^2$ for $\VEC{z} \in S$, and define
$h:S \to \RR$ by $\displaystyle h(\VEC{z}) = 4\|K_2\|_\infty^2$
 for all $\VEC{z} \in S$.
We have that $h \in L(S)$ because $S$ is compact.

Since $h_{\VEC{y}}(\VEC{z}) \rightarrow 0$ as
$\VEC{y} \rightarrow \VEC{x}$ for every $\VEC{z} \in S$,
and $0\leq h_{\VEC{y}}(\VEC{z}) \leq h(\VEC{z})$ for all
$\VEC{z}, \VEC{y}\in S$, it follows from the Lebesgue Dominated
Convergence Theorem that
\begin{align*}
\left| (T_{K_2} f)(\VEC{x}) - (T_{K_2} f)(\VEC{y})\right|
&\leq \|f\|_2 \, \left( \int_S \left|K_2(\VEC{x},\VEC{z}) -
K_2(\VEC{y},\VEC{z}) \right|^2 \dss{S}{z} \right)^{1/2} \\
&= \|f\|_2 \, \left( \int_S h_y(\VEC{z}) \dss{S}{z} \right)^{1/2}
\rightarrow 0 \quad \text{as} \quad \VEC{y} \rightarrow \VEC{x} \ .
\end{align*}
Thus, $T_{K_2}f$ is continuous at the arbitrary point $\VEC{x} \in S$.

\stage{ii} We prove that $f \in C(S)$.
Let $g = (f+ \lambda T_K f) - \lambda T_{K_2} f$.  We have that $g \in C(S)$
by hypothesis and ({\bfseries i}).  Moreover, $g=f+ \lambda T_{K_1} f$ by
definition of $K_1$ and $K_2$.  It follows from
Proposition~\ref{pot_lin_op_K} that
\begin{align*}
\| \lambda T_{K_1} \|_p & \leq
\begin{cases}
\lambda C \epsilon^{n-\alpha-1} \|A\|_{\infty} & \quad \text{if} \quad
\alpha>0 \\
\lambda C \epsilon^{n-1} \left( \|A\|_{\infty}
\left( 1 + |\ln(\epsilon)|\right) + \|B\|_\infty \right) & \quad \text{if} \quad
\alpha=0
\end{cases}
\end{align*}
for $p=2$ and $p=\infty$,  By taking $\epsilon$ smaller if needed, we
then have that $\displaystyle \| \lambda T_{K_1} \|_p < 1$.
It follows from
the Banach Lemma that
$\displaystyle \Id + \lambda T_{K_1}:L^2(S) \rightarrow L^2(S)$ is
invertible.  More precisely,
$\displaystyle (\Id + \lambda T_{K_1})^{-1} = \sum_{j=0}^\infty
(-\lambda T_{K_1})^j$.
Hence $\displaystyle f = \sum_{j=0}^\infty (-\lambda T_{K_1})^j g$, where the
convergence is in $\displaystyle L^2(S)$.

Since $g$ is continuous on the compact set $S$, $g$ is bounded.  It
follows from Proposition~\ref{pot_Kc_fb} that
$(-\lambda T_{K_1})^j g\in C(S)$ for all $j$.  Moreover, the series
$\displaystyle \sum_{j=0}^\infty \|\lambda T_{K_1}\|_\infty^j$
converges because $\|\lambda T_{K_1}\|_\infty <1$.  Hence,
given $\delta>0$, you can choose $N$ large enough such that
\[
\left\| \sum_{j=n}^m (-\lambda T_{K_1})^j g \right\|_\infty \leq
\sum_{j=n}^m \|\lambda T_{K_1}\|_\infty^j \|g\|_\infty < \delta
\]
for $m\geq n >N$.  Proving that
$\displaystyle
\left\{ \sum_{j=0}^n (-\lambda T_{K_1})^j g \right\}_{n=0}^\infty$ is a
Cauchy sequence in $C(S)$ with respect to the uniform norm.  Thus, the
limit $f$ of the series is in $C(S)$.
\end{proof}

\begin{rmk}
The results in this section are also valid if $S$
is the closure of a bounded and open subset of $\RR^{n-1}$.
The proofs are almost identical.  They are even simpler because
the interior of the sets of the form
$\displaystyle S_{\VEC{x},\epsilon} = \left\{\VEC{y} \in S :
\|\VEC{y}-\VEC{x}\| < \epsilon \right\}$ are open subsets of $\RR^{n-1}$.
So, there is no need to refer to an atlas for $S$ and to use the
relations (\ref{Sxepsint1}) and (\ref{Sxepsint1}).  Moreover, there is
no restriction associated to the Lebesgue number of an atlas.
\end{rmk}

\subsection{Double Layer Potential}

As in the previous section, we set $S = \partial \Omega$.  In addition
to being a compact $\displaystyle C^2$-manifold of dimension $n-1$ in
$\displaystyle \RR^n$ with the topology on $S$ induced from the
topology on $\displaystyle \RR^n$, $S$ is a measurable space.

\begin{defn} \label{pot_dblp_def}
The {\bfseries double layer potential} with moment $\phi \in C(S)$ is
\[
u(\VEC{x}) = \int_S
\pdydx{N}{\nu_{\VEC{y}}}(\VEC{x},\VEC{y}) \phi(\VEC{y}) \dss{S}{y}
\quad , \quad \VEC{x} \in \RR^n \setminus S \ ,
\]
where $\displaystyle \pdydx{}{\nu_{\VEC{y}}}$ denotes the
directional derivative of $\VEC{y} \mapsto N(\VEC{x},\VEC{y})$ in the
direction of the outward unit normal $\nu(\VEC{y})$ to
$S$ at $\VEC{y} \in S$ and $N$ is the fundamental solution for
the Laplace operator defined in (\ref{laplace_NXY}) and
Theorem~\ref{laplace_fund_sol}.
\end{defn}

We have that
\[
\pdydx{N}{\nu_{\VEC{y}}}(\VEC{x},\VEC{y}) =
\graD_{\VEC{y}} N(\VEC{x},\VEC{y}) \cdot \nu(\VEC{y}) =
\frac{-1}{\omega_n \|\VEC{x}-\VEC{y}\|^n}\,(\VEC{x}-\VEC{y})
\cdot \nu(\VEC{y})
\]
for $\VEC{x} \neq \VEC{y}$.  Hence,
\[
\Delta_{\VEC{x}} \left(\pdydx{N}{\nu_{\VEC{y}}}(\VEC{x},\VEC{y})\right) =
\graD_{\VEC{y}} \left( \Delta_{\VEC{x}} N(\VEC{x},\VEC{y}) \right) \cdot
\nu(\VEC{y}) = 0
\]
for $\VEC{x} \neq \VEC{y}$ because
$\displaystyle N:\RR^n \setminus \{ \VEC{0} \} \to \RR$ is harmonic.
Thus
$\displaystyle \VEC{x} \mapsto \pdydx{N}{\nu_{\VEC{y}}}(\VEC{x},\VEC{y})$
is harmonic on $\displaystyle \RR^n\setminus \{\VEC{y}\}$ for all
$\displaystyle \VEC{y} \in \RR^n$.  Moreover, since 
$\displaystyle (\VEC{x},\VEC{y}) \mapsto
\diff^\alpha_{\VEC{x}}
\left(\pdydx{N}{\nu_{\VEC{y}}}(\VEC{x},\VEC{y})\right)$
is continuous on $\displaystyle (\RR^n\setminus S)\times S$ for all
multi-index $\alpha$, we may interchange the integral with respect to
$\VEC{y}$ in the definition of $u$ with the operator
$\Delta_{\VEC{x}}$ to show that $u$ is harmonic on
$\displaystyle \RR^n \setminus S$. 

Since
$\displaystyle 
\pdydx{N}{\nu_{\VEC{y}}}(\VEC{x},\VEC{y}) = O(\|\VEC{x}\|^{1-n})$
as $\|\VEC{x}\| \rightarrow \infty$ uniformly for $\VEC{y}$ on the
compact set $S$ \footnote{
Let $\displaystyle M = \max_{\VEC{y} \in S} \|\VEC{y}\|$.
For $\|\VEC{x}\| \geq M$, we have that
$\|\VEC{x}-\VEC{y}\| \geq \|\VEC{x}\| - \|\VEC{y}\| \geq \|\VEC{x}\| - M$
for all $\VEC{y} \in S$.  Thus, for all $\VEC{y} \in S$,
we have that $\|\VEC{x} - \VEC{y}\|^{1-n}
\leq (\|\VEC{x}\| - M)^{1-n} < C \|\VEC{x}\|^{1-n}$ 
for a constant $C$ and $\|\VEC{x}\| > M$.}, we have that
$\displaystyle u(\VEC{x}) = O(\|\VEC{x}\|^{1-n})$
as $\|\VEC{x}\| \rightarrow \infty$.
It follows from Proposition~\ref{pot_infty_u} that $u$ is harmonic at
infinity.

Let
\begin{equation} \label{pot_K_defn}
K(\VEC{x},\VEC{y}) =
\begin{cases}
\displaystyle \pdydx{N}{\nu_{\VEC{y}}}(\VEC{x},\VEC{y}) &
\quad \text{if} \ \VEC{x},\VEC{y} \in S \ \text{and} \ \VEC{x} \neq \VEC{y} \\
\text{anything} & \quad \text{if} \ \VEC{x}=\VEC{y} \in S
\end{cases}
\end{equation}
This function will be useful later to study the double layer
potential on $S$.  Before studying the double layer potential, we need
some results about $K$ and $\displaystyle \pdydx{N}{\nu_{\VEC{y}}}$.

\begin{lemma} \label{pot_lem_dbl1}
There exists $C>0$ such that
$\displaystyle \left| (\VEC{x}-\VEC{y}) \cdot \nu(\VEC{x}) \right| \leq C
\|\VEC{x}-\VEC{y}\|^2$ for all $\VEC{x},\VEC{y} \in S$.
\end{lemma}

\begin{proof}
\stage{i}
Let $\displaystyle \BB = \left\{ \left(U_{\VEC{x}_j},
\psi_{\VEC{x}_j}\right) \right \}_{j=1}^J$ be the atlas for
$S$ described in Section~\ref{pot_int_man}.
Choose $\epsilon/2$ smaller or equal to the Lebesgue number associated
to the open cover $\displaystyle \{ V_{\VEC{x}_j}\}_{j=1}^J$ of $S$.  Thus,
for each $\VEC{x} \in S$, there exists an index $j$ such that
$\displaystyle
S_{\VEC{x},\epsilon} = B_{\epsilon}(\VEC{x}) \cap S \subset U_{\VEC{x}_j}$.

Choose $C_0$ such that $C_0\,\epsilon > 1$.  Since
$\nu(\VEC{x})$ is of norm one for all $\VEC{x}$, we get from
Schwarz inequality that
\[
\left| (\VEC{x}-\VEC{y}) \cdot \nu(\VEC{x}) \right|
\leq \|\VEC{x}-\VEC{y}\|
\leq C_0\,\epsilon \|\VEC{x}-\VEC{y}\|
\leq C_0 \|\VEC{x}-\VEC{y}\|^2
\]
for all $\VEC{x}, \VEC{y} \in S$ such that
$\|\VEC{x}-\VEC{y}\| \geq \epsilon$.
Thus, we only have to consider $\|\VEC{x}-\VEC{y}\| < \epsilon$.

\stage{ii}
Given $\VEC{x} \in S$.  We have that
$\displaystyle S_{\VEC{x},\epsilon} \subset U_{\VEC{x}_j}$ for some $j \in J$.
Since we assume that
$\displaystyle \psi_{\VEC{x}_j}^{-1}$ is of class $C^2$ on the compact set
$\displaystyle \overline{W_j}$, we
have that $\displaystyle C_j = \max_{\VEC{w} \in \overline{W_j}}
\left\| \diff^2 \psi_j^{-1}(\VEC{w}) \right\| < \infty$
\footnote{For a bounded bilinear mapping $B:\RR^n \times \RR^n \to \RR^m$,
$\displaystyle \|B\| = \max_{\|\VEC{x}\|= \|\VEC{y}\|=1}
\|B(\VEC{x},\VEC{y})\|$.
Hence, $\|B(\VEC{x},\VEC{y})\| \leq \|B\| \, \|\VEC{x}\|\, \|\VEC{y}\|$
for all $\VEC{x}, \VEC{y} \in \RR^n$.}.

Let $\displaystyle \breve{\VEC{x}} = \psi_{\VEC{x}_j}(\VEC{x})$ and, given
$\displaystyle \VEC{y} \in S_{\VEC{x},\epsilon}$,
let $\displaystyle \breve{\VEC{y}} = \psi_{\VEC{x}_j}(\VEC{y})$.
Using the Taylor expansion of $\psi_{\VEC{x}_j}^{-1}$ about
$\breve{\VEC{x}}$, we get
\[
\left|(\VEC{y}- \VEC{x})\cdot \nu(\VEC{x})\right|
= \left|\left( \diff \psi_{\VEC{x}_j}^{-1}(\breve{\VEC{x}})
    \left(\breve{\VEC{y}}- \breve{\VEC{x}}\right)
+ \diff^2 \psi_{\VEC{x}_j}^{-1}(\VEC{w})
\left(\breve{\VEC{y}}- \breve{\VEC{x}},\breve{\VEC{y}}- \breve{\VEC{x}}\right)
\right) \cdot \nu(\VEC{x})\right|
\]
for some $\displaystyle \VEC{w} \in W_{\VEC{x}_j}$.  However,
\begin{align*}
&\left( \diff \psi_{\VEC{x}_j}^{-1}(\breve{\VEC{x}})
\left(\breve{\VEC{y}}- \breve{\VEC{x}} \right)\right) \cdot
\nu(\VEC{x}) \\
&\qquad = \begin{pmatrix}
\displaystyle
\pdydx{\psi^{-1}_{\VEC{x}_j}}{w_1}(\breve{\VEC{x}}) \cdot \nu(\VEC{x}) &
\displaystyle
\pdydx{\psi^{-1}_{\VEC{x}_j}}{w_2}(\breve{\VEC{x}}) \cdot \nu(\VEC{x}) &
\ldots &
\displaystyle
\pdydx{\psi^{-1}_{\VEC{x}_j}}{w_{n-1}}(\breve{\VEC{x}}) \cdot \nu(\VEC{x})
\end{pmatrix} \cdot
\left(\breve{\VEC{y}}- \breve{\VEC{x}} \right) = 0
\end{align*}
because $\displaystyle \pdydx{\psi^{-1}_{\VEC{x}_j}}{w_k}(\breve{\VEC{x}})$
is in the tangent space of $S$ at $\VEC{x}$ and so orthogonal to
$\nu(\VEC{x})$ for all $k$.  Thus
\[
\left|(\VEC{y}- \VEC{x})\cdot \nu(\VEC{x})\right|
= \left|\left(\diff^2 \psi_{\VEC{x}_j}^{-1}(\VEC{w})
\left(\breve{\VEC{y}}- \breve{\VEC{x}},\breve{\VEC{y}}- \breve{\VEC{x}}\right)
\right) \cdot \nu(\VEC{x})\right|
\leq C_j \|\breve{\VEC{y}}- \breve{\VEC{x}}\|^2
\leq C_j \|\VEC{y}- \VEC{x}\|^2
\]
because $\|\nu(\VEC{x})\| = 1$, where we have used the fact that
$\displaystyle \|\breve{\VEC{y}}- \breve{\VEC{x}}\|
=\| \psi_{\VEC{x}_j}(\VEC{y})- \psi_{\VEC{x}_j}(\VEC{x})\|
\leq \| \VEC{y}- \VEC{x}\|$ since
$\displaystyle \psi_{\VEC{x}_j}$ is the projection of
$\displaystyle U_{\VEC{x}_j}$ along the vector
$\displaystyle \nu(\VEC{x}_j)$ onto
$\displaystyle W_{\VEC{x}_j}$.

\stage{iii} We get the conclusion of the lemma with
$\displaystyle C = \max_{0\leq j \leq J} C_j$.
\end{proof}

\begin{lemma} \label{pot_lem_dbl2}
$K: S \times S \rightarrow \RR$ defined in
(\ref{pot_K_defn}) is a continuous kernel of order $n-2$.
\end{lemma}

\begin{proof}
We have that $\displaystyle K(\VEC{x},\VEC{y}) = A(\VEC{x},\VEC{y})
\|\VEC{x}- \VEC{y}\|^{2-n}$, where
\[
A(\VEC{x},\VEC{y}) = \frac{-1}{\omega_n \|\VEC{x}-\VEC{y}\|^2}\,
\left(\VEC{x}-\VEC{y}\right)\cdot \nu(\VEC{y}) \  .
\]
From Lemma~\ref{pot_lem_dbl1}, there exists a constant $C$ such that
$\displaystyle \left| A(\VEC{x},\VEC{y})\right| \leq C/\omega_n$ for
all $\VEC{x},\VEC{y} \in S$ with $\VEC{x} \neq \VEC{y}$.
So we may expand $A$ to a bounded and measurable function
on $S \times S$.  Note that $\|(\VEC{x},\VEC{x}) : \VEC{x} \in S\}$
is a set of measure zero.  This shows that $K$ is a kernel of order
$n-2$.

By definition, $K$ is continuous at
$\{ (\VEC{x},\VEC{y}) \in S \times S : \VEC{x} \neq \VEC{y}\}$.
\end{proof}

\begin{lemma} \label{pot_lem_dbl3}
We have that
\begin{equation} \label{pot_lem_dbl3E1}
\int_S K(\VEC{x},\VEC{y}) \dss{S}{y} = \frac{1}{2} \quad , \quad
\VEC{x} \in S \ ,
\end{equation}
and
\begin{equation} \label{pot_lem_dbl3E2}
\int_S \pdydx{N}{\nu_{\VEC{y}}}(\VEC{x},\VEC{y}) \dss{S}{y} =
\begin{cases}
1 & \quad \text{if} \ \VEC{x} \in \Omega \\
0 & \quad \text{if} \ \VEC{x} \in \Omega^{\prime}
\end{cases}
\end{equation}
\end{lemma}

\begin{proof}
\stage{i} We first prove (\ref{pot_lem_dbl3E2}).
Given $\displaystyle \VEC{x} \in \Omega^{\prime}$, we have that
$w(\VEC{y}) = N(\VEC{x},\VEC{y})$ for $\VEC{y} \in \overline{\Omega}$
is harmonic on $\Omega$ and
$\displaystyle w \in C^2(\Omega) \cap C^1(\overline{\Omega})$
It follows from the Green's Identity (\ref{laplace_green1}) with
$v = 1$ and $u= w$ that
\[
\int_S \pdydx{N}{\nu_{\VEC{y}}}(\VEC{x},\VEC{y}) \dss{S}{y} =
\int_{\Omega} \Delta_{\VEC{y}} N(\VEC{x},\VEC{y}) \dx{\VEC{y}} = 0
\]
for $\displaystyle \VEC{x} \in \Omega^{\prime}$.

Given $\VEC{x} \in \Omega$, choose $\epsilon >0$ small enough to have
$\overline{B_\epsilon(\VEC{x})} \subset \Omega$.
Let $\displaystyle \Omega_\epsilon = \Omega \setminus
\overline{B_\epsilon(\VEC{x})}$.  We have that
$w(\VEC{y}) = N(\VEC{x},\VEC{y})$ for $\VEC{y} \in \overline{\Omega_\epsilon}$
is harmonic on $\Omega_\epsilon$ and
$\displaystyle w \in C^2\left(\Omega_\epsilon\right) \cap
C^1\left(\overline{\Omega_\epsilon}\right)$.
It follows from the Green's Identity (\ref{laplace_green1}) with
$v = 1$ and $u = w$ that
\[
\int_S \pdydx{N}{\nu_{\VEC{y}}}(\VEC{x},\VEC{y}) \dss{S}{y}
- \int_{\partial B_\epsilon(\VEC{x})}
\pdydx{N}{\nu_{\VEC{y}}}(\VEC{x},\VEC{y}) \dss{S}{y}
=
\int_{\Omega_\epsilon} \Delta_{\VEC{y}} N(\VEC{x},\VEC{y}) \dx{\VEC{y}} = 0 \ .
\]
However,
\[
\int_{\partial B_\epsilon(\VEC{x})}
\pdydx{N}{\nu_{\VEC{y}}}(\VEC{x},\VEC{y}) \dss{S}{y}
= \frac{1}{\omega_n \epsilon^{n-1}}
\int_{\|\VEC{y}\|_2=\epsilon} \dss{S}{y} = 1
\]
because $\VEC{x}-\VEC{y} = \epsilon \nu(\VEC{y})$ and
$\|\VEC{x}-\VEC{y}\| = \epsilon$.  Thus,
$\displaystyle
\int_S \pdydx{N}{\nu_{\VEC{y}}}(\VEC{x},\VEC{y}) \dss{S}{y} = 1$
for $\VEC{x} \in \Omega$.

\stage{ii} We now prove (\ref{pot_lem_dbl3E1}).
Given $\VEC{x} \in S$, let
$A_\epsilon = S \setminus \overline{B_\epsilon(\VEC{x})}$,
$B_\epsilon = \Omega \setminus \overline{B_\epsilon(\VEC{x})}$,
$C_\epsilon = \Omega \cap \partial B_\epsilon(\VEC{x})$ and
$\displaystyle D_\epsilon = \left\{\VEC{y} \in \partial B_\epsilon(\VEC{x}) :
  (\VEC{y}- \VEC{x})\cdot \nu(\VEC{x}) < 0 \right\}$
(Figure~\ref{dblLayerKN}).

Since the function defined by $w(\VEC{y}) = N(\VEC{x},\VEC{y})$ for
$\VEC{y} \in \overline{B_\epsilon}$ is
harmonic in $B_\epsilon$ and
$\displaystyle w \in C^2(B_\epsilon)\cap C^2\left(\overline{B_\epsilon}\right)$,
we may use the Green's Identity (\ref{laplace_green1}) with
$v = 1$ and $u = w$ to get
\[
\int_{A_\epsilon} K(\VEC{x},\VEC{y}) \dss{S}{y}
-\int_{C_\epsilon} \pdydx{N}{\nu_\VEC{y}}(\VEC{x},\VEC{y}) \dss{S}{y}
= \int_{B_\epsilon} \Delta_{\VEC{y}} N(\VEC{x},\VEC{y}) \dx{\VEC{y}} = 0 \ .
\]

Since $\VEC{y} \to K(\VEC{x},\VEC{y})$ defines a function in
$\displaystyle L^1(S)$
according to Proposition~\ref{pot_prop_KL1}, we may use the Lebesgue
Dominate Convergence theorem to conclude that
\[
\int_S  K(\VEC{x},\VEC{y}) \dss{S}{y}
= \lim_{\epsilon \to 0} \int_{A_\epsilon}  K(\VEC{x},\VEC{y}) \dss{S}{y} \ .
\]
Thus
\begin{equation} \label{pot_lem_dbl3E7}
\int_S  K(\VEC{x},\VEC{y}) \dss{S}{y}
= \lim_{\epsilon \to 0}
\int_{C_\epsilon} \pdydx{N}{\nu_\VEC{y}}(\VEC{x},\VEC{y}) \dss{S}{y} \ .
\end{equation}
We have that
\begin{align}
\int_{C_\epsilon} \pdydx{N}{\nu_\VEC{y}}(\VEC{x},\VEC{y}) \dss{S}{y}
&= \frac{1}{\omega_n \epsilon^{n-1}} \int_{C_\epsilon} \dss{S}{y} \nonumber \\
&= \frac{1}{\omega_n \epsilon^{n-1}} \left( \int_{D_\epsilon} \dss{S}{y}
- \int_{D_\epsilon \setminus C_\epsilon} \dss{S}{y}
+ \int_{C_\epsilon \setminus D_\epsilon} \dss{S}{y} \right) \ .
\label{pot_lem_dbl3E3} 
\end{align}
Moreover,
\begin{equation} \label{pot_lem_dbl3E4}
\frac{1}{\omega_n \epsilon^{n-1}} \int_{D_\epsilon} \dss{S}{y}
= \frac{1}{2 \omega_n \epsilon^{n-1}}
\int_{\partial B_\epsilon(\VEC{x})} \dss{S}{y} = \frac{1}{2} \ .
\end{equation}

Let
$\displaystyle
M_\epsilon = \max_{\VEC{y}\in S \cap B_\epsilon(\VEC{x})} \left|
  \left( \VEC{y} - \VEC{x}\right) \cdot \nu(\VEC{x}) \right|$
and
$Q_\epsilon = \left\{ \VEC{y} \in \partial B_\epsilon(\VEC{x}) :
\left| (\VEC{y} - \VEC{x}) \cdot \nu(\VEC{x}) \right| \leq
M_\epsilon \right\}$.
Since $D_\epsilon \setminus C_\epsilon \subset Q_\epsilon$, we get
that the portion $A_\epsilon$ of the sphere
$\partial B_\epsilon(\VEC{x})$ covered by $D_\epsilon \setminus C_\epsilon$ 
is smaller than the portion
$\displaystyle 4 \arctan(M_\epsilon/\sqrt{\epsilon^2 - M_\epsilon^2})$
of the sphere $\partial B_\epsilon(\VEC{x})$ covered by $Q_\epsilon$. 
However, we have from Lemma~\ref{pot_lem_dbl1} that
$M_\epsilon \leq C \epsilon^2$ for some constant $C$.  Thus,
$A_\epsilon \leq 4 \arctan(M_\epsilon/\sqrt{\epsilon^2 - M_\epsilon^2}) \to 0$ 
as $\epsilon \to 0$.  It follows that
\begin{equation} \label{pot_lem_dbl3E5}
\frac{1}{\omega_n \epsilon^{n-1}} \int_{D_\epsilon \setminus C_\epsilon} \dss{S}{y}
= A_\epsilon \to 0 \quad \text{as} \quad \epsilon \to 0 \ .
\end{equation}
Similarly,
\begin{equation} \label{pot_lem_dbl3E6}
\frac{1}{\omega_n \epsilon^{n-1}} \int_{C_\epsilon \setminus D_\epsilon} \dss{S}{y}
\to 0 \quad \text{as} \quad \epsilon \to 0 \ .
\end{equation}
If we let $\epsilon$ converge to $0$ in (\ref{pot_lem_dbl3E3}), and
use (\ref{pot_lem_dbl3E4}), (\ref{pot_lem_dbl3E5}) and
(\ref{pot_lem_dbl3E6}), we get (\ref{pot_lem_dbl3E1}) from
(\ref{pot_lem_dbl3E7}).
\end{proof}

\pdfF{potential/dbl_layer_KN}{Figure for the proof of Lemma~\ref{pot_lem_dbl3}}
{Figure for the proof of Lemma~\ref{pot_lem_dbl3}.  The dashed blue
line represents $C_\epsilon$.  The dashed black line represents
$D_\epsilon$.  The tick black line represents $A_\epsilon$.}{dblLayerKN}

\begin{rmk}
Looking at the proof of the previous lemma, it is clear that
(\ref{pot_lem_dbl3E1}) and (\ref{pot_lem_dbl3E2}) are still true if
$S$ is replaced by $\partial \Omega_j$ and $\Omega$ by $\Omega_j$,
where $\Omega_j$ is one of the components of $\Omega$.  For
(\ref{pot_lem_dbl3E2}), the set $\Omega^{\prime}$ is then replaced by
the complement of $\overline{\Omega_j}$.     \label{pot_lem_dbl3_rmk}
\end{rmk}

\begin{lemma} \label{pot_lem_dbl4}
There exists $B \in ]0,\infty[$ such that 
\[
\int_S \left| \pdydx{N}{\nu_{\VEC{y}}}(\VEC{x},\VEC{y})
\right| \dss{S}{y} \leq B
\]
for all $\VEC{x} \in \RR^n\setminus S$.
\end{lemma}

\begin{proof}
Given $\displaystyle \VEC{x} \in \RR^n\setminus S$, let
$\displaystyle d_{\VEC{x}} = \min_{\VEC{y} \in S}\|\VEC{y} - \VEC{x}\|$.
Since $S$ is compact, $\displaystyle d_{\VEC{x}} >0$.

Choose $\delta >0$ such that $\delta < 1/(4C)$ for $C$ given in
Lemma~\ref{pot_lem_dbl1} and such that
\[
O_{\delta} = \left\{ \VEC{y} + t\, \nu(\VEC{y}) : \VEC{y} \in S \ ,
\ |t|<\delta \right\}
\]
is a tubular neighbourhood of $S$.  Such a $\delta$ exists according
to Theorem~\ref{pot_TBN}.

\stage{i} If $\displaystyle d_{\VEC{x}} \geq \delta$, then
\[
\left| \pdydx{N}{\nu_{\VEC{y}}}(\VEC{x},\VEC{y})\right| 
= \left| \frac{-1}{\omega_n \|\VEC{x}-\VEC{y}\|^n}\,(\VEC{x}-\VEC{y})
\cdot \nu(\VEC{y}) \right|
\leq \frac{1}{\omega_n \|\VEC{x}-\VEC{y}\|^{n-1}}
\leq \frac{1}{\omega_n d_{\VEC{x}}^{n-1}}
\leq \frac{1}{\omega_n \delta^{n-1}} \ .
\]
Thus,
\begin{equation} \label{Bounddny1}
\int_S \left| \pdydx{N}{\nu_{\VEC{y}}}(\VEC{x},\VEC{y})
\right| \dss{S}{y} \leq \frac{1}{\omega_n \delta^{n-1}} \int_S \dx{S} \ .
\end{equation}

\stage{ii}  If $\displaystyle d_{\VEC{x}} < \delta$, then
$\VEC{x} \in O_{\delta}$.  So, there exist a unique
$\VEC{x}_o \in S$ and $t_o \in ]-\delta,\delta[$ such
that
$\displaystyle \VEC{x} = \VEC{x}_o + t_o\nu(\VEC{x}_o)$.
Let $\displaystyle S_{\VEC{x}_o,2\delta} = 
\left\{ \VEC{y} \in S : \|\VEC{y} - \VEC{x}_0\| < 2\delta \right\}$.

We have that
\[
\int_S \left| \pdydx{N}{\nu_{\VEC{y}}}(\VEC{x},\VEC{y}) \right|
\dss{S}{y}
= \int_{S_{\VEC{x}_o,2\delta}}
\left| \pdydx{N}{\nu_{\VEC{y}}}(\VEC{x},\VEC{y}) \right| \dss{S}{y}
+ \int_{S\setminus S_{\VEC{x}_o,2\delta}}
\left| \pdydx{N}{\nu_{\VEC{y}}}(\VEC{x},\VEC{y}) \right|
\dss{S}{y} \ .
\]

For $\VEC{y} \in S \setminus S_{\VEC{x}_o,2\delta}$, we have that
\[
\|\VEC{x} - \VEC{y} \|
\geq \|\VEC{x}_o-\VEC{y}\| - \|\VEC{x}-\VEC{x}_o\|
\geq 2\delta - \delta = \delta \ .
\]
Thus
\[
\left| \pdydx{N}{\nu_{\VEC{y}}}(\VEC{x},\VEC{y})\right| 
= \left| \frac{-1}{\omega_n \|\VEC{x}-\VEC{y}\|^n}\,(\VEC{x}-\VEC{y})
\cdot \nu(\VEC{y}) \right|
\leq \frac{1}{\omega_n \|\VEC{x}-\VEC{y}\|^{n-1}}
\leq \frac{1}{\omega_n \delta^{n-1}}
\]
and so
\begin{equation}  \label{Bounddny2}
\int_{S\setminus S_{\VEC{x}_o,2\delta}}
\left| \pdydx{N}{\nu_{\VEC{y}}}(\VEC{x},\VEC{y}) \right| \dss{S}{y}
\leq \frac{1}{\omega_n \delta^{n-1}} \int_S \dx{S} \ .
\end{equation}

For $\VEC{y} \in S_{\VEC{x}_o,2\delta}$, we get from
Lemma~\ref{pot_lem_dbl1} that
\begin{align}
\left| \pdydx{N}{\nu_{\VEC{y}}}(\VEC{x},\VEC{y})\right| 
&= \left| \frac{-1}{\omega_n \|\VEC{x}-\VEC{y}\|^n}\,(\VEC{x}-\VEC{y})
\cdot \nu(\VEC{y}) \right|
\leq \frac{1}{\omega_n \|\VEC{x}-\VEC{y}\|^n}\,\big(
\left|(\VEC{x}-\VEC{x}_o) \cdot \nu(\VEC{y}) \right|
+ \left| (\VEC{x}_o-\VEC{y}) \cdot \nu(\VEC{y}) \right| \big)
\nonumber \\
&\leq \frac{1}{\omega_n \|\VEC{x}-\VEC{y}\|^n}\,\left(
\left\|\VEC{x}-\VEC{x}_o) \right\|
+ C \left\| \VEC{x}_o-\VEC{y}) \right\|^2 \right)
\label{Bounddny3}
\end{align}
because $\VEC{x}_o, \VEC{y} \in S$.  We now find a lower bound for
$\displaystyle \|\VEC{x}-\VEC{y}\|^n$ with $\VEC{y} \in S_{\VEC{x}_o,2\delta}$.
We have that
\[
\|\VEC{x}-\VEC{y}\|^2 = \| (\VEC{x}-\VEC{x}_o) + (\VEC{x}_o-\VEC{y})\|^2
= \|\VEC{x}-\VEC{x}_o\|^2 + 2 (\VEC{x}-\VEC{x}_o) \cdot (\VEC{x}_o-\VEC{y})
+ \|\VEC{x}_o-\VEC{y}\|^2 \ .
\]
We also have that
\begin{align*}
2\left| (\VEC{x} - \VEC{x}_o) \cdot (\VEC{x}_o - \VEC{y}) \right|
&= 2\left|\, \|\VEC{x} - \VEC{x}_o\| \nu(\VEC{x}_o)
\cdot (\VEC{x}_o - \VEC{y}) \right|
= 2\|\VEC{x} - \VEC{x}_o\|\, \left|\nu(\VEC{x}_o)
\cdot (\VEC{x}_o - \VEC{y}) \right| \\
&\leq 2 C \|\VEC{x} - \VEC{x}_o\| \, \|\VEC{x}_o - \VEC{y}\|^2 \ ,
\end{align*}
where again we have used Lemma~\ref{pot_lem_dbl1} with
$\VEC{x}_o, \VEC{y} \in S$.  Since
$\|\VEC{x}_o - \VEC{y}\| < 2\delta < 1/(2C)$, we get that
\[
2\left| (\VEC{x} - \VEC{x}_o) \cdot (\VEC{x}_o - \VEC{y}) \right|
\leq \|\VEC{x} - \VEC{x}_o\| \, \|\VEC{x}_o - \VEC{y}\| \ .
\]
Hence,
\[
\|\VEC{x}-\VEC{y}\|^2 \geq \|\VEC{x}-\VEC{x}_o\|^2
- \|\VEC{x} - \VEC{x}_o\| \, \|\VEC{x}_o - \VEC{y}\| + \|\VEC{x}_o-\VEC{y}\|^2
\geq \frac{1}{2} \left( \|\VEC{x}-\VEC{x}_o\|^2
+ \|\VEC{x}_o-\VEC{y}\|^2 \right) \ .
\]
If we substitute this expression in (\ref{Bounddny3}), we get
\begin{align*}
\left| \pdydx{N}{\nu_{\VEC{y}}}(\VEC{x},\VEC{y})\right| 
&\leq \frac{2^{n/2}}{\omega_n \left(\|\VEC{x}-\VEC{x}_o\|^2 +
    \|\VEC{x}_o-\VEC{y}\|^2\right)^{n/2}}\,\left(
\left\|\VEC{x}-\VEC{x}_o) \right\|
  + C \left\| \VEC{x}_o-\VEC{y}) \right\|^2 \right) \\
&\leq \frac{2^{n/2} \left\|\VEC{x}-\VEC{x}_o\right\|}
{\omega_n \left(\|\VEC{x}-\VEC{x}_o\|^2 +
\|\VEC{x}_o-\VEC{y}\|^2\right)^{n/2}}
+ \frac{2^{n/2} C}{\omega_n \|\VEC{x}_o-\VEC{y}\|^{n-2}} \ .
\end{align*}
Therefore,
\begin{equation} \label{Bounddny4}
\begin{split}
\int_{S_{\VEC{x}_o,2\delta}}
\left| \pdydx{N}{\nu_{\VEC{y}}}(\VEC{x},\VEC{y}) \right| \dss{S}{y}
&\leq \frac{2^{n/2}}{\omega_n} \int_{S_{\VEC{x}_o,2\delta}}
\frac{\left\|\VEC{x}-\VEC{x}_o\right\|}
{\left(\|\VEC{x}-\VEC{x}_o\|^2 + \|\VEC{x}_o-\VEC{y}\|^2\right)^{n/2}}
\dss{S}{y} \\
&\qquad + \frac{2^{n/2} C}{\omega_n} \int_{S_{\VEC{x}_o,2\delta}}
\frac{1}{\|\VEC{x}_o-\VEC{y}\|^{n-2}} \dss{S}{y} \ .
\end{split}
\end{equation}
If we assume that $4\delta$ is smaller than the Lebesgue number
associated to the atlas that we have introduced in
Section~\ref{pot_int_man}, then we get from (\ref{Sxepsint1}) with
$\alpha = n-2$ that
\begin{equation} \label{Bounddny5}
\int_{S_{\VEC{x}_o,2\delta}} \frac{1}{\|\VEC{x}_o-\VEC{y}\|^{n-2}} \dss{S}{y}
\leq 2 Q \omega_{n-1} \delta \ .
\end{equation}
To find an upper bound for the first integral on the right side of
(\ref{Bounddny4}), we use the fact that
$S_{\VEC{x}_o,2\delta} \subset U_{\VEC{x}_j}$ for one of the chart of
the atlas that we have defined in Section~\ref{pot_int_man}.
Thus
\begin{align*}
&\int_{S_{\VEC{x}_o,2\delta}}
\frac{\left\|\VEC{x}-\VEC{x}_o\right\|}
{\left(\|\VEC{x}-\VEC{x}_o\|^2 + \|\VEC{x}_o-\VEC{y}\|^2\right)^{n/2}}
\dss{S}{y} \\
&\qquad = \int_{\psi_{\VEC{x}_j}(S_{\VEC{x}_o,2\delta})}
\frac{\left\|\VEC{x}-\VEC{x}_o\right\|}
{\left(\|\VEC{x}-\VEC{x}_o\|^2
+ \|\VEC{x}_o-\psi_{\VEC{x}_j}^{-1}(\VEC{w})\|^2\right)^{n/2}}
\left| \det \diff \psi_{\VEC{x}_j}^{-1}(\VEC{w}) \right| \dx{\VEC{w}} \\
&\qquad \leq Q \int_{B_{2\delta}(\psi_{\VEC{x}_j}(\VEC{x}_o))}
\frac{\left\|\VEC{x}-\VEC{x}_o\right\|}
{\left(\|\VEC{x}-\VEC{x}_o\|^2
+ \|\psi_{\VEC{x}_j}(\VEC{x}_o) -\VEC{w}\|^2\right)^{n/2}} \dx{\VEC{z}} \ ,
\end{align*}
where we have use (\ref{defnQdetpsim1}),
$\displaystyle \psi_{\VEC{x}_j}(S_{\VEC{x}_o,2\delta}) \subset
B_{2\delta}(\psi_{\VEC{x}_j}(\VEC{x}_o))$, and
$\displaystyle \|\VEC{x}-\VEC{y}\| 
\geq \|\psi_{\VEC{x}_j}(\VEC{x}) - \psi_{\VEC{x}_j}(\VEC{y})\|$ for all
$\displaystyle \VEC{x},\VEC{y} \in U_{\VEC{x}_j}$ to obtain the
inequality.  Using the change of variables
$\VEC{w} = \psi_{\VEC{x}_j}(\VEC{x}_o) + r \VEC{z}$ with $0 \leq r < 2\delta$
and $\VEC{z} \in \partial B_1(\VEC{0}) \subset \RR^{n-1}$, and posing
$a = \|\VEC{x}-\VEC{x}_o\|$, we get
\begin{align}
&\int_{S_{\VEC{x}_o,2\delta}}
\frac{\left\|\VEC{x}-\VEC{x}_o\right\|}
{\left(\|\VEC{x}-\VEC{x}_o\|^2 + \|\VEC{x}_o-\VEC{y}\|^2\right)^{n/2}}
\dss{S}{y}
\leq Q \int_{\partial B_1(\VEC{0})} \int_0^{2\delta}
\frac{a r^{n-2}}{(a^2 + r^2)^{n/2}} \dx{r} \dx{S} \nonumber \\
&\qquad
= Q \omega_{n-1} \int_0^{2\delta/a} \frac{s^{n-2}}{(1 + s^2)^{n/2}} \dx{s} 
\leq Q \omega_{n-1} \int_0^{\infty} \frac{s^{n-2}}{(1 + s^2)^{n/2}} \dx{s} \ ,
\label{Bounddny6}
\end{align}
where we have used the substitution $r = a s$.
The last integral converges because
\[
\int_0^R \frac{s^{n-2}}{(1 + s^2)^{n/2}} \dx{s}
= \int_0^1 \frac{s^{n-2}}{(1 + s^2)^{n/2}} \dx{s}
+ \int_1^R \frac{s^{n-2}}{(1 + s^2)^{n/2}} \dx{s}
\leq \underbrace{\int_0^1 \frac{s^{n-2}}{(1 + s^2)^{n/2}} \dx{s}}_{<\infty}
+ \underbrace{\int_1^R \frac{1}{s^2} \dx{s}}_{<1}
\]
for $R >1$.

It follows from (\ref{Bounddny2}), (\ref{Bounddny4}),
(\ref{Bounddny5}) and (\ref{Bounddny6}) that
\begin{equation} \label{Bounddny7}
\begin{split}
\int_S
\left| \pdydx{N}{\nu_{\VEC{y}}}(\VEC{x},\VEC{y}) \right| \dss{S}{y}
&\leq \frac{1}{\omega_n \delta^{n-1}} \int_S \dx{S} 
+ \frac{2^{n/2}Q \omega_{n-1}}{\omega_n} 
\int_0^{\infty} \frac{s^{n-2}}{(1 + s^2)^{n/2}} \dx{s} \\
&\qquad + \frac{2^{(n+2)/2} C Q \omega_{n-1} \delta}{\omega_n}
\end{split}
\end{equation}

\stage{iii} It follows from (\ref{Bounddny1}) and (\ref{Bounddny7})
that the Lemma is satisfied if $B$ if equal or larger than
\[
\frac{1}{\omega_n \delta^{n-1}} \int_S \dx{S} 
+ \frac{2^{n/2}Q \omega_{n-1}}{\omega_n} 
\int_0^{\infty} \frac{s^{n-2}}{(1 + s^2)^{n/2}} \dx{s}
+ \frac{2^{(n+2)/2} C Q \omega_{n-1} \delta}{\omega_n} \ . \qedhere
\]
\end{proof}

\begin{lemma} \label{pot_lem_dbl5}
If $\phi \in C(S)$ is zero at some point $\VEC{z} \in S$, then the
double layer potential $u$ is continuous at $\VEC{z}$.
\end{lemma}

\begin{proof}
According to Proposition~\ref{pot_prop_KL1}, there exists a constant $A$
such that $\displaystyle \int_S |K(\VEC{x},\VEC{y})| \dss{S}{y} < A$
for all $\VEC{x}$ since $K$ is a continuous kernel of order $n-2$.

Given $\epsilon > 0$, choose $\eta > 0$ such that
$\displaystyle |\phi(\VEC{x})| < \frac{\epsilon}{3(A+B)}$
for $\VEC{x} \in S_{\VEC{z},\eta} = S \cap B_{\eta}(\VEC{z})$, where $B$ is
given in Lemma~\ref{pot_lem_dbl4}.

We have that
\begin{equation} \label{potLemDbl5equ1}
\begin{split}
|u(\VEC{x}) - u(\VEC{z})|
& \leq \int_{S_{\VEC{z},\eta}} \left| \pdydx{N}{\nu_{\VEC{y}}}
(\VEC{x},\VEC{y}) - \pdydx{N}{\nu_{\VEC{y}}} (\VEC{z},\VEC{y})
\right|\,|\phi(\VEC{y})| \dss{S}{y} \\
&\qquad + \int_{S \setminus S_{\VEC{z},\eta}} \left| \pdydx{N}{\nu_{\VEC{y}}}
(\VEC{x},\VEC{y}) - \pdydx{N}{\nu_{\VEC{y}}} (\VEC{z},\VEC{y})
\right|\,|\phi(\VEC{y})| \dss{S}{y} \ ,
\end{split}
\end{equation}
where obviously
$\displaystyle \pdydx{N}{\nu_{\VEC{y}}} (\VEC{w},\VEC{y})
= K(\VEC{w},\VEC{y})$ when $\VEC{w}, \VEC{y} \in S$.

Since
$\displaystyle (\VEC{x},\VEC{y}) \to 
\pdydx{N}{\nu_{\VEC{y}}}
(\VEC{x},\VEC{y}) - \pdydx{N}{\nu_{\VEC{y}}} (\VEC{z},\VEC{y})$
is continuous on the compact set 
$\displaystyle
\overline{B_\eta(\VEC{z})} \times \left(S \setminus S_{\VEC{z},\eta}\right)$,
it is uniformly continuous on this set.  Thus,
$\displaystyle \VEC{y} \to
\pdydx{N}{\nu_{\VEC{y}}}
(\VEC{x},\VEC{y}) - \pdydx{N}{\nu_{\VEC{y}}} (\VEC{z},\VEC{y})$
converges uniformly to $0$ on $S \setminus S_{\VEC{z},\eta}$ as
$\VEC{x} \to \VEC{z}$.  Hence,
$\displaystyle
\int_{S \setminus S_{\VEC{z},\eta}} \left| \pdydx{N}{\nu_{\VEC{y}}}
(\VEC{x},\VEC{y}) - \pdydx{N}{\nu_{\VEC{y}}} (\VEC{z},\VEC{y})
\right|\,|\phi(\VEC{y})| \dss{S}{y} \to 0$ as
$\VEC{x} \to \VEC{z}$.

Choose $\delta > 0$ such that
$\displaystyle
\int_{S \setminus S_{\VEC{z},\eta}} \left| \pdydx{N}{\nu_{\VEC{y}}}
(\VEC{x},\VEC{y}) - \pdydx{N}{\nu_{\VEC{y}}} (\VEC{z},\VEC{y})
\right|\,|\phi(\VEC{y})| \dss{S}{y} < \frac{\epsilon}{3}$ for
$\|\VEC{x} - \VEC{z}\| < \delta$.
Since
\begin{align*}
&\int_{S_{\VEC{z},\eta}}
\left| \pdydx{N}{\nu_{\VEC{y}}} (\VEC{x},\VEC{y})
- \pdydx{N}{\nu_{\VEC{y}}} (\VEC{z},\VEC{y}) \right|\,
|\phi(\VEC{y})| \dss{S}{y} \\
& \qquad \leq \int_{S_{\VEC{z},\eta}}
\left| \pdydx{N}{\nu_{\VEC{y}}} (\VEC{x},\VEC{y}) \right|\,
|\phi(\VEC{y})| \dss{S}{y} 
+ \int_{S_{\VEC{z},\eta}}
\left| \pdydx{N}{\nu_{\VEC{y}}} (\VEC{z},\VEC{y}) \right|\,
|\phi(\VEC{y})| \dss{S}{y} \\
&\qquad \leq \frac{\epsilon}{3(A+B)} \left(
\int_{S_{\VEC{z},\eta}}
\left| \pdydx{N}{\nu_{\VEC{y}}} (\VEC{x},\VEC{y}) \right| \dss{S}{y} 
+ \int_{S_{\VEC{z},\eta}}
\left| \pdydx{N}{\nu_{\VEC{y}}} (\VEC{z},\VEC{y}) \right| \dss{S}{y}
\right)
\leq \frac{2\epsilon}{3} \ ,
\end{align*}
we get from (\ref{potLemDbl5equ1}) that
$|u(\VEC{x}) - u(\VEC{z})| < \epsilon$ for 
$\|\VEC{x} - \VEC{z}\| < \delta$.
\end{proof}

\begin{theorem} \label{pot_double_layer}
Consider the double layer potential given in
Definition~\ref{pot_dblp_def} with $\phi \in C(S)$.
Then $\displaystyle u\big|_\Omega$ has a continuous extension to
$\overline{\Omega}$ and $\displaystyle u\big|_{\Omega^{\prime}}$
has a continuous extension to $\displaystyle \overline{\Omega^{\prime}}$.

To be more precise, let
$\displaystyle u_t(\VEC{x}) = u(\VEC{x} + t \nu(\VEC{x}))$
for $\VEC{x} \in \partial \Omega$ and $t \in \RR$,
\begin{equation} \label{pot_dblpT1}
u^{[i]}(\VEC{x}) = \frac{1}{2} \phi(\VEC{x}) + \int_S
K(\VEC{x},\VEC{y}) \, \phi(\VEC{y}) \dss{S}{y}
\end{equation}
for $\VEC{x} \in S$, and
\begin{equation} \label{pot_dblpT2}
u^{[o]}(\VEC{x}) = - \frac{1}{2} \phi(\VEC{x}) + \int_S
K(\VEC{x},\VEC{y}) \, \phi(\VEC{y}) \dss{S}{y}
\end{equation}
for $\VEC{x} \in S$.
In other words, $\displaystyle u^{[i]} = \frac{1}{2} \phi + T_K \phi$ and
$\displaystyle u^{[o]} = -\frac{1}{2} \phi + T_K \phi$ respectively on $S$.
Then $u_t \rightarrow u^{[i]}$ uniformly on $S$ as $t\rightarrow 0^-$ and
$u_t \rightarrow u^{[o]}$ uniformly on $S$ as $t\rightarrow 0^+$.
\end{theorem}

\begin{proof}
In the discussion below, we assume that $|t|$ is small enough to have
$\VEC{x} + t \nu(\VEC{x})$ in a tubular neighbourhood of $S$ for
all $\VEC{x} \in S$.  

\stage{i}
For $\VEC{x} \in S$ and $t<0$, we have
\[
u_t(\VEC{x}) = \int_S \phi(\VEC{y})
\pdydx{N}{\nu_{\VEC{y}}}(\VEC{x} +t \nu(\VEC{x}),\VEC{y})
\dss{S}{y} \ .
\]
From Lemma~\ref{pot_lem_dbl3}, we get that
\begin{align*}
u_t(\VEC{x}) &= \phi(\VEC{x}) \int_S \pdydx{N}{\nu_{\VEC{y}}}
(\VEC{x} +t \nu(\VEC{x}),\VEC{y}) \dss{S}{y}
+ \int_S \left( \phi(\VEC{y}) - \phi(\VEC{x}) \right)
\pdydx{N}{\nu_{\VEC{y}}}(\VEC{x} +t \nu(\VEC{x}),\VEC{y})
\dss{S}{y} \\
&= \phi(\VEC{x}) + \int_S \left( \phi(\VEC{y}) - \phi(\VEC{x}) \right) 
\pdydx{N}{\nu_{\VEC{y}}}(\VEC{x} +t \nu(\VEC{x}),\VEC{y})
\dss{S}{y} \ .
\end{align*}
Because $\VEC{y} \mapsto \phi(\VEC{y})-\phi(\VEC{x})$ is null at
$\VEC{y}=\VEC{x}$, we have from Lemma~\ref{pot_lem_dbl5} that the function
\[
t \mapsto \int_{\partial \Omega}
\left( \phi(\VEC{y}) - \phi(\VEC{x}) \right)
\pdydx{N}{\nu_{\VEC{y}}}(\VEC{x} +t \nu(\VEC{x}),\VEC{y})
\dss{S}{y}
\]
is continuous at $t=0$ for $\VEC{x}$ fixed.  Thus,
\begin{align*}
\lim_{t\rightarrow 0^-} u_t(\VEC{x})
&= \phi(\VEC{x}) + \int_S K(\VEC{x},\VEC{y}) \left( \phi(\VEC{y})
- \phi(\VEC{x}) \right) \dss{S}{y} \\
&= \phi(\VEC{x}) +
\int_S K(\VEC{x},\VEC{y}) \phi(\VEC{y}) \dss{S}{y}
- \phi(\VEC{x}) \int_S K(\VEC{x},\VEC{y}) \dss{S}{y} \\
&= \frac{1}{2} \phi(\VEC{x})
+ \int_S K(\VEC{x},\VEC{y}) \phi(\VEC{y}) \dss{S}{y}
= u^{[i]}(\VEC{x}) \ ,
\end{align*}
where we have used Lemma~\ref{pot_lem_dbl3} for the last
equality.

\stage{ii}
For $\VEC{x} \in S$ and $t>0$, we have
\[
u_t(\VEC{x}) = \int_S \phi(\VEC{y})
\pdydx{N}{\nu_{\VEC{y}}}(\VEC{x} +t \nu(\VEC{x}),\VEC{y})
\dss{S}{y} \ .
\]
From Lemma~\ref{pot_lem_dbl3}, we get that
\begin{align*}
u_t(\VEC{x}) &= \phi(\VEC{x}) \int_S \pdydx{N}{\nu_{\VEC{y}}}
(\VEC{x} +t \nu(\VEC{x}),\VEC{y}) \dss{S}{y}
+ \int_S \left( \phi(\VEC{y}) - \phi(\VEC{x}) \right)
\pdydx{N}{\nu_{\VEC{y}}}(\VEC{x} +t \nu(\VEC{x}),\VEC{y})
\dss{S}{y} \\
&= \int_S \left( \phi(\VEC{y}) - \phi(\VEC{x}) \right) 
\pdydx{N}{\nu_{\VEC{y}}}(\VEC{x} +t \nu(\VEC{x}),\VEC{y})
\dss{S}{y} \ .
\end{align*}
From here, the proof that
\[
\lim_{t\rightarrow 0^+} u_t(\VEC{x}) = -\frac{1}{2} \phi(\VEC{x}) +
\int_S K(\VEC{x},\VEC{y}) \phi(\VEC{y}) \dss{S}{y} = u^{[o]}(\VEC{x})
\]
is very similar to the proof in (i).

\stage{iii}
We prove that the convergence $u_t \rightarrow u^{[i]}$ (resp.
$u_t \rightarrow u^{[o]}$) is uniform on $S$ as
$t\rightarrow 0^-$ (resp. $t\rightarrow 0^+$).

Let $B$ be the constant from Lemma~\ref{pot_lem_dbl4} and let $A$
be a constant such that \\
$\displaystyle \int_S \left| K(\VEC{x},\VEC{y}) \right|
\dss{S}{y} < A$ for $\VEC{x} \in S$.  Such a constant exists according
to Proposition~\ref{pot_prop_KL1}.

Given $\epsilon >0$, choose $\eta>0$ such that
\begin{equation} \label{pot_dblp_B2}
\left|\phi(\VEC{x}) - \phi(\VEC{y}) \right| < \frac{\epsilon}{2(A + B)}
\end{equation}
for $\VEC{x}, \VEC{y} \in S$ and $\|\VEC{x} - \VEC{y}\| < \eta$.
This is possible because $\phi$ is continuous on the compact
set $S$.  So $\phi$ is uniformly continuous on
$S$.

We may also assume that $\eta$ is small enough such that
$\displaystyle
O_\eta = \left\{ \VEC{x} + t \nu(\VEC{x}) : \VEC{x} \in S
\ \text{and} \ |t| < \eta \right\}$
is a tubular neighbourhood of $S$.

Since
$\displaystyle
(\VEC{z},\VEC{y}) \mapsto \pdydx{N}{\nu_{\VEC{y}}}(\VEC{z},\VEC{y})$ is
continuous (and so uniformly continuous) on the compact set
$\displaystyle
P = \left\{ (\VEC{z},\VEC{y}) \in \overline{O_{\eta/2}} \times S
: \| \VEC{z}-\VEC{y} \| \geq \eta/2 \right\}$,
there exists $\delta < \eta/2$ such that
\begin{equation} \label{pot_dblp_B1}
\left| \pdydx{N}{\nu_{\VEC{y}}}(\VEC{z}_1,\VEC{y})
 - K(\VEC{z}_2,\VEC{y})
\right| < \frac{\epsilon}{2} \left(\int_S
\left| \phi(\VEC{y}) \right| \dss{S}{y} + \max_{\VEC{x}\in S}
|\phi(\VEC{x})|\, \int_S \dss{S}{y} \right)^{-1}
\end{equation}
for $(\VEC{z}_1,\VEC{y}) , (\VEC{z}_2,\VEC{y}) \in P$ with
$\|\VEC{z}_1-\VEC{z}_2\| < \delta$ and $\VEC{z}_2 \in S$.

Hence, for all $\VEC{x} \in S$ and $|t|<\delta$, we have that
\begin{align*}
&\left| u_t(\VEC{x}) - u^{[i]}(\VEC{x}) \right| \\
&\quad = \left| \int_S \left( \phi(\VEC{y}) - \phi(\VEC{x}) \right) 
\pdydx{N}{\nu_{\VEC{y}}}(\VEC{x} +t \nu(\VEC{x}),\VEC{y})
\dss{S}{y} - \int_S K(\VEC{x},\VEC{y})
\left( \phi(\VEC{y}) - \phi(\VEC{x}) \right) \dss{S}{y} \right| \\
&\quad \leq \int_S \left|
\pdydx{N}{\nu_{\VEC{y}}}(\VEC{x} +t \nu(\VEC{x}),\VEC{y})
- K(\VEC{x},\VEC{y}) \right|
\left| \phi(\VEC{y}) - \phi(\VEC{x}) \right| \dss{S}{y} \\
&\quad\leq \int_{S \cap B_{\eta}(\VEC{x})} \left(
\left|
\pdydx{N}{\nu_{\VEC{y}}}(\VEC{x} +t \nu(\VEC{x}),\VEC{y}) \right|
+\left| K(\VEC{x},\VEC{y}) \right| \right)
\left| \phi(\VEC{y}) - \phi(\VEC{x}) \right| \dss{S}{y} \\
&\qquad + \int_{S \setminus B_{\eta}(\VEC{x})}
\left| \pdydx{N}{\nu_{\VEC{y}}}(\VEC{x} +t \nu(\VEC{x}),\VEC{y})
- K(\VEC{x},\VEC{y}) \right|
\left( \left| \phi(\VEC{y})\right|
+ \left|\phi(\VEC{x}) \right| \right) \dss{S}{y}
\leq \frac{\epsilon}{2} + \frac{\epsilon}{2} < \epsilon \ ,
\end{align*}
where we have used (\ref{pot_dblp_B2}) to find an upper bound for the
integral over $S \cap B_{\eta}(\VEC{x})$, and (\ref{pot_dblp_B1}) to
find an upper bound for the integral over $S \setminus B_{\eta}(\VEC{x})$.
Moreover, we note that
$\| \VEC{x} + t \nu(\VEC{x}) - \VEC{y} || \geq \eta/2$
for $|t| < \eta/2$ and $\|\VEC{x} - \VEC{y} \| \geq \eta$.  So
(\ref{pot_dblp_B1}) can be effectively used (Figure~\ref{pot_DBL_LYR}).
\end{proof}

\pdfF{potential/dbl_layer}{Double layer potential}
{Figure for the proof of Theorem~\ref{pot_double_layer}.}{pot_DBL_LYR}

\subsection{Single Layer Potential} \label{subsect_SLpot}

As in the previous two sections, $S = \partial \Omega$.

\begin{defn} \label{pot_spl_def}
The {\bfseries single layer potential} with moment
$\phi \in C(S)$ is
\[
u(\VEC{x}) = \int_S
N(\VEC{x},\VEC{y}) \phi(\VEC{y}) \dss{S}{y} \quad , \quad
\VEC{x} \in \RR^n \ ,
\]
where $N$ is the fundamental solution for the Laplace operator given in
(\ref{laplace_NXY}) and Theorem~\ref{laplace_fund_sol}.
\end{defn}

Recall the following facts about $N$.
\begin{enumerate}
\item $\VEC{y}\mapsto N(\VEC{x},\VEC{y})$ is harmonic in
$\displaystyle \RR^n \setminus \{ \VEC{x} \}$.
\item For $n>2$,
$\displaystyle N(\VEC{x},\VEC{y}) = O(\|\VEC{y}\|^{2-n})$ as
$\|\VEC{y}\|\rightarrow \infty$.  Hence $\VEC{y} \mapsto N(\VEC{x},\VEC{y})$
is harmonic at infinity according to Proposition~\ref{pot_infty_u}.
\item (1) and (2) imply that the
single layer potential $u$ defined in Definition~\ref{pot_spl_def} is
harmonic on $\displaystyle \RR^n \setminus S$ and at infinity for $n>2$. 
\item The single layer potential $u$ is well defined at $\VEC{x} \in S$.
Let $\displaystyle R \geq \diam S
= \max_{\VEC{x},\VEC{y} \in S} \|\VEC{x}-\VEC{y}\|$,
$\displaystyle \BB = \left\{ \left(U_{\VEC{x}_j},
\psi_{\VEC{x}_j}\right) \right \}_{j\in J}$ be the atlas for
$S$ described in Section~\ref{pot_int_man}, and
$\displaystyle \{ \phi_{\VEC{x}_j} \}_{j\in J}$ be a partition of unity 
for $S$ subordinated to $\BB$.

For $n>2$, we have that  
\begin{align*}
&\int_S \left| N(\VEC{x},\VEC{y}) \right|\, |\phi(\VEC{y})| \dss{S}{y}
\leq \frac{1}{(n-2)\omega_n} \, \|\phi\|_\infty
\int_S \|\VEC{x}-\VEC{y}\|^{2-n} \dss{S}{y} \\
&\qquad = \frac{1}{(n-2)\omega_n} \, \|\phi\|_\infty
\sum_{j\in J} \int_{W_{\VEC{x}_j}}
\phi_{\VEC{x}_j}(\psi_{\VEC{x}_j}^{-1}(\VEC{w})\,
\|\VEC{x}-\psi_{\VEC{x}_j}^{-1}(\VEC{w})\|^{2-n}
\,\left|\det\psi_{\VEC{x}_j}^{-1}(\VEC{w})\right| \dx{\VEC{w}} \\
&\qquad \leq \frac{Q}{(n-2)\omega_n} \, \|\phi\|_\infty
\sum_{j\in J} \int_{W_{\VEC{x}_j} }
\|\psi_{\VEC{x}_j}(\VEC{x})-\VEC{w}\|^{2-n}\dx{\VEC{w}} \\
&\qquad \leq \frac{Q}{(n-2)\omega_n} \, \|\phi\|_\infty
\sum_{j\in J} \int_{\partial B_1(\VEC{0})} \int_0^R
  r^{2-n} r^{n-2} \dx{r} \dx{S}
\leq \frac{Q |J|\, \omega_{n-1} R}{(n-2)\omega_n} \, \|\phi\|_\infty
< \infty
\end{align*}
for all $\phi\in C(\partial \Omega)$ and all $\VEC{x} \in S$, where we
have used (\ref{defnQdetpsim1}), $\displaystyle \|\VEC{x}-\VEC{y}\| 
\geq \|\psi_{\VEC{x}_j}(\VEC{x}) - \psi_{\VEC{x}_j}(\VEC{y})\|$ for all
$\displaystyle \VEC{x},\VEC{y} \in U_{\VEC{x}_j}$ and
$\displaystyle W_{\VEC{x}_j} = \psi_{\VEC{x}_j}(U_{\VEC{x}_j}) \subset
B_R(\psi_{\VEC{x}_j}(\VEC{x})) \subset \RR^{n-1}$.

For $n=2$, we have that
\begin{align*}
&\int_S \left| N(\VEC{x},\VEC{y}) \right|\, |\phi(\VEC{y})| \dss{S}{y}
\leq \frac{1}{\omega_2} \, \|\phi\|_\infty
\int_S |\ln(\|\VEC{x}-\VEC{y}\|)| \dss{S}{y} \\
&\qquad = \frac{1}{\omega_2} \, \|\phi\|_\infty
\sum_{j\in J} \int_{W_{\VEC{x}_j}}
\phi_{\VEC{x}_j}(\psi_{\VEC{x}_j}^{-1}(w)\,
\left|\ln(\|\VEC{x}-\psi_{\VEC{x}_j}^{-1}(w)\|)\right|
\,\left|\det\psi_{\VEC{x}_j}^{-1}(w)\right|  \dx{w} \\
& \qquad \leq \frac{Q}{\omega_2} \, \|\phi\|_\infty
\sum_{j\in J} \int_{W_{\VEC{x}_j}}
\left|\ln(|\psi_{\VEC{x}_j}(\VEC{x})-w|)\right| \dx{w}
\leq \frac{2Q}{\omega_2} \, \|\phi\|_\infty
\sum_{j\in J} \int_0^R |\ln(r)| \dx{r} \\
&\qquad
\leq \frac{2Q\,|J|\,
\left( 2 + \left( |\ln(R)| + 1 \right) R\right)}{\omega_2} \, \|\phi\|_\infty
< \infty \ . 
\end{align*}
for all $\phi\in C(\partial \Omega)$ and all $\VEC{x} \in S$, where
we have used (\ref{defnQdetpsim1}), $\displaystyle \|\VEC{x}-\VEC{y}\| 
\geq \|\psi_{\VEC{x}_j}(\VEC{x}) - \psi_{\VEC{x}_j}(\VEC{y})\|$ for all
$\displaystyle \VEC{x},\VEC{y} \in U_{\VEC{x}_j}$ and
$\displaystyle W_{\VEC{x}_j} = \psi_{\VEC{x}_j}(U_{\VEC{x}_j}) \subset
]\psi_{\VEC{x}_j}(\VEC{x}) -R ,  \psi_{\VEC{x}_j}(\VEC{x}) + R[$.
\end{enumerate}

\begin{prop} \label{pot_slp_cont}
If $\phi \in C(S)$, then the single layer potential
defined in Definition~\ref{pot_spl_def} is continuous on
$\displaystyle \RR^n$.
\end{prop}

\begin{proof}
We only have to show that $u$ is continuous at an arbitrary point
$\VEC{z} \in \partial \Omega$ since $u$ is continuous on
$\displaystyle \RR^n \setminus S$.  Given $\mu >0$, let
$\displaystyle S_{\VEC{z},\mu} = \{ \VEC{y} \in S : \|\VEC{z}-\VEC{y}\|<\mu\}
= B_\mu(\VEC{z}) \cap S$.  We have
\begin{equation} \label{potSlpContEq1}
\begin{split}
|u(\VEC{x}) - u(\VEC{z})| &= \left| \int_S
\left( N(\VEC{x},\VEC{y}) -  N(\VEC{z},\VEC{y}) \right)
\phi(\VEC{y}) \dss{S}{y} \right| \\
&\leq \int_{S_{\VEC{z},\mu}} \left| N(\VEC{x},\VEC{y}) \right|\,
|\phi(\VEC{y})| \dss{S}{y}
+ \int_{S_{\VEC{z},\mu}}\left| N(\VEC{z},\VEC{y}) \right|\,
|\phi(\VEC{y})| \dss{S}{y} \\
&\qquad + \int_{S \setminus S_{\VEC{z},\mu}}
\left| N(\VEC{x},\VEC{y}) - N(\VEC{z},\VEC{y}) \right|
|\phi(\VEC{y}) |\dss{S}{y} \ .
\end{split}
\end{equation}

Since $\phi$ is continuous on the compact set
$S$, we have that $\displaystyle \|\phi\|_\infty =
\max_{\VEC{x}\in S} |\phi(\VEC{x})| < \infty$.

Given $\epsilon>0$, we first choose $\mu$ small enough such that
the sum of the first two integrals is less than $2\epsilon/3$ for all
$\displaystyle \VEC{x} \in B_{\mu/2}(\VEC{z}) \subset \RR^n$ as we
explain below.

\stage{i} Choose $\mu$ such that $4\mu < 1$ is smaller than the Lebesgue
number associated to the atlas that we have introduced in
Section~\ref{pot_int_man}.  So $B_{2\mu}(\VEC{z}) \subset V_{\VEC{x}_j}$
for some $j$.  Since $\displaystyle
B_{2\mu}(\VEC{z}) \supset B_{3\mu/2}(\VEC{x})
\supset B_{\mu}(\VEC{z})$ for all $\VEC{x} \in B_{\mu/2}(\VEC{z})$, we
get that
$\displaystyle S_{\VEC{x},3\mu/2} =  B_{3\mu/2}(\VEC{x}) \cap S
\supset B_{\mu}(\VEC{z}) \cap S = S_{\VEC{z},\mu}$
for all $\VEC{x} \in B_{\mu/2}(\VEC{z})$.
Thus, we have for $n>2$ that
\begin{align*}
&\int_{S_{\VEC{z},\mu}}\left| N(\VEC{x},\VEC{y}) \right|\,
|\phi(\VEC{y})| \dss{S}{y}
\leq \|\phi\|_\infty \int_{S_{\VEC{x},3\mu/2}} \left| N(\VEC{x},\VEC{y}) \right|
\dss{S}{y}
= \frac{\|\phi\|_\infty}{(n-2)\omega_n}
\int_{S_{\VEC{x},3\mu/2}} \left\| \VEC{x} - \VEC{y}\right\|^{2-n} \dx{S}{y} \\
&\qquad = \frac{\|\phi\|_\infty}{(n-2)\omega_n}
\int_{\psi_{\VEC{x}_j}(S_{\VEC{x},3\mu/2})}
\left\| \VEC{x} - \psi_{\VEC{x}_j}^{-1}(\VEC{w})\right\|^{2-n}
\left| \det \diff \psi_{\VEC{x}_j}^{-1}(\VEC{w}) \right| \dx{\VEC{w}} \\
&\qquad \leq \frac{\|\phi\|_\infty Q}{(n-2)\omega_n}
\int_{\psi_{\VEC{x}_j}(S_{\VEC{x},3\mu/2})}
\left\| \psi_{\VEC{x}_j}(\VEC{x}) - \VEC{w}\right \|^{2-n} \dx{\VEC{w}} \\
&\qquad \leq \frac{\|\phi\|_\infty Q}{(n-2)\omega_n}
\int_{\partial B_1(\VEC{0})} \int_0^{3\mu/2} \dx{r} \dx{S}
= \frac{3\|\phi\|_\infty \omega_{n-1} Q\,\mu}{2(n-2)\omega_n}
\end{align*}
for all $\VEC{x} \in B_{\mu/2}(\VEC{z})$.  As is usual, for the
previous computations, we have used (\ref{defnQdetpsim1})
and $\displaystyle \|\VEC{x}-\VEC{y}\|
\geq \|\psi_{\VEC{x}_j}(\VEC{x}) - \psi_{\VEC{x}_j}(\VEC{y})\|$ for all
$\displaystyle \VEC{x},\VEC{y} \in V_{\VEC{x}_j}$, where we consider
$\psi_{\VEC{x}_j}$ to be the projection of $\displaystyle \RR^n$ onto
$T_{\VEC{x}_j}$ as defined in Section~\ref{pot_int_man}.

Similarly, we find for $n=2$ that
\begin{align*}
&\int_{S_{\VEC{z},\mu}}\left| N(\VEC{x},\VEC{y}) \right|\,
|\phi(\VEC{y})| \dss{S}{y}
\leq \|\phi\|_\infty \int_{S_{\VEC{x},3\mu/2}} \left| N(\VEC{x},\VEC{y}) \right|
\dss{S}{y}
= \frac{\|\phi\|_\infty}{\omega_2}
\int_{S_{\VEC{x},3\mu/2}}
\left| \ln\left( \left\| \VEC{x} - \VEC{y}\right\| \right) \right| \dx{S}{y} \\
&\qquad \leq \frac{\|\phi\|_\infty Q}{\omega_2} \int_0^{3\mu/2} |\ln(r)| \dx{r}
= \frac{3\|\phi\|_\infty Q\,(|\ln(3\mu/2)|+1) \mu}{2\omega_2}
\end{align*}
for all $\VEC{x} \in B_{\mu/2}(\VEC{z})$.

\stage{ii} We get directly from (\ref{Sxepsint1}) and
(\ref{Sxepsint2}) that
\[
\int_{S_{\VEC{z},\mu}} \left| N(\VEC{z},\VEC{y}) \right|\,
|\phi(\VEC{y})| \dss{S}{y}
\leq \frac{\|\phi\|_\infty}{(n-2)\omega_n} 
\int_{S_{\VEC{z},\mu}} \left\|\VEC{z} -\VEC{y} \right\|^{2-n} \dss{S}{y}
\leq \frac{\|\phi\|_\infty Q \omega_{n-1} \mu}{(n-2)\omega_n} 
\]
for $n>2$, and
\[
\int_{S_{\VEC{z},\mu}} \left| N(\VEC{z},\VEC{y}) \right|\,
|\phi(\VEC{y})| \dss{S}{y}
\leq \frac{\|\phi\|_\infty}{\omega_2} 
\int_{S_{\VEC{z},\mu}} \left| \ln\left(
\left\|\VEC{z} -\VEC{y} \right\| \right)\right|  \dss{S}{y}
\leq \frac{ \|\phi\|_\infty Q \left(|\ln(\mu)|+1\right)\mu}{\omega_2} 
\]
for $n= 2$.

\stage{iii}  It follows from (i) and (ii) that we can choose $\mu$
small enough such that the sum of the first two integrals
in (\ref{potSlpContEq1}) is less than $2\epsilon/3$ for all
$\displaystyle \VEC{x} \in B_{\mu/2}(\VEC{z}) \subset \RR^n$.

\stage{iv}
Since $(\VEC{x},\VEC{y}) \mapsto N(\VEC{x},\VEC{y})$ is continuous on
the compact set
$\overline{B_{\mu/2}(\VEC{z})} \times (S \setminus S_{\VEC{z},\mu})$,
it is uniformly continuous on this compact set.  So, there exists
$0 < \delta < \mu/2$ such that
\[
\sup_{\VEC{y} \in S \setminus S_{\VEC{z},\mu}}
\left| N(\VEC{x},\VEC{y}) - N(\VEC{z},\VEC{y}) \right| <
\epsilon\left( 3\|\phi\|_\infty \int_S \dss{S}{y}\right)^{-1}
\]
for all $\displaystyle \VEC{x} \in B_\delta(\VEC{z}) \subset \RR^n$.
Hence
\[
\int_{S \setminus S_{\VEC{z},\mu}}
\left| N(\VEC{x},\VEC{y}) - N(\VEC{z},\VEC{y}) \right|
\,|\phi(\VEC{y}) |\dss{S}{y}
\leq \|\phi\|_\infty \sup_{\VEC{y} \in S \setminus S_{\VEC{z},\mu}}
\left| N(\VEC{x},\VEC{y}) - N(\VEC{z},\VEC{y}) \right| 
\int_{S \setminus S_{\VEC{z},\mu}} \dss{S}{y}
< \frac{\epsilon}{3}
\]
for all $\VEC{x} \in B_\delta(\VEC{z})$.

\stage{v} We get from (\ref{potSlpContEq1}), (iii) and (iv) that
$\displaystyle \left| u(\VEC{x}) - u(\VEC{x}_0) \right| \leq \epsilon$ for all
$\VEC{x} \in B_\delta(\VEC{w})$.
\end{proof}

Let $O_\mu$ be a tubular neighbourhood of $S$.  For all
$\VEC{z_t} \in O_\mu$, we define a directional derivative by the formula
\[
\pdydxS{u}{\nu}(\VEC{z}_t)
\equiv \dfdx{u\big(\VEC{x}+t\nu(\VEC{x})\big)}{t}
= \graD u(\VEC{z}_t) \cdot \nu(\VEC{x})
= \graD u\big(\VEC{x}+t\nu(\VEC{x})\big) \cdot \nu(\VEC{x})
\]
where $\VEC{z}_t = \VEC{x} + t\nu(\VEC{x})$ with $\VEC{x} \in S$
is the unique representation of $\VEC{z}_t$ in the tubular neighbourhood
$O_\mu$.  The directional derivative of the single layer
potential $u$ at $\VEC{x}\in O_\mu \setminus S$ is
therefore
\[
\pdydxS{u}{\nu}(\VEC{x}) = \int_S 
 \phi(\VEC{y}) \pdydxS{N}{\nu_{\VEC{x}}}(\VEC{x},\VEC{y}) \dss{S}{y} \ ,
\]
where $\displaystyle \pdydxS{}{\nu_{\VEC{x}}}$ indicates that the
directional derivative is with respect to the variable $\VEC{x}$ while
$\VEC{y}$ is constant.

Let $\displaystyle K^\ast(\VEC{x},\VEC{y}) = K(\VEC{y},\VEC{x})$
for $\VEC{x},\VEC{y} \in S$,  where $K$ is defined in
(\ref{pot_K_defn}).  We have that
\[
K(\VEC{p},\VEC{q}) = \pdydx{N}{\nu_{\VEC{y}}}(\VEC{p},\VEC{q})
= - \frac{1}{\omega_n \|\VEC{p}-\VEC{q}\|^n}\,(\VEC{p}-\VEC{q})\cdot
\nu(\VEC{q})
= \frac{1}{\omega_n \|\VEC{q}-\VEC{p}\|^n}\,(\VEC{q}-\VEC{p})\cdot
\nu(\VEC{q})
= \pdydx{N}{\nu_{\VEC{x}}}(\VEC{q},\VEC{p})
\]
for $\VEC{p} , \VEC{q} \in S$ with $\VEC{p} \neq \VEC{q}$.  Thus
$\displaystyle K^\ast(\VEC{p},\VEC{q}) =
\pdydx{N}{\nu_{\VEC{x}}}(\VEC{p},\VEC{q})$ for
$\VEC{p} , \VEC{q} \in S$ with $\VEC{p} \neq \VEC{q}$.

We expand $\displaystyle \pdydxS{u}{\nu}$ to $O_\mu$ with
the definition
\begin{equation} \label{pot_dslp_def}
\pdydxS{u}{\nu}(\VEC{x}) = 
\begin{cases}
\displaystyle
\int_S \phi(\VEC{y}) \pdydxS{N}{\nu_{\VEC{x}}}(\VEC{x},\VEC{y})
\dss{S}{y} & \quad \text{if} \ \VEC{x} \in O_\mu \setminus S \\[0.8em]
\displaystyle
\int_S K^\ast(\VEC{x},\VEC{y}) \phi(\VEC{y})
\dss{S}{y} & \quad \text{if} \ \VEC{x} \in S
\end{cases}
\end{equation}
Since $K$ is a continuous kernel of order $n-2$, we also have that
$\displaystyle K^\ast$ is 
a continuous kernel of order $n-2$.  Hence, it follows from
Propositions~\ref{pot_lin_op_K} and \ref{pot_Kc_fb} that
$\displaystyle \pdydxS{u}{\nu} \bigg|_S$ is well defined as
$\displaystyle T_{K^\ast} \phi$, and
$\displaystyle \pdydxS{u}{\nu}\bigg|_S \in C(S)$ if $\phi\in C(S)$.

According to Proposition~\ref{pot_K_LtwoComp},
$\displaystyle T_K:L^2(S)\rightarrow L^2(S)$ and
$\displaystyle T_{K^\ast}:L^2(S)\rightarrow L^2(S)$ are
compact operators.  Moreover, since $K$ and $\displaystyle K^\ast$ are
real valued functions, we have
\begin{align*}
&\ps{\psi}{T_K\phi}_2 = \int_S \psi(\VEC{x}) \left(
\overline{\int_S K(\VEC{x},\VEC{y})\phi(\VEC{y}) \dss{S}{y}} \right) \dss{S}{x}
=\int_S \psi(\VEC{x}) \left(
\int_S K(\VEC{x},\VEC{y})\overline{\phi(\VEC{y})} \dss{S}{y} \right)
\dss{S}{x} \\
&\quad =\int_S \left( \int_S K(\VEC{x},\VEC{y}) \psi(\VEC{x}) \dss{S}{x}
\right) \overline{\phi(\VEC{y})} \dss{S}{y}
=\int_S \left(
\int_S K^\ast(\VEC{y},\VEC{x}) \psi(\VEC{x}) \dss{S}{x}
\right) \overline{\phi(\VEC{y})} \dss{S}{y} =
\ps{T_{K^\ast} \psi}{\phi}_2
\end{align*}
for $\phi,\psi \in L^2(S)$.
Thus $\displaystyle T_{K^\ast} = T_K^\ast$, the adjoint operator to $T_K$ on
$\displaystyle L^2(S)$.

\begin{lemma} \label{pot_inequ}
For $\eta$ small enough, there exists a constant $C_1$ such that
\[
\left\| \nu(\VEC{y}) - \nu(\VEC{x}) \right\| < C_1
\|\VEC{y}-\VEC{x}\|
\]
for all $\VEC{x}, \VEC{y} \in S$ with $\|\VEC{x} - \VEC{y}\|<\eta$.
Moreover, there exists a constant $C_2$ such that
\[
\left\| \big(\VEC{x} + t\nu(\VEC{x})\big) - \VEC{y} \right\|
\geq C_2 \|\VEC{x} - \VEC{y}\|
\]
for all $\VEC{x}, \VEC{y} \in S$ with $\|\VEC{x} - \VEC{y}\|<\eta$,
and $t \in \RR$.  
\end{lemma}

\begin{proof}
Let $\displaystyle \BB = \{ (U_{\VEC{x}_j}, \psi_{\VEC{x}_j}) \}_{j\in J}$
be the atlas for $S$ described in Section~\ref{pot_int_man}, and
let $\epsilon$ be the Lebesgue number associated to the atlas $\BB$.

Given $\VEC{x} \in S$, there exists an index $j$ such that
$\displaystyle S_{\VEC{x},\epsilon/2} = B_{\epsilon/2}(\VEC{x})
\cap S \subset U_{\VEC{x}_j}$.

We may assume that
$\displaystyle \nu \circ \psi_{\VEC{x}_j}^{-1}:
\overline{W_{\VEC{x}_j}} \to \RR^n$
is of class $\displaystyle C^1$ since we assume
$\displaystyle \psi_{\VEC{x}_j}^{-1} \in
C^2\left(\overline{W_{\VEC{x}_j}}\right)$
\footnote{If $\VEC{z} \in S \subset \RR^3$, we have that
$\displaystyle \nu(\VEC{z})
= \nu\left(\psi_{\VEC{x}_j}^{-1}(\breve{\VEC{z}})\right)
= (1/\|\VEC{v}\|) \VEC{v}$, where
$\breve{\VEC{z}} = \psi_{\VEC{x}_j}(\VEC{z})$ and
$\displaystyle \VEC{v} = \pdydx{\psi_{\VEC{x}_j}^{-1}}{\breve{z}_1}(
\breve{\VEC{z}})
\times \pdydx{\psi_{\VEC{x}_j}^{-1}}{\breve{z}_2}(\breve{\VEC{z}})$,
the vector product}.
Let $M_j$ be the maximum of
$\displaystyle \left\| \diff \left(\nu \circ \psi_j^{-1}\right)\right\|$
on the compact set $\displaystyle \overline{W_j}$.  For all
$\VEC{y} \in S_{\VEC{x},\epsilon/2}$, we have that
\[
  \left\| \nu(\VEC{y}) - \nu(\VEC{x}) \right\|
= \left\| \nu\left(\psi_{\VEC{x}_j}^{-1}(\breve{\VEC{y}})\right) -
 \nu\left(\psi_{\VEC{x}_j}^{-1}(\breve{\VEC{x}})\right) \right\|
\leq M_j \left\| \breve{\VEC{y}} - \breve{\VEC{x}} \right\|
\leq M_j \left\| \VEC{y} - \VEC{x} \right\| \ ,
\]
where $\breve{y} = \psi_{\VEC{x}_j}(\VEC{y})$ and
$\breve{x} = \psi_{\VEC{x}_j}(\VEC{x})$.  In the previous relation, we
have used $\displaystyle \|\VEC{x}-\VEC{y}\| 
\geq \|\psi_{\VEC{x}_j}(\VEC{x}) - \psi_{\VEC{x}_j}(\VEC{y})\|$ for all
$\displaystyle \VEC{x},\VEC{y} \in U_{\VEC{x}_j}$ to get the last
inequality.

We get the first conclusion of the lemma with
$\displaystyle C_1 = \max_{j \in J} M_j$ and $\eta \leq \epsilon /2$.

To get the second conclusion of the lemma, we may assume that the
$U_{\VEC{x}_j}$ in the definition of the atlas $\BB$ are small enough
to satisfy 
$\nu(\VEC{x}_1)\cdot \nu(\VEC{x}_2) \geq 1/\sqrt{2}$ for all
$\VEC{x}_1,\VEC{x}_2 \in U_{\VEC{x}_j}$.  This is possible because
$\displaystyle \nu:S \to \RR^n$ is of class $C^1$.  This means that
the cosine of the angle between the vectors
$\nu(\VEC{x}_1)$ and $\nu(\VEC{x}_2)$ is between $1/\sqrt{2}$ and
$1$ for all $\VEC{x}_1,\VEC{x}_2 \in U_{\VEC{x}_j}$.

Let $\VEC{y}_p$ be the orthogonal projection of $\VEC{y}$ on the line
$\displaystyle \left\{\VEC{x} + t\nu(\VEC{x}) : t \in \RR \right\}$,
and $\theta$ be the angle between the vectors $\VEC{y}_p - \VEC{y}$ and
$\VEC{x}- \VEC{y}$.  Then
\begin{equation} \label{pot_curvat}
\| \big(\VEC{x} + t\nu(\VEC{x})\big)-\VEC{y} \|
\geq \|\VEC{y}_p - \VEC{y}\| = \cos(\theta)
\|\VEC{y}-\VEC{x}\| \geq \frac{1}{\sqrt{2}} \|\VEC{y}-\VEC{x}\|
\end{equation}
for all $\VEC{y} \in S_{\VEC{x},\epsilon/2}$ as we can observe from
the sketch in Figure~\ref{POT_MAX_CURV}.    We note that
(\ref{pot_curvat}) is true for all $\VEC{x}$.  Thus, we get the second
conclusion of the lemma with $C_2 = 1/\sqrt{2}$ and $\eta \leq \epsilon/2$.

By shrinking $\eta$, we may have $C_2$ as close to $1$ as we want.
This is expected because we assume that $S$ is ``smooth''.
\end{proof}

\pdfF{potential/max_curv}{Sketch associated to the proof of
Theorem~\ref{pot_slp_exist}}{Sketch associated to the proof of
Theorem~\ref{pot_slp_exist} where $0 \leq \phi \leq \pi/4$ is the
angle between $\nu(\VEC{x})$ and $\nu(\VEC{y})$.}{POT_MAX_CURV}

\begin{theorem} \label{pot_slp_exist}
If $\phi \in C(S)$, then
$\displaystyle u\big|_{\overline{\Omega}} \in C_{\nu(\Omega)}$ and
$\displaystyle u\big|_{\overline{\Omega^{\prime}}} \in
C_{\nu(\Omega^{\prime})}$.
Moreover,
\[
\pdydx{u}{\nu^-}(\VEC{x}) = -\frac{1}{2} \phi(\VEC{x}) +
\int_{\partial \Omega} K^\ast(\VEC{x},\VEC{y}) \phi(\VEC{y}) \dss{S}{y}
\]
for $\VEC{x} \in S$ and
\[
\pdydx{u}{\nu^+}(\VEC{x}) = \frac{1}{2} \phi(\VEC{x}) +
\int_{\partial \Omega} K^\ast(\VEC{x},\VEC{y}) \phi(\VEC{y}) \dss{S}{y}
\]
for $\VEC{x} \in S$.  In other words,
$\displaystyle \pdydx{u}{\nu^-} = -\frac{1}{2} \phi + T^\ast_K \phi$
and
$\displaystyle \pdydx{u}{\nu^+} = \frac{1}{2} \phi + T^\ast_K \phi$
on $S$.

If $\displaystyle
w_t(\VEC{x}) = \pdydxS{u}{\nu}\big(\VEC{x}+t\nu(\VEC{x})\big)$
for $\VEC{x} \in S$ and $\VEC{x} + t \nu(\VEC{x}) \in O_\mu$,
then $\displaystyle w_t \rightarrow \pdydx{u}{\nu^-}$ uniformly
on $S$ as $t\rightarrow 0^-$ and
$\displaystyle w_t \rightarrow \pdydx{u}{\nu^+}$ uniformly on
$S$ as $t\rightarrow 0^+$.
\end{theorem}

\begin{proof}
From Proposition~\ref{pot_slp_cont}, we have that
$\displaystyle u \in C(\RR^n)$.
Hence, $u \in C(\overline{\Omega})$ and
$\displaystyle u \in C(\overline{\Omega^{\prime}})$.  By definition of
the single layer potential, we also have that
$\displaystyle u \in C^1(\Omega)$ and
$\displaystyle u \in C^1(\Omega^{\prime})$.

Consider the double layer potential of moment $\phi$ defined by
\[
v(\VEC{x}) = \int_S \phi(\VEC{y})
\pdydx{N}{\nu_{\VEC{y}}}(\VEC{x},\VEC{y}) \dss{S}{y}
\]
for $\displaystyle \VEC{x} \in \RR^n \setminus S$, and the function
\[
f(\VEC{x}) =
\begin{cases}
\displaystyle v(\VEC{x}) + \pdydxS{u}{\nu_{\VEC{x}}}(\VEC{x}) & \quad
\text{if} \quad \VEC{x} \in O_\mu \setminus S \\[0.7em]
\displaystyle
(T_K\phi)(\VEC{x}) + (T^\ast_K\phi)(\VEC{x}) & \quad \text{if} \quad
\VEC{x} \in S
\end{cases}
\]

\stage{i} We claim that $f$ is continuous on $O_\mu$.  Note that
$O_\mu$ is an open subset of $\RR^n$ because $S$ is a manifold of
dimension $n-1$.  It is clear
from its definition that $f$ is continuous on $O_\mu \setminus S$.
To prove that $f$ is continuous on $O_\mu$, it is enough to prove that
\[
\sup_{\VEC{x}\in S}\,
\left| f(\VEC{x}+t\nu(\VEC{x})) - f(\VEC{x})\right| \rightarrow
0 \quad \text{as} \quad t \rightarrow 0 \ .
\]
This will implies that
$\displaystyle f\big|_{O_\mu \setminus S}$ can be
continuously expanded to a continuous function on $O_\mu$ by
$\displaystyle f(\VEC{x}) = (T_K\phi)(\VEC{x}) + (T^\ast_K\phi)(\VEC{x})$
for $\VEC{x} \in S$.

We now prove the claim.  Let $\VEC{z}_t = \VEC{x}+t\nu(\VEC{x})$ for
$\VEC{x} \in S$ and $S_{\VEC{x},\eta} = S \cap B_\eta(\VEC{x})$ with
$\eta < \mu$ given in Lemma~\ref{pot_inequ}.  Recall from the proof of 
Lemma~\ref{pot_inequ} that $2\eta$ is smaller than the Lebesgue number
associated to the atlas $\BB$ defined in Section~\ref{pot_int_man}.
Thus, for each $\VEC{x}$, there exists $j$ such that
$S_{\VEC{x},\eta} \subset U_{\VEC{x}_j}$.  we have that
\begin{align*}
&\left|f(\VEC{z_t}) - f(\VEC{x})\right|\\
&\qquad = \left| \int_S \phi(\VEC{y}) \left(
\pdydx{N}{\nu_{\VEC{y}}}(\VEC{z_t},\VEC{y}) +
\pdydxS{N}{\nu_{\VEC{x}}}(\VEC{z_t},\VEC{y})\right) \dss{S}{y}
- \int_S \phi(\VEC{y}) \left(
\pdydx{N}{\nu_{\VEC{y}}}(\VEC{x},\VEC{y}) +
\pdydx{N}{\nu_{\VEC{x}}}(\VEC{x},\VEC{y})\right) \dss{S}{y}
\right| \\
&\qquad \leq I_1(\VEC{x},t,\eta) + I_2(\VEC{x},t,\eta) +
I_3(\VEC{x},t,\eta) + I_4(\VEC{x},t,\eta) \  ,
\end{align*}
where
\begin{align*}
I_1(\VEC{x},t,\eta) &= \int_{S\setminus S_{\VEC{x},\eta}}
\left|\phi(\VEC{y})\right|\,
\left| \pdydx{N}{\nu_{\VEC{y}}}(\VEC{z}_t,\VEC{y}) -
\pdydx{N}{\nu_{\VEC{y}}}(\VEC{x},\VEC{y})\right| \dss{S}{y} \ ,\\
I_2(\VEC{x},t,\eta) &= \int_{S\setminus S_{\VEC{x},\eta}}
\left|\phi(\VEC{y})\right|\,
\left| \pdydxS{N}{\nu_{\VEC{x}}}(\VEC{z}_t,\VEC{y}) -
\pdydx{N}{\nu_{\VEC{x}}}(\VEC{x},\VEC{y})\right| \dss{S}{y} \ ,\\
I_3(\VEC{x},t,\eta) &= \int_{S_{\VEC{x},\eta}}
\left|\phi(\VEC{y})\right|\,
\left| \pdydx{N}{\nu_{\VEC{y}}}(\VEC{z}_t,\VEC{y}) +
\pdydxS{N}{\nu_{\VEC{x}}}(\VEC{z}_t,\VEC{y})\right| \dss{S}{y}
\intertext{and}
I_4(\VEC{x},t,\eta) &= \int_{S_{\VEC{x},\eta}}
\left|\phi(\VEC{y})\right|\,
\left| \pdydx{N}{\nu_{\VEC{y}}}(\VEC{x},\VEC{y}) +
\pdydx{N}{\nu_{\VEC{x}}}(\VEC{x},\VEC{y})\right| \dss{S}{y} \ .
\end{align*}

\stage{i.a} We consider $I_3(\VEC{x},t,\eta)$.
Since $\phi$ is continuous on the compact set $S$, we have that
$\displaystyle
\left\|\phi\right\|_{\infty} = \sup_{\VEC{y}\in S}|\phi(\VEC{y}| < \infty$.
Hence,
\[
I_3(\VEC{x},t,\eta)
\leq \left\|\phi\right\|_{\infty} \int_{S_{\VEC{x},\eta}}
\left| \pdydx{N}{\nu_{\VEC{y}}}(\VEC{z}_t,\VEC{y}) +
\pdydxS{N}{\nu_{\VEC{x}}}(\VEC{z}_t,\VEC{y})\right| \dss{S}{y} \  .
\]
Using Lemma~\ref{pot_inequ}, we have that
\begin{align*}
&\left| \pdydx{N}{\nu_{\VEC{y}}}(\VEC{z}_t,\VEC{y}) +
\pdydxS{N}{\nu_{\VEC{x}}}(\VEC{z}_t,\VEC{y})\right|
= \left| \frac{1}{\omega_n \|\VEC{z}_t-\VEC{y}\|^n}\,(\VEC{z}_t-\VEC{y})\cdot
(\nu(\VEC{x}) - \nu(\VEC{y})) \right| \\
&\quad
\leq \frac{1}{\omega_n \|\VEC{z}_t-\VEC{y}\|^n}\,\|\VEC{z}_t-\VEC{y}\|\,
\|\nu(\VEC{x}) - \nu(\VEC{y})\|
\leq \frac{C_1}{\omega_n \|\VEC{z}_t-\VEC{y}\|^{n-1}}\,\|\VEC{x}-\VEC{y}\|
\leq \frac{C_1}{\omega_n C_2^{n-1}\|\VEC{x}-\VEC{y}\|^{n-2}}
\end{align*}
for $\|\VEC{x} - \VEC{y}\| < \eta$ and $|t| < \eta$.  It follows from
(\ref{Sxepsint1}) that
\[
I_3(\VEC{x},t,\eta)
\leq \frac{C_1 \left\|\phi\right\|_{\infty}}{\omega_n C_2^{n-1}}
\int_{S_{\VEC{x},\eta}} \|\VEC{x}-\VEC{y}\|^{2-n} \dss{S}{y}
= \frac{C_1 \left\|\phi\right\|_{\infty} \omega_{n-1} Q}{C_2^{n-1}\omega_n}
\, \eta
\]
for $\VEC{x} \in S$ and
$\VEC{z}_t = \VEC{x}+t\nu(\VEC{x})$ with $|t|< \eta$.

\stage{i.b} A similar but simpler reasoning to the one above yields
\[
I_4(\VEC{x},\VEC{t},\eta)
= \int_{S_{\VEC{x},\eta}} \left|\phi(\VEC{y})\right|\,
\left| \pdydx{N}{\nu_{\VEC{y}}}(\VEC{x},\VEC{y}) +
\pdydx{N}{\nu_{\VEC{x}}}(\VEC{x},\VEC{y})\right| \dss{S}{y}
\leq \frac{C_1 \left\|\phi\right\|_{\infty}\omega_{n-1} Q}{\omega_n}\, \eta
\]
for $\VEC{x} \in S$.

\stage{i.c} Given $\epsilon > 0$, choose $\eta$ small enough such that
$I_3(\VEC{x},t,\eta)$ and $I_4(\VEC{x},t,\eta)$ are smaller
than $\epsilon/4$ for $\VEC{x} \in S$ and
$\VEC{z}_t = \VEC{x}+t\nu(\VEC{x})$ with $|t|< \eta$.

\stage{i.d}
Since $\displaystyle (\VEC{x},\VEC{y}) \mapsto
\pdydx{N}{\nu_{\VEC{y}}}(\VEC{x},\VEC{y})$ is
continuous on the compact set
$\{(\VEC{x},\VEC{y}) \in \overline{O_\nu} \times S :
\|\VEC{x} - \VEC{y}\| \geq \eta$,
it is uniformly continuous on this compact set.  So, there exists
$0 < \delta_1 < \eta$ such that
\[
\left| \pdydx{N}{\nu_{\VEC{y}}}(\VEC{z}_t,\VEC{y}) -
\pdydx{N}{\nu_{\VEC{y}}}(\VEC{x},\VEC{y})\right|
=\left| \pdydx{N}{\nu_{\VEC{y}}}(\VEC{x} + t \nu(\VEC{x}),\VEC{y}) -
\pdydx{N}{\nu_{\VEC{y}}}(\VEC{x},\VEC{y})\right|
< \frac{\epsilon}{4} \left(
\int_{S\setminus S_{\VEC{x},\eta}}\left|\phi(\VEC{y})\right|
\dss{S}{y} \right)^{-1}
\]
for all $\displaystyle \VEC{y} \in S \setminus S_{\VEC{x},\eta}$,
$|t| < \delta_1$ and $\VEC{x} \in S$.
Hence,
$\displaystyle I_1(\VEC{x},t,\eta) \leq \epsilon/4$ for all $\VEC{x} \in S$
and $|t| < \delta_1$.

\stage{i.e} Since $\displaystyle (t,\VEC{x},\VEC{y}) \mapsto
\nabla N(\VEC{x} + t \nu(\VEC{x}) - \VEC{y}) \cdot \nu(\VEC{x})$
is continuous on the compact set
$[\eta/2,\eta/2] \times \{(\VEC{x},\VEC{y}) \in S \times S :
\|\VEC{x} - \VEC{y}\| \geq \eta$,
it is uniformly continuous on this compact set.  So, there exists
$0 < \delta_2 < \eta/2$ such that
\begin{align*}
\left| \pdydxS{N}{\nu_{\VEC{x}}}(\VEC{z}_t,\VEC{y}) -
\pdydx{N}{\nu_{\VEC{x}}}(\VEC{x},\VEC{y})\right|
&= \left| \nabla N(\VEC{x} + t \nu(\VEC{x}) - \VEC{y})
\cdot \nu(\VEC{x}) -
\nabla N(\VEC{x} - \VEC{y}) \cdot \nu(\VEC{x}) \right| \\
&\leq \frac{\epsilon}{4} \left(
\int_{S\setminus S_{\VEC{x},\eta}}\left|\phi(\VEC{y})\right|
\dss{S}{y} \right)^{-1}
\end{align*}
for all $\displaystyle \VEC{y} \in S \setminus S_{\VEC{x},\eta}$,
$|t| < \delta_2$ and $\VEC{x} \in S$.
Hence,
$\displaystyle I_2(\VEC{x},t,\eta) \leq \epsilon/4$ for all $\VEC{x} \in S$
and $|t| < \delta_2$.

\stage{i.f} Hence, for $\delta < \min\{\delta_1,\delta_2\}$, we have
that 
$\displaystyle \left| f(\VEC{z}_t) - f(\VEC{x}) \right|< \epsilon$ for
$\VEC{x} \in S$ and $|t|< \delta$.  This proves our claim.

\stage{ii} Since $f \in C(O_\mu)$, we have
\begin{align*}
(T_K \phi)(\VEC{x}) + (T^\ast_K \phi)(\VEC{x}) &=
\lim_{t\rightarrow 0^-} f\big(\VEC{x}+t\nu(\VEC{x})\big)
= \lim_{t\rightarrow 0^-} v\big(\VEC{x}+t\nu(\VEC{x})\big)
+ \lim_{t\rightarrow 0^-}
\pdydxS{u}{\nu_{\VEC{x}}}\big(\VEC{x}+t\nu(\VEC{x})\big) \\
&= \frac{1}{2} \phi(\VEC{x}) + (T_K \phi)(\VEC{x}) + 
\pdydx{u}{\nu^-}(\VEC{x})
\end{align*}
for all $\VEC{x} \in S$, where the last equality comes from
Theorem~\ref{pot_double_layer} and the definition of
$\displaystyle \pdydx{u}{\nu^-}$.  Thus,
\[
\pdydx{u}{\nu^-}(\VEC{x}) =
-\frac{1}{2} \phi(\VEC{x}) + (T^\ast_K \phi)(\VEC{x})
\]
for all $\VEC{x} \in S$.  Similarly,
\begin{align*}
(T_K \phi)(\VEC{x}) + (T^\ast_K \phi)(\VEC{x}) &=
\lim_{t\rightarrow 0^+} f\big(\VEC{x}+t\nu(\VEC{x})\big)
= \lim_{t\rightarrow 0^+} v\big(\VEC{x}+t\nu(\VEC{x})\big)
+ \lim_{t\rightarrow 0^+}
\pdydxS{u}{\nu_{\VEC{x}}}\big(\VEC{x}+t\nu(\VEC{x})\big) \\
&= -\frac{1}{2} \phi(\VEC{x}) + (T_K \phi)(\VEC{x}) + 
\pdydx{u}{\nu^+}(\VEC{x})
\end{align*}
for all $\VEC{x} \in S$, where the last equality comes from
Theorem~\ref{pot_double_layer} and the definition of
$\displaystyle \pdydx{u}{\nu^+}$.  Thus,
\[
\pdydx{u}{\nu^+}(\VEC{x}) =
\frac{1}{2} \phi(\VEC{x}) + (T^\ast_K \phi)(\VEC{x})
\]
for all $\VEC{x} \in S$.

\stage{iii}
From Theorem~\ref{pot_double_layer}, we have that
\[
\sup_{\VEC{x}\in S}
\left|v\big(\VEC{x}+t\nu(\VEC{x})\big) - v^{[i]}(\VEC{x})\right|
= \sup_{\VEC{x}\in S}
\left|v\big(\VEC{x}+t\nu(\VEC{x})\big) -
\frac{1}{2} \phi(\VEC{x}) - (T_K \phi)(\VEC{x}) \right|
\rightarrow 0 \quad \text{as} \quad t\rightarrow 0^-\ .
\]
From the claim in (i) above, we have that
\begin{align*}
&\sup_{\VEC{x}\in S}
\left| \left(v\big(\VEC{x}+t\nu(\VEC{x})\big)
+ \pdydxS{u}{\nu}\big(\VEC{x}+t\nu(\VEC{x})\big)\right) -
\bigg( (T_K \phi)(\VEC{x}) + (T_K^\ast \phi)(\VEC{x})\bigg) \right| \\
&\qquad = \sup_{\VEC{x}\in S}
\left| f\big(\VEC{x}+t\nu(\VEC{x})\big) - f(\VEC{x}) \right|
\rightarrow 0 \quad \text{as} \quad t \rightarrow 0^- \ .
\end{align*}
Thus,
\[
\sup_{\VEC{x}\in S} \left| w_t(\VEC{x})
- \pdydx{u}{\nu^-}(\VEC{x}) \right|
= \sup_{\VEC{x}\in S} \left| 
\pdydxS{u}{\nu}\big(\VEC{x}+t\nu(\VEC{x})\big)
+ \frac{1}{2} \phi(\VEC{x}) - (T_K^\ast \phi)(\VEC{x}) \right|
\to 0 \quad \text{as} \quad
t\rightarrow 0^- \ .
\]
This also proofs
that (\ref{pot_unif_ddn}) is satisfied and thus completes the proof that
$u \in C_{\nu}(\Omega)$.

The proof that $\displaystyle w_t \rightarrow \pdydx{u}{\nu^+}$
uniformly on $S$ as $t\rightarrow 0^+$ is similar.  We then also get
that (\ref{pot_ddn_unif}) is satisfied and thus complete the proof
that $u \in C_{\nu}(\Omega^{\prime})$.
\end{proof}

\begin{cor} \label{pot_cor_pdpdm}
In the previous theorem,
$\displaystyle \phi = \pdydx{u}{\nu^+}- \pdydx{u}{\nu^-}$ on $S$.
\end{cor}

The following two lemmas will be useful in the next section.

\begin{lemma} \label{pot_sol_zeroB}
If $\phi \in C(S)$ and $\displaystyle f = \frac{1}{2} \phi + T^\ast_K \phi$ on
$S$, where $K$ is defined in (\ref{pot_K_defn}), then
$\displaystyle \int_{S} f \dx{S} = \int_{S} \phi \dx{S}$.
\end{lemma}

\begin{proof}
The result is a consequence of Fubini's theorem and
Lemma~\ref{pot_lem_dbl3}.
\begin{align*}
\int_S f(\VEC{x}) \dx{S} &= \frac{1}{2} \int_S \phi(\VEC{x}) \dx{S} + 
\int_S \int_S K^\ast(\VEC{x},\VEC{y}) \phi(\VEC{y}) \dss{S}{y} \dss{S}{x} \\
&= \frac{1}{2} \int_S \phi(\VEC{x}) \dx{S} + 
\int_S \left( \int_S K^\ast(\VEC{x},\VEC{y}) \dss{S}{x} \right)
\phi(\VEC{y}) \dss{S}{y} \\
&= \frac{1}{2} \int_S \phi(\VEC{x}) \dx{S} + 
\int_S \left( \int_S K(\VEC{y},\VEC{x}) \dss{S}{x} \right)
\phi(\VEC{y}) \dss{S}{y} \\
&= \frac{1}{2} \int_S \phi(\VEC{x}) \dx{S}
+ \frac{1}{2} \int_S \phi(\VEC{y}) \dx{S}
= \int_S \phi(\VEC{x}) \dx{S} \ . \qedhere
\end{align*}
\end{proof}

\begin{lemma} \label{pot_1to1n2}
Suppose that $n=2$ and $\phi \in C(S)$.  Let $u$ be the
the single layer potential with moment $\phi$.
\begin{enumerate}
\item $u$ is harmonic at infinity if and only if
$\displaystyle \int_{S} \phi(\VEC{x}) \dx{S} =0$.
\item $u$ harmonic at infinity implies that $u$ vanishes at infinity.
\item If $\displaystyle \int_{S} \phi(\VEC{x}) \dx{S} =0$
and $u$ is constant on $\overline{\Omega}$, then $\phi=0$ on
$S$, and therefore $u=0$ on $\overline{\Omega}$.
\end{enumerate}
\end{lemma}

\begin{proof}
Since $S$ is a compact set, we have that
$\displaystyle M = \max_{\VEC{y} \in S} \|\VEC{y}\| < \infty$.
For $\|\VEC{x}\| > M$, we have that
\[
\|\VEC{x}\| + M \geq \|\VEC{x}\| + \|\VEC{y}\| \geq \|\VEC{x} - \VEC{y} \|
\geq \|\VEC{x}\| - \|\VEC{y}\|
\geq \|\VEC{x}\| - M
\]
and therefore
\[
\left| \ln\left(\|\VEC{x}-\VEC{y}\|\right) -
\ln\left(\|\VEC{x}\|\right) \right|
= \left| \ln\left(\frac{\|\VEC{x}-\VEC{y}\|}{\|\VEC{x}\|}\right) \right|
\leq \max \left\{
\left| \ln\left(1 + \frac{M}{\|\VEC{x}\|}\right) \right|,
\left| \ln\left(1 - \frac{M}{\|\VEC{x}\|}\right) \right| \right\}
\]
for all $\VEC{y} \in S$.   It follows that
$\ln\left(\|\VEC{x}-\VEC{y}\|\right) - \ln\left(\|\VEC{x}\|\right) \to 0$
uniformly for $\VEC{y} \in S$ as $\|\VEC{x}\| \to \infty$.
Hence,
\[
\lim_{\|\VEC{x}\|\to 0} \int_S \left( \ln(\|\VEC{x}-\VEC{y}\|) - \ln(\|\VEC{x}\|)
\right) \phi(\VEC{y}) \dss{S}{y} = 0 \ .
\]
We get that
\begin{align*}
\frac{u(\VEC{x})}{\ln(\|\VEC{x}\|)}
&= \frac{1}{\omega_2 \ln(\|\VEC{x}\|)}
\int_S \left( \ln(\|\VEC{x}-\VEC{y}\|) - \ln(\|\VEC{x}\|)
\right) \phi(\VEC{y}) \dss{S}{y}
+ \frac{1}{\omega_2} \int_S \phi(\VEC{y}) \dss{S}{y} \\
&\to \frac{1}{\omega_2} \int_S \phi(\VEC{y}) \dss{S}{y}
\quad \text{as} \quad \|\VEC{x}\| \to \infty \ .
\end{align*}
It follows from Proposition~\ref{pot_infty_u} that $u$ is harmonic at
infinity if and only if
$\displaystyle \int_S \phi(\VEC{y}) \dss{S}{y} = 0$.
Moreover, if $\displaystyle \int_S \phi(\VEC{y}) \dss{S}{y} = 0$, then
\begin{align*}
u(\VEC{x})
&= \frac{1}{\omega_2}
\int_S \left( \ln(\|\VEC{x}-\VEC{y}\|) - \ln(\|\VEC{x}\|)
\right) \phi(\VEC{y}) \dss{S}{y}
+ \frac{\ln(\|\VEC{x}\|)}{\omega_2} \int_S \phi(\VEC{y}) \dss{S}{y} \\
&= \frac{1}{\omega_2}
\int_S \left( \ln(\|\VEC{x}-\VEC{y}\|) - \ln(\|\VEC{x}\|)
\right) \phi(\VEC{y}) \dss{S}{y}
\to 0 \quad \text{as} \quad \|\VEC{x}\| \to \infty \ .
\end{align*}
Thus $u$ vanishes at infinity.

To prove the last statement of the lemma, suppose that
$u(\VEC{x}) = c$, a constant, for all $\VEC{x} \in \overline{\Omega}$.
Thus, in particular, $u(\VEC{x}) = c$ for all $\VEC{x} \in S$.
According to (1), $u$ is also harmonic at infinity.  Thus, $u$ is a
solution of the exterior Dirichlet problem (N2) with $f(\VEC{x}) = c$
for all $\VEC{x} \in S$.   But the solution of this problem is
unique and $u(\VEC{x}) = c$ for all
$\displaystyle \VEC{x} \in \Omega^{\prime}$ is a
solution.  Therefore, $u(\VEC{x}) = c$ for all
$\displaystyle \VEC{x} \in \RR^n$.
Since $u$ is constant, it follows from Corollary~\ref{pot_cor_pdpdm}
that $\phi(\VEC{x}) = 0$ for all $\VEC{x} \in S$.
\end{proof}

\subsection{Existence of Solutions}

The following subspaces of $\displaystyle L^2(\partial \Omega)$ will play a
fundamental role in the solutions of the problems (D1), (D2), (N1) and
(N2) given at the beginning of the chapter.
\begin{align*}
V_+ &= \left\{ \phi \in L^2(\partial \Omega) : T_K \phi = \frac{1}{2}
\phi \right\} \quad , \quad
W_+ = \left\{ \phi \in L^2(\partial \Omega) : T_K^\ast \phi = \frac{1}{2}
\phi \right\} \ , \\
V_- &= \left\{ \phi \in L^2(\partial \Omega) : T_K \phi = - \frac{1}{2}
\phi \right\} \quad \text{and} \quad
W_- = \left\{ \phi \in L^2(\partial \Omega) : T_K^\ast \phi = -\frac{1}{2}
\phi \right\} \ .
\end{align*}
It follows from Lemma~\ref{pot_lem_dbl2} and
Proposition~\ref{pot_K_LtwoComp} that $T_K$ and
$\displaystyle T_K^\ast$ are compact
linear operators from $\displaystyle L^2(\partial \Omega)$ to itself.

\begin{prop} \label{pot_split_prop}
We have
\begin{enumerate}
\item $\dim V_+ = \dim W_+ = m$, where $m$ is the number of connected
components of $\Omega$.
\item $\displaystyle \dim V_- = \dim W_- = m^{\prime}$, where $m'$ is
the number of bounded and connected components of
$\displaystyle \Omega^{\prime}$.
\item If $\displaystyle \VEC{a} \in \CC^{m^{\prime}}$, then there
exists a unique $\phi_{\VEC{a}} \in W_-$ such that the single layer
potential $w$ with moment $\phi_{\VEC{a}}$ satisfies
$\displaystyle w\big|_{\Omega_0^{\prime}}=0$ and
$\displaystyle w\big|_{\Omega_j^{\prime}} = a_j$ for
$1 \leq j \leq m^{\prime}$.
\item If $n>2$ and $\displaystyle \VEC{a} \in \CC^m$, then there
exists a unique $\phi_{\VEC{a}} \in W_+$ such that the single layer
potential $u$ with moment $\phi_{\VEC{a}}$ satisfies
$\displaystyle u\big|_{\Omega_j} = a_j$
for $1 \leq j \leq m$.
\item If $n=2$, there exists an $(m-1)$-dimensional subspace $X$ of
$\displaystyle \CC^m$ with the following properties.
\begin{enumerate}
\item $\displaystyle \CC^m = X \oplus \CC\left\{(1,1,1,\ldots,1)\right\}$.
\item If $\VEC{a}\in X$, there exists a unique
\[
\phi_{\VEC{a}} \in W_+^0 = \left\{ \phi \in W_+ :
\int_{\partial \Omega} \phi(\VEC{x}) \dss{S}{x} = 0 \right\}
\]
such that the single layer potential $u$ with moment $\phi_{\VEC{a}}$
satisfies $\displaystyle u\big|_{\Omega_j} = a_j$
for $1 \leq j \leq m$.
\end{enumerate}
\end{enumerate}
\end{prop}

\begin{proof}
Consider $\phi_i : \partial \Omega \rightarrow \RR$ for $1\leq i \leq m$, and
$\displaystyle \phi_j^{\prime} : \partial \Omega \rightarrow \RR$ for
$\displaystyle 1\leq j \leq m^{\prime}$, defined by
\begin{equation} \label{pot_phiI_phiJc}
\phi_i(\VEC{x}) =
\begin{cases}
1 & \quad \text{if} \quad \VEC{x} \in \partial \Omega_i \\
0 & \quad \text{otherwise}
\end{cases}
\qquad \text{and} \qquad
\phi_j^{\prime}(\VEC{x}) =
\begin{cases}
1 & \quad \text{if} \quad \VEC{x} \in \partial \Omega_j^{\prime} \\
0 & \quad \text{otherwise}
\end{cases}
\end{equation}
respectively.

\stage{i} It follows from Lemma~\ref{pot_lem_dbl3} (see
Remark~\ref{pot_lem_dbl3_rmk}) with
$\partial \Omega$ replaced by $\partial \Omega_i$ that
\[
(T_K \phi_i)(\VEC{x}) = \int_{\partial \Omega_i} K(\VEC{x},\VEC{y})
\dss{S}{y} = \frac{1}{2} \phi_i(\VEC{x})
\]
for all $\VEC{x} \in \partial \Omega$.
Thus $\phi_i \in V_+$ for $1 \leq i \leq m$.  Since the $\phi_i$
are linearly independent, $\dim V_+ \geq m$.

\stage{ii} We have that
\[
(T_K \phi_j^{\prime})(\VEC{x})
= \int_{\partial \Omega_j^{\prime}} K(\VEC{x},\VEC{y}) \dss{S}{y}
= - \sum_{i\in I} \int_{\partial \Omega_i} K(\VEC{x},\VEC{y}) \dss{S}{y} \ ,
\]
where $I\subset \{1,2,\ldots, m\}$ is the set of indices such that
$\displaystyle \partial \Omega_j^{\prime} = \bigcup_{i\in I} \partial \Omega_i$.
The minus sign in front of the sum comes from the change of orientation
between the integral over $\displaystyle \partial \Omega_j^{\prime}$
and the integrals over the $\partial \Omega_i$.
If $\displaystyle \VEC{x} \in \partial \Omega_j^{\prime}$, then
$\VEC{x} \in \partial \Omega_i$ for one and only one $i_0 \in I$.  In
particular, $\VEC{x}$ is in the complement of $\overline{\Omega_i}$
for all $i \neq i_0$.  As in (i), it follows from
Lemma~\ref{pot_lem_dbl3} with
$\partial \Omega$ replace by $\partial \Omega_{i}$ that
\[
\int_{\partial \Omega_i} K(\VEC{x},\VEC{y}) \dss{S}{y} =
\begin{cases}
1/2 & \quad \text{if} \ i = i_0 \\
0 & \quad \text{if} \ i \neq i_0
\end{cases}
\]
Thus
\[
  (T_K \phi_j^{\prime})(\VEC{x}) = - \frac{1}{2} \phi_j^{\prime}(\VEC{x})
\]
for all $\displaystyle \VEC{x} \in \partial \Omega^{\prime}$.
Hence, $\displaystyle \phi_j^{\prime} \in V_-$ for
$\displaystyle 1 \leq j \leq m^{\prime}$.
Since the $\displaystyle \phi_j^{\prime}$ are linearly independent,
$\displaystyle \dim V_- \geq m^{\prime}$.

\stage{iii}
Since $\displaystyle -\frac{1}{2} \Id + T_K$ and
$\displaystyle \frac{1}{2} \Id + T_K$ are compact operators, it
follows from Theorem~\ref{fu_an_comp_oper} that
$\displaystyle \dim\ \KE \left(-\frac{1}{2} \Id + T_K\right) =
\dim\ \KE \left(-\frac{1}{2} \Id + T_K^\ast \right)$
and $\displaystyle \dim\ \KE \left(\frac{1}{2} \Id + T_K\right) =
\dim\ \KE \left(\frac{1}{2} \Id + T_K^\ast \right)$.

Since $\displaystyle V_+ = \KE \left(-\frac{1}{2} \Id + T_K\right)$
and $\displaystyle W_+ = \KE \left(-\frac{1}{2} \Id + T_K^\ast \right)$, 
we get that $\dim V_+ = \dim W_+$.  Since
$\displaystyle V_- = \KE \left(\frac{1}{2} \Id + T_K\right)$
and $\displaystyle W_- = \KE \left(\frac{1}{2} \Id + T_K^\ast \right)$, 
we get that $\dim V_- = \dim W_-$.

\stage{iv} If $\phi \in W_+$ and $u$ is the single layer potential with
moment $\phi$, then $\displaystyle \pdydx{u}{\nu^-}(\VEC{x}) = 0$
for $\VEC{x} \in \partial \Omega$ as a consequence of
Theorem~\ref{pot_slp_exist}.  By uniqueness (modulo a
function which is constant on each connected component of $\Omega$)
of the solution for the interior Neumann problem (N1), we have that
$u$ is constant on each connected component of $\Omega$.  We may
therefore define the mapping
\begin{align*}
\Lambda : W_+ & \rightarrow \CC^m \\
\phi &\mapsto \left( u\big|_{\Omega_1}, u\big|_{\Omega_2},
\ldots, u\big|_{\Omega_m}\right)
\end{align*}

\stage{iv.a} $\Lambda$ is one-to-one for $n>2$.  Suppose that
$\Lambda u_1 = \Lambda u_2$, where $u_1$ and $u_2$ are the single
layer potentials with moments $\phi_1$ and $\phi_2$ respectively.
Then $u=u_1-u_2$ is the single layer potential with moment
$\phi = \phi_1-\phi_2$.  Moreover, $u = 0$ on each component of
$\Omega$.  Thus, $u = 0$ on $\overline{\Omega}$.  By uniqueness of
solution for the exterior Dirichlet problem (D2) given by
$\Delta v = 0$ on $\displaystyle \Omega^{\prime} \cup \{\infty\}$ 
with $\displaystyle v\big|_{\partial \Omega^{\prime}}=0$,
we have that $u=0$ on $\Omega^{\prime}$ because $v \equiv 0$ is a
solution of this problem.  Hence, $u=0$ on $\displaystyle \RR^n$.
We have seen in item (2) following Definition~\ref{pot_spl_def} that
$u$ is harmonic at infinity.
It follows from Corollary~\ref{pot_cor_pdpdm} that $\phi =0$ on
$\partial \Omega$.  Thus $\phi_1=\phi_2$ on $\partial \Omega$.

This shows that $\dim W_+ \leq \dim \Lambda(W_+) \leq m$.  Combined
with (i) and (iii), this shows that $\dim W_+ = m$.  Thus proving
{\em item (1) of the proposition for $n>2$}.

We also have that $\displaystyle \Lambda\big|_{W_+}$ is onto because
$\dim \Lambda(W_+) = m$.  Hence, {\em item (4) of the proposition} is
true because $\displaystyle \Lambda\big|_{W_+}$ is one-to-one and onto
$\displaystyle \CC^m$.

\stage{iv.b}  If $n=2$, $\Lambda$ defined above may not be one-to-one
because the single layer potential $u$ with moment $\phi$ may not be
harmonic at infinity.  Let
\[
W_+^0 = \left\{ \phi \in W_+ : \int_{\partial \Omega} \phi(\VEC{x})
\dss{S}{x} = 0 \right\} \ .
\]
Using Lemma~\ref{pot_1to1n2}, a reasoning very similar to the one
given in (iv.a) shows that
$\displaystyle \Lambda\big|_{W_+^0} : W_+^0 \rightarrow \CC^m$ is
one-to-one.  In particular, this proves that $\displaystyle \dim W_+^0 \leq m$.

Moreover, item (3) of Lemma~\ref{pot_1to1n2} shows that the space
$\displaystyle Y = \left\{ \alpha (1,1,\ldots,1) : \alpha \in \CC \right\}$
is not in the image $X$ of $\Lambda\big|_{W_+^0}$.
Thus $\displaystyle \dim W_+^0 \leq m-1$ because
$\Lambda\big|_{W_+^0}$ is one-to-one.
Also $\displaystyle \dim W_+^0 \geq \dim W_+ - 1$ because
$\displaystyle W_+^0$ is the kernel of the
linear functional
$\displaystyle \phi \mapsto \int_{\partial \Omega} \phi(\VEC{x}) \dss{S}{x}$
from $W_+$ to $\CC$.
Therefore, $\dim W_+ \leq m$.  Combined with (i) and (iii), this shows
that {\em item (1) of the proposition is true for $n=2$}.

Since $\dim W_+ = m$, it follows that $\displaystyle \dim W_+^0 = m - 1$.
Hence, $\displaystyle X = \Lambda(W_+^0)$ is of dimension $m-1$ because
$\displaystyle \Lambda\big|_{W_+^0}$ is one-to-one.
We then get that
$\displaystyle \CC^m = X \oplus \CC\left\{(1,1,1,\ldots,1)\right\}$.
This completes the proof of {\em item (5) of the proposition} because
$\displaystyle \Lambda\big|_{W_+^0}$ is one-to-one and onto $X$.

\stage{v}
We first claim that if $\phi \in W_-$ and $u$ is the single layer
potential with moment $\phi$, then $u$ is constant on each bounded
component $\displaystyle \Omega_j^{\prime}$ of
$\displaystyle \Omega^{\prime}$ and $u=0$ on the unbounded
component $\displaystyle \Omega_0^{\prime}$.  We have from
Theorem~\ref{pot_slp_exist}
that $\displaystyle \pdydx{u}{\nu^+}(\VEC{x}) = 0$
for $\displaystyle \VEC{x} \in \partial \Omega^{\prime}$.  For $n>2$,
the claim is a consequence of Proposition~\ref{pot_nec_Nprobl0}.  For $n=2$,
Proposition~\ref{pot_nec_Nprobl0} still implies that $u$ is constant
on each bounded component of $\displaystyle \Omega^{\prime}$ and also
on $\displaystyle \Omega_0^{\prime}$.
From Lemma~\ref{pot_sol_zeroB}, we have that
$\displaystyle \int_{\partial \Omega} \phi(\VEC{x}) \dss{S}{x} = 0$.
Hence, Lemma~\ref{pot_1to1n2} implies that $u$ vanishes at infinity.  Since
$u$ is constant on $\displaystyle \Omega_0^{\prime}$, $u=0$ on
$\displaystyle \Omega_0^{\prime}$.
This completes the proof of the claim.

We may therefore define the mapping
\begin{align*}
\Lambda : W_- & \rightarrow \CC^{m^{\prime}} \\
\phi &\mapsto \left( u\big|_{\Omega^{\prime}_1}, u\big|_{\Omega^{\prime}_2},
\ldots, u\big|_{\Omega^{\prime}_{m^{\prime}}}\right)
\end{align*}

We now show that $\Lambda$ is one-to-one.  Suppose that
$\Lambda u_1 = \Lambda u_2$, where $u_1$ and $u_2$ are the single
layer potentials with moments $\phi_1$ and $\phi_2$ respectively.
Then $u=u_1-u_2$ is the single layer potential with moment
$\phi = \phi_1-\phi_2$.  Moreover, $u = 0$ on each bounded component of
$\displaystyle \Omega^{\prime}$.  We also have that $u = 0$ on the unbounded
components since this is true for $u_1$ and $u_2$.   Thus,
$u=0$ on $\displaystyle \overline{\Omega^{\prime}}$.
By uniqueness of the interior Dirichlet problem (D1) given by
$\Delta v =0$ on $\Omega$ with $\displaystyle v\big|_{\partial \Omega}=0$,
we have that $u=0$ on $\Omega$ because $v\equiv 0$ is a solution of this
problem.   Hence, $u = 0$ on $\displaystyle \RR^n$.  It follows from
Corollary~\ref{pot_cor_pdpdm} that $\phi =0$ on
$\displaystyle \partial \Omega^{\prime}$.
Thus $\phi_1=\phi_2$ on $\displaystyle \partial \Omega^{\prime}$.

This shows that
$\displaystyle \dim W_- \leq \dim \Lambda(W_-) \leq m^{\prime}$.  Combined
with (ii), this shows that $\displaystyle \dim W_- = m^{\prime}$ and proves
{\em item (2) of the proposition}.

We also have that $\displaystyle \Lambda\big|_{W_-}$ is onto because
$\dim \Lambda(W_-) = m^{\prime}$.  Hence, 
{\em item (3) of the proposition} is true because
$\displaystyle \Lambda\big|_{W_-}$ is one-to-one and onto
$\displaystyle \CC^{m^{\prime}}$.
\end{proof}

\begin{prop} \label{pot_L2_split1}
\[
L^2(\partial \Omega) = V_+^\perp \oplus W_+ = V_-^\perp \oplus W_- \ .
\]
\end{prop}

In the previous proposition, the orthogonality is based on the scalar
product on $\displaystyle L^2(\partial \Omega)$ defined by
\[
\ps{f}{g} = \int_{\partial \Omega} f(\VEC{x})\,\overline{g(\VEC{x})} \dx{S}
\]
for $\displaystyle f,g \in L^2(\partial \Omega)$.

\begin{proof}
\stage{i} We first prove that
$\displaystyle L^2(\partial \Omega) = V_+^\perp \oplus W_+$.  It
follows from Proposition~\ref{pot_split_prop} that
$\displaystyle V_+^\perp$ is a
closed subspace of codimension $m$ in
$\displaystyle L^2(\partial \Omega)$ and $W_+$
is a subspace of dimension $m$ in
$\displaystyle L^2(\partial \Omega)$.  Thus, we
only have to prove that $\displaystyle V_+^\perp \cap W_+ = \{0\}$.

Suppose that $\displaystyle \phi \in V_+^\perp \cap W_+$.
Since $\phi \in W_+$, we have that
$\displaystyle \phi \in \KE \left(T^\ast_K - \frac{1}{2} \Id\right)$.
Since $\displaystyle \phi \in V_+^\perp
= \left(\KE \left(T_k -\frac{1}{2} \Id\right)\right)^\perp$,
it follows from Theorem~\ref{fu_an_RIOrth} that
$\displaystyle \phi \in \IMG \left( T^\ast_K - \frac{1}{2} \Id\right)$
because $\displaystyle T_k^\ast$ is a compact operator
and so $\displaystyle \IMG \left( T^\ast_K - \frac{1}{2} \Id\right)$
is a close subspace according to Theorem~\ref{fu_an_comp_oper}.  Thus, there
exists $\displaystyle \psi \in L^2(\partial \Omega)$ such that
$\displaystyle \phi = T^\ast_K \psi - \frac{1}{2} \psi$.  Moreover,
since $\displaystyle T^\ast_K\phi - \frac{1}{2} \phi = 0$, it follows
from Proposition~\ref{pot_compt_cont_K} that
$\phi \in C(\partial \Omega)$.  We can use 
Proposition~\ref{pot_compt_cont_K} a second time to conclude that
$\psi \in C(\partial \Omega)$ because
$\displaystyle T^\ast_K \psi - \frac{1}{2} \psi = \phi$.

Let $u$ and $v$ be the simple layer potentials with moments $\phi$ and
$\psi$ respectively.  We have that $\displaystyle u,v \in C(\RR^n)$
according to Proposition~\ref{pot_slp_cont}.

From Theorem~\ref{pot_slp_exist}, we have
that
$\displaystyle \pdydx{u}{\nu^-} = T^\ast_K \phi - \frac{1}{2} \phi = 0$
and
$\displaystyle \pdydx{v}{\nu^-} = T^\ast_K \psi - \frac{1}{2} \psi
= \phi = T^\ast_K \phi + \frac{1}{2} \phi
= \pdydx{u}{\nu^+}$
on $\partial \Omega$.  Hence,
\begin{equation} \label{pot_L2_split1_E1}
\int_{\partial \Omega} \left( u \pdydx{v}{\nu^-}
- v \pdydx{u}{\nu^-} \right) \dx{S}
= \int_{\partial \Omega} u \pdydx{u}{\nu^+} \dx{S} \ ,
\end{equation}
where the positive direction on $\partial \Omega$ is such that the
normal to $\partial \Omega$ points outside $\Omega$, so inside
$\displaystyle \Omega^{\prime}$.

From the Green's identity (\ref{laplace_green2}), we have that
\begin{equation} \label{pot_L2_split1_E2}
\int_{\partial \Omega} \left( u \pdydx{v}{\nu^-}
- v \pdydx{u}{\nu^-} \right) \dx{S}
= \int_\Omega \left( u \Delta v -  v \Delta u \right)
\dx{\VEC{x}} = 0
\end{equation}
because $u$ and $v$ are harmonic on $\Omega$.

To compute the integral on the right side in (\ref{pot_L2_split1_E1}),
we choose $r>0$ such that
$\displaystyle \partial B_r(\VEC{0}) \subset \Omega_0^{\prime}$,
the unbounded component of $\displaystyle \Omega^{\prime}$.
We get from the Green's identities (\ref{laplace_green1}) that
\begin{align}
\int_{\partial \Omega} u \pdydx{u}{\nu^+} \dx{S} 
&= -\int_{\Omega^{\prime} \cap B_r(\VEC{0})}
\left( u \Delta u + \graD u \cdot \graD u \right)
\dx{\VEC{x}} + \int_{\partial B_r(\VEC{0})} u \pdydx{u}{\nu} \dx{S}
\nonumber \\
&= -\int_{\Omega^{\prime} \cap B_r(\VEC{0})} \|\graD u\|_2^2 \dx{\VEC{x}}
+ \int_{\partial B_r(\VEC{0})} u \pdydx{u}{\nu} \dx{S}
\label{pot_L2_split1_E3}
\end{align}
because $u$ is harmonic in $\displaystyle \Omega^{\prime}$.  The
normal to $\displaystyle \partial (\Omega^{\prime} \cap B_r(\VEC{0}))$
points inside $\displaystyle \Omega^{\prime} \cap B_r(\VEC{0})$.

We now show that the last integral in (\ref{pot_L2_split1_E3}) converges
to $0$ as $r \to \infty$.  We get from item (2) following
Definition~\ref{pot_spl_def} that $u$ is harmonic at infinity for
$n>2$.  To prove that $u$ is harmonic at infinity for $n=2$ requires a
little bit more work.  Since
$\Omega^{\prime} \cap B_r(\VEC{0})\phi \in V_+^\perp$, we have that
\[
\int_{\partial \Omega} \phi \dx{S}
= \int_{\partial \Omega} \phi \left( \sum_{i=0}^m \phi_i\right) \dx{S}
= \sum_{i=0}^m \int_{\partial \Omega} \phi \,\phi_i \dx{S} = 0 \ ,
\]
where the functions $\phi_i \in V_+$ are defined at the beginning of
the proof of Proposition~\ref{pot_split_prop}.  Note that
$\displaystyle \sum_{i=0}^m \phi_i(\VEC{x}) = 1$ for all
$\VEC{x} \in \partial \Omega$.  It follows from
Lemma~\ref{pot_1to1n2} that $u$ is harmonica at infinity.

If $n>2$, we have from Proposition~\ref{pot_infty_u} and
Proposition~\ref{pot_infty_ddu} that
$\Omega^{\prime} \cap B_r(\VEC{0})|u(\VEC{x})| \leq C_1\|\VEC{x}\|^{2-n}$ and
$\displaystyle \left| \pdydx{u}{\nu}(\VEC{x})\right| \leq
C_2 \|\VEC{x}\|^{1-n}$ as $\|\VEC{x}\| \rightarrow \infty$ for some
constants $C_1$ and $C_2$.  Hence,
\[
\left| \int_{\partial B_r(\VEC{0})} u \pdydx{u}{\nu} \dx{S} \right|
\leq \int_{\partial B_r(\VEC{0})} |u|\,\left|\pdydx{u}{\nu} \right| \dx{S}
\leq C_1C_2 r^{3-2n} \int_{\partial B_r(\VEC{0})} \dx{S}
= C_1C_2 \omega_n r^{2-n}  \to 0 
\]
as $r \to \infty$.
If $n=2$, we have from Proposition~\ref{pot_infty_u} and
Proposition~\ref{pot_infty_ddu} that
$|u(\VEC{x})| \leq C_3|\ln(\|\VEC{x}\|)|$ and
$\displaystyle \left| \pdydx{u}{\nu}(\VEC{x})\right| \leq
C_4 \|\VEC{x}\|^{-2}$ as $\|\VEC{x}\| \rightarrow \infty$ for some
constants $C_3$ and $C_4$..  Hence,
\[
\left| \int_{\partial B_r(\VEC{0})} u \pdydx{u}{\nu} \dx{S} \right|
\leq \int_{\partial B_r(\VEC{0})} |u|\,\left|\pdydx{u}{\nu} \right| \dx{S}
\leq C_3C_4 |\ln(r)| r^{-2} \int_{\partial B_r(\VEC{0})} \dx{S}
= C_3C_4 \omega_2 \frac{|\ln(r)|}r \to 0 
\]
as $r \to \infty$.

Therefore, if we let $r \to \infty$ in (\ref{pot_L2_split1_E3}), we
get from (\ref{pot_L2_split1_E1}) and (\ref{pot_L2_split1_E2}) that\\
$\displaystyle \int_{\Omega^{\prime}} \|\graD u\|_2^2 \dx{\VEC{x}} = 0$.
Thus $\graD u = 0$ on $\displaystyle \Omega^{\prime}$.  This shows that $u$ is
constant on each component of $\displaystyle \Omega^{\prime}$.  From
this, we get that $\displaystyle \phi = \pdydx{u}{\nu^+} = 0$.

\stage{ii} The proof that
$\displaystyle L^2(\partial \Omega) = V_-^\perp \oplus W_-$ is very
similar to the proof given in (i). It
follows from Proposition~\ref{pot_split_prop} that $V_-^\perp$ is a
closed subspace of codimension $\displaystyle m^{\prime}$ in
$\displaystyle L^2(\partial \Omega)$ and $W_-$
is a subspace of dimension $\displaystyle m^{\prime}$ in
$\displaystyle L^2(\partial \Omega)$.  Thus, we
only have to prove that $\displaystyle V_-^\perp \cap W_- = \{0\}$.

Suppose that $\displaystyle \phi \in V_-^\perp \cap W_-$.
Since $\phi \in W_-$, we have that
$\displaystyle \phi \in \KE \left(T^\ast_K + \frac{1}{2} \Id\right)$.
Since $\displaystyle \phi \in V_-^\perp
= \left(\KE \left(T_k + \frac{1}{2} \Id\right)\right)^\perp$,
it follows from Theorem~\ref{fu_an_RIOrth} that
$\displaystyle \phi \in \IMG \left( T^\ast_K + \frac{1}{2} \Id\right)$ because
$\displaystyle T_k^\ast$ is a compact operator
and so $\displaystyle \IMG \left( T^\ast_K + \frac{1}{2} \Id\right)$
is a close subspace according to Theorem~\ref{fu_an_comp_oper}.  Thus, there
exists $\displaystyle \psi \in L^2(\partial \Omega)$ such that
$\displaystyle \phi = T^\ast_K \psi + \frac{1}{2} \psi$.  Moreover,
since $\displaystyle T^\ast_K\phi + \frac{1}{2} \phi = 0$, it follows
from Proposition~\ref{pot_compt_cont_K} that
$\phi \in C(\partial \Omega)$.  We can use 
Proposition~\ref{pot_compt_cont_K} a second time to conclude that
$\psi \in C(\partial \Omega)$ because
$\displaystyle T^\ast_K \psi + \frac{1}{2} \psi = \phi$.

Let $u$ and $v$ be the simple layer potentials with moments $\phi$ and
$\psi$ respectively.    We have that $\displaystyle u,v \in C(\RR^n)$
according to Proposition~\ref{pot_slp_cont}.

From Theorem~\ref{pot_slp_exist}, we have that
\[
  \pdydx{u}{\nu^+} = T^\ast_K \phi + \frac{1}{2} \phi = 0
\quad \text{and} \quad
  \pdydx{v}{\nu^+} = T^\ast_K \psi + \frac{1}{2} \psi
= \phi = - T^\ast_K \phi + \frac{1}{2} \phi
= - \pdydx{u}{\nu^-}
\]
on $\partial \Omega$.  Hence,
\begin{equation} \label{pot_L2_split1_E4}
\int_{\partial \Omega} \left( u \pdydx{v}{\nu^+}
- v \pdydx{u}{\nu^+} \right) \dx{S}
= - \int_{\partial \Omega} u \pdydx{u}{\nu^-} \dx{S} \ ,
\end{equation}
where the positive direction on $\partial \Omega$ is such that the
normal to $\partial \Omega$ points outside
$\displaystyle \Omega^{\prime}$, so inside $\Omega$.

From the Green's identity (\ref{laplace_green1}), we get that
\begin{equation} \label{pot_L2_split1_E5}
\int_{\partial \Omega} u \pdydx{u}{\nu^-} \dx{S} 
= -\int_{\Omega} \left( u \Delta u + \graD u \cdot \graD u \right)
\dx{\VEC{x}}
= -\int_{\Omega} \|\graD u\|_2^2 \dx{\VEC{x}}
\end{equation}
because $u$ is harmonic on $\Omega$.  The normal to $\partial \Omega$
points inside $\Omega$.

To compute the integral on the left side in (\ref{pot_L2_split1_E4}),
we choose $r>0$ such that
$\displaystyle \partial B_r(\VEC{0}) \subset \Omega_0^{\prime}$.
We get from the Green's identities (\ref{laplace_green2}) that
\begin{align}
\int_{\partial \Omega} \left( u \pdydx{v}{\nu^+}
- v \pdydx{u}{\nu^+} \right) \dx{S}
&= \int_{\Omega^{\prime} \cap B_r(\VEC{0})}
\left( u \Delta v + v \Delta u \right) \dx{\VEC{x}}
- \int_{\partial B_r(\VEC{0})} \left( u \pdydx{v}{\nu}
- v \pdydx{u}{\nu} \right) \dx{S}
\nonumber \\
&= - \int_{\partial B_r(\VEC{0})} \left( u \pdydx{v}{\nu}
- v \pdydx{u}{\nu} \right) \dx{S}
\label{pot_L2_split1_E6}
\end{align}
because $u$ and $v$ are harmonic in $\displaystyle \Omega^{\prime}$.
The normal to $\displaystyle \partial (\Omega^{\prime} \cap B_r(\VEC{0}))$
points outside $\displaystyle \Omega^{\prime} \cap B_r(\VEC{0})$.

We now show that the integral in (\ref{pot_L2_split1_E6}) converges
to $0$ as $r \to \infty$.  We get from item (2) following
Definition~\ref{pot_spl_def} that $u$ and $v$ are harmonic at infinity for
$n>2$.  We now prove that $u$ and $v$ are harmonic at infinity for $n=2$.
Since $\phi \in W_-$, we get from Lemma~\ref{pot_sol_zeroB} that
$\displaystyle \int_{\partial \Omega} \phi \dx{S} = 0$.
Moreover, since $\displaystyle T^\ast_K \psi + \frac{1}{2} \psi = \phi$,
we can use Lemma~\ref{pot_sol_zeroB} again to conclude that
$\displaystyle \int_{\partial \Omega} \psi \dx{S} = 0$.
It follows from Lemma~\ref{pot_1to1n2} that $u$ and $v$ are harmonica
at infinity.  Proceeding exactly as we did in (i), we can show that
$\displaystyle \int_{\partial B_r(\VEC{0})} u \pdydx{v}{\nu} \dx{S}$
and
$\displaystyle \int_{\partial B_r(\VEC{0})} v \pdydx{u}{\nu}  \dx{S}$
converge to $0$ as $r \to \infty$.

Therefore, if we let $r \to \infty$ in (\ref{pot_L2_split1_E6}), we
get from (\ref{pot_L2_split1_E4}) and (\ref{pot_L2_split1_E5}) that
$\displaystyle \int_{\Omega} \|\graD u\|_2^2 \dx{\VEC{x}} = 0$.
Thus $\graD u = 0$ on $\Omega$.  This shows that $u$ is
constant on each component of $\Omega$.  From this, we
that $\displaystyle \phi = \pdydx{u}{\nu^-} = 0$.
\end{proof}

\begin{cor} \label{pot_cor_splitL2}
\[
L^2(\partial \Omega) = \IMG \left( -\frac{1}{2} \Id + T_K\right)
\oplus V_+ = \IMG \left( \frac{1}{2} \Id + T_K\right) \oplus V_- \ .
\]
\end{cor}

\begin{proof}
Since $\displaystyle T_K$ is a compact operator, it
follows from Theorems~\ref{fu_an_comp_oper} and \ref{fu_an_RIOrth} that\\
$\displaystyle  \IMG \left(-\frac{1}{2} \Id + T_K\right)
= \left( \KE \left(-\frac{1}{2} \Id + T_K^\ast \right) \right)^\perp
= W_+^\perp$
and 
$\displaystyle \IMG \left(\frac{1}{2} \Id + T_K\right)
= \left( \KE \left(\frac{1}{2} \Id + T_K^\ast \right) \right)^\perp
= W_-^\perp$.
It is therefore enough to show that
$\displaystyle W_+^\perp \cap V_+ = W_-^\perp \cap V_- = \{0\}$
because $\displaystyle W_+^\perp$ has codimension $m$ and $V_+$ has
dimension $m$, and $\displaystyle W_-^\perp$ has codimension
$\displaystyle m^{\prime}$ and $V_-$ has dimension $\displaystyle m^{\prime}$.

Suppose that $\displaystyle \psi \in W^\perp_+ \cap V_+$.  It follows from
Proposition~\ref{pot_L2_split1} that $\psi = \psi_1+\psi_2$, where
$\displaystyle \psi_1 \in V^\perp_+$ and $\psi_2 \in W_+$.  Thus
\[
\ps{\psi}{\psi_1}_2 = \int_{\partial \Omega}
\psi(\VEC{x})\overline{\psi_1(\VEC{x})} \dss{S}{x} = 0
\]
because $\psi \in V_+$ and
\[
\ps{\psi}{\psi_2}_2 = \int_{\partial \Omega}
\psi(\VEC{x})\overline{\psi_2(\VEC{x})} \dss{S}{x} = 0
\]
because $\psi \in W_+^\perp$.  Hence
\[
\ps{\psi}{\psi}_2 = \int_{\partial \Omega}
\left| \psi(\VEC{x}) \right|^2 \dss{S}{x} = 0
\]
by linearity.  Thus, $\psi=0$ almost everywhere on $\partial \Omega$.
A similar proof yields $\displaystyle W_-^\perp \cap V_- = \{0\}$.
\end{proof}

\begin{theorem} \label{pot_exist_uniqu_TH}
Referring to the statement of the Dirichlet and Neumann problems given
at the beginning of the chapter, we have the following results.
\begin{enumerate}
\item ({\bfseries D1}) has a unique solution for $f \in  C(\partial \Omega)$.
\item ({\bfseries D2}) has a unique solution for
$\displaystyle f \in  C(\partial \Omega^{\prime})$.
\item ({\bfseries N1}) has a solution for $f \in  C(\partial \Omega)$
if and only if
$\displaystyle \int_{\partial \Omega_i} f(\VEC{x})\dss{S}{x}=0$
for $1 \leq i \leq m$.  The solution is unique modulo a
function which is constant on each connected component of $\Omega$.
\item ({\bfseries N2}) has a solution for
$\displaystyle f \in  C(\partial \Omega^{\prime})$ if and only if
$\displaystyle \int_{\partial \Omega_j^{\prime}} f(\VEC{x})\dss{S}{x}=0$
for $\displaystyle 1\leq j \leq m^{\prime}$ when $n>2$, and
$\displaystyle 0\leq j \leq m^{\prime}$ when $n=2$.  The solution is
unique modulo a function $h$ such that $h(\VEC{x}) = c$, a fixed
constant, on each connected component
$\displaystyle \Omega_j^{\prime}$ for $\displaystyle 1\leq j \leq m^{\prime}$.
\end{enumerate}
\end{theorem}

\begin{proof}
Uniqueness results has been proved in In
Section~\ref{pot_sect_unique}.  The need for the necessary conditions
is proved in Section~\ref{pot_sect_necessary}.  We only need to
consider the existence of the solution.

\stage{N1}  Let $\phi_i \in V_+$ for $1 \leq i \leq m$ be the
functions defined in (\ref{pot_phiI_phiJc}).  We have shown in the
proof of Proposition~\ref{pot_split_prop} that
$\displaystyle \{\phi_i\}_{i=1}^m$ is a basis of $V_+$.  Hence,
\[
V_+^\perp = \left\{ g \in L^2(\partial \Omega) :
\int_{\partial \Omega} g(\VEC{x}) \phi_i(\VEC{x}) \dx{S}
= \int_{\partial \Omega_i} g(\VEC{x}) \dx{S} = 0 \ \text{for}
\ 1 \leq i \leq m \right\} \ .
\]
From Theorems~\ref{fu_an_comp_oper} and \ref{fu_an_RIOrth},
we have that $\displaystyle g \in \IMG \left(-\frac{1}{2} \Id +T^\ast_K\right)$
if and only if\\
$\displaystyle g \in \left( \KE \left( -\frac{1}{2} \Id +
T_K\right)\right)^\perp = V_+^\perp$.

By hypothesis $\displaystyle f \in V_+^\perp$.  Therefore, there exists
$\displaystyle \phi \in L^2(\partial \Omega)$ such that
\[
\boxed{
f = -\frac{1}{2}\phi +T^\ast_K \phi \  .
}
\]
Since $f\in C(\partial \Omega)$, we have that
$\phi \in C(\partial \Omega)$ by Proposition~\ref{pot_compt_cont_K}.
Finally, from Theorem~\ref{pot_slp_exist}, the single layer potential 
with moment $\phi$ defined by
\[
\boxed{
u(\VEC{x}) = \int_{\partial \Omega} N(\VEC{x},\VEC{y})
\phi(\VEC{y}) \dss{S}{y} \quad , \quad \VEC{x} \in \overline{\Omega} \ ,
}
\]
is the solution of (N1).

\stage{N2}  Let $\displaystyle \phi_j^{\prime} \in V_-$ for
$\displaystyle 1 \leq j \leq m^{\prime}$
be the functions defined in (\ref{pot_phiI_phiJc}).  We have shown in
the proof of Proposition~\ref{pot_split_prop} that
$\displaystyle \{\phi_j^{\prime}\}_{j=1}^{m^{\prime}}$ is a basis of
$V_-$.  Hence,
\[
V_-^\perp = \left\{ g \in L^2(\partial \Omega) :
\int_{\partial \Omega^{\prime}} g(\VEC{x}) \phi_j^{\prime}(\VEC{x}) \dss{S}{x}
= \int_{\partial \Omega_j^{\prime}} g(\VEC{x}) \dss{S}{x} = 0 \ \text{for}
\ 1 \leq j \leq m^{\prime} \right\} \ .
\]
From Theorems~\ref{fu_an_comp_oper} and \ref{fu_an_RIOrth},
we have that $\displaystyle g \in \IMG \left(\frac{1}{2} \Id +T^\ast_K\right)$
if and only if\\
$\displaystyle g \in \left(\KE \left(\frac{1}{2} \Id +T_K\right)\right)^\perp
= V_-^\perp$.

By hypothesis $\displaystyle f \in V_-^\perp$.  Therefore, there exists
$\displaystyle \phi \in L^2(\partial \Omega)$ such that
\[
\boxed{
f = \frac{1}{2}\phi +T^\ast_K \phi \ .
}
\]
Since $\displaystyle f\in C(\partial \Omega^{\prime})$, we have that
$\displaystyle \phi \in C(\partial \Omega^{\prime})$ by
Proposition~\ref{pot_compt_cont_K}.
Finally, from Theorem~\ref{pot_slp_exist}, the single layer potential 
with moment $\phi$ defined by
\[
\boxed{
u(\VEC{x}) = \int_{\partial \Omega} N(\VEC{x},\VEC{y})
\phi(\VEC{y}) \dss{S}{y} \quad , \quad \VEC{x} \in \overline{\Omega^{\prime}}
\ ,
}
\]
is the solution of (N2).  If $n>2$, we showed in item (2) following
Definition~\ref{pot_spl_def} that $u$ is harmonica at infinity.  If
$n=2$, the extra condition
$\displaystyle \int_{\partial \Omega_0^{\prime}} f(\VEC{x})\dss{S}{x} = 0$ 
implies that $u$ is harmonic at infinity (and vanishes at infinity)
according to Lemmas~\ref{pot_sol_zeroB} and \ref{pot_1to1n2}.

\stage{D1} From Corollary~\ref{pot_cor_splitL2}, we may write
\[
\boxed{
f = \frac{1}{2} \phi + T_K \phi + \sum_{j=1}^{m^{\prime}} a_j \phi_j^{\prime}
}
\]
for some $\displaystyle \phi \in L^2(\partial \Omega)$ and $a_j \in \CC$.
Since $f\in C(\partial \Omega)$ and
$\displaystyle \phi_j^{\prime} \in C(\partial \Omega)$
for all $j$, we have that
$\displaystyle
\phi + 2T_K \phi = 2f - 2 \sum_{j=1}^{m^{\prime}} a_j \phi_j^{\prime}
\in C(\partial \Omega)$.
It follows from Proposition~\ref{pot_compt_cont_K} that
$\phi \in C(\partial \Omega)$.  From Theorem~\ref{pot_double_layer},
the double layer potential with moment $\phi$,
\[
\boxed{
v(\VEC{x}) = \int_{\partial \Omega}
\pdydx{N}{\nu_{\VEC{y}}}(\VEC{x},\VEC{y}) \phi(\VEC{y})
\dss{S}{y} \quad , \quad \VEC{x} \in \Omega \ ,
}
\]
can be extended to $\overline{\Omega}$ to give a continuous solution for the
interior Dirichlet problem (D1) with 
$\displaystyle \frac{1}{2} \phi + T_K \phi$ instead of $f$ as boundary
condition.
By item (3) of Proposition~\ref{pot_split_prop}, there
exists a unique $\psi \in W_-$ such that the single layer potential
with moment $\psi$,
\[
\boxed{
w(\VEC{x}) = \int_{\partial \Omega} N(\VEC{x},\VEC{y})
\psi(\VEC{y}) \dss{S}{y} \quad , \quad \VEC{x} \in \RR^n \ ,
}
\]
satisfies $\displaystyle w\big|_{\Omega_j^{\prime}} = a_j$ for
$\displaystyle 1\leq j \leq m^{\prime}$, and
$\displaystyle w\big|_{\Omega_0^{\prime}} = 0$.
Since $\displaystyle w \in C(\RR^n)$ by
Proposition~\ref{pot_slp_cont}, we have that
$\displaystyle 
w\big|_{\partial \Omega} = \sum_{j=1}^{m^{\prime}} a_j \phi_j^{\prime}$.
Hence,
\[
\boxed{
u(\VEC{x}) = v(\VEC{x}) + w(\VEC{x}) \quad , \quad \VEC{x} \in
\overline{\Omega} \ ,
}
\]
is a solution of (D1).

\stage{D2}  From Proposition~\ref{pot_cor_splitL2}, we may write
\[
\boxed{
f = - \frac{1}{2} \phi + T_K \phi + \sum_{i=1}^m a_j \phi_i
}
\]
for some $\displaystyle \phi \in L^2(\partial \Omega)$ and $a_i \in \CC$.
Since $f\in C(\partial \Omega)$ and $\phi_i \in C(\partial \Omega)$
for all $j$, we have that
$\displaystyle
\phi -2 T_K \phi = -2f + 2 \sum_{j=1}^m a_j \phi_j \in C(\partial \Omega)$.
It follows from Proposition~\ref{pot_compt_cont_K} that
$\phi \in C(\partial \Omega)$.  From Theorem~\ref{pot_double_layer},
the double layer potential with moment $\phi$
\[
\boxed{
v(\VEC{x}) = \int_{\partial \Omega}
\pdydx{N}{\nu_{\VEC{y}}}(\VEC{x},\VEC{y}) \phi(\VEC{y})
\dss{S}{y} \quad , \quad \VEC{x} \in \Omega^{\prime} \ ,
}
\]
can be extended to $\overline{\Omega^{\prime}}$ to give a continuous
solution of the exterior Dirichlet problem (D2) with
$\displaystyle -\frac{1}{2} \phi + T_k \phi$ instead of $f$ as boundary
condition.

The rest of the proof is split in two cases: $n>2$ and $n=2$.

\stage{$\mathbf{n>2}$}
By item (4) of Proposition~\ref{pot_split_prop}, there
exists a unique $\psi \in W_+$ such that the single layer potential
with moment $\psi$,
\[
\boxed{
w(\VEC{x}) = \int_{\partial \Omega} N(\VEC{x},\VEC{y})
\psi(\VEC{y}) \dss{S}{y} \quad , \quad \VEC{x} \in \RR^n \ ,
}
\]
satisfies $\displaystyle w\big|_{\Omega_i} = a_i$ for
$1\leq i \leq m$.
Since $\displaystyle w \in C(\RR^n)$ by Proposition~\ref{pot_slp_cont},
$\displaystyle 
w\big|_{\partial \Omega^{\prime}} = \sum_{i=1}^\infty a_i \phi_i$.
Therefore,
\[
\boxed{
u(\VEC{x}) = v(\VEC{x}) + w(\VEC{x}) \quad , \quad \VEC{x} \in
\overline{\Omega^{\prime}} \ ,
}
\]
is a solution of (D2).

\stage{$\mathbf{n=2}$}
Since
$\displaystyle \sum_{i=1}^m \phi_i(\VEC{x}) = 1$ for
$\VEC{x} \in \partial \Omega$, we may write
\[
\boxed{
\sum_{i=1}^m a_i \phi_i(\VEC{x}) = \sum_{i=1}^m b_i \phi_i(\VEC{x}) + c \ ,
}
\]
where $c$ is a constant and $(b_1,b_2, \ldots, b_m)$ is an element of
the set $X$ defined in Proposition~\ref{pot_split_prop}.  Hence, by
item (5) of Proposition~\ref{pot_split_prop}, there
exists $\displaystyle \psi \in W_+^0$ such that the single layer
potential with moment $\psi$,
\[
\boxed{
w(\VEC{x}) = \int_{\partial \Omega} N(\VEC{x},\VEC{y})
\psi(\VEC{y}) \dss{S}{y} \quad , \quad \VEC{x} \in \RR^n \  ,
}
\]
satisfies $\displaystyle w\big|_{\Omega_i} = b_i$ for
$1\leq i \leq m$.  Since
$\displaystyle \int_{\partial \Omega} \psi(\VEC{y}) \dss{S}{y} = 0$
because $\psi\in W_+^0$, we also have from Lemma~\ref{pot_1to1n2} that
$w$ is harmonic at infinity.  Since $w \in C(\RR^n)$ by
Proposition~\ref{pot_slp_cont},
$\displaystyle 
w\big|_{\partial \Omega^{\prime}} = \sum_{i=1}^m b_i \phi_i$.
Therefore,
\[
\boxed{
u(\VEC{x}) = v(\VEC{x}) + w(\VEC{x}) + c \quad , \quad \VEC{x} \in
\overline{\Omega^{\prime}}
}
\]
is a solution of (D2).
\end{proof}

%%% Local Variables: 
%%% mode: latex
%%% TeX-master: "notes"
%%% End: 
