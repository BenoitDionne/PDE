\nonumchapter{Preface}

This book is for an advanced undergraduate course on partial
differential equations or for a beginning graduate course on partial
differential equations. 

There are excellent books on partial differential equations for
graduate students (or advanced undergraduate students) who plan to
pursue the study of partial differential equations for their 
thesis.  However, these books often require a solid
background in measure and integration, and functional analysis that
students who are required to study partial differential equations do
not have.  As a consequence, it takes a lot of time for these students
to study and understand these books because of their condensed style.
This is often frustrating for students who do not plan to pursue
their research in partial differential equations but are only taking a
course on partial differential equation as an elective in their program.  The
present book covers must of the important topics found in books on
partial differential equations, using most of the time the same 
approach, but provide a lot more details so that students do not have
to struggle to understand the theory.

As we said above,
university students now a day do not always learn as much measure and
integration, and functional analysis as it used to be.  Most of the
excellent books on partial differential equations require a good
knowledge of these topics.  The present book is accessible to
students with a minimal introduction to measure and integration, and
functional analysis, and who are willing to accept some of the
fundamentals results which are presented in
Chapter~\ref{ChapterRevFunctAnal}.  Authors often assume that their
readers will fill in the standard details in analysis.  Unfortunately,
students will get stuck trying to understand the proofs or, worse,
will completely miss the subtleties of some proofs.  For instance,
they will assume
that all series and integrals converge, or that derivative and
integral can always be interchanged.  Every computation becomes a formal
computation.  We provide a lot of the details in
analysis that are often assumed in books on partial differential
equations.  Hopefully, students will learn to appreciate the rigour of
analysis and its raison d'être.

This book contains must of the material on partial differential
equations usually taught at the undergraduate level for science and
engineering students.  The reason to include this material is that
many mathematics students do not have the chance do study partial
differential equations at the undergraduate level.  Thus, their first
contact with partial differential equations is at the graduate level
where all the undergraduate material on partial differential equations
is assumed.  The sections
covering the undergraduate material are accessible to readers with
only a calculus background.  The presentation in these sections is
usually more formal.  They are therefore appropriate for an
introduction to partial differential equations for science and
engineering students.  A first undergraduate course could 
possibly include:
{
\begin{itemize} \setlength{\itemsep}{0ex} \setlength{\parskip}{0ex}  
\item parts of Chapter~\ref{ChapCaract};
\item the introduction of Chapter~\ref{ChapShock} and
Section~\ref{sectShockIntro};
\item Chapter~\ref{ChapClassifPDE}.
\item Sections~\ref{SectGenL2} and \ref{SectClassFourierSer} (limited
  to the study of the classical Fourier series);
\item part of Chapter~\ref{ChapSpecFunct};
\item Chapter~\ref{ChapWaveEqu1D} excluding Section~\ref{SectWaveEq1DStSol};
\item Chapter~\ref{ChapWaveEqunD} excluding Section~\ref{SectWavenDLaplOp}
and the higher dimensional case in Section~\ref{SectWavenDExSol};
\item Section~\ref{SectLaplaceHarmFunct}, \ref{SectLaplaceMaxPrinc} and
\ref{SectLaplaceSepVar}; and
\item Chapter~\ref{ChapHeatEqu} excluding the higher dimension heat
equation in Section~\ref{SectHeatEqCauchyPr} and Section~\ref{SectHeatFundSol}.
\end{itemize}
}

For the more advanced topics, the readers are expected to be familiar
with all the concept of vector calculus in $\displaystyle \RR^n$.
These topics are still accessible to readers with a more standard
vector calculus course in $\displaystyle \RR^2$ and
$\displaystyle \RR^3$ if they assume that $n$ is $2$ or
$3$ when reading the sections covering these advanced topics.
The readers may however find the presentation very technical in
comparison to the simpler presentation which is limited to $n=2$ or
$n=3$.

Except for Chapters~\ref{ChapCaract} and \ref{ChapShock}, this book
is abound linear partial differential equations.  So, there is no
mention of reaction diffusion equations, solitons, inverse scattering,
etc.  In particular, there is no mention of semi-group theory.  On
this subject, we strongly recommend \cite{He}.  Energy methods are
used a few times in the book but not as extensively as you will find
in some books on partial differential equation.

It would be a sacrilege to not motivate heuristically the major partial
differential equations that have had a major influence on the theory
of partial differential equations.  I am talking about the wave
equation, the heat equation and the Laplace equation.  We present them
in the next sections.

\section*{Wave equation}

We consider a guitar or violin string of length $L$ at rest.  Since
the string is really thin, we consider only the linear dimension of
the string.  Such strings are homogeneous.  So, we may assume that the
string is of constant density $\rho$.
Moreover, we assume that the string is perfectly elastic when
subject to very small vibration.  Therefore, the tension in the string
is tangent to the string.   Moreover, because of the perfect
elasticity, we also assume that each point on the string is only moving
up and down in a vertical plane as illustrated in the figure below.
There is no horizontal motion.  We assume that the vertical motion is
really small.

\pdfbox{preface/wave1}

The vertical position of the string at a position $x$ on the string
and at a time $t$ is denoted $u(x,t)$, and the vector tension in the string
at that position and time $t$ is denoted $\VEC{T}(x,t)$.  We consider a
very small section of the string and apply Newton's law
``force equals mass times acceleration'' at the end points.   

Let $T(x,t) = \|\VEC{T}(x,t)\|_2$ be the Euclidean length of the
vector tension $\VEC{T}(x,t)$ at position $x$ and time $t$.  So
$T(x,t)$ is the magnitude of the tension at position $x$ and time $t$.

\pdfbox{preface/wave2}

Since there is no horizontal motion, we get that the difference between
the horizontal component of the force at the end points is null; namely,
$T(x_2,t) \cos(\theta_2) - T(x_1,t) \cos(\theta_1) = 0$.  Since
$\displaystyle \cos(\theta_1) = \frac{1}{\sqrt{1+u_x^2(x_1,t)}}$ and
$\displaystyle \cos(\theta_2) = \frac{1}{\sqrt{1+u_x^2(x_2,t)}}$, we
get
\begin{equation} \label{prefaceEq1}
\frac{T(x_1,t)}{\sqrt{1+u_x^2(x_1,t)}}
= \frac{T(x_2,t)}{\sqrt{1+u_x^2(x_2,t)}} \ .  \tag{$1$}
\end{equation}

For the vertical motion, the difference between
the vertical component of the force at the end points is equal to the
mass of the section of the string time its acceleration; namely,
$\displaystyle T(x_2,t) \sin(\theta_2) - T(x_1,t) \sin(\theta_1) =
\int_{x_1}^{x_2} \rho u_{tt}(x,t) \dx{x}$.  Since
$\displaystyle \sin(\theta_1) = \frac{u_x(x_1,t)}{\sqrt{1+u_x^2(x_1,t)}}$ 
and
$\displaystyle \sin(\theta_2) = \frac{u_x(x_2,t)}{\sqrt{1+u_x^2(x_2,t)}}$,
we get
\begin{equation} \label{prefaceEq2}
\frac{T(x_2,t) u_x(x_2,t)}{\sqrt{1+u_x^2(x_2,t)}} -
\frac{T(x_1,t) u_x(x_1,t)}{\sqrt{1+u_x^2(x_1,t)}} = 
\int_{x_1}^{x_2} \rho u_{tt}(x,t) \dx{x} \ . \tag{$2$}
\end{equation}

Since we assume that the vertical motion is really small, we may
assume that $u_x(x,t) \approx 0$ for all $x$ and $t$.  Hence, (\ref{prefaceEq1})
yields $T(x_1,t) \approx T(x_2,t)$.  Since we assume that this is true
for all $x_1$, $x_2$ and $t$, we may conclude that the magnitude of
the tension is constant along the sting; namely, $T(x,t) \equiv T$, a constant,
for all $x$ and $t$.

With all our assumption, we may rewrite (\ref{prefaceEq2}) as
\begin{equation} \label{prefaceEq3}
T \big( u_x(x_2,t) - u_x(x_1,t) \big) = \rho \int_{x_1}^{x_2}
u_{tt}(x,t) \dx{x} \ .  \tag{$3$}
\end{equation}
If we divide both sides of (\ref{prefaceEq3}) by $x_2-x_1$ and let
$x_2 \to x_1$, we get from the definition of the derivative that
$T u_{xx}(x_1,t) = \rho u_{tt}(x_1,t)$.  Since $x_1$ is arbitrary, we
get the famous wave equation
\begin{equation} \label{prefaceEq4}
  u_{tt} = c^2 u_{xx} \tag{$4$}
\end{equation}
for $0 < x < L$ and $t > 0$, where $c = \sqrt{T/\rho}$.  This partial
differential equation is normally accompanied by initial and boundary
conditions.  For instance, the initial position of the string at time
$t=0$ may be given by $u(x,0) = f(x)$ for $0 \leq x \leq L$.
Moreover, the ends of the string may be fixed.  So $u(0,t) = u(L,t) = 0$
for $t\geq 0$.  There are many other choices of boundary conditions.
Moreover, the partial differential equation in (\ref{prefaceEq4}) can
be modified to include damping.  We will address some of these issues
in this book.

A similar reasoning can be used to deduce the wave equation for the
vibrating membrane of a drum; namely,
\[
u_{tt} = c^2 \big( u_{xx} + u_{yy} \big)
\]
for $(x,y) \in D$ and $t > 0$,  where $c = \sqrt{T/\rho}$ and $D$ is
the region representing the membrane in $\RR^2$.

\section*{Heat Equation}

To study the diffusion of heat in a thin rod, it is easier and
equivalent to consider the diffusion of dye in the water contained in a
thin and long tube.  According to Fick's law, the diffusion of the dye
is from a region of higher density to a region of lower density, and
the rate of diffusion is proportional to the concentration difference.
For the heat equation, the diffusion is from the region of higher
temperature to the region of lower temperature, and the rate of
diffusion is proportional to the difference of temperature.
We should point out that for the rod, we assume that diffusion is only
possible at the ends of the rod, there is not diffusion along the rod.

Since the tube is really thin, we consider only the linear dimension
of the tube of length $L$.  We consider a small section of a tube of
length $L$.

\pdfbox{preface/heat1}

Let $u(x,t)$ be the density of dye at the position $x$ and time $t$,
the mass of dye in the little section is given by
$\displaystyle M(x_1,x_2,t) = \int_{x_1}^{x_2} u(x,t) \dx{x}$.
The variation of the mass of dye in the little section is
due to dye flowing in or out at the ends of the section.  Therefore,
\[
  \int_{x_1}^{x_2} u_t(x,t) \dx{x} 
= \dfdx{\int_{x_1}^{x_2} u(x,t) \dx{x}}{t} 
= k \big( u_x(x_2,t) - u_x(x_1,t) \big) \ ,
\]
where $k$ is a constant.  If we divide both sides of this equation by
$x_2-x_1$ and let $x_2 \to x_1$, we get from the definition of the
derivative that $\displaystyle u_t(x_1,t) = k u_{xx}(x_1,t)$.
Since $x_1$ is arbitrary, we get the heat equation
\begin{equation} \label{prefaceEq5}
  u_t = k u_{xx} \tag{$5$}
\end{equation}
for $0 < x < L$ and $t > 0$.  It is called the heat equation because
$u(x,t)$ traditionally represent the temperature in a thin and long
rod at position $x$ and time $t$.  But, as we have just seen and as it
is typical in mathematics, the same partial differential equation can
be used to study different phenomena. The partial differential
equation in (\ref{prefaceEq5}) is usually accompanied of initial and
boundary conditions.   The initial temperature in the rod may be given
by $u(x,0) = f(x)$ for $0 \leq x \leq L$.  There may be boundary
conditions at the end of the rod.  For instance, the temperature at
both ends of the rod may be fixed; so $u(0,t) = g_1(t)$ and
$u(L,t) = g_2(t)$ for $t \geq 0$, where $g_1(t)$ is the temperature
outside the left side of the rode at time $t$ and $g_2(t)$ is the temperature
outside the right side of the rode at time $t$.  Or, there may be no
diffusion at the end of the rod; so $u_x(0,t) = u_x(L,t) = 0$ for
$t \geq 0$.  Another possibility is that
the rate of diffusion is proportional to the difference of temperature
in the rod and outside the rod according to Newton's law of cooling.
Thus, $u_x(0,t) = k (u(0,t) - g_1(t))$ and 
$u_x(L,t) = -k (u(L,t) - g_2(t))$ for $t \geq 0$, where $k>0$ is a
constant of proportionality.

A similar reasoning can be used to deduce the heat equation in a thin
plate; namely,
\[
u_t = k \big( u_{xx} + u_{yy} \big)
\]
for $(x,y) \in D$ and $t > 0$, where $k$ is the constant of
proportionality and $D$ is the region representing the thin plate in
$\displaystyle \RR^2$.

\section*{Laplace Equation}

The Laplace equation comes from the steady state case of the wave
equation and the heat equation; namely, we assume that the function
$u$ does not depend on time.  We get
\[
  u_{xx} + u_{yy} = 0
\]
for $(x,y) \in D$, where $D$ is an open subset of
$\displaystyle \RR^2$.  The solutions of this
equation are called ``harmonic functions.''\  We will study these
functions intensively in this book.  The Laplace equation is usually
accompanied of boundary conditions.  For instance, $u=0$ on the
boundary $\partial D$ of the region $D$.

\section*{Other Equations}

Partial differential equations are ubiquitous in physics.  Several
examples are mentioned in \cite{Str}.  To name a few of them, we have
the Schrödinger equation that describe the motion of an electron
around a proton, the Maxwell's equations in electromagnetism, the
Navier-Stokes equation in fluid dynamics, and the Klein-Gordon
equation in particle physics.   For an excellent introduction to fluid
mechanics, we strongly recommend \cite{ChMa}.  As the previous list
suggest, the first impetus for the study of partial differential
equations came from physics.  However, partial differential equations are
beginning to play an important role in other scientific fields.  For an
introduction to the role play by partial differential equation (and
integro-differential equations) in mathematical biology, we recommend
\cite{Ho}.  For a nice introduction to the applications of partial
differential equations in several area of science, economics, and so
on, we recommend \cite{Sal}.

\section*{Suggested Exercises}

We could have created exercises very similar to those found in the
literature on partial differential equations.  Moreover, some
exercises are classical problems that are found in mostly all good textbooks on
partial differential equations and thus should be included.  So, instead of
(re)producing some exercises on partial differential equation, we have
chosen to refer the reader to the exercises that are found in some of
the good textbooks on partial differential equations.  Our choice of
books is arbitrary and many other good textbooks could have been selected.

We have selected exercises that the reader can solve after having read
the related chapter in the book and without having to read the book
used as the source for the exercises.  We have tried to avoid exercises
that refer specifically to the content of the textbook used as the
source for the exercises.  There are a few exceptions where
we have selected exercises that cover important material that has not
been included in the present book.  As for any book on partial
differential equations, we could not cover every topics on the
subject.  We had to make a personal choice.

\section*{Some Terminology}

The {\bfseries $\mathbf{k}$-norm}\index{$k$-norm} of
$\displaystyle \VEC{x} \in \RR^n$ is defined by
\[
\|\VEC{x}\|_k =
\begin{cases}
\displaystyle \left(\sum_{i=1}^n |x_i|^k\right)^{1/k} & \quad \text{if}\ k > 0 \\
\displaystyle \max_{1\leq i \leq n} |x_i| & \quad \text{if} \ k = \infty  
\end{cases}
\]
We adopt the convention that $\|\cdot\|$ will denote the
{\bfseries $\mathbf{2}$-norm}\index{$2$-norm} or
{\bfseries Euclidean norm}\index{Euclidean Norm} $\|\cdot\|_2$.

The open ball of radius $r$ centred at $\displaystyle \VEC{y} \in \RR^n$
is denoted $\displaystyle B_r(\VEC{y})$; namely,\\
$\displaystyle B_r(\VEC{y}) = \left\{ \VEC{x} \in \RR^n :
\left\|\VEC{x}-\VEC{y} \right\| < r \right\}$.  We will also use this
notation to denote open balls in any metric spaces.

Let $V$ be a subset of $\displaystyle \RR^n$ and $f:V \to \RR$.  The
supremum norm of $f$ on $V$ is defined by
$\displaystyle \|f\|_{\infty,V} = \sup_{\VEC{x}\in V} |f(\VEC{x})|$.
We will simply write $\|f\|_\infty$ when it is clear from the context
what is the set $V$ used in the definition of the supremum.

If $V$ be an open subset of $\displaystyle \RR^n$ and $f:V \to \RR$ is
sufficiently differentiable, we define the differential operator
$\displaystyle \diff^\alpha$ for $\displaystyle \alpha \in \NN^n$
by
\[
\diff^\alpha f\left(\VEC{x}\right)
= \frac{\partial^{|\alpha|} f}{\partial x_1^{\alpha_1}
\partial x_2^{\alpha_2} \ldots \partial x_n^{\alpha_n}}\left(\VEC{x}\right)
\]
for $\VEC{x} \in V$, where $\displaystyle |\alpha| = \sum_{j=1}^n \alpha_j$.

\begin{defn*}
Let $V$ be an open subset of $\displaystyle \RR^n$.  For $m\in \NN$,
$\displaystyle C^m(\overline{V})$ is the space of all
$\displaystyle f \in C^m(V) \cap C(\overline{V})$ such that
$\displaystyle \diff^\alpha f$ can be continuously extended
to $\overline{V}$ for
all multi-index $\displaystyle \alpha \in \NN^n$ such that $|\alpha|\leq m$.

$\displaystyle C^\infty(\overline{V})$ is the space of all
$\displaystyle f \in C^\infty(V) \cap C(\overline{V})$ such that
$\displaystyle \diff^\alpha f$ can be continuously extended to
$\overline{V}$ for all multi-index $\displaystyle \alpha \in \NN^n$.
\end{defn*}

\begin{defn*}
Let $V$ be an open subset of $\displaystyle \RR^n$ and $k$ be a
positive integer.  The space of {\bfseries bounded continuous
functions}\index{Bounded Continuous Functions}, denoted
$\displaystyle C_b^k(\overline{V})$, is the set of all functions
$\displaystyle f\in C^k\left(\overline{V}\right)$ such that
\[
\|f\|_{k,\infty,V} \equiv \max_{|\alpha|\leq k}
\|\diff^\alpha f\|_{\infty,\overline{V}} < \infty \ .
\]
\end{defn*}

It is not hard to prove that the space
$\displaystyle C^k_b(\overline{V})$ is a Banach space
with respect to the norm $\displaystyle \|f\|_{k,\infty,V}$.

\begin{defn*}
Let $V$ be an open subset of $\displaystyle \RR^n$ and $k$ be a
positive integer.  The space of {\bfseries bounded uniformly continuous
functions}\index{Bounded Uniformly Continuous Functions}, denoted
$\displaystyle C_{bu}^k(\overline{V})$, is the set of all functions
$\displaystyle f\in C_b^k\left(\overline{V}\right)$ such that
$\displaystyle \diff^\alpha f:\overline{V} \to \RR$ is uniformly
continuous.
\end{defn*}

It is also not hard to prove that
$\displaystyle C_{bu}^k(\overline{V})$ is a closed subset of
$\displaystyle C_b^k\left(\overline{V}\right)$ with respect to the
norm $\|\cdot\|_{k,\infty,V}$.

\begin{defn}
If $V$ is an open subset of $\displaystyle \RR^n$ and $f:V \to \RR$ is a
sufficiently differentiable function, then the
{\bfseries gradient}\index{Gradient} of $u$ is defined by
$\displaystyle \graD f(\VEC{x})
= \left( \pdydx{f}{x_1}(\VEC{x}) , \pdydx{f}{x_2}(\VEC{x}) ,
\ldots, \pdydx{f}{x_n}(\VEC{x}) \right)$ and the
{\bfseries Laplacian}\index{Laplacian} of $f$ is defined by
$\displaystyle \Delta f(\VEC{x})
= \pdydxn{f}{x_1}{2}(\VEC{x}) + \pdydxn{f}{x_2}{2}(\VEC{x}) +
\ldots + \pdydxn{f}{x_n}{2}(\VEC{x})$ for all $\VEC{x} \in V$.

If $\displaystyle F:V \to \RR^n$ is a sufficiently differentiable
function, then the {\bfseries divergence}\index{Divergence} of $F$ is
defined by $\displaystyle \diV F(\VEC{x}) = \nabla \cdot F(\VEC{x})
= \pdydx{F_1}{x_1}(\VEC{x}) + \pdydx{F_2}{x_2}(\VEC{x}) +
\ldots + \pdydx{F_n}{x_n}(\VEC{x})$ for all $\VEC{x} \in V$..
\end{defn}

We have that $\Delta f(\VEC{x}) = \diV (\graD f(\VEC{x})))$
if $f:V \to \RR$ is sufficiently differentiable.

\subI{Warning} In this book, $\NN = \{0,1,2,3, \ldots\}$.

%%% Local Variables: 
%%% mode: latex
%%% TeX-master: "notes"
%%% End: 
