\chapter{Review of Functional Analysis} \label{ChapterRevFunctAnal}

The following material is normally presented in a functional analysis
course.   We assume that the reader has already taken an introductory
course in topology and one in measure theory.  Good references for
the material of this section are \cite{Br,ReeSim,Ru,RuFA}.  We
particularly suggest \cite{Br}.  It contains a lot of exercises with
detailed solutions to many of them.

We assume that the reader has learned the basic concepts of
topological, metric, linear and normed spaces.  In particular, we assume that
the reader is familiar know what is a
{\bfseries Banach space}\index{Banach Space}; namely, of complete
normed space.  This is the starting point for this chapter.

\section{Hilbert Spaces}

\begin{defn}
An {\bfseries inner product space}\index{Inner Product Space}
$H$ over the complex numbers is a vector
space over $\CC$ equipped with a mapping
$\ps{\cdot}{\cdot}:H\times H \rightarrow \CC$ such that
\begin{enumerate}
\item $\ps{x}{y} \geq 0$ for all $x,y \in H$.
\item $\ps{x}{x} = 0$ if and only if $x=0 \in H$.
\item $\ps{x}{y} = \overline{\ps{x}{y}}$ for all $x,y \in H$.
\item $\ps{x+y}{z} = \ps{x}{z} + \ps{y}{z}$ for all $x,y,z \in H$.
\item $\ps{\alpha x}{y} = \alpha \ps{x}{y}$ for all $x,y \in H$ and
$\alpha \in \CC$.
\end{enumerate}
The mapping $\ps{\cdot}{\cdot}:H\times H \rightarrow \CC$ is called a
{\bfseries scalar product}\index{Scalar Product} or an
{\bfseries inner product}\index{Inner Product} on $H$.
\end{defn}

Hilbert spaces are metric spaces.  We can define a norm and a distance
in a natural way.

\begin{theorem}
If $H$ is an inner product space over the complex numbers, then
$\displaystyle \| x \| = \sqrt{\ps{x}{x}}$ for $x \in H$ defines a
norm on $H$.  This norm is called the
{\bfseries norm Induced}\index{Norm Induced} by the scalar product. 
\end{theorem}

As expected for a norm, the norm induced by a scalar product has the
following properties.

\begin{theorem}
If $H$ is an inner product space over the complex numbers, then
\begin{enumerate}
\item $|\ps{x}{y}| \leq \|x\|\,\|y\|$ for all $x,y\in H$.  This is
called the {\bfseries Schwarz inequality}\index{Schwarz Inequality}.
\item $\|x+y\| \leq \|x\| + \|y\|$ for all $x,y\in H$.  This is
called the {\bfseries triangle inequality}\index{Triangle Inequality}.
\end{enumerate}
\end{theorem}

Since an inner product space $H$ is a metric space with the metric
$d(\VEC{x},\VEC{y}) = \|\VEC{x} - \VEC{y}\|$ for all $\VEC{x},\VEC{y} \in H$,
we can define the convergence of sequences
$\displaystyle \{ \VEC{x}_j \}_{j=0}^\infty$ in $H$ exactly has it is
done in any metric space.  The metric $d$ defines a topology on $H$.

\begin{defn}
Let $H$ is an inner product space over the complex numbers with the
scalar product $\ps{\cdot}{\cdot}:H\times H \rightarrow \CC$.  If $H$
is complete with respect to the norm induce by this scalar product
(i.e.\ every Cauchy sequence in $H$ is converging to an element of
$H$), then $H$ is called an
{\bfseries Hilbert space}\index{Hilbert Space} over the complex
numbers.
\end{defn}

Since the inner product defines a metric on an inner product space, we
may define the continuity of a function $f:H_1 \to H_2$ from an inner
product space $H_1$ to an inner product space $H_2$ as it is normally
done for functions between two metric spaces.

\begin{prop}
Let $H$ be an Hilbert space over the complex numbers and let
$y\in H$, then
\begin{enumerate}
\item $\displaystyle x \mapsto \ps{x}{y}$ is a continuous linear mapping.
\item $\displaystyle x \mapsto \ps{y}{x}$ is a continuous
conjugate-linear mapping.
\item $\displaystyle x \mapsto \|x\|$ is a continuous mapping.
\end{enumerate}
\end{prop}

In addition to open or close subsets of an Hilbert space, there is
another type of subsets of an Hilbert space that will play an
important role later on when solving partial differential equations.

\begin{defn}
A subset $V$ of a vector space $H$ is {\bfseries convex}\index{Convex Space}
if $ \lambda x + (1-\lambda) y \in V$ for all $\lambda \in [0,1]$ and all
$x,y \in V$.    
\end{defn}

\begin{theorem} \label{fu_an_ccnes}
If $V$ is a closed, convex and non-empty subset of and Hilbert space
$H$, then there exists a unique $y \in V$ such that
$\displaystyle \|y\| = \inf_{x\in V} \|x\|$.
\end{theorem}

In the previous theorem, $V$ may not be compact.

\begin{defn} \label{defnCLFunctHilbert}
A continuous linear function from an Hilbert space $H$ to $\CC$ is
called a {\bfseries continuous linear functional}\index{Continuous
Linear Functional} on $H$.  The space of all continuous linear
functional on $H$ is denote $\displaystyle H^\ast$.  This space is
called the {\bfseries dual space}\index{dual space} of $H$.
\end{defn}

\begin{theorem}[Riesz Representation Theorem] \label{fu_an_RieszRT}
Let $H$ be an Hilbert space and let $g:H\rightarrow \CC$ be a
continuous linear functional, then there exists a unique $y\in H$ such
that $g(x) = \ps{x}{y}$ for all $x\in H$.  \index{Riesz Representation Theorem}
\end{theorem}

Since an inner product is defined on an Hilbert space, it is possible
to define the angle between elements of an Hilbert space; in
particular, it is possible to define when two elements are orthogonal
to each other.

\begin{defn}
Let $H$ be an Hilbert space.  Two elements $x_1, x_2 \in H$ are
{\bfseries orthogonal}\index{Orthogonal Vectors} or
{\bfseries perpendicular}\index{Perpendicular Vectors} if $\ps{x_1}{x_2} = 0$.
A set $\displaystyle S = \{ x_\alpha \}_{\alpha \in A} \subset H$,
where $A$ is some index set, is
{\bfseries orthogonal}\index{Orthogonal Set} if
$\ps{x_\alpha}{x_\beta} = 0$ for $\alpha\neq\beta$ and
$\ps{x_\alpha}{x_\beta} \neq 0$ for $\alpha=\beta$.  The set $S$ is
{\bfseries orthonormal}\index{Orthonormal Set} if
\[
\ps{x_\alpha}{x_\beta} =
\begin{cases}
0 & \quad \text{if} \ \alpha \neq \beta \\
1 & \quad \text{if} \ \alpha = \beta \\
\end{cases}
\]
\end{defn}

\begin{theorem}
Let $H$ be an Hilbert space and let $\displaystyle \{x_j\}_{j=1}^J$ be a finite
orthonormal subset of $H$.  Given $x\in H$, we have
\[
\left\| x - \sum_{j=1}^J \ps{x}{x_j} x_j \right\| 
\leq \left\| x - \sum_{j=1}^J \lambda_j x_j \right\|
\]
for all $\lambda_j \in \CC$, and
\[
\left\| x - \sum_{j=1}^J \ps{x}{x_j} x_j \right\| 
= \left\| x - \sum_{j=1}^J \lambda_j x_j \right\|
\]
implies that $\lambda_j = \ps{x}{x_j}$ for all $j$.
\end{theorem}

\begin{cor}
Let $H$ be an Hilbert space and
$\displaystyle S = \{ x_\alpha \}_{\alpha \in A}$,
where $A$ is some index set, be an orthonormal subset of $H$.
Then
\begin{equation} \label{fu_an_intsum}
\sum_{\alpha \in A} |\ps{x}{x_\alpha}|^2 \leq \|x\|^2
\end{equation}
for all $x \in H$.
\end{cor}

The sum in (\ref{fu_an_intsum}) is in fact the square of the norm of
the function 
\begin{align*}
F_x:A & \rightarrow \CC \\
\alpha &\mapsto \ps{x}{x_\alpha}
\end{align*}
in the space $\displaystyle \ell^2(A)$ of square integrable functions
with respect to the counting measure on $A$.  In other words,
(\ref{fu_an_intsum}) is
\[
\int_A \left| F_x(\alpha)\right|^2\dx{\nu(\alpha)} \leq \|x\|^2 \ ,
\]
where $\nu$ is the counting measure on $A$.  The previous corollary
states that $\displaystyle F_x \in \ell^2(A)$ for all $x\in H$.

Since $\displaystyle F_x \in \ell^2(A)$, it follows that the set of all $\alpha$
such that $F_x(\alpha) \neq 0$ is at most countable.  More
precisely, for each positive integer $n$, the set of all $\alpha$ such
that $|F_x(\alpha)| > 1/n$ is finite or empty.

\begin{theorem} \label{fu_an_complset}
Let $H$ be an Hilbert space and
$\displaystyle S = \{ x_\alpha \}_{\alpha \in A}$,
where $A$ is some index set, be an orthonormal subset of $H$.
The following statement are equivalent
\begin{enumerate}
\item $S$ is a maximal orthonormal subset of $H$.
\item The set of all finite linear combinations of elements of $S$ is
dense in $H$.
\item For every $x\in H$, we have
$\displaystyle \|x\|^2 = \sum_{\alpha \in A} |\ps{x}{x_\alpha}|^2$.
\item For every $x, y \in H$, we have
$\displaystyle
\ps{x}{y} = \sum_{\alpha \in A} \ps{x}{x_\alpha}\, \ps{y}{x_\alpha}$.
\end{enumerate}
\end{theorem}

\begin{defn}
Let $H$ be an Hilbert space.  A
{\bfseries complete orthonormal set}\index{Complete Orthonormal Set}
or an {\bfseries orthonormal basis}\index{Orthonormal Basis} for $H$
is an orthonormal set
$\displaystyle S = \{ x_\alpha \}_{\alpha \in A} \subset H$, where $A$ is
some index set, such that $S$ satisfies any of the equivalent statements in
Theorem~\ref{fu_an_complset}.

A {\bfseries complete orthogonal set}\index{Complete Orthogonal Set}
or a {\bfseries orthogonal basis}\index{Orthogonal Basis} for
$H$ is an orthogonal set $S = \{ x_\alpha \}_{\alpha \in A}$, where $A$ is
some index set, such that
$\displaystyle \{ \|x_\alpha\|^{-1} x_\alpha \}_{\alpha \in A}$ is an
orthonormal basis.
\end{defn}

A consequence of the previous definition is that a complete orthogonal set
$S$ is an orthogonal set such that all finite linear combinations of elements
of $S$ is dense in $H$.

If $\displaystyle S = \{ x_\alpha \}_{\alpha \in A}$, where $A$ is some index
set, is an orthonormal basis of an Hilbert space $H$, it follows from
Theorem~\ref{fu_an_complset} that
\begin{equation} \label{fu_an_isoL2}
\begin{split}
T: H & \rightarrow \ell^2(A) \\
 x &\mapsto F_x
\end{split}
\end{equation}
is a one-to-one and onto isometry, and moreover
$\displaystyle \ps{x}{y} = \int_A F_x(\alpha)F_y(\alpha) \dx{\nu(\alpha)}$,
where $\nu$ is the counting measure on $A$.  A consequence of this
isomorphism is that orthonormal bases in $H$ are isomorphic (i.e.\ have
the same cardinality).

\section{General $\displaystyle L^2$ Spaces}\label{SectGenL2}

Suppose that $\mu$ is a measure on a measurable space $\Omega$.
Let $\displaystyle L^2(\Omega) \equiv L^2(\Omega, \mu)$ be the space
of measurable functions $f:\Omega \rightarrow \CC$ such that
$\displaystyle \int_\Omega |f|^2 \dx{\mu}$ is finite.  We can define
a scalar product on $\displaystyle L^2(\Omega)$ by
\[
\ps{f}{g} = \int_\Omega f\, \overline{g} \dx{\mu} \quad , \quad
f, g \in L^2(\Omega) \ .
\]
The associated $\displaystyle L^2$-norm is
\[
\| f \|_2 = \sqrt{\ps{f}{f}} = \left(\int_\Omega |f|^2 \dx{\mu}\right)^{1/2}
\quad , \quad  f \in L^2(\Omega) \ .
\]
Equipped with this norm, $\displaystyle L^2(\Omega)$ is an Hilbert space.

\begin{rmk}
We use the term ``function'' for an element
$\displaystyle f \in L^2(\Omega)$.  The
reader should remember that in fact $f$ represents the equivalence
class of functions which are equal to $f$ almost everywhere. 
\end{rmk}

It follows from Theorem~\ref{fu_an_complset}, that an orthonormal set
of functions $\displaystyle \{ v_\alpha \}_{\alpha \in A}$, where $A$
is some index set, is complete if
$\ps{f}{v_\alpha} = 0$ for all $\alpha \in A$ implies that $f = 0$
almost everywhere on $\Omega$.

Let $\displaystyle S= \{ v_\alpha \}_{\alpha \in A}$, where $A$ is some
index set, be an orthonormal basis of $\displaystyle L^2(\Omega)$.  
The {\bfseries (generalized) Fourier series}%
\index{Generalized Fourier Series} of a function
$\displaystyle f \in L^2(\Omega)$ with respect to the orthonormal basis $S$ is
\[
\sum_{\alpha \in A} a_\alpha v_\alpha \ ,
\]
where
\[
a_\alpha = \ps{f}{v_\alpha} 
= \int_\Omega f\, \overline{v}_\alpha \dx{\mu} \quad, \quad \alpha \in A \ .
\]
Recall that the set of all $\alpha$ such that $a_\alpha \neq 0$ is at
most countable.  So, the sum is an ordinary sum.  More precisely, take
any ordering $\displaystyle \{ \alpha_j\}_{j=0}^\infty$ of the set of
index $\alpha \in A$ such that $a_\alpha \neq 0$.  Because of the
isometry defined in (\ref{fu_an_isoL2}), we have
\begin{equation} \label{fu_an_convFS}
\left\| f - \sum_{j=0}^{J} a_{\alpha_j} v_{\alpha_j} \right\|_2
= \left( \int_{j \in B_J} \left| F_f(j)\right|^2 \dx{\nu} \right)^{1/2}
\rightarrow 0 \quad \text{as} \quad J \rightarrow \infty \ ,
\end{equation}
where $F_f:\NN \rightarrow \CC$ is defined by
$F_f(j) = \ps{f}{v_{\alpha_j}}$ for all
$j\in \NN$, $B_J = \{J+1, J+2, \ldots\}$ and $\nu$ is the
counting measure on $\NN$.  The convergence to $0$ of the integral
above is due to the fact that $\displaystyle F_f \in \ell^2(\NN)$.
In other words, (\ref{fu_an_convFS}) states that
\[
\left( \int_\Omega
\left|f - \sum_{j=1}^{J} \: a_{\alpha_j} v_{\alpha_j} \right|^2
\dx{\mu} \right)^{1/2}
= \left( \sum_{j=J+1}^\infty |a_{\alpha_j}|^2 \right)^{1/2} 
\rightarrow 0 \quad \text{as} \quad
J \rightarrow \infty
\]
whatever the ordering $\displaystyle \{ \alpha_j\}_{j=0}^\infty$ of
the set of indices $\alpha \in A$ such that $a_\alpha \neq 0$.
We write
\[
  f = \sum_{\alpha \in A} a_\alpha v_\alpha
\]
whenever \ref{fu_an_convFS} is satisfied.   This does not mean that we
have pointwise convergence.  We address this issue in
Section~\ref{sectPtWsConv} below.

\subsection{$\displaystyle L^2$ Spaces on the Real Line}

An important special case of the general theory of $\displaystyle L^2$ spaces is
given by $\Omega=[a,b]$ equipped with the measure $\mu$ defined on
$[a,b]$ by
\[
  \mu(I) = \int_I p(x) \dx{x}
\]
for all Lebesgue measurable subsets $I$ of $[a,b]$, where
$p:[a,b]\rightarrow \RR$ is a piecewise continuous function on the
interval $[a,b]$ such that $p(x) > 0$ for almost all $x \in [a,b]$.
We could also have used an open interval $]a,b[$, bounded or not
bounded.  The results are the same.

Recall that a function $f$ is
{\bfseries piecewise continuous}\index{Piecewise Continuous} on an
interval $I$ if $f$ is continuous on the interval $I$ except at a
finite number of points and
$\displaystyle f(c^+) = \lim_{x\to c^+} f(x)$ and
$\displaystyle f(c^-) = \lim_{x\to c^-} f(x)$ for all $c \in I$; in
particular at the points where $f$ is discontinuous.

$\displaystyle L^2[a,b] \equiv L^2([a,b],\mu)$ is the space of
measurable functions $f:[a,b]\rightarrow \RR$ such that
$\displaystyle \int_a^b f^2 \dx{\mu} = \int_a^b f^2(x) p(x) \dx{x}$
is finite.  The scalar product on $\displaystyle L^2[a,b]$ is 
\begin{equation} \label{fu_an_r_scp}
\ps{f}{g} = \int_a^b f(x) g(x) p(x) \dx{x} \quad , \quad f, g \in L^2[a,b] \ .
\end{equation}
The associated $\displaystyle L^2$-norm is
\begin{equation} \label{fu_an_r_2norm}
\| f \|_2 = \sqrt{\ps{f}{f}} = \left(\int_a^b f^2(x) p(x) \dx{x}\right)^{1/2}
\quad , \quad f \in L^2[a,b] \ .
\end{equation}

The $\displaystyle L^2[a,b]$ spaces have countable orthonormal bases.
Suppose that $\displaystyle \{v_n : n \in \NN \} \subset L^2[a,b]$
is an orthonormal base for $\displaystyle L^2[a,b]$, the Fourier
series of a function $\displaystyle f \in L^2[a,b]$ with respect to
this bases is
\begin{equation} \label{fsOne}
\sum_{n=0}^{\infty} a_{n} v_n \ ,
\end{equation}
where
\[
a_n = \ps{f}{v_n} = \int_a^b f(x) v_n(x) p(x) \dx{x}
\quad, \quad n \in \NN \ .
\]

Sometime, we only have a complete orthogonal set of
functions $\displaystyle \{v_n : n \in \NN \} \subset L^2[a,b]$.
In this case, the Fourier series of $\displaystyle f \in L^2[a,b]$
with respect to this set of functions is
\begin{equation} \label{fsTwo}
\sum_{n=0}^{\infty} a_{n} v_n \ ,
\end{equation}
where
\[
a_n = \frac{\ps{f}{v_n}}{\| v_n \|^2} 
= \left( \int_a^b (v_n(x))^2 p(x) \dx{x} \right)^{-1}
\int_a^b f(x) v_n(x) p(x) \dx{x}
\quad, \quad n \in \NN \ .
\]

The convergence of the series in (\ref{fsOne}) and (\ref{fsTwo})
implies that
\[
\left\| f - \sum_{n=0}^{N} a_n v_n \right\|_2
= \left( \int_a^b \left( f(x) - \sum_{n=0}^{N} \: a_n v_n(x) \right)^2
p(x) \dx{x} \right)^{1/2} \rightarrow 0 \quad \text{as} \quad
N \rightarrow \infty \ .
\]
We say that $\displaystyle \sum_{n=0}^{N} \: a_n v_n$ converges in
$\displaystyle L^2$ to $f$ as $N\rightarrow \infty$, and we write
\[
 f = \sum_{n=0}^\infty a_n v_n \ .
\]
Recall that this equality is not pointwise but it is an equality in
$\displaystyle L^2[a,b]$.

\subsection{$\displaystyle L^2$ Spaces on the Unit Circle}

An important class of $\displaystyle L^2$ spaces is the space of
measurable complex valued functions $\displaystyle L^2(T) \equiv L^2(T,\nu)$,
where $T$ is the unit circle in $\CC$ and the Lebesgue measure $\nu$
on $T$ is obtained from the arc length divided by $2\pi$.  In
particular, $\nu(T) =1$.

Let $P$ be the space of $2\pi$-periodic complex valued functions on
$\RR$ such that
\[
\left(\int_{-\pi}^\pi |f|^2 \dx{\mu}\right)^{1/2} < \infty \ ,
\]
where $\mu$ is the Lebesque measure on $[-\pi,\pi]$ divided by $2\pi$.
If the norm on $P$ is defined by
\[
\|f\|_2 = \left(\int_{-\pi}^\pi |f|^2 \dx{\mu}\right)^{1/2} \ ,
\]
we have an isometry between $\displaystyle L^2(T,\nu)$ and $P$ defined by
\begin{align*}
Q: L^2(T,\nu) & \rightarrow P \\
  f & \mapsto f(e^{x\,i})
\end{align*}
We have
\[
\left(\int_T |f(z)|^2\dx{\nu}(z) \right)^{1/2}
= \left(\int_{-\pi}^\pi |f(e^{x\,i})|^2\dx{\mu}(x) \right)^{1/2}  \ ,
\]
Every measurable set $B \subset T$ is of the form
$\displaystyle B = e^{Ai} \equiv \{e^{x\,i} : x\in A\}$ where
$A \subset [-\pi,\pi]$
is a measurable set, and vice-versa.   We also have that
$\displaystyle \nu(e^{Ai}) = \mu(A)$.

The following definition is justified by the isometry above.

\begin{defn}
$P$ is denoted $\displaystyle L^2(T) = L^2(T,\nu)$.
\end{defn}

The context will determine which of the two spaces we may be working with.

The set $\displaystyle \left\{ e^{n x\, i} : n \in \ZZ \right\}$ is a
complete orthonormal set in $\displaystyle L^2(T)$.  The
{\bfseries classical Fourier expansion}\index{Classical Fourier Expansion}
of a complex valued function $\displaystyle f \in L^2(T)$ is
\[
\sum_{n\in \ZZ} c_n e^{n x\, i} \ ,
\]
where
\begin{equation} \label{fa_coeff_cfs}
c_n = \frac{1}{2\pi} \int_{-\pi}^{\pi} \, f(x) e^{-n x\, i} \dx{x} \quad
, \quad n \in \ZZ \ .
\end{equation}

\subsection{Pointwise Convergence of the Classical Fourier Series}
\label{sectPtWsConv}

In this section, we prove the following theorem.

\begin{theorem}\label{faFsPwc}
Let $f:\RR\to \CC$ be a $2\pi$-periodic, piecewise differentiable
function.  Then
\[
\frac{f(x^+) + f(x^-)}{2} = \lim_{N\to \infty} \sum_{n=-N}^N c_n e^{n x\, i} \ ,
\]
where the $c_n$ are defined in (\ref{fa_coeff_cfs}),
$\displaystyle f(x^+) = \lim_{s\to x^+} f(s)$ and
$\displaystyle f(x^-) = \lim_{s\to x^-} f(s)$.
\end{theorem}

Less restrictive conditions are given in \cite{RuPMA} for the
pointwise convergence of the Fourier series.  Recall that $f$ is
{\bfseries piecewise differentiable}\index{Piecewise Differentiable}
on $[a,b]$ if $f$ and $f'$ are piecewise continuous on $[a,b]$.

Before proving this theorem, we will need a very famous Lemma.

\begin{lemma}[Riemann-Lebesgue]
If $f:[a,b] \to \RR$ is piecewise differentiable, then
\[
\lim_{R\to \infty} \int_a^b f(y) \sin(Ry) \dx{y} = 0 \ .
\]
\end{lemma}

\begin{proof}
Let $a < y_1 < y_2 < \ldots < y_{m-1} < b$ be the points where $f$ is
not differentiable.  Let $y_0 = a$ and $y_m = b$.  Using integration
by parts, we have
\begin{align*}
&\int_a^b  f(y) \sin(Ry) \dx{y}
= \sum_{j=0}^{m-1} \int_{y_j}^{y_{j+1}} f(y) \sin(Ry) \dx{y} \\
& \quad = \frac{1}{R} \sum_{j=0}^{m-1} \left( -f(y_{j+1}^+)\cos(Ry_{j+1})
+ f(y_j^-)\cos(Ry_j) + \int_{y_j}^{y_{j+1}} f'(y) \cos(Ry) \dx{y} \right) \ .
\end{align*}
Let $M_1 = \sup_{y\in[a,b]} |f(y)|$ and $M_2 = \sup_{y\in[a,b]}|f'(y)|$.
$M_1$ and $M_2$ are finite because $f$ is piecewise differentiable and
thus also piecewise continuous.  We have
\begin{align*}
\left| \int_a^b  f(y) \sin(Ry) \dx{y} \right|
& \leq \frac{1}{R} \sum_{j=0}^{m-1} \left( |f(y_{j+1}^+)| + |f(y_j^-)|
+ \int_{y_j}^{y_{j+1}} |f'(y)| \dx{y} \right) \\
& \leq \frac{1}{R} \left( 2m M_1 + M_2 (b-a) \right) \to 0 \quad \text{as}
\quad R \to \infty \ .  \qedhere
\end{align*}
\end{proof}

Let
\[
S_N(x) = \sum_{n=-N}^N c_n e^{n x\, i}
\]
for $N$ a positive integer and $\VEC{x} \in \RR$.  We get from
(\ref{fa_coeff_cfs}) that
\[
S_N(x) = \frac{1}{2\pi} \int_{-\pi}^{\pi} f(t) \sum_{n=-N}^N
e^{n(x-t)\, i} \dx{t}
= \frac{1}{2\pi} \int_{-\pi}^{\pi} f(t) D_N(x-t) \dx{t} \ ,
\]
where
\[
D_n(y) = \sum_{n=-N}^N e^{n y \, i} \ .
\]
$D_N$ is called the {\bfseries Dirichlet kernel}\index{Dirichlet kernel}.
From
\[
(1-e^{y\, i}) \sum_{n=-N}^N e^{n y \, i} = 
\sum_{n=-N}^N \left( e^{n y \, i} - e^{(n+1) y \, i} \right)
= e^{-N y \, i} - e^{(N+1) y \, i} \ ,
\]
we get
\[
D_N(y) = \frac{e^{-N y \, i} - e^{(N+1) y \, i}}{1-e^{y\, i}}
= \frac{e^{-(N+1/2) y \, i} - e^{(N+1/2) y \, i}}{e^{-(y/2)\,i}-e^{(y/2)\, i}}
= \frac{ \sin( (N+1/2)y)}{\sin(y/2)}
\]
for $y \neq 2 n \pi$ with $n \in \ZZ$.

The function $D_N(y)$ is even and
$\displaystyle \int_{-\pi}^{\pi} D_N(y) \dx{y} = 2\pi$.

\begin{proof}[Proof of Theorem~\ref{faFsPwc}]
Given $x \in [-\pi,\pi]$.
\begin{align*}
S_N(x) - \frac{f(x^+)-f(x^-)}{2}
&= \frac{1}{2\pi} \int_{-\pi}^{\pi} f(t) D_N(x-t)\dx{t}
- \frac{1}{2\pi} \int_{-\pi}^{\pi} \frac{f(x^+)+f(x^-)}{2} D_N(t)\dx{t} \\
&= \frac{1}{2\pi} \int_{-\pi}^{\pi} f(x-t) D_N(t)\dx{t}
- \frac{1}{2\pi} \int_0^{\pi} (f(x^+)+f(x^-)) D_N(t)\dx{t} \ ,
\end{align*}
where we have used the substitution $t \to x-t$ in the first integral
and the fact that $f$ and $D_N$ are $2\pi$-periodic.  For the second
integral, we have use the fact that $D_N$ is an even function.  Thus
\begin{align*}
&S_N(x) - \frac{f(x^+)-f(x^-)}{2}
= \frac{1}{2\pi} \int_0^{\pi} f(x-t) D_N(t)\dx{t}
- \frac{1}{2\pi} \int_0^{\pi} f(x^-) D_N(t)\dx{t} \\
&\qquad + \frac{1}{2\pi} \int_{-\pi}^0 f(x-t) D_N(t)\dx{t}
- \frac{1}{2\pi} \int_0^{\pi} f(x^+) D_N(t)\dx{t} \\
&\quad = \frac{1}{2\pi} \int_0^{\pi} f(x-t) D_N(t)\dx{t}
- \frac{1}{2\pi} \int_0^{\pi} f(x^-) D_N(t)\dx{t} \\
&\qquad + \frac{1}{2\pi} \int_0^{\pi} f(x+t) D_N(t)\dx{t}
- \frac{1}{2\pi} \int_0^{\pi} f(x^+) D_N(t)\dx{t} \\
&\quad = \frac{1}{2\pi} \int_0^{\pi} \left(f(x-t) - f(x^-) \right) D_N(t)\dx{t}
+ \frac{1}{2\pi} \int_0^{\pi} \left( f(x+t)- f(x^+)\right) D_N(t)\dx{t} \ .
\end{align*}
Let
\begin{align*}
I_1(N) &= \frac{1}{2\pi} \int_0^{\pi} \left(f(x-t) - f(x^-) \right) D_N(t)\dx{t}
\intertext{and}
I_2(N) &= \frac{1}{2\pi} \int_0^{\pi} \left( f(x+t)- f(x^+)\right) D_N(t)\dx{t} \ .
\end{align*}

\stage{i} We prove that $I_2(N) \to 0$ as $N \to \infty$.
Namely, given $\epsilon > 0$, we find $N_0>0$ such that
$|I_2(N)| < \epsilon$ for $N > N_0$.

Given $0 < \delta < \pi$, we have that
\[
I_2(N) = I_{2,1}(N,\delta) + I_{2,2}(N,\delta) \ ,
\]
where
\begin{align*}
I_{2,1}(M, \delta)
&= \frac{1}{2\pi} \int_0^{\delta} \left( f(x+t)- f(x^+)\right) D_N(t)\dx{t}
\intertext{and}
I_{2,2}(N,\delta)
&= \frac{1}{2\pi} \int_{\delta}^{\pi} \left( f(x+t)- f(x^+)\right) D_N(t)\dx{t} \ .
\end{align*}
Choose $\delta_0 \in ]0,\pi[$ such that $f'$ is continuous on
$]x,x+\delta_0[$.  Let $M = \sup_{t\in]x,x+\delta_0[}|f'(t)|$.  By the Mean
Value Theorem, we have that $\displaystyle |f(x+t) - f(x^+)| \leq M t$
for $0 < t < \delta_0$.
We also have $0 \leq t \leq \pi \sin(t/2)$ for $0 \leq t \leq \pi$.
Hence, for $0 < \delta < \min( \delta_0, \epsilon/M)$, we get
\[
|I_{2,1}(N,\delta)|
\leq \frac{1}{2\pi} \int_0^{\delta} M t | D_N(t) | \dx{t}
\leq \frac{M}{2} \int_0^{\delta} \frac{t}{\pi \sin(t/2)} \dx{t}
\leq \frac{M\delta}{2} < \frac{\epsilon}{2}
\]
We fix $\delta$ between $0$ and $\min( \delta_0, \epsilon/M)$.
From Riemann-Lebesgue's Lemma, we can find $N_0>0$ such that
\[
|I_{2,2}(N,\delta)|
= \left|
\frac{1}{2\pi} \int_{\delta}^{\pi} 
\left( \frac{f(x+t)- f(x^+)}{\sin(t/2)} \right) 
\sin((N+1/2)t) \dx{t} \right| < \frac{\epsilon}{2}
\]
for $N > N_0$.  Hence, for $N> N_0$, we have $|I_2(N)| < \epsilon$.

We had to split the integral $I_2(N)$ because
$\displaystyle (f(x+t) - f(x^+))/\sin(t/2)$ or is derivative may not
be bounded at the origin.  So, $\displaystyle (f(x+t) - f(x^+))/\sin(t/2)$
may not be piecewise differentiable on $[0,\pi]$.

\stage{ii} A proof similar to the one given in (i) yields
$I_1(N) \to 0$ as $N\to \infty$.
\end{proof}

\section{Classical Fourier Series of real Valued
Functions}\label{SectClassFourierSer}

We consider the space $\displaystyle L^2[-L,L] = L^2([-L,L],m)$, where
$m$ is the Lebesgue measure on the interval $[-L,L]$.  A complete
orthogonal set for this space is given by the trigonometric functions
\[
\left\{ \cos\left(\frac{n\pi x}{L}\right) : n \geq 0 \right\}
\cup \left\{ \sin\left(\frac{n\pi x}{L}\right) : n \geq 1 \right\} \ .
\]
To be more precise, we have
\begin{align*}
\int_{-L}^L \sin\left(\frac{i \pi x}{L}\right)
\sin\left(\frac{j \pi x}{L}\right)\dx{x} & =
\begin{cases}
0 & \mbox{ if } i \neq j \mbox{ and } i,j > 0\\
L & \mbox{ if } i = j > 0
\end{cases} \\
\int_{-L}^L \cos\left(\frac{i \pi x}{L}\right)
\cos\left(\frac{j \pi x}{L}\right)\dx{x} & =
\begin{cases}
0 & \mbox{ if } i \neq j \mbox{ and } i,j \geq 0 \\
L & \mbox{ if } i = j > 0 \\
2L & \mbox{ if } i = j = 0
\end{cases}
\intertext{and}
\int_{-L}^L \sin\left(\frac{i \pi x}{L}\right)
\cos\left(\frac{j \pi x}{L}\right)\dx{x} & =
 0 \quad \text{if} \quad i,j \geq 0 \ .
\end{align*}

From (\ref{fsTwo}), we get the
{\bfseries classical Fourier expansion}\index{Classical Fourier expansion}
of a function $\displaystyle f \in L^2[-L,L]$.
\[
a_0 + \sum_{n=1}^{\infty} \left( a_n
\cos\left(\frac{n\pi x}{L}\right) + b_n
\sin\left( \frac{n\pi x}{L}\right) \right) \ ,
\]
where
$\displaystyle a_0 = \frac{1}{2L} \int_{-L}^{L} f(x) \dx{x}$,
$\displaystyle a_n = \frac{1}{L} \int_{-L}^{L} f(x)
\cos\left(\frac{n\pi x}{L}\right) \dx{x}$ for $n \geq 1$, and\\
$\displaystyle b_n = \frac{1}{L} \int_{-L}^{L} f(x)
\sin\left( \frac{n\pi x}{L}\right) \dx{x}$ for $n \geq 1$.

If $f:\RR\rightarrow \RR$ is periodic of period $T = 2L$ and
$\displaystyle f\big|_{[-L,L]} \in L^2[-L,L]$, the coefficients of the Fourier
expansion are also given by
$\displaystyle a_0 = \frac{1}{2L} \int_{\alpha}^{\alpha+2L} f(x) \dx{x}$,
$\displaystyle a_n = \frac{1}{L} \int_{\alpha}^{\alpha+2L} f(x)
\cos\left(\frac{n\pi x}{L}\right) \dx{x}$ for $n \geq 1$ and
$\displaystyle b_n = \frac{1}{L} \int_{\alpha}^{\alpha+2L} f(x)
\sin \left(\frac{n\pi x}{L}\right) \dx{x}$ for $n \geq 1$,
where $\alpha$ may be any real number.

\begin{egg}
Find the Fourier series of the periodic function $f$ of period $2\pi$
defined by
\[
f(x) = \begin{cases}
0 & \quad \text{if} \ -\pi \leq x \leq 0 \\
1 & \quad \text{if} \ 0 < x < \pi
\end{cases}
\]

We have
\begin{align*}
a_0 &= \frac{1}{2\pi} \int_{-\pi}^\pi f(x) \dx{x} =
\frac{1}{2\pi} \int_0^\pi \dx{x} = \frac{1}{2} \ , \\
a_n &= \frac{1}{\pi} \int_{-\pi}^\pi f(x) \cos(nx) \dx{x}
= \frac{1}{\pi} \int_0^\pi \cos(nx) \dx{x}
= \frac{\sin(nx)}{n\pi}\bigg|_0^\pi = 0
\intertext{and}
b_n &= \frac{1}{\pi} \int_{-\pi}^\pi f(x) \sin(nx) \dx{x}
= \frac{1}{\pi} \int_0^\pi \sin(nx) \dx{x}
= -\frac{\cos(nx)}{n\pi}\bigg|_0^\pi
= \begin{cases}
2/(n\pi) & \quad \text{if $n$ is odd} \\
0 & \quad \text{if $n$ is even}
\end{cases}
\end{align*}
Hence,
\[
f = \frac{1}{2} + \sum_{m=1}^\infty
  \frac{2}{(2m-1)\pi}\sin((2m-1)x) \ .
\]
As predicted by Theorem~\ref{faFsPwc}, we have that the value of the
series is $1/2$ when $x=n\pi$ with $n\in \ZZ$.
\end{egg}

\begin{egg}
Find the Fourier series of the periodic function $f$ of period $6$
defined by
\[
f(x) = \begin{cases}
1 & \quad \text{if} \ -3 \leq x \leq 0 \\
x & \quad \text{if} \ 0 < x < 3
\end{cases}
\]

We have
\begin{align*}
a_0 &= \frac{1}{6} \int_{-3}^3 f(x) \dx{x} =
\frac{1}{6} \int_{-3}^0 \dx{x} + \frac{1}{6} \int_0^3 x \dx{x}
= \frac{5}{4} \ , \\
a_n &= \frac{1}{3} \int_{-3}^3 f(x) \cos\left(\frac{n\pi x}{3}\right) \dx{x}
= \frac{1}{3} \int_{-3}^0 \cos\left(\frac{n\pi x}{3}\right) \dx{x}
+ \frac{1}{3} \int_0^3 x\cos\left(\frac{n\pi x}{3}\right) \dx{x} \\
&= \frac{1}{n\pi}\sin\left(\frac{n\pi x}{3}\right)\bigg|_{-3}^0
+ \left( \frac{x}{n\pi} \sin\left(\frac{n\pi x}{3}\right)
+ \frac{3}{n^2\pi^2} \cos\left(\frac{n\pi x}{3}\right) \right)\bigg|_0^3 \\
&= \frac{3}{n^2\pi^2} (\cos(n\pi)-1)
= \begin{cases}
0 & \quad \text{if $n$ is even} \\
-6/(n^2 \pi^2) & \quad \text{if $n$ is odd}
\end{cases}
\intertext{and}
b_n &= \frac{1}{3} \int_{-3}^3 f(x) \sin\left(\frac{n\pi x}{3}\right) \dx{x}
= \frac{1}{3} \int_{-3}^0 \sin\left(\frac{n\pi x}{3}\right) \dx{x}
+ \frac{1}{3} \int_0^3 x \sin\left(\frac{n\pi x}{3}\right) \dx{x} \\
&= -\frac{1}{n\pi} \cos\left(\frac{n\pi x}{3}\right)\bigg|_{-3}^0
+ \left( -\frac{x}{n\pi} \cos\left(\frac{n\pi x}{3}\right)
+ \frac{3}{n^2\pi^2} \sin\left(\frac{n\pi x}{3}\right) \right)\bigg|_0^3 \\
&= \left( -\frac{1}{n\pi} + \frac{1}{n\pi} \cos(n\pi)\right)
- \frac{3}{n\pi}\cos(n\pi)
= \begin{cases}
1/(n\pi) & \quad \text{if $n$ is odd} \\
-3/(n\pi) & \quad \text{if $n$ is even}
\end{cases}
\end{align*}
Hence,
\begin{align*}
f &= \frac{5}{4} - \sum_{m=1}^\infty
  \frac{6}{(2m-1)^2\pi^2}\cos\left(\frac{(2m-1)\pi x}{3}\right)
- \sum_{m=1}^\infty \frac{3}{2m\pi} \sin\left(\frac{2m\pi x}{3}\right) \\
&\quad + \sum_{m=1}^\infty \frac{1}{(2m-1)\pi}
\sin\left(\frac{(2m-1)\pi x}{3}\right) \ .
\end{align*}
From Theorem~\ref{faFsPwc}, we have that
\[
\frac{1}{2} = \frac{5}{4} - \sum_{m=1}^\infty \frac{6}{(2m-1)^2\pi^2} \ .
\]
This is the value of the series at $x = 6n$ with $n \in \ZZ$.
Moreover,
\[
2 = \frac{5}{4} + \sum_{m=1}^\infty \frac{6}{(2m-1)^2\pi^2} \ .
\]
This is the value of the series at $x = 3+ 6n$ with $n \in \ZZ$.
\end{egg}

\subsection{Odd and Even Functions}

We may use the properties that
$\displaystyle \int_{-L}^L g(t) \dx{t} = 0$ if $g$ is an odd
function and $\displaystyle \int_{-L}^L g(t) \dx{t} = 2 \int_0^L g(t) \dx{t}$
if $g$ is an even function to obtain simpler formulae for the Fourier
series of odd and even functions.

If $\displaystyle f \in L^2[-L,L]$ is an even function, its Fourier expansion is
\begin{equation} \label{OEfunctEq1}
a_0 + \sum_{n=1}^{\infty} \,  a_n \cos\left(\frac{n\pi x}{L}\right)  \ ,
\end{equation}
where
$\displaystyle a_0 = \frac{1}{L} \int_{0}^{L} f(x) \dx{x}$ and
$\displaystyle a_n = \frac{2}{L} \int_{0}^{L} \, f(x)
\cos\left( \frac{n\pi x}{L}\right) \dx{x}$ for $n \geq 1$.

If $\displaystyle f\in L^2[-L,L]$ is an odd function, its Fourier expansion is
\begin{equation} \label{OEfunctEq2}
\sum_{n=1}^{\infty} \, b_n \sin \left( \frac{n\pi x}{L}\right) \ ,
\end{equation}
where $\displaystyle b_n = \frac{2}{L} \int_{0}^{L} \, f(x)
\sin \left( \frac{n\pi x}{L}\right) \dx{x}$ for $n \geq 1$.

The {\bfseries Fourier sine series}\index{Fourier Sine Series} of a function
$f:[0,L] \rightarrow \RR$ is the Fourier series of the odd function
$F:[-L,L]\rightarrow \RR$ defined by
\[
F(x) =
\begin{cases}
f(x) & \quad \text{if} \ 0 \leq x \leq L \\
-f(-x) & \quad \text{if} \ -L < x < 0
\end{cases}
\]
if $\displaystyle F \in L^2[-L,L]$.
The {\bfseries Fourier cosine series}\index{Fourier Cosine Series} of
a function $f:[0,L] \rightarrow \RR$ is the Fourier series of the even
function $F:[-L,L]\rightarrow \RR$ defined by
\[
F(x) =
\begin{cases}
f(x) & \quad \text{if} \ 0 \leq x \leq L \\
f(-x) & \quad \text{if} \ -L \leq x < 0
\end{cases}
\]
if $\displaystyle F\in L^2[-L,L]$.

The Fourier series in (\ref{OEfunctEq1}) and (\ref{OEfunctEq1})
converge to $\displaystyle f = F\big|_{[0,L]}$ in
$\displaystyle L^2[0,L]$ because they converge to
$F$ in $\displaystyle L^2[-L,L]$.

\begin{egg}
Find the Fourier series of the periodic function $f$ of period $\pi$
defined by $f(x) = |\sin(x)|$ whose graph is given in
Figure~\ref{abssin}.

\pdfF{funct_anal/abssin}{Graph of the $\pi$-period function
define by $f(x) = |\sin(x)|$}{Graph of the $\pi$-period function
define by $f(x) = |\sin(x)|$}{abssin}

We note that $f$ is an even function.  Thus, the coefficients $b_n$ of
the Fourier series are all null.  We have $L = \pi/2$ in the formulae
for the $a_n$.  Hence
\begin{align*}
a_0 &= \frac{2}{\pi} \int_0^{\pi/2} f(x) \dx{x} =
\frac{2}{\pi} \int_0^{\pi/2} \sin(x) \dx{x} =
-\frac{2}{\pi} \cos(x) \bigg|_0^{\pi/2} = \frac{2}{\pi}
\intertext{and}
a_n &= \frac{4}{\pi} \int_0^{\pi/2} f(x) \cos(2n x) \dx{x}
= \frac{4}{\pi} \int_0^{\pi/2} \sin(x) \cos(2n x) \dx{x} \\
&= \frac{2}{\pi} \int_0^{\pi/2} \left(
\sin( (2n+1)x) - \sin((2n-1)x) \right) \dx{x} \\
&= \frac{2}{\pi} \left( -\frac{\cos( (2n+1)x)}{2n+1}
 + \frac{\cos((2n-1)x)}{2n-1} \right)\bigg|_0^{\pi/2}
= \frac{4}{(1-4n^2)\pi} \ .
\end{align*}
We get
\[
f = \frac{2}{\pi} + 
\sum_{n=1}^\infty \frac{4}{(1-4n^2)\pi} \cos(2n x) \ .
\]
\end{egg}

\begin{egg}
Find the Fourier cosine series of the function $\displaystyle f(x) = e^x$ for
$0 \leq x <1$.

In other words, we want the Fourier series of the even function $F$ of
period $2$ defined by
\[
F(x) = \begin{cases}
e^x & \quad \text{if} \ 0\leq x \leq 1 \\
e^{-x} & \quad \text{if} \ -1\leq x < 0
\end{cases}
\]
Since $F$ is an even function, the coefficients $b_n$ of the Fourier
series are all null.  We have $L = 1$ in the formulae for the $a_n$.
Hence
\begin{align*}
a_0 &= \int_0^1 f(x) \dx{x} = \int_0^1 e^x \dx{x} = e - 1
\intertext{and}
a_n &= 2 \int_0^1 f(x) \cos(n \pi x) \dx{x}
= 2 \int_0^1 e^x \cos(n\pi x) \dx{x} \\
&= \frac{2 (n\pi)^2}{(n\pi)^2+1}
\left( e^x \frac{\sin(n\pi x)}{n\pi}
+ e^x \frac{\cos(n\pi x)}{(n\pi)^2}\right)\bigg|_0^1 
= \frac{2}{(n\pi)^2+1} \left( e \cos(n\pi) - 1\right) \\
&= \begin{cases}
\displaystyle \frac{2(e-1)}{(n\pi)^2+1} &\quad \text{if $n$ is even}\\
\displaystyle \frac{-2(e+1)}{(n\pi)^2+1} &\quad \text{if $n$ is odd}\\
\end{cases}
\end{align*}
We get
\[
e^x = e - 1 + \sum_{m=1}^\infty \frac{2(e-1)}{(2m\pi)^2+1} \cos(2m \pi x)
- \sum_{m=1}^\infty \frac{2(e+1)}{((2m-1)\pi)^2+1} \cos((2m-1) \pi x)
\]
for $0 \leq x < 1$.
\end{egg}

\subsection{More Special Cases}\label{period4L}

By expanding the definition of a function $f:[0,L] \to \RR$ to an
interval larger than $[0,L]$, it is possible to obtain different
Fourier series of $f$ on $[0,L]$ in terms of the sine and cosine
functions.

For instance, if we expand a function $\displaystyle f \in L^2[0,L]$ to an even
function $F[-2L,2L]\rightarrow \RR$ as follows
\[
F(x) = \begin{cases}
-f(2L-x) & \quad \text{if} \ L<x\leq 2L \\
f(x) &  \quad \text{if} \ 0 \leq x\leq L \\
f(-x) & \quad \text{if} \ -L\leq x < 0 \\
-f(2L+x) & \quad \text{if} \ -2L\leq x < -L
\end{cases}
\]
then $\displaystyle F\in L^2[-2L,2L]$ and the Fourier cosine series of $F$ on
$[0,2L]$ restricted to $[0,L]$ gives the following series
representation of $f$ on $[0,L]$.
\begin{equation} \label{moreFSegg1}
\sum_{n=1}^{\infty} \, a_n \cos\left(\frac{(2n-1)\pi x}{2L}\right)
\end{equation}
for $x\in [0,L]$, where
$\displaystyle a_n = \frac{2}{L} \int_{0}^{L} \, f(x)
\cos\left(\frac{(2n-1) \pi x}{2L}\right) \dx{x}$ for $n \geq 1$.
To obtain this formula, we need to note that
\[
\cos\left(\frac{m \pi x}{2L}\right)
= \begin{cases}
\displaystyle
\cos\left(\frac{m \pi (2L-x)}{2L}\right) & \quad \text{if $m$ even} \\
\displaystyle
-\cos\left(\frac{m \pi (2L-x)}{2L}\right) & \quad \text{if $m$ odd}
\end{cases}
\]
for $L \leq x \leq 2L$.  Hence,
\[
F(x) \cos\left(\frac{m \pi x}{2L}\right)
= F(x) \cos\left(\frac{m \pi (2L-x)}{2L}\right)
= -F\left( 2L-x\right) \cos\left(\frac{m \pi (2L-x)}{2L}\right)
\]
for $m$ even and $L \leq x \leq 2L$, and
\[
F(x) \cos\left(\frac{m \pi x}{2L}\right)
= -F(x) \cos\left(\frac{m \pi (2L-x)}{2L}\right)
= F\left( 2L-x\right) \cos\left(\frac{m \pi (2L-x)}{2L}\right)
\]
for $m$ odd and $L \leq x \leq 2L$.  We then get that
\[
a_m = \frac{2}{2L} \int_{0}^{2L} \, f(x)
\cos\left(\frac{m \pi x}{2L}\right) \dx{x}
= \begin{cases}
\displaystyle \frac{2}{L} \int_{0}^{L} \, f(x)
\cos\left(\frac{m \pi x}{2L}\right) \dx{x}
& \quad \text{if $m$ odd} \\
0 & \quad \text{if $m$ even}
\end{cases}
\]

If we expand a function $\displaystyle f \in L^2[0,L]$ to an odd function
$F[-2L,2L]\rightarrow \RR$ as follows
\[
F(x) = \begin{cases}
f(2L-x) & \quad \text{if} \ L<x\leq 2L \\
f(x) &  \quad \text{if} \ 0 \leq x\leq L \\
-f(-x) & \quad \text{if} \ -L\leq x < 0 \\
-f(2L+x) & \quad \text{if} \ -2L\leq x < -L
\end{cases}
\]
then $\displaystyle F\in L^2[-2L,2L]$ and the Fourier sine series of $F$ on
$[0,2L]$ restricted to $[0,L]$ gives the following series
representation of $f$ on $[0,L]$.
\begin{equation} \label{moreFSegg2}
\sum_{n=1}^{\infty} \, b_n \sin\left(\frac{(2n-1)\pi x}{2L}\right)
\end{equation}
for $x\in [0,L]$, where
$\displaystyle b_n = \frac{2}{L} \int_{0}^{L} \, f(x)
\sin\left(\frac{(2n-1)\pi x}{2L}\right) \dx{x}$ for $n \geq 1$.
One may proceed as we did above to obtain this results.

The Fourier series in (\ref{moreFSegg1}) and (\ref{moreFSegg2})
converge to $\displaystyle f = F\big|_{[0,L]}$ in $\displaystyle L^2[0,L]$
because they converge to $F$ in $\displaystyle L^2[-2L,2L]$.

\subsection{Fourier Series of Functions of Two Variables}

If, in the general definition of $\displaystyle L^2$ spaces, we have
$\Omega =[a,b]\times[b,c]$ and $\mu$ is the Lebesgue measure on
$\Omega$, then we get the space
$\displaystyle L^2([a,b]\times[b,c]) = L^2([a,b]\times[b,c],\mu)$ of
$\displaystyle L^2$ integrable functions on the rectangle
$[a,b]\times[b,c]$.  We invite the reader
to write the definitions of scalar product, norm, orthogonal set,
complete set, \ldots in the special case of
$\displaystyle L^2([a,b]\times[c,d])$.

We can extend in a natural way the concept of classical Fourier series
to functions in $\displaystyle L^2([a,b]\times[b,c])$.  For instance,
the Fourier sine series for a function
$\displaystyle f \in L^2([0,a]\times[0,b])$ is
\[
\sum_{m=1}^{\infty} \sum_{n=1}^{\infty} \, c_{m,n}
\sin\left(\frac{m \pi x}{a} \right)\sin\left(\frac{n \pi y}{b} \right)
\ ,
\]
where $\displaystyle c_{m,n} = \frac{4}{ab} \int_{0}^{b} \int_{0}^{a} f(x,y)
\sin \left(\frac{m \pi x}{a}\right) \sin\left(\frac{n \pi y}{b}\right)
\dx{x}\dx{y}$ for $n,m \geq 1$.

We leave it to the reader to provide other examples of classical
Fourier series in $\displaystyle L^2([a,b]\times[c,d])$.

\section{Operator Theory}

\subsection{Bounded Operators}

We start be reviewing some definitions about mapping between Banach spaces.
In this section, $X$ and $Y$ are two Banach spaces over the complex
numbers.

\begin{defn}
A {\bfseries linear operator}\index{Linear Operator}
or {\bfseries linear mapping}\index{Linear Mapping} from $X$
to $Y$ is a function $A:X \rightarrow Y$ such that
$A(\lambda_1 x_1 + \lambda_2 x_2) = \lambda_1 A(x_1) + \lambda_2 A(x_2)$
for all $\lambda_1, \lambda_2 \in \CC$ and $x_1, x_2 \in X$.
If $Y=\CC$, then the linear operator $A$ is called a
{\bfseries linear functional}\index{Linear Functional}.

The {\bfseries range}\index{Linear Mapping!Range} of $A$ is the
linear space
$\displaystyle \IMG(A) = \left\{ y \in Y : y = A(x) \ \text{for some} \ x\in X\right\}$.
The {\bfseries kernel}\index{Linear Mapping!Kernel} of $A$ is
the linear space
$\displaystyle \KE(A) = \left\{ x \in X : A(x) = 0 \right\}$.
\end{defn}

\begin{defn}
A linear operator $A:X\rightarrow Y$ is
{\bfseries bounded}\index{Linear Mapping!Bounded} if there
exist a constant $C\geq 0$ such that $\|A(x)\| \leq C \|x\|$ for all
$x\in X$. The space of bounded linear operators from $X$ to $Y$ is
denoted $\LL(X,Y)$.  If $Y=X$, The space of bounded linear operators
on $X$ is simply denoted $\LL(X)$.
\end{defn}

The following results are fundamental results about bounded linear
operators between Banach spaces.

\begin{prop}
Let
\[
\|A\| = \sup_{\substack{x\in X\\ \|x\|=1}} \|A(x)\| \quad ,
\quad A \in \LL(X,Y) \ .
\]
\begin{enumerate}
\item $\|\cdot\|$ defines a norm on $\LL(X,Y)$ called the
{\bfseries induce norm from $X$ and $Y$}\index{Induce Norm}.
\item Equipped with the norm $\|\cdot\|$, $\LL(X,Y)$ is a Banach
space.
\end{enumerate}
\end{prop}

\begin{theorem}[Open Mapping Theorem]
If $A\in \LL(X,Y)$ is onto, then the image of an open set is an
open set.    \index{Open Mapping Theorem}
\end{theorem}

\begin{theorem}[Bounded Inverse Theorem]
If $A\in \LL(X,Y)$ is one-to-one and onto, then
$\displaystyle A^{-1} \in \LL(Y,X)$.
\index{Bounded Inverse Theorem}  
\end{theorem}

\begin{defn}
The {\bfseries graph}\index{Linear Mapping!Graph} of a linear operator
$A:X\rightarrow Y$ is the subspace
\[
\GR(A) = \left\{ (x,A(x)) : x \in X \right\} \subset X \times Y  \ .
\]
A linear operator $A:X\rightarrow Y$ is
{\bfseries closed}\index{Linear Mapping!Closed} if its
graph is closed in $X\times Y$.  In other words, a linear operator
$A:X \rightarrow Y$ is closed if, for any converging sequence
$\displaystyle \left\{ (x_n, A(x_n)) \right\}_{n=0}^\infty \subset \GR(A)$,
we have that
$\displaystyle (x,y) = \lim_{n\to\infty} (x_n, A(x_n)) \in \GR(A)$; namely,
$y=A(x)$.
\end{defn}

\begin{theorem}[Closed Graph Theorem]
If $A:X\rightarrow Y$ is a closed linear operator, then $A \in \LL(X,Y)$.
\index{Closed Graph Theorem}
\end{theorem}

\begin{defn}
The {\bfseries resolvent set}\index{Linear Mapping!Resolvent Set}
$\rho(A)$ of $A \in \LL(X)$ is the set
of all $\lambda \in \CC$ such that $A-\lambda \Id: X \rightarrow X$ is
one-to-one and onto.
The {\bfseries spectrum}\index{Linear Mapping!Spectrum}
of $A \in \LL(X)$ is the set $\sigma(A) = \CC \setminus \rho(A)$.
\end{defn}

From the Open Mapping Theorem, we have that
$\displaystyle (A-\lambda \Id)^{-1} \in \LL(X)$ if $\lambda$ is in the
resolvent set of $A$.

\begin{defn}
If $\lambda$ is in the resolvent set of $A$, then
$\displaystyle R_\lambda(A) = (A-\lambda\Id)^{-1}$ is called the
{\bfseries resolvent}\index{Linear Mapping!Resolvent}
of $A$.
\end{defn}

$\lambda \in \sigma(A)$ if $A-\lambda \Id$ is not one-to-one, or
$A-\lambda \Id$ is not onto, or both.

\begin{defn}
If $A-\lambda \Id$ is not one-to-one, then $\lambda$ is called an
{\bfseries eigenvalue}\index{Linear Mapping!Eigenvalue} of $A$.  We have that
$\KE(A-\lambda\Id) \neq \{0\}$ and the elements of 
$\KE(A-\lambda\Id) \setminus \{0\}$ are called the
{\bfseries eigenvectors}\index{Linear Mapping!Eigenvectors} associated
to $\lambda$.  $\KE(A-\lambda\Id)$
is called the {\bfseries eigenspace}\index{Linear Mapping!Eigenspace}
associated to $\lambda$.
\end{defn}

If $x$ is en eigenvector associated to $\lambda$, then
$Ax = \lambda x$.

\begin{prop}
If $A \in \LL(X)$ and $\lambda \in \sigma(A)$, then $|\lambda|\leq \|A\|$.
\end{prop}

\begin{defn}
The {\bfseries spectral radius}\index{Linear Mapping!Spectral Radius}
of $A\in \LL(X)$ is defined as
$\displaystyle r_\sigma(A) = \sup_{\lambda\in \sigma(A)} |\lambda|$.
\end{defn}

From the previous proposition, $r_\sigma(A) \leq \|A||$.

We list below some of the important properties of
the spectrum of a bounded linear operator.

\begin{prop}
If $A \in \LL(X)$, then $\sigma(A) \subset \CC$ is compact.
\end{prop}

\begin{prop}
If $A \in \LL(X)$ and $X\neq \{0\}$, then $\sigma(A) \neq \emptyset$.
\end{prop}

The following definition generalizes the definition of dual space
given in Definition~\ref{defnCLFunctHilbert}.

\begin{defn}
Let $X$ be a Banach space.  The {\bfseries dual space}\index{Dual Space}
$\displaystyle X^\ast$ of $X$ is the set of all bounded linear
functionals on $X$.
\end{defn}

\begin{egg}
If $H$ is an Hilbert space, we have from the Riesz Representation
Theorem, Theorem~\ref{fu_an_RieszRT}, that there exists a mapping
$\displaystyle T:H^\ast \to H$ such that $u(x) = \ps{x}{T(u)}$ for all
$x \in H$.  It follows from Schwarz inequality that
$|u(x)| \leq \|x\|\,\|T(u)\|$ for all $x \in H$.  Thus,
$\|u\| \leq \|T(u)\|$ where the norm on the left is the norm of
the linear mapping $u:H \to \RR$ and the norm on the right is the norm
of $T(u) \in H$.  In fact, we have that $\|u\|=\|T(u)\|$ because
$\displaystyle u\left( \|T(u)\|^{-1} T(u)\right) = \|T(u)\|$.
Hence, $\displaystyle T:H^\ast \to H$ is an isometry from
$\displaystyle H^\ast$ into $H$.
\end{egg}

\begin{defn}
The {\bfseries adjoint}\index{Linear Mapping!Adjoint} of
$A\in \LL(X,Y)$ is the linear mapping
$\displaystyle A^\ast : Y^\ast \rightarrow X^\ast$ defined by
\[
A^\ast(w)(x) = w(A(x)) \quad , \quad x \in X \ \text{and} \ w \in Y^\ast \ .
\]
\end{defn}

\begin{theorem}
If $A \in \LL(X,Y)$, then $\displaystyle A^\ast \in \LL(Y^\ast, X^\ast)$ and
$\displaystyle \|A\| = \|A^\ast\|$
\end{theorem}

Obviously, in the previous theorem,
\[
\| A^\ast \| =  \sup_{\substack{w\in Y^\ast\\ \|w\|=1}} \|A^\ast(w)\| 
\quad , \quad A^\ast \in \LL(Y^\ast,X^\ast) \ ,
\]
and
\[
\| w \| = \sup_{\substack{y\in Y\\ \|y\|=1}} |w(y)| 
\quad , \quad w \in Y^\ast \ ,
\]
with a similar definition for the norm on $\displaystyle X^\ast$.

\begin{defn}
A linear operator $A \in \LL(X,Y)$ is
{\bfseries compact}\index{Linear Mapping!Compact} if
$\overline{A(V)}$ is compact for every bounded set $V \subset X$.
The set of all compact linear operators from $X$ to
$Y$ is denoted $\KK(X,Y)$ and the set of all compact linear operators from $X$
to itself is denoted $\KK(X)$.
\end{defn}

A little review of some topological concepts are useful for future
applications.

\begin{defn}
A set $U \subset X$ is {\bfseries precompact}\index{Precompact Set} if
$\overline{U}$ is compact.  A set $U \subset X$ is
{\bfseries sequentially compact}\index{Sequentially Compact Set}
if every sequence in $U$ has a subsequence converging in $X$.
\end{defn}

Inside a Banach space, these two concepts are equivalent.

\begin{prop} \label{fu_an_Kclose}
$\KK(X,Y)$ is a closed subspace of $\LL(X,Y)$ with respect to the
operator norm induced by the norms on $X$ and $Y$.
\end{prop}

\begin{prop} \label{fu_an_comp_cont}
Let $X$, $Y$ and $Z$ be three Banach spaces.
\begin{enumerate}
\item If $A \in \KK(X,Y)$ and $B \in \LL(Y,Z)$, then $B\circ A \in \KK(X,Z)$.
\item If $A \in \LL(X,Y)$ and $B \in \KK(Y,Z)$, then $B\circ A \in \KK(X,Z)$.
\end{enumerate}
\end{prop}

\begin{proof}
We prove (2).  Let $U\subset X$ be a bounded set, then
$A(U) \subset Y$ is bounded because $A$ is bounded\footnote{If
$B_r(\VEC{0}) \supset U$, then $B_{\|A\|r}(\VEC{0}) \supset A(U)$.}.
Thus, $\overline{B(A(U))}$ is a compact set since $B$ is a compact
operator and $A(U)$ is a bounded set.
\end{proof}

\begin{theorem} \label{fa_finit_compact}
Let $A$ be a linear operator from $X$ to $Y$.
\begin{enumerate}
\item If $A \in \LL(X,Y)$ and $\dim \IMG(A) < \infty$, then $A \in \KK(X,Y)$.
\item If $A \in \KK(X,Y)$ and $\IMG(A)$ is closed, then
$\dim \IMG(A) < \infty$.
\item $A \in \KK(X,Y)$ if and only if
$\displaystyle A^\ast \in \KK(Y^\ast, X^\ast)$.
\end{enumerate}
\end{theorem}

Surprisingly, the next theorem will be quite useful for the study of
differential equations.  For instance, it will be used to describe the
eigenvalues for the Sturm-Liouville problem and the Laplace operator.

\begin{theorem} \label{fu_an_comp_oper}
Consider $A \in \KK(X)$.
\begin{enumerate}
\item If $\lambda \neq 0$, then $\dim \KE(A-\lambda \Id) < \infty$.
\item If $\lambda \neq 0$, then $\IMG(A-\lambda \Id)$ is closed.
\item If $\lambda \neq 0$, then
$\displaystyle \dim \KE(A-\lambda \Id) = \dim \KE(A^\ast-\lambda \Id)$.
\item If $r>0$ and $E$ is the set of eigenvalues $\lambda$ of $A$ such
that $|\lambda|>r$, then $E$ is finite and $\IMG(A-\lambda \Id) \neq X$
for all $\lambda \in E$.
\item If $\lambda \in \sigma(A)$ and $\lambda \neq 0$, then $\lambda$
is an eigenvalue.
\item $\sigma(A)$ is compact and at most countable.  If $\sigma(A)$ is
countable, then $0$ is the only limit point of $\sigma(A)$.
\end{enumerate}
\end{theorem}

\subsection{Unbounded Operators}

As in the previous section, $X$ and $Y$ are two Banach spaces over the
complex numbers.  We consider linear operators from a subspace $Z$ of
$X$ to $Y$ where $Z$ may not be equal to $X$.  We generalize many
of the definitions and results that we have presented in the previous
section to these new linear operators.

\begin{defn}
A {\bfseries linear operator}\index{Linear Operator} or
{\bfseries linear mapping}\index{Linear Mapping} from $X$
to $Y$ is a pair $(\DO(A),A)$, where $\DO(A)$ is a subspace of $X$
and $A:\DO(A) \rightarrow Y$ is such that
$A(\lambda_1 x_1 + \lambda_2 x_2) = \lambda_1 A(x_1) + \lambda_2 A(x_2)$
for all $\lambda_1, \lambda_2 \in \CC$ and $x_1, x_2 \in \DO(A)$.

$\DO(A)$ is called the {\bfseries domain}\index{Linear Mapping!Domain}
of $A$.  The {\bfseries range}\index{Linear Mapping!Range} of $A$ is
the linear space
$\displaystyle \IMG(A) = \left\{ y \in Y : y = A(x) \ \text{for some}
\ x\in \DO(A)\right\}$.
The {\bfseries kernel}\index{Linear Mapping!Kernel} of $A$ is the
linear space
$\displaystyle \KE(A) = \left\{ x \in \DO(A) : A(x) = 0 \right\}$.
\end{defn}

\begin{defn}
A linear operator $(\DO(A),A)$ is
{\bfseries bounded}\index{Linear Mapping!Bounded} if there
exists a constant $C\geq 0$ such that $\|A(x)\| \leq C \|x\|$ for all
$x\in \DO(X)$.
\end{defn}

The following result is well know for linear operators between finite
dimensional vector spaces.

\begin{prop}
If $(\DO(A),A)$ is a linear operator such that $\dim \DO(A) < \infty$,
then $(\DO(A),A)$ is bounded.
\end{prop}

\begin{defn}
A linear operator $(\DD(B),B)$ from $X$ to $Y$ is an
{\bfseries extension}\index{Linear Mapping!Extension} of a
linear operator $(\DD(A),A)$ from $X$ to $Y$ if $\DO(A) \subset \DO(B)$ and 
$A(x)=B(x)$ for all $x \in \DO(A)$.
\end{defn}

\begin{theorem}
Given a linear operator $(\DO(A),A)$ such that
$\DO(A)$ is a closed subspace of $X$ and
\begin{equation} \label{FA_UBOnorm}
M = \sup_{\substack{x\in\DO(A)\\ \|x\|=1}} \|A(x)|\ < \infty \ ,
\end{equation}
then there exists a bounded linear operator $B \in \LL(X,Y)$ such that
$(X,B)$ is an extension of $(\DO(A),A)$ and $\|B\| = M$.
$M$ defined by (\ref{FA_UBOnorm}) is the norm of $A$ acting on the
subspace $\DD(A)$.
\end{theorem}

\begin{defn}
The {\bfseries graph}\index{Linear Mapping!Graph} of a linear operator
$(\DO(A),A)$ is the subspace
\[
\GR(A) = \left\{ (x,A(x)) : x \in \DO(A) \right\} \subset X \times Y  \ .
\]
\end{defn}

\begin{defn}
A linear operator $(\DO(A),A)$ is
{\bfseries closed}\index{Linear Mapping!Closed} if its graph is
closed in $X\times Y$.  In other words, a linear operator $(\DO(A),A)$
is closed if, for any converging sequence
$\displaystyle \left\{ (x_n, A(x_n)) \right\}_{n=0}^\infty \subset \GR(A)$,
we have
$\displaystyle (x,y) = \lim_{n\to \infty} (x_n, A(x_n)) \in \GR(A)$;
namely, $x\in \DO(A)$ and $y=A(x)$.

The {\bfseries closure}\index{Linear Mapping!Closure} of an operator
$(\DO(A),A)$ is the smallest closed extension of $(\DO(A),A)$ which we
denoted by $(\DO(\,\overline{A}\,),\overline{A})$.
\end{defn}

\begin{theorem}[Closed Graph Theorem]
If $(\DO(A),A)$ is a closed linear operator such that $\DO(A)$ is
dense in $X$, then $(\DO(A),A)$ is bounded.
\index{Closed Graph Theorem}
\end{theorem}

\begin{defn}
The {\bfseries resolvent set}\index{Linear Mapping!Resolvent Set}
$\rho(A)$ of a closed linear operator
$(\DO(A),A)$ from $X$ to itself is the set of all $\lambda \in \CC$
such that $A-\lambda \Id: \DO(A) \rightarrow X$ is one-to-one and
onto, and $\displaystyle (A-\lambda\Id)^{-1} \in \LL(X)$.
The {\bfseries spectrum}\index{Linear Mapping!Spectrum} of $A$ is the set
$\sigma(A) = \CC \setminus \rho(A)$.
If $\lambda$ is in the resolvent set of $A$, then
$\displaystyle R_\lambda(A) \equiv (A-\lambda\Id)^{-1}$ is
called the {\bfseries resolvent}\index{Linear Mapping!Resolvent} of $A$.
\end{defn}

\begin{prop}
$\lambda \in \sigma(A)$ falls into one of the following three classes.
\begin{enumerate}
\item The {\bfseries point spectrum}\index{Linear Mapping!Point Spectrum}
  or {\bfseries discrete spectrum}\index{Linear Mapping!Discrete Spectrum}
  is the set $\sigma_{ps}(A)$ of all $\lambda \in \sigma(A)$ such that
  $A-\lambda \Id$ is not one-to-one.
\item The {\bfseries residual spectrum}\index{Linear Mapping!Residual Spectrum}
or {\bfseries compression spectrum}\index{Linear Mapping!Compression Spectrum}
is the set $\sigma_{rs}(A)$ of all $\lambda$ such that $A-\lambda \Id$
is one-to-one but $\IMG(A-\lambda \Id)$ is not a dense subspace of $X$.
\item The {\bfseries continuous spectrum}\index{Linear
Mapping!Continuous Spectrum} 
is the set $\sigma(A) \setminus ( \sigma_{ps}(A) \cup \sigma_{rs}(A))$.
\end{enumerate}
\end{prop}

\begin{defn}
If $\lambda$ is in the point spectrum, then $\lambda$ is called
an {\bfseries eigenvalue}\index{Linear Mapping!Eigenvalue} of $A$.  We
have that $\KE(A-\lambda\Id) \neq \{0\}$ and the elements of
$\KE(A-\lambda\Id) \setminus \{0\}$ are
called the {\bfseries eigenvectors}\index{Linear Mapping!Eigenvectors}
associated to $\lambda$.  $\KE(A-\lambda\Id)$ is called the
{\bfseries eigenspace}\index{Linear Mapping!Eigenspace} associated to
$\lambda$.
\end{defn}

If $x$ is en eigenvector associated to $\lambda$, then $Ax = \lambda x$.
As we have in finite dimension, eigenvectors associated to distinct
eigenvalues are independent.

\begin{defn}
Let $(\DO(A),A)$ be a linear mapping from $X$ into $Y$ such that
$\DO(A)$ is dense in $X$.  Let $S$ be the set of all bounded linear
functionals $\displaystyle w \in Y^\ast$ such that there exists a
bounded linear functional $\displaystyle v \in X^\ast$ satisfying
$w(A(x)) = v(x)$ for all $x \in \DO(A)$.
Since $\DO(A)$ is dense, $v$ is uniquely defined.  The
{\bfseries adjoint}\index{Linear Mapping!Adjoint}
of $A$ is the mapping $\displaystyle (S,A^\ast)$ from $\displaystyle Y^\ast$
to $\displaystyle X^\ast$ defined by $\displaystyle A^\ast(w) = v$ for
$w\in S$.  We have that $\displaystyle \DO(A^\ast) = S$.
\end{defn}

\begin{theorem}
Let $(\DO(A),A)$ be a linear mapping from $X$ into $Y$ such that
$\DO(A)$ is dense in $X$, then $\displaystyle (\DO(A^\ast),A^\ast)$ is a closed
linear operator.  Moreover, if $X$ and $Y$ are
{\bfseries reflexive}\index{Reflexive Space}
(i.e.\ $\displaystyle X^{\ast\ast}=X$ and $\displaystyle Y^{\ast\ast} = Y$),
then $\displaystyle \DO(A^\ast)$ is dense in
$\displaystyle Y^\ast$ and $\displaystyle A^{\ast\ast} = A$.
\end{theorem}

\begin{defn}
Let $(\DO(A),A)$ be a bounded linear operator from $X$ to $Y$.  The
linear operator $(\DO(A),A)$ is
{\bfseries compact}\index{Linear Mapping!Compact} if
$\overline{A(V)}$ is compact for every bounded set $V \subset \DD(A)$.
\end{defn}

The next theorem is not surprising if we think about linear mappings
between finite dimensional vector spaces.

\begin{theorem}
Let $(\DO(A),A)$ be a linear operator from $X$ to $Y$, then
\begin{enumerate}
\item $(\DO(A),A)$ is a compact linear operator if $(\DO(A),A)$ is
bounded and $\IMG(A)$ is finite dimensional.
\item $(\DO(A),A)$ is a compact linear operator if $\DO(A)$ is finite
dimensional.
\end{enumerate}
\end{theorem}

\subsection{Operators on Hilbert Spaces}

In the present section, $H$, $H_1$, $H_2$, \ldots will be Hilbert
spaces over the complex numbers.

Obviously, the definitions and results of the previous
sections about operators between Banach spaces applied are also valid
for operators between Hilbert spaces.
However, since there is a conjugate-linear isomorphism between
$\displaystyle H^\ast$ and $H$ provided by the Riesz Representation
Theorem, it is traditional to redefine the adjoint of a linear
operator between two Hilbert spaces.  Recall that a linear mapping $L$
between two vector spaces is
{\bfseries Conjugate-linear}\index{Conjugate-Linear mapping}
if $L(\lambda_1v_1 + \lambda_2 v_2) = \overline{\lambda_1}\, L(v_1)
+ \overline{\lambda_2}\, L(v_2)$ for all $\lambda_1, \lambda_2 \in \CC$ and
all vectors $v_1,v_2$.

\begin{defn} \label{fu_an_dualHB_def}
Let $(\DO(A),A)$ be a linear mapping from $H_1$ into $H_2$ such that
$\DO(A)$ is dense in $H_1$.  Let $S \subset H_2$ be the set of all
elements $w$ of $H_2$ for which there exists an element $v \in H_1$
satisfying $\displaystyle \ps{A(x)}{w}_{H_2} = \ps{x}{v}_{H_1}$ for $x
\in \DO(A)$.  Since $\DO(A)$ is dense, $v$ is uniquely defined.  The
{\bfseries (Hilbert space) adjoint}\index{Linear Mapping!Hilbert Space
Adjoint} of $A$ is the mapping $\displaystyle (S,A^\bullet)$ from
$H_2$ to $H_1$ defined by $\displaystyle A^\bullet(w) = v$ for $w\in S$.
We have $\displaystyle \DO(A^\bullet) = S$.
\end{defn}

Consider the conjugate-linear application
\begin{align*}
\Lambda_H : H &\rightarrow H^\ast \\
v & \mapsto  \ps{\cdot}{v}
\end{align*}
As a consequence of Riesz Representation Theorem, this is a one-to-one
and onto mapping.

Consider a linear operator $(\DO(A),A)$ from $H_1$ to $H_2$ such
that $\DO(A)$ is dense in $H_1$.  Let
$\displaystyle (\DO(A^\ast),A^\ast)$ be the
adjoint of $(\DO(A),A)$ in the sense of operators between Banach
spaces and $\displaystyle (\DO(A^\bullet),A^\bullet)$ be the adjoint
of $(\DO(A),A)$ in the sense of operators between Hilbert spaces.  We have
\begin{equation} \label{fu_an_dualHB}
A^\ast = \Lambda_{H_1} \circ A^\bullet \circ \Lambda_{H_2}^{-1}
\end{equation}
on $\displaystyle \DD(A^\ast)$.  To prove this, we note that
\[
w(A(x)) = A^\ast(w)(x) \Leftrightarrow
\ps{A(x)}{\Lambda_{H_2}^{-1}(w)}_{H_2} =
\ps{x}{\Lambda_{H_1}^{-1}(A^\ast(w))}_{H_1}
\]
for all $x \in \DO(A)$ and $\displaystyle w\in \DO(A^\ast)$.  Thus
$\displaystyle \Lambda_{H_2}^{-1}(w) \in \DO(A^\bullet)$
for $\displaystyle w \in \DD(A^\ast)$ and
\[
A^\bullet(\Lambda_{H_2}^{-1}(w)) = \Lambda_{H_1}^{-1}(A^\ast(w))
\]
for $\displaystyle w\in \DO(A^\ast)$ by definition of
$\displaystyle (\DO(A^\bullet),A^\bullet)$.  This proves (\ref{fu_an_dualHB}).

It follows from Definition~\ref{fu_an_dualHB_def} that
the mapping $\displaystyle A \mapsto A^\bullet$ if conjugate-linear
because
\[
A^\bullet(w) = v
\Leftrightarrow \ps{A(x)}{w}_{H_2} = \ps{x}{v}_{H_1}
\Leftrightarrow \ps{\lambda A(x)}{w}_{H_2} = \ps{\lambda x}{v}_{H_1}
= \ps{x}{\overline{\lambda} v}_{H_1}
\Leftrightarrow (\lambda A)^\bullet = \overline{\lambda} v
\]
for all $\lambda \in \CC$.  We have
however that the mapping $\displaystyle A \mapsto A^\ast$ is linear.

From now on, we will follow the tradition in functional analysis and
use $\displaystyle A (\DO(A^\ast),A^\ast)$ to also designate
$\displaystyle A (\DO(A^\bullet),A^\bullet)$.  The context will determine if we are
talking of the dual in the sense of operators between Banach spaces or
between Hilbert spaces.

\begin{theorem} \label{fu_an_RIOrth}
Let $(\DO(A),A)$ be a linear operator from $H$ to itself
such that $\DO(A)$ is dense in $H$.  We have
$\displaystyle \IMG(A)^\perp = \KE(A^\ast)$ and
$\displaystyle \IMG(A^\ast)^\perp = \KE(A)$.
Moreover, $\displaystyle \KE(A^\ast)^\perp = \overline{\IMG(A)}$ and
$\displaystyle \KE(A)^\perp = \overline{\IMG(A^\ast)}$.
\end{theorem}

\begin{defn}
Let $(\DO(A),A)$ be a linear operator from $H$ into itself
such that $\DO(A)$ is dense in $H$.  We say that $(\DO(A),A)$ is
{\bfseries symmetric}\index{Linear Mapping!Symmetric}
if $\displaystyle (\DO(A^\ast),A^\ast)$ is an extension of $(\DO(A),A)$.  Thus,
$\displaystyle \ps{A(x)}{y} = \ps{x}{A(y)}$ for $x,y \in \DO(A)$.

We say that $(\DO(A),A)$ is
{\bfseries self-adjoint}\index{Linear Mapping!Self Adjoint} if
$\displaystyle (\DO(A^\ast),A^\ast) = (\DO(A),A)$; namely, if
$\displaystyle \DO(A)=\DO(A^\ast)$ and
$\displaystyle A(x)=A^\ast(x)$ for all $x \in \DO(A)$.
\end{defn}

\begin{theorem}
Let $(\DO(A),A)$ be a symmetric linear operator from $H$
to itself such that $\DO(A)$ is dense in $H$.  We have
\begin{enumerate}
\item $\ps{A(x)}{x} \in \RR$ for all $x \in \DD(A)$.
\item The eigenvalues of $A$ are real.
\item Eigenvectors associated to distinct eigenvalues are orthogonal.
\item The continuous spectrum of $A$ is a subset of $\RR$.
\end{enumerate}
If $(\DO(A),A)$ is also self-adjoint, we have
\begin{enumerate}
\item The spectrum of $A$ is a subset of $\RR$.
\item The residual spectrum of $A$ is empty.
\end{enumerate}
\end{theorem}

Consider a self-adjoint linear operator $A \in \KK(H)$.  According to
Theorem~\ref{fu_an_comp_oper}, every $\lambda \in \sigma(A)$ with
$\lambda \neq 0$ is an eigenvalue with
$\dim \KE(A-\lambda\Id) < \infty$, and these
eigenvalues may be arranged in a monotone decreasing sequence
$\displaystyle |\lambda_1| \geq |\lambda_2| \geq |\lambda_2| \geq \ldots$
\ Moreover, if $\sigma(A)$ is infinite, then
$\lambda_n \rightarrow 0$ as $n\rightarrow \infty$.

If we choose for each eigenspace $\KE(A-\lambda\Id)$, where
$\lambda \in \sigma(A)$ with $\lambda \neq 0$, an orthonormal
basis and do the union of these basis, we get an orthonormal set
$\displaystyle \{ \phi_j \}_{j=1}^\infty$ such that $\phi_j \neq 0$ for only a finite
number of $j$ when $\sigma(A)$ is finite.
For each eigenvector $\phi_j$, we denote the eigenvalue associate to
it by $\tilde{\lambda}_j$.  For instance,
for each $\phi_j \in \KE(A- \lambda_n \Id)$, we will have that 
$\tilde{\lambda}_j = \lambda_n$.

\begin{theorem}[Hilbert-Schmidt Theorem] \label{fu_an_HStheorem}
Consider a self-adjoint linear operator $A \in \KK(H)$.
\index{Hilbert-Schmidt Theorem}
Let $\displaystyle S = \left\{ \phi_1, \phi_2, \phi_3, \ldots \right\}$ be
an orthonormal set of eigenvectors as defined in the previous
paragraph.
\begin{enumerate}
\item $S$ is an orthonormal basis for $H$.
\item $\displaystyle Au =
\sum_{j=1}^{\infty} \tilde{\lambda}_j \ps{u}{\phi_j}\phi_j$ for $u \in H$.
\item Given $f\in H$, let
$\displaystyle f_0 = f - \sum_{j=1}^{\infty} \ps{f}{\phi_j}\phi_j$.
We have that $f_0 \in \KE(A)$.

If $\lambda$ is not an eigenvalue, there is a unique solution
to $\displaystyle Au - \lambda u = f$ given by
\[
u = \sum_{j=1}^{\infty} \frac{1}{\tilde{\lambda}_j-\lambda}
\ps{f}{\phi_j} \phi_j - \frac{1}{\lambda} f_0 \ .
\]
If $\lambda$ is an eigenvalue, there is a solution to
$\displaystyle Au - \lambda u = f$ if and only if
$\displaystyle \ps{f}{\phi_j} = 0$ for $\tilde{\lambda}_j = \lambda$.
If this is satisfied, there are infinitely many solutions given by
\[
u = \sum_{\substack{j=1\\ \tilde{\lambda}_j\neq \lambda}}^\infty
\frac{1}{\tilde{\lambda}_j-\lambda} \ps{f}{\phi_j} \phi_j -
\frac{1}{\lambda} f_0
+ \sum_{\substack{j=1\\ \tilde{\lambda}_j  = \lambda}}^\infty c_j \phi_j \ ,
\]
where the $c_j$'s are arbitrary constants.
\end{enumerate}
\end{theorem}

\begin{theorem}
If $A \in \KK(H)$, there exists a sequence
$\displaystyle \{A_n\}_{n-1}^\infty$ of
operators $A_n \in \LL(H)$, each of finite rank, converging to $A$ in
the induced operator norm; namely,
$\displaystyle \lim_{n\rightarrow \infty} \|A_n-A\| = 0$.
\end{theorem}

We complete this section with an important class of compact operators.

\begin{theorem} \label{fu_an_HSKern}
Let $\Omega$ be a measurable space and
$\displaystyle K \in L^2(\Omega\times \Omega)$.  Then
$\displaystyle T_K:L^2(\Omega)\rightarrow L^2(\Omega)$ defined by
\[
(T_K f)(x) = \int_{\Omega} K(x,y) f(y) \dx{y} \quad , \quad for x \in \Omega
\ \text{and} \ f \in L^2(\Omega) \ ,
\]
is a compact operator and $\|T_K\| \leq \|K\|_2$.
\end{theorem}

The function $\displaystyle K \in L^2(\Omega\times \Omega)$ in the
previous theorem is called a
{\bfseries Hilbert-Schmidt kernel}\index{Hilbert-Schmidt Kernel}. 

\begin{proof}
\stage{i} We first prove that $T_K$ is effectively a bounded operator
from $\displaystyle L^2(\Omega)$ into itself.

Since $K:\Omega \time \Omega \to \RR$ is measurable, we have that
$\VEC{y} \mapsto K(\VEC{x},\VEC{y})$ is also measurable for almost all
$\VEC{x}$.  Given any $\displaystyle f\in L^2(\Omega)$, we have that
$\VEC{y} \mapsto K(\VEC{x},\VEC{y})f(\VEC{y})$ is
measurable for almost all $\VEC{x}$.

From Fubini's Theorem, we have that
$\displaystyle \int_{\Omega} |K(\VEC{x},\VEC{y})|^2\dx{\VEC{y}}$
exists almost everywhere in $\VEC{x}$ because
$\displaystyle K \in L^2(\Omega\times \Omega)$.  Hence, given any
$\displaystyle f \in L^2(\Omega)$, we have from Schwarz inequality in
$\displaystyle L^2(\Omega)$ that
\[
\int_{\Omega} |K(\VEC{x},\VEC{y})|\,|f(\VEC{y})| \dx{\VEC{y}}
\leq \left(\int_{\Omega} |K(\VEC{x},\VEC{y})|^2\dx{\VEC{y}}\right)^{1/2}
\left(\int_{\Omega} |f(\VEC{y})|^2\dx{\VEC{y}}\right)^{1/2} < \infty
\]
almost everywhere.  Hence,
$\VEC{y} \mapsto K(\VEC{x},\VEC{y})f(\VEC{y})$ is in
$\displaystyle L^1(\Omega)$
for almost all $\VEC{x}$.  So, $T_K f:\Omega \to \RR$ is well defined
almost everywhere.  Moreover, from Fubini's Theory,
$T_K f:\Omega \to \RR$ is measurable. 

From
\[
\left| (T_K f)(\VEC{x}) \right|
\leq \int_{\Omega} |K(\VEC{x},\VEC{y})|\,|f(\VEC{y})| \dx{\VEC{y}}
\leq \left(\int_{\Omega} |K(\VEC{x},\VEC{y})|^2\dx{\VEC{y}}\right)^{1/2}
\left(\int_{\Omega} |f(\VEC{y})|^2\dx{\VEC{y}}\right)^{1/2}
\]
for almost all $x$, we get
\[
\| T_K f \|_2^2 =
\int_{\Omega} \left| (T_K f)(\VEC{x}) \right|^2 \dx{\VEC{x}}
\leq \int_{\Omega}\int_{\Omega} |K(\VEC{x},\VEC{y})|^2\dx{\VEC{y}}\dx{\VEC{x}}
\ \int_{\Omega} |f(\VEC{y})|^2\dx{\VEC{y}}
= \| K \|_2^2 \|f\|_2^2 < \infty
\]
for all $\displaystyle f\in L^2(\Omega)$.  Thus
$\displaystyle T_K f \in L^2(\Omega)$ for all
$\displaystyle f\in L^2(\Omega)$ and
\begin{equation} \label{faHilbSchA}
\|T_K \| \leq \|K\|_2
= \left( \int_{\Omega}\int_{\Omega}
|K(\VEC{x},\VEC{y})|^2\dx{\VEC{y}}\dx{\VEC{x}} \right)^{1/2} \ .
\end{equation}

\stage{ii} We now prove that $T_K$ is compact.

Let $\displaystyle \{\phi_j\}_{j=0}^\infty$ be an orthonormal basis of
$\displaystyle L^2(\Omega)$.  We consider the functions
$\psi_{j,k}:\Omega \times \Omega \to \RR$ defined by
$\displaystyle \psi_{j,k}(\VEC{x},\VEC{y}) = \phi_j(\VEC{x})\phi_k(\VEC{y})$ for
$(\VEC{x},\VEC{y}) \in \Omega\times \Omega$.  The functions
$\psi_{j,k}$ are measurable.  Using Fubini's Theorem and
Theorem~\ref{fu_an_complset}, one can show that 
$\displaystyle \{\psi_{j,k}\}_{(j,k)\in \NN\times \NN}$ is an
orthonormal basis of $\displaystyle L^2(\Omega \times \Omega)$.

We can write $K$ as
$\displaystyle K = \sum_{(j,k)\in \NN\times \NN} a_{j,k} \psi_{j,k}$
for some $a_{j,k} \in \RR$.  The convergence of the previous series is
in $\displaystyle L^2(\Omega\times \Omega)$.  In particular,
$\displaystyle \|K\|_2^2 = \sum_{(j,k)\in \NN\times \NN} a_{j,k}^2$
since $\displaystyle \{\psi_{j,k}\}_{(j,k)\in \NN\times \NN}$ is an
orthonormal basis.  The series converges independently of the order
of the summation.  In particular, for every $\epsilon >0$, there
exists $M>0$ such that $\displaystyle \sum_{i+j > m} a_{j,k}^2 < \epsilon$
for $m\geq M$.  Thus $\displaystyle \sum_{i+j > N} a_{j,k}^2 \to 0$ as
$N \to \infty$.

Let $\displaystyle
K_N(\VEC{x}, \VEC{y}) = \sum_{i+j\leq N} a_{j,k} \psi_{j,k}(\VEC{x},\VEC{y})$
for $N \geq 0$.  Since the image of $T_{K_N}$ is included in the
span of $\displaystyle \{\phi_j\}_{j=0}^N$, we have that
$\dim \IMG(T_{K_N}) \leq N +1 < \infty$.  Thus
$\displaystyle T_{K_N}$ is a compact
operator according to Theorem~\ref{fa_finit_compact}.  We can prove, as
we proved (\ref{faHilbSchA}), that 
\[
\|T_K - T_{K_N} \| \leq \| K - K_N\|_2 = \sum_{j+k>N} a_{j,k}^2 \ .
\]
Thus $T_{K_N}$ converges to $T_K$ as $N\to \infty$.  Since the space
of compact operators is closed with respect to the induce operator
norm according to Theorem~\ref{fu_an_Kclose},
we have that $T_K$ is compact.
\end{proof}

\section{Lax-Milgram Theorem}

\begin{defn}
Let $H$ be an Hilbert space over $\CC$.  A
{\bfseries bounded (or continuous) sequilinear form}%
\index{Sequilinear Form!Bounded}\index{Sequilinear Form!Continuous}
on $H$ is a mapping $B:H\times H \rightarrow \CC$ such that
\begin{enumerate}
\item For each $u\in H$, the mapping $v \mapsto B(u,v)$ is conjugate-linear.
\item For each $v\in H$, the mapping $u \mapsto B(u,v)$ is linear.
\item There exists a constant $C$ such that
$\displaystyle \left| B(u,v) \right| \leq C \| u\| \, \| v\|$
for all $\displaystyle (u,v) \in H \times H$, where $\|\cdot\|$ is the
norm induced by the scalar product $\ps{\cdot}{\cdot}$ on $H$.
\end{enumerate}

The bounded sequilinear form $B$ is
{\bfseries symmetric}\index{Sequilinear Form!Symmetric} if
$\displaystyle B(u,v) = \overline{B(v,u)}$ for all $u,v \in H$.

The bounded sequilinear form $B$ is
{\bfseries coercive}\index{Sequilinear Form!Coercive} if there
exists a positive constant $c$ such that
$\displaystyle \RE B(u,u) \geq c \|u||^2$ for all $u \in H$.  If
$B:H\times H \rightarrow \RR$, then the condition is simply
$\displaystyle B(u,u) \geq c \|u||^2$.
\end{defn}

\begin{prop} \label{fu_an_pre_lax}
Let $H$ be an Hilbert space and $K$ be a non-empty, convex and
closed subset of $H$.  Given $w\in H$, there exists a unique element
$u \in K$ such that
\begin{equation} \label{fu_an_lax3}
\| w-u\| = \min_{v\in K} \|w-v\| \ .
\end{equation}
Moreover, this element $u$ as the property of being the unique element
of $K$ satisfying
\begin{equation} \label{fu_an_lax4}
\RE \ps{w-u}{v-u} \leq 0
\end{equation}
for all $v \in K$.
\end{prop}

\begin{proof}
\stage{i}
Since $w-K$ is a non-empty, convex and closed subset of $H$, it has an
unique element of smallest norm according to
Theorem~\ref{fu_an_ccnes}.  If $\breve{u} \in w-K$ is this element, then
$u= w - \breve{u}$.

\stage{ii}
Suppose that $u$ satisfies (\ref{fu_an_lax3}) and take $v \in K$.  Let
$\displaystyle v_t = (1-t)u+t v$ for $0\leq t \leq 1$.  we have that
$v_t \in K$ for all $t \in [0,1]$ because $K$ is convex.  Thus
\begin{align*}
\| w-u\|^2 &\leq \| w - v_t\|^2 = \|(w-u) -t(v-u)\|^2 \\
&= \|w-u\|^2 - 2t \RE \ps{w-u}{v-u} + t^2 \|u-v\|^2 \ .
\end{align*}
We get that
\[
2\RE \ps{w-u}{v-u} \leq t \|u-v\|^2
\]
for $0<t \leq 1$.  If $t\rightarrow 0$, we get (\ref{fu_an_lax4}).

Conversely, if $u\in K$ satisfies (\ref{fu_an_lax4}), then
\begin{align*}
&\|w-u\|^2 - \|w-v\|^2
= \|w-u\|^2 - \|(w-u) + (u-v)\|^2 \\
&\qquad = -2 \RE \ps{w-u}{u-v}  - \|u-v\|^2
= 2 \RE \ps{w-u}{v-u} - \|u-v\|^2\leq 0
\end{align*}
for $v\in K$.  Thus (\ref{fu_an_lax3}) is satisfied.

The uniqueness of $u$ satisfying (\ref{fu_an_lax4}) comes from the
uniqueness of $u$ satisfying (\ref{fu_an_lax3}).  If there are two
distinct elements $u_1$ and $u_2$ of $K$ satisfying
(\ref{fu_an_lax4}), they must also satisfies (\ref{fu_an_lax3}) since
(\ref{fu_an_lax3}) and (\ref{fu_an_lax4}) are equivalent.  This is a
contradiction of the uniqueness of the unique element of minimal norm
in $w - K$.
\end{proof}

\begin{rmk}
The uniqueness of the element $u$ satisfying (\ref{fu_an_lax4}) could
also be proved directly.  Suppose that there exist $u_1, u_2 \in K$
such that
$\displaystyle \RE \ps{w-u_1}{v-u_1} \leq 0$
and $\displaystyle \RE \ps{w-u_2}{v-u_2} \leq 0$ for all $v \in K$.
If we substitute $v=u_2$ in the first inequality, $v=u_1$ in the
second inequality and add the resulting inequalities, we get
$\displaystyle \ps{u_1-u_2}{u_1-u_2} \leq 0$.  Thus $u_1=u_2$.
\end{rmk}

\begin{theorem}[Stampacchia] \label{fu_an_stamp}
Let $B:H\times H\rightarrow \RR$ be a coercive bounded sequilinear form
on an Hilbert space $H$.  \index{Stampacchia's Theorem}
Moreover, let $K$ be a non-empty, convex and
closed subset of $H$.  Then, for any $\displaystyle \tilde{u} \in H^\ast$, there
exists a unique $u\in K$ such that
\begin{equation} \label{fu_an_lax1}
\RE B(v-u,u) \geq \RE \tilde{u}(v-u)
\end{equation}
for all $v \in K$.  Moreover, if the sequilinear form $B$ is also
symmetric, then $u$ is the unique element of $K$ such that
\begin{equation} \label{fu_an_lax8}
\frac{1}{2}B(u,u) - \RE\tilde{u}(u) = \min_{v\in K} \left\{
\frac{1}{2}B(v,v) - \RE\tilde{u}(v) \right\} \ .
\end{equation}
\end{theorem}

\begin{proof}
\stage{i} Since $v \mapsto B(v,u)$ is a bounded linear function on
$H$ for each $u\in H$, we may use the Riesz Representation Theorem
to define a unique mapping $A:H \rightarrow H$ such that
$\displaystyle B(v,u) = \ps{v}{Au}$ for all $v \in H$.
$Au$ is the unique element of $H$ such that the previous relation is
satisfied.

We now prove that $A$ is a bounded, one-to-one linear mapping.
\begin{enumerate}
\item By definition of $A$, we have for $a_1, a_2 \in \CC$ and
$u_1, u_2 \in H$ that
\begin{align*}
\ps{v}{A\left( a_1 u_1 + a_2 u_2 \right)}
&= B(v,a_1 u_1 + a_2 u_2) = \overline{a_1} B(v,u_1) + \overline{a_2} B(v,u_2) \\
&= \overline{a_1} \ps{v}{A u_1} + \overline{a_2} \ps{v}{A u_2}
= \ps{v}{a_1 A u_1 + a_2 A u_2}
\end{align*}
for all $v \in H$.  Thus
$A\left( a_1 u_1 + a_2 u_2 \right) = a_1 A u_1 + a_2 A u_2$ by
uniqueness of the representation of the elements in $\displaystyle H^\ast$ by
elements in $H$ as given by the Riesz Representation Theorem.  This
show that $A$ is a linear mapping.
\item Since
\[
\|Au\|^2 = \ps{Au}{Au} = B(Au,u) \leq C \|u||\|Au\|
\]
for all $u \in H$, where $C$ is the constant in the definition of a
bounded sequilinear form, we get
\begin{equation} \label{fu_an_lax6}
\|Au\| \leq C \|u||
\end{equation}
for all $u \in H$.  Thus, the induced norm of $A$ satisfies
$\|A\| \leq C$ and, therefore, $A$ is a bounded linear mapping.
\item We have from the coercive property of $B$ that $A$ is
one-to-one.  In fact, we have that
\begin{equation} \label{fu_an_lax7}
\RE \ps{u_1-u_2}{A(u_1-u_2)}
= \RE B(u_1-u_2,u_1-u_2) \geq c \|u_1-u_2\|^2
\end{equation}
for any $u_1, u_2 \in H$, where $c$ is the constant given in the
definition of a coercive bounded sequilinear form.
If $A u_1 = A u_2$, we then get
\[
0 = \RE \ps{u_1-u_2}{A(u_1-u_2)} \geq c \|u_1-u_2\|^2 \ .
\]
Thus $u_1=u_2$.
\end{enumerate}

\stage{ii} According to Riesz Representation Theorem,
Theorem~\ref{fu_an_RieszRT}, there exists a unique $\breve{u} \in H$ such that 
$\displaystyle \tilde{u}(v) = \ps{v}{\breve{u}}$ for all $v \in H$.

Proving that there exists a unique $u\in K$ satisfying 
(\ref{fu_an_lax1}) is equivalent to proving that there exists a unique
$u\in K$ satisfying
\begin{equation} \label{fu_an_lax2}
\RE \ps{v-u}{Au} \geq \RE \ps{v-u}{\breve{u}}
\end{equation}
for all $v \in K$.  Let $p$ be a positive number.  Proving that there
exists a unique $u\in K$ satisfying (\ref{fu_an_lax2}) is equivalent
to proving that there exists a unique $u\in K$ satisfying
\begin{equation} \label{fu_an_lax5}
\RE \ps{v-u}{ \left(p\breve{u}-p Au +u\right)-u} \leq 0
\end{equation}
for all $v \in K$.

Let $P:K \rightarrow K$ be the function defined by
$\displaystyle P(w) = u$ where $u$ satisfies (\ref{fu_an_lax4}).
Because of the uniqueness of $u$, the function $P$ is well defined.
Proving that there exists a unique $u\in K$ satisfying
(\ref{fu_an_lax5}) is thus equivalent to proving that there exists a
unique fixed point $u$ of
\begin{align*}
F: K &\rightarrow K \\
 v & \mapsto P(p\breve{u}-p Av +v)
\end{align*}
To prove this last statement, it is enough to prove that $F$ is a
strict contraction on the closed set $K$.  That is what we now do.  We
will choose the constant $p$ to get a strict contraction.

We first prove that $\|P(w_1) - P(w_2) \| \leq \|w_1 - w_2\|$ for all
$w_1, w_2 \in H$.  We have that $P(w_1)$ and $P(w_2)$ are defined by
$\displaystyle \RE \ps{w_1- P(w_1)}{v- P(w_1)} \leq 0$ and
$\displaystyle \RE \ps{w_2- P(w_2)}{v- P(w_2)} \leq 0$
for all $v \in K$ respectively.  If we substitute $v=P(w_2)$ in the
first inequality, $v=P(w_1)$ in the second inequality and add the
resulting inequalities, we get
\begin{align*}
&\RE \ps{w_1- P(w_1)}{P(w_2)- P(w_1)} + \RE \ps{w_2- P(w_2)}{P(w_1)- P(w_2)} \\
&\qquad \qquad = \RE \ps{-w_1 + w_2 + P(w_1) - P(w_2)}{P(w_1)- P(w_2)} \leq 0 \\
&\qquad \Leftrightarrow
\ps{P(w_1) - P(w_2)}{P(w_1)- P(w_2)} = \RE \ps{P(w_1) - P(w_2)}{P(w_1)- P(w_2)} \\
&\qquad \qquad \leq \RE \ps{w_1- w_2}{P(w_1)- P(w_2)}
\end{align*}
Thus,
\begin{align*}
\|P(w_1) - P(w_2) \|^2 &= \ps{P(w_1)- P(w_2)}{P(w_1)- P(w_2)} \\
& \leq \RE \ps{w_1- w_2}{P(w_1)- P(w_2)}
\leq \|w_1 - w_2 \| \, \|P(w_1) - P(w_2) \| \ ,
\end{align*}
where the last inequality comes from Schwarz inequality.
Hence $\displaystyle \|P(w_1) - P(w_2)\| \leq \| w_1 - w_2 \|$.

It follows that
\begin{align*}
\| F(v_1) - F(v_2) \|^2 &= \| P(p\breve{u}-p Av_1 +v_1) -
P(p\breve{u}-p Av_2 +v_2) \|^2 \\
& \leq \| p (Av_2 - Av_1) - (v_2 - v_1) \|^2 \\
&= p^2 \| Av_2 - Av_1\|^2 -2p \RE\ps{Av_2 - Av_1}{v_2 - v_1}
+ \|v_2 - v_1\|^2 \\
&\leq p^2 C^2 \| v_2-v_1 \|^2 - 2pc \|v_2-v_1\|^2 + \|v_2-v_1\|^2 \\
& = (p^2 C^2 - 2pc +1 ) \|v_2-v_2\|^2
\end{align*}
for all $v_1, v_2 \in K$, where we have used
(\ref{fu_an_lax6}) and (\ref{fu_an_lax7}) for the last
inequality.  If we take $p$ such that
$\displaystyle 0< p^2 C^2 - 2pc +1 <1$, we
get that $F$ is a strict contraction on $K$.  Such a $p$ exists
because
\[
0 > p(pC^2 -2c) \geq -c^2/C^2 \geq -1
\]
for all $p$ is between the two roots, $0$ and $\displaystyle 2c/C^2$, of the
quadratic polynomial $\displaystyle p(pC^2 -2c)$.  The reader should verify
that we must have $0<c\leq C$.

\stage{iii} When $B$ is also a symmetric bounded sequilinear form,
proving that there exists a unique $u\in K$ satisfying
(\ref{fu_an_lax1}) is equivalent to proving that there exists a unique
$u\in K$ satisfying (\ref{fu_an_lax8}).

Let
\[
\ps{v}{u}_B = B(v,u) 
\]
for all $u,v \in H$.  Since $B$ is a coercive and symmetric bounded
sequilinear form on $H$, $\ps{\cdot}{\cdot}_B$ is a scalar product on
$H$ equivalent to the standard scalar product on $H$.  Namely,
$\ps{\cdot}{\cdot}_B$ and $\ps{\cdot}{\cdot}$ define the same topology
on $H$.
We may therefore use the Riesz Representation Theorem to conclude
that there exists a unique $\check{u} \in H$ such that
$\displaystyle \tilde{u}(v) = \ps{v}{\check{u}}_B$ for all $v \in H$.
Hence, (\ref{fu_an_lax1}) becomes
\begin{equation} \label{fu_an_lax10}
\RE \ps{v-u}{\check{u}-u}_B \leq 0
\end{equation}
for all $v \in K$.

According to Proposition~\ref{fu_an_pre_lax}, proving that there exists
a unique $u\in K$ satisfying (\ref{fu_an_lax10}) is equivalent to
proving that there exists a unique $u\in K$ satisfying
\[
\| \check{u}-u\|_B = \min_{v\in K} \|\check{u}-v\|_B \ ,
\]
where $\|v\|_B = \sqrt{B(v,v)}$ for $v \in H$.  This is equivalent to
proving that there exists a unique $v \in K$ where
\[
v \mapsto B(\check{u}-v,\check{u}-v) = B(\check{u},\check{u})
-2 \RE B(v,\check{u}) + B(v,v)
\]
reaches its minimum, which is equivalent to proving that there exists
a unique $v \in K$ where
\[
v \mapsto  -2 \RE B(v,\check{u}) + B(v,v) = B(v,v) - 2 \RE \tilde{u}(v)
\]
reaches its minimum.  This yields (\ref{fu_an_lax8}).
\end{proof}

\begin{cor}[Lax-Milgram Theorem] \label{fu_an_LaxMilgTh}
Let $B:H\times H\rightarrow \CC$ be a bounded, coercive sequilinear form
on an Hilbert space $H$.  \index{Lax-Milgram Theorem}
Then, for every $\displaystyle \tilde{u} \in H^\ast$, there
exists a unique $u\in H$ such that
\begin{equation} \label{fu_an_lax9}
B(v,u) = \tilde{u}(v)
\end{equation}
for all $v \in H$.  Moreover, if the bilinear form $B$ is also
symmetric, then $u$ is the unique element of $H$ such that
\begin{equation} \label{fu_an_lax13}
\frac{1}{2}B(u,u) - \RE \tilde{u}(u) = \min_{v\in H} \left\{
\frac{1}{2}B(v,v) - \RE \tilde{u}(v) \right\} \  .
\end{equation}
\end{cor}

\begin{proof}
This is Stampacchia's theorem with $K=H$.  To prove
(\ref{fu_an_lax9}), we note that (\ref{fu_an_lax1}) now says that
there exists a unique $u\in H$ such that
$\displaystyle \RE B(v-u,u) \geq \RE \tilde{u}(v-u)$ for all $v\in H$.
Since $H-u = H$, the previous sentence is equivalent to saying that
there exists a unique $u\in H$ such that
\begin{equation} \label{fu_an_lax11}
\RE B(v,u) \geq \RE \tilde{u}(v)
\end{equation}
for all $v\in H$.
 
If $v = w$ in (\ref{fu_an_lax11}), we get
\begin{equation}\label{fu_an_lax14}
\RE B(w,u) \geq \RE \tilde{u}(w) \ .
\end{equation}
If $v = -w$ in (\ref{fu_an_lax11}), we get
$\RE B(-w,u) \geq \RE \tilde{u}(-w)$.
Therefore,
\begin{equation}\label{fu_an_lax15}
\RE B(w,u) = - \RE B(-w,u) \leq - \RE \tilde{u}(-w) = \RE \tilde{u}(w) \ .
\end{equation}
(\ref{fu_an_lax14}) and (\ref{fu_an_lax15}) implies that
$\displaystyle \RE B(w,u) = \RE \tilde{u}(w)$ for all $w \in H$.
Thus, $w\mapsto B(w,u)$ and $\tilde{u}$ denote the same linear
functional \footnote{
$\displaystyle \RE B(w,u) = \RE \tilde{u}(w)$ for all $w \in H$
if and only if $\displaystyle \RE B(i w,u) = \RE \tilde{u}(iw)$ for all $w \in H$
if and only if $\displaystyle \RE (i B(w,u)) = \RE (i\tilde{u}(w))$ for all $w \in H$
if and only if $\displaystyle \IM B(w,u) = \IM \tilde{u}(w)$ for all $w \in H$.
}.  Hence, (\ref{fu_an_lax9}) is true.

If $u' \neq u$ satisfies $B(v,u') = \tilde{u}(v)$ for $v \in H$, then 
(\ref{fu_an_lax11}) is satisfied with $u'$ instead of $u$.  This is a
contradiction that $u$ satisfying (\ref{fu_an_lax11}) is unique.

Obviously, (\ref{fu_an_lax13}) is (\ref{fu_an_lax8}) with $K = H$.
\end{proof}

\begin{rmk}
If we are only interested in the existence and uniqueness of $u\in H$
satisfying (\ref{fu_an_lax9}), there is a short proof of this result.
It is almost the first part of the proof of Theorem~\ref{fu_an_stamp}.
\label{fu_an_LaxMilgTh1}

According to Riesz Representation Theorem, there
exists a unique $\breve{u} \in H$ such that
$\displaystyle \tilde{u}(v) = \ps{v}{\breve{u}}$ for all $v \in H$.

Similarly, since $v \mapsto B(v,u)$ is a bounded linear function on
$H$ for each $u\in H$, we may use the Riesz Representation Theorem
to define a unique mapping $A:H \rightarrow H$ such that
$\displaystyle B(v,u) = \ps{v}{Au}$ for all $v \in H$.
$Au$ is the unique element of $H$ such that the previous relation is
satisfied.

To prove the existence and uniqueness of $u\in H$ satisfying
(\ref{fu_an_lax9}), it suffices to prove that $A$ is a linear bijection of
$H$ onto itself.  The requested $u$ in (\ref{fu_an_lax9}) is
$\displaystyle A^{-1}(\breve{u})$.  The proof that $A$ is a bounded, one-to-one
linear mapping is given in the first part of the proof of
Theorem~\ref{fu_an_stamp}.  The coercive nature of $B$ implies that
$A(H)$ is closed as we now explain.  If
$\displaystyle \left\{A u_j\right\}_{j=1}^\infty$ converges to
$w\in H$, we have that
\[
\|A(u_j)\|\,\|u_j\| \geq |\ps{u_j}{A(u_j)}| = |B(u_j,u_j)| \geq c
\|u_j\|^2
\]
for $j \geq 1$.
Thus $\displaystyle \|A(u_j)\| \geq c \|u_j\|$ for all $j$.  Since 
$\displaystyle \left\{A u_j\right\}_{j=1}^\infty$ converges, it is a
Cauchy sequence.  Thus
$\displaystyle \left\{u_j\right\}_{j=1}^\infty$ is a Cauchy sequence 
in the Hilbert space $H$.  Hence, the sequence
$\displaystyle \left\{u_j\right\}_{k=1}^\infty$ converges.  Let $u$ be the
limit of this converging sequence.  Since $A$ is continuous, we have
\[
w = \lim_{j\rightarrow \infty} A u_{j}
= A\left( \lim_{j\rightarrow \infty} u_j\right) = A(u) \  .
\]
Thus $w \in A(H)$.

Finally, the range of $A$ is dense in $H$.  Suppose that
$\displaystyle w \in A(H)^\perp$,
then
\[
c \|w\|^2 \leq |B(w,w)| = |\ps{w}{Aw}| = 0 \  .
\]
Since $R(A)$ is closed and dense in $H$, we have that $A(H)=H$.
\end{rmk}

\begin{rmk}
To prove that the mapping
\begin{align*}
T: H^\ast &\rightarrow H \\
\tilde{u} &\mapsto u
\end{align*}
defined by (\ref{fu_an_lax9}) in Corollary~\ref{fu_an_LaxMilgTh} is a
bounded linear mapping, we show that $T$ is the composition of two
bounded linear mappings.              \label{fu_an_LaxMilgTh2}

Let $\displaystyle R:H^\ast \rightarrow H$ be the bounded linear
mapping defined by $R(\tilde{u})$ is the unique element given by the
Riesz Representation Theorem such that
$\tilde{u}(v) = \ps{v}{R(\tilde{u})}$ for all $v \in H$.
We have $\|R\|=1$ because $\|\tilde{u}\| = \|R(\tilde{u})\|$ for all
$\displaystyle \tilde{u}\in H^\ast$.  Let $A:H\rightarrow H$ be the
bounded mapping defined in Remark~\ref{fu_an_LaxMilgTh1}.  Since $A$
is a linear bijection of $H$ onto $H$, we have by the Open Mapping
Theorem that $\displaystyle A^{-1}$ is continuous.  Finally, we have that
$\displaystyle T=A^{-1}\circ R$.
\end{rmk}

\section{Exercises}

The problems that we suggest cover only $\displaystyle L^2$ spaces and
in particular the classical Fourier series.  For exercises on the
general theory of functional analysis, the reader should consult books like
\cite{Br,ReeSim,Ru,RuFA}.  Some of the books on
partial differential equations listed in the bibliography have
sections on functional analysis with some basic exercises.  In general,
they are not substitute to good books in function analysis.  There is one
exception, \cite{Br} has a nice introduction to the important results
of functional analysis and also include a lot of exercises with
detailed solutions for many of them.  We strongly recommend this book.

Suggested exercises:

\begin{itemize}
\item In \cite{Str}: numbers 1 to 7 in Section 5.1; numbers 1, 2, 9 to
15 in Section 5.2; number 2 in Section 5.3; numbers 1 to 16 in Section 5.4. 
\end{itemize}


%%% Local Variables: 
%%% mode: latex
%%% TeX-master: "notes"
%%% End: 
