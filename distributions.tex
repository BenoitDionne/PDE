\chapter{Distributions and Strong Solutions}
\label{ChapterDistributions}

\section{Introduction}

Consider the linear partial differential equation
\begin{equation} \label{distr_lPDE}
L(\VEC{x}, \diff)u = \sum_{|\VEC{\alpha}|\leq m}
a_{\VEC{\alpha}}(\VEC{x}) \diff^{\VEC{\alpha}} u = f
\end{equation}
where $\displaystyle u \in C^m(\RR^n)$, $\displaystyle f \in C(\RR^n)$
and $\displaystyle a_{\VEC{\alpha}} \in C^{|\VEC{\alpha}|}(\RR^n)$ for all
multi-indices involved in the sum.

If we multiply both sides of the equation in (\ref{distr_lPDE}) by a
function $\displaystyle \phi \in C_c^\infty(\RR^n)$
(functions in $\displaystyle C^\infty(\RR^n)$ with compact supports) and
use integration by parts, we get
\begin{align*}
\int_{\RR^n} f(\VEC{x})\phi(\VEC{x}) \dx{\VEC{x}}
&= \sum_{|\VEC{\alpha}|\leq m} \int_{\RR^n} a_{\VEC{\alpha}}(\VEC{x})
\diff^{\VEC{\alpha}} u(\VEC{x})\phi(\VEC{x}) \dx{\VEC{x}} \\
&= \sum_{|\VEC{\alpha}|\leq m} (-1)^{|\VEC{\alpha}|}\int_{\RR^n}
u(\VEC{x}) \diff^{\VEC{\alpha}}
\left(a_{\VEC{\alpha}}(\VEC{x})\phi(\VEC{x})\right) \dx{\VEC{x}} \ .
\end{align*}

Conversely, if there exists a function $\displaystyle u \in C^m(\RR^n)$
such that
\begin{equation} \label{distr_intE}
\int_{\RR^n} f(\VEC{x})\phi(\VEC{x}) \dx{\VEC{x}}
= \sum_{|\VEC{\alpha}|\leq m} (-1)^{|\VEC{\alpha}|}\int_{\RR^n}
u(\VEC{x}) \diff^{\VEC{\alpha}}
\left(a_{\VEC{\alpha}}(\VEC{x})\phi(\VEC{x})\right) \dx{\VEC{x}}
\end{equation}
for all functions $\displaystyle \phi \in C_c^\infty(\RR^n)$,
then integrations by parts allows us to rewrite this equation as
\[
\int_{\RR^n} f(\VEC{x})\phi(\VEC{x}) \dx{\VEC{x}}
= \sum_{|\VEC{\alpha}|\leq m} \int_{\RR^n} a_{\VEC{\alpha}}(\VEC{x})
\diff^{\VEC{\alpha}} u(\VEC{x})\phi(\VEC{x}) \dx{\VEC{x}}
\]
for all functions $\displaystyle \phi \in C_c^\infty(\RR^n)$.
Since these functions are dense in the space of continuous complex valued
functions over $\displaystyle \RR^n$, (\ref{distr_intE}) implies that
(\ref{distr_lPDE}) is satisfied by $u$ in the classical sense.

The previous discussion suggests the following approach to solve
(\ref{distr_lPDE}).
\begin{enumerate}
\item Find a function $u$ which satisfies (\ref{distr_intE}) for all
function $\displaystyle \phi \in C_c^\infty(\RR^n)$.
\item Show, if possible, that $u$ found in (1) is in
$\displaystyle C^m(\RR^n)$.
\end{enumerate}

Let $\displaystyle L^\ast(\VEC{x}, \diff)$ be the linear partial
differential operator defined by
\[
L^\ast(\VEC{x}, \diff)\phi = \sum_{|\VEC{\alpha}|\leq m} (-1)^{|\VEC{\alpha}|}
\diff^{\VEC{\alpha}} \left( a_{\VEC{\alpha}}(\VEC{x}) \phi \right)
\]
for all $\displaystyle \phi \in C_c^\infty(\RR^n)$.  The
operator $\displaystyle L^\ast(\VEC{x}, \diff)$ is called the
{\bfseries adjoint}\index{Linear Partial Differential Operator!Adjoint} of
$L(\VEC{x}, \diff)$.  Hence, if $u \in C^m(\RR^n)$ is a solution of
(\ref{distr_lPDE}), we have
\[
\int_{\RR^n} \big( L(\VEC{x}, \diff)u(\VEC{x}) \big) \phi(\VEC{x}) \dx{\VEC{x}}
= \int_{\RR^n} u(\VEC{x}) \big( L^\ast(\VEC{x}, \diff)\phi(\VEC{x})
\big) \dx{\VEC{x}}
\]
for all $\displaystyle \phi \in C_c^\infty(\RR^n)$.
This is reminiscing of the relation that we have in linear algebra between an
\nn matrix $A$ and its adjoint $\displaystyle A^\ast$; namely,
$\displaystyle \ps{A\VEC{x}}{\VEC{y}} = \ps{\VEC{x}}{A^\ast\VEC{y}}$
for all $\displaystyle \VEC{x}, \VEC{y} \in \RR^n$.

A {\bfseries classical solution}\index{Classical Solution}
of (\ref{distr_lPDE}) is a function
$\displaystyle u\in C^m(\RR^n)$ that satisfies (\ref{distr_lPDE}).
A {\bfseries strong solution}\index{Strong Solution} or a
{\bfseries solution in the sense of distributions}\index{Solution in
the Sense of Distribution} of (\ref{distr_lPDE}) is a measurable
function $\displaystyle u:\RR^n \to \CC$ that satisfies (\ref{distr_intE}).
Sometime, only strong solutions of (\ref{distr_lPDE}) can be found.

We should warn the reader that our use of strong solution is not
universal.  What we called strong solution is often called
``weak solution'' in the literature.   However, we have chosen to
reserve the expression ``weak solution'' for the solution of the
``variational problem'' associated to a partial differential equation
as we will see in Chapter~\ref{elliptic_PDEs}.  The concept of
Strong solution is more general than the concept of ``weak solution''
though the are often the same.

The simple idea of solving an integral equation like
(\ref{distr_intE}) instead of its associated differential equation
like (\ref{distr_lPDE}) has lead to a lot of research in partial
differential equations during the twentieth century.  The
{\bfseries theory of distributions}\index{Theory of Distributions} that we
briefly present in this chapter and the
{\bfseries Sobolev spaces}\index{Sobolev Space} that we will
study in a future chapter have come out of the research done in
that century.  The modern theory of numerical analysis for partial
differential equations is based on the theory of distributions and the
Sobolev spaces.

Good references for the material presented in this chapter are
\cite{ReeSim,RuFA}.  You will find in these references the proofs of
the results that we state without proofs in the presentation below.

\section{Test Functions} \label{SectTestFnct}

From now on, unless otherwise stated, $\displaystyle \Omega \subset \RR^n$ is an
open set.

The space $\displaystyle C_c^\infty(\Omega)$ of infinitely
differentiable functions with compact support in $\Omega$ is denoted
$\DD(\Omega)$.
Recall that the support of a function $f:\Omega \rightarrow \CC$, denoted
$\supp f$, is the set
\[
\supp f = \overline{\left\{ \VEC{x} \in \Omega : f(\VEC{x}) \neq 0 \right\}} \ ,
\]
where $\overline{A}$ denotes the closure of the set $A\subset \Omega$.
The functions in $\DD(\Omega)$ are called the
{\bfseries test functions}\index{Test Functions} on $\Omega$. 

The convergence (and so the topology) on $\DD(\Omega)$ is defined as
follows.  We say that sequence
$\displaystyle \left\{\phi_j\right\}_{j=0}^\infty$
{\bfseries converges}\index{Test Functions!Convergence}
toward $\phi$ in $\DD(\Omega)$ and write
$\displaystyle \phi_j \rightarrow \phi$ if
\begin{enumerate}
\item there exists a compact set $K$ such that
$\supp \phi_j \subset K$ for all $j$ and $\supp \phi \subset K$.
\item $\displaystyle \lim_{j\rightarrow \infty}
\left\| \diff^{\VEC{\alpha}} \phi_j
- \diff^{\VEC{\alpha}} \phi\right\|_{\infty,K} =0$
for all multi-indices
$\VEC{\alpha} = (\alpha_1,\alpha_2, \ldots, \alpha_n) \in \NN^n$.
\end{enumerate}
Recall that, for all continuous functions $\psi:\Omega \to \CC$ and
compact sets $K\subset \Omega$,
$\displaystyle \left\| \psi \right\|_{\infty,K} =
\max_{\VEC{x}\in K} \left| \psi(\VEC{x}) \right|$.

A complete definition of the locally convex topology on $\DD(\Omega)$
is presented in \cite{RuFA,ReeSim}.  We have only stated the definition of
convergence in $\DD(\Omega)$ that is convenient for the presentation
of distribution theory.

\section{Distributions}

A {\bfseries distribution}\index{Distribution} on $\Omega$ is a
continuous linear functional on $\DD(\Omega)$, where the continuity is
defined as follows.  An application $u:\DD(\Omega) \rightarrow \CC$ is
continuous at $\phi \in \DD(\Omega)$ if
$\displaystyle \lim_{j\rightarrow \infty}|u(\phi_j)-u(\phi)| = 0$
for all sequences $\displaystyle \left\{ \phi_j\right\}_{j=0}^\infty$ in
$\DD(\Omega)$ converging toward $\phi$ in $\DD(\Omega)$.
The space of all distributions on $\Omega$ is denoted $\DD'(\Omega)$.

Let $K$ be a compact subset of $\displaystyle \RR^n$.  The space
$\DD_K$ is defined by\\
$\displaystyle \DD_K = \left\{ u \in \DD(\RR^n) : \supp u \subset K \right\}$.

\begin{prop} \label{distr_cont_cond}
A linear functional $u$ on $\DD(\Omega)$ is continuous if and only if
for every compact set $K \subset \Omega$ there exist a constant $C_K$
and an integer $N_K$ such that
$\displaystyle \left| u(\phi) \right| \leq C_K \max_{|\VEC{\alpha}|\leq N_K}
\left\| \diff^{\VEC{\alpha}} \phi \right\|_{\infty,K}$
for all $\phi \in \DD_K$.
\end{prop}

The proof of this result is included in the proof of Theorem 6.8 in
\cite{RuFA}.

\begin{rmk}
The following notation is often used.  For $u \in \DD'(\Omega)$ and
$\phi \in \DD(\Omega)$, we define $\ps{u}{\phi}$ as
$\ps{u}{\phi} = u(\phi)$.  We will use this notation occasionally.
\end{rmk}

\begin{egg}
Let $u$ be a function in $\displaystyle L^1_{loc}(\RR^n)$.  The
function $u$ defines a distribution on $\displaystyle \RR^n$ by
\[
\ps{u}{\phi} = u(\phi) =
\int_{\RR^n} u(\VEC{x}) \phi(\VEC{x}) \dx{\VEC{x}} \quad ,
\quad \phi \in \DD(\RR^n) \ .
\]
It is traditional to use the same letter to designate the function
and the distribution.  The context determines to which one we refer to.

To verify this statement, we have to show that
$\displaystyle u:\DD(\RR^n)\rightarrow \CC$ is well defined, $u$ is
linear, and $u$ is continuous.

If $\displaystyle \phi \in \DD(\RR^n)$ and $K = \supp \phi$, then 
\begin{align*}
\left|u(\phi)\right|
&= \left| \int_{\RR^n} u(\VEC{x}) \phi(\VEC{x}) \dx{\VEC{x}} \right|
\leq \int_{K} \left|u(\VEC{x})\right|\,
\left| \phi(\VEC{x})\right| \dx{\VEC{x}}
\leq \left\|\phi(\VEC{x})\right\|_{\infty,K}
\int_{K} \left| u(\VEC{x})\right| \dx{\VEC{x}} <\infty
\end{align*}
because
$\displaystyle \left\|\phi(\VEC{x})\right\|_{\infty,K} < \infty$ since
$\phi$ is continuous on the compact set $K$, and
$\displaystyle \int_{K} \left| u(\VEC{x})\right| \dx{\VEC{x}}$ is bounded
since $\displaystyle u\in L^1_{loc}(\RR^n)$.  Thus,
$\displaystyle u:\DD(\RR^n)\rightarrow \CC$ is well defined.

$\displaystyle u:\DD(\RR^n)\rightarrow \CC$ is linear because the integral is a
linear mapping.

To show that $\displaystyle u:\DD(\RR^n)\rightarrow \CC$ is
continuous, let $\phi$ be any element of $\displaystyle \DD(\RR^n)$ and
$\displaystyle \left\{ \phi_j\right\}_{j=0}^\infty$ be any sequence
$\displaystyle \DD(\RR^n)$ converging toward $\phi$ in
$\displaystyle \DD(\RR^n)$.  By definition
of convergence in $\displaystyle \DD(\RR^n)$, there exists a compact set
$\displaystyle K \subset \RR^n$ such that $\supp \phi_j \subset K$ for all $j$,
$\supp \phi \subset K$, and
\[
\lim_{j\rightarrow \infty}
\left\| \diff^{\VEC{\alpha}} \phi_k
- \diff^{\VEC{\alpha}} \phi \right\|_{\infty,K} = 0
\]
for all multi-indices $\displaystyle \VEC{\alpha} \in \NN^n$.
In particular, for $\VEC{\alpha} = \VEC{0}$, we get
\[
\lim_{j\rightarrow \infty} \left\| \phi_k - \phi \right\|_{\infty,K} = 0 \ .
\]
Hence,
\begin{align*}
\left|u(\phi_j) - u(\phi)\right| &=
\left| \int_{\RR^n} u(\VEC{x})
\left(\phi_j(\VEC{x}) - \phi(\VEC{x})\right) \dx{\VEC{x}} \right|
\leq \int_K \left| u(\VEC{x}) \right|\,
\left|\phi_j(\VEC{x}) - \phi(\VEC{x})\right| \dx{\VEC{x}} \\
& \leq \left\| \phi_k - \phi \right\|_{\infty,K} 
\int_K \left| u(\VEC{x}) \right|\ \dx{\VEC{x}} \rightarrow 0
\end{align*}
as $j \to \infty$.  Note that
$\displaystyle \int_K \left| u(\VEC{x}) \right| \dx{\VEC{x}}$ is finite
because $\displaystyle u\in L^1_{loc}(\RR^n)$.
\end{egg}

\begin{egg}
The {\bfseries Dirac delta function}\index{Dirac Delta Function} is
the distribution on $\displaystyle \RR^n$ defined by
\[
\ps{\delta}{\phi} = \delta(\phi) = \phi(\VEC{0}) \quad , \quad
\phi \in \DD(\RR^n) \ .
\]
\end{egg}

\begin{rmk}
The Dirac delta function is often introduced in engineering books
as a function which is zero everywhere but at the origin and such that
$\displaystyle \int_{-\infty}^\infty \delta(x)f(x) \dx{x} = f(0)$ for
all functions $f$.  Obviously, this does not make any sense.  However,
the formal manipulations done using this ``definition'' of the
Dirac delta function can be rigorously proved with the correct
definition of this function given in the previous example.

Since most of the distributions $\displaystyle u \in \DD'(\RR)$ used in
engineering comes from integrable functions, the integral notation
$\displaystyle \int_{-\infty}^\infty u(x) \phi(x)\dx{x}$ is often used
instead of $u(\phi)$ or $\ps{u}{\phi}$ even when the distribution
$u$ is not an integrable function as is the case for the Dirac delta
function.  We will not use these notation in the present text.
\end{rmk}

The next proposition states that we may defined the
{\bfseries product}\index{Distribution!Product} of $u \in \DD'(\Omega)$ and
$\displaystyle f \in C^\infty(\Omega)$ as the distribution $f u$ defined by
\[
(f u)(\phi) = u(f \phi) \quad , \quad \phi \in \DD(\Omega) \ .
\]

\begin{prop} \label{distr_product}
If $\displaystyle f\in C^\infty(\Omega)$ and $u \in \DD'(\Omega)$, then
\[
v(\phi) = u(f \phi) \quad , \quad \phi \in \DD(\Omega) \ ,
\]
defines a distribution on $\Omega$.  We denote this distribution by
$f u$, the product of the function $f$ and the distribution $u$.
\end{prop}

\begin{proof}
\stage{i} $(f u)(\phi)$ is well defined for all $\phi \in \DD(\Omega)$
because $f \phi \in \DD(\Omega)$ for all $\phi\in \DD(\Omega)$.

\stage{ii} It is easy to show that $f u$ is a linear functional.

\stage{iii} To prove that $f u$ is continuous, we show that $f u$
satisfies Proposition~\ref{distr_cont_cond}.

Let $K$ be a compact subset of $\Omega$.  Since $u\in \DD'(\Omega)$,
it follows from Proposition~\ref{distr_cont_cond} that there exist a
constant $C_K$ and an integer $N_K$ such that
$\displaystyle \left| u(\phi) \right| \leq C_K \max_{|\VEC{\alpha}|\leq N_K}
\left\| \diff^{\VEC{\alpha}} \phi \right\|_{\infty,K}$ for all
$\phi \in \DD_K$.  According to Leibniz formula
\[
\diff^{\VEC{\alpha}} (f \phi) = \sum_{\VEC{\beta} \leq \VEC{\alpha}}
c_{\VEC{\alpha},\VEC{\beta}}
\diff^{\VEC{\alpha}-\VEC{\beta}}f \,\diff^{\VEC{\beta}} \phi \ ,
\]
where the coefficients $c_{\VEC{\alpha},\VEC{\beta}}$ are constants.
This is a formula similar to the binomial formula.  Hence, there
exists a constant $C_f$ such that
\[
\| \diff^{\VEC{\alpha}} (f \phi) \|_{\infty} \leq C_f
\max_{\VEC{\beta}\leq\VEC{\alpha}} \|\diff^{\VEC{\beta}} \phi\|_{\infty,K}
\]
for all $\phi \in \DD_K$ and $|\VEC{\alpha}|\leq N_K$.  The constant $C_f$
depends only of
$\displaystyle \left\|\diff^{\VEC{\alpha}-\VEC{\beta}} f\right\|_{\infty,K}$
and $c_{\VEC{\alpha},\VEC{\beta}}$ for $\VEC{\beta} \leq \VEC{\alpha}$.
We finally get that
\[
\left| (f u)(\phi) \right| =
\left| u(f \phi) \right| \leq C_K \max_{|\VEC{\alpha}|\leq N_K}
\left\| \diff^{\VEC{\alpha}} (f \phi) \right\|_{\infty,K}
\leq C_K C_f \max_{|\VEC{\alpha}|\leq N_K}
\left\| \diff^{\VEC{\alpha}} \phi \right\|_{\infty,K}
\]
for all $\phi \in \DD_K$.
\end{proof}

The {\bfseries support}\index{Distribution!Support} of a distribution
$u\in \DD'(\Omega)$ is the complement of the largest open set
$V\subset \Omega$ such that $u(\phi)=0$ for all $\phi \in \DD(\Omega)$ with
$\supp \phi \subset V$.  The space of distributions on $\Omega$
with compact support is denoted $\EE'(\Omega)$; in other words,
$\displaystyle \EE'(\Omega) = \left\{ u \in \DD'(\Omega) : \supp u
\text{ is compact} \right\}$.

The space of distributions is equipped with the following (weak) topology.
We say that the sequence
$\displaystyle \left\{ u_j\right\}_{j=0}^\infty$ of distribution on
$\Omega$ {\bfseries converge (weakly)}\index{Distribution!Convergence} to
$u\in \DD'(\Omega)$ if
$\displaystyle \lim_{j\rightarrow \infty} \left|u_j(\phi) - u(\phi) \right| = 0$
for all $\phi \in \DD(\Omega)$.  The space of distribution on $\Omega$
is ``kind of complete'' according to this topology.

\begin{prop}
Let $\displaystyle \left\{ u_j \right\}_{j=0}^\infty$ be a sequence
in $\DD'(\Omega)$ such that the sequence
$\displaystyle \left\{ u_j(\phi) \right\}_{j=0}^\infty$ of complex numbers
converges for all $\phi \in \DD(\Omega)$.  Let
$u : \DD(\Omega) \to \CC$ be the function defined
by $\displaystyle u(\phi) = \lim_{j\to \infty} u_j(\phi)$ for all
$\phi \in \DD(\Omega)$.  Then $u\in \DD'(\Omega)$ and the sequence 
$\displaystyle \left\{ u_j \right\}_{j=0}^\infty$ converges to $u$ in
$\DD'(\Omega)$.
\end{prop}

The proof of this result is part of the proof of Theorem 6.17 in
\cite{RuFA}.

\begin{egg}
Consider the function        \label{chia_delta}
\[
\chi_a(x) =
\begin{cases}
1/(2a) & \quad \text{if} \ |x|<a \\
0 & \quad \text{if} \ |x| >a
\end{cases}
\]
It is often stated in engineering books that the Dirac delta function
$\delta$ is the limit of $\chi_a$ as $a \to 0$.

We can rigorously prove this statement in the context of our theory.
We have that $\chi_a$ defined a distribution on $\RR$ by
\[
  \chi_a(\phi) = \int_{\RR} \chi_a(x) \phi(x) \dx{x}
\]
for all $\phi \in \DD(\RR)$.  We now prove that this distribution
converges to $\delta$ as $ a \to 0$.

Given $\phi \in \DD(\RR)$, let
$\displaystyle \Phi(x) = \int_0^x \phi(x) \dx{x}$.
We have that
\[
\lim_{a\to 0} \chi_a(\phi)
= \lim_{a\to 0}\frac{1}{2a} \int_{-a}^a \phi(x) \dx{x} 
= \lim_{a\to 0}\frac{1}{2a} \left( \Phi(a) - \Phi(-a)\right)
= \Phi'(0) = \phi(0) = \delta(\phi) \ .
\]
Since this is true for all $\phi \in \DD(\RR)$, we get that
$\chi_a \to \delta$ in $\DD'(\RR)$.
\end{egg}

\section{Derivatives of Distributions}

\begin{prop} \label{distr_der_def}
If $u \in \DD'(\Omega)$ and $\VEC{\alpha}$ is a multi-index, then
\[
v(\phi) = (-1)^{|\VEC{\alpha}|}u(\diff^{\VEC{\alpha}} \phi) \quad ,
\quad \phi \in \DD(\Omega) \ ,
\]
defines a distribution on $\Omega$.  We denote this distribution by
$\displaystyle \diff^{\VEC{\alpha}} u$, the
{\bfseries partial derivative}\index{Distribution!Partial Derivative}
of $u\in \DD'(\Omega)$.
\end{prop}

\begin{proof}
\stage{i} $\displaystyle (\diff^{\VEC{\alpha}} u)(\phi)$ is well defined for all
$\phi \in \DD(\Omega)$ because
$\displaystyle \diff^{\VEC{\alpha}} \phi \in \DD(\Omega)$
for all $\phi\in \DD(\Omega)$.

\stage{ii} It is easy to show that $\displaystyle \diff^{\VEC{\alpha}} u$ is a
linear functional.

\stage{iii} To prove that $\displaystyle \diff^{\VEC{\alpha}} u$ is
continuous, we show that $\displaystyle \diff^{\VEC{\alpha}} u$ satisfies
Proposition~\ref{distr_cont_cond}.

Let $K$ be a compact subset of $\Omega$.  Since $u\in \DD'(\Omega)$,
it follows from Proposition~\ref{distr_cont_cond} that there exist a
constant $C_K$ and an integer $N_K$ such that
$\displaystyle \left| u(\phi) \right| \leq C_K \max_{|\VEC{\beta}|\leq N_K}
\left\| \diff^{\VEC{\beta}} \phi \right\|_{\infty,K}$
for all $\phi \in \DD_K$.  Hence
\[
\left| (\diff^{\VEC{\alpha}} u)(\phi) \right| =
  \left| (-1)^{|\VEC{\alpha}|} u(\diff^{\VEC{\alpha}} \phi) \right|
\leq C_K \max_{|\VEC{\beta}|\leq N_K}
\left\| \diff^{\VEC{\beta}+\VEC{\alpha}} \phi \right\|_{\infty,K}
\leq C_K \max_{|\VEC{\beta}|\leq N_K+|\VEC{\alpha}|}
\left\| \diff^{\VEC{\beta}} \phi \right\|_{\infty,K}
\]
for all $\phi \in \DD_K$.
\end{proof}

The distributions are infinitely differentiable.  The following
example justifies this definition of derivative. 

\begin{egg}
Suppose that $\displaystyle u \in C^k(\RR^n)$.  Show that the partial
derivatives $\displaystyle \diff^{\VEC{\alpha}} u$ of the distribution $u$ on
$\displaystyle \RR^n$ with $|\VEC{\alpha}|\leq k$
corresponds to the standard partial derivatives of the function $u$.  More
precisely, show that the derivative of the distribution $u$ on
$\displaystyle \RR^n$ is the distribution given by the standard
derivative of the function $u$.

Note that $\displaystyle u \in C^k(\RR^n)$ implies that standard derivative
$\displaystyle \diff^{\VEC{\alpha}} u$ of $u$ is in $\displaystyle C(\RR^n)$
for all $|\VEC{\alpha}|\leq k$ and
so it is in $\displaystyle L_{loc}^1(\RR^n)$ for all $|\VEC{\alpha}|\leq k$.
Thus, the standard derivative $\displaystyle \diff^{\VEC{\alpha}} u$
can define a distribution for all $|\VEC{\alpha}|\leq k$.

Let $\VEC{\alpha} = (1,0,\ldots,0)$, the reasoning for other first order
partial derivatives is similar. There are two ways to look at
$\displaystyle \diff^{\VEC{\alpha}} u$.

\stage{i} The function $u$ can be used to define a
distribution on $\displaystyle \RR^n$ because
$\displaystyle u\in L_{loc}^1(\RR^n)$.  Hence, the 
derivative of $u$ in the sense of distributions is
\[
\ps{\diff^{\VEC{\alpha}} u}{\phi} = -\ps{u}{\diff^{\VEC{\alpha}} \phi} =
-\int_{\RR^n} u(\VEC{x}) \pdydx{\phi}{x_1}(\VEC{x}) \dx{\VEC{x}}
\quad , \quad \phi \in \DD(\Omega) \ .
\]

\stage{ii} The function $\displaystyle \diff^{\VEC{\alpha}} u$ can be
used to define a distribution on $\displaystyle \RR^n$ because
$\displaystyle \diff^{\VEC{\alpha}} u\in L_{loc}^1(\RR^n)$.
Hence, by definition,
\[
\ps{\diff^{\VEC{\alpha}} u}{\phi} =
\int_{\RR^n} (\diff_{x_1}u)(\VEC{x}) \phi(\VEC{x}) \dx{\VEC{x}}
\quad , \quad \phi \in \DD(\Omega) \ .
\]

However, for all $\displaystyle \phi \in \DD(\RR^n)$, we have
\begin{align*}
&-\int_{\RR^n} u(\VEC{x}) \pdydx{\phi}{x_1}(\VEC{x}) \dx{\VEC{x}}
= -\int_{\RR^{n-1}} \left( \int_{-\infty}^\infty u(x_1,\VEC{y})
\pdydx{\phi}{x_1}(x_1,\VEC{y})\dx{x_1}\right) \dx{\VEC{y}} \\
&\qquad = -\int_{\RR^{n-1}} \left(
\lim_{x_1\rightarrow +\infty}u(x_1,\VEC{y})\phi(x_1,\VEC{y})
- \lim_{x_1\rightarrow -\infty}u(x_1,\VEC{y})\phi(x_1,\VEC{y}) \right. \\
&\qquad \qquad \left. - \int_{-\infty}^\infty
\diff^{\VEC{\alpha}} u(x_1,\VEC{y})\phi(x_1,\VEC{y})\dx{x_1} \right)
\dx{\VEC{y}}\\
&\qquad = \int_{\RR^{n-1}} \left(\int_{-\infty}^\infty
\diff^{\VEC{\alpha}} u(x_1,\VEC{y})\phi(x_1,\VEC{y})\dx{x_1} \right)
\dx{\VEC{y}}
= \int_{\RR^n} \diff^{\VEC{\alpha}} u(\VEC{x})\phi(\VEC{x}) \dx{\VEC{x}} \ ,
\end{align*}
where we have used integration by parts for the differentiable functions
$u$ and $\phi$, and
\[
\lim_{x_1\rightarrow +\infty}u(x_1,\VEC{y})\phi(x_1,\VEC{y})
= \lim_{x_1\rightarrow -\infty}u(x_1,\VEC{y})\phi(x_1,\VEC{y}) = 0
\]
because $\phi$ has a compact support, so the support of $\phi$ is bounded.

Thus, the action of the derivative of the distribution $u$ given in
(i) is equal to the action of the distribution
$\displaystyle \diff^{\VEC{\alpha}} u$
given in (ii) on $\displaystyle \DD(\RR^n)$.

By Induction, one can show that this is also true for higher order
partial derivatives.
\end{egg}

\begin{egg}
Let $\displaystyle k(x_1,x_2) = \frac{1}{2\pi(x_1 + i x_2)}$ and
$L = \diff_{x_1} + i \diff_{x_2}$.

Given $\displaystyle \phi \in \DD(\RR^2)$, we have that
\begin{align*}
\ps{L k}{\phi} &= - \ps{k}{L\phi}
= \frac{-1}{2\pi} \int_{\RR^2} \left( \pdydx{\phi}{x_1} + i
\pdydx{\phi}{x_2} \right) \frac{1}{x_1 + i x_2} \dx{\VEC{x}} \\
&= \lim_{r\to 0} \frac{-1}{2\pi} \int_{\|\VEC{x}\|_2\geq r}
\left( \pdydx{\phi}{x_1} + i \pdydx{\phi}{x_2} \right)
\frac{1}{x_1 + i x_2} \dx{\VEC{x}}  \ ,
\end{align*}
where the limit is justified by the Lebesgue Dominate Convergence
Theorem.  Using the Green's Theorem on the disk defined by
$\displaystyle \|\VEC{x}\|_2 \leq r$, we get
\begin{align*}
\int_{\|\VEC{x}\|_2\geq r}
\left( \pdydx{\phi}{x_1} + i \pdydx{\phi}{x_2} \right) \frac{1}{x_1 + i x_2}
\dx{\VEC{x}}
&= -i \int_{\|\VEC{x}\|_2\geq r}
\left( \pdfdx{\left(\frac{i \phi}{x_1 + i x_2}\right)}{x_1}
- \pdfdx{\left(\frac{\phi}{x_1 + i x_2}\right)}{x_2}  \right) \dx{\VEC{x}} \\
&= -i \int_{\|\VEC{x}\|_2 = r}
\frac{\phi}{x_1 + i x_2} \dx{x_1} + \frac{i\phi}{x_1 + i x_2} \dx{x_2}
\end{align*}
where the orientation of the circle $\|\VEC{x}\|_2 = r$
is clockwise.  If we use the parametric representation
$(x_1,x_2) = (r\cos(\theta),r\sin(\theta)$ for
$0 \leq \theta \leq 2\pi$, a parametric representation associated to
the counterclockwise orientation of the circle, we get
\begin{align*}
&\int_{\|\VEC{x}\|_2\geq r}
\left( \pdydx{\phi}{x_1} + i \pdydx{\phi}{x_2} \right)
\frac{1}{x_1 + i x_2} \dx{\VEC{x}} 
= i \int_0^{2\pi} \frac{\phi(r\cos(\theta),r\sin(\theta))}{r e^{i \theta}}
\left( -r \sin(\theta) + i r \cos(\theta) \right) \dx{\theta} \\
&\qquad \qquad
= - \int_0^{2\pi} \phi(r\cos(\theta),r\sin(\theta)) \dx{\theta} \ .
\end{align*}
Thus,
\[
\ps{L k}{\phi} = \lim_{r\to 0} \frac{1}{2\pi}
\int_0^{2\pi} \phi(r\cos(\theta),r\sin(\theta)) \dx{\theta}
= \frac{1}{2\pi} \int_0^{2\pi} \phi(\VEC{0}) \dx{\theta}
= \phi(\VEC{0}) = \delta(\phi)
\]
for all $\displaystyle \phi \in \DD(\RR^2)$.  This means that
$L k = \delta$.
\end{egg}

\begin{lemma} % If the label is defined here before the first sentence
% of the definition, TeX does not register it.  Why?
Let $]a,b[$ be an open interval of $\RR$.  If     \label{distr_der0_eq_const}
$\displaystyle f \in L^1_{loc}(]a,b[)$ satisfies
\begin{equation} \label{distr_fp0_const}
\ps{f}{\phi'} = \int_a^b f(x) \phi'(x) \dx{x} = 0
\end{equation}
for all $\phi \in \DD(]a,b[)$, then there exists a constant $C$ such
that $f=C$ almost everywhere on $]a,b[$.
\end{lemma}

\begin{proof}
Choose $\psi \in \DD(]a,b[)$ such that
$\displaystyle \int_a^b \psi(x)\dx{x}=1$.  For each
$\phi \in \DD(]a,b[)$, let
\[
\theta(x) = \int_a^x \left( \phi(y) - \psi(y) \int_a^b
\phi(z)\dx{z}\right) \dx{y} \ .
\]
We prove that $\theta \in \DD(]a,b[)$. Let $K = \supp \psi \cup \supp \phi$
and choose $m, M \in ]a,b[$ such $m < x < M$ for all $x \in K$.
This is possible because $\phi$ and $\psi$ have compact supports in
the open set $]a,b[$.  We then have that
$\theta(x) = 0$ for $x<m$ because $\phi(x) = \psi(x) = 0$ for $x<m$,
and $\theta(x) = 0$ for $x>M$ because
\begin{align*}
\int_a^x \left( \phi(y) - \psi(y) \int_a^b \phi(z)\dx{z}\right) \dx{y} &=
\int_a^b \left( \phi(y) - \psi(y) \int_a^b \phi(z)\dx{z}\right) \dx{y} \\
&= \int_a^b \phi(y) \dx{y} - \left(\int_a^b \psi(y) \dx{y}\right)
\left( \int_a^b \phi(z)\dx{z}\right) = 0
\end{align*}
for all $x >M$, where we have used $\phi(x)=\psi(x) = 0$ for $x>M$.

It follows from (\ref{distr_fp0_const}) that
\begin{align*}
0 &= \ps{f}{\theta'} = \int_a^b f(x) \left( \phi(x) - \psi(x) \int_a^b
\phi(z)\dx{z}\right) \dx{x} \\
&= \int_a^b f(x) \phi(x) \dx{x} -
\left( \int_a^b f(x) \,\psi(x) \dx{x}\right)
\left(\int_a^b \phi(z)\dx{z}\right) \\
&= \int_a^b f(x) \phi(x) \dx{x} -
\left( \int_a^b f(z) \, \psi(z) \dx{z}\right)
\left(\int_a^b \phi(x)\dx{x}\right) \\
&= \int_a^b \left( f(x) - \left( \int_a^b f(z)\,\psi(z) \dx{z}\right) \right)
\phi(x)\dx{x}
\end{align*}
for all $\phi \in \DD(]a,b[)$.  Thus
\[
f(x) - \left( \int_a^b f(z)\,\psi(z) \dx{z}\right) = 0
\]
almost everywhere in $]a,b[$.
The constant $C$ in the conclusion of the lemma is given by
$\displaystyle C = \int_a^b f(z)\,\psi(z) \dx{z}$.  The readers should
convince themselves that this is independent of $\psi$.
\end{proof}

The previous proposition can be restated as follows.  If
$u \in \DD'(]a,b[)$ is defined by
$\displaystyle u(\phi) = \int_a^b f(x) \phi(x) \dx{x}$ for
$\phi \in \DD(]a,b[)$, where $\displaystyle f \in L_{loc}^1(]a,b[)$, and
$\diff u = 0$ in the sense of distributions, then there exists a constant
$C$ such that $f(x) = C$ almost everywhere on $]a,b[$.  Thus, we may
say that $u$ is a distribution given by the constant function
$f(x) = C$ for all $x \in ]a,b[$.

\begin{prop} \label{distr_cdistr_cder}
Suppose that $u \in C(\Omega)$.  Moreover, suppose that the derivative
in the sense of distributions $\displaystyle \diff^{\VEC{\alpha}} u$ can be
represented by a continuous function $g_{\VEC{\alpha}}:\Omega \to \CC$ for
$|\VEC{\alpha}|\leq k$; namely,
\[
\ps{\diff^{\VEC{\alpha}} u}{\phi} = (-1)^{|\VEC{\alpha}|}
\ps{u}{\diff^{\VEC{\alpha}}\phi}
= \ps{g_{\VEC{\alpha}}}{\phi} = \int_{\Omega} g_{\VEC{\alpha}}(\VEC{x})
\phi(\VEC{x}) \dx{\VEC{x}}
\]
for all $\phi \in \DD(\Omega)$.  Then
$\displaystyle u \in C^k(\Omega)$.
\end{prop}

\begin{proof}
\stage{$\mathbf{k=1}$}
Let $g_{\VEC{\alpha}} \in C(\Omega)$ be the function representing
$\displaystyle \diff^{\VEC{\alpha}} u$ for $\VEC{\alpha} = (1.0.\ldots,0)$, the
partial derivative of the distribution $u$ on $\Omega$ with respect to
$x_1$.  The reasoning is similar for the other multi-indices $\VEC{\alpha}$
such that $|\VEC{\alpha}|=1$.

Since the standard derivative is a local concept, we may assume that
$\displaystyle \Omega = \prod_{j=1}^n ]a_j,b_j[$.

Choose $\VEC{y} \in \Omega$. Let
\[
v(\VEC{x}) = \int_{y_1}^{x_1} g_{\VEC{\alpha}}(s,\tilde{\VEC{x}}) \dx{s} +
u(y_1,\tilde{\VEC{x}})
\]
for $\VEC{x} = (x_1, \tilde{\VEC{x}}) \in \Omega$.
Since $g_{\VEC{\alpha}} \in C(\Omega)$, we have that
\begin{align*}
]a_1,b_1[ & \rightarrow \CC \\
x & \mapsto v(x,\tilde{\VEC{x}})
\end{align*}
is differentiable in the classical sense for all
$\displaystyle \tilde{\VEC{x}} \in \prod_{j=2}^n ]a_j,b_j[$.
More precisely, $\displaystyle \pdydx{v}{x_1}(x,\tilde{\VEC{x}}) =
g_{\VEC{\alpha}}(x,\tilde{\VEC{x}})$ for all $(x, \tilde{\VEC{x}}) \in \Omega$.
The goal is to show that $u=v$ in $\Omega$.

Consider
\begin{equation} \label{distr_der_der}
\int_\Omega \left(u(\VEC{x}) - v(\VEC{x})\right) \diff^{\VEC{\alpha}}
\phi(\VEC{x}) \dx{\VEC{x}}
= \int_\Omega u(\VEC{x}) \diff^{\VEC{\alpha}} \phi(\VEC{x}) \dx{\VEC{x}} -
\int_\Omega v(\VEC{x}) \diff^{\VEC{\alpha}} \phi(\VEC{x}) \dx{\VEC{x}}
\end{equation}
for $\phi \in \DD(\Omega)$.
By definition of the derivative in the sense of distributions, we have
\begin{equation} \label{distr_der_derA}
-\int_\Omega u(\VEC{x}) \diff^{\VEC{\alpha}} \phi(\VEC{x}) \dx{\VEC{x}} =
\int_\Omega g_{\VEC{\alpha}}(\VEC{x}) \phi(\VEC{x}) \dx{\VEC{x}}
\end{equation}
for all $\phi \in \DD(\Omega)$ by hypothesis.
Moreover, using integration by parts, we have
\begin{align}
&\int_\Omega v(\VEC{x}) \diff^{\VEC{\alpha}} \phi(\VEC{x}) \dx{\VEC{x}} =
\int_{\prod_{j=2}^n]a_j,b_j[} \int_{a_1}^{b_1}
v(x_1,\tilde{\VEC{x}}) \pdydx{\phi}{x_1}(x_1,\tilde{\VEC{x}})
\dx{x_1} \dx{\tilde{\VEC{x}}} \nonumber \\
&\quad = \int_{\prod_{j=2}^n]a_j,b_j[} \left( v(x_1, \tilde{\VEC{x}})\,
\phi(x_1,\tilde{\VEC{x}})\bigg|_{x_1=a_1}^{x_1=b_1}
- \int_{a_1}^{b_1} \pdydx{v}{x_1}(x_1,\tilde{\VEC{x}})\,
\phi(x_1,\tilde{\VEC{x}}) \dx{x_1} \right) \dx{\tilde{\VEC{x}}}
\nonumber \\
&\quad = - \int_\Omega \pdydx{v}{x_1}(\VEC{x})\,\phi(\VEC{x}) \dx{\VEC{x}}
= - \int_\Omega g_{\VEC{\alpha}}(\VEC{x})\,\phi(\VEC{x}) \dx{\VEC{x}}
\label{distr_der_derB}
\end{align}
for all $\phi \in \DD(\Omega)$, where we have use the fact that $\phi$
has a compact support in $\Omega$ to obtain
\[
v(x_1,\tilde{\VEC{x}})\, \phi(x_1,\tilde{\VEC{x}})
\bigg|_{x_1=a_1}^{x_1=b_1}
= \lim_{x_ \to b_1} v(x_1,\tilde{\VEC{x}})\, \phi(x_1,\tilde{\VEC{x}})
- \lim_{x_ \to a_1} v(x_1,\tilde{\VEC{x}})\, \phi(x_1,\tilde{\VEC{x}}) = 0 \ .
\]
It follows from (\ref{distr_der_der}), (\ref{distr_der_derA}) and
(\ref{distr_der_derB}) that
\begin{equation} \label{distr_der_der2}
\int_\Omega \left(u(\VEC{x}) - v(\VEC{x})\right) \diff^{\VEC{\alpha}}
\phi(\VEC{x}) \dx{\VEC{x}} = 0
\end{equation}
for all $\phi \in \DD(\Omega)$.

If we consider $\phi \in \DD(\Omega)$ of the form
$\displaystyle \phi(\VEC{x}) = \phi_1(x_1) \, \phi_2(\tilde{\VEC{x}})$ for all
$\VEC{x} = (x_1,\tilde{\VEC{x}}) \in \Omega$, where
$\phi_1 \in \DD(]a_1,b_1[)$ and
$\displaystyle \phi_2 \in \DD\left(\prod_{j=2}^n]a_j,b_j[\right)$, we
can rewrite (\ref{distr_der_der2}) as
\[
\int_{\prod_{j=2}^n]a_j,b_j[} \left( \int_{a_1}^{b_1}
\left(u(x_1,\tilde{\VEC{x}}) - v(x_1,\tilde{\VEC{x}})\right)
\diff^{\VEC{\alpha}} \phi_1(x_1) \dx{x_1} \right) \phi_2(\tilde{\VEC{x}})
\dx{\tilde{\VEC{x}}} = 0
\]
for all $\phi_1 \in \DD(]a_1,b_1[)$ and
$\displaystyle \phi_2 \in \DD\left(\prod_{j=2}^n]a_j,b_j[\right)$.
It follows that, for almost all
$\displaystyle \tilde{\VEC{x}} \in \prod_{j=2}^n]a_j,b_j[$, we have
\[
\int_{a_1}^{b_1}
\left(u(x_1,\tilde{\VEC{x}}) - v(x_1,\tilde{\VEC{x}})\right) \diff^{\VEC{\alpha}}
\phi_1(x_1) \dx{x_1} = 0
\]
for all $\phi_1 \in \DD(]a_1,b_1[)$.
Hence, from Lemma~\ref{distr_der0_eq_const}, there exists a constant $C$ such
that $u-v = C$ almost everywhere in $\Omega$.  Since $v$ and $u$ are
continuous on $\Omega$, it follows that
$u(\VEC{x})-v(\VEC{x}) = C$ for all $\VEC{x} \in \Omega$.
However, $u(y_1,x_2,\ldots,x_n) = v(y_1,x_2, \ldots,x_n)$.  Thus
$C=0$.

\stage{$\mathbf{k>1}$} A proof by induction on $k$ takes care of
higher order partial derivatives.
\end{proof}

\section{Convolution} \label{sectConvolution}

\begin{theorem}[Generalized Young's Inequality] \label{distr_GyoungI}
Let $\left(X,\mu\right)$ be a measure space,
$1\leq p \leq \infty$, and $K:X\times X \rightarrow \CC$ be a
measurable function.  Suppose that there exists $C>0$ such that
\[
\sup_{x\in X} \int_X \left|K(x,y)\right| \dx{\mu(y)} \leq C
\quad \text{and} \quad
\sup_{y\in X} \int_X \left|K(x,y)\right| \dx{\mu(x)} \leq C \ .
\]
If $\displaystyle f \in L^p(X)$, then $Tf:X\rightarrow \CC$ defined by
\[
(Tf)(x) = \int_X K(x,y)f(y) \dx{\mu(y)}
\]
is defined almost everywhere, $\displaystyle Tf \in L^p(X)$ and
$\displaystyle \| Tf \|_p \leq C \|f\|_p$.
\index{Generalized Young's Inequality}
\end{theorem}

\begin{proof}
\stage{$0<p<\infty$}  Let $q$ be the
{\bfseries conjugate exponent}\index{Conjugate Exponent} of
$p$; namely. $\displaystyle \frac{1}{p} + \frac{1}{q} = 1$.  Since
\[
K(x,y)f(y) = K(x,y)^{1/q} K(x,y)^{1/p}f(y)
\]
for all $x,y \in X$, we get from Hölder's inequality that
\begin{align*}
\int_X \left| K(x,y)f(y)\right| \dx{\mu(y)}
&\leq \left( \int_X \left| K(x,y)\right| \dx{\mu(y)}\right)^{1/q}
\left(\int_X \left| K(x,y)\right|\,\left|f(y)\right|^p
\dx{\mu(y)}\right)^{1/p} \\
& \leq C^{1/q} 
\left(\int_X \left| K(x,y)\right|\,\left|f(y)\right|^p
\dx{\mu(y)}\right)^{1/p}
\end{align*}
for all $x \in X$.  Hence, from Fubini's theorem, we have
\begin{align}
&\int_X \left( \int_X \left| K(x,y)f(y)\right| \dx{\mu(y)} \right)^p \dx{\mu(x)}
\leq C^{p/q} \int_X \left(\int_X \left| K(x,y)\right|\,\left|f(y)\right|^p
\dx{\mu(y)}\right) \dx{\mu(x)} \nonumber \\
& \quad = C^{p/q} \int_X \left(\int_X \left| K(x,y)\right| \dx{\mu(x)}\right)
\,\left|f(y)\right|^p \dx{\mu(y)}
\leq C^{1+p/q} \int_X \left|f(y)\right|^p \dx{\mu(y)} < \infty \  .
\label{distr_g_young}
\end{align}
Thus
\[
\int_X \left| K(x,y)f(y)\right| \dx{\mu(y)} < \infty
\]
for almost all $x\in X$.  It follows that
$\displaystyle y \mapsto K(x,y)f(y) \in L^1(\mu)$ for almost all $x\in X$.
So $Tf$ is defined almost everywhere.  Moreover, since
\[
\left|(Tf)(x)\right| \leq \int_X \left| K(x,y)f(y)\right| \dx{\mu(y)} \ ,
\]
it follows from (\ref{distr_g_young}) that
\[
\int_X \left|(Tf)(x)\right|^p \dx{\mu(x)} \leq
C^{1+p/q} \int_X \left|f(y)\right|^p \dx{\mu(y)} \ .
\]
After taking the $p^{th}$ root on both sides of the previous
inequality, we get
\[
\| Tf\|_p \leq C^{1/p+1/q} \| f\|_p = C \|f \|_p \ .
\]

\stage{$p=1$}  We have from Fubini's theorem that
\begin{align}
\int_X \int_X \left| K(x,y)f(y)\right| \dx{\mu(y)} \dx{\mu(x)}
& = \int_X \left(\int_X \left| K(x,y)\right| \dx{\mu(x)}\right)
\,\left|f(y)\right| \dx{\mu(y)} \nonumber \\
& \leq C \int_X \left|f(y)\right|\dx{\mu(y)} < \infty \ .
\label{distr_g_young1}
\end{align}
Thus
\[
\int_X \left| K(x,y)f(y)\right| \dx{\mu(y)} < \infty
\]
for almost all $x\in X$.
It follows that $\displaystyle y \mapsto K(x,y)f(y) \in L^1(\mu)$ for
almost all $x\in X$.  So $Tf$ is defined almost everywhere.  Moreover, since
\[
\left|(Tf)(x)\right| \leq \int_X \left| K(x,y)f(y)\right| \dx{\mu(y)} \ ,
\]
we have from (\ref{distr_g_young1}) that
$\displaystyle \| Tf\| \leq  C \|f \|$.

\stage{$p=\infty$} We have
\[
\int_X \left| K(x,y)f(y)\right| \dx{\mu(y)}
\leq \|f\|_\infty \int_X \left| K(x,y)\right| \dx{\mu(y)}
\leq C \|f\|_\infty
\]
for all $x\in X$.
It follows that $\displaystyle y \mapsto K(x,y)f(y) \in L^1(\mu)$ for
all $x\in X$.  So $Tf$ is defined (almost) everywhere.  Moreover,
$\displaystyle \| (Tf)(x)\|_\infty \leq C \| f\|_\infty$.

The norm $\| \cdot \|_\infty$ above is the essential supremum of the
measurable function.
\end{proof}

The {\bfseries convolution}\index{Functions!Convolution} of $f$ and $g$ in
$\displaystyle L^1(\RR^n)$ is defined by
\[
(f\ast g)\left(\VEC{x}\right) = \int_{\RR^n}
f\left(\VEC{x} - \VEC{y}\right)  g\left(\VEC{y}\right) \dx{\VEC{y}}
\]
for all $\displaystyle \VEC{x} \in \RR^n$ where the integral
converges.  It can be proved that the integral is defined almost
everywhere.

\begin{theorem}[Young's Inequality] \label{distr_sp_young}
if $\displaystyle f \in L^1(\RR^n)$ and $\displaystyle g\in L^p(\RR^n)$
\index{Young's Inequality}
for $1\leq p \leq \infty$, then $\displaystyle f\ast g \in L^p(\RR^n)$ and
\[
\| f\ast g \|_p \leq \|f \|_1 \, \|g\|_p \ .
\]
\end{theorem}

\begin{proof}
It is a special case of the Generalized Young's Inequality where
$\displaystyle X=\RR^n$ and $K(x,y) = f\left(\VEC{x}-\VEC{y}\right)$.
\end{proof}

The following proposition is often used to expand to a larger class of
functions results proved for a special class of functions.

\begin{prop} \label{distr_limit_convol}
Suppose that $\displaystyle \phi \in L^1(\RR^n)$ and let
$\displaystyle a = \int_{\RR^n} \phi(\VEC{x})\dx{\VEC{x}}$.  For
$\epsilon >0$, let
\[
\phi_\epsilon(\VEC{x}) = \epsilon^{-n}
\phi\left(\epsilon^{-1}\VEC{x}\right) \ .
\]
If $\displaystyle f \in L^p(\RR^n)$ with $1 \leq p < \infty$, then
$\displaystyle f\ast \phi_\epsilon \rightarrow a f$ in
$\displaystyle L^p(\RR^n)$ as $\epsilon \rightarrow 0$.

If $\displaystyle f \in L^\infty(\RR^n)$ and $f$ is uniformly continuous on
an open set $V\subset \RR^n$, then
$\displaystyle f\ast \phi_\epsilon \rightarrow a f$ uniformly on
$K$ as $\epsilon \rightarrow 0$ for all compact sets $K\subset V$.
\end{prop}

\begin{proof}
The substitution $\VEC{y} = \epsilon^{-1} \VEC{x}$ shows that
\[
\int_{\RR^n} \phi_\epsilon(\VEC{x}) \dx{\VEC{x}} 
= \int_{\RR^n} \epsilon^{-n} \phi(\epsilon^{-1} \VEC{x}) \dx{\VEC{x}} 
= \int_{\RR^n} \phi(\VEC{y}) \dx{\VEC{y}} = a
\]
for every $\epsilon >0$.

\stage{i} If $\displaystyle f\in L^p(\RR^n)$ with $1\leq p < \infty$,
the substitution $\VEC{y} = \epsilon\, \VEC{z}$ yields
\[
(f\ast \phi_\epsilon)(\VEC{x}) - a f(\VEC{x}) =
\int_{\RR^n} \left( f(\VEC{x}-\VEC{y}) - f(\VEC{x})
\right)\phi_\epsilon(\VEC{y}) \dx{\VEC{y}}
= \int_{\RR^n} \left( f(\VEC{x}-\epsilon \VEC{z}) - f(\VEC{x})
\right)\phi(\VEC{z}) \dx{\VEC{z}} \ .
\]
Using Jensen's Inequality (Theorem 3.3 in \cite{Ru}), we get
\begin{align*}
\left| (f\ast \phi_\epsilon)(\VEC{x}) - a f(\VEC{x}) \right|^p
&\leq \left( \int_{\RR^n} \left| f(\VEC{x}-\epsilon \VEC{z}) - f(\VEC{x})
\right| \,\left| \phi(\VEC{z})\right| \dx{\VEC{z}} \right)^p \\
&\leq \int_{\RR^n} \left| f(\VEC{x}-\epsilon\VEC{z}) - f(\VEC{x})
\right|^p \,\left| \phi(\VEC{z})\right| \dx{\VEC{z}} \ .
\end{align*}
If we define $\displaystyle f_{\VEC{z}} \in L^p(\RR^n)$ by
$f_{\VEC{z}}(\VEC{x}) = f(\VEC{x}-\VEC{z})$ for
$\VEC{x}, \VEC{z}\in \RR^n$, we then have that
\[
\| f\ast \phi_\epsilon -a f \|_p^p
\leq \int_{\RR^n} \int_{\RR^n} \left| f(\VEC{x}-\epsilon \VEC{z})
 - f(\VEC{x})\right|^p\, \left| \phi(\VEC{z})\right| \dx{\VEC{z}}
\dx{\VEC{x}}
\leq \int_{\RR^n} \| f_{\epsilon\VEC{z}} - f \|_p^p
\left|\phi(\VEC{z})\right|
\dx{\VEC{z}} \ .
\]
However,
\[
\| f_{\epsilon\VEC{z}} - f \|_p^p \left|\phi(\VEC{z})\right| \leq
(2 \|f\|_p)^p \left|\phi(\VEC{z})\right| \in L^1(\RR^n)
\]
and
\[
\| f_{\epsilon\VEC{z}} - f \|_p^p \left|\phi(\VEC{z})\right| \rightarrow 0
\quad \text{as} \quad \epsilon \rightarrow 0
\]
for all $\VEC{z}$ because
$\displaystyle \| f_{\epsilon\VEC{z}} - f \|_p \rightarrow 0$
as $\epsilon \rightarrow 0$ (Theorem~9.5 in \cite{Ru}).  It follows from the
Lebesgue Dominate Convergence Theorem that
\[
\| f\ast \phi_\epsilon -a f \|_p \rightarrow 0 \quad \text{as} \quad
\epsilon \rightarrow 0 \ .
\]

\stage{ii} Suppose that $\displaystyle f \in L^\infty(\RR^n)$ and $f$
is uniformly continuous on $\displaystyle V\subset \RR^n$.
Let $K \subset V$ be a compact set.

Given $\delta >0$, choose a compact set
$\displaystyle W \subset \RR^n$ such that
\[
\int_{\RR^n\setminus W} |\phi(\VEC{x})| \dx{\VEC{x}} <
\frac{\delta}{4\|f\|_\infty} \ .
\]
This is possible because $\displaystyle \phi \in L^1(\RR^n)$.

Since $f$ is uniformly continuous on the open set $V$, we may choose
$\epsilon$ small enough such that $K+\epsilon W \subset V$ and
\[
\sup_{\VEC{x}\in K,\VEC{y}\in W} \left| f(\VEC{x}+\epsilon \VEC{y})
-f(\VEC{x})\right| < \frac{\delta}{2\|\phi\|_1} \ .
\]
Then,
\begin{align*}
&\sup_{\VEC{x}\in K} \left| (f\ast \phi_\epsilon)(\VEC{x}) - a f(\VEC{x}) \right|
\leq \sup_{\VEC{x}\in K} 
\int_{\RR^n} \left| f(\VEC{x}-\epsilon \VEC{y})
 - f(\VEC{x})\right| \, \left|\phi(\VEC{y})\right| \dx{\VEC{y}} \\
&\quad \leq \sup_{\VEC{x}\in K, \VEC{y} \in W} 
\left| f(\VEC{x}-\epsilon \VEC{y}) - f(\VEC{x})\right|
\, \int_{W} \left|\phi(\VEC{y})\right| \dx{\VEC{y}} + 2 \| f \|_\infty
\int_{\RR^n\setminus W}\left|\phi(\VEC{y})\right| \dx{\VEC{y}} \\
&\quad \leq \frac{\delta}{2 \|\phi\|_1} \,
\int_{W} \left|\phi(\VEC{y})\right| \dx{\VEC{y}}
+ 2 \| f \|_\infty \, \frac{\delta}{4\|f\|_\infty} < \delta \ .
\end{align*}
The role of $\epsilon$ and $\delta$ are interchanged from the usual
role that they play in a proof of convergence.  Instead of ``for all
$\epsilon$ there exists $\delta$ \ldots'', we have ``for all $\delta$
there exists $\epsilon$ \ldots''
\end{proof}

\begin{defn}
The functions $\phi_\epsilon$ for $\epsilon>0$ defined in
Proposition~\ref{distr_limit_convol} are called
{\bfseries approximate identity}\index{Approximate Identity} when $a=1$.
\end{defn}

Given $\displaystyle \phi:\RR^n \rightarrow \CC$, we define
$\displaystyle \phi^\vee$ and
$\displaystyle \phi_{\VEC{z}}$ in $\DD(\RR^n)$ by
$\displaystyle \phi^{\vee}(\VEC{x}) = \phi(-\VEC{x})$ for
$\displaystyle \VEC{x} \in \RR^n$ and
$\displaystyle \phi_{\VEC{z}}(\VEC{x}) = \phi(\VEC{x}-\VEC{z})$
for $\displaystyle \VEC{x}, \VEC{r} \in \RR^n$.

This motivates the following definition.  Given
$\displaystyle u \in \DD'(\RR^n)$,
we define $\displaystyle u^\vee$ and $u_{\VEC{z}}$ in
$\displaystyle \DD'(\RR^n)$ by
$\displaystyle u^\vee(\phi) = \ps{u^\vee}{\phi} = \ps{u}{\phi^\vee}$
and
$\displaystyle u_{\VEC{z}}(\phi) = \ps{u_{\VEC{z}}}{\phi}
= \ps{u}{\phi_{-\VEC{z}}}$ for $\displaystyle \phi \in \DD(\RR^n)$.

For two functions $f$ and $g$ in $\displaystyle L^1(\RR^n)$, we have
\[
(f\ast g)\left(\VEC{x}\right) = \int_{\RR^n} f\left(\VEC{y}\right)
(g^\vee)_{\VEC{x}}\left(\VEC{y}\right)\dx{\VEC{y}} \ .
\]
It therefore makes sense to define the
{\bfseries convolution}\index{Distribution!Convolution} of
$\displaystyle u\in \DD'(\RR^n)$ with $\displaystyle \phi\in \DD(\RR^n)$ by
\[
\left( u \ast \phi\right)(\VEC{x}) =
\ps{u}{\left(\phi^{\vee}\right)_{\VEC{x}}}
\]
for all $\displaystyle \VEC{x} \in \RR^n$.
We will see later that this definition of convolution is the usual
definition of convolution when $\displaystyle u\in L^1(\RR^n)$.

Suppose that $\displaystyle u \in \EE'(\RR^n)$ and that $K$ is the
compact support of $u$.  Choose $\displaystyle \psi \in \DD(\RR^n)$
such that $\psi(z) =1$ for all $z \in K$.  We have that $\psi u = u$ because
$\ps{\psi u}{\phi} = \ps{u}{\psi \phi} = \ps{u}{\phi}$ for all
$\displaystyle \phi \in \DD(\RR^n)$.  This is obviously independent of
the function $\psi$ chosen.  We can use this property to extend $u$ in a
unique way to a continuous linear functional on
$\displaystyle C^\infty(\RR^n)$,
where the convergence in $\displaystyle C^\infty(\RR^n)$ is defined by uniform
convergence on compact sets.  We define
$\displaystyle u:C^\infty(\RR^n) \to \CC$ by
$u(\phi) = u(\psi \phi)$ for $\displaystyle \phi \in C^\infty(\RR^n)$.

\begin{prop} \label{distr_smooth_convol}
\begin{enumerate}
\item If $\displaystyle u\in \DD'(\RR^n)$ and
$\displaystyle \psi\in \DD(\RR^n)$, then $u\ast \psi$
is in $\displaystyle C^\infty(\RR^n)$.
\item If $\displaystyle u\in \EE'(\RR^n)$ and
$\displaystyle \psi\in \DD(\RR^n)$, then
$\displaystyle u \ast \psi \in \DD(\RR^n)$.
\item If $\displaystyle u\in \EE'(\RR^n)$ and
$\displaystyle \psi\in C^\infty(\RR^n)$, then
$\displaystyle u \ast \psi \in C^\infty(\RR^n)$.
\end{enumerate}
\end{prop}

\begin{prop} \label{distr_comm_der}
If $\displaystyle u\in \DD'(\RR^n)$ and
$\displaystyle \phi\in \DD(\RR^n)$, then
$\displaystyle \diff^{\VEC{\alpha}}(u\ast \phi)
= u \ast \diff^{\VEC{\alpha}} \phi = \diff^{\VEC{\alpha}} u \ast \phi$
for all multi-indices $\VEC{\alpha}$.
\end{prop}

We may also define the convolution of two distributions if one of them
has compact support.  Consider $\displaystyle u,v \in \DD'(\RR^n)$.
If one of them has compact support, then the function
$\displaystyle L: \DD(\RR^n) \to C^\infty(\RR^n)$
defined by $\displaystyle L(\phi) = u \ast (v \ast \phi)$ for
$\displaystyle \phi \in \DD(\RR^n)$ is well defined.  To justify this
statement, we note that if $v$ has compact support, then
$\displaystyle v \ast \phi \in \DD(\RR^n)$
by (2) of Proposition~\ref{distr_smooth_convol} and
$\displaystyle u \ast (v \ast \phi) \in C^\infty(\RR^n)$ by (1) of
Proposition~\ref{distr_smooth_convol}.  Moreover, if $u$ has compact
support and $v$ does not have compact support, then
$\displaystyle v \ast \phi \in C^\infty(\RR^n)$ by (1) of
Proposition~\ref{distr_smooth_convol} and again 
$\displaystyle u \ast (v \ast \phi) \in C^\infty(\RR^n)$ by (3) of
Proposition~\ref{distr_smooth_convol}.

If follows from Theorem~6.33 in \cite{RuFA} that there exists a unique
distribution on $\displaystyle \RR^n$, denoted $u\ast v$, such that
\[
L(\phi) = u \ast ( v \ast \phi) = (u \ast v ) \ast \phi
\]
for all $\displaystyle \phi \in \DD(\RR^n)$.  In particular, we have
\[
\ps{(u \ast v)}{\phi} = ( (u\ast v) \ast \phi^\vee)(\VEC{0}) 
= (u \ast (v \ast \phi^\vee))(\VEC{0}) 
= \ps{u}{ (v \ast \phi^\vee)^\vee}
\]
for all $\displaystyle \phi \in \DD(\RR^n)$.

\begin{prop} \label{uvequvu}
If $\displaystyle u\in \EE'(\RR^n)$ and
$\displaystyle \psi\in C^\infty(\RR^n)$, then
$u \ast \psi = \psi \ast u$.
\end{prop}

\begin{proof}
First, we have to expend our definition of integration.
Suppose that $\Omega$ is a measurable space with measure $\mu$, that $E$
is a {\bfseries toplogical vector space}\index{Topological Vector
Space} \footnote{We will not give a definition of topological
vector spaces.  This will lead us too dip into analysis.  The reader
can find a definition of topological vector space in
\cite{ReeSim,RuFA,Tr} and many other graduate textbooks in Analysis.
A Banach space is a topological vector space.}
and that $f:\Omega \to E$ is a measurable
function.  The integral of $f$ over $\Omega$ is the element $y \in E$,
if such a $y$ exists, such that
\begin{equation} \label{intVectFunct}
  u(y) = \int_\Omega u\circ f \dx{\mu}
\end{equation}
for all $\displaystyle u \in E^\ast$.   We assume that
$\displaystyle E^\ast$ separates points in $E$
\footnote{Namely, for any $y_1$ and $y_2$ in
$E$ with $y_1 \neq y_2$, there exists $\displaystyle u \in E^\ast$
such that $u(y_1) \neq u(y_2)$.} to ensure the uniqueness of $y$ if it
exists.  There is a more traditional approach to define the integral
in (\ref{intVectFunct}) using the equivalent of ``Riemann sums.''\
This approach can be found in more traditional textbooks in functional
analysis.

Let
\[
g(x) = \int_{\RR^n} \psi(\VEC{w}) \phi(\VEC{x} - \VEC{w}) \dx{\VEC{w}}
\]
for $\displaystyle \VEC{x} \in \RR^n$.  According to the definition of
$u \ast \psi$, we have
\[
\big( (u\ast \psi) \ast \phi \big)(\VEC{z})
= \big( u \ast ( \psi \ast \phi) \big)(\VEC{z})
= \left( u \ast g \right)(\VEC{z})
\]
for $\displaystyle \VEC{z} \in \RR^n$ because $\psi$ and $\phi$ are of class
$\displaystyle C^\infty(\RR^n)$ with one of them having a compact support.
So $\displaystyle \psi \ast \phi \in \DD(\RR^n)$.
We then have that
\[
\big( (u\ast \psi) \ast \phi \big)(\VEC{z})
= \left( u \ast g \right)(\VEC{z})
= \ps{u}{ \left(g^\vee\right)_{\VEC{z}}}
= \ps{u}{h} \ ,
\]
where
\[
h(\VEC{x}) = g(\VEC{z}- \VEC{x})
= \int_{\RR^n} \psi(\VEC{w}) \phi(\VEC{z} - \VEC{x} - \VEC{w}) \dx{\VEC{w}}
\]
for all $\displaystyle \VEC{x} \in \RR^n$.  If $\displaystyle f(\VEC{w})$ for
$\displaystyle \VEC{w} \in \RR^n$ is the function in $\displaystyle \DD(\RR^n)$
defined by
\[
f(\VEC{w})(\VEC{x}) = \psi(\VEC{w}) \phi(\VEC{z} - \VEC{x} - \VEC{w})
\]
for all $\displaystyle \VEC{x} \in \RR^n$, then we get
\[
\big( (u\ast \psi) \ast \phi \big)(\VEC{z}) = \ps{u}{h}
= \int_{\RR^n} u \circ f \dx{\VEC{w}} \ ,
\]
where we have used (\ref{intVectFunct}) with
$\displaystyle \Omega = \RR^n$ and $\displaystyle E = \DD(\RR^n)$.
If $\displaystyle q(\VEC{w})$ for
$\displaystyle \VEC{w} \in \RR^n$ is the function in $\displaystyle \DD(\RR^n)$
defined by
\[
q(\VEC{w})(\VEC{x}) = \phi(\VEC{z} - \VEC{x} - \VEC{w})
\]
for all $\displaystyle \VEC{x} \in \RR^n$, then
\begin{align*}
\big( (u\ast \psi) \ast \phi \big)(\VEC{z}) 
& = \int_{\RR^n} \psi(\VEC{w}) \ps{u}{q(\VEC{w})} \dx{\VEC{w}}
= \int_{\RR^n} \psi(\VEC{w})
\ps{u}{\left(\phi^\vee\right)_{\VEC{z}-\VEC{w}}} \dx{\VEC{w}} \\
&= \int_{\RR^n} \psi(\VEC{w}) \big( u \ast \phi \big)(\VEC{z} - \VEC{w})
\dx{\VEC{w}}
= \big( \psi \ast (u \ast \phi) \big)(\VEC{z})
\end{align*}
for $\displaystyle \VEC{z} \in \RR^n$ and all
$\displaystyle \phi \in \DD(\RR^n)$.  Thus,
$\psi \ast (u \ast \phi) = (u\ast \psi) \ast \phi$ for all
$\displaystyle \phi \in \DD(\RR^n)$.  But, by definition of the
convolution of two distributions with one having a compact support,
this states that $\psi \ast u = u\ast \psi$.  
\end{proof}

\section{Fourier Transforms} \label{sectionFT}

If $\displaystyle f \in L^1(\RR^n)$, the
{\bfseries Fourier transform}\index{Functions!Fourier Transform} of $f$ is the
function defined by
\begin{equation} \label{distr_fourierT}
\F(f)(\VEC{y}) \equiv \hat{f}(\VEC{y}) = \int_{\RR^n} e^{-i \VEC{x}\cdot\VEC{y}}
f(\VEC{x}) \dx{\mu(\VEC{x})} \quad , \quad \VEC{y} \in \RR^n \ ,
\end{equation}
where $\mu$ is the Lebesgue measure $m$ divided by $\displaystyle (2\pi)^{n/2}$.
To be able to state aesthetically pleasing relations between
Fourier transform and convolution, we redefined the
{\bfseries convolution}\index{Functions!Convolution} of
two functions $\displaystyle f,g \in L^1(\RR^n)$ as
\[
(f\ast g)\left(\VEC{x}\right) = \int_{\RR^n}
f\left(\VEC{x} - \VEC{y}\right)  g\left(\VEC{y}\right) \dx{\mu(\VEC{y})}
\]
for all $\displaystyle \VEC{x} \in \RR^n$ where the integral converges.  We have
added the factor $\displaystyle (2\pi)^{-n/2}$ to the previous definition of
convolution.   We will use this definition of convolution only inside
this section and revert to the previous definition in the rest of the
text.  The reader shall therefore not be surprise by the appearance
of the factor $\displaystyle (2\pi)^{n/2}$ in the formulae involving
convolution and Fourier transform later in the text.

\begin{egg}
Suppose that $f \in C(\RR)$ has a compact support.

We first show that the function $\displaystyle g:\RR \to \RR$ defined by
$\displaystyle g(x) = \sum_{k\in \ZZ} f(x+ 2 \pi k)$ is a
continuous periodic function of period $2\pi$.  Since $f$ has a
compact support, the summation is finite on every bounded open set.
It follows that $g \in C(\RR)$ since continuity is a local property.
Since
\[
g(x+2\pi) = \sum_{k\in \ZZ} f(x+2\pi(k+1))
= \sum_{k\in \ZZ} f(x+2 \pi k) = g(x)
\]
for all $x$, we have that $g$ is periodic of period $2 \pi$.

We now show that the coefficient $a_n$ of the Fourier series
$\displaystyle \sum_{n\in \ZZ} a_n e^{n x i}$ of $g$ is given by
$\displaystyle a_n = \frac{1}{2\pi} \hat{f}(n)$ for all $n$.  It is a simple
computation from the definition of the coefficient of the series.
\begin{align*}
a_n &= \frac{1}{2\pi} \int_{-\pi}^{\pi} g(x) e^{- n x i} \dx{x}
= \frac{1}{2\pi}
\sum_{k\in \ZZ} \int_{-\pi}^{\pi} f(x+ 2 \pi k) e^{-n x i} \dx{x} \\
&= \frac{1}{2\pi} 
\sum_{k\in \ZZ} \int_{(2k-1)\pi}^{(2k+1)\pi} f(y)
e^{-n (y-2 \pi k) i} \dx{y}
= \frac{1}{2\pi} 
\sum_{k\in \ZZ} \int_{(2k-1)\pi}^{(2k+1)\pi} f(y) e^{-n y i} \dx{y} \\
&= \frac{1}{2\pi} \int_{-\infty}^{\infty} f(y) e^{- n y i} \dx{y}
= \frac{1}{2\pi} \hat{f}(n)
\end{align*}
for all $n \in \ZZ$.  Note that the previous summations are finite.

It follows that
\begin{equation} \label{eggPoissonSF}
  \frac{1}{2\pi} \sum_{n\in \ZZ} \hat{f}(n) = \sum_{n\in\ZZ} f(n) \ .
\end{equation}
Since $g$ is continuous we have that
\[
\frac{1}{2\pi} \sum_{n\in \ZZ} \hat{f}(n) e^{n x i}
= \sum_{n\in \ZZ} a_n e^{n x i} = g(x) =  \sum_{n\in\ZZ} f(x+n)
\]
for all $x$.  If $x = 0$, we get (\ref{eggPoissonSF}).  The formula 
(\ref{eggPoissonSF}) is known as the
{\bfseries Poisson summation formula}\index{Poisson Summation Formula}.
\end{egg}

\begin{egg}
As an example, we compute the Fourier transform of
$\displaystyle f(x) = e^{-\|\VEC{x}\|^2}$ for $\displaystyle \VEC{x} \in \RR^n$.
This Fourier transform will play an important role in the study of the
heat equation.  \label{FTexpx2}
Since
\[
  \hat{f}(\VEC{y}) = \int_{\RR^n} e^{-i \VEC{x}\cdot\VEC{y}}
e^{-\|\VEC{x}\|^2} \dx{\mu(\VEC{x})} 
= \prod_{j=1}^n \left( \frac{1}{\sqrt{2\pi}}
  \int_{-\infty}^\infty  e^{-i x_j y_j - x_j^2} \dx{x_j} \right) \ ,
\]
it suffices to compute
\[
\frac{1}{\sqrt{2\pi}} \int_{-\infty}^\infty  e^{-i x y - x^2} \dx{x} \ .
\]
If we use the substitution $x = r/2$, we get
\[
\frac{1}{\sqrt{2\pi}} \int_{-\infty}^\infty  e^{-i x y - x^2} \dx{x}
= \frac{1}{2\sqrt{2\pi}} \int_{-\infty}^\infty  e^{-(2 i r y + r^2)/4} \dx{r}
= \frac{1}{2\sqrt{2\pi}} e^{-y^2/4}
\int_{-\infty}^\infty  e^{-(r+ y i)^2/4} \dx{r} \ .
\]
To compute the integral, we note that $\displaystyle e^{-z^2/4}$ with
$z \in \CC$ is an entire function and use Cauchy's theorem for
integrals along a close path to conclude that
$\displaystyle \int_{\Gamma} e^{-z^2/4} \dx{z} = 0$ where $\Gamma$ is
the close path defined by
$\displaystyle \Gamma = \Gamma_1 + \Gamma_2 - \Gamma_3 - \Gamma_4$
with $\Gamma_1 = \{r + 0 i : -R \leq r \leq R\}$,
$\Gamma_2=\{R + s i : 0 \leq s \leq y \}$,
$\Gamma_3 = \{r + y i : -R \leq r \leq R\}$ and
$\Gamma_4 = \{-R + s i : 0 \leq s \leq y\}$ (Figure~\ref{distr_FIG2}).
We have that
\[
\int_{\Gamma_1} e^{-z^2/4} \dx{z}
= \int_{-R}^R e^{-r^2/4} \dx{r} \to \int_{-\infty}^\infty e^{-r^2/4} \dx{r}
= 2 \int_{-\infty}^\infty e^{-t^2} \dx{t} = 2\sqrt{\pi}
\quad \text{as} \quad R \to \infty \ ,
\]
\[
\int_{\Gamma_3} e^{-z^2/4} \dx{z}
= \int_{-R}^R e^{-(r+y i)^2/4} \dx{r} \to
\int_{-\infty}^\infty e^{-(r+yi)^2/4} \dx{r}
\quad \text{as} \quad R \to \infty \ ,
\]
\[
\left| \int_{\Gamma_2} e^{-z^2/4} \dx{z} \right|
\leq \int_0^y \left| e^{-(R+s i)^2/4}\right| \dx{s}
= e^{-R^2/4} \int_0^y e^{s^2/4} \dx{s} \to 0
\quad \text{as} \quad R \to \infty \ ,
\]
and similarly
\[
\left| \int_{\Gamma_4} e^{-z^2/4} \dx{z} \right|
\leq \int_0^y \left| e^{-(-R+s i)^2/4}\right| \dx{s}
= e^{-R^2/4} \int_0^y e^{s^2/4} \dx{s} \to 0
\quad \text{as} \quad R \to \infty \ .
\]
Thus,
\[
\int_{-\infty}^\infty  e^{-(r+ y i)^2/4} \dx{r} = 2 \sqrt{\pi} \ .
\]
We finally get that
\[
\hat{f}(\VEC{y}) = 
\prod_{j=1}^n \left( \frac{1}{\sqrt{2}} e^{-y_j^2/4} \right)
= \frac{1}{2^{n/2}} e^{-\|\VEC{y}\|^2/4} \ .
\]

More generally, we can compute the Fourier transform of
$\displaystyle f(x) = e^{-a \|\VEC{x}\|^2}$ for
$\displaystyle \VEC{x} \in \RR^n$
and $a$ a positive constant.  We have that
\[
\hat{g}(\VEC{y}) = \int_{\RR^n} e^{-i \VEC{x}\cdot\VEC{y}}
e^{-a\|\VEC{x}\|^2} \dx{\mu(\VEC{x})} \ .
\]
If we use the change of variables $\VEC{w} = \sqrt{a}\, \VEC{x}$, we get
\begin{equation} \label{FTexpx2eq1}
\hat{g}(\VEC{y}) = \frac{1}{a^{n/2}}
\int_{\RR^n} e^{-i \VEC{w}\cdot (a^{-1/2}\VEC{y})}
e^{-\|\VEC{w}\|^2} \dx{\mu(\VEC{w})}
= \frac{1}{a^{n/2}} \hat{f}(a^{-1/2}\VEC{y})
= \frac{1}{(2a)^{n/2}} e^{-\|\VEC{y}\|^2/(4a)} \ .
\end{equation}
\end{egg}

\pdfF{distributions/distr_fig2}{Contour of integration for an Inverse
Fourier Transform}{According to Cauchy's theorem for integrals along a
close path, we have that
$\displaystyle \int_{\Gamma} = \int_{\Gamma_1} + \int_{\Gamma_2}
- \int_{\Gamma_3} - \int_{\Gamma_4} = 0$.}{distr_FIG2}

The Fourier transform has the following properties.

\begin{prop} \label{FTLofPs}
\begin{enumerate}
\item If $f$ and $g$ are in $\displaystyle L^1(\RR^n)$, then
$\displaystyle (f\ast g)^\wedge = \hat{f}\,\hat{g}$.
\item If $f$ is in $\displaystyle L^1(\RR^n)$, then
$\displaystyle (f_{\VEC{z}})^\wedge(\VEC{y}) =
e^{-i \VEC{z}\cdot\VEC{y}} \hat{f}(\VEC{y})$
\item If $\displaystyle f \in C_c^m(\RR^n)$, $\VEC{\alpha}$ is a multi-index
such that $|\VEC{\alpha}|\leq m$, and $p$ is the polynomial
$\displaystyle p(\VEC{x}) = (-i\VEC{x})^{\VEC{\alpha}}$,
then $\displaystyle (\diff^{\VEC{\alpha}} f)^\wedge = p \hat{f}$ and
$\displaystyle (p f)^\wedge = \diff^{\VEC{\alpha}}\hat{f}$.
\end{enumerate}
\end{prop}

\begin{proof}
Left to the reader.
\end{proof}

In a course for engineer students, the Fourier transform and the
``inverse Fourier transform'' are introduced without considerations
for the domain and range of these transformations.  The following
theorem states under which conditions the inversion formula is valid.
Before stating this theorem, we need to define the space
$\displaystyle C_0(\RR^n)$ of continuous functions that
{\bfseries vanish at infinity}\index{Functions!Vanish at Infinity}.

The space $\displaystyle C_0(\RR^n)$ is the space of continuous real valued
functions on $\displaystyle \RR^n$ satisfying the following property.  For every
$\epsilon >0$, there exists a compact set $K$ such that
$|f(\VEC{x})|<\epsilon$ for $\displaystyle \VEC{x} \in \RR^n \setminus K$.

\begin{prop} \label{distr_C0_four_transf}
If $\displaystyle f\in L^1(\RR^n)$, then $\hat{f} \in C_0(\RR^n)$ and
$\|\hat{f}\|_\infty \leq \|f\|_1$.
\end{prop}

\begin{prop}[Inverse Fourier Transform] \label{distr_inv_four_transf}
If $\displaystyle f \in L^1(\RR^n)$, $\displaystyle \hat{f} \in L^1(\RR^n)$ and
\[
g(\VEC{y}) = \int_{\RR^n} e^{i \VEC{x}\cdot\VEC{y}} \,
\hat{f}(\VEC{x}) \dx{\mu(\VEC{x})}
\]
for all $\displaystyle \VEC{y} \in \RR^n$, then
$\displaystyle g \in C_0(\RR^n)$.  Moreover, $f = g$ almost everywhere
in $\displaystyle \RR^n$ and $\|f\|_\infty \leq \|\hat{f}\|_1$.
\end{prop}

Given $\displaystyle f\in L^1(\RR^n)$, let
\[
\F^{-1}(f)(\VEC{y}) \equiv \check{f}(\VEC{y}) = 
\int_{\RR^n} e^{i \VEC{x}\cdot\VEC{y}} \, f(\VEC{x}) \dx{\mu(\VEC{x})}
\quad , \quad \VEC{y} \in \RR^n \ .
\]
The previous theorem states that $\displaystyle \F^{-1}(\F(f)) = f$
almost everywhere when $\displaystyle f \in L^1(\RR^n)$
and $\displaystyle \hat{f} \in L^1(\RR^n)$.

It is possible to extend the Fourier transform to $\displaystyle L^2(\RR^n)$.
This will be useful when talking about Sobolev spaces later.

\begin{theorem}[Plancherel] \label{distr_plancherel}
To each $\displaystyle f\in L^2(\RR^n)$ can be associated a function
$\displaystyle \hat{f} \in L^2(\RR^n)$ such that   \index{Plancherel's Theorem}
\begin{enumerate}
\item For $\displaystyle f \in L^1(\RR^n) \cap L^2(\RR^n)$. $\hat{f}$ is the
Fourier transform given in (\ref{distr_fourierT}).
\item $\| f \|_2 = \|\hat{f}\|_2$ for all $\displaystyle f \in L^2(\RR^n)$.
In fact, $\displaystyle \int_{\RR^n} f g \dx{\VEC{x}} =
\int_{\RR^n} \hat{f} \hat{g} \dx{\VEC{x}}$ for all
$\displaystyle f,g \in L^2(\RR^n)$.
\item The linear mapping $\displaystyle \F: L^2(\RR^n) \rightarrow L^2(\RR^n)$
defined by $\F(f) = \hat{f}$ for all $\displaystyle f \in L^2(\RR^n)$
is an Hilbert space isomorphism.
\end{enumerate}
\end{theorem}

In fact, (2) and (3) in Theorem~\ref{distr_plancherel} implies that 
$\displaystyle \F: L^2(\RR^n) \rightarrow L^2(\RR^n)$ is an isometry.

The space $\displaystyle \SS(\RR^n)$ of
{\bfseries rapidly decreasing functions}\index{Functions!Rapidly
Decreasing Functions} on $\displaystyle \RR^n$ (also called 
{\bfseries Schwartz class}\index{Functions!Schwartz Class}) is the space of all
functions $\displaystyle f \in C^\infty(\RR^n)$ such that
\[
\sup_{\VEC{x} \in \RR^n} \left| \VEC{x}^{\VEC{\alpha}}
\diff^{\VEC{\beta}} f(\VEC{x}) \right| < \infty
\]
for all multi-indices $\VEC{\alpha}$ and $\VEC{\beta}$.  A locally
convex topology is defined on $\SS(\RR^n)$ with the help of the norms
defined by
\[
\sup_{|\VEC{\alpha}|\leq k} \sup_{\VEC{x} \in \RR^n}
\left( 1 + \|\VEC{x}\|_2^2\right)^2
\left| \diff^{\VEC{\alpha}} f(\VEC{x}) \right|
\]
for $k \geq 0$.  We will not elaborate on this subject but the
interested reader may consult \cite{RuFA,ReeSim} for more information.

If $\displaystyle f \in \SS(\RR^n)$, we have that
$\displaystyle \VEC{x}^{\VEC{\alpha}} \diff^{\VEC{\beta}} f$ itself
vanishes at infinity for all multi-indices $\VEC{\alpha}$ and $\VEC{\beta}$.
Moreover,
$\displaystyle \VEC{x}^{\VEC{\alpha}}\diff^{\VEC{\beta}} f \in L^1(\RR^n)$
for all multi-indices $\VEC{\alpha}$ and $\VEC{\beta}$.  To prove this, choose
$k > n/2$ and $C>0$.  Since $f$ vanishes at infinity, there exists $r>0$
such that
$\displaystyle \|\VEC{x}\|^{2k}\, \left| f(\VEC{x}) \right| < C$ for
all $\|\VEC{x}\| > r$.  Hence,
\begin{align*}
\int_{\RR^n} \left| f(\VEC{x})\right| \dx{\VEC{x}} &=
\int_{\|\VEC{x}\|\leq r} \left| f(\VEC{x})\right| \dx{\VEC{x}} +
\int_{\|\VEC{x}\|> r} \left| f(\VEC{x})\right| \dx{\VEC{x}} \\
&= \int_{\|\VEC{x}\|\leq r} \left| f(\VEC{x})\right| \dx{\VEC{x}} +
\int_{\|\VEC{x}\|> r} \|\VEC{x}\|^{2k}\,\left| f(\VEC{x})\right|
\|\VEC{x}\|^{-2k} \dx{\VEC{x}} \\
&\leq \int_{\|\VEC{x}\|\leq r} \left| f(\VEC{x})\right| \dx{\VEC{x}} +
C \int_{\|\VEC{x}\|> r} \|\VEC{x}\|^{-2k} \dx{\VEC{x}} < \infty
\end{align*}
because $f$ is locally integrable, the integral of $|f|$ on
$\{ \VEC{x} : \|\VEC{x}\| \leq r \}$ is finite, and\\
$\displaystyle \int_{\|\VEC{x}\|> r} \|\VEC{x}\|^{-2k} \dx{\VEC{x}} < \infty$
since $2k >n$.  For all multi-indices $\VEC{\alpha}$ and $\VEC{\beta}$, this
reasoning can be applied to
$\displaystyle \VEC{x}^{\VEC{\alpha}} \diff^{\VEC{\beta}} f$ instead of $f$ to
show that
$\displaystyle \VEC{x}^{\VEC{\alpha}} \diff^{\VEC{\beta}} f \in L^1(\RR^n)$.

Hence, the Fourier transform of rapidly decreasing functions and their
derivatives exists.

\begin{prop} \label{distr_four_f}
If $\displaystyle f \in L^1(\RR^n)$ has compact support, then $\hat{f}$ is the
restriction to $\displaystyle \RR^n$ of an entire function on
$\displaystyle \CC^n$.  If
$\displaystyle f\in \DD(\RR^n)$, then $\hat{f}$ is a rapidly decreasing
function on any subset of the form
$\displaystyle \left\{ \VEC{z}\in \CC^n : \| \IM \VEC{z} \|_2 \leq c \right\}$ 
for $c>0$.  Recall that
$\displaystyle \IM \VEC{z} =
\begin{pmatrix} \IM z_1 & \IM z_2 & \ldots & \IM z_n \end{pmatrix}^\top$
for $\displaystyle \VEC{z} \in \CC^n$.
\end{prop}

\begin{proof}
The integral
\[
\int_{\RR^n} e^{-i \VEC{z}\cdot \VEC{x}} f(\VEC{x}) \dx{\mu(\VEC{x})} \ ,
\]
converges for all $\displaystyle \VEC{z} \in \CC^n$ because $f$ has a compact
support so the integration is over a bounded set and the
exponential factor is bounded on this set.
Hence, we may define $\hat{f}(\VEC{z})$ for $\displaystyle \VEC{z} \in \CC^n$ by
\[
  \hat{f}(\VEC{z}) =
\int_{\RR^n} e^{-i \VEC{z}\cdot \VEC{x}} f(\VEC{x}) \dx{\mu(\VEC{x})} \ .
\]
$\hat{f}$ is an entire function because
$\displaystyle\VEC{z} \mapsto  e^{i \VEC{z}\cdot \VEC{x}}$ is an entire function
and, for any multi-index $\VEC{\alpha}$, complex
derivatives $\displaystyle \diff_{\VEC{z}}^{\VEC{\alpha}}$ and integration
can be interchanged.

If $\displaystyle f \in \DD(\RR^n)$, choose $r$ large enough such that
$\supp f \subset \{ \VEC{x} : \|\VEC{x}\| < r \}$.  Then, for any
multi-index $\VEC{\alpha}$,
\begin{align}
\left| \VEC{z}^{\VEC{\alpha}} \hat{f}(\VEC{z}) \right|
&= \left| (i)^{|\VEC{\alpha}|} \big(\diff^{\VEC{\alpha}}
f\big)^\wedge(\VEC{z}) \right|
= \left| \int_{\RR^n} e^{-i \VEC{z}\cdot\VEC{x}} \diff^{\VEC{\alpha}} f(\VEC{x})
  \dx{\mu(\VEC{x})} \right| \nonumber \\
&\leq \int_{\|\VEC{x}\| < r} \left| e^{-i \VEC{z}\cdot\VEC{x}}\right|\,
\left|\diff^{\VEC{\alpha}} f(\VEC{x})\right| \dx{\mu(\VEC{x})}
\leq e^{r \|\IM \VEC{z}\|_2}\,\int_{\|\VEC{x}\| < r}
\left|\diff^{\VEC{\alpha}} f(\VEC{x})\right| \dx{\mu(\VEC{x})} \nonumber \\
& = e^{r \|\IM \VEC{z}\|_2}\,\int_{\RR^n}
\left|\diff^{\VEC{\alpha}} f(\VEC{x})\right| \dx{\mu(\VEC{x})}
\label{proof_distr_four_f} 
\end{align}
for all $\displaystyle \VEC{z} \in \CC^n$.
Hence, for $\|\IM \VEC{z} \|_2 \leq c$, we have
\[
\left| \VEC{z}^{\VEC{\alpha}} \hat{f}(\VEC{z}) \right|
\leq e^{r c}
\int_{\RR^n} \left|\diff^{\VEC{\alpha}} f(\VEC{x})\right| \dx{\mu(\VEC{x})}
< \infty \ .
\]

Given any two multi-indices $\VEC{\alpha}$ and $\VEC{\beta}$, to prove that
$\displaystyle \VEC{z}^{\VEC{\alpha}} \diff^{\VEC{\beta}} \hat{f}(\VEC{z})$ is
uniformly bounded for $\|\IM \VEC{z} \|_2 \leq c$, we use the relation
\[
\VEC{z}^{\VEC{\alpha}} \diff^{\VEC{\beta}} \hat{f}(\VEC{z}) =
\VEC{z}^{\VEC{\alpha}} \big( (-i \VEC{x})^{\VEC{\beta}} f\big)^{\wedge} (\VEC{z})
= (-i)^{|\VEC{\beta}|} \VEC{z}^{\VEC{\alpha}}
\left(\VEC{x}^{VEC{\beta}} f \right)^{\wedge}(\VEC{z})
\]
to reduce the problem to the previous case with
$\displaystyle \VEC{x}^{\VEC{\beta}} f$
instead of $f$ where $\displaystyle \VEC{x}^{\VEC{\beta}} f \in \DD(\RR^n)$.
\end{proof}

The Fourier transform can only be defined on a subset of the space
$\displaystyle \DD'(\RR^n)$ of distributions on $\displaystyle \RR^n$.

\begin{prop} \label{DD_S_IncDense}
\begin{enumerate}
\item $\displaystyle \DD(\RR^n) \subset \SS(\RR^n)$ and the inclusion
$\displaystyle i:\DD(\RR^n) \subset \SS(\RR^n)$ is continuous.
\item $\displaystyle \DD(\RR^n)$ is dense in $\displaystyle \SS(\RR^n)$.
\end{enumerate}
\end{prop}

In the statement of the previous proposition, the topology on
$\displaystyle \DD(\RR^n)$ is the topology defined in
Section~\ref{SectTestFnct} and the topology on
$\displaystyle \SS(\RR^n)$ is the locally convex topology that we
have mentioned above.

If $u$ is a continuous linear functional on $\displaystyle \SS(\RR^n)$, then
$u \circ i$ is a continuous linear functional on
$\displaystyle \DD(\RR^n)$; namely,
$u\circ i \in \DD'(\RR)$.  Since $\displaystyle \DD(\RR^n)$ is dense in
$\displaystyle \SS(\RR^n)$, each continuous linear functional on
$\displaystyle \SS(\RR^n)$ is associated in this manner to a unique
element of $\displaystyle \DD'(\RR^n)$.
The {\bfseries tempered distributions}\index{Distribution!Tempered}
on $\displaystyle \RR^n$ are the distributions on $\displaystyle \RR^n$
associated to continuous linear functionals on
$\displaystyle \SS(\RR^n)$.  For this reason, the space of all the tempered
distributions is denoted $\displaystyle \SS'(\RR^n)$.  Another way to
look at the tempered distributions is to say that
$\displaystyle u \in \DD'(\RR^n)$ is a
tempered distribution if it can be extended to a continuous linear
functional on $\displaystyle \SS(\RR^n)$.

\begin{egg}
Here are a couple of examples of tempered distributions.

If $\nu$ is a positive measure on $\displaystyle \RR^n$ such that
$\displaystyle \int_{\RR^n} (1 + \|\VEC{x}\|^2)^{-k} \dx{\nu} < \infty$
for some positive integer $k$, then $\nu$ defines a tempered
distribution on $\displaystyle \RR^n$.  More precisely,
$\displaystyle u : \SS(\RR^n) \to \CC$ defined by
$\displaystyle u(\phi) = \int_{\RR^n} \phi(\VEC{x}) \dx{\nu(\VEC{x})}$
for $\displaystyle \phi\in \SS(\RR^n)$ is a temperate distribution on
$\displaystyle \RR^n$.

If $\displaystyle g:\RR^n \to \RR$ is a measurable functions such that
$\displaystyle \int_{\RR^n} \left| (1+\|\VEC{x}\|^2)^{-k} g(\VEC{x})
\right|^p \dx{\VEC{x}} < \infty$ for a positive integer $k$ and a real
number $p \geq 1$, then $g$ defines a tempered distribution on
$\displaystyle \RR^n$.
More precisely, $\displaystyle u : \SS(\RR^n) \to \CC$ defined by
$\displaystyle u(\phi) = \int_{\RR^n} \phi(\VEC{x}) g(\VEC{x}) \dx{\VEC{x}}$
for $\displaystyle \phi\in \SS(\RR^n)$ is a temperate distribution on
$\displaystyle \RR^n$.

Every $\displaystyle g \in L^p(\RR^n)$ for $1 \leq p \leq \infty$
defines a tempered distribution on $\displaystyle \RR^n$.
\end{egg}

The following proposition states that all distributions with compact
support are also tempered distribution but there are more tempered
distributions.

\begin{prop}
$\displaystyle \EE'(\RR^n) \subset \SS'(\RR^n) \subset \DD'(\RR^n)$
\end{prop}

The Fourier transform of a rapidly decreasing function has the
following properties.

\begin{prop} \label{distr_frr_tempT}
\begin{enumerate}
\item The Fourier transform (\ref{distr_fourierT}) is a continuous
linear bijection from $\displaystyle \SS(\RR^n)$ to itself and its inverse is
continuous.
\item If $\phi$ and $\psi$ are in $\displaystyle \SS(\RR^n)$, then
$\displaystyle \phi \ast \psi \in \SS(\RR^n)$ and 
$\displaystyle (\phi\ast \psi)^\wedge = \hat{\phi}\,\hat{\psi}$.
\end{enumerate}
\end{prop}

The {\bfseries convolution}\index{Distribution!Convolution} of a
temperate distribution $\displaystyle u \in \SS'(\RR^n)$ with a
function $\displaystyle \phi \in \SS(\RR^n)$ is defined by
\[
  (u \ast \phi)(\VEC{x}) = \ps{u}{(\phi^\vee)_{\VEC{x}}}
\]
for all $\displaystyle \VEC{x} \in \RR^n$.
The {\bfseries Fourier transform}\index{Distribution!Fourier Transform}
of a tempered distribution $\displaystyle u \in \SS'(\RR^n)$ is the
distribution $\hat{u}$ defined by
\begin{equation} \label{distr_fourierTD}
\ps{\hat{u}}{\phi} = \ps{u}{\hat{\phi}} \quad , \quad \phi \in \SS(\RR^n) \ .
\end{equation}
As for the standard distributions, we use the notation
$\ps{u}{\phi} = u(\phi)$ for $\displaystyle u \in \SS'(\RR^n)$ and
$\displaystyle \phi \in \SS(\RR^n)$.

\begin{prop} \label{distr_frr_tempD}
\begin{enumerate}
\item The Fourier transform (\ref{distr_fourierTD}) is a continuous
linear bijection from $\displaystyle \SS'(\RR^n)$ into itself and its
inverse is continuous.
\item If $\displaystyle u \in \SS'(\RR^n)$, $\VEC{\alpha}$ is a multi-index,
and $p$ is the polynomial
$\displaystyle p(\VEC{x}) = (i\VEC{x})^{\VEC{\alpha}}$, then
$\displaystyle (\diff^{\VEC{\alpha}} u)^\wedge = p\, \hat{u}$ and
$\displaystyle (p^\vee u)^\wedge = \diff^{\VEC{\alpha}}\hat{u}$.
\item If $\displaystyle u \in \SS'(\RR^n)$ and
$\displaystyle \phi \in \SS(\RR^n)$, then
$\displaystyle u\ast \phi \in C^\infty(\RR^n)$ and
$\displaystyle u\ast \phi \in \SS'(\RR^n)$.
\item If $\displaystyle u \in \SS'(\RR^n)$ and
$\displaystyle \phi \in \SS(\RR^n)$, then
$\displaystyle (u\ast \phi)^\wedge = \hat{\phi}\,\hat{u}$ and
$\displaystyle \hat{u}\ast \hat{\phi} = (\phi\,u)^\wedge$.
\end{enumerate}
\end{prop}

\begin{proof}
We will only prove the first part of (2).  The reader should consult
the references for the proofs of the other results.

By definition, we have that
$\displaystyle (\diff^{\VEC{\alpha}} u)(\phi) = (-1)^{\VEC{\alpha}}
u(\diff^{\VEC{\alpha}} \phi)$
for all $\displaystyle \phi \in \SS(\RR^n)$.  Hence,
\[
(\diff^{\VEC{\alpha}} u)^\wedge (\phi)
= (\diff^{\VEC{\alpha}} u)(\hat{\phi})
= (-1)^{|\VEC{\alpha}|} u\big(\diff^{\VEC{\alpha}} \hat{\phi}\big)
= (-1)^{|\VEC{\alpha}|} u\big((-i\VEC{x})^{\VEC{\alpha}} \hat{\phi})
= (i\VEC{x})^{\VEC{\alpha}} u\big( \hat{\phi})
= (i\VEC{x})^{\VEC{\alpha}} \hat{u}(\phi)
\]
for all $\displaystyle \phi \in \SS(\RR^n)$.  Thus,
$\displaystyle (\diff^{\VEC{\alpha}} u)^\wedge = p \hat{u}$ in the sense of
distributions.
\end{proof}

\begin{egg}
The Dirac delta function $\delta$ is a distribution with compact
support, so we can compute its Fourier transform.  Since
\[
\ps{\hat{\delta}}{\phi} = \ps{\delta}{\hat{\phi}} = \hat{\phi}(\VEC{0})
= \int_{\RR^n} \phi(\VEC{x}) \dx{\mu(\VEC{x})} =
\ps{f}{\phi}
\]
for all $\displaystyle \phi \in \SS(\RR^n)$, where
$\displaystyle f \in \SS'(\RR^n)$ is the
tempered distribution defined by the measure $\mu$; namely,
$\displaystyle f(\phi) = \int_{\RR^n} \phi(\VEC{x}) \dx{\mu(\VEC{x})}
= \frac{1}{(2\pi)^{n/2}} \int_{\RR^n} \phi(x) \dx{\VEC{x}}$ for all
$\displaystyle \phi \in \SS(\RR^n)$, then $\hat{\delta} = f$.

Note that in the literature, it is often stated that $\hat{\delta} = 1$.
The reason behind the difference with our statement is that they do
not use the factor $\displaystyle (2\pi)^{-n/2}$ in their definition
of the Fourier transform as we have done with the measure $\mu$.
\end{egg}

\begin{egg}
Using Example~\ref{chia_delta}, it is also possible to prove that
$\hat{\delta} = \mu$ in $\DD'(\RR)$.   This is often the proof given
for this result.

Since the Fourier transform is continuous on $\SS'(\RR)$ and since
$\chi_a$ defined in Example~\ref{chia_delta} converges to $\delta$
in $\DD(\RR)$ as $a \to 0$, and so in $\SS(\RR)$ because of (1) in
Proposition~\ref{DD_S_IncDense}, we have that
$\displaystyle (\chi_a)^\wedge$ converges to $\hat{\delta}$ in $\SS'(\RR)$.

Since $\displaystyle \chi_a \in L^1(\RR)$, its Fourier transform is
given by
\[
(\chi_a)^\wedge(x) = \frac{1}{2a \sqrt{2\pi}}\int_{-a}^a e^{-ix y}\dx{y}
= \frac{-1}{2a \sqrt{2\pi}\, i x}\, e^{-i x y}\bigg|_{-a}^a
= \frac{\sin(a x)}{a x \sqrt{2\pi}}
\]
for $x \neq 0$ and
\[
(\chi_a)^\wedge(x) = \frac{1}{2a \sqrt{2\pi}}\int_{-a}^a \dx{y}
= \frac{1}{\sqrt{2\pi}}
\]
for $x = 0$.   Thus
$\displaystyle \lim_{a\to 0} (\chi_a)^\wedge(x) = 1/\sqrt{2\pi}$
for all $x \in \RR$.

Thus $\displaystyle \lim_{a\to 0} (\chi_s)^\wedge(x) = 1/\sqrt{2\pi}$
for all $x \in \RR$.  We have shown that
$\displaystyle (\chi_a)^\wedge$ converges pointwise to the function
$f:\RR\to \RR$ defined by $f(x) = 1/\sqrt{2\pi}$ for
all $x$.  We cannot immediately conclude that $\hat{\delta} = f$
because pointwise convergence is not the convergence in $\SS'(\RR)$.
However, if the pointwise limit exists, then the limit in $\SS'(\RR)$
must be equal almost everywhere to the pointwise limit.  Thus,
$\hat{\delta} = f$ almost everywhere as expected.
\end{egg}

\section{Fundamental Solutions}

Our presentation is based on \cite{FoPDE} and make used of the Fourier
transform.  This approach is analytic.  A more elementary and algebraic
presentation is given in \cite{Smo}.

Let $\displaystyle \Omega \subset \RR^n$ be an open set and consider
the linear partial differential equation
\begin{equation} \label{distr_lPDE2}
L(\VEC{x},\diff)u = \sum_{|\VEC{\alpha}|\leq m} a_{\VEC{\alpha}}
\diff^{\VEC{\alpha}} u = f
\end{equation}
where $u:\Omega \rightarrow \CC$, $f:\Omega \rightarrow \CC$ is
continuous and $a_{\VEC{\alpha}}:\Omega \rightarrow \CC$ is of class
$\displaystyle C^{|\VEC{\alpha}|}$ for all multi-indices involved in the sum.

As in the introduction of the present chapter, let
$\displaystyle L^\ast(\VEC{x}, \diff)$ 
be the adjoint of $L(\VEC{x}, \diff)$ defined by
\[
L^\ast(\VEC{x},\diff)\phi = \sum_{|\VEC{\alpha}|\leq m} (-1)^{|\VEC{\alpha}|}
\diff^{\VEC{\alpha}} \left(a_{\VEC{\alpha}} \phi \right) \quad , \quad \phi \in
\DD(\Omega) \ .
\]
A {\bfseries strong solution}\index{Strong Solution} or a
{\bfseries solution in the sense of distributions}\index{Solution in
the Sense of Distribution}
of (\ref{distr_lPDE2}) is a distribution $u$ on $\Omega$ such that
\[
\ps{L(\VEC{x},\diff)u}{\phi} = \ps{u}{L^\ast(\VEC{x},\diff)\phi} =
\ps{f}{\phi}
\]
for all $\phi \in \DD(\Omega)$.

Consider the linear partial differential equation with constant coefficients
\begin{equation} \label{distr_lPDE3}
L(\VEC{x},\diff)u = \sum_{|\VEC{\alpha}|\leq m} a_{\VEC{\alpha}}
\diff^{\VEC{\alpha}} u = f \  ,
\end{equation}
where $a_{\VEC{\alpha}} \in \RR$ for all multi-indices $\VEC{\alpha}$
involved in the sum, $\displaystyle u:\RR^n \rightarrow \CC$ and
$\displaystyle f \in \DD(\RR^n)$.

We assume that the coordinates are such that
$\displaystyle \{ \VEC{x} \in \RR^n : x_n=0 \}$ is not a characteristic surface.
This implies that $a_{(0,0,\ldots,0,m)} \neq 0$.  Without loss of
generality, we may even assume that $a_{(0,0,\ldots,0,m)} = 1$.
We show that there exists a solution $\displaystyle u \in C^\infty(\RR^n)$ for
(\ref{distr_lPDE3}).

We have that $\displaystyle L(\VEC{x},\diff)\hat{u} = (p\, u)^\wedge$, where
\begin{equation} \label{distr_lPDE3_ex1}
p(\VEC{y}) = \sum_{|\VEC{\alpha}|\leq m} a_{\VEC{\alpha}}
(-i\VEC{y})^{\VEC{\alpha}} \ .
\end{equation}
Let $\lambda_1(\breve{\VEC{y}})$, $\lambda_2(\breve{\VEC{y}})$, \ldots,
$\lambda_m(\breve{\VEC{y}})$ be the roots of
$\displaystyle g(z) = p(\breve{\VEC{y}}, z)$ for
$\displaystyle \breve{\VEC{y}}\in \RR^{n-1}$.
By Rouch\'e's theorem, the roots
$\displaystyle \lambda_i:\RR^{n-1} \rightarrow \CC$
are continuous functions.

\begin{lemma} \label{distr_lPDE3_ex2}
There exists a measurable function $\kappa:\RR^{n-1}\rightarrow [-m,m]$
such that
\[
\min_{\substack{\breve{\VEC{y}} \in \RR^{n-1}\\ 1\leq j \leq m}}
\left\{ \left|\kappa(\breve{\VEC{y}})
- \IM \lambda_j(\breve{\VEC{y}})\right| \right\} \geq 1 \ .
\]
\end{lemma}

\begin{proof}
Given $\displaystyle \breve{\VEC{y}} \in \RR^{n-1}$, there exists a
least one interval
$\displaystyle I_k = [2k-m-1, 2k-m+1]$ with $0\leq k \leq m$ such that
$\IM \lambda_j(\breve{\VEC{y}}) \not\in I_k$ for all $j$ because there
are $m+1$ intervals and at most $m$ distinct zeros.  Let
\[
V_k = \left\{ \breve{\VEC{y}} \in \RR^{n-1} :
\IM \lambda_j(\breve{\VEC{y}}) \not\in I_k \ \text{for} \ 1 \leq j
\leq m \right\}
\]
for $0 \leq k \leq m$,  We have that
$\displaystyle \RR^{n-1} = \bigcup_{1\leq k \leq m} V_k$ and
the $V_k$ are measurable because
\[
V_k = \bigcap_{1\leq j \leq m} \lambda_j^{-1}\big(\CC \setminus
(\RR + i\,[2k-m-1,2k-m+1])\big)
\]
and the $\lambda_j$ are continuous.

The function $\kappa$ is defined by
$\displaystyle \kappa(\breve{\VEC{y}}) = 2k-m$ if
$\displaystyle \breve{\VEC{y}} \in V_k \setminus \bigcup_{0\leq j<k} V_j$.
It is a measurable function because
$\displaystyle \kappa^{-1}\big(\{2k-m\}\big)
= V_k \setminus \bigcup_{0\leq j<k} V_j$
is a measurable set.
\end{proof}

\begin{lemma} \label{distr_lPDE3_ex3}
Let $\displaystyle q(z) = z^m + g(z)$, where $g$ is a polynomial of
degree less than $m$ and $g(0) \neq 0$.  Suppose that $\lambda_1$,
$\lambda_2$, \ldots, $\lambda_m$ are the roots of $q$.  Then,
$\displaystyle \left| q(0)\right| \geq \left(\frac{\lambda}{2}\right)^m$,
where $\displaystyle \lambda = \min_{1\leq j\leq m} |\lambda_j|$.
\end{lemma}

\begin{proof}
The factorization of $q$ is
$\displaystyle q(z) = (z-\lambda_1)(z-\lambda_2)\ldots(z-\lambda_m)$.
Hence,
\[
\left| q(0) \right| = \left| \prod_{j=1}^m \lambda_j \right|
= \prod_{j=1}^m \left| \lambda_j \right| \geq
\lambda^m \geq \left(\frac{\lambda}{2}\right)^m \ .  \qedhere
\]
\end{proof}

\begin{rmk}
A more convoluted but cute proof of the previous lemma was given in
\cite{FoPDE}.  The factorization of $q$ is
$\displaystyle q(z) = (z-\lambda_1)(z-\lambda_2)\ldots(z-\lambda_m)$.
Hence,
\[
\left| \frac{q(z)}{q(0)} \right| = \prod_{j=1}^m
\left| 1 - \frac{z}{\lambda_j} \right| \leq 2^m
\]
for $|z|= \lambda$ since $|z/\lambda_j| = \lambda/|\lambda_j| \leq 1$,
so $z/\lambda_j \in \overline{B_1(0)} \subset \CC$ and
$|z/\lambda_j - 1| \leq 2$.  Since $\displaystyle m! = q^{(m)}(z)$,
Cauchy Integral Formula gives
\begin{align*}
m! & = |q^{(m)}(0)| = \left| \frac{m!}{2\pi i}
\int_{|z|=\lambda} \frac{q(z)}{z^{m+1}} \dx{z} \right|
= \frac{m!}{2\pi} \left| \int_0^{2\pi}
\frac{q(\lambda e^{i\theta})}{(\lambda e^{i\theta})^{m+1}}
\lambda i e^{i\theta} \dx{\theta}\right| \\
& \leq \frac{m!}{2\pi \lambda^m} \int_0^{2\pi}
|q(\lambda e^{i\theta})| \dx{\theta}
\leq \frac{m!\, 2^m |q(0)|}{2\pi \lambda^m} \int_0^{2\pi} \dx{\theta}
= \frac{m!\, 2^m |q(0)|}{\lambda^m} \ .
\end{align*}
Thus, $\displaystyle \left(\frac{\lambda}{2}\right)^m \leq |q(0)|$.
\end{rmk}

\begin{theorem} \label{distr_lPDE3_ex4}
There exists a solution of class $\displaystyle C^\infty$ for
(\ref{distr_lPDE3}).
\end{theorem}

\begin{proof}
Let
\[
u(\VEC{x}) = (2\pi)^{-n/2} \int_{\RR^{n-1}}
\int_{\IM y_n = \kappa(\breve{\VEC{y}})}
e^{-i\, \VEC{x}\cdot\VEC{y}}
\frac{\hat{f}(\VEC{y})}{p(\VEC{y})} \dx{y_n}\dx{\breve{\VEC{y}}}
\]
for $\displaystyle \VEC{x} \in \RR^n$, where $\kappa$ is defined in
Lemma~\ref{distr_lPDE3_ex2} and $p$ is defined in
(\ref{distr_lPDE3_ex1}).  In the previous integral, we
assume that $y_n \in \CC$.

\stage{i} Given $\displaystyle \VEC{y} \in \RR^n$, we first prove that
$\displaystyle |p(\VEC{y})| \geq 2^{-m}$ for
$\IM y_n = \kappa(\breve{\VEC{y}})$.
Let $q(z) = p(\breve{\VEC{y}}, y_n+z)$.
From Lemma~\ref{distr_lPDE3_ex3}, we have that
\[
\left| p(\VEC{y}) \right| = \left| q(0) \right| \geq
\left(\frac{\lambda}{2}\right)^m \ ,
\]
where
$\displaystyle
\lambda = \min_{1\leq j\leq m} \left\{ \left|\lambda_j\right| : \lambda_j
\ \text{is a root of} \ q \right\}$.
However, the roots of the present polynomial $q$ are
$\lambda_j(\breve{\VEC{y}}) - y_n$, where $\lambda_j(\breve{\VEC{y}})$ for
$1\leq j \leq m$ are the roots of the polynomial
$\displaystyle g(z) = p(\breve{\VEC{y}}, z)$. Hence,
for $\IM y_n = \kappa(\breve{\VEC{y}})$, we get from
Lemma~\ref{distr_lPDE3_ex2} that
\[
\lambda = \min_{1\leq j \leq m}
\left| \lambda_j(\breve{\VEC{y}}) - y_n \right|
\geq \min_{1\leq j \leq m}
\left|\IM \lambda_j(\breve{\VEC{y}}) - \kappa(\breve{\VEC{y}}) \right|
\geq 1 \ .
\]

\stage{ii}
From Proposition~\ref{distr_four_f}, $\hat{f}$ is rapidly decreasing
on the set $\displaystyle \{\VEC{y} \in \CC^n : \|\IM \VEC{y}\|_2 \leq m \}$
that contains the domain of integration
$\displaystyle \{ \VEC{y} = (\breve{\VEC{y}}, y_n) \in \RR^{n-1}\times \CC :
\IM y_n = \kappa(\breve{\VEC{y}}) \}$ because
$|\IM y_n| \leq m$ since the range of $\kappa$ is $[-m.m]$.

It follows from (i) and (ii) that $\displaystyle
\VEC{y} \mapsto e^{-i\, \VEC{x}\cdot\VEC{y}} \frac{\hat{f}(\VEC{y})}{p(\VEC{y})}$
is bounded and rapidly decreasing on the domain of integration
\[
S = \{\VEC{y} = (\breve{\VEC{y}}, y_n) \in \RR^{n-1}\times \CC :
|y_n| \leq m \} \ .
\]
Since $|\IM y_n|\leq m$ in the domain of integration, the
exponential in the definition of $u$ is bounded by $\displaystyle e^{m x_n}$.
It follows that
$\displaystyle \VEC{y} \mapsto \diff_{\VEC{x}}^{\VEC{\alpha}}
e^{-i\, \VEC{x}\cdot\VEC{y}} \frac{\hat{f}(\VEC{y})}{p(\VEC{y})}$
is rapidly decreasing on $S$ for all multi-indices $\VEC{\alpha}$.
Hence, we may interchange derivative and integral in the definition of
$u$.  Thus $u$ is of class $\displaystyle C^\infty$.

We have
\begin{align}
(L(\VEC{x},\diff)u)(\VEC{x}) &= 
(2\pi)^{-n/2} \int_{\RR^{n-1}} \int_{\IM y_n = \kappa(\breve{\VEC{y}})}
\left( L(\VEC{x},\diff) e^{-i\, \VEC{x}\cdot\VEC{y}} \right)
\frac{\hat{f}(\VEC{y})}{p(\VEC{y})} \dx{y_n}\dx{\breve{\VEC{y}}}
\nonumber \\
&= (2\pi)^{-n/2} \int_{\RR^{n-1}} \int_{\IM y_n = \kappa(\breve{\VEC{y}})}
e^{-i\, \VEC{x}\cdot\VEC{y}} \hat{f}(\VEC{y}) \dx{y_n}\dx{\breve{\VEC{y}}}
\label{distrlPDE3eq1}
\end{align}
for $\displaystyle \VEC{x} \in \RR^n$.

Since $\displaystyle y_n \mapsto e^{i\VEC{x}\cdot\VEC{y}} \hat{f}(\VEC{y})$ is a
entire function, we may use Cauchy's theorem for
integrals along a close path to conclude that
\[
\int_{y_n \in \Gamma}
e^{-i\, \VEC{x}\cdot\VEC{y}} \hat{f}(\VEC{y}) \dx{y_n} = 0 \ ,
\]
where $\Gamma = \Gamma_1 + \Gamma_2 - \Gamma_3 - \Gamma_4$ with
$\Gamma_1 = \{ r+ 0 i : -R \leq r \leq R \}$,
$\Gamma_2 = \{ R+ s i : 0 \leq s \leq \kappa(\breve{\VEC{y}}) \}$,
$\Gamma_3 = \{ r+ \kappa(\breve{\VEC{y}}) i : -R \leq r \leq R \}$
and
$\Gamma_4 = \{ -R+ s i : 0 \leq s \leq \kappa(\breve{\VEC{y}}) \}$
(Figure~\ref{distr_FIG1}).
We have that
\[
\int_{y_n \in \Gamma_1}
e^{-i\, \VEC{x}\cdot\VEC{y}} \hat{f}(\VEC{y}) \dx{y_n}
\to
\int_{-\infty}^{\infty}
e^{-i\, \VEC{x}\cdot\VEC{y}} \hat{f}(\VEC{y}) \dx{y_n}
\quad \text{as} \quad R \to \infty
\]
and
\[
\int_{y_n \in \Gamma_3}
e^{-i\, \VEC{x}\cdot\VEC{y}} \hat{f}(\VEC{y}) \dx{y_n}
\to 
\int_{\IM y_n = \kappa(\breve{\VEC{y}})}
e^{-i\, \VEC{x}\cdot\VEC{y}} \hat{f}(\VEC{y}) \dx{y_n}
\quad \text{as} \quad R \to \infty \ .
\]
Moreover, since
$\displaystyle \VEC{y} \mapsto 2^{i\VEC{x}\cdot\VEC{y}} \hat{f}(\VEC{y})$ is
rapidly decreasing on the set $S$, we have that there exists a
constant $C>0$ such that
$\displaystyle \left| 2^{i\VEC{x}\cdot\VEC{y}} \hat{f}(\VEC{y}) \right| \leq
C/ \|\VEC{y}\|_2^2$ for all $\VEC{z} \in S$.  Thus, for
$\displaystyle R^2>2m^2$, we have that
\begin{align*}
\left| \int_{y_n \in \Gamma_2}
e^{-i\, \VEC{x}\cdot\VEC{y}} \hat{f}(\VEC{y}) \dx{y_n} \right|
& \leq C \int_0^{\kappa(\breve{\VEC{y}})}
\frac{1}{\|(\breve{\VEC{y}},R+is)\|_2^2} \dx{s}
\leq C \int_0^m \frac{1}{ R^2 - s^2} \dx{s} \\
& \leq C \int_0^m \frac{1}{ R^2 - m^2} \dx{s}
= \frac{2Cm}{R^2} \to 0
\quad \text{as} \quad R \to \infty
\end{align*}
and similarly
\begin{align*}
\left| \int_{y_n \in \Gamma_4}
e^{-i\, \VEC{x}\cdot\VEC{y}} \hat{f}(\VEC{y}) \dx{y_n} \right|
&\leq C \int_0^{\kappa(\breve{\VEC{y}})}
\frac{1}{\|(\breve{\VEC{y}},-R+is)\|_2^2} \dx{s}
\leq C \int_0^m \frac{1}{ R^2 - s^2} \dx{s} \\
& \leq C \int_0^m \frac{1}{ R^2 - m^2} \dx{s}
= \frac{2Cm}{R^2} \to 0
\quad \text{as} \quad R \to \infty \ .
\end{align*}
Hence,
\[
\int_{\IM y_n = \kappa(\breve{\VEC{y}})}
e^{-i\, \VEC{x}\cdot\VEC{y}} \hat{f}(\VEC{y}) \dx{y_n}
= 
\int_{-\infty}^{\infty}
e^{-i\, \VEC{x}\cdot\VEC{y}} \hat{f}(\VEC{y}) \dx{y_n} \ .
\]
It follows from (\ref{distrlPDE3eq1}) that
\[
(L(\VEC{x},\diff)u)(\VEC{x})
= \int_{\RR^n} e^{-i\, \VEC{x}\cdot\VEC{y}} \hat{f}(\VEC{y}) \dx{\mu(\VEC{y})}
= f(\VEC{x})
\]
for $\displaystyle \VEC{x} \in \RR^n$, where we have used the Formula for the
Inverse Fourier transform.
\end{proof}

\pdfF{distributions/distr_fig1}{Contour of integration}{
According to Cauchy's theorem for integrals along a
close path, we have that
$\displaystyle \int_{\Gamma} = \int_{\Gamma_1} + \int_{\Gamma_2}
- \int_{\Gamma_3} - \int_{\Gamma_4} = 0$.}{distr_FIG1}

\begin{defn}
A {\bfseries fundamental solution}\index{Fundamental Solution} for the
differential operator $L(\VEC{x},\diff)$ defined in (\ref{distr_lPDE3}) is a
distribution $F$ on $\displaystyle \RR^n$ such that
$L(\VEC{x},\diff)F = \delta$.
\end{defn}

\begin{egg}
Let $R = \{\VEC{x} \in \RR^2 : x_i > 0 \ \text{for} \ i=1,2 \}$ and
$\displaystyle L(\VEC{x},\diff) = \diff^{(1,1)}
= \pdydxnm{}{x_1}{x_2}{2}{}{}$.  We have that
$\Chi_R$ is a fundamental solution of the differential operator
$L(\VEC{x},\diff)$ because
\begin{align*}
&\ps{L(\VEC{x},\diff) \Chi_R}{\phi}
= \ps{\Chi_R}{\diff^{(1,1)} \phi}
= \int_0^\infty \int_0^\infty \pdydxnm{\phi}{x_1}{x_2}{2}{}{}(x_1,x_2)
\dx{x_2}\dx{x_1} \\
&\qquad 
= \int_0^\infty \left( \pdydx{\phi}{x_1}(x_1,x_2) \bigg|_0^\infty\right)\dx{x_1}
= -\int_0^\infty \pdydx{\phi}{x_1}(x_1,0) \dx{x_1}
= -\phi(x_1,0) \bigg|_0^\infty = \phi(0,0) = \delta(\phi)
\end{align*}
for all $\phi \in \DD(\RR^2)$.
\end{egg}

\begin{rmk}
If $\displaystyle F\in \DD'(\RR^n)$ is a fundamental solution for
$L(\VEC{x},\diff)$, then a solution of $L(\VEC{x},\diff)u = f$, where
$\displaystyle f \in \DD(\RR^n)$, is the 
distribution $u = F \ast f$ because
\[
L(\VEC{x},\diff) u = L(\VEC{x},\diff)(F \ast f)
= (L(\VEC{x},\diff) F) \ast f = \delta \ast f = f \ .
\]
\end{rmk}

\begin{theorem}[Malgrange-Ehrempreis] \label{malgr_Ehr}
Every differential linear operator with constant coefficients
\begin{equation} \label{distr_lPDE4}
L(\VEC{x},\diff) = \sum_{|\VEC{\alpha}|\leq m} a_{\VEC{\alpha}} \diff^{\VEC{\alpha}}
\end{equation}
has a fundamental solution on $\displaystyle \RR^n$.
\index{Malgrange-Ehrempreis' Theorem}
\end{theorem}

\begin{proof}
We define a linear functional $F$ on $\displaystyle \DD(\RR^n)$ by
\[
F(\phi) = (2\pi)^{-n/2} \int_{\RR^{n-1}} \int_{\IM y_n = \kappa(\breve{\VEC{y}})}
\frac{\hat{\phi}^\vee(\VEC{y})}{p(\VEC{y})}\dx{y_n}\dx{\breve{\VEC{y}}}
\quad , \quad \phi \in \DD(\RR^n) \ ,
\]
where $\kappa$ is defined in Lemma~\ref{distr_lPDE3_ex2} and $p$ is
defined in (\ref{distr_lPDE3_ex1}).  In the previous integral, we
assume that $y_n \in \CC$.  To prove that
$\displaystyle F \in \DD'(\RR^n)$, we have
to show that $F$ is continuous on
$\displaystyle \DD(\RR^n)$.  We will also show
that $L(\VEC{x},\diff)F = \delta$.

\stage{i} Let $\displaystyle K \subset \RR^n$ be a compact set.  We will find a
constant $C_K$ and an integer $N_K$, such that
\[
\left|F(\phi)\right| \leq C_K \sum_{|\VEC{\alpha}|\leq N_K} \|\diff^{\VEC{\alpha}}
\phi\|_{\infty,K}
\]
for all $\phi \in \DD_K$.
It will follow from Proposition~\ref{distr_cont_cond} that $F$ is
continuous on $\displaystyle \DD(\RR^n)$ and so
$\displaystyle F\in \DD'(\RR^n)$.

As in the proof of Theorem~\ref{distr_lPDE3_ex4}, we can use
Proposition~\ref{distr_four_f} to conclude that $\hat{\phi}$ is
rapidly decreasing on the set
$\displaystyle \{\VEC{y} \in \CC^n : \|\IM \VEC{y}\|_2 \leq m \}$.
Moreover, we have shown in the proof of Theorem~\ref{distr_lPDE3_ex4} that
$\displaystyle \left|p(\VEC{y})\right| \geq 2^{-m}$ for
$\IM y_n = \kappa(\breve{\VEC{y}})$.  Hence, 
\begin{align*}
&\left| F(\phi) \right|
\leq \frac{1}{C_1(2\pi)^{n/2}}
\int_{\RR^{n-1}} \int_{-\infty}^\infty
\left| \hat{\phi}^\vee(\breve{\VEC{y}},r + \kappa(\breve{\VEC{y}})\,i) \right|
\dx{r}\dx{\breve{\VEC{y}}} \\
&\quad\leq \bigg( \underbrace{ \frac{1}{C_1\,(2\pi)^{n/2}}
\int_{\RR^{n-1}} \int_{-\infty}^\infty
\left( 1+\|\breve{\VEC{y}}\|^2 + r^2\right)^{-n-1} \dx{r}
  \dx{\breve{\VEC{y}}}}_{=C_2} \bigg)
\, \sup_{\|\IM \VEC{y}\| \leq m } \left\{
\left( 1+ \|\VEC{y}\|^2  \right)^{n+1}
\left| \hat{\phi}^\vee(\VEC{y}) \right| \right\} \\
&\quad = C_2 \sup_{\|\IM \VEC{y}\| \leq m} \left\{
\left( 1+\|\VEC{y}\|^2 \right)^{n+1}
\left| \hat{\phi}^\vee(\VEC{y}) \right| \right\}
\leq C_3 \sum_{|\VEC{\alpha}|\leq 2(n+1)} \|\diff^{\VEC{\alpha}} \phi\|_\infty
\end{align*}
for all $\phi \in \DD_K$ and some constant $C_3$.
The last inequality above is proved using the relation
\[
\left| \VEC{y}^{\VEC{\alpha}} \hat{\phi}(\VEC{y}) \right|
\leq e^{r\|\IM \VEC{y}\|_2}\,\int_{\RR^n}
\left|\diff^{\VEC{\alpha}} \phi(\VEC{x})\right| \dx{\mu(\VEC{x})}
\]
for $\displaystyle \VEC{y} \in \CC^n$ and
$\displaystyle \VEC{\alpha} \in \NN^n$ that we have obtained in
(\ref{proof_distr_four_f}) of the proof of Proposition~\ref{distr_four_f},
where $r$ is such that $K \subset B_r(\VEC{0})$.
Thus, the constant $C_3$ depends on $K$.

\stage{ii} We have
$\displaystyle \ps{L(\VEC{x},\diff) F}{\phi}
= \ps{F}{L^\ast(\VEC{x},\diff) \phi}$
for $\phi \in \DD(\RR^n)$ where \\
$\displaystyle L^\ast(\VEC{x},\diff)
= \sum_{|\VEC{\alpha}|\leq m} (-1)^{|\VEC{\alpha}|} a_{\VEC{\alpha}}
\diff^{\VEC{\alpha}}$.
Let $\displaystyle p^\ast(\VEC{y}) = \sum_{|\VEC{\alpha}|\leq m}
a_{\VEC{\alpha}} (i \VEC{y})^{\VEC{\alpha}}$.
Then
\[
((L^\ast(\VEC{x},\diff) \phi))^\wedge(-\VEC{y})
= p^\ast(-\VEC{y}) \hat{\phi}(-\VEC{y})
= p(\VEC{y}) \hat{\phi}(-\VEC{y})
\]
for $\displaystyle \VEC{y}\in \RR^n$.  Hence
\begin{align*}
\ps{L(\VEC{x},\diff)F}{\phi} &= \ps{F}{L^\ast(\VEC{x},\diff)\phi}
= (2\pi)^{-n/2} \int_{\RR^{n-1}} \int_{\IM y_n = \kappa(\breve{\VEC{y}})}
\frac{(L^\ast(\VEC{x},\diff)\phi)^\wedge(-\VEC{y})}{p(\VEC{y})}
\dx{y_n}\dx{\breve{\VEC{y}}} \\
&= (2\pi)^{-n/2} \int_{\RR^{n-1}} \int_{\IM y_n = \kappa(\breve{\VEC{y}})}
\hat{\phi}(-\VEC{y}) \dx{y_n}\dx{\breve{\VEC{y}}}
= (2\pi)^{-n/2} \int_{\RR^n} \hat{\phi}(-\VEC{y}) \dx{\VEC{y}} \\
&= \int_{\RR^n} \hat{\phi}(\VEC{y}) \dx{\mu(\VEC{y})}
= \phi(\VEC{0}) = \delta(\phi)
\end{align*}
for all $\displaystyle \phi \in \DD(\RR^n)$.  The fourth equality comes from
the Cauchy's theorem for integral along a close path as we have done 
in Theorem~\ref{distr_lPDE3_ex4} for instance.
Note that $\hat{\phi}$ is the restriction of an entire function which
is rapidly decreasing on the set
$\displaystyle \{ \VEC{y} \in \CC^n : \|\IM y\|_2 \leq m \}$ according to 
Proposition~\ref{distr_four_f}.  The second to last
equality is a consequence of the inverse Fourier transform.
\end{proof}

\begin{defn}
Let $\Omega$ be an open subset of $\displaystyle \RR^n$.  The
differential operator
$\displaystyle L(\VEC{x},\diff)u = \sum_{|\VEC{\alpha}|\leq m}
a_{\VEC{\alpha}} \diff^{\VEC{\alpha}} u$,
where the $\displaystyle a_{\VEC{\alpha}}\in C^\infty(\Omega)$, is
{\bfseries hypoelliptic}%
\index{Linear Partial Differential Operator!Hypoelliptic} if
$\displaystyle L(\VEC{x},\diff)u \in C^\infty(\Omega)$ implies that
$\displaystyle u \in C^\infty(\Omega)$.
\end{defn}

Recall that we say that $u \in \DD'(\Omega)$ is of class
$\displaystyle C^\infty(\Omega)$ and write
$\displaystyle u \in C^\infty(\Omega)$ if there exists
a function $\displaystyle f \in C^\infty(\Omega)$ such that
$\displaystyle u(\phi) = \int_{\Omega} f(x) \phi(x) \dx{x}$
for all $\phi \in \DD(\Omega)$.

The next lemma will be used to give a sufficient condition to
determine if a differential operator is hypoelliptic.

\begin{lemma} \label{distr_lemma_hypo}
If $\displaystyle f \in \DD'(\RR^n)$ is given by
$\displaystyle f\in C^\infty(\RR^n\setminus \{\VEC{0}\})$
and $\displaystyle v \in \EE'(\RR^n)$, then
$\displaystyle f\ast v \in C^\infty(\RR^n\setminus \supp v)$.
\end{lemma}

\begin{proof}
Consider $\VEC{z} \not\in \supp v$ and choose $\epsilon >0$ such that
$\displaystyle B_\epsilon(\VEC{z}) \cap \supp v = \emptyset$.
Choose $\displaystyle \psi \in \DD(\RR^n)$ such that $\psi(\VEC{x}) = 1$ for
$\displaystyle \VEC{x} \in B_{\epsilon/4}(\VEC{0})$ and
$\displaystyle \supp \psi \subset B_{\epsilon/2}(\VEC{0})$.

Then $f\ast v = (\psi f)\ast v + ( (1-\psi)f) \ast v$ where
$\displaystyle ((1-\psi) f) \ast v \in C^\infty(\RR^n)$ according to
item (3) of Proposition~\ref{distr_smooth_convol} and
Proposition~\ref{uvequvu} because
$\displaystyle (1-\psi) f \in C^\infty(\RR^n)$.  Moreover,
\[
\supp \big( (\psi f)\ast v\big) \subset D = \left\{ \VEC{x} + \VEC{y}
: \VEC{x} \in \supp \psi \ , \ \VEC{y} \in \supp v \right\}
\]
and $\displaystyle D \cap B_{\epsilon/2}(\VEC{z}) = \emptyset$.
Hence $((\phi f)\ast v)(\phi) = 0$ for all
$\displaystyle \phi \in \DD\left(B_{\epsilon/2}(\VEC{z})\right)$.
Thus $\displaystyle f\ast v = ((1-\phi)f) \ast v$ on
$\displaystyle \DD\left(B_{\epsilon/2}(\VEC{z})\right)$ and so
$\displaystyle f\ast v \in C^\infty\left(B_{\epsilon/2}(\VEC{z})\right)$.

Since $\VEC{z} \not\in \supp v$ is arbitrary, we get
$\displaystyle f\ast v \in C^\infty(\RR^n\setminus \supp v)$.
\end{proof}

\begin{theorem} \label{distr_hypoTH}
Consider the differential operator
\[
L(\VEC{x},\diff)u = \sum_{|\VEC{\alpha}|\leq m} a_{\VEC{\alpha}}
\diff^{\VEC{\alpha}} u \ ,
\]
where $a_{\VEC{\alpha}} \in \RR$ for all multi-indices $\VEC{\alpha}$.
The following sentences are equivalent.
\begin{enumerate}
\item There exists a fundamental solution of $L(\VEC{x},\diff)$ of
class $\displaystyle C^\infty$ on $\displaystyle \RR^n\setminus \{\VEC{0}\}$.
\item Every fundamental solution of $L(\VEC{x},\diff)$ is of class
$\displaystyle C^\infty$ on $\displaystyle \RR^n\setminus \{\VEC{0}\}$.
\item $L(\VEC{x},\diff)$ is hypoelliptic.
\end{enumerate}
\end{theorem}

\begin{proof}
\stage{$\mathbf{(3)\Rightarrow (2)}$}
If $F$ is a fundamental solution, then
$\displaystyle L(\VEC{x},\diff)F = \delta \in C^\infty(\RR^n
\setminus \{\VEC{0}\})$ and (3)
implies that $\displaystyle F \in C^\infty(\RR^n \setminus \{\VEC{0}\})$.

\stage{$\mathbf{(2) \Rightarrow (1)}$}
This is trivial because of Theorem~\ref{malgr_Ehr}.

\stage{$\mathbf{(1) \Rightarrow (3)}$}
Let $\displaystyle F \in C^\infty(\RR^n\setminus \{\VEC{0}\}$ be a fundamental
solution of $L(\VEC{x},\diff)$.  Suppose that $u\in \DD'(\Omega)$ is such that
$\displaystyle L(\VEC{x},\diff)u \in C^\infty(\Omega)$.

Given $\VEC{z} \in \Omega$ and $\epsilon >0$ such that
$\displaystyle B_\epsilon(\VEC{z}) \subset \Omega$ , we prove that
$\displaystyle u \in C^\infty(B_{\epsilon/2}(\VEC{z}))$.  Choose
$\displaystyle \phi \in \DD(\RR^n)$ such that
$\displaystyle \supp \phi \subset B_\epsilon(\VEC{z})$ and
$\phi(\VEC{x})=1$ for all $\displaystyle \VEC{x} \in B_{\epsilon/2}(\VEC{z})$.

We have
\[
L(\VEC{x},\diff)(\phi u) = \phi\, L(\VEC{x},\diff)u + v \  ,
\]
where $v$ is an expression involving the partial derivatives of $\phi$
(but not $\phi$ itself) and the partial derivatives of $u$ of order
less than $m$.

It follows from item (1) of Proposition~\ref{distr_smooth_convol} that
$\displaystyle F\ast (\phi \, L(\VEC{x},\diff)u) \in C^\infty(\RR^n)$ because
$\displaystyle \phi\,L(\VEC{x},\diff)u \in \DD(\RR^n)$.

We have that $v(\phi) = 0$ for all $\displaystyle \phi \in \DD(\RR^n)$ with
$\displaystyle \supp \phi \subset \overline{B_{\epsilon/2}(\VEC{z})}
\cup \left( \RR^n \setminus B_\epsilon(\VEC{z}) \right)$
because $\phi(\VEC{x})=0$ for
$\displaystyle \VEC{x} \not\in B_\epsilon(\VEC{z})$ and $\phi(\VEC{x})=1$ for
$\displaystyle \VEC{x} \in B_{\epsilon/2}(\VEC{z})$.  Hence, from
Lemma~\ref{distr_lemma_hypo}, we have that
$\displaystyle F\ast v \in C^\infty(B_{\epsilon/2}(\VEC{z}))$ 
because $\displaystyle \supp v \subset \overline{B_\epsilon(\VEC{z})} \setminus
B_{\epsilon/2}(\VEC{z})$.

Therefore,
\begin{align*}
\phi u &= \delta \ast (\phi u) = \big(L(\VEC{x},\diff) F\big) \ast (\phi u)
= F \ast \big( L(\VEC{x},\diff)(\phi u) \big) \\
&= F \ast (\phi L(\VEC{x},\diff) u)
+ F\ast v \in C^\infty(B_{\epsilon/2}(\VEC{z}))
\end{align*}
and, since $\phi(\VEC{x}) = 1$ for
$\displaystyle \VEC{x} \in B_{\epsilon/2}(\VEC{z})$, we get that
$\displaystyle u \in C^\infty(B_{\epsilon/2}(\VEC{z}))$.
\end{proof}

\section{Exercises}

We only suggest elementary problems about distribution and the Fourier
transform of functions.  For more advanced problems; in particular
concerning the Fourier transform of distributions, we suggest
\cite{ReeSim,RuFA}.

The remark made in the exercise section of Chapter~\ref{ChapShock} is
worth repeating here.  Namely, some of the problems below refer to
weak solutions of partial differential equations which for us means
solutions in the sense of distribution or strong solutions. 
The expression ``weak solutions'' is reserved to solutions of
variational problems that we will study in Chapter~\ref{elliptic_PDEs}.

Suggested exercises:

\begin{itemize}
\item In \cite{J}: numbers 1 in Section 3.6 ; all the numbers
in Section 1.9.
\item In \cite{McO}: numbers 3 to 5, 7 to 9 in Sections 2.3.
\item In \cite{Str}: all the numbers in Sections 12.1 and 12.4
(Warning: the definition of Fourier transform in \cite{Str} is
slightly different than the definition given above).
\end{itemize}


%%% Local Variables: 
%%% mode: latex
%%% TeX-master: "notes"
%%% End:
