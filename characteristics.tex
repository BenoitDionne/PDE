\chapter{Method of Characteristics}\label{ChapCaract}

\section[Quasi-Linear First Order Equations]{Quasi-Linear First Order
Partial Differential Equations}

Consider the {\bfseries partial differential equation}
\index{Partial Differential Equation}
\begin{equation}\label{char_QLPDE}
a(x,y,u)\,\pdydx{u}{x} + b(x,y,u)\,\pdydx{u}{y} = c(x,y,u) \ ,
\end{equation}
where $\displaystyle a:\RR^3\rightarrow \RR$ and $b:\RR^3\rightarrow \RR$ are
two given differentiable functions, and
$\displaystyle u:\RR^2\rightarrow \RR$ is an
unknown differentiable function.

Equations of the form (\ref{char_QLPDE}) are called
{\bfseries quasi-linear partial differential equations}%
\index{First Order Partial Differential Equation!Quasi-Linear}.
If $a$ and $b$ are independent of $u$, and
$c(x,y,u) = p(x,y) u + q(x,y)$ for some differentiable functions
$\displaystyle p:\RR^2\rightarrow \RR$ and
$\displaystyle q:\RR^2\rightarrow \RR$, then the
partial differential equation is called
{\bfseries linear}\index{First Order Partial Differential Equation!Linear}.

The goal is to find a differentiable solution, also called surface,
$u=u(x,y)$ that satisfy (\ref{char_QLPDE}) and contains a given curve
$\Gamma$ defined by
\[
\Gamma = \left\{ (x,y,u) = \left(x_0(s), y_0(s),u_0(s)\right) : s\in I \right\}
\]
for some differentiable functions $x_0$, $y_0$ and $u_0$ defined on 
an open interval $I$.  The problems where the
solutions (if any) of the partial differential equation must contain a
given curve are called
{\bfseries Cauchy problems}%
\index{First Order Partial Differential Equation!Cauchy Problem}.

\begin{defn}
A surface $u=u(x,y)$ satisfying (\ref{char_QLPDE}) and containing a
given curve $\Gamma$ is called an
{\bfseries integral surface}%
\index{First Order Partial Differential Equation!Integral Surface} for
(\ref{char_QLPDE}).
\end{defn}

We may rewrite (\ref{char_QLPDE}) as
$\displaystyle (a,b,c) \cdot \left(\pdydx{u}{x}, \pdydx{u}{y}, -1 \right) = 0$.
The vector $\displaystyle \left(\pdydx{u}{x}, \pdydx{u}{y}, -1 \right)$
evaluated at $(x,y)$ is perpendicular to the surface $u=u(x,y)$ at
$\displaystyle \left(x,y,u(x,y)\right)$.  Hence, the vector
$(a,b,c)$ evaluated at $(x,y,u(x,y))$ is in the tangent plan to the integral
surface $u=u(x,y)$ at the point $\displaystyle \left(x,y,u(x,y)\right)$.

If $\displaystyle \gamma = \left\{ (x(t),y(t),u(t)) : t \in \RR \right\}$
is a curve on the integral surface $u=u(x,y)$ then
$\left(x'(t),y'(t),u'(t)\right)$ is in the tangent
plane (more precisely, parallel to the tangent plane) to the integral
surface at the point $\left(x(t),y(t),u(t)\right)$.

If follows from the previous discussion that we should consider the
following family of curves.  For each fixed value of $s \in I$,
let $\gamma_s$ be the curve defined by
\[
\gamma_s = \left\{ (x,y,u) = \left(x(t,s), y(t,s), u(t,s) \right) : t \in \RR
\right\} \ ,
\]
where
\begin{equation} \label{char_CE}
\begin{split}
\pdydx{x}{t}(t,s) &= a(x(t,s),y(t,s),u(t,s)) \\
\pdydx{y}{t}(t,s) &= b(x(t,s),y(t,s),u(t,s)) \\
\pdydx{u}{t}(t,s) &= c(x(t,s),y(t,s),u(t,s))
\end{split}
\end{equation}
and
\begin{equation} \label{char_CE_init}
\left(x(0,s), y(0,s), u(0,s)\right) =
\left(x_0(s), y_0(s), u_0(s)\right) \ .
\end{equation}
For $s$ fix, (\ref{char_CE}) is a system of {\bfseries ordinary
differential equations}\index{Ordinary Differential Equation}
depending on a parameter $s$ and satisfying the initial condition
(\ref{char_CE_init}).  For this reason, in the literature, the partial
derivatives are often replaced by the derivatives
$\displaystyle \dydx{x}{t}$, $\displaystyle \dydx{y}{t}$ and
$\displaystyle \dydx{u}{t}$.

\begin{defn}
The equations in (\ref{char_CE}) are called
{\bfseries characteristic equations}%
\index{First Order Partial Differential Equation!Characteristic Equations}
and, for each $s$, the curve $\gamma_s$ is called a
{\bfseries characteristic curve}%
\index{First Order Partial Differential Equation!Characteristic Curve}.
\end{defn}

We end up with a family of curves
$\displaystyle \left\{ \gamma_s \right\}_{s\in I}$.  If we assume that
\begin{equation} \label{char_IFTQL_cond}
\frac{\partial(x,y)}{\partial(t,s)}\bigg|_{(t,s)} = \det
\begin{pmatrix}
\displaystyle \pdydx{x}{t} & \displaystyle \pdydx{x}{s} \\[0.7em]
\displaystyle \pdydx{y}{t} & \displaystyle \pdydx{y}{s}
\end{pmatrix}\bigg|_{(t,s)} \neq 0 \ ,
\end{equation}
then it follows from the Inverse Mapping Theorem that $s$ and $t$ may
be expressed (locally) in function of $x$ and $y$.  Substituting these
expressions of $s$ and $t$ in terms of $x$ and $y$ into $u=u(t,s)$
yields the equation of a surface $u=u(x,y)$ (Figure~\ref{char_fig1}).

To simplify the notation, we use $u$ to denote both the function of
$t$ and $s$, and the function of $x$ and $y$.  The context will
determine which one is used.  We have $u(x,y) = u(t,s)$ where
$x=x(t,s)$ and $y=y(t,s)$.

\pdfF{characteristics/char_fig1}{Characteristic curves}{Schematic
representation of the characteristic curves and the integral
surface.}{char_fig1}

\begin{theorem}
Let $\Gamma$ be a curve defined by
\[
\Gamma = \left\{ (x,y,u) = \left(x_0(s), y_0(s),u_0(s)\right) : s \in I
\right\} \ ,
\]
where $I$ is an open interval and $\displaystyle x_0, y_0, u_0 \in C_b^1(I)$.
For $s \in I$, let
\[
\gamma_s = \left\{ \left(x(t,s), y(t,s), u(t,s) \right) : t \in J_s \right\}
\]
be the {\bfseries orbit}\index{Ordinary Differential Equation!Orbit}
of the solution $t \mapsto \left(x(t,s), y(t,s), u(t,s) \right)$ 
of (\ref{char_CE}) with the initial condition
(\ref{char_CE_init}), where $J_s \subset \RR$ is the domain of the
solution.  If
\[
\det \begin{pmatrix}
a\left(x(t,s),y(t,s),u(t,s)\right) & \displaystyle \pdydx{x}{s}(t,s) \\[0.7em]
b\left(x(t,s),y(t,s),u(t,s)\right) & \displaystyle \pdydx{y}{s}(t,s)
\end{pmatrix} \neq 0
\]
for all $(t,s)$ \footnotemark, then the surface $S$ given by the
family of curves $\displaystyle \left\{ \gamma_s \right\}_{s\in I}$ is the
integral surface of (\ref{char_QLPDE}) containing the curve $\Gamma$.
There is only one integral surface containing $\Gamma$.
\label{qlfoPDEintsurf}
\end{theorem}

\footnotetext{From now on, we will often state ``for all
$t$ and $s$'' to mean ``for all $t$ and $s$ where $x(t,s)$, $y(t,s)$
and $u(t,s)$ are defined.''}

\begin{proof}
\stage{i} Given $(\tilde{t},\tilde{s}) \in J_s \times I$, it follows from
the theorems of existence and uniqueness of solutions for ordinary
differential equations that there exist two open intervals $I_b \subset I$
and $J_b \subset J_s$ such that $(\tilde{t},\tilde{s}) \in J_b \times I_b$
and the solutions $x$ and $y$ are defined and continuously
differrentiable on $J_b \times I_b$.  Since $J_b \times I_b$ is simply
connected\footnote{There is no ``holes'' in the surface.  The reader
may want to consult the chapter on homotopy theory in a good
textbook on topology.}, we have that the surface
\[
  S_b = \left(\{ \big( x(t,s),y(t,s),u(t,s)\big) :
    (t,s) \in J_b \times I_b \right\}
\]
is simply connected.  The integral surface $S$ is the union of the
surfaces $S_b$.  The surface $S$ is covered by at most a countable
collections of surfaces $S_b$.

\stage{ii} As we mentioned before, the existence of a differentiable
surface $u=u(x,y)$ comes from the Inverse Mapping Theorem because
\[
\frac{\partial(x,y)}{\partial(t,s)}\bigg|_{(\tilde{t},\tilde{s})} = \det
\begin{pmatrix}
\displaystyle \pdydx{x}{t} & \displaystyle \pdydx{x}{s} \\[0.7em]
\displaystyle \pdydx{y}{t} & \displaystyle \pdydx{y}{s}
\end{pmatrix}\bigg|_{(\tilde{t},\tilde{s})}
= \det \begin{pmatrix}
a\left(x(\tilde{t},\tilde{s}),y(\tilde{t},\tilde{s}),
 u(\tilde{t},\tilde{s})\right) &
\displaystyle \pdydx{x}{s}(\tilde{t},\tilde{s}) \\[0.7em]
b\left(x(\tilde{t},\tilde{s}),y(\tilde{t},\tilde{s}),
 u(\tilde{t},\tilde{s})\right) &
\displaystyle \pdydx{y}{s}(\tilde{t},\tilde{s})
\end{pmatrix} \neq 0
\]
by hypothesis.  We can write $s$ and $t$ in a neighbourhood of
$(\tilde{t},\tilde{s})$ as functions of $x$ and $y$.  Substituting
these expressions of $s$ and $t$ in $u=u(t,s)$ gives the equation
$u=u(x,y)$ to describe the surface $S$.

\stage{iii} This surface $u=u(x,y)$ contains $\Gamma$ by construction.
In fact, $\Gamma = \{ (x(0,s),y(0,s),u(0,s) : s \in I \}$.
Moreover, the points on the surface $S$ are of the form
\[
(x,y,u) = \left(x(t,s),y(t,s), u(t,s)\right)
= \left(x(t,s),y(t,s), u\big(x(t,s),y(t,s)\big)\right)
\]
for some value of $s$ and $t$.  Hence,
\begin{align*}
c\left(x,y, u\right)
&= c\left(x(t,s),y(t,s),u(t,s)\right) = \pdfdx{u(t,s)}{t} 
= \pdfdx{u(x(t,s),y(t,s))}{t} \\
&= \pdydx{u}{x}\left(x(t,s),y(t,s)\right) \pdydx{x}{t}\left(t,s\right)
+ \pdydx{u}{y}\left(x(t,s),y(t,s)\right) \pdydx{y}{t}\left(t,s\right) \\
&= a(x(t,s),y(t,s),u(t,s)) \,\pdydx{u}{x}\left(x(t,s),y(t,s)\right) \\
& \qquad
+ b(x(t,s),y(t,s),u(t,s)) \,\pdydx{u}{y}\left(x(t,s),y(t,s)\right) \\
&= a(x,y,u) \,\pdydx{u}{x}\left(x,y\right)
+  b(x,y,u) \,\pdydx{u}{y}\left(x,y\right) \ .
\end{align*}
Thus (\ref{char_QLPDE}) is satisfied.

\stage{iv} To prove the uniqueness of the integral surface containing
$\Gamma$, we show that a characteristic curve that starts on an
integral surface $u=v(x,y)$ of (\ref{char_QLPDE}) containing the curve
$\Gamma$ stays on the integral surface.  By uniqueness of
characteristic curves, it will follow that the integral surface
$u=v(x,y)$ is the integral surface $u=u(x,y)$ that we have found
above.

Let
\[
g(t,s) = u(t,s) - v(x(t,s),y(t,s)) \ ,
\]
where
\[
\gamma_s = \left\{ (x,y,u) = (x(t,s), y(t,s) ,u(t,s)) : t \in \RR \right\}
\]
is a characteristic curve, where we assume that $s$ is fix.
We have that (\ref{char_CE}) and (\ref{char_CE_init}) are satisfied
by $x$, $y$ and $u$.  Hence,
\begin{align}
\pdydx{g}{t}(t,s) &= 
\pdydx{u}{t}(t,s) - \pdydx{v}{x}\left(x(t,s),y(t,s)\right) \pdydx{x}{t}(t,s)
- \pdydx{v}{y}\left(x(t,s),y(t,s)\right) \pdydx{y}{t}(t,s) \nonumber \\
&= c(x(t,s),y(t,s),u(t,s)) - a(x(t,s),y(t,s),u(t,s))\,
\pdydx{v}{x}\left(x(t,s),y(t,s)\right) \nonumber \\
&\qquad - b(x(t,s),y(t,s),u(t,s))\,\pdydx{v}{y}\left(x(t,s),y(t,s)\right)
\nonumber \\
&= c(x(t,s),y(t,s),g(t,s)+v(x(t,s),y(t,s)) \nonumber \\
&\qquad - a(x(t,s),y(t,s),g(t,s)+v(x(t,s),y(t,s))\,
\pdydx{v}{x}(x(t,s),y(t,s) \nonumber \\
&\qquad -  b(x(t,s),y(t,s),g(t,s)+v(x(t,s),y(t,s))\,\pdydx{v}{y}(x(t,s),y(t,s)
\label{char_CE_unique} \ .
\end{align}
This is a first order ordinary differential equation for $g$ with the
initial condition
\[
g(0,s) = u(0,s) - v(x(0,s),y(0,s)) = u_0(s) - v(x_0(s),y_0(s)) = 0
\]
because $u=v(x,y)$ is a surface containing the curve $\Gamma$.
By uniqueness of solutions for ordinary differential equations and the
fact that $v$ satisfies (\ref{char_QLPDE}), we have that $g(t,s) = 0$
for all $t \in J_s$ because it is a solution of (\ref{char_CE_unique}). Thus
$v(x(t,s),y(t,s)) = u(t,s) = u(x(t,s),y(t,s))$ for all $t$ and $s$
since $s$ is arbitrary; namely, $v = u$.
\end{proof}

\begin{rmk}
It is important to remember that, for each characteristic curve\\
$\displaystyle
\gamma_s =  \left\{ \left(x(t,s),y(t,s),u(t,s)\right) : t \in J_s\right\}$,
the vector
$\displaystyle
\left(\pdydx{x}{t}(t,s), \pdydx{y}{t}(t,s), \pdydx{u}{t}(t,s)\right) $
is in the tangent plane of the integral surface $u=u(x,y)$ (more
precisely, parallel to the tangent plane) at
$\left(x(t,s),y(t,s),u(t,s)\right)$.
\end{rmk}

\begin{rmk}
If (\ref{char_IFTQL_cond}) is not satisfied at $(t,s) = (0,\tilde{s})$, then
\begin{align}
0 &= \frac{\partial(x,y)}{\partial(t,s)}\bigg|_{(0,\tilde{s})}
= \det \begin{pmatrix}
\displaystyle \pdydx{x}{t} & \displaystyle \pdydx{x}{s} \\[0.7em]
\displaystyle \pdydx{y}{t} & \displaystyle \pdydx{y}{s}
\end{pmatrix}\bigg|_{(0,\tilde{s})}
= \det \begin{pmatrix}
a(x(0,\tilde{s}), y(0,\tilde{s}), u(0,\tilde{s}))
& \displaystyle \pdydx{x}{s}(0,\tilde{s}) \\[0.7em]
b(x(0,\tilde{s}), y(0,\tilde{s}), u(0,\tilde{s}))
& \displaystyle \pdydx{y}{s}(0,\tilde{s})
\end{pmatrix} \nonumber \\
&= \det \begin{pmatrix}
a(x_0(\tilde{s}), y_0(\tilde{s}), u_0(\tilde{s}))
& \displaystyle x_0'(\tilde{s}) \\
b(x_0(\tilde{s}), y_0(\tilde{s}), u_0(\tilde{s}))
& \displaystyle y_0'(\tilde{s})
\end{pmatrix} \nonumber \\
&= a(x_0(\tilde{s}), y_0(\tilde{s}), u_0(\tilde{s}))y_0'(\tilde{s})
- b(x_0(\tilde{s}), y_0(\tilde{s}),u_0(\tilde{s})) x_0'(\tilde{s})
\ . \label{char_idt1}
\end{align}

If $\Gamma$ is on the integral surface $u=u(x,y)$, then
$u_0(s) = u(x_0(s), y_0(s))$ and thus
\begin{equation} \label{char_idt2}
u_0'(s) = \pdydx{u}{x}(x_0(s), y_0(s))x_0'(s)
+\pdydx{u}{y}(x_0(s), y_0(s))y_0'(s) \ .
\end{equation}
Moreover, we have from (\ref{char_QLPDE}) that
\begin{equation} \label{char_idt3}
a(x, y, u)\pdydx{u}{x}(x, y) + b(x, y, u)\pdydx{u}{y}(x, y)
= c(x, y, u)
\end{equation}
at $(x,y,u)= (x_0(s), y_0(s), u_0(s))$ for $s \in I$.

Hence, from (\ref{char_idt2}) and (\ref{char_idt3}), we have
\begin{align}
& b(x_0(\tilde{s}), y_0(\tilde{s}),u_0(\tilde{s})) u_0'(\tilde{s}) -
c(x_0(\tilde{s}), y_0(\tilde{s}),u_0(\tilde{s})) y_0'(\tilde{s}) \nonumber \\
&\qquad = b(x_0(\tilde{s}), y_0(\tilde{s}),u_0(\tilde{s}))
\left(\pdydx{u}{x}(x_0(\tilde{s}), y_0(\tilde{s}))x_0'(\tilde{s})
+\pdydx{u}{y}(x_0(\tilde{s}), y_0(\tilde{s}))y_0'(\tilde{s})\right) \nonumber \\
&\qquad \quad- \left(a(x, y, u)\pdydx{u}{x}(x,y) + b(x, y, u) \pdydx{u}{y}(x,y)
\right)\bigg|_{(x,y,u)= (x_0(\tilde{s}), y_0(\tilde{s}), u_0(\tilde{s}))}
y_0'(\tilde{s})
\nonumber \\
&\qquad = \bigg( b(x_0(\tilde{s}), y_0(\tilde{s}),u_0(\tilde{s}))x_0'(\tilde{s})
- a(x_0(\tilde{s}), y_0(\tilde{s}), u_0(\tilde{s}))y_0'(\tilde{s}) \bigg)
\pdydx{u}{x}(x_0(\tilde{s}), y_0(s)) \nonumber \\
&\qquad = 0 \ ,   \label{char_idt8}
\end{align}
where the last identity comes from (\ref{char_idt1}).  Similarly,
\begin{align}
& a(x_0(\tilde{s}), y_0(\tilde{s}),u_0(\tilde{s})) u_0'(\tilde{s}) -
c(x_0(\tilde{s}), y_0(\tilde{s}),u_0(\tilde{s})) x_0'(\tilde{s}) \nonumber \\
&\qquad = a(x_0(\tilde{s}), y_0(\tilde{s}),u_0(\tilde{s}))
\left(\pdydx{u}{x}(x_0(\tilde{s}), y_0(\tilde{s}))x_0'(\tilde{s})
+\pdydx{u}{y}(x_0(\tilde{s}), y_0(\tilde{s}))y_0'(\tilde{s})\right) \nonumber \\
&\qquad \quad- \left(a(x, y, u)\pdydx{u}{x}(x,y) + b(x, y, u) \pdydx{u}{y}(x,y)
\right)\bigg|_{(x,y,u)= (x_0(\tilde{s}), y_0(\tilde{s}),u_0(\tilde{s}))}
x_0'(\tilde{s}) \nonumber \\
&\qquad = \bigg( a(x_0(\tilde{s}), y_0(\tilde{s}),u_0(\tilde{s}))y_0'(\tilde{s})
- b(x_0(\tilde{s}), y_0(\tilde{s}), u_0(\tilde{s}))x_0'(\tilde{s}) \bigg)
\pdydx{u}{y}(x_0(\tilde{s}), y_0(s)) \nonumber \\
&\qquad = 0 \ .   \label{char_idt9}
\end{align}
Hence, from (\ref{char_idt1}), (\ref{char_idt8} and (\ref{char_idt9}),
we deduce that
\[
\left(a(x,y,u), b(x,y,u),  c(x,y,u)\right)
\big|_{(x,y,u)=(x_0(\tilde{s}), y_0(\tilde{s}), u_0(\tilde{s}))} \times
\left(x_0'(\tilde{s}), y_0'(\tilde{s}), u_0'(\tilde{s}) \right) = 0
\]
and thus these two vectors are parallel.

If (\ref{char_IFTQL_cond}) is not satisfied at $(t,s) = (0,\tilde{s})$
and $\Gamma$ is a characteristic curve, then there may be infinitely
integral surfaces, one for each initial curve $\displaystyle \Gamma^\ast$
intersecting $\Gamma$ transversely that we chose.
However, if (\ref{char_IFTQL_cond}) is not satisfied at
$(t,s) = (0,\tilde{s})$ and $\Gamma$ is not a characteristic, then there is
no integral surface.
\end{rmk}

\begin{egg}
Solve the partial differential equation
\[
\pdydx{u}{x} + \pdydx{u}{y} = 2
\]
with the condition $\displaystyle u(x,0)=x^2$.

The characteristic equations are $x'(t) = 1$, $y'(t) = 1$ and $u'(t) = 2$.
The initial curve is given by $x(0) = s$, $y(0) = 0$ and $u(0) = s^2$.

The equation $x'(t) = 1$ with $x(0)=s$ gives $x = x(t,s) = t + s$,
the equation $y'(t) = 1$ with $y(0)=0$ gives $y = y(t,s) = t$, and
the equation $u'(t) = 2$ with $\displaystyle u(0)=s^2$ gives
$\displaystyle u = u(t,s) = 2t+s^2$.
We can solve $x = t+s$ and $y=t$ in terms of $s$ and $t$ to get
$t=y$ and $s= x-y$.  If we substitute these values of $s$ and $t$ in the
expression for $u$, we get the integral surface
$\displaystyle u= u(x,y) = 2y + (x-y)^2$.
\end{egg}

\begin{egg}
Solve the partial differential equation
\[
x \pdydx{u}{x} + \pdydx{u}{y} = u
\]
with the condition $u(x,0)=\ln(x)$ for $x>0$.

The characteristic equations are $x'(t) = x(t)$, $y'(t) = 1$ and
$u'(t) = u(t)$.  The initial curve is given by are
$x(0) = s$, $y(0) = 0$ and $u(0) = \ln(s)$.

The equation $x'(t) = x(t)$ with $x(0)=s$ gives
$\displaystyle x = x(t,s) = s e^t$,
the equation $y'(t) = 1$ with $y(0)=0$ gives $y = y(t,s) = t$, and
the equation $u'(t) = u(t)$ with $u(0)=\ln(s)$ gives
$\displaystyle u = u(t,s) = \ln(s)e^t$.
We can solve $\displaystyle x = se^t$ and $y=t$ in terms of $s$ and $t$ to get
$t=y$ and $\displaystyle s= xe^{-y}$.  If we substitute these values
of $s$ and $t$ in the expression for $u$, we get the integral surface
\[
u= u(x,y) = \ln(xe^{-y}) e^y = \ln(x) e^y -y e^y \ .
\]
\end{egg}

\begin{egg}
Solve the partial differential equation
\[
\pdydx{u}{x} + \pdydx{u}{y} = u
\]
with the condition $u(x,0)=\cos(x)$.

The characteristic equations are $x'(t) = 1$, $y'(t) = 1$ and $u'(t) = u(t)$.
The initial curve is given by $x(0) = s$, $y(0) = 0$ and $u(0) = \cos(s)$.

The equation $x'(t) = 1$ with $x(0)=s$ gives $x = x(t,s) = t + s$,
the equation $y'(t) = 1$ with $y(0)=0$ gives $y = y(t,s) = t$, and
the equation $u'(t) = u(t)$ with $u(0)=\cos(s)$ gives
$\displaystyle u = u(t,s) = \cos(s)e^t$.
We can solve $x = t+s$ and $y=t$ in terms of $s$ and $t$ to get
$t=y$ and $s= x-y$.  If we substitute these values of $s$ and $t$ in the
expression for $u$, we get the integral surface
$\displaystyle u= u(x,y) = \cos(x-y) e^y$.
\end{egg}

\begin{egg}
Solve the partial differential equation
\[
x^2 \pdydx{u}{x} + y^2 \pdydx{u}{y} = u^2
\]
with the condition $u(x,2x)=1$.

The characteristic equations are
$\displaystyle x'(t) = x^2(t)$, $\displaystyle y'(t) = y^2(t)$ and
$\displaystyle u'(t) = u^2(t)$.
The initial curve is given by $x(0) = s$, $y(0) = 2s$ and $u(0) = 1$.

The equation $\displaystyle x'(t) = x^2(t)$ with $x(0)=s$ gives
\begin{equation} \label{char_idt4}
x = x(t,s) = \frac{s}{1-st} \ .
\end{equation}
Similarly, the equation $\displaystyle y'(t) = y^2(t)$ with $y(0)=2s$ gives
\begin{equation} \label{char_idt5}
y = y(t,s) = \frac{2s}{1-2st} \ .
\end{equation}
Finally, the equation $\displaystyle u'(t) = u^2(t)$ with $u(0)=1$ gives
$\displaystyle u = u(t,s) = \frac{1}{1-t}$.
We can solve (\ref{char_idt4}) and (\ref{char_idt5}) for $s$ and $t$
in terms of $x$ and $y$ to get
$\displaystyle t = \frac{1}{x} - \frac{2}{y}$ and
$\displaystyle s = \frac{xy}{2y-2x}$.
If we substitute these values of $s$ and $t$ in the
expression for $u$, we get the integral surface
\[
u= u(x,y) = \frac{xy}{xy-y+2x} \ .
\]
\end{egg}

\subsection{Lagrange Method}

The Lagrange method produce the
{\bfseries general solution}%
\index{First Order Partial Differential Equation!General Solution}
(namely, a family of integral surfaces that are not required to
contain the initial curve $\Gamma$) of a quasi-Linear first order ordinary
differential equation like (\ref{char_QLPDE}).  We assume that the
characteristic curves are given by the intersection of two families of
surfaces of the form
\[
\phi(x,y,u) = \alpha \quad \text{and} \quad \psi(x,y,u) = \beta \ .
\]
The characteristic curve associated to the initial condition
$(x_0(s),y_0(s),u_0(s)) \in \Gamma$ is the intersection of the
surfaces $\phi(x,y,u) = \alpha_s$ and $\psi(x,y,u) = \beta_s$.
for some pair $(\alpha_s, \beta_s)$ as illustrated in
Figure~\ref{lagrange_fig1}.

\pdfF{characteristics/lagrange_fig1}{Lagrange Method}{The
intersection of the two surfaces $\phi(x,y,u)=\alpha_s$ and
$\psi(x,y,u)=\beta_s$ is the characteristic curve $\gamma_s$
corresponding to the initial condition $(x_0(s),y_0(s),u_0(s))$.
We have $F(\alpha_s,\beta_s) = 0$.}{lagrange_fig1}

The set $\displaystyle \left\{ (\alpha_s,\beta_s) : s \in I \right\}$
is a curve in the $\alpha$,$\beta$ plane.  If this curve is given by
$F(\alpha,\beta)=0$, the integral surface of the quasi-linear equation
containing $\Gamma$ is given by
\[
F(\phi(x,y,u), \psi(x,y,u)) = 0 \ .
\]
The problem is to find $\phi$ and $\psi$.  In some rare cases, we can
find two linearly independent vector valued functions
$\displaystyle v_1:\RR^3\rightarrow \RR^3$ and
$\displaystyle v_2:\RR^3\rightarrow \RR^3$ such that
$v_1(x,y,u)$ and $v_1(x,y,u)$ are orthogonal to
$\big(a(x,y,u),b(x,y,u),c(x,y,u)\big)$ for all $(x,y,u)$, and
$\curL(v_1) = \curL(v_2)=0$.  Since $v_1$ and $v_2$ are curl free,
there exist differentiable functions
$\displaystyle \phi:\RR^3\rightarrow \RR$ and
$\displaystyle \psi:\RR^3\rightarrow \RR$ such that $\graD \phi = v_1$ and
$\graD \psi = v_2$.  We have that $\phi$ and $\psi$ are constant along
the characteristic curves because
\begin{multline*}
\pdfdx{ \phi(x(t,s),y(t,s),u(t,s)) }{t}
= \graD \phi(x(t,s),y(t,s),u(t,s)) \cdot \left(\pdydx{x}{t}(t,s),
\pdydx{y}{t}(t,s), \pdydx{u}{t}(t,s) \right) \\
= v_1(x(t,s),y(t,s),u(t,s)) \cdot
(a(x,y,u),b(x,y,u),c(x,y,u))\bigg|_{(x,y,u)=(x(t,s),y(t,s),u(t,s))} = 0
\end{multline*}
and, similarly,
\[
\pdfdx{ \psi(x(t,s),y(t,s),u(t,s)) }{t} = 0
\]
for all $t$ and $s$.

\begin{egg}
Find the integral surface of the partial differential equation
\[
-y \pdydx{u}{x} + x \pdydx{u}{y} = 0
\]
with the condition $\displaystyle u(x,x)=e^x$.

We have
\[
\left(a(x,y,u),b(x,y,u),c(x,y,u)\right) = (-y,x,0) \ .
\]
The vectors $v_1(x,y,u)=(x,y,0)$ and $v_2(x,y,u)=(0,0,1)$ are two
linearly independent vector valued functions satisfying
$\curL(v_1) = \curL(v_2) = 0$ and orthogonal to\\
$(a(x,y,u),b(x,y,u),c(x,y,u))$ for all $(x,y,u)$.
We have that $\displaystyle \phi(x,y,u)=(x^2+y^2)/2$
satisfies $\graD \phi = v_1$ and $\psi(x,y,u)=u$ satisfies
$\graD \psi = v_2$.  The integral surfaces of the partial differential
equation are of the form
\begin{equation} \label{LagrMegg}
F(x^2+y^2, u) = 0
\end{equation}
for functions $\displaystyle F:\RR^2 \rightarrow \RR$.  

Suppose that $\displaystyle \pdydx{F}{\beta}(\alpha,\beta) \neq 0$,
then we can solve $F(\alpha,\beta)=0$ for $\beta$ as a function of
$\alpha$.  Namely, there exists a function $f:\RR\rightarrow \RR$
such that $F(\alpha,f(\alpha))=0$.  It then follows from (\ref{LagrMegg})
that $\displaystyle u(x,y) =f(x^2+y^2)$.  From the initial condition
$\displaystyle u(x,x)= e^x$, we
get $\displaystyle e^x = f(2x^2)$.  With $\displaystyle z = 2x^2$, we get
$\displaystyle f(z) = e^{\sqrt{z/2}}$ for $z \geq 0$. 
We finally get the integral surface
$\displaystyle u(x,y) = e^{\sqrt{(x^2+y^2)/2}}$.
\end{egg}

\section{Envelope of a Family of Surfaces}

Our analysis of the general first order partial differential equation
in the next sections will use the concept of the envelope of a family
of surfaces.   This is the subject of this section.

We consider the family of surfaces described by
\begin{equation} \label{char_env1}
F(x,y,z,\lambda) = z - G(x,y,\lambda) = 0 \quad , \quad \lambda \in \RR \ .
\end{equation}
Namely, the collection of surfaces $S_\lambda$ for $\lambda \in \RR$
given by
\[
S_\lambda = \left\{ (x,y,z) : F(x,y,z,\lambda) = 0 \right\} \ .
\]
The differential equation
\begin{equation} \label{char_env2}
\pdydx{G}{\lambda}(x,y,\lambda) = 0
\end{equation}
combined with (\ref{char_env1}) yields a curve $\gamma_\lambda$ for
each fixed value of $\lambda$.
The {\bfseries envelope}\index{First Order Partial Differential
Equation!Envelope}
of the family of surfaces is the union of these curves.  The word
``envelope'' refers generally to a surface that wrap up something.
Is our definition of the envelope of a family of surfaces a real surface
in $\displaystyle \RR^3$?

Suppose that
$\displaystyle \pdydx{G}{\lambda}(x_1,y_1,\lambda_1) = 0$
and
$\displaystyle \pdydxn{G}{\lambda}{2}(x_1,y_1,\lambda_1) \neq 0$, then
we may use the Implicit Function Theorem to find open neighbourhoods
$\displaystyle V \subset \RR^2$ of $(x_1,y_1)$ and
$\displaystyle W \subset \RR^3$ of
$(x_1,y_1,\lambda_1)$, and a unique continuously
differentiable function $\lambda:V\to \RR$ such that
$\lambda(x_1,y_1) = \lambda_1$,
$(x,y,\lambda(x,y)) \in W$ for all $(x,y) \in V$ and
\begin{equation} \label{char_env3}
\pdydx{G}{\lambda}\left(x,y,\lambda(x,y)\right) = 0
\end{equation}
for $(x,y) \in V$.  Moreover, all solutions of (\ref{char_env2}) in $W$
are given by this function $\lambda$.

For $\lambda_2$ closed to $\lambda_1$, the curve $\gamma_{\lambda_2}$
is defined by
\begin{align*}
\gamma_{\lambda_2} &= \left\{ (x,y,z) : (x,y) \in V \ ,
\ \lambda(x,y) = \lambda_2 \ \text{and} \ F(x,y,z,\lambda_2) = 0 \right\} \\
&= \left\{ (x,y,G(x,y,\lambda(x,y))) : (x,y) \in V \ \text{and}
\ \lambda(x,y) = \lambda_2 \right\} \ .
\end{align*}
The union of these curves for $\lambda_2$ close to $\lambda_1$ form a
``patch'' of the envelope $E$ for the family of surfaces in
(\ref{char_env1}).  If we substitute $\lambda = \lambda(x,y)$ in
(\ref{char_env1}), we get the equation
\[
F(x,y,z,\lambda(x,y)) = z - G(x,y,\lambda(x,y)) = 0
\]
of a local ``patch'' of the envelope $E$.

Note that $\gamma_{\lambda_2}$ is the intersection of the
surface $S_{\lambda_2}$ with the envelope $E$ of the family of
surface.

\begin{rmk}
We have from (\ref{char_env3}) that               \label{char_env_tgp}
\[
\pdfdx{G(x,y,\lambda(x,y))}{x}
= \pdydx{G}{x}(x,y,\lambda(x,y)) + \pdydx{G}{\lambda}(x,y,\lambda(x,y))\,
 \pdydx{\lambda}{x}(x,y) = \pdydx{G}{x}(x,y,\lambda(x,y))
\]
and
\[
\pdfdx{G(x,y,\lambda(x,y))}{y}
= \pdydx{G}{y}(x,y,\lambda(x,y)) + \pdydx{G}{\lambda}(x,y,\lambda(x,y))\,
 \pdydx{\lambda}{y}(x,y) = \pdydx{G}{y}(x,y,\lambda(x,y)) \ .
\]
Hence, if we let $H(x,y,z) = F(x,y,z,\lambda(x,y))$, a normal to the
envelope $E$ at a point
$(\tilde{x},\tilde{y},\tilde{z}
= (\tilde{x},\tilde{y},G(\tilde{x},\tilde{y},\lambda(\tilde{x},\tilde{y})))$
of $E$ is given by
\begin{align*}
\graD H(\tilde{x},\tilde{y},\tilde{z}) &=
\left( \pdydx{H}{x}(\tilde{x},\tilde{y},\tilde{z}) ,
\pdydx{H}{y}(\tilde{x},\tilde{y},\tilde{z}) ,
\pdydx{H}{z}(\tilde{x},\tilde{y},\tilde{z}) \right) \\
&= \left( -\pdfdx{G(\tilde{x},\tilde{y},\lambda(\tilde{x},\tilde{y}))}{x} ,
-\pdfdx{G(\tilde{x},\tilde{y},\lambda(\tilde{x},\tilde{y}))}{y}, 1 \right) \\
&= \left( -\pdydx{G}{x}(\tilde{x},\tilde{y},\lambda(\tilde{x},\tilde{y})) ,
-\pdydx{G}{y}(\tilde{x},\tilde{y},\lambda(\tilde{x},\tilde{y})), 1 \right) \ .
\end{align*}
This is also a normal vector to the surface described by
$F(x,y,z,\tilde{\lambda}) = z - G(x,y,\tilde{\lambda}) = 0$
at the point $(\tilde{x},\tilde{y},\tilde{z})$ of this surface where
$\lambda(\tilde{x},\tilde{y}) = \tilde{\lambda}$; namely, a point    
$(\tilde{x},\tilde{y},\tilde{z}) \in \gamma_{\tilde{\lambda}}$.
\end{rmk}

\begin{egg}
Find the envelope of the family of spheres
\[
x^2 + (y-\lambda)^2 + z^2 = 4 \ .
\]

If we assume that $z \geq 0$, we can write the equation of this
family of spheres as
\begin{equation} \label{char_env4}
F(x,y,x,\lambda) = z - \sqrt{4-x^2-(y-\lambda)^2} = 0 \ .
\end{equation}
Thus $\displaystyle G(x,y,\lambda) = \sqrt{4-x^2-(y-\lambda)^2}$ in
(\ref{char_env1}).  To find the envelope, we need to consider
(\ref{char_env2}) for this function $G$; namely,
\[
\pdydx{G}{\lambda}(x,y,\lambda) =
\frac{(y-\lambda)}{\sqrt{4-x^2-(y-\lambda)^2}} = 0 \ .
\]
Thus $\lambda = y$.  If we substitute $\lambda = y$ in
(\ref{char_env4}), we get $\displaystyle z = \sqrt{4-x^2}$.
Similarly, for $z<0$, we get $\displaystyle z = -\sqrt{4-x^2}$.
Therefore, the envelope is the surface $\displaystyle x^2 + z^2 = 4$.
It is the cylinder of radius $2$ whose axis is the $y$-axis.
\end{egg}

\section{General First Order Partial Differential Equations}

We consider partial differential equation of the form
\begin{equation} \label{char_GPDE}
F\left(x,y,u,\pdydx{u}{x}, \pdydx{u}{y}\right) = 0 \ ,
\end{equation}
where $\displaystyle F:\RR^5 \rightarrow \RR$ is sufficiently differentiable.

The goal is to find a differentiable surface $u=u(x,y)$ that satisfies
(\ref{char_GPDE}) and contains a given curve $\Gamma$ defined by
\[
\left\{ (x,y,u) = (\tilde{x}(s) , y_0(s) , u_0(s)) : s \in I \right\} \ ,
\]
where $I$ is an open interval, and $x_0$, $y_0$, $u_0$ are
differentiable functions.  This is the
{\bfseries Cauchy problem}%
\index{First Order Partial Differential Equation!Cauchy Problem} for
the general first-order partial differential equation.  As for the
quasi-linear partial differential equation, we have the following definition.

\begin{defn}
A surface $u=u(x,y)$ satisfying (\ref{char_GPDE}) and containing a
given curve $\Gamma$ is called an
{\bfseries integral surface}%
\index{First Order Partial Differential Equation!Integral Surface}
for (\ref{char_GPDE}).
\end{defn}

As for the quasi-linear partial differential equation, we will find a family of curves
$\displaystyle \left\{ \gamma_s \right\}_{s\in I}$ that
generates an integral surface $u=u(x,y)$ for (\ref{char_GPDE}) and
where each curve $\gamma_s$ intersects transversely $\Gamma$ at
$(x_0(s),y_0(s),u_0(s))$.
Unfortunately, the ordinary differential equation that determine this
family of curves cannot be read directly from the partial differential
equation as it was done for the quasi-linear partial differential equation.

We did not have to explicitly consider the tangent plane to the
integral surface for the quasi-linear partial differential equation
but we have to do so for the general partial differential equation.
Suppose that $u=u(x,y)$ is a surface satisfying (\ref{char_GPDE}) and let
\[
(\tilde{x},\tilde{y},\tilde{u})
= \left(\tilde{x},\tilde{y},u(\tilde{x},\tilde{y})\right)
\]
be a point on this surface.  The equation of the tangent plane to the
integral surface at $(\tilde{x},\tilde{y},\tilde{u})$ is
\begin{equation} \label{char_tp_monge}
\pdydx{u}{x}(\tilde{x},\tilde{y})(x-\tilde{x})
+ \pdydx{u}{y}(\tilde{x},\tilde{y}) (y-\tilde{y}) - (u-\tilde{u}) = 0 \ .
\end{equation}
(\ref{char_GPDE}) gives a relation between
$\displaystyle \pdydx{u}{x}(\tilde{x},\tilde{y})$ and
$\displaystyle \pdydx{u}{y}(\tilde{x},\tilde{y})$; namely,
\[
F\left( \tilde{x}, \tilde{y}, \tilde{u}, \pdydx{u}{x}(\tilde{x},\tilde{y}),
\pdydx{u}{y}(\tilde{x},\tilde{y}) \right) = 0 \ .
\]
Combined with the derivative along the curve $\Gamma$, we will be able
to determine the tangent plane to the integral surface.

\subsection{Equations for $x$, $y$ and $u$}

The comments of the previous paragraph can be used to deduce
conditions that must be satisfied by the curves $\gamma_s$ with
$s\in I$.  The equation $F(\tilde{x},\tilde{y},\tilde{u},p,q)=0$
represents a curve in the $p$,$q$ plane.  We get a one-parameter
family of tangent planes defined by
\[
p(x-\tilde{x}) + q (y-\tilde{y}) - (u-\tilde{u}) = 0 \quad \text{and}
\quad F(\tilde{x},\tilde{y},\tilde{u},p,q)=0 \ .
\]
This family of tangent planes, all containing the point
$(\tilde{x},\tilde{y},\tilde{u})$,
generates a ``cone'' with its summit at
$(\tilde{x},\tilde{y},\tilde{u})$.  This cone
is called a {\bfseries Monge Cone}\index{First Order Partial
Differential Equation!Monge Cone}.
To visualize this cone, think of the region that is enveloped by this
one-parameter family of tangent planes all containing the point
$(\tilde{x},\tilde{y},\tilde{u})$. This is illustrate in Figure~\ref{char_fig2}.

\pdfF{characteristics/char_fig2}{The Monge Cone}{Assuming that we
can write the solution of the equation
$F(\tilde{x},\tilde{y},\tilde{u},p,q)=0$ as
$q=q(p)$, the figure represent some of the tangent planes containing
the point $(\tilde{x},\tilde{y},\tilde{u})$ and enveloping the Monge
Cone.}{char_fig2}

Let
\[
S = \left\{ (p,q) : F(\tilde{x},\tilde{y},\tilde{u},p,q) = 0 \right\} \ .
\]
We assume that $F$ in (\ref{char_GPDE}) is not degenerate on $S$;
namely
\[
\left(\pdydx{F}{p}(\tilde{x},\tilde{y},\tilde{u},p,q),
\pdydx{F}{q}(\tilde{x},\tilde{y},\tilde{u},p,q) \right) \neq (0,0)
\]
for all $(p,q) \in S$.  We may assume without loss of generality that
\[
\pdydx{F}{q}(\tilde{x},\tilde{y},\tilde{u},p,q) \neq 0
\]
locally along the curve $S$.  We can then write $q=q(p)$ locally.

If we derive with respect to $p$ the equation
\begin{equation} \label{char_tp1}
p(x-\tilde{x}) + q(y-\tilde{y}) = u-\tilde{u}
\end{equation}
of the tangent plane at $(\tilde{x},\tilde{y},\tilde{u})$, we get
\[
(x-\tilde{x}) + \dydx{q}{p}(y-\tilde{y}) = 0 \ .
\]
By the Implicit Function Theorem, since we assume that
$\displaystyle \pdydx{F}{q}(\tilde{x},\tilde{y},\tilde{u},p,q) \neq 0$,
we have 
\[
\dydx{q}{p} =
- \frac{\displaystyle \pdydx{F}{p}(\tilde{x},\tilde{y},\tilde{u},p,q)}
{\displaystyle \pdydx{F}{q}(\tilde{x},\tilde{y},\tilde{u},p,q)} \ .
\]
Hence,
\begin{equation} \label{char_tp2}
\frac{(x-\tilde{x})}{\displaystyle \pdydx{F}{p}
(\tilde{x},\tilde{y},\tilde{u},p,q)} = 
\frac{(y-\tilde{x})}{\displaystyle \pdydx{F}{q}
(\tilde{x},\tilde{y},\tilde{u},p,q)} \ .
\end{equation}
Moreover, if we divide (\ref{char_tp1}) by $(y-\tilde{y})$, we get
\[
\frac{u-\tilde{u}}{y-\tilde{y}} = p\frac{x-\tilde{x}}{y-\tilde{y}} + q = 
p\, \frac{\displaystyle \pdydx{F}{p}(\tilde{x},\tilde{y},\tilde{u},p,q)}
{\displaystyle \pdydx{F}{q}(\tilde{x},\tilde{y},\tilde{u},p,q)} + q \ .
\]
Thus
\[
\left(\frac{u-\tilde{u}}{y-\tilde{y}}\right)
\pdydx{F}{q}(\tilde{x},\tilde{y},\tilde{u},p,q) = 
p\, \pdydx{F}{p}(\tilde{x},\tilde{y},\tilde{u},p,q)
+ q\, \pdydx{F}{q}(\tilde{x},\tilde{y},\tilde{u},p,q) \ .
\]
This finally gives
\begin{equation} \label{char_tp3}
\frac{u-\tilde{u}}{\displaystyle p\,
\pdydx{F}{p}(\tilde{x},\tilde{y},\tilde{u},p,q) +
q\, \pdydx{F}{q}(\tilde{x},\tilde{y},\tilde{u},p,q)}
= \frac{y-\tilde{y}}
{\displaystyle \pdydx{F}{q}(\tilde{x},\tilde{y},\tilde{u},p,q)} \ .
\end{equation}

Equations (\ref{char_tp2}) and (\ref{char_tp3}) yield the
standard representation of the line defined by (\ref{char_tp1}).  This
line is in the tangent plane of the integral surface
$u=u(x,y)$ at $(\tilde{x},\tilde{y},\tilde{u})$ if
\[
(p.q) = \left( \pdydx{u}{x}(\tilde{x},\tilde{y}) ,
\pdydx{u}{y}(\tilde{x},\tilde{y}) \right) \ .
\]
The direction of this line at $(\tilde{x},\tilde{y},\tilde{u})$ is
\[
\left( \pdydx{F}{p}(\tilde{x},\tilde{y},\tilde{u},p,q),
\pdydx{F}{q}(\tilde{x},\tilde{y},\tilde{u},p,q) ,
p\, \pdydx{F}{p}(\tilde{x},\tilde{y},\tilde{u},p,q)
+ q\, \pdydx{F}{q}(\tilde{x},\tilde{y},\tilde{u},p,q) \right) \ .
\]

As for the quasi-linear first order partial differential equation,
this suggests that the integral surface is the union of the family of curves
$\displaystyle \left\{\gamma_s\right\}_{s\in I}$, where
each curve $\gamma_s$ is defined by
\[
\gamma_s = \left\{ (x,y,u) = (x(t,s) , y(t,s) , u(t,s)) : t \in J_s \right\}
\]
such that
\begin{equation} \label{char_CE_GPDE1}
\begin{split}
\pdydx{x}{t}(t,s) &= \pdydx{F}{p}(x(t,s),y(t,s),u(t,s),p(t,s),q(t,s)) \ , \\
\pdydx{y}{t}(t,s) &= \pdydx{F}{q}(x(t,s),y(t,s),u(t,s),p(t,s),q(t,s)) \ , \\
\pdydx{u}{t}(t,s) &= p(t,s) \pdydx{F}{p}(x(t,s),y(t,s),u(t,s),p(t,s),q(t,s)) \\
& \qquad + q(t,s)  \pdydx{F}{q}(x(t,s),y(t,s),u(t,s),p(t,s),q(t,s))
\end{split}
\end{equation}
and
\[
\left(x(0,s),y(0,s),u(0,s)\right) = \left(x_0(s), y_0(s), u_0(s)\right) \ ,
\]
where $p$ and $q$ are unknown functions to be determined later and
$J_s$ is the domain of the solution for $s$ given.
To determine $p$ and $q$, we need two additional differential
equations related to the tangent planes of the integral surface
sought.

If we assume for now that we have also determined $p$ and $q$, and
that
\begin{equation} \label{char_IFTG_cond}
\frac{\partial(x,y)}{\partial(t,s)}\bigg|_{(t,s)} = \det
\begin{pmatrix}
\displaystyle \pdydx{x}{t} & \displaystyle \pdydx{x}{s} \\[0.7em]
\displaystyle \pdydx{y}{t} & \displaystyle \pdydx{y}{s}
\end{pmatrix}\bigg|_{(t,s)} \neq 0 \ ,
\end{equation}
then we can use the Inverse Mapping Theorem to express locally $s$ and
$t$ as functions of $x$ and $y$.  Substituting these expressions of
$s$ and $t$ into $u=u(s,t)$, $p=p(t,s)$ and $q=q(s,t)$ yields
$u=u(x,y)$, $p=p(x,y)$ and $q=q(x,y)$.  As we did for the quasi-linear
first order partial differential equation, we keep $u$, $p$ and $q$ to
denote the functions of $s$ and $t$, as well as the functions of $x$
and $y$.  We have $u(x,y) = u(t,s)$ for $x=x(s,t)$ and $y=y(s,t)$, and similar
equalities for $p$ and $q$.

\subsection{Equations for $p$ and $q$}

The functions $p$ and $q$ in (\ref{char_CE_GPDE1}) must satisfy
\begin{equation} \label{char_Gidx3}
p(t,s) = \pdydx{u}{x}(x(t,s),y(t,s)) \quad \text{and} \quad
q(t,s) = \pdydx{u}{y}(x(t,s),y(t,s))
\end{equation}
if we want to satisfy the consistency conditions
\[
p(x,y) = \pdydx{u}{x}(x,y) \quad \text{and} \quad q(x,y) = \pdydx{u}{y}(x,y) \ .
\]

If we assume that (\ref{char_Gidx3}) is satisfied and derive the
equations in (\ref{char_Gidx3}) with respect to $t$, we get
\begin{align}
\pdydx{p}{t}(t,s) &= \pdydxn{u}{x}{2}(x(t,s),y(t,s)) \pdydx{x}{t}(t,s)
+ \pdydxnm{u}{y}{x}{2}{}{}(x(t,s),y(t,s)) \pdydx{y}{t}(t,s) \nonumber \\
&= \pdydxn{u}{x}{2}(x(t,s),y(t,s))
\pdydx{F}{p}\left(x(t,s),y(t,s),u(t,s),p(t,s),q(t,s)\right) \nonumber \\
& \qquad + \pdydxnm{u}{y}{x}{2}{}{}(x(t,s),y(t,s))
\pdydx{F}{q}\left(x(t,s),y(t,s),u(t,s),p(t,s),q(t,s)\right)
\label{char_Gidx1}
\end{align}
and
\begin{align}
\pdydx{q}{t}(t,s) &= \pdydxnm{u}{x}{y}{2}{}{}(x(t,s),y(t,s)) \pdydx{x}{t}(t,s)
+ \pdydxn{u}{y}{2}(x(t,s),y(t,s)) \pdydx{y}{t}(t,s) \nonumber \\
&= \pdydxnm{u}{x}{y}{2}{}{}(x(t,s),y(t,s))
\pdydx{F}{p}\left(x(t,s),y(t,s),u(t,s),p(t,s),q(t,s)\right) \nonumber \\
& \qquad + \pdydxn{u}{y}{2}(x(t,s),y(t,s))
\pdydx{F}{q}\left(x(t,s),y(t,s),u(t,s),p(t,s),q(t,s)\right) \ .
\label{char_Gidx2}
\end{align}

Moreover, if $u$ is an integral surface, we have
$\displaystyle F\left(x,y,u(x,y),p(x,y),q(x,y)\right) = 0$ with
$\displaystyle p(x,y) = \pdydx{u}{x}(x,y)$ and
$\displaystyle q(x,y) = \pdydx{u}{y}(x,y)$.  If we derive
$\displaystyle F\left(x,y,u(x,y),p(x,y),q(x,y)\right) = 0$ 
with respect to $x$, we get
\[
\pdydx{F}{x} + \pdydx{F}{u}\pdydx{u}{x} + \pdydx{F}{p}\pdydx{p}{x}
+ \pdydx{F}{q} \pdydx{q}{x} = 0
\]
at $(x,y,u(x,y),p(x,y),q(x,y))$.  Hence,
\begin{equation} \label{char_Gidx4}
-\pdydx{F}{x} - \pdydx{F}{u}\pdydx{u}{x}
= \pdydx{F}{p}\pdydx{p}{x} + \pdydx{F}{q} \pdydx{q}{x}
= \pdydxn{u}{x}{2} \pdydx{F}{p} + \pdydxnm{u}{y}{x}{2}{}{} \pdydx{F}{q}
\end{equation}
at $(x,y,u(x,y),p(x,y),q(x,y))$.  Similarly, if we derive
$\displaystyle F\left(x,y,u(x,y),p(x,y),q(x,y)\right) = 0$
with respect to $y$, we get
\[
\pdydx{F}{y} + \pdydx{F}{u}\pdydx{u}{y} + \pdydx{F}{p}\pdydx{p}{y}
+ \pdydx{F}{q} \pdydx{q}{y} = 0
\]
at $(x,y,u(x,y),p(x,y),q(x,y))$.  Hence,
\begin{equation} \label{char_Gidx5}
-\pdydx{F}{y} - \pdydx{F}{u}\pdydx{u}{y}
= \pdydx{F}{p}\pdydx{p}{y} + \pdydx{F}{q} \pdydx{q}{y}
= \pdydxnm{u}{x}{y}{2}{}{} \pdydx{F}{p} + \pdydxn{u}{y}{2}
\pdydx{F}{q}
\end{equation}
at $(x,y,u(x,y),p(x,y),q(x,y))$.  If we substitute (\ref{char_Gidx4})
and (\ref{char_Gidx5}) into (\ref{char_Gidx1}) and (\ref{char_Gidx2})
respectively, and use the relations $p(x,y) = p(t,s)$,
$q(x,y) = q(t,s)$ and $u(x,y) = u(t,s)$ for $(x,t) = (x(t,s), y(t,s))$,
we get
\begin{equation} \label{char_CE_GPDE2}
\begin{split}
\pdydx{p}{t}(t,s)
&=-\pdydx{F}{x}\left(x(t,s),y(t,s),u(t,s),p(t,s),q(t,s)\right) \\
& \qquad - p(t,s) \,
\pdydx{F}{u}\left(x(t,s),y(t,s),u(t,s),p(t,s),q(t,s)\right) \ , \\
\pdydx{q}{t}(t,s)
&=-\pdydx{F}{y}\left(x(t,s),y(t,s),u(t,s),p(t,s),q(t,s)\right) \\
& \qquad - q(t,s)\,
\pdydx{F}{u}\left(x(t,s),y(t,s),u(t,s),p(t,s),q(t,s)\right) \ .
\end{split}
\end{equation}

(\ref{char_CE_GPDE1}) and (\ref{char_CE_GPDE2}) define a system of five
ordinary differential equations for the five unknown functions
$x(t,s)$, $y(t,s)$, $u(t,s)$, $p(t,s)$ and $q(t,s)$.

\subsection{initial conditions}

The only thing missing for the system of differential equations
(\ref{char_CE_GPDE1}) and (\ref{char_CE_GPDE2}) is a set of initial
conditions.  Since $\Gamma$ must be on the integral surface, we
require $x(0,s) = x_0(s)$, $y(0,s) = y_0(s)$ and $u(0,s) = u_0(s)$ for
$s\in I$.  Moreover, if (\ref{char_Gidx3}) is true, then
\begin{align*}
p_0(s) = p(0,s) &= p(x(0,s),y(0,s)) = \pdydx{u}{x}(x_0(s),y_0(s)) \\
\intertext{and}
q_0(s) = q(0,s) &= q(x(0,s),y(0,s)) = \pdydx{u}{y}(x_0(s),y_0(s))
\end{align*}
must satisfy
\begin{equation} \label{char_sic1}
F\left( x_0(s), y_0(s), u_0(s), p_0(s), q_0(s)\right) = 0
\end{equation}
since
\[
F\left( x_0(s), y_0(s), u(x_0(s),y_0(s)), \pdydx{u}{x}(x_0(s),y_0(s)),
\pdydx{u}{y}(x_0(s),y_0(s)) \right) = 0
\]
for all $s$.

Moreover,
\[
\dydx{u_0}{s}(s) = \dfdx{u(x_0(s),y_0(s))}{s}
= \pdydx{u}{x}(x_0(s),y_0(s))\,\dydx{x_0}{s}(s)
+ \pdydx{u}{y}(x_0(s),y_0(s))\,\dydx{y_0}{s}(s)
\]
yields
\begin{equation} \label{char_sic2}
\dydx{u_0}{s}(s) = p_0(s)\dydx{x_0}{s}(s)+ q_0(s)\dydx{y_0}{s}(s) \ .
\end{equation}
The initial conditions for $p$ and $q$ at $t=0$ should therefore be
$p_0(s)$ and $q_0(s)$, the solution of (\ref{char_sic1}) and
(\ref{char_sic2}).

\begin{defn} \label{char_gdefn}
The equations in (\ref{char_CE_GPDE1}) and (\ref{char_CE_GPDE2}) are
called the
{\bfseries characteristic equations}%
\index{First Order Partial Differential Equation!Characteristic Equations}
for the general partial differential equation in (\ref{char_GPDE}).  A solution of the
characteristic equations is called a
{\bfseries characteristic strip}%
\index{First Order Partial Differential Equation!Characteristic Strip}.  The
{\bfseries characteristic curve}%
\index{First Order Partial Differential Equation!Characteristic Curve} are the
projection in the $x$, $y$, $u$ space of the solutions of the
characteristic equations with the initial condition
\[
\left( x(0,s) , y(0,s) , u(0,s) , p(0,s) , q(0,s) \right) =
\left( x_0(s) , y_0(s) , u_0(s) , p_0(s) , q_0(s) \right) \ ,
\]
where $p_0(s)$ and $q_0(s)$ are given by
\begin{equation} \label{char_sic1D}
F\left(x_0(s),y_0(s),u_0(s), p_0(s),q_0(s) \right) = 0
\end{equation}
and
\begin{equation} \label{char_sic2D}
\dydx{u_0}{s}(s) = p_0(s)\dydx{x_0}{s}(s)+ q_0(s)\dydx{y_0}{s}(s)
\end{equation}
for $s \in I$.
\end{defn}

\begin{rmk}
The system of equations (\ref{char_sic1D}) and (\ref{char_sic2D}) may
have no solution, a unique solution or many solutions.  Namely,
there may be no choice, only one choice or many choices possible for
$p_0(s)$ and $q_0(s)$.  Therefore, there may be zero, one or many
integral surfaces containing the initial curve $\Gamma$.
\end{rmk}

\begin{theorem}
Let $\Gamma$ be a curve defined by
\[
\left\{ (x,y,u) = \left(x_0(s), y_0(s),u_0(s)\right) : s\in I \right\} \ ,
\]
where $I$ is an open interval and $x_0, y_0, u_0 \in C_b^1(I)$.
Let $p_0(s)$ and $q_0(s)$ be the solutions, if any, of
\begin{equation} \label{char_idxIN1}
F\left( x_0(s), y_0(s), u_0(s), p_0(s), q_0(s)\right) = 0
\end{equation}
and
\begin{equation} \label{char_idxIN2}
\dydx{u_0}{s}(s) = p_0(s)\dydx{x_0}{s}(s)+ q_0(s)\dydx{y_0}{s}(s)
\end{equation}
for $s\in I$.  (see Remark~\ref{char_PQCON} below.)\quad For $s \in I$
fixed, let
\[
\gamma_s = \left\{ \left(x(t,s), y(t,s), u(t,s) \right) : t \in J_s \right\}
\]
be the projection in the $x$,$y$,$u$ space of the orbit of the solution
\[
(x,y,u,p,q) = \left(x(t,s), y(t,s), u(t,s), p(t,s), q(t,s) \right)
\]
of (\ref{char_CE_GPDE1}) and (\ref{char_CE_GPDE2}) with the initial
condition
\begin{equation} \label{char_init_GPDE}
(x(0,s),y(0,s),u(0,s),p(0,s),q(0,s))
= \left(x_0(s), y_0(s), u_0(s), p_0(s), q_0(s) \right) \ ,
\end{equation}
where $J_s \subset \RR$ is the domain of the solution.  If
\begin{equation} \label{char_GPDE_cond}
\det
\begin{pmatrix}
\displaystyle \pdydx{F}{p}(x(t,s),y(t,s),u(t,s),p(t,s),q(t,s)) &
\displaystyle \pdydx{x}{s}(t,s) \\[0.7em]
\displaystyle \pdydx{F}{q}(x(t,s),y(t,s),u(t,s),p(t,s),q(t,s)) &
\displaystyle \dydx{y}{s}(t,s)
\end{pmatrix} \neq 0
\end{equation}
for all $(t,s)$, then the surface $S$ given by the union of the family
of curves $\displaystyle \left\{ \gamma_s \right\}_{s\in I}$ is the
integral surface for (\ref{char_GPDE}) containing the curve
$\Gamma$.  There is only one integral surface containing $\Gamma$.
\end{theorem}

\begin{proof}
\stage{i} The proof that $S$ is a nice surface is identical to the
proof given in (i) of the proof of Proposition~\ref{qlfoPDEintsurf}

\stage{ii} The existence of a surface
$u=u(x,y)$ comes from the Inverse Mapping Theorem because
\begin{align*}
\frac{\partial(x,y)}{\partial(t,s)}\bigg|_{(\tilde{t},\tilde{s})} &= \det
\begin{pmatrix}
\displaystyle \pdydx{x}{t} & \displaystyle \pdydx{x}{s} \\[0.7em]
\displaystyle \pdydx{y}{t} & \displaystyle \pdydx{y}{s}
\end{pmatrix}\bigg|_{(\tilde{t},\tilde{s})} \\
&= \det \begin{pmatrix}
\displaystyle
\pdydx{F}{p}(x(\tilde{t},\tilde{s}),y(\tilde{t},\tilde{s}),
u(\tilde{t},\tilde{s}),p(\tilde{t},\tilde{s}),q(\tilde{t},\tilde{s})) &
\displaystyle \pdydx{x}{s}(\tilde{t},\tilde{s}) \\[0.7em]
\displaystyle \pdydx{F}{q}(x(\tilde{t},\tilde{s}),y(\tilde{t},\tilde{s}),
u(\tilde{t},\tilde{s}),p(\tilde{t},\tilde{s}),q(\tilde{t},\tilde{s})) &
\displaystyle \pdydx{y}{s}(\tilde{t},\tilde{s})
\end{pmatrix} \neq 0
\end{align*}
by hypothesis. We can write $s$ and $t$ in a neighbourhood of
$(\tilde{t},\tilde{s})$ as functions of $x$ and $y$.  Substituting
these expressions of $s$ and $t$ in $u=u(t,s)$ gives the equation
$u=u(x,y)$ to describe the surface $S$.  We can also substitute these
expressions of $s$ and $t$ in $p=p(t,s)$ and $q=q(t,s)$ to get
$p=u(x,y)$ and $q=q(x,y)$.

\stage{iii} We claim that
\begin{equation} \label{char_idxCON3}
F\left(x(t,s),y(t,s),u(t,s),p(t,s),q(t,s)\right)=0
\end{equation}
for all $t$ and $s$.  We have from (\ref{char_CE_GPDE1}) and
(\ref{char_CE_GPDE2}) that
\begin{align*}
\pdfdx{F\left(x,y,u,p,q\right)}{t}
&= \pdydx{F}{x}\left(x,y,u,p,q\right)\pdydx{x}{t}
+\pdydx{F}{y}\left(x,y,u,p,q\right)\pdydx{y}{t}
+\pdydx{F}{u}\left(x,y,u,p,q\right)\pdydx{u}{t} \\
& \qquad +\pdydx{F}{p}\left(x,y,u,p,q\right)\pdydx{p}{t}
+\pdydx{F}{q}\left(x,y,u,p,q\right)\pdydx{q}{t} \\
&=\pdydx{F}{x}\left(x,y,u,p,q\right)\pdydx{F}{p}\left(x,y,u,p,q\right)
+\pdydx{F}{y}\left(x,y,u,p,q\right)\pdydx{F}{q}\left(x,y,u,p,q\right) \\
&\qquad +\pdydx{F}{u}\left(x,y,u,p,q\right)\left(
p\, \pdydx{F}{p}(x,y,u,p,q) + q\,\pdydx{F}{q}(x,y,u,p,q)\right) \\
&\qquad +\pdydx{F}{p}\left(x,y,u,p,q\right)\left(
-\pdydx{F}{x}\left(x,y,u,p,q\right) -p\,\pdydx{F}{u}\left(x,y,u,p,q\right)
\right) \\
&\qquad +\pdydx{F}{q}\left(x,y,u,p,q\right)\left(
-\pdydx{F}{y}\left(x,y,u,p,q\right)- q\,\pdydx{F}{u}\left(x,y,u,p,q\right)
\right) =0
\end{align*}
for all $(t,s)$.  Thus $F\left(x(t,s),y(t,s),u(t,s),p(t,s),q(t,s)\right)$
is constant with respect to $t$ whatever $s \in I$.  Since  
\[
F\left(x(0,s),y(0,s),u(0,s),p(0,s),q(0,s)\right)=
F\left(x_0(s),y_0(s),u_0(s),p_0(s),q_0(s)\right)= 0
\]
for $s\in I$ by hypothesis (\ref{char_idxIN1}), we have that
$F\left(x(t,s),y(t,s),u(t,s),p(t,s),q(t,s)\right) = 0$ for all $t \in J_s$
and $s\in  I$.

\stage{iv} We need to prove that our new surface $u=u(x,y)$ satisfies
\[
p(x,y) = \pdydx{u}{x}(x,y) \quad \text{and} \quad
q(x,y) = \pdydx{u}{y}(x,y) \ .
\]
To do this, it is enough to show that
\begin{equation} \label{char_idxCON2}
\pdydx{u}{t}(t,s) = p(t,s)\pdydx{x}{t}(t,s) + q(t,s) \pdydx{y}{t}(t,s)
\end{equation}
and
\begin{equation} \label{char_idxCON1}
\pdydx{u}{s}(t,s) = p(t,s)\pdydx{x}{s}(t,s) + q(t,s) \pdydx{y}{s}(t,s) \ ,
\end{equation}
because the system
\begin{align*}
\pdydx{u}{t}(t,s) &= A(t,s)\pdydx{x}{t}(t,s) + B(t,s)
\pdydx{y}{t}(t,s) \\
\pdydx{u}{s}(t,s) &= A(t,s)\pdydx{x}{s}(t,s) + B(t,s) \pdydx{y}{s}(t,s)
\end{align*}
as the solution
\begin{align*}
A(t,s) &= \pdydx{u}{x}(x(t,s),y(t,s)) \\
B(t,s) &= \pdydx{u}{x}(x(t,s),y(t,s))
\end{align*}
according the Chain Rule, and the solution is (locally) unique since
$\displaystyle \frac{\partial(x,y)}{\partial(t,s)}\bigg|_{(t,s)} \neq 0$.

(\ref{char_idxCON2}) is a direct consequence of the three equations in
(\ref{char_CE_GPDE1}).  To prove (\ref{char_idxCON1}), let
\[
R(t,s) = \pdydx{u}{s}(t,s) - p(t,s)\pdydx{x}{s}(t,s)
- q(t,s) \pdydx{y}{s}(t,s) \ . 
\]
We have
\begin{align*}
\pdydx{R}{t} &=
\pdydxnm{u}{t}{s}{2}{}{} - \pdydx{p}{t}\pdydx{x}{s}
- p\pdydxnm{x}{t}{s}{2}{}{} - \pdydx{q}{t}\pdydx{y}{s}
- q\pdydxnm{y}{t}{s}{2}{}{} \\
&= \pdfdx{ \left( \pdydx{u}{t} - p\pdydx{x}{t}
- q \pdydx{y}{t} \right)}{s}
- \pdydx{p}{t}\pdydx{x}{s} - \pdydx{q}{t}\pdydx{y}{s}
+ \pdydx{p}{s}\pdydx{x}{t} + \pdydx{q}{s}\pdydx{y}{t} \\
&= -\pdydx{p}{t}\pdydx{x}{s}
- \pdydx{q}{t}\pdydx{y}{s}
+ \pdydx{p}{s}\pdydx{x}{t}
+ \pdydx{q}{s}\pdydx{y}{t}
\end{align*}
for all $(t,s)$ due to (\ref{char_idxCON2}).  We then get from
(\ref{char_CE_GPDE1}) and (\ref{char_CE_GPDE2}) that
\begin{align*}
\pdydx{R}{t} &=  \left(\pdydx{F}{x}\left(x,y,u,p,q\right)
+ p\,\pdydx{F}{u}\left(x,y,u,p,q\right) \right)\pdydx{x}{s}
+\left(\pdydx{F}{y}\left(x,y,u,p,q\right) \right. \\
& \qquad \left. + q \pdydx{F}{u}\left(x,y,u,p,q\right) \right)\pdydx{y}{s}
+\pdydx{F}{p}\left(x,y,u,p,q\right)\pdydx{p}{s}
+\pdydx{F}{q}\left(x,y,u,p,q\right)\pdydx{q}{s} \\
&=\pdydx{F}{u}\left(x,y,u,p,q\right) \left(p\,\pdydx{x}{s}
+ q\pdydx{y}{s} \right)
+\pdydx{F}{x}\left(x,y,u,p,q\right)\pdydx{x}{s}
+\pdydx{F}{y}\left(x,y,u,p,q\right)\pdydx{y}{s} \\
&\qquad +\pdydx{F}{p}\left(x,y,u,p,q\right)\pdydx{p}{s}
+\pdydx{F}{q}\left(x,y,u,p,q\right)\pdydx{q}{s} \\
&=\pdydx{F}{u}\left(x,y,u,p,q\right) \left( p\,\pdydx{x}{s} +
q\pdydx{y}{s} - \pdydx{u}{s} \right)
+\pdydx{F}{x}\left(x,y,u,p,q\right)\pdydx{x}{s}
+\pdydx{F}{y}\left(x,y,u,p,q\right)\pdydx{y}{s} \\
&\qquad +\pdydx{F}{u}\left(x,y,u,p,q\right)\pdydx{u}{s}
+\pdydx{F}{p}\left(x,y,u,p,q\right)\pdydx{p}{s}
+\pdydx{F}{q}\left(x,y,u,p,q\right)\pdydx{q}{s} \\
&= - \pdydx{F}{u}\left(x,y,u,p,q\right) \,R +
\pdfdx{F\left(x,y,u,p,q\right)}{s}
\end{align*}
for all $(t,s)$.  It follows from (\ref{char_idxCON3}) that
\[
\pdfdx{F\left(x(t,s),y(t,s),u(t,s),p(t,s),q(t,s)\right)}{s} = 0
\]
for all $(t,s)$.  Thus, we get the separable ordinary differential equation
\[
\pdydx{R}{t}(t,s) =
-\pdydx{F}{u}\left(x(t,s),y(t,s),u(t,s),p(t,s),q(t,s)\right) \,R(t,s) \ .
\]
Its solutions is $\displaystyle R(t,s) = C e^{H(t,s)}$, where $C$ is a
constant and
\[
H(t,s) = -\int_0^t
\pdydx{F}{u}\left(x(\tau,s),y(\tau,s),u(\tau,s),p(\tau,s),q(\tau,s)\right)
\dx{\tau} \ .
\]
From (\ref{char_idxIN2}), we have $R(0,s) = 0$ for $s \in I$.  Thus
the constant $C$ is null and $R(t,s) = 0$ for all $t \in J_s$ and
$s\in I$.  This prove (\ref{char_idxCON1}).

\stage{v} The surface $u=u(x,y)$ contains $\Gamma$ by construction.
Moreover, (\ref{char_GPDE}) is satisfied by $u=u(x,y)$ because
(\ref{char_idxCON3}) is true, and
$u(t,s) = u(x,y)$, $\displaystyle p(t,s) = p(x,y) = \pdydx{u}{x}(x,y)$
and $\displaystyle q(t,s) = q(x,y) = \pdydx{u}{y}(x,y)$
for $x=x(t,s)$ and $y=y(t,s)$.
\end{proof}

\begin{rmk}
Given $p_0(s)$ and $q_0(s)$ for $s\in I$, the integral surface
containing $\Gamma$ is unique.
\end{rmk}

\begin{rmk}
In the previous proposition, we may only assume that     \label{char_PQCON}
(\ref{char_idxIN1}), (\ref{char_idxIN2}) and (\ref{char_GPDE_cond})
are satisfied for one value $s=\tilde{s}$.  In fact, the existence of $p_0(s)$
and $q_0(s)$ for $s\neq \tilde{s}$ is not even required.

Let
\begin{align*}
G(p,q,s) &= \dydx{u_0}{s}(s) - p \dydx{x_0}{s}(s) - q \dydx{y_0}{s}(s) \\
H(p,q,s) &= F(x_0(s),y_0(s),u_0(s),p,q)
\end{align*}
Since
\begin{align*}
& \frac{\partial(G.H)}{\partial(p,q)}
\bigg|_{(p,q,s)=(p_0(\tilde{s}),q_0(\tilde{s}),\tilde{s})} \\
&\quad = \left. \det
\begin{pmatrix}
\displaystyle -\dydx{x_0}{s}(s) &
\displaystyle -\dydx{y_0}{s}(s) \\[0.7em]
\displaystyle \pdydx{F}{p}(x_0(s),y_0(s),u_0(s),p,q) &
\displaystyle \pdydx{F}{q}(x_0(s),y_0(s),u_0(s),p,q) \\
\end{pmatrix} \right|_{(p,q,s)=(p_0(\tilde{s}),q_0(\tilde{s}),\tilde{s})}\\
&\quad \neq 0
\end{align*}
by hypothesis, we may use the Implicit Function
Theorem to find $p=p_0(s)$ and $q=q_0(s)$ for $s$ near $\tilde{s}$ such that
$G(p_0(s), q_0(s),s) = 0$ and $H(p_0(s),q_0(s),s)=0$; namely,
(\ref{char_idxIN1}) and (\ref{char_idxIN2}) are true for $s$ near $\tilde{s}$.
By continuity, (\ref{char_GPDE_cond}) will therefore be true for $s$
close enough to $\tilde{s}$.  By taking a small enough interval $I$
containing $\tilde{s}$, we have the hypothesis of the previous proposition.
\end{rmk}

\begin{rmk}
In the previous proposition, not only do we find the characteristic
curves but we also find the equation of the tangent plane of the
integral surface along the characteristic curves; namely,
\[
p(t,s)(x-x(t,s)) + q(t,s)(y-y(t,s)) = (u - u(t,s)) \ .
\]
This is the reason why the solutions of the characteristic equations
are called characteristic strips.
\end{rmk}

\begin{egg}
Solve the first order non-linear partial differential equation
\[
\left(\pdydx{u}{x}\right)^2 + \pdydx{u}{y} = 0
\]
with the condition $u(x,0)=x$.

This is a partial differential equation of the form (\ref{char_GPDE})
with $\displaystyle F(x,y,u,p,q)=p^2+q$.  Hence, the characteristic
equations are
\begin{align*}
\pdydx{x}{t} &= \pdydx{F}{p} = 2p \\
\pdydx{y}{t} &= \pdydx{F}{q} = 1 \\
\pdydx{u}{t} &= p\, \pdydx{F}{p} + q \pdydx{F}{q} = 2p^2 +q \\
\pdydx{p}{t} &= -\pdydx{F}{x} - p \, \pdydx{F}{u} = 0 \\
\pdydx{q}{t} &= -\pdydx{F}{y} - q \, \pdydx{F}{u} = 0
\end{align*}
with the initial condition $x(0,s) = x_0(s) = s$, $y(0,s) = y_0(s) = 0$
and $u(0,s) = u_0(s) = s$, and where $p_0(s)$ and
$q_0(s)$ are the solutions of
\[
F\left(x_0(s),y_0(s),u_0(s), p_0(s),q_0(s) \right) = p_0^2(s) + q_0(s) = 0
\]
and
\[
\dydx{u_0}{s}(s) - p_0(s)\dydx{x_0}{s}(s)- q_0(s)\dydx{y_0}{s}(s)
= 1 - p_0(s) = 0 \ .
\]
We find $p_0(s) = - q_0(s) = 1$ for all $s$.  Starting from the last
characteristic equation above, we find
$q(t,s) = -1$, $p(t,s) = 1$, $u(t,s) = t + s$, $y(t,s) = t$ and
$x(t,s) = 2t + s$.  Solving the equations for $x$ and $y$, we get
$t=y$ and $s=x-2y$.  Substituting these expressions of $s$ and $t$ in
$u$, $p$ and $q$, we find $u(x,y) = x -y$, $p(x,y) = 1$ and $q(x,y) = -1$.
Note that $\displaystyle \pdydx{u}{x}(x,y) = p(x,y) = 1$ and 
$\displaystyle \pdydx{u}{y}(x,y) = q(x,y) = -1$ for all $x$ and $y$ as
expected.
\end{egg}

\begin{egg}
Consider the wave equation                   \label{char_egg_eikonal}
\[
\pdydxn{u}{t}{2} = c^2(x,y) \Delta u \ ,
\]
where $c(x,y)$ is the speed of a wave at the point $(x,y)$ in the
plane.  As we will see, solutions of the wave equations are linear
combinations of terms of the form
$\displaystyle u_\omega = e^{i\omega t} \psi(x,y)$.
If $\omega = k c_0$ and $c(x,y) = c_0/n(x,y)$, where $c_0$ is the
average speed of a wave in the medium, then the wave equation gives
\begin{equation} \label{char_eikon1}
\Delta \psi + k^2 n^2(x,y) \psi = 0 \ .
\end{equation}
The function $n$ is called the
{\bfseries refraction index}\index{Refraction Index} and the
number $k$ is called the {\bfseries wave number}\index{Wave Number}.  The
{\bfseries wavelength}\index{Wavelength} is $1/(2\pi k)$.  In
particular, in the case of a lens, the wavelength is not more than one
micron while the other lengths (e.g.\ thickness and curvature of the
lens) are several millimetres.  Rescaling with respect to the largest
length in the system, we may assume that $k$ is very large compared to
the other lengths in the system.

We seek solutions of (\ref{char_eikon1}) of the form
$\displaystyle \psi(x,y) = \rho(x,y,k) e^{ik\phi(x,y)}$.
If we substitute this expression of $\psi$ in (\ref{char_eikon1}) and
divide the result by $\displaystyle e^{ik\phi(x,y)}$, we get
\[
\rho_{xx}+\rho_{yy} + 2ik\left(\rho_x\phi_x+\rho_y\phi_y\right)
+ ik\rho \left(\phi_{xx} + \phi_{yy}\right)
-k^2 \rho\left( \phi_x^2 + \phi_y^2\right) + k^2 n^2\rho = 0 \  .
\]
Thus,
\[
\rho\left( \phi_x^2 + \phi_y^2 - n^2\right) =
\frac{1}{k^2} \left(\rho_{xx}+\rho_{yy}\right) +
\frac{1}{k} \left( 2i\left(\rho_x\phi_x+\rho_y\phi_y\right)
+ i\rho \left(\phi_{xx} + \phi_{yy}\right) \right) =
O\left(\frac{1}{k}\right) \ .
\]
Since we assume that $k$ is much larger than the order lengths in the
system, we have approximately
\begin{equation} \label{char_eikonal}
\phi_x^2 + \phi_y^2 - n^2 = 0 \ .
\end{equation}
This partial differential equation is called the
{\bfseries eikonal equation}\index{Eikonal Equation} and was first
proposed by W. R. Hamilton in the nineteen century.  This equation is
fundamental in geometrical optics.  The surfaces defined by $\phi = a$,
where $a$ is a constant, are the {\bfseries wavefronts}\index{Wavefronts}.
More information on the eikonal equation can be found in
\cite{J,McO,PinRub}.

\stage{a}
We will solve the eikonal equation for $n$ constant (i.e.\ when the
speed $c(x,y)$ of a wave is independent of the position in the medium)
and under the initial conditions $\phi(x,2x)=1$.

The eikonal equation is of the form
$\displaystyle F(x,y,u,p,q) = p^2 + q^2 - n^2=0$ for $u=\phi$.  Hence, the
characteristic equations are
\begin{equation} \label{char_eik_charEQU}
\begin{split}
\pdydx{x}{t} &= \pdydx{F}{p}(x,y,u,p,q) = 2 p \\
\pdydx{y}{t} &= \pdydx{F}{q}(x,y,u,p,q) = 2 q \\
\pdydx{u}{t} &= p \pdydx{F}{p}(x,y,u,p,q) + q \pdydx{F}{q}(x,y,u,p,q)
= 2\left( p^2 + q^2 \right) \\
\pdydx{p}{t} &= -\pdydx{F}{x}(x,y,u,p,q) - p \, \pdydx{F}{u}(x,y,u,p,q) = 0 \\
\pdydx{q}{t} &= -\pdydx{F}{y}(x,y,u,p,q) - q \, \pdydx{F}{u}(x,y,u,p,q) = 0
\end{split}
\end{equation}
We have $\Gamma = \left\{ (x_0(s), y_0(s), u_0(s) : s \in I\right\}$,
where $x_0(s) = s$, $y_0(s) = 2s$ and $u_0(s) = 1$ for $s \in I=\RR$.  As for
$p_0(s)$ and $q_0(s)$, they satisfy
\[
F\left(x_0(s),y_0(s),u_0(s), p_0(s),q_0(s) \right) = p_0^2(s)+q_0^2(s)
- n^2 = 0
\]
and
\[
0 = \dydx{u_0}{s}(s) = p_0(s)\dydx{x_0}{s}(s)+ q_0(s)\dydx{y_0}{s}(s)
=p_0(s) + 2q_0(s)
\]
for $s \in \RR$.  We get $p_0(s) = -2 q_0(s)$ and
$q_0(s) = \pm n/\sqrt{5}$.

We have respectively that
\begin{align*}
\pdydx{p}{t}(t,s) &= 0 \quad \text{with} \quad p(0,s)=-2n/\sqrt{5}
\quad \Rightarrow \quad p(t,s) = -2n/\sqrt{5} \ , \\
\pdydx{q}{t}(t,s) &= 0 \quad \text{with} \quad q(0,s)=n/\sqrt{5}
\quad \Rightarrow \quad q(t,s) = n/\sqrt{5} \ , \\
\pdydx{x}{t}(t,s) &= 2p(t,s) = -4n/\sqrt{5} \quad \text{with} \quad
x(0,s)=s \quad \Rightarrow \quad x(t,s) = -4nt/\sqrt{5} + s \ , \\
\pdydx{y}{t}(t,s) &= 2q(t,s) = 2n/\sqrt{5} \quad \text{with} \quad
y(0,s)=2s \quad \Rightarrow \quad y(t,s) = 2nt/\sqrt{5} + 2s
\intertext{and}
\pdydx{u}{t}(t,s) &= 2\left( p^2(t,s) + q^2(t,s) \right) = 2n^2
\quad \text{with} \quad u(0,s)=1 \quad \Rightarrow \quad
u(t,s) = 2n^2t + 1 \ .
\end{align*}                   
We solve $x = -4nt/\sqrt{5} + s$ and $y=2nt/\sqrt{5} + 2s$ in
terms of $s$ and $t$ to get
$\displaystyle t = (y-2x)/(2n\sqrt{5})$ and
$\displaystyle s = (x+2y)/5$.
If we substitute these values of $s$ and $t$ in the
expression for $u$, we get the integral surface
$\displaystyle u(x,y) = (n(y-2x))/\sqrt{5} +1$.
A similar reasoning with $p_0(s) = 2n/\sqrt{5}$ and
$q_0(s) = -n/\sqrt{5}$ gives
$\displaystyle u(x,y) = -(n(y-2x))/\sqrt{5} +1$.

\stage{b}  The possible values of $p_0(s)$, $q_0(s)$ on the curve
$\Gamma = \left\{ (x_0(s), y_0(s), u_0(s) : s \in \RR\right\}$ 
are given by
\[
F\left(x_0(s),y_0(s),u_0(s), p_0(s),q_0(s) \right) = p_0^2(s)+q_0^2(s)
- n^2 = 0
\]
and
\[
u_0'(s) = \dydx{u_0}{s}(s) = p_0(s)\dydx{x_0}{s}(s)+ q_0(s)\dydx{y_0}{s}(s)
= x_0'(s) p_0(s) + y_0'(s) q_0(s) \ .
\]
The first equation says that $\displaystyle \|(p_0(s),q_0(s))\|_2 = n$
and the second says that the projection of
$(x_0'(s),y_0'(s))$ on $(p_0(s),q_0(s))$ is
\[
\frac{ \ps{(x_0'(s),y_0'(s))}{(p_0(s),q_0(s))} }{\|(p_0(s),q_0(s))\|}
= \frac{ u_0'(s) }{n} \ .
\]
For this to be possible, the norm of the vector
$(x_0'(s),y_0'(s))$ must be greater or equal than $|u_0'(s)|/n$ as can
be seen in Figure~\ref{char_eik_PROJ} (or from Schwarz inequality).
Thus, we need
\begin{equation} \label{char_eik_EC}
\left(x_0'(s)\right)^2 + \left(y_0'(s)\right)^2
\geq \left(\frac{u_0'(s)}{n}\right)^2 \ .
\end{equation}
When (\ref{char_eik_EC}) is satisfied, we say that $\Gamma$ is
spacelike and the Cauchy problem has at least one solution.  When
(\ref{char_eik_EC}) is not satisfied, we say that $\Gamma$ is timelike
and the Cauchy problem does not have any solution.

In part (a) above, we had $x_0(s) = s$, $y_0(s) = 2s$ and
$u_0(s) = 1$.  So (\ref{char_eik_EC}) is strictly satisfied and the
curve $\Gamma$ is spacelike.  This explains the two
solutions that we have found.

\pdfF{characteristics/char_eik_proj}{The projection for the eikonal
equation}{The projection of the vector $(x_0'(s),y_0'(s))$ on the
vector $(p_0(s), q_0(s))$ is of norm smaller or equal to
$\|p_0(s),q_0(s)\|$.}{char_eik_PROJ}

\stage{c} If we consider the curve
$\Gamma = \left\{ (x_0(s), y_0(s), u_0(s) : s \in I\right\}$, where
$x_0(s)= \cos(s)$, $y_0(s) = \sin(s)$ and $u_0(s)=0$ for $s \in I = \RR$.
Since (\ref{char_eik_EC}) is strictly satisfied, $\Gamma$ is spacelike
and we should find at least one solution; in fact, we will find two
solutions.

$p_0(s)$ and $q_0(s)$ satisfy
\[
F\left(x_0(s),y_0(s),u_0(s), p_0(s),q_0(s) \right) = p_0^2(s)+q_0^2(s)
- n^2 = 0
\]
and
\[
0 = \dydx{u_0}{s}(s) = p_0(s)\dydx{x_0}{s}(s)+ q_0(s)\dydx{y_0}{s}(s)
= - p_0(s)\,\sin(s) + q_0(s)\, \cos(s)
\]
for $s \in \RR$.  We get $p_0(s) = n\cos(s)$ and $q_0(s) = n\sin(s)$,
or $p_0(s) = -n\cos(s)$ and $q_0(s) = -n\sin(s)$.

We still have the characteristic equation given in
(\ref{char_eik_charEQU}).  
\begin{align*}
\pdydx{p}{t}(t,s) &= 0 \quad \text{with} \quad p(0,s)=n\cos(s)
\quad \Rightarrow \quad p(t,s) = n\cos(s) \ , \\
\pdydx{q}{t}(t,s) &= 0 \quad \text{with} \quad q(0,s)=n\sin(s)
\quad \Rightarrow \quad q(t,s) = n\sin(s) \ , \\
\pdydx{x}{t}(t,s) &= 2p(t,s) = 2n\cos(s) \quad \text{with} \quad
x(0,s)=\cos(s) \quad \Rightarrow \quad x(t,s) = (2nt+1)\cos(s) \ , \\
\pdydx{y}{t}(t,s) &= 2q(t,s) = 2n\sin(s) \quad \text{with} \quad
y(0,s)=\sin(s) \quad \Rightarrow \quad y(t,s) = (2nt+1)\sin(s)
\intertext{and}
\pdydx{u}{t}(t,s) &= 2\left( p^2(t,s) + q^2(t,s) \right) = 2n^2
\quad \text{with} \quad u(0,s)=0 \quad \Rightarrow \quad
u(t,s) = 2n^2t \ .
\end{align*}                   

We can solve $x = (2nt+1)\cos(s)$ and $y=(2nt+1)\sin(s)$ in
terms of $s$ and $t$ to get
\[
t = \frac{\sqrt{x^2+y^2}}{2n} - \frac{1}{2n} \quad \text{and} \quad
s =
\begin{cases}
\displaystyle \arctan\left(\frac{y}{x}\right) & \quad x>0 \\
\displaystyle \frac{\pi}{2} & \quad x=0, y>0 \\
\displaystyle \arctan\left(\frac{y}{x}\right) + \pi & \quad x<0 \\
\displaystyle \frac{3\pi}{2} & \quad x=0, y<0
\end{cases}
\]
If we substitute these values of $s$ and $t$ in the
expression for $u$, we get the integral surface
\[
u(x,y) = n\sqrt{x^2+y^2} - n \ .
\]

A similar reasoning with $p_0(s) = -n\cos(s)$ and $q_0(s) = -n\sin(s)$
yields
\[
u(x,y) = -n\sqrt{x^2+y^2} + n \ .
\]

We have that $\displaystyle x^2+y^2 = (1 \pm 2nt)^2$ represents the
wavefront because $u$ is constant on these circles.
$\displaystyle x^2+y^2 = (1 + 2nt)^2$ is the wavefront moving away of
the origin as $t$ increases while
$\displaystyle x^2+y^2 = (1 - 2nt)^2$ is the wavefront moving toward
the origin to reach it at $t= 1/2n$.
\end{egg}

\begin{rmk}
The eikonal equation is of the form
$\displaystyle F(x,y,u,p,q) = p^2 + q^2 - n^2=0$.  Because of the
special form of this equation, we can describe the Monge Cones for
this problem.  They are cones with an axis parallel to the $u$ axis
and such that the angle $\theta$ between the axis and the side of the
cone satisfies $\cot(\theta)=n$ (Figure~\ref{char_DIR_DER}).
Effectively, each vector $(p,q,-1)$ with $\displaystyle p^2+q^2=n^2$ is
perpendicular to a tangent plane enveloping the Monge Cone.
Moreover, the angle $\psi$ between the vectors $(p,q,-1)$ and the
vertical axis satisfies $\displaystyle \tan(\psi) = \sqrt{p^2+q^2}$.  Since a
rotation by $\pi/2$ in the vertical plane containing the vector
$(p,q,-1)$ and the vertical axis transforms this vector into a vector
in the direction of the line of intersection between the Monge Cone
and the tangent plane perpendicular to $(p,q,-1)$, the angle $\theta$
satisfies $\displaystyle \cot(\theta) = \sqrt{p^2 + q^2} = n$.
\end{rmk}

\pdfF{characteristics/char_dir_der}{The Monge Cones for the eikonal
equation}{Schematic representation of a Monge Cone for the eikonal
equation.  The curve $\Gamma$ is spacelike.  The curve $\Gamma$ is
inside the cone when $\Gamma$ is timelike.}{char_DIR_DER}

\section{Solution Generated as Envelope}

Before addressing the subject of this section, we review what we know
about the envelope of a family of integral surfaces.  We suppose that
\[
u = f(x,y,\lambda) \quad , \quad \lambda \in \RR \ ,
\]
describes a one-parameter family of integral surfaces for
\ref{char_GPDE}.  Moreover, we assume that its envelope $E$ is given
by $u = f(x,y,\lambda(x,y))$ for some function
$\displaystyle \lambda: \RR^2 \to \RR$.

For each point
$\displaystyle \left(\tilde{x},\tilde{y},\tilde{u}\right) \in E$, we have from
Remark~\ref{char_env_tgp} that the tangent plane to the envelope $E$
at $\displaystyle \left(\tilde{x},\tilde{y},\tilde{u}\right)$ is also
the tangent plane of the integral surface given by
$u= f(x,y,\tilde{\lambda})$ at 
$\displaystyle \left(\tilde{x},\tilde{y},\tilde{u}\right)$, where
$\displaystyle \tilde{\lambda} = \lambda\left(\tilde{x},\tilde{y}\right)$.

Moreover, if
$\displaystyle h(x,y) = f\left(x,y,\lambda(x,y)\right)$, then
\[
\pdydx{h}{x}\left(\tilde{x},\tilde{y}\right) =
\pdydx{f}{x}\left(\tilde{x},\tilde{y},\tilde{\lambda}\right) \quad
\text{and} \quad \pdydx{h}{y}\left(\tilde{x},\tilde{y}\right) =
\pdydx{f}{y}\left(\tilde{x},\tilde{y},\tilde{\lambda}\right)
\]
by Remark~\ref{char_env_tgp}.
Thus,
\[
F\left(\tilde{x},\tilde{y},h(\tilde{x},\tilde{y}),
\pdydx{h}{x}(\tilde{x},\tilde{y}),
\pdydx{h}{y}(\tilde{x},\tilde{y})\right) 
= F\left(\tilde{x},\tilde{y},f(\tilde{x},\tilde{y},\tilde{\lambda}),
\pdydx{f}{x}(\tilde{x},\tilde{y},\tilde{\lambda}),
\pdydx{f}{y}(\tilde{x},\tilde{y},\tilde{\lambda})\right) = 0 \ .
\]
Since $\left(\tilde{x},\tilde{y}\right)$ is arbitrary, we get that
the envelope of a family of integral surfaces for
(\ref{char_GPDE}) is also an integral surface for (\ref{char_GPDE}).

We now explain how we may find the integral surface of
(\ref{char_GPDE}) containing a specific curve $\Gamma$ as the envelope
of a family of integral surfaces of (\ref{char_GPDE}).  We first
need a method to find a proper family of integral surfaces.

A {\bfseries complete integral}%
\index{First Order Partial Differential Equation!Complete Integral} of
(\ref{char_GPDE}) is a two-parameter family of surfaces
\begin{equation} \label{char_cint}
u = G(x,y,\alpha, \beta)
\end{equation}
satisfying (\ref{char_GPDE}) and such that the mapping
\begin{equation} \label{char_ci_rank}
\begin{split}
\RR^2 &\rightarrow \RR^3  \\
(\alpha,\beta) &\mapsto \left(G(x,y,\alpha,\beta),
\pdydx{G}{x}(x,y,\alpha,\beta) , \pdydx{G}{y}(x,y,\alpha,\beta) \right)
\end{split}
\end{equation}
is of rank $2$ for all $(x,y)$.  This implies that the parameters
$\alpha$ and $\beta$ are independent (i.e.\ there is no functional
relation between them).  There is not any easy method to find complete
integrals.

Given a complete integral like (\ref{char_cint}) for the partial
differential equation (\ref{char_GPDE}), we may define many
one-parameter sub-families of integral surfaces.  Suppose that
\[
  u = f(x,y,s) = G\left(x,y,\alpha(s),\beta(s)\right)
\]
for $\alpha, \beta : I \to \RR$ describes a one-parameter
sub-family of integral surfaces.  We obviously have that
$F(x,y,u,p,q) = 0$ if $u = f(x,y,s)$,
$\displaystyle
p = \pdydx{f}{x}(x,y,s) = \pdydx{G}{x}\left(x,y,\alpha(s), \beta(s)\right)$
and
$\displaystyle
q = \pdydx{f}{y}(x,y,s) = \pdydx{G}{y}\left(x,y,\alpha(s), \beta(s)\right)$
for all $s \in I$.  The envelope of this sub-family of integral
surfaces is deduced from
\begin{align*}
u &= f(x,y,s) = G\left(x,y,\alpha(s),\beta(s)\right)
\intertext{and}
0 &= \pdfdx{f}{s}(x,y,s)
= \pdydx{G}{\alpha}\left(x,y,\alpha(s), \beta(s)\right)\, \dydx{\alpha}{s}(s)
+ \pdydx{G}{\beta}\left(x,y,\alpha(s), \beta(s)\right)\,\dydx{\beta}{s}(s)
\end{align*}
by solving for $s$ as a function of $x$ and $y$.

The complete integral can be used to find an integral surface
containing a curve $\displaystyle \Gamma = \{x_0(s),y_0(s),u_0(s) : s \in I\}$
for some open interval $I$.  The first step is to extract from the
complete integral a one-parameter family of integral surfaces
defined by $\displaystyle u = G\left(x,y,\alpha(s), \beta(s)\right)$
such that $u_0(s) = G\left(x_0(s),y_0(s), \alpha(s), \beta(s)\right)$
for $s \in I$.
The envelope of this one-parameter family of integral surface is the
integral surface containing the curve $\Gamma$.  In other words, we
first find a curve
\begin{align*}
I & \rightarrow \RR^2 \\
s &\mapsto (\alpha(s), \beta(s))
\end{align*}
such that
\begin{align}
u_0(s) &= G\left(x_0(s), y_0(s), \alpha(s), \beta(s)\right) \label{char_compl1}
\intertext{and}
\dydx{u_0}{s}(s)
&= \pdydx{G}{x}\left(x_0(s),y_0(s), \alpha(s), \beta(s)\right)\,
\dydx{x_0}{s}(s) \nonumber \\
&\qquad + \pdydx{G}{y}\left(x_0(s),y_0(s), \alpha(s), \beta(s)\right)\,
\dydx{y_0}{s}(s)
\label{char_compl2}
\end{align}
for $s \in I$, and then use
\begin{align}
u &= G(x,y,\alpha(s),\beta(s)) \label{char_compl3}
\intertext{and}
0 &= \pdfdx{G}{s}\left(x,y,\alpha(s),\beta(s) \right) \nonumber \\
&= \pdydx{G}{\alpha}\left(x,y, \alpha(s), \beta(s)\right)\,\dydx{\alpha}{s}(s)
+ \pdydx{G}{\beta}\left(x,y, \alpha(s), \beta(s)\right)\,\dydx{\beta}{s}(s)
\label{char_compl4}
\end{align}
to find the envelope of the sub-family of surfaces described by
$\displaystyle u = G\left(x, y, \alpha(s), \beta(s)\right)$
that will contains $\Gamma$.

To prove that each point
$\displaystyle \left(\tilde{x},\tilde{y},\tilde{u}\right) \in \Gamma$
is on the envelope of the family of integral surfaces given by
$u = G(x,y,\alpha(s),\beta(s))$ for $s \in I$, we note that
\[
\left(\tilde{x},\tilde{y},\tilde{u}\right)
= \left(x_0\left(\tilde{s}\right),y_0\left(\tilde{s}\right),
u_0\left(\tilde{s}\right)\right)
\]
for some $\tilde{s} \in I$.  It follows from (\ref{char_compl1}) that
\[
\tilde{u} = u_0\left(\tilde{s}\right) =
G\left(x_0\left(\tilde{s}\right),y\left(\tilde{s}\right),
\alpha\left(\tilde{s}\right),\beta\left(\tilde{s}\right)\right) =
G\left(\tilde{x},\tilde{y},\alpha\left(\tilde{s}\right),
\beta\left(\tilde{s}\right)\right) \ .
\]
Hence, (\ref{char_compl3}) is satisfied by
$\displaystyle \left(\tilde{x},\tilde{y},\tilde{u}\right)$
and $s=\tilde{s}$.

If we derive (\ref{char_compl1}) with respect to $s$, we get
\begin{align*}
\dydx{u_0}{s}(s)
&= \pdydx{G}{x}\left(x_0(s), y_0(s), \alpha(s), \beta(s)\right)
\dydx{x_0}{s}(s) + \pdydx{G}{y}\left(x_0(s), y_0(s), \alpha(s), \beta(s)\right)
\dydx{y_0}{s}(s) \\
&\quad + \pdydx{G}{\alpha}\left(x_0(s), y_0(s), \alpha(s), \beta(s)\right)
\dydx{\alpha}{s}(s)
+ \pdydx{G}{\beta}\left(x_0(s), y_0(s), \alpha(s), \beta(s)\right)
\dydx{\beta}{s}(s) \\
&= \dydx{u_0}{s}(s)
+ \pdydx{G}{\alpha}\left(x_0(s), y_0(s), \alpha(s), \beta(s)\right)
\dydx{\alpha}{s}(s)
+ \pdydx{G}{\beta}\left(x_0(s), y_0(s), \alpha(s), \beta(s)\right)
\dydx{\beta}{s}(s) ,
\end{align*}
where we have used (\ref{char_compl2}) to get the second equality.  Hence,
for $s=\tilde{s}$, we get
\begin{align*}
0 &= \pdydx{G}{\alpha}\left(x_0\left(\tilde{s}\right),
y_0\left(\tilde{s}\right), \alpha\left(\tilde{s}\right),
\beta\left(\tilde{s}\right)\right)\,\dydx{\alpha}{s}\left(\tilde{s}\right) +
\pdydx{G}{\beta}\left(x_0\left(\tilde{s}\right), y_0\left(\tilde{s}\right),
\alpha\left(\tilde{s}\right), \beta\left(\tilde{s}\right)\right)\,
\dydx{\beta}{s}\left(\tilde{s}\right) \\
&= \pdydx{G}{\alpha}\left(\tilde{x}, \tilde{y},\alpha\left(\tilde{s}\right),
\beta\left(\tilde{s}\right)\right)\, \dydx{\alpha}{s}\left(\tilde{s}\right)
+ \pdydx{G}{\beta}\left(\tilde{x}, \tilde{y},\alpha\left(\tilde{s}\right),
\beta\left(\tilde{s}\right)\right)\, \dydx{\beta}{s}\left(\tilde{s}\right)
= \pdfdx{G}{s}\left(\tilde{x}, \tilde{y}, \alpha\left(\tilde{s}\right),
\beta\left(\tilde{s}\right)\right) \ .
\end{align*}
Thus, (\ref{char_compl4}) is satisfied by
$\displaystyle \left(\tilde{x},\tilde{y},\tilde{u}\right)$
and $s=\tilde{s}$.

\begin{rmk}
The one-parameter family of surfaces $u=G(x,y,\alpha(s),\beta(s))$
defined by (\ref{char_compl1}) and (\ref{char_compl2}) satisfies
\[
F\left(x,y,G(x,y,\alpha(s),\beta(s)),
\pdydx{G}{x}(x,y,\alpha(s),\beta(s)),
\pdydx{G}{y}(x,y,\alpha(s),\beta(s))\right) = 0
\]
and
\begin{align*}
&\left(\dydx{x_0}{s}(s) ,\dydx{y_0}{s}(s),\dydx{u_0}{s}(s)\right) \\
&\quad
\cdot \left(\pdydx{G}{x}\left(x_0(s),y_0(s),\alpha(s),\beta(s)\right),
\pdydx{G}{y}\left(x_0(s),y_0(s),\alpha(s),\beta(s)\right),-1\right) = 0
\end{align*}
for $s\in I$.  In particular, if
$\displaystyle p_0(s) = \pdydx{G}{x}(x_0(s),y_0(s),u_0(s),\alpha(s),\beta(s))$
and \\
$\displaystyle q_0(s) = \pdydx{G}{y}(x_0(s),y_0(s),u_0(s),\alpha(s),\beta(s))$,
then
\begin{align*}
F\left(x_0(s),y_0(s),u_0(s),p_0(s),q_0(s)\right) &= 0
\intertext{and}
\left(\dydx{x_0}{s}(s) ,\dydx{y_0}{s}(s),\dydx{u_0}{s}(s)\right)
\cdot\left(p_0(s), q_0(s),-1\right) & = 0 \ .
\end{align*}
The last equation says that the normal direction
$\displaystyle \left(p_0(s), q_0(s),-1\right)$ of the
tangent plane to the integral surface of (\ref{char_GPDE}) along the
curve $\Gamma$ is perpendicular to the tangential direction
$\displaystyle
\left(\dydx{x_0}{s}(s),\dydx{y_0}{s}(s),\dydx{u_0}{s}(s)\right)$ along
$\Gamma$.

The possible values for $p_0(s)$ and $q_0(s)$ are therefore given by
the intersection of the perpendicular plane to 
$\displaystyle
\left(\dydx{x_0}{s}(s),\dydx{y_0}{s}(s),\dydx{u_0}{s}(s)\right)$
with the ``cone'' produced by the vector $(p.q.-1)$ satisfying
\[
F(x_0(s),y_0(s),u_0(s),p,q)=0
\]
as it is shown in Figure~\ref{char_fig3}.  In
Figure~\ref{char_fig3}, the cone and the perpendicular plane have been
translated from the origin to $(x_0(s),y_0(s),u_0(s))$.
\end{rmk}

\pdfF{characteristics/char_fig3}{Possible initial tangent planes to
the integral surfaces}{The possible initial tangent planes to the
integral surfaces for the general Cauchy problem (\ref{char_GPDE}) are
associated to the possible normal direction marked by the four thick
dots.}{char_fig3}

\begin{rmk}
Given a complete integral $u=G(x,y,\alpha,\beta)$ and a curve $\Gamma$
defined by
\[
\left\{ (x,y,u) = (x_0(s), y_0(s), u_0(s)) : s \in I \right\} \ ,
\]
where $I$ is an open interval, consider the mapping
\begin{align*}
H:I \times \RR^2 & \rightarrow \RR^2 \\
\begin{pmatrix}
s \\
\alpha \\
\beta
\end{pmatrix}
&\mapsto
\begin{pmatrix}
u_0(s) - G(x_0(s), y_0(s), \alpha, \beta) \\[0.7em]
\displaystyle  \dydx{u_0}{s}(s) - \pdydx{G}{x}(x_0(s),y_0(s), \alpha, \beta)\,
\dydx{x_0}{s}(s) + \pdydx{G}{y}(x_0(s),y_0(s), \alpha, \beta)\,
\dydx{y_0}{s}(s)
\end{pmatrix} \ .
\end{align*}
Suppose that
$\displaystyle \left(\tilde{s}, \tilde{\alpha}, \tilde{\beta}\right)
\in I \times \RR^2$ is a solution of
\begin{equation} \label{char_impl_f_th}
H\begin{pmatrix} s \\ \alpha \\ \beta \end{pmatrix} =
\begin{pmatrix} 0 \\ 0 \end{pmatrix}  \ .
\end{equation}
The equation in (\ref{char_impl_f_th}) is equivalent to the union of
equations in (\ref{char_compl1}) and (\ref{char_compl2}).

We now prove that there exists a differentiable curve
$s\mapsto (\alpha(s),\beta(s))$ defined for $s$ in a neighbourhood
$J \subset I$ of $\tilde{s}$ such that
$\displaystyle \left(\alpha\left(\tilde{s}\right),
\beta\left(\tilde{s}\right)\right)
= (\tilde{\alpha},\tilde{\beta})$ and
\begin{equation} \label{char_impl_f_thLocal}
  H\begin{pmatrix} s \\ \alpha(s) \\ \beta(s) \end{pmatrix} =
\begin{pmatrix} 0 \\ 0 \end{pmatrix}
\end{equation}
for $s \in J$.  Thus, $s\mapsto (\alpha(s),\beta(s))$ satisfies
(\ref{char_compl1}) and (\ref{char_compl2}) for $s \in J$.

Since the mapping defined in (\ref{char_ci_rank}) is of rank $2$, we
(generally) have that
\begin{align*}
&\det \diff_{\alpha,\beta} H(\tilde{s},\alpha,\beta)
\bigg|_{(\alpha,\beta) = (\tilde{\alpha},\tilde{\beta})}
= \det \begin{pmatrix}
\displaystyle \pdydx{G}{\alpha} & \displaystyle \pdydx{G}{\beta} \\[0.7em]
\displaystyle \pdydxnm{G}{x}{\alpha}{2}{}{}\, \dydx{x_0}{s}
+ \pdydxnm{G}{y}{\alpha}{2}{}{}\, \dydx{y_0}{s} &
\displaystyle \pdydxnm{G}{x}{\beta}{2}{}{}\, \dydx{x_0}{s}
+ \pdydxnm{G}{y}{\beta}{2}{}{}\, \dydx{y_0}{s} &
\end{pmatrix} \\
&\qquad \qquad = \left( \pdydx{G}{\alpha} \pdydxnm{G}{x}{\beta}{2}{}{}
- \pdydx{G}{\beta} \pdydxnm{G}{x}{\alpha}{2}{}{} \right)
\dydx{x_0}{s} + \left(\pdydx{G}{\alpha} \pdydxnm{G}{y}{\beta}{2}{}{}
- \pdydx{G}{\beta} \pdydxnm{G}{y}{\alpha}{2}{}{} \right)
\dydx{y_0}{s} \neq 0 \ ,
\end{align*}
where $G$ and its derivatives are evaluated at
$\displaystyle (x,y,\alpha,\beta)
= \left(x_0\left(\tilde{s}\right), y_0\left(\tilde{s}\right),
\tilde{\alpha}, \tilde{\beta} \right)$,
and $\displaystyle \dydx{y_0}{s}$ and $\displaystyle \dydx{y_0}{s}$
are evaluated at $s = \tilde{s}$.  The content of at
least one of the two parentheses in the expression above is non-null
because the mapping defined in (\ref{char_ci_rank}) is of rank $2$.
It then follows from the implicit function theorem that there exists a
unique differentiable curve $s\mapsto (\alpha(s),\beta(s))$, defined
in an open neighbourhood $J$ of $\tilde{s}$, such that
(\ref{char_impl_f_thLocal}) is satisfied.
and $\displaystyle \left(\alpha\left(\tilde{s}\right)
\beta\left(\tilde{s}\right)\right)
= (\tilde{\alpha},\tilde{\beta})$ as expected.
\end{rmk}

\begin{egg}
Find the complete integral of
\begin{equation} \label{char_egg_ms}
\pdydx{u}{x}\,\pdydx{u}{y} = u
\end{equation}
The function $u(x,y)=0$ for all $x$ and $y$ satisfies
(\ref{char_egg_ms}) and the condition $u(x,x)=0$.  Are there
non-trivial solutions to (\ref{char_egg_ms}) that satisfy $u(x,x)=0$?
Use the complete integral to find a non-trivial solution $u$ of
(\ref{char_egg_ms}) which satisfies $u(x,x) = 0$.

To find the complete integral, we use the method of separation of
variables.  If we substitute $u(x,y) = F(x)G(y)$ in
(\ref{char_egg_ms}), we get $F'(x)G(y)\, F(x)G'(y) = F(x)G(y)$.  If
we divide both sides by $F(x)G(y)$, we get $F'(x) \, G'(y) = 1$. Thus
$\displaystyle F'(x) = \frac{1}{G'(y)}$.  Since the right hand side is
independent of $x$ and the left hand side is independent of $y$, we
must have $\displaystyle F'(x) = \frac{1}{G'(y)} = a$, where $a$ is a
constant.  We get two first order ordinary differential equation:
$F'(x)= a$ and $G'(y) = 1/a$.
Their general solutions are $F(x) = ax+b$ and $G(y) = y/a + c$.  Hence,
\[
u(x,y) = (ax+b)\left(\frac{y}{a} + c\right)
= xy + ac\, x + \frac{b}{a} \, y + bc
= xy + \alpha x + \beta y + \alpha \beta
\]
for $\alpha = ac$ and $\beta = b/a$.  The function
\[
u = G(x,y,\alpha,\beta) = xy + \alpha x + \beta y + \alpha \beta
\]
is a complete integral because
\[
\det
\begin{pmatrix}
\displaystyle \pdydxnm{G}{x}{\alpha}{2}{}{} &
\displaystyle \pdydxnm{G}{x}{\beta}{2}{}{} \\[0.7em]
\displaystyle \pdydxnm{G}{y}{\alpha}{2}{}{} &
\displaystyle \pdydxnm{G}{y}{\beta}{2}{}{}
\end{pmatrix}
= \det
\begin{pmatrix}
1 & 0 \\ 0 & 1
\end{pmatrix} = 1 \neq 0 \ .
\]
Thus, the mapping defined in (\ref{char_ci_rank}) is of rank $2$.

To find a non-trivial solution $u$ of (\ref{char_egg_ms}) which
satisfies $u(x,x) = 0$. we first find a family of integral surfaces
that touch the curve $\Gamma = \{ (x,x,0) : x \in \RR \}$
defined by $x=s$, $y=s$ and $u=0$ for $s \in \RR$.  This
family is defined by the system of equations
\begin{align*}
0 &= s^2 + \alpha \, s + \beta \, s + \alpha \, \beta \\
0 &= 2s + \alpha + \beta
\end{align*}
obtained from (\ref{char_compl3}) and (\ref{char_compl4}).
We get $\alpha(s) = -s$ and $\beta(s) = -s$.  The one-parameter family
of integral surfaces that we are looking for is given by
\[
u = G(x,y, -s, -s) = xy -s x -s y + s^2 \ .
\]
The envelope of this family of integral surfaces is given by the
system of equations
\begin{align*}
u &= G(x,y, -s, -s) = xy - s x - s y + s^2 \\
0 &= \pdydx{G(x,y, -s,-s)}{s} = -x - y + 2s 
\end{align*}
obtained from (\ref{char_env1}) and (\ref{char_env2}).
From the second equation, we find $s = (x+y)/2$.  If we substitute
this value of $s$ in the first equation, we get the required integral
surface
\[
u = xy - \frac{x(x+y)}{2} - \frac{y(x+y)}{2} + \frac{(x+y)^2}{4}
= - \frac{(x-y)^2}{4}
\]
contained $\Gamma$.
\end{egg}

As the following two examples illustrate, the method of separation of
variables does not always give a complete integral.

\begin{egg}
Use the method of separation of variables to find a solution of
\[
\pdydx{u}{x} + \pdydx{u}{y} = 0 \ ,
\]
and show that it is not a complete integral.

Let $u(x,y) = F(x)G(y)$.  We get $F'(x) G(y) + F(x) G'(y) = 0$.  If we
divide both sides of this equality by $F(x)G(y)$ and isolate the
functions of $x$ on one side and those of $y$ on the other side, we
get
\[
\frac{F'(x)}{F(x)} = - \frac{G'(y)}{G(y)}
\]
for all $x$ and $y$ such that $F(x) \neq 0$ and $G(x) \neq 0$.  Since
the right hand side is independent of $x$ and the left hand side is
independent of $y$, we must have
\[
\frac{F'(x)}{F(x)} = - \frac{G'(y)}{G(y)} = \alpha \ ,
\]
where $\alpha$ is a constant.  We get two first order ordinary
differential equations, $F'(x) -\alpha F(x) = 0$ and
$G'(y) + \alpha G(y) = 0$,
whose general solutions are $F(x) = \beta_1 e^{\alpha x}$ and
$\displaystyle G(y) = \beta_2 e^{-\alpha y}$ respectively.  A solution is
$\displaystyle u(x,y) = F(x)G(y) = \beta e^{\alpha(x-y)}$, where
$\beta = \beta_1 \beta_2$.

However, we have
\[
\det
\begin{pmatrix}
\displaystyle \pdydxnm{u}{x}{\alpha}{2}{}{} &
\displaystyle \pdydxnm{u}{x}{\beta}{2}{}{} \\[0.7em]
\displaystyle \pdydxnm{u}{y}{\alpha}{2}{}{} &
\displaystyle \pdydxnm{u}{y}{\beta}{2}{}{}
\end{pmatrix}
= \det
\begin{pmatrix}
\beta (1+\alpha (x-y)) e^{\alpha(x-y)} & \alpha e^{\alpha(x-y)} \\
-\beta(1+\alpha (x-y)) e^{\alpha(x-y)} & -\alpha e^{\alpha(x-y)}
\end{pmatrix} = 0
\]
for all $x$ and $y$. Thus, (\ref{char_ci_rank}) is not satisfied.
\end{egg}

\begin{egg}
Use the method of separation of variables to find a solution of
\[
x\pdydx{u}{x} - y\pdydx{u}{y} = 0 \ ,
\]
and show that it is not a complete integral.

Let $u(x,y) = F(x)G(y)$.  We get $xF'(x) G(y) - y F(x) G'(y) = 0$.  If we
divide both sides of this equality by $F(x)G(y)$ and isolate the
functions of $x$ on one side and those of $y$ on the other side, we
get
\[
x\frac{F'(x)}{F(x)} = y\frac{G'(y)}{G(y)}
\]
for all $x$ and $y$ such that $F(x) \neq 0$ and $G(x) \neq 0$.  Since
the right hand side is independent of $x$ and the left hand side is
independent of $y$, we must have
\[
x \frac{F'(x)}{F(x)} = y \frac{G'(y)}{G(y)} = \alpha \ ,
\]
where $\alpha$ is a constant.  We get two first order ordinary
differential equations of the same form:
$x F'(x) - \alpha F(x) = 0$ and $y G'(y) - \alpha G(y) = 0$.
If we consider the first differential equation, we may write
\[
\frac{F'(x)}{F(x)} = \frac{\alpha}{x} \ .
\]
Integrating both sides with respect to $x$ gives
$\ln|F(x)| = \alpha \ln|x| + \gamma_1$
for some constant $\gamma_1$.  Thus
$\displaystyle F(x) = \beta_1 |x|^\alpha$, where
$\displaystyle \beta_1 = \pm e^{\gamma_1}$.  We may also accept
$\beta_1 = 0$ to get
the trivial solution $u(x,y)=0$ for all $x$ and $y$.  Similarly,
$\displaystyle G(y) = \beta_2 |y|^\alpha$.   A solution is
$u(x,y) = F(x)G(y) = \beta |xy|^{\alpha}$, where
$\beta = \beta_1 \beta_2$.

However, we have
\[
\det
\begin{pmatrix}
\displaystyle \pdydxnm{u}{x}{\alpha}{2}{}{} &
\displaystyle \pdydxnm{u}{x}{\beta}{2}{}{} \\[0.7em]
\displaystyle \pdydxnm{u}{y}{\alpha}{2}{}{} &
\displaystyle \pdydxnm{u}{y}{\beta}{2}{}{}
\end{pmatrix}
= \det
\begin{pmatrix}
\beta (1+\alpha\ln(x y)) x^{\alpha-1}y^\alpha &
\alpha x^{\alpha-1}y^\alpha \\
\beta (1+\alpha\ln(x y)) x^\alpha y^{\alpha-1} &
\alpha x^\alpha y^{\alpha-1}
\end{pmatrix} = 0
\]
for all $x > 0$ and $y > 0$ with similar results for the other
choices of $x$ and $y$.   Thus, (\ref{char_ci_rank}) is not
satisfied.
\end{egg}

\begin{egg}
We revisit the eikonal equation of Example~\ref{char_egg_eikonal} with
our knowledge of envelope and complete integral.

The family of integral surfaces for the eikonal equation given by
\[
u = G(x,y,\alpha,\beta) = n(x\cos(\alpha) + y \sin(\alpha)) + \beta
\]
is a complete integral because
\[
\begin{split}
\RR^2 &\rightarrow \RR^3  \\
(\alpha,\beta) &\mapsto \left(G(x,y,\alpha,\beta),
\pdydx{G}{x}(x,y,\alpha,\beta) , \pdydx{G}{y}(x,y,\alpha,\beta) \right)
\end{split}
\]
is a mapping of rank $2$ for all $(x,y)$.  Effectively, the matrix
\[
\begin{pmatrix}
\displaystyle \pdydx{G}{\alpha} & \displaystyle \pdydx{G}{\beta} \\
\displaystyle \pdydxnm{G}{x}{\alpha}{2}{}{} &
\displaystyle \pdydxnm{G}{x}{\beta}{2}{}{} \\
\displaystyle \pdydxnm{G}{y}{\alpha}{2}{}{} &
\displaystyle \pdydxnm{G}{y}{\beta}{2}{}{} \\
\end{pmatrix}
=
\begin{pmatrix}
n(- x\sin(\alpha) + y \cos(\alpha)) & 1 \\
-n\sin(\alpha) & 0 \\
n\cos(\alpha) & 0
\end{pmatrix}
\]
is of rank $2$ because at least one of the determinants
$\displaystyle \det
\begin{pmatrix} n(- x\sin(\alpha) + y \cos(\alpha)) & 1 \\ -n\sin(\alpha) & 0
\end{pmatrix}$
and
$\displaystyle \det
\begin{pmatrix} n(- x\sin(\alpha) + y \cos(\alpha)) & 1 \\ n\cos(\alpha) & 0
\end{pmatrix}$
is non null for all $(x,y)$.

As in Example~\ref{char_egg_eikonal}, we consider the Cauchy problem
with the curve \\
$\displaystyle \Gamma = \left\{ (x_0(s), y_0(s), u_0(s) : s \in I\right\}$,
where $x_0(s)= \cos(s)$, $y_0(s) = \sin(s)$ and $u_0(s)=0$ for $s \in I = \RR$.

The sub-family of surfaces obtained from the complete integral above
with $\alpha = \alpha(s) = s$ and $\beta= \beta(s) = -n$ for all $s$
is the one-parameter family of integral surfaces needed to find the
integral surface that contains $\Gamma$ using the envelope of this
family.  To verify this, we note that
\[
G(x_0(s),y_0(s),\alpha(s),\beta(s)) = n(\cos(s)\cos(s)
+ \sin(s)\sin(s)) + \beta = 0 = u_0(s)
\]
and
\begin{align*}
&\pdydx{G}{x}(x_0(s),y_0(s), \alpha(s), \beta(s))\, \dydx{x_0}{s}(s)
+ \pdydx{G}{y}(x_0(s),y_0(s), \alpha, \beta)\, \dydx{y_0}{s}(s) \\
&\qquad = -n\cos(s)\sin(s) + n \sin(s)\cos(s) = 0 = u_0'(s)
\end{align*}
for all $s$.

The envelope to this family of integral surfaces is given by the
equations
\begin{align}
u &= G(x,y,\alpha(s),\beta(s)) = n(x\cos(s) + y \sin(s)) - n \label{char_Eik1}
\intertext{and}
0 &= \pdfdx{G}{s}(x,y,\alpha(s), \beta(s)) = -nx\sin(s) +ny\cos(s)
\ . \label{char_Eik2}
\end{align}
(\ref{char_Eik2}) gives $\tan(s) = y/x$.  Thus
$\displaystyle \cos(s) = x/\sqrt{x^2+y^2}$ and
$\displaystyle \sin(s) = y/\sqrt{x^2+y^2}$.
If we substitute these values of $\cos(s)$ and $\sin(s)$ in
(\ref{char_Eik1}), we get
\[
u = \frac{nx^2}{\sqrt{x^2+y^2}} + \frac{ny^2}{\sqrt{x^2+y^2}} - n
= n\sqrt{x^2+y^2} - n \ .
\]
This is one of the integral surfaces that we found in
Example~\ref{char_egg_eikonal}.  The other integral surface is given
by $\alpha = \alpha(s) = -s$ and $\beta= \beta(s) = -n$ for all $s$.
\end{egg}

\section{Addendum}

The rest of this book is about linear partial differential equations.
However, it is worth having a short chapter on nonlinear partial
differential equations.  This is Chapter~\ref{ChapShock} at the end of
the book.  The theory on linear partial differential
equations is not easy.  The theory on nonlinear partial differential
equations is harder.

\section{Exercises}

Suggested exercises:

\begin{itemize}
\item In \cite{J}: numbers 1 and 2 in Section 1.6; all the numbers
in Section 1.9.
\item In \cite{McO}: all the numbers in Sections 1.1 and 1.3.
\item In \cite{PinRub}: all the numbers in Sections 1.7 and 2.10. 
\item In \cite{Str}: numbers 1 to 4 in Section 1.1; all the
numbers in Section 1.2; numbers 4 to 6 in Section 1.5. 
\end{itemize}


%%% Local Variables: 
%%% mode: latex
%%% TeX-master: "notes"
%%% End: 
