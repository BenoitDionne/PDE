\chapter{Linear Elliptic Partial Differential Equations}
\label{elliptic_PDEs}

To simplify the presentation of the results about linear elliptic partial
differential equations in this chapter, we introduce a new notation
for the derivatives in the sense of distributions. Let $\Omega$ be an
open subset of $\displaystyle \RR^n$ and $u$ be a distribution on $\Omega$.  If
$\VEC{\alpha} = \VEC{e}_j$, then we denote the derivative
$\diff^{\VEC{\alpha}} u$ in the sense of distribution by $\diff_{x_j} u$.  If
$\VEC{\alpha} = \VEC{e}_{j_1} + 3 \VEC{e}_{j_2}$,
then we denote the derivative $\displaystyle \diff^{\VEC{\alpha}} u$ in the
sense of distribution by $\diff_{x_{j_1}x_{j_2}x_{j_2}x_{j_2}} u$.  And so on.
Obviously, if $u$ is sufficiently differentiable with locally
integrable derivatives, then $\displaystyle \diff_{x_j} u = \pdydx{u}{x_j}$
and $\displaystyle \diff_{x_{j_1}x_{j_2}x_{j_2}x_{j_2}} u
= \pdydxnm{u}{x_{j_1}}{x_{j_2}}{4}{}{3}$, and so on.

The main source of information for this chapter was
\cite{Br,FoPDE,RenRog,Tr}; in particular the first two
citations.

\section{Introduction to Variational Methods}

There are four steps in a {\bfseries variational method}.  The steps
can be roughly described as follows.
\begin{enumerate}
\item Choose a ``variational problem'' such that a
classical solutions (if any) is also a solution of the
``variational problem.''\  This includes choosing the right Sobolev
space to use with the variational problem.  Solutions of the
``variational problem'' are called ``weak solution.''
\item Prove the existence and uniqueness of a weak solution $u$ of the
variational equation; for instance, with the help of the Lax-Milgram Theorem.
\item \label{ell_wsd1} Determine the Sobolev space of the highest
order that contains the weak solution $u$.  ``Regularity results''
are generally used at this stage.
\item Determine the highest order $k$ (if any) of the space of
$\displaystyle C^k$ functions that contains the weak solution $u$.
``Imbedding results'' for Sobolev spaces are generally used at this stage.
If $k$ is equal or larger than the order of the original partial
differential equation, then the weak solution is a classical solution
of the original partial differential equation.
\end{enumerate}

We now illustrate the variational approach to solve a linear elliptic
partial differential equations with a couple of simple examples.
These examples will illustrate the approach that will be used to
develop a general theory to solve linear elliptic partial differential
equations.  Obviously, more technical results will be needed to
solve the general problems as we will see in the next sections.
Nevertheless, the basic principle stays the same. 

\begin{egg}
Consider the boundary value problem
\begin{equation} \label{ell_example1_eq1}
\begin{split}
- u'' + u &= f  \quad , \quad a < x < b \\
 u(a)&=u(b) = 0
\end{split}
\end{equation}
where $\displaystyle f \in L^2(]a,b[)$.  We assume that all functions are real
valued functions.                   \label{ell_example1}

If $f\in C([a,b])$, then the classical solution of
(\ref{ell_example1_eq1}) is a function $\displaystyle u\in C^2([a,b])$ that
satisfies (\ref{ell_example1_eq1}).  Recall that $\displaystyle C^k([a,b])$,
$k\in \NN$, is the set of all
$\displaystyle u \in C([a,b]) \cap C^k(]a,b[)$ such
that $\displaystyle \dydxn{u}{x}{j}$ can be continuously extended
to $[a,b]$ for $0 \leq j \leq k$.

If $f\in C([a,b])$, then (\ref{ell_example1_eq1}) is a simple second
order ordinary differential equation with constant coefficients that
can easily be solved using basic techniques to solve such ordinary
differential equations.  We ignore this fact.

\subI{Variational Problem}
If $\displaystyle u \in C^2([a,b])$ is a classical solution of
(\ref{ell_example1_eq1}), then
\[
\int_a^b \left( -u''(x) + u(x) - f(x) \right)v(x)\dx{x} = 0
\]
for all $\displaystyle v \in C^1_c(]a,b[)$.  A simple integration by parts gives
\[
\int_a^b \left( u'(x)\,v'(x) + u(x)\, v(x) -f(x)\, v(x)
\right) \dx{x} = 0
\]
for all $\displaystyle v \in C^1_c(]a,b[)$.
This expression makes sense if we assume that
$u$ and $v$ are in the Sobolev space $\displaystyle H_0^{1}(]a,b[)$, and the
derivatives are in the sense of distributions.  We now return to the
assumption that $\displaystyle f \in L^2([a.b])$.  Recall that
$\displaystyle H^1_0(]a,b[)$ is the completion of $\DD(]a,b[)$ in
$\displaystyle L^2(]a.b[)$ with respect to the norm
\begin{equation} \label{ell_example1_eq3}
\| f \|_{1,2} = \left( \sum_{|\VEC{\alpha}|\leq 1} \| \diff^{\VEC{\alpha}} f \|_2^2
\right)^{1/2} \ ,
\end{equation}
where $\|\cdot\|_2$ is the usual norm on $\displaystyle L^2(]a,b[)$.
$\displaystyle H^1_0(]a,b[)$ is a Banach space with respect to the norm
(\ref{ell_example1_eq3}).  As we have seen in the previous chapter,
$\displaystyle u \in H^1_0(]a,b[)$ generalizes the notion of
$\displaystyle u\in C^1([a,b])$ with $u(a)=u(b)=0$.

The {\bfseries variational formulation} of
(\ref{ell_example1_eq1}) is to find $\displaystyle u \in H^1_0(]a,b[)$ such that
\begin{equation} \label{ell_example1_eq4}
\int_a^b \left( \diff_x u\,\diff_x v + u\, v -f\, v \right) \dx{x} = 0
\end{equation}
for all $\displaystyle v \in H^{1}_0(]a,b[)$.
This is also called a {\bfseries variational problem}.
(\ref{ell_example1_eq4}) is a {\bfseries variational equation}.  The
solution $\displaystyle u \in H^1_0(]a,b[)$ of
(\ref{ell_example1_eq4}) is called a {\bfseries weak solution}.

We show that a classical solution is a weak solution.  If
$\displaystyle u \in C^2([a,b])$ is a solution of (\ref{ell_example1_eq1}), then
\[
\int_a^b \left( - u''(x) + u(x) - f(x) \right)v(x)\dx{x} = 0
\]
for all $\displaystyle v \in C^1_c(]a,b[)$.
A simple integration by parts yields
\[
\int_a^b \left( u'(x)\, v'(x)+ u(x) v(x) -
f(x)v(x)\right) \dx{x} = 0
\]
for all $\displaystyle v \in C^1_c(]a,b[)$.
Since $\displaystyle C^1_c(]a,b[)\supset \DD(]a,b[)$ is dense in
$\displaystyle H^1_0(]a,b[)$, we get (\ref{ell_example1_eq4}).  To be
more precise, if
$\displaystyle \left\{v_i\right\}_{i=1}^\infty \subset C^1_c([a,b])$
converges to $\displaystyle v \in H^1_0(]a,b[)$ with respect to the norm
$\|\cdot\|_{1,2,]a,b[}$, then
$\displaystyle \|v_i - v\|_{2,]a,b[} \to 0$ and
$\displaystyle \|\diff_x v_i - \diff_x v\|_{2,]a,b[} \to 0$ as $i \to 0$.
Hence, from Schwarz inequality, we get that
\begin{align*}
&\left|\int_a^b \diff_x u \, \diff_x v_i \dx{x} - 
\int_a^b \diff_x u\, \diff_x v \dx{x} \right|
\leq \int_a^b \left|\diff_x u\right| \,
\left| \diff_x v_i -\diff_x v\right| \dx{x} \\
&\qquad \quad \leq \left(\int_a^b \left|\diff_x u\right|^2 \dx{x} \right)^{1/2}
\left( \int_a^b \left| \diff_x v_i -\diff_x v\right|^2 \dx{x} \right)^{1/2}
\to 0 \quad \text{as} \quad i \to \infty \ ,
\end{align*}
because $\displaystyle u' \in L^2(]a,b[)$ since $u\in C^2([a,b])$.  Similarly,
\[
\int_a^b u v_i \dx{x} \to \int_a^b u v \dx{x} \quad \text{and} \quad
\int_a^b f v_i \dx{x} \to \int_a^b f v \dx{x} \quad \text{as} \quad
i \to \infty \ .
\]
Thus,
\[
\int_a^b \left( \diff_x u\, \diff_x v_i + u\, v_i  - f\, v_i\right) \dx{x} \to
\int_a^b \left( \diff_x u \, \diff_x v + u\, v - f\, v \right) \dx{x}
\quad \text{as} \quad i \to \infty \ .
\]
It also follows from Proposition~\ref{sob_w0_trad2} that
$\displaystyle u \in H^1_0(]a,b[)$ because \\
$\displaystyle u \in C^2([a,b]) \subset H^1(]a,b[) \cap C([a,b])$ and
$u(a)=u(b)=0$.

\subI{Existence and uniqueness of the weak solution}
We show that Lax-Milgram Theorem, Corollary~\ref{fu_an_LaxMilgTh}, can
be used to conclude that the variational problem
(\ref{ell_example1_eq4}) has a unique weak solution
$\displaystyle u \in H^1_0(]a,b[)$ given by
\begin{equation} \label{ell_example1_eq5}
\begin{split}
&\frac{1}{2} \int_a^b \left( \left(\diff_x u\right)^2 +
\left(u\right)^2\right) \dx{x} - \int_a^b f\, u \dx{x}\\
&\qquad \quad = \min_{v \in H^1_0(]a,b[)} \left\{
\frac{1}{2} \int_a^b \left( \left(\diff_x v\right)^2 +
\left(v\right)^2\right) \dx{x} - \int_a^b f\, v \dx{x} \right\} \ .
\end{split}
\end{equation}
This property is known as the {\bfseries Dirichlet principle}.

Consider the bilinear form
\[
B(v,u) = \int_a^b \left( \diff_x u\, \diff_x v + u \, v \right) \dx{x}
= \int_a^b \left( \diff_x u, u\right)\cdot\left(\diff_x v,v \right) \dx{x}
\]
for all $\displaystyle u,v \in H^1_0(]a,b[)$.
It is clear that $B$ is bilinear.  To prove that $B$ is
bounded, we use Schwarz inequality in $\displaystyle \RR^2$ and
$\displaystyle L^2(]a.b[)$ to get
\begin{align*}
|B(v,u)| &\leq \int_a^b \left| \left( \diff_x u, u\right) \cdot
\left(\diff_x v, v \right) \right| \dx{x}
\leq \int_a^b \left(\left( \diff_x u\right)^2
+\left(u\right)^2\right)^{1/2} \, \left(\left( \diff_x v\right)^2
+\left(v\right)^2\right)^{1/2} \dx{x} \\
&\leq \left( \int_a^b \left(\left( \diff_x u\right)^2
+\left(u\right)^2\right) \dx{x} \right)^{1/2} \,
\left( \int_a^b \left(\left(\diff_x v\right)^2
+\left(v\right)^2\right) \dx{x} \right)^{1/2}
= \| u\|_{1,2} \, \|v\|_{1,2}
\end{align*}
for all $\displaystyle u,v \in H^1_0(]a,b[)$.
Since $\displaystyle B(u,u) = \|u\|_{1,2}^2$ for
$\displaystyle u\in H^1_0(]a,b[)$, it is obvious
that $B$ is coercive.  It is also obvious that $B$ is
symmetric; namely, $B(u,v)=B(v,u)$ for all $\displaystyle u,v \in H^1_0(]a,b[)$.
Moreover, the function $\displaystyle f \in L^2(]a,b[)$ yields a
bounded functional on $\displaystyle H^1_0(]a,b[)$ defined by
\[
\tilde{f}(u) = \ps{u}{f}_2 = \int_a^b f(x)u(x)\dx{x}
\]
for all $\displaystyle u \in H^1_0(]a,b[)$.
We let the reader prove that $f$ effectively defines a bounded
functional.  We therefore can apply Lax-Milgram Theorem to the
bilinear form $B$ to get (\ref{ell_example1_eq5}).

\subI{Regularity of the solution}
The regularity of the solution is deduced from
Theorem~\ref{ell_regular} that will present later.  We will see
that $\displaystyle u\in H^2(]a.b[)$.

\subI{Classical Solution}
We prove that if $f \in C([a,b])$, then the weak solution $u$
belongs to $\displaystyle C^2([a,b])$.  According to
Corollary~\ref{sob_cor_sob_lem},
we get that $\displaystyle u \in C^1([a,b])$ because
$\displaystyle u\in H^2(]a,b[)$.  We also
have from $B(v,u)=\ps{v}{f}_2$ with
$\displaystyle v\in \DD(]a,b[) \subset H^1_0(]a,b[)$
that $\diff_x(\diff_x u) = u - f \in C([a,b])$, where
the derivatives are in the sense of distributions.  Hence, from
Proposition~\ref{distr_cdistr_cder} and Theorem~\ref{sob_ader}
(or Corollary~\ref{sob_TheSobLemma_cor1}), we get that
$\displaystyle \diff_x u \in C^1([a,b])$.
Thus $\displaystyle u\in C^2([a,b])$.

We now prove that a weak solution
$\displaystyle u \in C^2([a,b]) \cap H^1_0(]a,b[)$
is a classical solution.  If
$\displaystyle u \in C^2([a,b]) \cap H^1_0(]a,b[)$ be
a solution of (\ref{ell_example1_eq4}), then
\[
\int_a^b \left( u'(x)\,v'(x) + u(x)\, v(x) -f(x)\, v(x) \right) \dx{x} = 0
\]
for all $\displaystyle v \in C^1_c(]a,b[)$.
A simple integration by parts shows that
\[
\int_a^b \left( -u''(x) + u(x) -f(x)\right) v(x) \dx{x} = 0
\]
for all $\displaystyle v \in C^1_c(]a,b[)$.
Since $\displaystyle C^1_c(]a,b[) \supset \DD(]a,b[)$ is dense in
$\displaystyle L^2(]a,b[)$, we obtain that
$-u'' + u -f = 0$ almost everywhere in $]a,b[$.  Since
$\displaystyle u\in C^2([a,b])$, we have that 
$- u'' + u \in C([a,b])$.  So, we may assume that 
$f \in C([a,b])$ and $-u'' + u -f = 0$
everywhere on $]a,b[$.  Finally, we get from
Proposition~\ref{sob_w0_trad2} that $u(a)=u(b)= 0$ because
$\displaystyle u \in H^{1,2}_0(]a,b[) \cap C([a,b])$.  Thus $u$ is the classical
solution of (\ref{ell_example1_eq1}).
\end{egg}

\begin{egg}
Consider the Sturm-Liouville problem
\begin{equation} \label{ell_exampleSL_eq1}
\begin{split}
-(p\, u')' + q\,u &= f \quad , \quad 0<x < 1 \\
u(0)&= u(1)=0
\end{split}
\end{equation}
where $\displaystyle f\in L^2(]0,1[)$, $q\in C([0,1])$ and
$\displaystyle p\in C^1([0.1])$.
Moreover, we assume that $q(x)\geq 0$ for all $x\in [0,1]$ and that
there exists $\alpha>0$ such that $p(x) \geq \alpha$ for all
$x\in[0,1]$.   \label{ell_exampleSL}

\subI{Variational Problem}
The weak solution is the solution $\displaystyle u\in H^{1,2}_0(]0,1[)$ of the
equation
\begin{equation} \label{ell_exampleSL_eq2}
\int_0^1 \left( p\, \diff_x u \, \diff_x v + q\, u\, v
- f\, v \right) \dx{x} = 0
\end{equation}
for all $\displaystyle v \in H^{1,2}_0(]0,1[)$.

We prove that a classical solution is a weak solution.  
If $\displaystyle u\in C^2([0,1])$ is a classical solution of
(\ref{ell_exampleSL_eq1}), then it follows from integration by parts that
\[
\int_0^1 \left( p\,u'\, v' + q\, u\, v - f\, v \right) \dx{x}
\int_0^1 \left( -(p\,u')'\, v + q\, u\, v - f\, v \right) \dx{x} = 0
\]
for all $\displaystyle v \in C^1_c(]0,1[)$.   Since
$\displaystyle C^1_c(]0,1[) \supset \DD(]0,1[)$ is dense in
$\displaystyle H^{1,2}_0(]0,1[)$, we have that
(\ref{ell_exampleSL_eq2}) is true.  Moreover, it follows from 
Proposition~\ref{sob_w0_trad2} that
$\displaystyle u\in H^{1,2}_0(]0,1[)$ because
$u\in C([0,1])$ and $u(0)=u(1)=0$.

\subI{Existence and uniqueness of the weak solution}
\begin{align*}
B:H^{1,2}_0(]0,1[) \times H^{1,2}_0(]0,1[) &\rightarrow \RR \\
(u,v) &\mapsto \int_0^1 \left( p(x)\,\diff_x u(x)\,\diff_x v(x) + q(x)\,u(x)
v(x)\right)\dx{x}
\end{align*}
is a coercive, symmetric bounded bilinear mapping.  The prove that $B$
is a symmetric bilinear form is easy and is left to the reader.  To
prove that $B$ is bounded, let
\[
M = \max\left\{ \max_{x\in[0,1]} |p(x)| , \max_{x\in[0,1]} |q(x)| \right\} \ .
\]
Then, using Schwarz inequality in $\RR^2$ and $\displaystyle L^2([0,1])$, we get
\begin{align*}
|B(u,v)| &\leq \int_0^1 \left| p\, \diff_x u\,\diff_x v + q\,u\,v \right| \dx{x}
\leq M \int_0^1 \left( \left| \diff_x u\, \diff_x v \right|\
+ \left|u\, v\right|\right) \dx{x} \\
&= M \int_0^1 \left( \left|\diff_x u\right|,|u| \right)
\cdot \left( \left|\diff_x v\right|, |v| \right) \dx{x}
\leq M \int_0^1 \left\| \left(\diff_x u, u\right)\right\|_2
\left\| \left(\diff_x v, v\right)\right\|_2 \dx{x} \\
&\leq M \left( \int_0^1 \left\|\left(\diff_x u, u\right)\right\|_2^2
\dx{x} \right)^{1/2} \, \left( \int_0^1 \left\|
\left(\diff_x v, v\right)\right\|_2^2 \dx{x}\right)^{1/2} \\
&= M\left( \int_0^1 \left(\left|\diff_x u\right|^2+ \left|u\right|^2
\right) \dx{x} \right)^{1/2} \, \left( \int_0^1 \left(
\left|\diff_x v\right|^2 + \left|v\right|^2 \right) \dx{x} \right)^{1/2}
= M\| u\|_{1,2} \, \|v\|_{1,2}
\end{align*}
for all $\displaystyle u,v \in H^{1,2}_0(]0,1[)$.
To prove that $B$ is coercive, we note that
\[
B(u,u) = \int_0^1 \left( p\,|\diff_x u|^2 + q\,|u|^2 \right)\dx{x}
\geq \alpha \int_0^1 |\diff_x u|^2 \dx{x} \geq \frac{\alpha}{C}
\|u\|_{1,2}^2
\]
for all $\displaystyle u \in H^{1,2}_0(]0,1[)$,
where $C$ is the constant given by the Poincar\'e's inequality,
Theorem~\ref{sob_pt_carre}.  Moreover, the function
$\displaystyle f \in L^2(]0,1[)$ yields a bounded functional on
$\displaystyle H^{1,2}_0(]0,1[)$ defined by
\[
\tilde{f}(u) = \ps{u}{f}_2 = \int_\Omega f\, u \dx{x}
\]
for all $\displaystyle u \in H^{1,2}_0(]0,1[)$.
Thus, we may apply Lax-Milgram Theorem, Corollary~\ref{fu_an_LaxMilgTh},
to (\ref{ell_exampleSL_eq2}) to
conclude that there exists a unique weak solution
$\displaystyle u \in H^{1,2}_0(]0,1[)$ given by
\begin{align*}
&\frac{1}{2} \int_0^1 \left( p\,\diff_x u\, \diff_x u + q\,u\,u \right) \dx{x}
-\int_0^1 f\, u \dx{x} \\
& \qquad = \min_{v\in H^{1,2}_0(]0,1[)} \left\{ \frac{1}{2}
\int_0^1 \left( p\,\diff_x v\,\diff_x v + q\,v\,v \right) \dx{x}
-\int_0^1 f \, v \dx{x} \right\} \ .
\end{align*}

\subI{Regularity of the solution}
As a consequence of Theorem~\ref{ell_regular} that we will present later,
we have that $\displaystyle u\in H^{2,2}(]0,1[)$.

\subI{Classical solution}
We prove that if $f \in C([0,1])$, then
the weak solution $u$ belongs to $\displaystyle C^2([0,1])$ and it is
a classical solution of (\ref{ell_exampleSL_eq1}).

According to Corollary~\ref{sob_cor_sob_lem},
$\displaystyle u\in H^{2,2}(]0,1[)$
implies that $\displaystyle u \in C^1([0,1])$.  Since\\
$\displaystyle H^{1,2}_0(]0,1[) \supset \DD(]0,1[)$,
we also have from $B(v,u)=0$ for all $v \in \DD(]0,1[)$ that
$\diff_x(p\, \diff_x u) = q\,u - f \in C([0,1])$.  Hence, from
Proposition~\ref{distr_cdistr_cder} and Theorem~\ref{sob_ader},
$u'\in C^1([0,1]$.  It follows that $\displaystyle u \in C^2([0,1])$.

To prove that our weak solution
$\displaystyle u \in C^2([0,1]) \cap H^1_0(]0,1[)$
satisfies (\ref{ell_exampleSL_eq1}) in the classical sense, we use
integration by parts on (\ref{ell_exampleSL_eq2}) to get
\[
\int_0^1 \left( -(p\,u')' + q\,u - f\right) v \dx{x} =
\int_0^1 \left( p\,u'\, v' + q\,u\,v - f\, v\right) \dx{x} = 0
\]
for all $\displaystyle v \in C^1_c([0,1[)$.
Since $\displaystyle C^1_c(]0,1[)\supset \DD(]0,1[)$ is dense in
$\displaystyle L^2(]0,1[)$, we
get $\displaystyle -(p\,u')' + q\,u - f = 0$ almost
everywhere.  Since $f$, $p'$ and $q$ are all continuous functions,
$\displaystyle -(pu')' + q u -f = 0$ everywhere on $]0,1[$ because
$\displaystyle u\in C^2(]0,1[)$.  It also follows from
Proposition~\ref{sob_w0_trad2} that $u(0)=u(1)=0$ because
$\displaystyle u \in H^1_0(]0,1[)\cap C^2([0,1])$.  Thus $u$ is the classical
solution of (\ref{ell_exampleSL_eq1}). 
\end{egg}

\section[Classical, Strong and Weak Solutions]{Classical, Strong and
Weak Solutions of Elliptic Problems} \label{CSWSsection}

We consider strongly elliptic linear operators of the form
\begin{equation} \label{ell_op1} 
L(\VEC{x}, \diff) = \sum_{|\VEC{\gamma}|\leq m} a_{\VEC{\gamma}}(\VEC{x})
\diff^{\VEC{\gamma}} \ ,
\end{equation}
where the coefficients $a_{\VEC{\gamma}}$ are all of class
$\displaystyle C^\infty(\RR^n)$.
Recall that an elliptic linear operator paused
on a bounded open set $\Omega$ is strongly elliptic on $\Omega$ if
$a_{\VEC{\gamma}}:\overline{\Omega} \to \RR$ is continuous for all
$\VEC{\gamma}$.
According to Proposition~\ref{classif_ell_even},, it is also of even
order; namely, $m$ is even.  If $n>2$ and some of the coefficients are
complex valued functions, then the order is also even.
We assume that any one of these necessary conditions is satisfied.

It is useful to write $L(\VEC{x},\diff)$ in the {\bfseries divergence form}%
\index{Linear Partial Differential Operator!Divergence Form}
\begin{equation} \label{ell_op2}
L(\VEC{x},\diff)u = \sum_{|\VEC{\alpha}|,|\VEC{\beta}|\leq m/2}
(-1)^{|\VEC{\alpha}|} \diff^{\VEC{\alpha}}
\left( b_{\VEC{\alpha},\VEC{\beta}}(\VEC{x}) \diff^{\VEC{\beta}} u\right) \ ,
\end{equation}
where $\displaystyle b_{\VEC{\alpha},\VEC{\beta}} \in C^\infty(\Omega)$ for all
multi-indices $\VEC{\alpha}$ and $\VEC{\beta}$.

If the linear differential operator $L(\VEC{x},\diff)$ is given in the
divergence form (\ref{ell_op2}), then the principal symbol takes the
form
\[
Q_{df}(\VEC{x}, \VEC{\xi}) = \sum_{|\VEC{\alpha}|=|\VEC{\beta}|=m/2}
\VEC{\xi}^{\VEC{\alpha}}
b_{\VEC{\alpha},\VEC{\beta}}(\VEC{x}) \VEC{\xi}^{\VEC{\beta}} \quad ,
\quad \VEC{x} \in \RR^n \ \text{and} \ \VEC{\xi} \in \RR^n \ .
\]
If we compare (\ref{ell_op1}) and (\ref{ell_op2}), we find that
$\displaystyle a_{\VEC{\gamma}} = (-1)^{m/2}
\sum_{\VEC{\alpha}+\VEC{\beta}=\VEC{\gamma}} b_{\VEC{\alpha},\VEC{\beta}}$
for $|\VEC{\gamma}|= m$.
We may therefore modify Definition~\ref{strong_ell_form} as
follows.  The linear differential operator $L(\VEC{x},\diff)$ given in
(\ref{ell_op2}) is {\bfseries strongly elliptic}\index{Linear Partial
Differential Operator!Strongly Elliptic}
on an open set $\displaystyle \Omega \subset \RR^n$ if there exists a
constant $C>0$ such that
$\displaystyle \RE Q_{df}(\VEC{x}, \VEC{\xi}) \geq C \|\VEC{\xi}\|^m$
for all $\VEC{x} \in \Omega$ and $\displaystyle \VEC{\xi} \in \RR^n$.
Note that there is no factor $\displaystyle (-1)^{m/2}$ on the left
side because we have a factor of $\displaystyle (-1)^{m/2}$ in the
definition of $Q(\VEC{x},\VEC{\xi})$.  So, the factor on the left hand side is
$\displaystyle (-1)^m =1$ since $m$ is even.
In the future, we will not use different notations for the principal
symbol associated to a partial differential equation in standard form
or in divergence form.  The context will determine which form of the
principal symbol is used.

For convenience, we also adopt in the following definition for an
elliptic partial differential operator in divergence form.  We will
say that (\ref{ell_op2}) is {\bfseries elliptic}\index{Linear Partial
Differential Operator!Elliptic} on an
open set $\Omega$ if $Q_{df}(\VEC{x}, \VEC{\xi}) > 0$ for all
$\VEC{x} \in \Omega$ and $\displaystyle \VEC{\xi} \in \RR^n$ with
$\VEC{\xi} \neq \VEC{0}$.

Except otherwise stated, we always assume that the linear differential
operators in this section are in divergence form (\ref{ell_op2}).

\subsection{Dirichlet Problem}

In this section and the next one, we assume that $\Omega$ is an open
subset of $\displaystyle \RR^n$ and that $\partial \Omega$ is a
$\displaystyle C^\infty$-manifold, or at least that $\Omega$ satisfies
a strong $q$-extension property with $q\geq m/2$.  Let $k = m/2$.
We consider the Dirichlet problem
\begin{equation} \label{ell_PDE1}
\begin{split}
L(\VEC{x}, \diff)u &= \sum_{|\VEC{\alpha}|,|\VEC{\beta}|\leq k}
(-1)^{|\alpha|} \diff^{\VEC{\alpha}}
\left( b_{\VEC{\alpha},\VEC{\beta}}(\VEC{x}) \diff^{\VEC{\beta}} u\right)
= f \quad , \quad
\VEC{x} \in \Omega \ ,\\
\pdydxn{u}{\nu}{j}(\VEC{x}) &= 0 \quad , \quad
\VEC{x} \in \partial \Omega \ \text{and}\  0 \leq j < k \ ,
\end{split}
\end{equation}
where $L(\VEC{x}, \diff)$ is at least elliptic and $\nu(\VEC{x})$ is
the outward unit normal vector to $\partial \Omega$ at
$\VEC{x} \in \partial \Omega$.

There are many interpretations of solving this problem.  We have already
considered two interpretations of solving problems like (\ref{ell_PDE1})
which we summarize in the following two definitions.

\begin{defn}
If $f \in C_b(\Omega)$, a {\bfseries classical solution}
\index{Dirichlet Problem!Classical Solution}
for (\ref{ell_PDE1}) is a function
$\displaystyle u \in C^{2k}_b(\Omega) \cap C^{k-1}_b(\overline{\Omega})$
such that
\[
L(\VEC{x},\diff)u(\VEC{x}) = \sum_{|\VEC{\alpha}|,|\VEC{\beta}|\leq k}
(-1)^{|\VEC{\alpha}|} \diff^{\VEC{\alpha}}
\left( b_{\VEC{\alpha},\VEC{\beta}}(\VEC{x}) \diff^{\VEC{\beta}}
u(\VEC{x}) \right) = f(\VEC{x}) \quad , \quad \VEC{x} \in \Omega \ ,
\]
and
\[
\pdydxn{u}{\nu}{j}(\VEC{x}) = 0 \quad , \quad
\VEC{x} \in \partial \Omega \ \text{and}\ 0 \leq j < k \ .
\]
All partial and directional derivatives are in the classical sense.
\end{defn}

\begin{defn}
If $\displaystyle f\in L_{loc}^2(\Omega)$, a {\bfseries strong solution}
\index{Dirichlet Problem!Strong Solution}
for (\ref{ell_PDE1}) is a distribution
$\displaystyle u \in H^{2k,2}(\Omega) \cap H^{k,2}_0(\Omega) \subset
\DD'(\Omega)$ such that
\[
\ps{ L(\VEC{x},\diff)u}{\phi} =
\sum_{|\VEC{\alpha}|,|\VEC{\beta}|\leq k} \ps{ (-1)^{|\VEC{\alpha}|}
\diff^{\VEC{\alpha}} \left( b_{\VEC{\alpha},\VEC{\beta}}(\VEC{x})
\diff^{\VEC{\beta}} u \right) }{\phi} = \ps{f}{\phi}
\]
for all $\displaystyle \phi \in \DD(\Omega)$,
where $\ps{v}{\phi} \equiv v(\phi)$ for $v\in \DD'(\Omega)$ and
$\phi \in \DD(\Omega)$.  Obviously, all derivatives are in the sense
of distributions.
\end{defn}

There is a third approach to study (\ref{ell_PDE1}).  It is the
variational approach that we will consider in the rest of this
section.  We consider the {\bfseries variational formulation} of
(\ref{ell_PDE1}) which consists in finding
$\displaystyle u\in H^{k,2}_0(\Omega)$ such that
\begin{equation} \label{ell_PDE2}
B(v,u) = \ps{v}{f}_2
\end{equation}
for all $\displaystyle \displaystyle v \in H^{k,2}_0(\Omega)$, where
\begin{equation} \label{ell_PDE3}
\begin{split}
B : H^{k,2}(\Omega) \times H^{k,2}(\Omega) & \rightarrow \CC \\
(v,u) &\mapsto
\sum_{|\VEC{\alpha}|,|\VEC{\beta}|\leq k}
\ps{\diff^{\VEC{\alpha}} v}{b_{\VEC{\alpha},\VEC{\beta}}(\VEC{x})
\diff^{\VEC{\beta}} u}_2
\end{split}
\end{equation}
and $\displaystyle \ps{v}{u}_2
= \int_\Omega v(\VEC{x})\overline{u(\VEC{x})} \dx{\VEC{x}}$
for $\displaystyle u,v \in L^2(\Omega)$ is the scalar product in
$\displaystyle L^2(\Omega)$.  Moreover, we assume that the
coefficients $b_{\VEC{\alpha},\VEC{\beta}}$ are bounded on $\Omega$.

The mapping $B$ is well defined.  Suppose that
$\left| b_{\VEC{\alpha},\VEC{\beta}}(\VEC{x})\right| \leq M$ for all 
$\VEC{x}\in \Omega$, where $M$ is a constant.  Then,
\begin{align*}
\left| B(v,u) \right| &=
\left| \sum_{|\VEC{\alpha}|,|\VEC{\beta}|\leq k}
\int_\Omega \overline{b_{\VEC{\alpha},\VEC{\beta}}(\VEC{x})
\diff^{\VEC{\beta}} u(\VEC{x})}\,
\diff^{\VEC{\alpha}} v(\VEC{x}) \dx{\VEC{x}} \right|
\leq \sum_{|\VEC{\alpha}|,|\VEC{\beta}|\leq k} M
\int_\Omega \left| \overline{\diff^{\VEC{\beta}} u(\VEC{x})}\,
\diff^{\VEC{\alpha}} v(\VEC{x})\right| \dx{\VEC{x}} \\
&\leq \sum_{|\VEC{\alpha}|,|\VEC{\beta}|\leq k} M
\left( \int_\Omega \left|\diff^{\VEC{\beta}} u(\VEC{x})\right|^2
\dx{\VEC{x}}\right)^{1/2}
\left( \int_\Omega \left|\diff^{\VEC{\alpha}} v(\VEC{x})\right|^2
\dx{\VEC{x}}\right)^{1/2} \\
&\leq \left(\sum_{|\VEC{\alpha}|,|\VEC{\beta}|\leq k} M \right)
\| u\|_{k,2}\, \| v\|_{k,2}
= C \| u\|_{k,2} \, \| v\|_{k,2} < \infty
\end{align*}
for all $\displaystyle (v,u) \in H^{k,2}(\Omega) \times H^{k,2}(\Omega)$,
where we have used Schwarz inequality for the second inequality and
set $\displaystyle C = \sum_{|\VEC{\alpha}|,|\VEC{\beta}|\leq k} M $.

Since
\begin{enumerate}
\item the mapping $u \mapsto B(v,u)$ is conjugate-linear for each
$u\in H^{k,2}(\Omega)$,
\item the mapping $v \mapsto B(v,u)$ is linear for each
$v\in H^{k,2}(\Omega)$, and
\item there exists a constant $C$ such that
$\displaystyle \left| B(v,u) \right| \leq C \| u\|_{k,2} \, \| v\|_{k,2}$
for $(v,u) \in H^{k,2}(\Omega) \times H^{k,2}(\Omega)$,
\end{enumerate}
we get the following result.

\begin{prop} \label{ell_Bblf}
The mapping $B$ defined in (\ref{ell_PDE3}) is a bounded sequilinear form.
\end{prop}

The sequilinear form $B$ defined in (\ref{ell_PDE3}) is
called a {\bfseries Dirichlet form}\index{Dirichlet Form} of order $k$.

\begin{rmk}
A partial differential operator $L(\VEC{x},\diff)$ may have more than one
Dirichlet form associated to it.  Consider the Laplacian
$\displaystyle \Delta$ in $\displaystyle \RR^2$.  Let $\eta=x-y$ and
$\xi = x+y$.
Then $B(v,u) = \ps{\diff_x v}{\diff_x u} + \ps{\diff_y v}{\diff_y u}$ and
$B(v,u) = 4\ps{\diff_\xi v}{\diff_\eta u}$ are two Dirichlet forms associated
to the Laplacian.  Recall that $\diff_x$, $\diff_y$, \ldots denote derivatives
in the sense of distributions.
\end{rmk}

Suppose that $\displaystyle b_{\VEC{\alpha},\VEC{\beta}}\in
C^{|\VEC{\alpha}|}_b(\Omega)$
for all multi-indices $\VEC{\alpha}$ and $\VEC{\beta}$ involved in the sum in
(\ref{ell_PDE1}).  Moreover, suppose that $f\in C_b(\Omega)$, and
$\displaystyle u \in C^{2k}_b(\Omega)\cap C^{k-1}_b(\overline{\Omega})$
satisfy (\ref{ell_PDE2}) for all $\displaystyle v \in C^{k}_c(\Omega)$.  Using
integration by parts, (\ref{ell_PDE2}) yields
\[
\int_\Omega \left( \overline{\sum_{|\VEC{\alpha}|,|\VEC{\beta}|\leq k}
(-1)^{\VEC{\alpha}} \diff^{\VEC{\alpha}}
\left( b_{\VEC{\alpha},\VEC{\beta}}(\VEC{x}) \diff^{\VEC{\beta}}
u(\VEC{x})\right)}\right)
v(\VEC{x}) \dx{\VEC{x}} = \int_\Omega \overline{f(\VEC{x})} v(\VEC{x})
\dx{\VEC{x}}
\]
for all $\displaystyle v \in C^{k}_c(\Omega)$.
Since $\displaystyle C^{k}_c(\Omega) \supset \DD(\Omega)$ is
dense in $\displaystyle L^2(\Omega)$, and $L(\VEC{x},\diff)u$ and $f$
are continuous on $\Omega$, we must have that
\[
L(\VEC{x},\diff)u(\VEC{x}) =
\sum_{|\VEC{\alpha}|,|\VEC{\beta}|\leq k} (-1)^{|\VEC{\alpha}|} \diff^{\VEC{\alpha}}
\left( b_{\VEC{\alpha},\VEC{\beta}}(\VEC{x}) \diff^{\VEC{\beta}}
u(\VEC{x})\right) = f(\VEC{x})
\quad , \quad \VEC{x} \in \Omega \ .
\]
Moreover, since $\displaystyle u \in H^{k,2}_0(\Omega)$ because we assume that
$u$ is a solution of (\ref{ell_PDE2}), it follows from
Proposition~\ref{sob_w0_trad2} that the boundary
conditions in (\ref{ell_PDE1}) are satisfied in the classical sense.
So $u$ is the classical solution of (\ref{ell_PDE1}).

\begin{defn} \label{ell_wf_Dir_probl}
If $\displaystyle b_{\VEC{\alpha},\VEC{\beta}} \in C^\infty_b(\Omega)$ and
$\displaystyle f \in L^2(\Omega)$,
a {\bfseries weak solution}\index{Dirichlet Problem!Weak Solution}
for (\ref{ell_PDE2}) is a function
$\displaystyle u \in H^{k,2}_0(\Omega)$ such that
$\displaystyle B(v,u) = \ps{v}{f}_{2}$ for all
$\displaystyle v \in H^{k,2}_0(\Omega)$.
\end{defn}

The variational approach is more general than the classical
approach because we look for solutions in a larger space
of functions; namely, $\displaystyle H^{k,2}_0(\Omega)$ instead of
$\displaystyle C^{2k}_b(\Omega)\cap C^{k-1}_b(\overline{\Omega})$.

\subsection{Neumann Problem}

In this section, we consider the Neumann problem
\begin{equation} \label{ell_PDE4}
\begin{split}
L(\VEC{x}, \diff)u(\VEC{x}) &= -\sum_{i,j=1}^n
\diff_{x_i} \left( b_{i,j} \diff_{x_j} u \right)(\VEC{x})
+ b_0(\VEC{x})u(\VEC{x}) = f(\VEC{x})
\quad , \quad \VEC{x} \in \Omega \ ,\\
& \sum_{i,j=1}^n b_{i,j}(\VEC{x}) \left( \diff_{x_j}u(\VEC{x}) \right)\,
\nu_i(\VEC{x})= 0
\quad , \quad \VEC{x} \in \partial \Omega \ ,
\end{split}
\end{equation}
where $\nu(\VEC{x})$ is the outward unit normal to
$\partial \Omega$ at $\VEC{x} \in \partial \Omega$.  We also assume
that $\displaystyle b_{i,j} \in C^\infty(\overline{\Omega})$ for all
$i$ and $j$, and $\displaystyle b_0 \in C^\infty(\overline{\Omega})$.

If we let $B$ be the \nn matrix with components $b_{i,j}$ for
$1\leq i,j\leq n$, then the boundary condition in (\ref{ell_PDE4}) can
be rewritten
\[
\left( \graD u(\VEC{x}) \right)^\top B(\VEC{x})\, \nu(\VEC{x}) = 0
\quad , \quad \VEC{x} \in \partial \Omega \ .
\]

\begin{egg}
If $b_{i,j} = 0$ for $i \neq j$ and              \label{ell_PDE_egg1}
$b_{i,i} = h$ for $1 \leq i \leq n$, where
$\displaystyle h \in C^{\infty}(\Omega)$
and $h(\VEC{x}) \neq 0$ for all $\VEC{x} \in \Omega$, then
\[
L(\VEC{x},\diff) u(\VEC{x})= -\Delta u(\VEC{x}) + \sum_{i=1}^n
b_i(\VEC{x}) \diff_{x_i} u(\VEC{x}) +
\frac{b_0(\VEC{x})}{h(\VEC{x})}u(\VEC{x}) = \frac{f(\VEC{x})}{h(\VEC{x})}
\quad , \quad \VEC{x} \in \Omega \ ,
\]
where $\displaystyle b_i = \frac{1}{h}\pdydx{h}{x_i}$ for $1 \leq i \leq n$,
and the boundary condition becomes
$\displaystyle \pdydx{u}{\nu} = 0$ on $\partial \Omega$.
\end{egg}

As for the Dirichlet Problem, We have already considered two
interpretations of solving problems like (\ref{ell_PDE4}) which we summarize
in the following two definitions.

\begin{defn}
If $f \in C_b(\Omega)$, a {\bfseries classical solution}
\index{Neumann Problem!Classical Solution} for
(\ref{ell_PDE4}) is a function
$\displaystyle u \in C^2_b(\Omega) \cap C^1_b(\overline{\Omega})$ such that
\[
L(\VEC{x}, \diff)u(\VEC{x}) = -\sum_{i,j=1}^n
\diff_{x_i} \left( b_{i,j} \diff_{x_j} u \right)(\VEC{x})
+ b_0(\VEC{x}) u(\VEC{x}) = f(\VEC{x}) \quad , \quad \ \VEC{x} \in \Omega \ ,
\]
and
\[
\sum_{i,j=1}^n b_{i,j}(\VEC{x}) \left( \diff_{x_j} u(\VEC{x}) \right)
\, \nu_i(\VEC{x})= 0  \quad , \quad \VEC{x} \in \partial \Omega \ .
\]
All partial derivatives are in the classical sense.
\end{defn}

\begin{defn}
If $\displaystyle f\in L_{loc}^2(\Omega)$, a {\bfseries strong solution}
\index{Neumann Problem!Strong Solution} for
(\ref{ell_PDE4}) is a distribution
$\displaystyle u \in H^{2,2}(\Omega) \subset \DD'(\Omega)$ such that
\[
\ps{ L(\VEC{x},\diff)u}{\phi} = -\sum_{i,j=1}^n
\ps{\diff_{x_i}\left( b_{i,j} \diff_{x_j} u \right)}{\phi}
+ \ps{b_0 \, u}{\phi} = \ps{f}{\phi}
\]
for all $\displaystyle \phi \in \DD(\Omega)$, and
\[
\sum_{i,j=1}^n b_{i,j} \left( \diff_{x_j} u \right)\, \nu_i= 0
\]
on $\partial \Omega$.  Obviously, all derivatives
are in the sense of distributions.
\end{defn}

Since we consider elliptic linear partial differential equations, we
can use a change of coordinates as we did
in Chapter~\ref{ChapClassifPDE} to reduce the problem in
(\ref{ell_PDE4}) to one of the form given in
Example~\ref{ell_PDE_egg1}.

As for the Dirichlet problem, there is a variational formulation of
(\ref{ell_PDE4}).  The {\bfseries variational formulation} of
(\ref{ell_PDE4}) consists in finding
$\displaystyle u\in H^{1,2}(\Omega)$ such that
\begin{equation} \label{ell_PDE5}
B(v,u) = \ps{v}{f}_2
\end{equation}
for all $\displaystyle v \in H^{1,2}(\Omega)$, where
\begin{equation} \label{ell_PDE6}
\begin{split}
B : H^{1,2}(\Omega) \times H^{1,2}(\Omega) & \rightarrow \CC \\
(v,u) &\mapsto
\sum_{i,j=1}^n \ps{\diff_{x_i} v}{b_{i,j} \diff_{x_j} u}_2 + \ps{v}{b_0 u}_2
\end{split}
\end{equation}
and, as usual, the scalar product $\ps{\cdot}{\cdot}_2$ is defined by
$\displaystyle \ps{v}{u}_2
= \int_\Omega v(\VEC{x})\overline{u(\VEC{x})} \dx{\VEC{x}}$ for
$\displaystyle u,v \in L^2(\Omega)$.
Moreover, we assume that the coefficients $b_{i,j}$ are
bounded on $\Omega$.  As we did for the Dirichlet problem, we can show
that $B$ is a bounded sequilinear form on $\displaystyle H^{1,2}(\Omega)$.

\begin{defn}
If $\displaystyle b_{i,j} \in C^\infty_b(\overline{\Omega})$,
$\displaystyle b_i \in C^\infty_b(\Omega)$ and
$\displaystyle f \in L^2(\Omega)$,
a {\bfseries weak solution}\index{Neumann Problem!Weak Solution}
for (\ref{ell_PDE5}) is a function
$\displaystyle u \in H^{1,2}(\Omega)$ such that
$\displaystyle B(v,u) = \ps{v}{f}_2$ for all
$\displaystyle v \in H^{1,2}(\Omega)$.
\end{defn}

Consider the variational problem (\ref{ell_PDE5}), where
$\displaystyle b_{i,j}\in C^{1}_b(\overline{\Omega})$, $b_0 \in C_b(\Omega)$ and
$f\in C_b(\Omega)$.  Suppose that
$\displaystyle u \in C^2_b(\Omega)\cap C^1_b(\overline{\Omega})$ is a
solution of \ref{ell_PDE5}.

For all $v \in C^1_b(\overline{\Omega})$, the divergence theorem yields
\begin{align*}
\int_{\partial \Omega}
\sum_{i=1}^n \left( \overline{b_{i,j}}\,\pdydx{\overline{u}}{x_j}
\,v \, \nu_i \right) \dx{\VEC{x}}
&= \int_\Omega \sum_{i=1}^n \pdfdx{\left( \overline{b_{i,j}}\,
\pdydx{\overline{u}}{x_j} \,v\right)}{x_i} \dx{\VEC{x}} \\
&= \int_\Omega \sum_{i=1}^n \left( \pdfdx{\left( \overline{b_{i,j}}\,
\pdydx{\overline{u}}{x_j}\right)}{x_i} \,v +
\overline{b_{i,j}}\,\pdydx{\overline{u}}{x_j} \pdydx{v}{x_i} \right)
\dx{\VEC{x}} \\
&= \sum_{i=1}^n \int_\Omega \pdfdx{\left( \overline{b_{i,j}}\,
\pdydx{\overline{u}}{x_j}\right)}{x_i} \,v \dx{\VEC{x}}
+ \sum_{i=1}^n \int_\Omega
\overline{b_{i,j}}\,\pdydx{\overline{u}}{x_j} \pdydx{v}{x_i}
\dx{\VEC{x}} \ .
\end{align*}
Hence,
\[
\sum_{i,j=1}^n \int_\Omega
\overline{b_{i,j}}\,\pdydx{\overline{u}}{x_j} \pdydx{v}{x_i} \dx{\VEC{x}}
= \int_{\partial \Omega}
\sum_{i,j=1}^n \left( \overline{b_{i,j}}\,\pdydx{\overline{u}}{x_j}
\,v \, \nu_i \right) \dx{\VEC{x}}
- \sum_{i,j=1}^n \int_\Omega \pdfdx{\left( \overline{b_{i,j}}\,
\pdydx{\overline{u}}{x_j}\right)}{x_i} \,v \dx{\VEC{x}} 
\]
for all $v \in C^1_b(\overline{\Omega})$.  It then follows from
$B(v,u) = \ps{v}{f}$ that
\begin{equation} \label{ell_Neum_CWC}
\ps{v}{f}_2 
= \int_\Omega \overline{\left(-
\sum_{i,j=1}^n \pdfdx{\left( b_{i,j} \pdydx{u}{x_j} \right)}{x_i}
+ b_0 u \right)} \, v \dx{\VEC{x}} + \int_{\partial \Omega}
\sum_{i,j=1}^n \left( \overline{b_{i,j}}\,\pdydx{\overline{u}}{x_j}
\,v \, \nu_i \right) \dx{\VEC{x}} 
\end{equation}
for $\displaystyle v \in C^{1}_b(\overline{\Omega})$.
From (\ref{ell_Neum_CWC}) with $\displaystyle v \in C^1_b(\Omega)$, we get
\[
\ps{v}{f}_2 = \int_\Omega \overline{\left(-
\sum_{i,j=1}^n \pdfdx{\left( b_{i,j} \pdydx{u}{x_j} \right)}{x_i}
+ b_0 u \right)} \, v \dx{\VEC{x}}
\]
for $\displaystyle v \in C^{1}_b(\Omega)$.
Since $\displaystyle C^{1}_b(\Omega) \supset \DD(\Omega)$ is dense in
$\displaystyle L^2(\Omega)$, and $L(\VEC{x},\diff)u$ and $f$ are
continuous on $\Omega$, we find that
\begin{equation} \label{ell_Neum_CWCB}
L(\VEC{x}, \diff)u = - \sum_{i,j=1}^n
\pdfdx{\left( b_{i,j} \pdydx{u}{x_j} \right)}{x_i}(\VEC{x})
+ b_0(\VEC{x}) u(\VEC{x}) = f(\VEC{x})
\end{equation}
for $\VEC{x} \in \Omega$.  This means that
the integrals over $\Omega$ in (\ref{ell_Neum_CWC}), including
$\displaystyle \ps{v}{f} = \int_\Omega v\,f \dx{\VEC{x}}$, do not
depend on the value of $\displaystyle v \in C^1_b(\overline{\Omega})$ on the
boundary $\partial \Omega$ (a set of measure zero).  For
(\ref{ell_Neum_CWC}) to be true for all $v \in C^1_b(\overline{\Omega})$,
we must have that
\[
\int_{\partial \Omega}
\sum_{i,j=1}^n \left( \overline{b_{i,j}}\,\pdydx{\overline{u}}{x_j}
\,v \, \nu_i \right) \dx{\VEC{x}} = 0
\]
for all $\displaystyle v \in C^{1}_b(\overline{\Omega})$.
Finally, since $\displaystyle C_b^1(\overline{\Omega})\big|_{\partial \Omega}$
is dense in $\displaystyle L^2(\partial \Omega)$, and the $b_{i,j}$ and
$\displaystyle \pdydx{u}{x_j}$ are continuous on $\partial \Omega$, we
get the boundary condition
\begin{equation} \label{ell_rm_bdr_Neum}
\sum_{i,j=1}^n b_{i,j}(\VEC{x})\,\pdydx{u}{x_j}(\VEC{x})\,
\nu_i(\VEC{x}) = 0
\end{equation}
for all $\VEC{x} \in \partial \Omega$.
So $u$ is the classical solution of (\ref{ell_PDE1}).

In Example~\ref{eggNewmanBdr} below, we will illustrate for the
Laplace operator why no boundary conditions are needed for the
variational formulation.

\section{Adjoint Boundary Conditions}

For the Dirichlet and Neumann problems presented in the previous
section, the boundary conditions were such that the variational
formulation of the partial differential equation did not involve any
boundary conditions.  This is not always possible or desirable.

In this section, we present a general theory to deduce the variational
problem with its adjoint boundary conditions associated to an elliptic partial
differential equation.  However, as we have seen in previous examples
and as we will see in future examples, it is often easier to directly 
deduce the variational problem with its adjoint boundary conditions
without referring to the general results of the present section.

We assume that $\Omega$ is a bounded open
subset of $\displaystyle \RR^n$ satisfying at least a strong
$m$-extension property with $m\geq k$, and $X$ is a subspace of
$\displaystyle H^{k,2}(\Omega)$ containing
$\displaystyle H^{k,2}_0(\Omega)$.  We consider the elliptic partial 
differential equation
\[
L(\VEC{x}, \diff)u = \sum_{|\VEC{\alpha}|,|\VEC{\beta}|\leq k}
(-1)^{|\VEC{\alpha}|} \diff^{\VEC{\alpha}}
\left( b_{\VEC{\alpha},\VEC{\beta}}(\VEC{x}) \diff^{\VEC{\beta}} u\right) = f
\]
on $\Omega$, where the coefficients $b_{\VEC{\alpha},\VEC{\beta}}$ are
all of class $\displaystyle C^\infty_b(\Omega)$ and $f\in L^2(\Omega)$.

\begin{defn}
A set $\{P_j(\VEC{x},\diff) \}_{j\in J}$ of differential operators defined
on an open set containing $\partial \Omega$, where $J$ is some index set, is
{\bfseries normal}\index{Normal Set of Differential Operators} on
$\partial \Omega$ if
\begin{enumerate}
\item $P_j(\VEC{x},\diff)$ is a differential operator of exactly order $j$
for all $j$. 
\item For each $j$, $\partial \Omega$ is not a characteristic surface
of the operator $P_j(\VEC{x},\diff)$.
\end{enumerate}
\end{defn}

The best known normal set of operators for a partial differential
equation of order $m=2k$ in divergence form is
$\displaystyle \left\{ \pdydxnS{}{\nu}{j} \right\}_{j=0}^{k-1}$
in a tubular neighbourhood $O_\mu$ of $\partial \Omega$,
where $\nu(\VEC{x})$ is the outward unit normal to
$\partial \Omega$ at $\VEC{x}\in \partial \Omega$.
Recall that we have defined $\displaystyle \pdydxS{u}{\nu}$
for $u:\Omega_\mu \to \RR$ in Section~\ref{subsect_SLpot} as
\[
\pdydxS{u}{\nu}(\VEC{z}_t)
\equiv \dfdx{u\big(\VEC{x}+t\nu(\VEC{x})\big)}{t}
= \graD u(\VEC{z}_t) \cdot \nu(\VEC{x})
= \graD u\big(\VEC{x}+t\nu(\VEC{x})\big) \cdot \nu(\VEC{x})
\]
for $\VEC{z}_t = \VEC{x} + t\nu(\VEC{x})$ with $\VEC{x} \in \partial \Omega$
and $|t|< \mu$.

\begin{theorem}
Let
\begin{equation} \label{ell_adj_BCs}
\begin{split}
B : H^{k,2}(\Omega) \times H^{k,2}(\Omega) & \rightarrow \CC \\
(v,u) &\mapsto
\sum_{|\VEC{\alpha}|,|\VEC{\beta}|\leq k}
\ps{\diff^{\VEC{\alpha}} v}{b_{\VEC{\alpha},\VEC{\beta}}(\VEC{x})
\diff^{\VEC{\beta}} u}_2
\end{split}
\end{equation}
be the Dirichlet form associated to $L(\VEC{x},\diff)$.  Given a normal
set of differential operators
$\displaystyle \{ P_j(\VEC{x},\diff)\}_{j=0}^{k-1}$,
there exists a set of differential operators
$\displaystyle \{ \tilde{P}_j(\VEC{x},\diff) \}_{j=k}^{2k-1}$ such that
\[
B(v,u) - \ps{v}{L(\VEC{x},\diff)u}_2
= \sum_{j=0}^{k-1} \int_{\partial \Omega} P_j(\VEC{x},\diff)v(\VEC{x})
\ \overline{\tilde{P}_{2k-1-j}(\VEC{x},\diff)u(\VEC{x})} \dx{S}
\]
for all $\displaystyle u,v \in C^{2k}(\overline{\Omega})$.
Moreover, if $B$ is coercive, the set of differential operators
$\displaystyle \{ \tilde{P}_j(\VEC{x},\diff) \}_{j=k}^{2k-1}$ is normal.
\end{theorem}

A long proof of this theorem is given in \cite{FoPDE}.

\begin{theorem} \label{ell_normal_BCs}
Suppose that $\Omega$ is a bounded open subset of $\displaystyle \RR^n$
satisfying at a strong $k$-extension property.
Let $\displaystyle \{ P_j(\VEC{x},\diff) \}_{j=0}^{k-1}$, be a normal set of
differential operators associated to $L(\VEC{x},\diff)u=f$.  Given
$\displaystyle g_j\in C^\infty(\partial \Omega)$ with $0\leq j<k$, there
exists $\displaystyle g \in C^\infty_c(\RR^n)$ such that
$P_j(\VEC{x},\diff)g=g_j$ on $\partial \Omega$ for $0\leq j<k$.
\end{theorem}

\begin{proof}
\stage{i} Suppose that
$\displaystyle P_j(\VEC{x},\diff) = \pdydxnS{}{\nu}{j}$ on a
tubular neighbourhood $O_\epsilon$ of $\partial \Omega$ as given
by Theorem~\ref{pot_TBN}.  Choose $\phi\in C^\infty_c(\RR)$ such that
$\displaystyle \supp \phi \subset ]-\epsilon/2, \epsilon/2[$ and
$\phi(x) = 1$ for $x\in]-\epsilon/4, \epsilon/4[$.

Consider the function $\displaystyle g:\RR^n \rightarrow \RR$ defined by
\[
g(\VEC{x}) =
\begin{cases}
\displaystyle \phi(t) \sum_{j=0}^k \frac{t^j}{j!} g_j(\VEC{y}) &\quad
\text{if} \ \VEC{x} = \VEC{y} + t\, \nu(\VEC{y}) \
\text{with} \ \VEC{y} \in \partial \Omega \ \text{and} \
-\epsilon < t < \epsilon \\[0.7em]
0 & \quad \text{if} \ \VEC{x} \not\in O_\epsilon
\end{cases}
\]
We have that $\displaystyle g\in C^\infty_c(\RR^n)$,
$\displaystyle \supp g \subset O_\epsilon$ and
\[
\big( P_j(\VEC{x},\diff)g\big)(\VEC{y}) = \pdydxnS{g}{\nu}{j}(\VEC{y})
= \pdydxn{g}{t}{j}\big(\VEC{y}+t\, \nu(\VEC{y})\big)\bigg|_{t=0}
= g_j(\VEC{y})
\]
for $\VEC{y} \in \partial \Omega$.

\stage{ii} The general form of $P_j(\VEC{x},\diff)$ is
\[
P_j(\VEC{x},\diff) = \sum_{i=0}^j Q_{j,i}(\VEC{x},\diff)\,
\pdydxn{}{\nu}{i} \  ,
\]
where $Q_{j,i}(\VEC{x},\diff)$ is a differential operator involving only
directional derivatives in directions tangent to $\partial \Omega$ and
of order smaller than or equal to $j-i$.  In particular,
$Q_{j,j}(\VEC{x},\diff)$ is simply a function
$\displaystyle q_j:\RR^n\rightarrow \RR$
such that $q_j(\VEC{x}) \neq 0$ for all $\VEC{x} \in \partial \Omega$
because $\partial \Omega$ is not a characteristic surface for the
operator $P_j(\VEC{x},\diff)$ by definition of a normal set of differential
operators.

Hence $P_j(\VEC{x},\diff)g=g_j$ for $0\leq j < k$ is equivalent to
$\displaystyle \pdydxn{g}{\nu}{j} = f_j$ for $0\leq j < k$, where
the $f_j$ are given recursively by $\displaystyle f_0 = \frac{g_0}{q_0}$
and
$\displaystyle f_j = \frac{1}{q_j} \left( g_j - \sum_{i=0}^{j-1}
Q_{j,i}(\VEC{x},\diff) f_i \right)$
for $j=1$, $2$, \ldots, $k-1$.

The problem is reduced to (i); namely, to find $g$ such
that $\displaystyle \pdydxn{g}{\nu}{j} = f_j$ for $0\leq j < k$.
\end{proof}

The following examples were suggested in \cite{FoPDE}.

\begin{egg}
Consider the partial differential equation
\begin{equation} \label{ABCegg1eq0}
L(\VEC{x}, \diff)u(\VEC{x}) = \sum_{|\VEC{\alpha}|,|\VEC{\beta}|\leq k}
(-1)^{|\VEC{\alpha}|} \diff^{\VEC{\alpha}}
\left( b_{\VEC{\alpha},\VEC{\beta}}(\VEC{x}) \diff^{\VEC{\beta}}
u(\VEC{x})\right) = f(\VEC{x})
\quad , \quad \VEC{x} \in \Omega \ ,
\end{equation}
where no boundary conditions are given.  To find the
{\bfseries free or natural boundary conditions}%
\index{Elliptic Partial Differential Equation!Free Boundary Conditions}%
\index{Elliptic Partial Differential Equation!Natural Boundary Conditions}
associated to this
problem, consider the normal family of differential operators 
$\displaystyle \left\{ \pdydxnS{}{\nu}{j} \right\}_{j=0}^{k-1}$.
According to Theorem~\ref{ell_adj_BCs}, there exists a set of
differential operators $\displaystyle \{ \tilde{P}_j\}_{j=k}^{2k-1}$
such that
\begin{equation} \label{ABCegg1eq1}
B(v,u) - \ps{v}{L(\VEC{x},\diff)u} =
\sum_{j=0}^{k-1} \int_{\partial \Omega} \pdydxnS{v}{\nu}{j}(\VEC{x}) \,
\overline{\tilde{P}_{2k-1-j}(u)(\VEC{x})} \dx{S}
\end{equation}
for all $\displaystyle v,u \in C^{2k}_c(\overline{\Omega})$,
where $B$ is defined in (\ref{ell_adj_BCs}).
Hence the variational formulation of $L(\VEC{x},\diff)u=f$ is to find
$\displaystyle u\in H^{k,2}(\Omega)$ such that
\begin{equation} \label{ABCegg1eq2A}
B(v,u) = \ps{v}{f}
\end{equation}
for all $\displaystyle v\in H^{k,2}(\Omega)$ with
\begin{equation} \label{ABCegg1eq2B}
\tilde{P}_{j}(u)(\VEC{x}) = 0 \quad , \quad \VEC{x} \in \partial \Omega \
\text{and} \ k\leq j < 2k \ .
\end{equation}
It follows form (\ref{ABCegg1eq1}) that a solution $u$ of
(\ref{ABCegg1eq2A}) and (\ref{ABCegg1eq2B}) is a solution of
\[
\ps{v}{L(\VEC{x},\diff)u} = \ps{v}{f}
\]
for all $\displaystyle  v\in H^{k,2}(\Omega)$.
One can show that the solution $u$ is in fact in
$\displaystyle H^{2k,2}(\Omega)$ if $f$ is nice\\
enough \footnote{The main result of Section~\ref{ell_sect_RB} on
regularity at the boundary may be used when $k=1$.}, and so the
previous free boundary conditions $\tilde{P}_{j}(u) = 0$ on
$\partial \Omega$ for $k \leq j < 2k$ make sense.
\end{egg}

\begin{egg}
We may create more complex Dirichlet problems than those mentioned
before.  Consider a normal family of differential operators 
$\displaystyle \left\{ P_j \right\}_{j=0}^{k-1}$ for
$L(\VEC{x},\diff)u = f$ in $\Omega$, where $\displaystyle f \in L^2(\Omega)$.

Let $J_1$ and $J_2$ be two non-empty, disjoint subsets of
$J=\{0,1,2,\ldots,k-1\}$ such that $J=J_1\cup J_2$.  Our goal is to
find the variational formulation of
\begin{equation} \label{ell_egg1_adj_BCs}
\begin{split}
L(\VEC{x},\diff)u(\VEC{x}) &= f(\VEC{x}) \quad , \quad \VEC{x} \in \Omega \ , \\
P_j(u)(\VEC{x}) &= 0 \quad , \quad \VEC{x} \in \partial \Omega \ \text{and}
\ j\in J_1 \ ,
\end{split}
\end{equation}
where the linear partial differential operator $L(\VEC{x},\diff)$ is
defined in (\ref{ABCegg1eq0}).
Consider the space $X$ defined as the closure in
$\displaystyle H^{k,2}(\Omega)$ (always
with respect to the norm $\|\cdot\|_{k,2}$) of the set
\[
S = \left\{ u \in C^\infty(\overline{\Omega}) : P_j(u) = 0 \ \text{on}
\ \partial \Omega \ \text{for} \ j\in J_1 \right\} \ .
\]
According to Theorem~\ref{ell_adj_BCs}, there exists a set of
differential operators $\displaystyle \{ \tilde{P}_j\}_{j=k}^{2k-1}$
such that
\[
B(v,u) - \ps{v}{L(\VEC{x},\diff)u} =
\sum_{j\in J} \int_{\partial \Omega} P_j(v)(\VEC{x}) \,
\overline{\tilde{P}_{2k-1-j}(u)(\VEC{x})} \dx{S}
\]
for all $\displaystyle v,u \in C^\infty(\overline{\Omega})$.
Hence
\begin{equation} \label{ABCegg2eq1}
B(v,u) - \ps{v}{L(\VEC{x},\diff)u} =
\sum_{j\in J_2} \int_{\partial \Omega} P_j(v)(\VEC{x}) \,
\overline{\tilde{P}_{2k-1-j}(u)(\VEC{x})} \dx{S}
\end{equation}
for all $\displaystyle v,u \in X \cap C^\infty(\overline{\Omega})$.
Given $\displaystyle u\in X \cap C^\infty(\overline{\Omega})$, we may use
Theorem~\ref{ell_normal_BCs} to find
$\displaystyle v \in C^\infty(\overline{\Omega})$ such that
$P_j(v)=0$ for $j\in J_1$ and $P_j(v) = \tilde{P}_{2k-1-j}(u)$ for $j\in J_2$.
However, to get the variational formulation, the right hand side of
(\ref{ABCegg2eq1}) must be null for all
$\displaystyle v\in X \cap C^\infty(\overline{\Omega})$.  Therefore,
we must require that $\displaystyle \tilde{P}_{2k-1-j}(u) = 0$ on
$\partial \Omega$ for all $j \in J_2$ to get the variational formulation.
Hence, the variational formulation of (\ref{ell_egg1_adj_BCs}) is to
find $u\in X$ such that
\begin{equation} \label{ABCegg2eq2A}
  B(v,u) = \ps{v}{f}
\end{equation}
for all $\displaystyle v\in X$ with
\begin{equation} \label{ABCegg2eq2B}
\tilde{P}_{2k-1-j}(u)(\VEC{x}) = 0  \quad , \quad \VEC{x} \in \partial \Omega
\ \text{and} \ j \in J_2 \ .
\end{equation}
It follows form (\ref{ABCegg2eq1}) that a solution $u$ of
(\ref{ABCegg2eq2A}) and (\ref{ABCegg2eq2B}) is a solution of
\[
\ps{v}{L(\VEC{x},\diff)u} = \ps{v}{f}
\]
for all $\displaystyle v\in X$.
As for the previous example, the missing step is to show that the
solution $u$ is in fact in $\displaystyle H^{2k,2}(\Omega)$ and so the
boundary conditions on $\partial \Omega$ for $k \leq j < 2k$ make sense.
\end{egg}

\section{Weakly Coercive Bilinear Mappings}

As usual, let $\Omega$ be a bounded open subset of $\displaystyle \RR^n$ with a
sufficiently smooth boundary.  Recall that a bounded sequilinear mapping
$\displaystyle B:H^{k,2}(\Omega)\times H^{k,2}(\Omega) \rightarrow \CC$
is coercive if there exists a constant $C>0$ such that 
$\displaystyle \RE B(f,f) \geq C \|f||_{k,2}^2$ for all
$\displaystyle f \in H^{k,2}(\Omega)$.

\begin{defn} \label{ell_weakcoerc}
Let $\displaystyle H = H^{k,2}(\Omega)$ or $H=H^{k,2}_0(\Omega)$.
We say that a bounded sequilinear mapping
$B:H \times H \rightarrow \CC$ is {\bfseries weakly coercive}
\index{Weakly Coercive Bounded Sequilinear Mapping} if 
there exist two constants $C>0$ and $\lambda >0$ such that 
$\displaystyle \RE B(f,f) \geq C \|f||_{k,2,\Omega}^2 - \lambda
\|f||_{2,\Omega}^2$ for $f \in H$.
\end{defn}

The main result of this section is G\r{a}rding's inequality.  Before
stating and proving this result, we need several lemmas.

\begin{lemma} \label{ell_garding4}
For $r<s<t$ and $\epsilon>0$, there exists $C>0$ such that
\[
\|f\|_{s,\rho}^2 \leq \epsilon \| f\|_{t,\rho}^2 + C \| f \|_{r,\rho}^2
\]
for $\displaystyle f \in H^{t,2}(\RR^n)$.
We also have for $r<s<t$, non-negative integers, that there exists
$C>0$ such that
\[
\|f\|_{s,2,\RR^n}^2 \leq \epsilon \| f\|_{t,2,\RR^n}^2 + C \| f \|_{r,2,\RR^n}^2
\]
for $\displaystyle f \in H^{t,2}(\RR^n)$.
\end{lemma}

\begin{proof}
\stage{i} Choose $R>0$ such that $(1+r^2)^s < \epsilon (1+r^2)^t$ for $r> R$.
This is possible because
\[
\lim_{r\rightarrow \infty} \frac{(1-r^2)^s}{(1-r^2)^t} = 0
\]
for $t>s$.  Let $C= (1+R^2)^{s-r}$.  Then
\[
(1+\|\VEC{y}\|^2)^s \leq C (1+\|\VEC{y}\|^2)^r \leq
\epsilon (1+\|\VEC{y}\|^2)^t + C (1+\|\VEC{y}\|^2)^r
\]
for $\|\VEC{y}\|\leq R$ and
\[
(1+\|\VEC{y}\|^2)^s \leq \epsilon (1+\|\VEC{y}\|^2)^t \leq
\epsilon (1+\|\VEC{y}\|^2)^t + C (1+\|\VEC{y}\|^2)^r
\]
for $\|\VEC{y}\|> R$.  It follows that
\[
\int_{\RR^n} \left|\hat{f}(\VEC{y})\right|^2 (1+\|\VEC{y}\|^2)^s \dx{\VEC{y}}
\leq \epsilon \int_{\RR^n} \left|\hat{f}(\VEC{y})\right|^2
(1+\|\VEC{y}\|^2)^t \dx{\VEC{y}}
+ C \int_{\RR^n} \left|\hat{f}(\VEC{y})\right|^2
(1+\|\VEC{y}\|^2)^r \dx{\VEC{y}}
\]
for $\displaystyle f \in H^{t,2}(\RR^n)$.

\stage{ii}  Suppose that $r<s<t$ are non-negative integers.
According to Theorem~\ref{sob_wk2_wk2} about the equivalence of the
Sobolev norms, there exist constants $Q_1$, $Q_2$ and $Q_3$ such that
$\|\cdot\|_{s,2} \leq Q_1 \|\cdot\|_{s,\rho}$,
$\|\cdot\|_{t,\rho} \leq Q_2 \|\cdot\|_{t,2}$ and
$\|\cdot\|_{t,\rho} \leq Q_3 \|\cdot\|_{r,2}$.

If we apply (i) with $\epsilon$ replaced by $\epsilon/(Q_1 Q_1)$, we
find that there exists a constant $C_1$ such that
\[
\|f\|_{s,\rho}^2 \leq \frac{\epsilon}{Q_1 Q_2} \| f\|_{t,\rho}^2
+ C_1 \| f \|_{r,\rho}^2
\]
for $\displaystyle f \in H^{t,2}(\RR^n)$.  Hence,
\[
\frac{1}{Q_1}\, \|f\|_{s,2,\RR^n}^2 \leq \frac{\epsilon}{Q_1}
\| f\|_{t,2,\RR^n}^2 + C_1 Q_3 \| f \|_{r,2,\RR^n}^2
\]
for $\displaystyle f \in H^{t,2}(\RR^n)$.  We then get that
\[
\|f\|_{s,2,\RR^n}^2 \leq \epsilon \| f\|_{t,2,\RR^n}^2 + C \| f \|_{r,2,\RR^n}^2
\]
for $\displaystyle f \in H^{t,2}(\RR^n)$, where $C= C_1 Q_1 Q_3$.
\end{proof}

\begin{lemma} \label{ell_garding2}
Let $\Omega$ be a bounded open subset of $\displaystyle \RR^n$.  Suppose that
\[
L(\VEC{x},\diff)u = \sum_{|\VEC{\alpha}|,|\VEC{\beta}|= k} (-1)^{\VEC{\alpha}}
\diff^{\VEC{\alpha}} \left( b_{\VEC{\alpha},\VEC{\beta}}
\diff^{\VEC{\beta}} u\right)
\]
is strongly elliptic on $\overline{\Omega}$, where
$b_{\VEC{\alpha},\VEC{\beta}} \in \CC$ for all $\VEC{\alpha}$ and
$\VEC{\beta}$.  Then
\[
\begin{split}
B : H^{k,2}_0(\Omega) \times H^{k,2}_0(\Omega) & \rightarrow \CC \\
(v,u) &\mapsto
\sum_{|\VEC{\alpha}|,|\VEC{\beta}|= k}
\ps{\diff^{\VEC{\alpha}} v}{b_{\VEC{\alpha},\VEC{\beta}} \diff^{\VEC{\beta}} u}_2
\end{split}
\]
is coercive.  Namely, there exists $K>0$ such that
\[
\RE B(u,u) = \RE \sum_{|\VEC{\alpha}|,|\VEC{\beta}|= k}
\ps{\diff^{\VEC{\alpha}} u}{b_{\VEC{\alpha},\VEC{\beta}}
\diff^{\VEC{\beta}} u}_2 \geq K \|u\|_{k,2,\Omega}^2
\]
for $\displaystyle u  \in H^{k,2}_0(\Omega)$.
\end{lemma}

\begin{proof}
For $\displaystyle u \in H^{k,2}_0(\Omega)$, let
$\displaystyle \underline{u}: \RR^n \rightarrow \CC$ be the function defined by
\[
\underline{u}(\VEC{x}) =
\begin{cases}
u(\VEC{x}) & \quad \text{if} \ \VEC{x} \in \Omega \\
0 & \quad \text{if} \ \VEC{x} \in \RR^n \setminus \Omega
\end{cases}
\]
From Proposition~\ref{sob_expand_WKP}, we have that
$\displaystyle \underline{u}\in H^{k,2}(\RR^n)$.
Using Plancherel theorem, Theorem~\ref{distr_plancherel}, we may write
\begin{align*}
B(u,u) &= \sum_{|\VEC{\alpha}|,|\VEC{\beta}|= k}
\overline{b_{\VEC{\alpha},\VEC{\beta}}} \int_{\Omega}
\overline{\diff^{\VEC{\beta}} u(\VEC{x})}\, \diff^{\VEC{\alpha}} u(\VEC{x})
\dx{\VEC{x}}
= \sum_{|\VEC{\alpha}|,|\VEC{\beta}|= k} \overline{b_{\VEC{\alpha},\VEC{\beta}}}
\int_{\RR^n} \overline{\diff^{\VEC{\beta}} \underline{u}(\VEC{x})}\,
\diff^{\VEC{\alpha}} \underline{u}(\VEC{x}) \dx{\VEC{x}} \\
&= \sum_{|\VEC{\alpha}|,|\VEC{\beta}|= k} \overline{b_{\VEC{\alpha},\VEC{\beta}}}
\int_{\RR^n} \overline{(\diff^{\VEC{\beta}} \underline{u})^\wedge(\VEC{x})}\,
(\diff^{\VEC{\alpha}} \underline{u})^\wedge(\VEC{x}) \dx{\VEC{x}}
= \sum_{|\VEC{\alpha}|,|\VEC{\beta}|= k} \overline{b_{\VEC{\alpha},\VEC{\beta}}}
\int_{\RR^n} (i\VEC{x})^{\VEC{\beta}} (-i\VEC{x})^{\VEC{\alpha}}
\left| \hat{\underline{u}}(\VEC{x}) \right|^2 \dx{\VEC{x}} \\
&= \int_{\RR^n} \left( \sum_{|\VEC{\alpha}|,|\VEC{\beta}|= k}
\overline{b_{\VEC{\alpha},\VEC{\beta}}} \,\VEC{x}^{\VEC{\beta}}
\VEC{x}^{\VEC{\alpha}} \right)
\left| \hat{\underline{u}}(\VEC{x}) \right|^2 \dx{\VEC{x}}
\end{align*}
for $\displaystyle u \in H^{k,2}_0(\Omega)$.
Since $L(\VEC{x},\diff)$ is strongly elliptic on $\overline{\Omega}$,
there exists $\theta_1 >0$ such that
\begin{equation} \label{ell_garding2_theta}
\RE\, Q(\VEC{x}, \VEC{\xi}) = \RE
\sum_{|\VEC{\alpha}|,|\VEC{\beta}|=k} b_{\VEC{\alpha},\VEC{\beta}}
\VEC{\xi}^{\VEC{\beta}} \VEC{\xi}^{\VEC{\alpha}} \geq \theta_1 \|\VEC{\xi}\|^{2k}
\end{equation}
for $\displaystyle \xi \in \RR^n$.  Thus,
\begin{align*}
\RE B(u,u) &\geq \theta_1 \int_{\RR^n} \| \VEC{x} \|^{2k}
\left| \hat{\underline{u}}(\VEC{x}) \right|^2 \dx{\VEC{x}}
\geq \theta_2 \sum_{|\VEC{\alpha}|=k} \int_{\RR^n}
\left| \VEC{x}^{\VEC{\alpha}} \right|^2 
\left| \hat{\underline{u}}(\VEC{x}) \right|^2 \dx{\VEC{x}} \\
&= \theta_2 \sum_{|\VEC{\alpha}|=k} \int_{\RR^n} \left|
(\diff^{\VEC{\alpha}} \underline{u})^\wedge (\VEC{x}) \right|^2 \dx{\VEC{x}} 
= \theta_2 \sum_{|\VEC{\alpha}|=k} \int_{\RR^n}
\left| \diff^{\VEC{\alpha}} \underline{u}(\VEC{x}) \right|^2 \dx{\VEC{x}}
\geq \theta_3 \| \underline{u} \|_{k,2,\RR^n}^2             
\end{align*}
for $\displaystyle u \in H^{k,2}_0(\Omega)$, where
we have used (\ref{ell_garding2_theta}) for the first inequality and
Plancherel theorem for the second equality.  The constant $\theta_3$
for the last inequality is $\theta_3 = \theta_2 C$ where $C$ comes
from the Poincaré's inequality, Theorem~\ref{sob_pt_carre}.
Moreover, for the second inequality, we have used the relation
\[
\| \VEC{x} \|^{2k} = \left( \sum_{j=1}^n x_j^2 \right)^{k}
\geq C \left( \sum_{|\VEC{\alpha}|=k} \VEC{x}^{2\VEC{\alpha}} \right)
= C \left( \sum_{|\VEC{\alpha}|=k} \left|\VEC{x}^{\VEC{\alpha}}\right|^2 \right)
\]
for $\displaystyle \VEC{x} \in \RR^n$,
where $\displaystyle 1/C = \max_{\|\VEC{x}\|=1}
\sum_{|\VEC{\alpha}|=k} \VEC{x}^{2\alpha}$,
and taken $\theta_2 = C \theta_1$.

Since $\underline{u}(\VEC{x}) = 0$ for
$\displaystyle \VEC{x} \in \RR^n \setminus \Omega$, we get
$\displaystyle \RE B(u,u) \geq \theta_3 \|u\|_{k,2,\Omega}^2$ for all
$\displaystyle u \in H^{k,2}_0(\Omega)$ with $\theta_3>0$.  We take
$K=\theta_3$ to get the conclusion of the lemma.
\end{proof}

\begin{lemma} \label{ell_garding3}
Let $\Omega$ be a bounded open subset of $\displaystyle \RR^n$.  Suppose that
\[
L(\VEC{x},\diff)u = \sum_{|\VEC{\alpha}|,|\VEC{\beta}|= k}
(-1)^{\VEC{\alpha}} \diff^{\VEC{\alpha}}
\left( b_{\VEC{\alpha},\VEC{\beta}}(\VEC{x}) \diff^{\VEC{\beta}} u\right)
\]
is strongly elliptic on $\overline{\Omega}$, where
$b_{\VEC{\alpha},\VEC{\beta}} \in C(\overline{\Omega})$ for all
$\VEC{\alpha}$ and $\VEC{\beta}$.  Then
\[
\begin{split}
B : H^{k,2}_0(\Omega) \times H^{k,2}_0(\Omega) & \rightarrow \RR \\
(v,u) &\mapsto
\sum_{|\VEC{\alpha}|,|\VEC{\beta}|= k}
\ps{\diff^{\VEC{\alpha}} v}{b_{\VEC{\alpha},\VEC{\beta}}(\VEC{x})
\diff^{\VEC{\beta}} u}_2
\end{split}
\]
is weakly coercive.
\end{lemma}

\begin{proof}
\stage{i} We prove that there exists a constant $C_1>0$ such that
$\displaystyle \RE B(u,u) \geq C_1 \|u\|_{k,2,\Omega}^2$ for all
$\displaystyle u\in H^{k,2}_0(\Omega)$ such that $\supp u$ is small enough.

Given $\VEC{z} \in \Omega$, we write $B(u,u)$ as
\begin{equation} \label{ell_garding3_B}
B(u,u) =
\sum_{|\VEC{\alpha}|,|\VEC{\beta}|= k}
\ps{\diff^{\VEC{\alpha}} u}{
\left(b_{\VEC{\alpha},\VEC{\beta}}(\VEC{x})
-b_{\VEC{\alpha},\VEC{\beta}}(\VEC{z})\right)\diff^{\VEC{\beta}} u}_2
+ \sum_{|\VEC{\alpha}|,|\VEC{\beta}|= k}
\ps{\diff^{\VEC{\alpha}} u}{b_{\VEC{\alpha},\VEC{\beta}}(\VEC{z})
\diff^{\VEC{\beta}} u}_2
\end{equation}
for $\displaystyle u \in H^{k,2}_0(\Omega)$.
Since $L(\VEC{x},\diff)$ is strongly elliptic on $\overline{\Omega}$,
there exists $\theta_1 >0$ such that
\begin{equation} \label{ell_garding3_theta1}
\RE\, Q(\VEC{z}, \VEC{\xi}) = \RE \sum_{|\VEC{\alpha}|,|\VEC{\beta}|=k}
b_{\VEC{\alpha},\VEC{\beta}}(\VEC{z}) \VEC{\xi}^{\VEC{\beta}}
\VEC{\xi}^{\VEC{\alpha}} \geq \theta_1 \|\VEC{\xi}\|^{2k}
\end{equation}
for all $\displaystyle \VEC{\xi} \in \RR^n$ and $\VEC{z}\in \overline{\Omega}$.
It follows that $\theta_1$ in (\ref{ell_garding2_theta}) in
the proof of Lemma~\ref{ell_garding2} may be assumed to be independent of
$\VEC{z} \in \overline{\Omega}$.  Therefore, we may assume that the
constant $K>0$ for the coercive property in Lemma~\ref{ell_garding2}
is independent of $\VEC{z} \in \overline{\Omega}$; namely,
\begin{equation} \label{ell_garding_local1}
\RE \sum_{|\VEC{\alpha}|,|\VEC{\beta}|= k}
\ps{\diff^{\VEC{\alpha}} u}
{b_{\VEC{\alpha},\VEC{\beta}}(\VEC{z}) \diff^{\VEC{\beta}} u}_2
\geq K \|u\|_{k,2,\Omega}^2
\end{equation}
for $\displaystyle u \in H^{k,2}_0(\Omega)$ and $\VEC{z} \in \Omega$.
Since the $b_{\VEC{\alpha},\VEC{\beta}}$ are continuous on
$\overline{\Omega}$ by hypothesis and $\overline{\Omega}$ is compact
because $\Omega$ is bounded, we have that the
$b_{\VEC{\alpha},\VEC{\beta}}$ are uniformly continuous on
$\overline{\Omega}$.  Thus, there exists $\delta >0$ such that
$\displaystyle \sum_{|\VEC{\alpha}|,|\VEC{\beta}|= k}
\left|b_{\VEC{\alpha},\VEC{\beta}}(\VEC{x}_1)
-b_{\VEC{\alpha},\VEC{\beta}}(\VEC{x}_2)\right| < \frac{K}{2}$
for $\displaystyle \left| \VEC{x}_1- \VEC{x}_2 \right| < \delta$ and
$\VEC{x}_1, \VEC{x}_2 \in \overline{\Omega}$.
Hence, for all $\VEC{z} \in \Omega$ and all
$\displaystyle u \in H^{k,2}_0(\Omega)$ with
$\supp u \subset B_\delta(\VEC{z}) \cap \Omega$, we have
\begin{align}
&\left| \sum_{|\VEC{\alpha}|,|\VEC{\beta}|= k} \int_{\Omega}
\overline{\left(b_{\VEC{\alpha},\VEC{\beta}}(\VEC{x})
-b_{\VEC{\alpha},\VEC{\beta}}(\VEC{z})\right)
\diff^{\VEC{\beta}} u(\VEC{x})}\, \diff^{\VEC{\alpha}} u(\VEC{x})
\dx{\VEC{x}} \right|
\nonumber \\
& \leq \sum_{|\VEC{\alpha}|,|\VEC{\beta}|= k} \int_{\Omega}
\left|b_{\VEC{\alpha},\VEC{\beta}}(\VEC{x})
-b_{\VEC{\alpha},\VEC{\beta}}(\VEC{z})\right|
\left| \diff^{\VEC{\beta}} u(\VEC{x})\right|\,
\left| \diff^{\VEC{\alpha}} u(\VEC{x}) \right|\dx{\VEC{x}} \nonumber \\
& \leq \sup_{\VEC{x}\in B_\delta(\VEC{z}) \cap \Omega}
\left( \sum_{|\VEC{\alpha}|,|\VEC{\beta}|= k}
\left|b_{\VEC{\alpha},\VEC{\beta}}(\VEC{x})
-b_{\VEC{\alpha},\VEC{\beta}}(\VEC{z})\right| \right)
\int_{\Omega} \left| \diff^{\VEC{\beta}} u(\VEC{x})\right|\,
\left| \diff^{\VEC{\alpha}} u(\VEC{x}) \right|\dx{\VEC{x}} \nonumber \\
& \leq \sup_{\VEC{x}\in B_\delta(\VEC{z}) \cap \Omega}
\left( \sum_{|\VEC{\alpha}|,|\VEC{\beta}|= k}
\left|b_{\VEC{\alpha},\VEC{\beta}}(\VEC{x})
-b_{\VEC{\alpha},\VEC{\beta}}(\VEC{z})\right| \right)
\left( \int_{\Omega} \left| \diff^{\VEC{\beta}} u(\VEC{x})\right|^2 \dx{\VEC{x}}
\right)^{1/2}
\left( \int_{\Omega} \left| \diff^{\VEC{\alpha}} u(\VEC{x}) \right|^2 \dx{\VEC{x}}
\right)^{1/2} \nonumber \\
& < \frac{K}{2} \|u\|_{k,2,\Omega}^2 \ . \label{ell_garding_local2}
\end{align}
It follows from (\ref{ell_garding3_B}), (\ref{ell_garding_local1}) and
(\ref{ell_garding_local2}) that
\begin{equation} \label{ell_garding3_delta}
\RE B(u,u) \geq \frac{K}{2} \|u\|_{k,2,\Omega}
\end{equation}
for all $\displaystyle u \in H^{k,2}_0(\Omega)$ such that there
exists $z\in \Omega$ with $\supp u \subset B_{\delta}(\VEC{z}) \cap \Omega$.
Let $C_1=K/2$.

\stage{ii} We now prove the lemma.  Since $\overline{\Omega}$ is
compact, there exists a finite subset
$\displaystyle \{\VEC{z}_j\}_{j=1}^J$ of $\Omega$ such that
$\displaystyle \{B_\delta(\VEC{z}_j)\}_{j=1}^J$ is an open cover of
$\overline{\Omega}$.  Let $\displaystyle \{\phi_j\}_{j=1}^J$ be a
partition of unity subordinate to this open cover.  Since
$\displaystyle \sum_{j=1}^J \phi(\VEC{x}) = 1$ for all
$\VEC{x} \in \overline{\Omega}$,
we have that $\displaystyle \sum_{j=1}^J |\phi(\VEC{x})|^2 > 0$ for all
$\VEC{x} \in \overline{\Omega}$.  Let
$\displaystyle \psi_j(\VEC{x}) = \phi_j(\VEC{x})
\left( \sum_{j=1}^J |\phi(\VEC{x})|^2\right)^{-1}$
for $\VEC{x} \in \overline{\Omega}$.
We have that $\psi_j \in C^\infty(\overline{\Omega})$ for all $j$ and
$\displaystyle \sum_{j=1}^J |\psi_j(\VEC{x})|^2 = 1$ for all
$\VEC{x}\in \overline{\Omega}$.  Hence,
\begin{align*}
B(u,u) &= \sum_{|\VEC{\alpha}|,|\VEC{\beta}|= k}
\ps{\diff^{\VEC{\alpha}} u}{ \sum_{j=1}^J |\psi_j(\VEC{x})|^2\,
b_{\VEC{\alpha},\VEC{\beta}}(\VEC{x}) \diff^{\VEC{\beta}} u}_2
= \sum_{|\VEC{\alpha}|,|\VEC{\beta}|= k} \sum_{j=1}^J \ps{\diff^{\VEC{\alpha}} u}
{|\psi_j(\VEC{x})|^2\, b_{\VEC{\alpha},\VEC{\beta}}(\VEC{x})
\diff^{\VEC{\beta}} u}_2 \\
&= \sum_{|\VEC{\alpha}|,|\VEC{\beta}|= k} \sum_{j=1}^J
\ps{\psi_j \diff^{\VEC{\alpha}} u}{b_{\VEC{\alpha},\VEC{\beta}}(\VEC{x})
\,\psi_j \diff^{\VEC{\beta}} u}_2 = Q_1(u) + Q_2(u)
\end{align*}
for $\displaystyle u \in H^{k,2}_0(\Omega)$, where
\begin{align*}
Q_1(u) &= \sum_{|\VEC{\alpha}|,|\VEC{\beta}|= k} \sum_{j=1}^J
\ps{\diff^{\VEC{\alpha}}\left( \psi_j\, u\right)}
{b_{\VEC{\alpha},\VEC{\beta}}(\VEC{x})\,\diff^{\VEC{\beta}}
\left( \psi_j\, u\right)}_2
\intertext{and}
Q_2(u) &= \sum_{|\VEC{\alpha}|,|\VEC{\beta}|= k} \sum_{j=1}^J
\ps{\diff^{\VEC{\alpha}}\left( \psi_j\, u\right)}
{b_{\VEC{\alpha},\VEC{\beta}}(\VEC{x})
\left(\psi_j \diff^{\VEC{\beta}} u - \diff^{\VEC{\beta}}
\left( \psi_j\, u\right)\right)}_2 \\
&\qquad + \sum_{|\VEC{\alpha}|,|\VEC{\beta}|= k} \sum_{j=1}^J
\ps{\psi_J \diff^{\VEC{\alpha}} u - \diff^{\VEC{\alpha}}\left( \psi_j\, u\right)}
{b_{\VEC{\alpha},\VEC{\beta}}(\VEC{x})\psi_j \diff^{\VEC{\beta}} u}_2 \ .
\end{align*}
Since $\displaystyle \psi_j\,u \in H^{k.2}_0(\Omega)$ has compact support in
$B_\delta(\VEC{z}_j)$ for all $j$, we have from part (i) that
\begin{equation} \label{ell_garding3_idx1}
\RE Q_1(u) \geq C_1 \sum_{j=1}^J  \| \psi_j \, u \|_{k,2,\Omega}^2 \ .
\end{equation}
However,
\begin{align}
\sum_{j=1}^J  \| \psi_j \, u \|_{k,2,\Omega}^2 &\geq
\sum_{j=1}^J  \sum_{|\VEC{\alpha}| = k}
\ps{\diff^{\VEC{\alpha}} \left( \psi_j \, u\right)}
{\diff^{\VEC{\alpha}} \left( \psi_j \, u\right)}_2 \nonumber \\
&= \sum_{j=1}^J  \sum_{|\VEC{\alpha}| = k}
\ps{\psi_j \,\diff^{\VEC{\alpha}} u}{\psi_j \, \diff^{\VEC{\alpha}} u}_2
+ \sum_{j=1}^J  \sum_{|\VEC{\alpha}|= k}
\ps{\diff^{\VEC{\alpha}}\left( \psi_j \, u\right)}
{\diff^{\VEC{\alpha}}\left( \psi_j \, u\right)
- \psi_j \diff^{\VEC{\alpha}} u}_2 \nonumber \\
&\qquad + \sum_{j=1}^J  \sum_{|\VEC{\alpha}|= k}
\ps{\diff^{\VEC{\alpha}}\left( \psi_j \, u\right)
- \psi_j \diff^{\VEC{\alpha}} u}{\psi_j \diff^{\VEC{\alpha}} u}_2
\ , \label{ell_garding_idx4}
\end{align}
where
\[
\sum_{j=1}^J  \sum_{|\VEC{\alpha}| = k}
\ps{\psi_j \,\diff^{\VEC{\alpha}} u}{\psi_j \, \diff^{\VEC{\alpha}} u}_2
= \sum_{|\VEC{\alpha}| = k}
\ps{\diff^{\VEC{\alpha}} u}{ \sum_{j=1}^J |\psi_j|^2 \,\diff^{\VEC{\alpha}} u}_2
= \sum_{|\VEC{\alpha}| = k} \ps{\diff^{\VEC{\alpha}} u}{\diff^{\VEC{\alpha}} u}_2
\geq C_2 \|u\|_{k,2,\Omega}^2
\]
for $\displaystyle u \in H^{k,2}_0(\Omega)$ and
some constant $C_2>0$ provided by the Poincaré's ineguality,
Theorem~\ref{sob_pt_carre}, and
\begin{align*}
& \left| \sum_{j=1}^J \sum_{|\VEC{\alpha}|= k}
\ps{\diff^{\VEC{\alpha}}\left( \psi_j \, u\right)}
{\diff^{\VEC{\alpha}}\left( \psi_j \, u\right) - \psi_j \diff^{\VEC{\alpha}} u}_2
+ \sum_{j=1}^J  \sum_{|\VEC{\alpha}|= k}
\ps{\diff^{\VEC{\alpha}}\left( \psi_j \, u\right)
- \psi_j \diff^{\VEC{\alpha}} u}{\psi_j \diff^{\VEC{\alpha}} u}_2 \right| \\
&\qquad \leq \sum_{\substack{|\VEC{\alpha}|,|\VEC{\beta}|\leq k\\
|\VEC{\alpha}|+|\VEC{\beta}|<2k}}
\int_\Omega \left| A_{\VEC{\alpha},\VEC{\beta}}(\VEC{x})\right|
\left| \diff^{\VEC{\alpha}} u(\VEC{x}) \right|
\left| \diff^{\VEC{\beta}} u(\VEC{x}) \right| \dx{\VEC{x}} \\
&\qquad \leq \sum_{\substack{|\VEC{\alpha}|,|\VEC{\beta}|\leq k\\
|\VEC{\alpha}|+|\VEC{\beta}|<2k}} \sup_{\VEC{x} \in \overline{\Omega}}
\left| A_{\VEC{\alpha},\VEC{\beta}}(\VEC{x})\right|
\int_\Omega \left| \diff^{\VEC{\alpha}} u(\VEC{x}) \right|
\left| \diff^{\VEC{\beta}} u(\VEC{x}) \right| \dx{\VEC{x}} \\
&\qquad \leq \sum_{\substack{|\VEC{\alpha}|,|\VEC{\beta}|\leq k\\
|\VEC{\alpha}|+|\VEC{\beta}|<2k}} \sup_{\VEC{x} \in \overline{\Omega}}
\left| A_{\VEC{\alpha},\VEC{\beta}}(\VEC{x})\right|
\left( \int_\Omega \left| \diff^{\VEC{\alpha}} u(\VEC{x}) \right|^2
\dx{\VEC{x}}\right)^{1/2}
\left( \int_\Omega \left| \diff^{\VEC{\beta}} u(\VEC{x}) \right|^2 \dx{\VEC{x}}
\right)^{1/2} \\
&\qquad \leq \sum_{\substack{|\VEC{\alpha}|,|\VEC{\beta}|\leq k\\
|\VEC{\alpha}|+|\VEC{\beta}|<2k}} \sup_{\VEC{x} \in \overline{\Omega}}
\left| A_{\VEC{\alpha},\VEC{\beta}}(\VEC{x})\right|
\|u\|_{k,2,\Omega} \|u\|_{k-1,2,\Omega}
\leq C_3 \|u\|_{k,2,\Omega} \|u\|_{k-1,2,\Omega} 
\end{align*}
for $\displaystyle u \in H^{k,2}_0(\Omega)$, where
\[
C_3 = \sum_{\substack{|\VEC{\alpha}|,|\VEC{\beta}|\leq
k\\|\VEC{\alpha}|+|\VEC{\beta}|<2k}} \sup_{\VEC{x} \in \overline{\Omega}}
\left| A_{\VEC{\alpha},\VEC{\beta}}(\VEC{x})\right| \geq 0 \  .
\]
Note that $C_3 < \infty$ because the $A_{\VEC{\alpha},\VEC{\beta}}$ are
polynomial expressions in $\displaystyle \diff^{\VEC{\gamma}} \psi_j$ for
$1\leq j \leq J$ and multi-indices $\VEC{\gamma}$ such that
$|\VEC{\gamma}|\leq k$.  Thus the $A_{\VEC{\alpha},\VEC{\beta}}$ are
continuous functions on $\overline{\Omega}$.  Hence, for
all multi-indices $\VEC{\alpha}$ and $\VEC{\beta}$, the
maximum of $|A_{\VEC{\alpha},\VEC{\beta}}|$
on $\overline{\Omega}$ exists.  We have shown that
\[
\RE Q_1(u) \geq C_1C_2 \|u\|_{k,2}^2
- C_1C_3 \|u\|_{k,2,\Omega} \|u\|_{k-1,2,\Omega} 
\]
for $\displaystyle u \in H^{k,2}_0(\Omega)$.

A reasoning similar to the one used to study the last two series in
(\ref{ell_garding_idx4}) shows that there exists a finite constant
$C_4 \geq 0$ such that
\[
\RE Q_2(u) \leq C_4 \|u\|_{k,2,\Omega} \|u\|_{k-1,2,\Omega} 
\]
for $\displaystyle u \in H^{k,2}_0(\Omega)$.

Hence
\begin{equation} \label{ell_garding_idx2}
\RE B(u,u) \geq C_1C_2 \|u\|_{k,2,\Omega}^2 - \left( C_1C_3 + C_4\right)
\|u\|_{k,2,\Omega} \|u\|_{k-1,2,\Omega}
\end{equation}
for $\displaystyle u \in H^{k,2}_0(\Omega)$.

If we substitute $a= \sqrt{C_1C_2} \|u\|_{k,2,\Omega}$ and
$\displaystyle b = \frac{C_1C_3 + C_4}{\sqrt{C_1C_2}}\|u\|_{k-1,2,\Omega}$ in
$\displaystyle ab \leq \frac{1}{2}(a^2+b^2)$, we get
\[
\left( C_1C_3 + C_4\right) \|u\|_{k,2,\Omega} \|u\|_{k-1,2,\Omega}
\leq \frac{C_1C_2}{2} \|u\|_{k,2,\Omega}^2 +
\frac{(C_1C_3 + C_4)^2}{2C_1C_2}\|u\|_{k-1,2,\Omega}^2
\]
for $\displaystyle u \in H^{k,2}_0(\Omega)$.

The relation (\ref{ell_garding_idx2}) becomes
\begin{equation} \label{ell_garding_idx3}
\RE B(u,u) \geq \frac{C_1C_2}{2} \|u\|_{k,2,\Omega}^2
 - \frac{(C_1C_3 + C_4)^2}{2C_1C_2}\|u\|_{k-1,2,\Omega}^2
\end{equation}
for $\displaystyle u \in H^{k,2}_0(\Omega)$.

If $\underline{u}:\RR^n\rightarrow \RR$ is defined by
\[
\underline{u}(\VEC{x}) =
\begin{cases}
u(\VEC{x}) & \quad \text{if} \ \VEC{x} \in \Omega \\
0 & \quad \text{if} \ \VEC{x} \in \RR^n \setminus \Omega
\end{cases}
\]
as in Proposition~\ref{sob_expand_WKP}, then
$\|u\|_{s,2,\Omega} = \|\underline{u}\|_{s,2,\RR^n}$ for all
$s \leq k$.  Hence, it
follows from Lemma~\ref{ell_garding4} with $r=0$, $s=k-1$, $t=k$
and $\displaystyle \epsilon = \frac{(C_1C_2)^2}{2(C_1C_3 + C_4)^2}$ that
there exists $C_5$ such that
\[
\frac{(C_1C_3 + C_4)^2}{2C_1C_2}\|u\|_{k-1,2,\Omega}^2 \leq 
\frac{C_1C_2}{4} \|u\|_{k,2,\Omega}^2 + \frac{(C_1C_3 + C_4)^2 C_5}{2C_1C_2}
\|u\|_{2,\Omega}^2
\]
for $\displaystyle u \in H^{k,2}_0(\Omega)$.
Using this relation, we finally get from (\ref{ell_garding_idx3}) that
\[
\RE B(u,u) \geq \frac{C_1C_2}{4} \|u\|_{k,2,\Omega}^2
 - \frac{(C_1C_3 + C_4)^2 C_5}{2C_1C_2}\|u\|_{2,\Omega}^2
\]
for $\displaystyle u \in H^{k,2}_0(\Omega)$.
This is Definition~\ref{ell_weakcoerc} with $C=C_1C_2/4$ and
$\lambda = (C_1C_3 + C_4)^2 C_5/(2C_1C_2)$.
\end{proof}

\begin{theorem}[G\r{a}rding's inequality]  \label{ell_garding1}
Let $\Omega$ be a bounded open subset of $\displaystyle \RR^n$ with a
boundary of class $\displaystyle C^2$.  Suppose that
\[
L(\VEC{x},\diff)u = \sum_{|\VEC{\alpha}|,|\VEC{\beta}|\leq k}
(-1)^{\VEC{\alpha}} \diff^{\VEC{\alpha}}
\left( b_{\VEC{\alpha},\VEC{\beta}}(\VEC{x}) \diff^{\VEC{\beta}} u\right) \  ,
\]
is strongly elliptic on $\overline{\Omega}$.  Moreover, suppose that
$\displaystyle b_{\VEC{\alpha},\VEC{\beta}} \in C(\overline{\Omega})$ for
$|\VEC{\alpha}|=|\VEC{\beta}|=k$ and
$\displaystyle b_{\VEC{\alpha},\VEC{\beta}} \in L^\infty(\Omega)$ for
$|\VEC{\alpha}|+|\VEC{\beta}|<2k$ with $|\VEC{\alpha}|,|\VEC{\beta}|\leq k$.
Then
$\displaystyle B : H^{k,2}_0(\Omega) \times H^{k,2}_0(\Omega)\rightarrow \CC$
defined in (\ref{ell_PDE3}) is weakly coercive.
\index{G\r{a}rding's Inequality}
\end{theorem}

\begin{proof}
Let
\[
B(v,u) = \sum_{|\VEC{\alpha}|,|\VEC{\beta}|\leq k}
\ps{\diff^{\VEC{\alpha}} v}{b_{\VEC{\alpha},\VEC{\beta}}(\VEC{x})
\diff^{\VEC{\beta}} u}_2  = B_1(v,u) + B_2(v,u)
\]
for $\displaystyle u, v \in H^{k,2}_0(\Omega)$, where
\[
B_1(v,u) = \sum_{|\VEC{\alpha}|,|\VEC{\beta}|= k}
\ps{\diff^{\VEC{\alpha}} v}{b_{\VEC{\alpha},\VEC{\beta}}(\VEC{x})
\diff^{\VEC{\beta}} u}_2
\quad \text{and} \quad
B_2(v,u) = \sum_{\substack{|\VEC{\alpha}|,|\VEC{\beta}|\leq k\\
|\VEC{\alpha}|+|\VEC{\beta}|<2k}}
\ps{\diff^{\VEC{\alpha}} v}{b_{\VEC{\alpha},\VEC{\beta}}(\VEC{x})
\diff^{\VEC{\beta}} u}_2 \ .
\]
From Lemma~\ref{ell_garding3}, there exist $C_1>0$ and $\lambda_1 >0$
such that
\begin{equation} \label{ell_garding_idx6}
\RE B_1(u,u) \geq C_1\|u\|_{k,2,\Omega}^2 - \lambda_1 \|u\|_{2,\Omega}^2
\end{equation}
for $\displaystyle u \in H^{k,2}_0(\Omega)$.

Moreover, using Schwarz inequality, we get
\begin{align}
\left| B_2(u,u) \right|
&= \bigg| \sum_{\substack{|\VEC{\alpha}|,|\VEC{\beta}|\leq k\\
|\VEC{\alpha}|+|\VEC{\beta}|<2k}}
\ps{\diff^{\VEC{\alpha}} u}{b_{\VEC{\alpha},\VEC{\beta}}(\VEC{x})
\diff^{\VEC{\beta}} u}_2 \bigg|
\leq \sum_{\substack{|\VEC{\alpha}|,|\VEC{\beta}|\leq k\\
|\VEC{\alpha}|+|\VEC{\beta}|<2k}}
\int_{\Omega} \left| b_{\VEC{\alpha},\VEC{\beta}}(\VEC{x})\right|\,
\left| \diff^{\VEC{\beta}} u(\VEC{x}) \right| \,
\left| \diff^{\VEC{\alpha}} u(\VEC{x}) \right| \dx{\VEC{x}} \nonumber \\
&\leq \sum_{\substack{|\VEC{\alpha}|,|\VEC{\beta}|\leq k\\
|\VEC{\alpha}|+|\VEC{\beta}|<2k}}
\| b_{\VEC{\alpha},\VEC{\beta}} \|_{\infty}
\int_{\Omega} \left| \diff^{\VEC{\beta}} u(\VEC{x}) \right| \,
\left| \diff^{\VEC{\alpha}} u(\VEC{x}) \right| \dx{\VEC{x}} \nonumber \\
&\leq \sum_{\substack{|\VEC{\alpha}|,|\VEC{\beta}|\leq k\\
|\VEC{\alpha}|+|\VEC{\beta}|<2k}}
\| b_{\VEC{\alpha},\VEC{\beta}} \|_{\infty}
\left( \int_\Omega \left| \diff^{\VEC{\alpha}} u(\VEC{x}) \right|^2
\dx{\VEC{x}}\right)^{1/2}
\left( \int_\Omega \left| \diff^{\VEC{\beta}} u(\VEC{x}) \right|^2 \dx{\VEC{x}}
\right)^{1/2} \nonumber \\
&\leq \sum_{\substack{|\VEC{\alpha}|,|\VEC{\beta}|\leq k\\
|\VEC{\alpha}|+|\VEC{\beta}|<2k}}
\| b_{\VEC{\alpha},\VEC{\beta}} \|_{\infty}
\|u\|_{k,2,\Omega} \|u\|_{k-1,2,\Omega} \leq C_2 \|u\|_{k,2,\Omega}
\|u\|_{k-1,2,\Omega} \label{ell_garding_idx8}
\end{align}
for $\displaystyle u \in H^{k,2}_0(\Omega)$, where
\[
C_2 = \sum_{\substack{|\VEC{\alpha}|,|\VEC{\beta}|\leq k\\
|\VEC{\alpha}|+|\VEC{\beta}|<2k}}
\| b_{\VEC{\alpha},\VEC{\beta}} \|_{\infty} \geq 0 \ .
\]
Note that the essential supremum
$\displaystyle \| b_{\VEC{\alpha},\VEC{\beta}} \|_{\infty}$ is finite
for all multi-indices $\VEC{\alpha}$ and $\VEC{\beta}$ because
$b_{\VEC{\alpha},\VEC{\beta}} \in L^\infty(\Omega)$ for
$|\VEC{\alpha}|,|\VEC{\beta}|\leq k$
and $|\VEC{\alpha}|+|\VEC{\beta}|<2k$.

If we substitute $a= \sqrt{C_1} \|u\|_{k,2,\Omega}$ and
$\displaystyle b = \frac{C_2}{\sqrt{C_1}}\|u\|_{k-1,2,\Omega}$ in
$\displaystyle ab \leq \frac{1}{2}(a^2+b^2)$, we get
\[
C_2\|u\|_{k,2,\Omega} \|u\|_{k-1,2,\Omega}
\leq \frac{C_1}{2} \|u\|_{k,2,\Omega}^2 + \frac{C_2^2}{2C_1}\|u\|_{k-1,2,\Omega}^2
\]
for $\displaystyle u \in H^{k,2}_0(\Omega)$.  It follows from
(\ref{ell_garding_idx8}) that
\begin{equation} \label{ell_garding_idx5}
\left|B_2(u,u)\right|
\leq \frac{C_1}{2} \|u\|_{k,2,\Omega}^2 + \frac{C_2^2}{2C_1}\|u\|_{k-1,2,\Omega}^2
\end{equation}
for $\displaystyle u \in H^{k,2}_0(\Omega)$.

If $\underline{u}:\RR^n\rightarrow \RR$ is defined by
\[
\underline{u}(\VEC{x}) =
\begin{cases}
u(\VEC{x}) & \quad \text{if} \ \VEC{x} \in \Omega \\
0 & \quad \text{if} \ \VEC{x} \in \RR^n \setminus \Omega
\end{cases}
\]
as in Proposition~\ref{sob_expand_WKP}, then
$\|u\|_{s,2,\Omega} = \|\underline{u}\|_{s,2,\RR^n}$ for $s\leq k$
Hence, it follows from Lemma~\ref{ell_garding4} with $r=0$, $s=k-1$, $t=k$
and $\displaystyle \epsilon = \frac{C_1^2}{2C_2^2}$ that
there exists $C_3$ such that
\[
\frac{C_2^2}{2C_1}\|u\|_{k-1,2,\Omega}^2 \leq 
\frac{C_1}{4} \|u\|_{k,2,\Omega}^2 + \frac{C_2^2 C_3}{2C_1}
\|u\|_{2,\Omega}^2
\]
for $\displaystyle u \in H^{k,2}_0(\Omega)$.  We then get from
(\ref{ell_garding_idx5}) that
\begin{equation} \label{ell_garding_idx7}
\left|B_2(u,u)\right| \leq \frac{3C_1}{4} \|u\|_{k,2,\Omega}^2
 + \frac{C_2^2C_3}{2C_1}\|u\|_{2,\Omega}^2
\end{equation}
for $\displaystyle u \in H^{k,2}_0(\Omega)$.

We conclude from (\ref{ell_garding_idx6}) and (\ref{ell_garding_idx7}) that
\begin{align*}
\RE B(u,u) &= \RE B_1(u,u) + \RE B_2(u,u) \geq
C_1\|u\|_{k,2,\Omega}^2 - \lambda_1 \|u\|_{2,\Omega}^2 - 
\frac{3C_1}{4} \|u\|_{k,2,\Omega}^2 - \frac{C_2^2C_3}{2C_1}\|u\|_{2,\Omega}^2 \\
&\geq \frac{C_1}{4} \|u\|_{k,2,\Omega}^2 - \left(\lambda_1 +
\frac{C_2^2C_3}{2C_1}\right) \|u\|_{2,\Omega}^2
\end{align*}
for $\displaystyle u \in H^{k,2}_0(\Omega)$.
This is Definition~\ref{ell_weakcoerc} with $C=C_1/4$ and
$\lambda = \lambda_1 + (C_2^2 C_3)/(2C_1C_2)$.
\end{proof}

It is interesting to note that there is a converse to
G\r{a}rding's inequality.

\begin{theorem}
Let $\Omega$ be a bounded open subset of $\displaystyle \RR^n$.  consider
\[
L(\VEC{x},\diff)u = \sum_{|\VEC{\alpha}|,|\VEC{\beta}|\leq k}
(-1)^{\VEC{\alpha}} \diff^{\VEC{\alpha}}
\left( b_{\VEC{\alpha},\VEC{\beta}}(\VEC{x}) \diff^{\VEC{\beta}} u\right) \ ,
\]
where $\displaystyle b_{\VEC{\alpha},\VEC{\beta}} \in C(\overline{\Omega})$ for
$|\VEC{\alpha}|=|\VEC{\beta}|=k$ and
$\displaystyle b_{\VEC{\alpha},\VEC{\beta}} \in L^\infty(\Omega)$ for
$|\VEC{\alpha}|+|\VEC{\beta}|<2k$ with $|\VEC{\alpha}|,|\VEC{\beta}|\leq k$.
If $\displaystyle B : H^{k,2}_0(\Omega) \times H^{k,2}_0(\Omega)
\rightarrow \CC$ defined in (\ref{ell_PDE3}) is weakly coercive, then
$L(\VEC{x},\diff)$ is strongly elliptic on $\Omega$.
\end{theorem}

A proof of this result can be found in \cite{FoPDE} for instance.

\section{Existence and Uniqueness of Solutions}

We begin this section by presenting some results concerning the
existence of classical solutions.

Consider the following partial differential equation on a bounded
open set $\Omega$.
\begin{equation} \label{ellEUclassic}
\begin{split}
L(\VEC{x}, \diff)u(\VEC{x})
&= \sum_{i,j=1}^n a_{i,j}(\VEC{x}) \diff_{x_i,x_j} u(\VEC{x})
+ \sum_{i=1}^n a_i(\VEC{x}) \diff_{x_i} u(\VEC{x}) \\
& \qquad \quad + a_0(\VEC{x})u(\VEC{x}) = f(\VEC{x})
\quad , \quad \VEC{x} \in \Omega \\
u(\VEC{x}) &= g(\VEC{x}) \quad , \quad \VEC{x} \in \partial \Omega
\end{split}
\end{equation}
where $\displaystyle b_{i,j} \in C^{0,\alpha}(\overline{\Omega})$ for
all $1\leq i, j \leq n$ and
$\displaystyle b_i \in C^{0,\alpha}(\overline{\Omega})$ for all
$0 \leq i \leq n$ and some $\alpha \geq 0$.  If $L(\VEC{x},\diff)$ is
strongly elliptic in $\Omega$, $a_0(\VEC{x}) \leq 0$ for all
$\VEC{x} \in \Omega$, $g \in C(\partial \Omega)$ and
$\displaystyle f \in C^{0,\alpha}(\overline{\Omega})$,
then there exists a unique solution
$\displaystyle u \in C^{2,\alpha}(\overline{\Omega})$
to (\ref{ellEUclassic}).  The uniqueness of the solution comes from
the relation
\[
  \|u\|_{\infty,\overline{\Omega}} \leq \|g\|_{\infty,\partial \Omega}
  + Q \|f\|_{\infty,\overline{\Omega}} \ ,
\]
where the constant $Q$ depends on $\Omega$, the constant $C$ in the
definition of strongly elliptic in Definition~\ref{strong_ell_form},
and the maximum of $a_{i,j}$, $a_i$ on $\overline{\Omega}$.
An accessible proof of these results can be found in \cite{Smo}.

The next results in this section are more general and are not limited to
classical solutions.

\begin{theorem} \label{ell_exist_th1}
Let $\Omega$ be an open subset of $\displaystyle \RR^n$ and let $X$ be a closed
subspace of $\displaystyle H^{k,2}(\Omega)$ such that
$\displaystyle H^{k,2}_0(\Omega) \subset X$.
Given $\displaystyle f \in L^2(\Omega)$, let
\begin{equation} \label{ell_exist_1}
B(v,u) = \sum_{|\VEC{\alpha}|,|\VEC{\beta}|\leq k}
\ps{\diff^{\VEC{\alpha}} v}{b_{\VEC{\alpha},\VEC{\beta}}(\VEC{x})
\diff^{\VEC{\beta}} u}_2 = \ps{v}{f}_2 \quad , \quad u,v \in X \ ,
\end{equation}
be a variational formulation of the problem
\begin{equation} \label{ell_exist_2}
L(\VEC{x},\diff)u = \sum_{|\VEC{\alpha}|,|\VEC{\beta}|\leq k}
(-1)^{\VEC{\alpha}} \diff^{\VEC{\alpha}}
\left( b_{\VEC{\alpha},\VEC{\beta}}(\VEC{x}) \diff^{\VEC{\beta}} u\right) = f \  ,
\end{equation}
where $\displaystyle b_{\VEC{\alpha},\VEC{\beta}} \in L^\infty(\Omega)$ for all
$\VEC{\alpha}$ and $\VEC{\beta}$.
If $B$ is coercive on $X$, then there exists a one-to-one, bounded
linear mapping $F:L^2(\Omega) \rightarrow X$ such that $F(f)$ is a
weak solution of (\ref{ell_exist_1}); namely,
$\displaystyle B(v,F(f)) = \ps{v}{f}_2$ for all $\displaystyle v \in X$.
\end{theorem}

\begin{proof}
Given $\displaystyle g \in L^2(\Omega)$, we have that
$\tilde{g}:X \to \RR$ defined by
$\tilde{g}(v) = \ps{v}{g}_2$ for all $v \in X$ is a bounded linear
functional.  To prove this statement, we use Schwarz inequality to get
$\displaystyle |\tilde{g}(v)| =|\ps{v}{g}_2|
\leq \|g\|_2 \, \|v\|_2 \leq \|g\|_2 \|v\|_{k,2}$
for all $\displaystyle v \in H^{k,2}(\Omega)$.  Thus 
$\|\tilde{g}|| \leq \|g\|_2$.  In particular, this shows that
$\displaystyle T_1:L^2(\Omega) \to X^\ast$ defined by
$T_1(g) = \tilde{g}$ is a continuous linear mapping of norm at most $1$. 

The application $T_1$ is one-to-one onto its image.  Suppose that
$\displaystyle f_1, f_2 \in L^2(\Omega)$ satisfy
$\ps{v}{f_1}_2 = \ps{v}{f_2}_2$ for all $v\in X$, then 
$\ps{v}{f_1}_2 = \ps{v}{f_2}_2$ for all $v\in \DD(\Omega)$ because
$\displaystyle X\supset H^{k,2}_0(\Omega)$.  Thus $f_1=f_2$ almost
everywhere on $\Omega$ because $\DD(\Omega)$ is dense in
$L^2(\Omega)$ \footnote{Note that we also have that $T_1$ is an
isometry into $X$.  Take $v \in X$ such that $\ps{v}{f}_2>0$. then
$\displaystyle T_1\left(\big(\|f\|_2 / \ps{v}{f}_2\big) v\right) = \|f\|_2$.}.

Since $X$ is an Hilbert space with respect to the norm
$\|\cdot\|_{k,2}$ and $B:X\times X \rightarrow \CC$ is a bounded
coercive sequilinear form according to Proposition~\ref{ell_Bblf}, then we may
use Lax-Milgram Theorem, Corollary~\ref{fu_an_LaxMilgTh}
and Remark~\ref{fu_an_LaxMilgTh2}, to conclude that there exists a
one-to-one, bounded linear mapping $T_2:X \rightarrow X^\ast$ such that
$\displaystyle B(v,T_2(\tilde{f})) = \tilde{f}(v)$ for all $v \in X$.

We have that $F = T_2 \circ T_1$.
\end{proof}

Let $X$ be a closed subspace of $\displaystyle H^{k,2}(\Omega)$ containing
$\displaystyle H^{k,2}_0(\Omega)$ and let $f$ be an element of
$\displaystyle L^2(\Omega)$ as in
the previous theorem.  The formal adjoint operator associated to
\[
L(\VEC{x},\diff)u = \sum_{|\VEC{\alpha}|,|\VEC{\beta}|\leq k}
(-1)^{\VEC{\alpha}} \diff^{\VEC{\alpha}}
\left( b_{\VEC{\alpha},\VEC{\beta}}(\VEC{x}) \diff^{\VEC{\beta}} u\right)
\]
is
\[
L^\ast(\VEC{x},\diff)v = \sum_{|\VEC{\alpha}|,|\VEC{\beta}|\leq k}
(-1)^{\VEC{\beta}} \diff^{\VEC{\beta}}
\left( \overline{b_{\VEC{\alpha},\VEC{\beta}}(\VEC{x})}
\diff^{\VEC{\alpha}} v\right) \  .
\]
A variational formulation of $L(\VEC{x},\diff)u = f$ is to find
$u\in X$ such that
\begin{equation} \label{ell_var_adj1}
B(v,u) = \ps{v}{f}_2
\end{equation}
for all $v \in X$, where $B$ is defined in the previous theorem.  Similarly, A
variational formulation of $L^\ast(\VEC{x},\diff)v = f$ is to find
$v\in X$ such that
\begin{equation} \label{ell_var_adj2}
B^\ast(u,v) = \sum_{|\VEC{\alpha}|,|\VEC{\beta}|\leq k} 
\ps{\diff^{\VEC{\beta}} u}{\overline{b_{\VEC{\alpha},\VEC{\beta}}(\VEC{x})}
\diff^{\VEC{\alpha}} v}_2 = \ps{u}{f}_2
\end{equation}
for all $u \in X$.  Since
\begin{equation} \label{ell_Bs_B}
B^\ast(u,v) = \overline{B(v,u)}
\end{equation}
for all $u,v \in X$,
$\displaystyle B^\ast$ is a bounded sequilinear form. The variational problem
(\ref{ell_var_adj2}) is the {\bfseries adjoint (variational) problem}
\index{Adjoint Variational Problem}
to the variational problem (\ref{ell_var_adj1}).  The variational
formulation (\ref{ell_var_adj1}) is {\bfseries self adjoint}
\index{Adjoint Variational Problem!Self Adjoint} if
$\displaystyle B^\ast=B$.  This implies that
$\displaystyle L(\VEC{x},\diff)=L^\ast(\VEC{x},\diff)$.  To
prove this, consider $\displaystyle B^\ast(v,u) = B(v,u)$ for
$u,v \in \DD(\Omega) \subset X$ and use integration by parts.

Moreover, from (\ref{ell_Bs_B}), we get that $B$ is coercive
if and only if $\displaystyle B^\ast$ is coercive.  We can even take the same
constant $C$ in the definition of a coercive sequilinear form applied
to $B$ and $\displaystyle B^\ast$.  Similarly, $B$ is weakly coercive
if and only if $\displaystyle B^\ast$ is weakly coercive.  Again, we
can take the same constants $C$ and $\lambda$ in the definition of a
weakly coercive sequilinear form applied to $B$ and $\displaystyle B^\ast$.

If $B$ is coercive, it follows from Theorem~\ref{ell_exist_th1}
that there exists a bounded linear mapping
$\displaystyle F_B:L^2(\Omega)\to X$ such that
$\displaystyle  B(v,F_B(f)) = \ps{v}{f}_2$ for all $v \in X$.
Similarly, since $\displaystyle B^\ast$ is also coercive, there exists a bounded
linear mapping $\displaystyle F_{B^\ast}:L^2(\Omega)\to X$ such that
$\displaystyle  B^\ast(u,F_{B^\ast}(f)) = \ps{u}{f}_2$ for all
$u \in X$.  Let $\displaystyle i:X\to L^2(\Omega) = H^{0,2}(\Omega)$ be
the inclusion mapping.  We have from Rellich's theorem,
Theorem~\ref{sob_rell_th}, that $i$ is a compact operator.
It follows from Proposition~\ref{fu_an_comp_cont} that
$\displaystyle K_1=i \circ F_B:L^2(\Omega) \to L^2(\Omega)$
and $\displaystyle K_2=i \circ F_{B^\ast}:L^2(\Omega) \to L^2(\Omega)$ are
compact operators.  The following sequence of equalities proves that
$\displaystyle K_1^\ast = (i\circ F_B)^\ast = i \circ F_{B^\ast} = K_2$.
\begin{align*}
\ps{(i\circ F_B)(g)}{f}_2 &= \ps{F_B(g)}{f}_2 =
B^\ast(F_B(g), F_{B^\ast}(f))
= \overline{B(F_{B^\ast}(f), F_B(g))} =
\overline{\ps{F_{B^\ast}(f)}{g}_2} \\
&= \ps{g}{F_{B^\ast}(f)}_2 = \ps{g}{(i\circ F_{B^\ast})(f)}_2
\end{align*}
for all $\displaystyle f,g \in L^2(\Omega)$.

\begin{theorem} \label{ell_exist_th2}
Let $\Omega$ be an open subset of $\displaystyle \RR^n$ and let $X$ be a closed
subspace of $\displaystyle H^{k,2}(\Omega)$ containing
$\displaystyle H^{k,2}_0(\Omega)$.
Consider the variational formulation (\ref{ell_exist_1}) of
(\ref{ell_exist_2}), where $B$ is weakly coercive.  Let
\[
V = \left\{ u \in X : B(v,u)=0 \ \text{for all}\ v \in X \right\}
\]
and
\[
W = \left\{ u \in X : B(u,v)=0 \ \text{for all}\ v \in X \right\}\ .
\]
Then,
\begin{enumerate}
\item $\dim V = \dim W < \infty$.
\item Given $\displaystyle f \in L^2(\Omega)$, there exists a solution of
(\ref{ell_exist_1}) if and only if $\ps{v}{f}_2=0$ for all $v\in W$.
The solution is unique up to the addition of an element of $V$.
\item Given $\displaystyle f \in L^2(\Omega)$, if $V=W=\{0\}$, then
there exists a unique solution of (\ref{ell_exist_1}).
\end{enumerate}
\end{theorem}

\begin{proof}
Since $B$ is weakly coercive, there exist $C>0$ and $\lambda>0$ such
that $\displaystyle \RE B(u,u) \geq C \|u\|_{k,2,\Omega}^2
- \lambda \|u\|_{2,\Omega}^2$
for all $u \in X$.  The sequilinear form
\begin{equation} \label{ell_Bhat_sql}
\tilde{B}(v,u) = B(v,u) + \lambda\ps{v}{u}_2 \quad , \quad u,v \in X \ ,
\end{equation}
is therefore coercive on $X$ because
$\displaystyle \RE \tilde{B}(u,u) \geq C \|u\|_{k,2,\Omega}^2$
for all $u \in X$.  According to the discussion before the statement
of the theorem with $B$ replaced by $\tilde{B}$, there exists a
compact operator
$\displaystyle K_1:L^2(\Omega)\rightarrow L^2(\Omega)$ such that
$\IMG(K_1) \subset X$ and
\[
\tilde{B}(v,K_1(f)) = \ps{v}{f}_2
\]
for all $v\in X$ and $\displaystyle f\in L^2(\Omega)$.

Given $\displaystyle f\in L^2(\Omega)$, we have that $u\in X$ satisfies
\begin{align}
B(v,u) = \ps{v}{f}_2 \quad , \quad v\in X\quad &\Leftrightarrow \quad
\tilde{B}(v,u) = \ps{v}{f}_2 + \lambda\ps{v}{u}_2 \quad , \quad v \in X
\nonumber \\
&\Leftrightarrow \quad \tilde{B}(v,u) = \ps{v}{f+\lambda u}_2 \quad ,
\quad v \in X \nonumber \\
&\Leftrightarrow \quad u = K_1(f+\lambda u) \nonumber \\
&\Leftrightarrow \quad \lambda^{-1} u - K_1(u) = \lambda^{-1} K_1(f) \ .
\label{ell_BK1u_K1f}
\end{align}

With $f=0$, we get $u\in V$ if and only if $\displaystyle u\in L^2(\Omega)$
satisfies $\lambda^{-1} u = K_1(u)$ because
$u = \lambda K_1(u) \in \IMG(K_1) \subset X$.
Hence
\[
V = \left\{ u \in L^2(\Omega) : K_1(u) = \lambda^{-1} u \right\} 
= \KE(K_1-\lambda^{-1}\Id) \  .
\]
A similar reasoning with $\displaystyle B^\ast$ instead of $B$ shows that there
exists a compact operator
$\displaystyle K_2:L^2(\Omega) \rightarrow L^2(\Omega)$
with $\IMG(K_2) \subset X$ such that
\[
W = \left\{ u \in L^2(\Omega) : K_2(u) = \lambda^{-1} u \right\}
= \KE(K_2-\lambda^{-1}\Id) \  .
\]
From the discussion before the statement of this theorem, we have that
$\displaystyle K_2= K_1^\ast$.  Item (1) of the theorem is a consequence of
item (3) of Theorem~\ref{fu_an_comp_oper}.

We have shown in (\ref{ell_BK1u_K1f}) that $u\in X$ satisfies
(\ref{ell_exist_1}) if and
only if $\displaystyle K_1(f) = u - \lambda K_1(u)
= -\lambda ( K_1(u) - \lambda^{-1} u)$.
Therefore, (\ref{ell_exist_1}) has a solution if and only if
$\displaystyle K_1(f) \in \IMG(K_1 - \lambda^{-1}\Id)$.  Since
$\displaystyle \IMG(K_1 -\lambda^{-1}\Id)$ is closed in
$\displaystyle L^2(\Omega)$ according to
item (2) of Theorem~\ref{fu_an_comp_oper}, we get
from Theorems~\ref{fu_an_RIOrth} that
$\displaystyle \IMG(K_1-\lambda^{-1}\Id) = \KE(K_1^\ast-\lambda^{-1}\Id)^\perp
= \KE(K_2-\lambda^{-1}\Id)^\perp = W^\perp$ in
$\displaystyle L^2(\Omega)$.  Thus, $u\in X$ satisfies
(\ref{ell_exist_1}) if and only if $K_1(f) \in W^\perp$.  Finally,
since
\[
\ps{K_1(f)}{w}_2 = \ps{f}{K_1^\ast(w)}_2 = \lambda^{-1} \ps{f}{w}_2
\]
for all $w \in W$ and $\displaystyle f \in L^2(\Omega)$,
we have $\displaystyle K_1(f) \in W^\perp$ if and only if
$\displaystyle f\in W^\perp$.  This is
item (2) of the theorem.  Item (3) is a consequence of (2).
\end{proof}

\begin{rmk}
In our previous discussion, we have assumed that the boundary
conditions of the Dirichlet problem (\ref{ell_PDE1}) were homogeneous.
More precisely, we have assumed that
$\displaystyle \pdydxn{u}{\nu}{j} = 0$ on
$\partial \Omega$ for $0 \leq j < k$.  However, more
general boundary conditions can be considered.   \label{ellRmkNhom}

The variational approach that we have developed before can also be
used to solve Dirichlet problems with non-homogeneous boundary
conditions.  Consider
\begin{equation} \label{ell_nh_dirichlet}
\begin{split}
L(\VEC{x}, \diff)u(\VEC{x})
&= \sum_{|\VEC{\alpha}|,|\VEC{\beta}|\leq k} (-1)^{|\VEC{\alpha}|}
\diff^{\VEC{\alpha}} \left( b_{\VEC{\alpha},\VEC{\beta}}(\VEC{x})
\diff^{\VEC{\beta}} u(\VEC{x})\right) = f(\VEC{x}) \quad
, \quad \VEC{x} \in \Omega \ ,\\
\pdydxn{u}{\nu}{j}(\VEC{x}) &= g_j(\VEC{x}) \quad , \quad
\VEC{x} \in \partial \Omega \ \text{and}\ 0 \leq j < k \ ,
\end{split}
\end{equation}
where the functions $g_j:\partial \Omega \rightarrow \CC$ are nice
functions.  For instance, $g_j \in C^\infty(\partial \Omega)$ for all $j$.

First, find $\displaystyle g\in H^{2k,2}(\RR^n)$ such that
$\displaystyle \pdydxn{g}{\nu}{j} = g_j$ on $\partial \Omega$ for
$0 \leq j < k$.  According to Theorem~\ref{ell_normal_BCs}, such a
function $g$ can be found if
$\displaystyle g_j \in C^\infty(\partial \Omega)$ for
all $j$.  Then, $u$ is a solution of (\ref{ell_nh_dirichlet}) if and
only if $v=u-g$ is a solution
\begin{align*}
L(\VEC{x}, \diff)v(\VEC{x}) &= f(\VEC{x}) - L(\VEC{x},\diff) g(\VEC{x}) \quad
, \quad \VEC{x} \in \Omega \ ,\\
\pdydxn{v}{\nu}{j}(\VEC{x}) &= 0 \quad , \quad
\VEC{x} \in \partial \Omega \ \text{and}\ 0 \leq j < k\ .
\end{align*}
\end{rmk}

The existence and uniqueness of the solution for the following
examples is a direct consequence of Theorem~\ref{ell_exist_th1}.
However, for the sake of illustrating with concrete examples the proof
of Theorem~\ref{ell_exist_th1} and to present the variational approach
to study elliptic partial differential equations, we deduce the
variational problem associated to the elliptic partial differential
equation given in each example and use the Lax-Milgram Theorem to
prove the existence and uniqueness of solutions.

\begin{egg}
Let $\Omega$ be a bounded subset of $\displaystyle \RR^n$.  Find the
weak solution of the Dirichlet Problem
\begin{equation} \label{ell_example2_eq1}
\begin{split}
-\Delta u + u &= f \quad \text{in} \quad \Omega \\
u&= 0 \quad \text{on} \quad \partial \Omega
\end{split}
\end{equation}
where $\displaystyle f \in L^2(\Omega)$.       \label{ell_example2}

\subI{Variational Problem}
The weak solution is the solution $\displaystyle u\in H^{1,2}_0(\Omega)$ of the
equation
\begin{equation} \label{ell_example2_eq2}
\int_\Omega \left( \graD u \cdot \graD v) + u\, v - f\, v \right)
\dx{\VEC{x}} = 0
\end{equation}
for all $\displaystyle v \in H^{1,2}_0(\Omega)$.

We prove that a classical solution is a weak solution.  If
$\displaystyle u\in C^2(\overline{\Omega})$ is a classical solution of
(\ref{ell_example2_eq1}), integrating (\ref{ell_example2_eq1})
multiplied by $\displaystyle v \in C^1_c(\Omega)$ over $\Omega$ gives
\[
\int_\Omega \left( \graD u \cdot \graD v
+ u\, v - f\, v \right) \dx{\VEC{x}} = 0
\]
for all $\displaystyle v \in C^1_c(\Omega)$.
The integral was computed using Green's identity (\ref{laplace_green1}).
Since $\displaystyle C^1_c(\Omega)$ is dense in
$\displaystyle H^{1,2}_0(\Omega)$, we have that
(\ref{ell_example2_eq2}) is true \footnote{See Example~\ref{ell_example1}
for the details of a similar case.}.
Moreover, it follows from Proposition~\ref{sob_w0_trad2} that
$\displaystyle u\in H^{1,2}_0(\Omega)$ because
$\displaystyle u\in C^2(\overline{\Omega})$ and $u(\VEC{x})=0$ for all
$\VEC{x} \in \partial \Omega$.

\subI{Existence and uniqueness of the weak solution}
\begin{align*}
B:H^{1,2}_0(\Omega) \times H^{1,2}_0(\Omega) &\rightarrow \RR \\
(u,v) &\mapsto \int_\Omega \left( \graD u \cdot \graD v
+ u\, v \right) \dx{\VEC{x}}
= \int_\Omega \left( \graD u ,u \right)\cdot \left( \graD v, v \right)
\dx{\VEC{x}}
\end{align*}
is a coercive, symmetric bounded bilinear mapping.  That $B$ is bounded is
proved as in Example~\ref{ell_example1} with two applications of
Schwarz inequality, one in $\displaystyle \RR^n$ and the other in
$\displaystyle L^2(\Omega)$;
namely,
\begin{align*}
|B(u,v)| &\leq \int_\Omega
\left| \left(\graD u, u\right) \cdot \left(\graD v, v\right) \right|
\dx{\VEC{x}}
\leq \int_\Omega \left\|\left(\graD u, u \right)\right\|_2\,
\left\| \left(\graD v, v\right)\right\|_2 \dx{\VEC{x}} \\
&\leq \left( \int_\Omega \left\|
\left(\graD u, u\right)\right\|_2^2 \dx{\VEC{x}} \right)^{1/2} \,
\left( \int_\Omega \left\|\left(\graD v, v \right)\right\|_2^2
\dx{\VEC{x}}\right)^{1/2} \\
&= \left( \int_\Omega \left( \sum_{j=1}^n
\left|\pdydx{u}{x_j}\right|^2+ \left|u\right|^2
\right) \dx{\VEC{x}} \right)^{1/2} \,
\left( \int_\Omega \left( \sum_{j=1}^n
\left|\pdydx{v}{x_j}\right|^2+ \left|v\right|^2
\right) \dx{\VEC{x}} \right)^{1/2}
= \| u\|_{1,2} \, \|v\|_{1,2}
\end{align*}
for all $\displaystyle u,v \in H^{1,2}_0(\Omega)$.
Since $\displaystyle B(u,u) = \|u\|_{1,2}^2$ for
$\displaystyle u\in H^{1,2}_0(\Omega)$, it is
obvious that $B$ is coercive.  It is also obvious that $B$ is
symmetric.  The function $\displaystyle f \in L^2(\Omega)$ yields a bounded
functional on $\displaystyle H^{1,2}_0(\Omega)$ defined by
\[
\tilde{f}(u) = \ps{u}{f}_2 = \int_\Omega f\,u \dx{\VEC{x}}
\]
for all $\displaystyle u \in H^{1,2}_0(\Omega)$.
We may thus apply Lax-Milgram Theorem to (\ref{ell_example2_eq2}) to
conclude that there exists a unique weak solution 
$\displaystyle u \in H^{1,2}_0(\Omega)$ given by
\[
\frac{1}{2} \int_\Omega \left( \graD u \cdot \graD u 
+u\, u \right) \dx{\VEC{x}} -\int_\Omega f\, u \dx{\VEC{x}}
= \min_{v\in H^{1,2}_0(\Omega)} \left\{ \frac{1}{2}
\int_\Omega \left( \graD v \cdot \graD v +v\, v \right) \dx{\VEC{x}}
-\int_\Omega f\, v \dx{\VEC{x}} \right\} \ .
\]

\subI{Regularity of the solution}
As a consequence of Theorem~\ref{ell_regular}, $\displaystyle u\in H^2(\Omega)$.

\subI{Classical solution}
If the weak solution $u$ is in $\displaystyle C^2(\overline{\Omega})$
and assuming that $\Omega$ has a boundary of class
$\displaystyle C^1$, it follows from
Proposition~\ref{sob_w0_trad2} that $u(\VEC{x})=0$ for
all $\VEC{x}\in \partial \Omega$ because
$\displaystyle u \in H^{1,2}_0(\Omega)\cap C^2(\overline{\Omega})$.
Using Green's identity (\ref{laplace_green1}), we can show
that (\ref{ell_example2_eq2}) becomes
\[
\int_\Omega \left( \Delta u + u - f\right)
v \dx{\VEC{x}} = 0
\]
for all $\displaystyle v \in C^1_c(\Omega)$.
Since $\displaystyle C^1_c(\Omega)$ is dense in $C_c(\Omega)$, we
get that $\displaystyle \Delta u + u - f = 0$ almost
everywhere.  If $f \in C(\Omega)$, we have
$\displaystyle -\Delta u + u -f = 0$ everywhere on $\Omega$ since
$\displaystyle u\in C^2(\Omega)$.  Thus $u$ is the classical solution of
(\ref{ell_example2_eq1}).
\end{egg}

\begin{egg}
Let $\Omega$ be a bounded subset of $\displaystyle \RR^n$ with a
boundary of class $\displaystyle C^2$.
Find the weak solution of the Neumann Problem          \label{eggNewmanBdr}
\begin{equation} \label{ell_example4_eq1}
\begin{split}
-\Delta u + u &= f \quad \text{in} \quad \Omega \\
\pdydx{u}{\nu} &= 0 \quad \text{on} \quad \partial \Omega
\end{split}
\end{equation}
where $\displaystyle f \in L^2(\Omega)$.

\subI{Variational Problem}
The weak solution is the solution $\displaystyle u\in H^{1,2}(\Omega)$ of
the equation
\begin{equation} \label{ell_example4_eq2}
\int_\Omega \left( \graD u \cdot \graD v
+ u\, v - f\, v \right) \dx{\VEC{x}} = 0
\end{equation}
for all $\displaystyle v \in H^{1,2}(\Omega)$.

We prove that a classical solution is a weak solution.  Suppose that
$\displaystyle u \in C^2(\overline{\Omega})$ is a classical solution of
(\ref{ell_example4_eq1}).  We then have that
\begin{equation} \label{ell_example4_eq3}
\int_\Omega \left( -\Delta u + u - f\right)
v \dx{\VEC{x}} = 0
\end{equation}
for all $\displaystyle v \in C^1(\overline{\Omega})$.
We get from Green's identity (\ref{laplace_green1}) that
\[
\int_{\partial \Omega} v\,\pdydx{u}{\nu} \dss{S}{x}
= \int_\Omega \left( v \Delta u + \graD u \cdot \graD v \right)
\dx{\VEC{x}}
\]
for all $\displaystyle v \in C^1(\overline{\Omega})$.
Since $\displaystyle \pdydx{u}{\nu} = 0$ on $\partial \Omega$, we
get
\[
\int_\Omega  v \Delta u \dx{\VEC{x}} =
- \int_\Omega \graD u \cdot \graD v \dx{\VEC{x}}
\]
for all $\displaystyle v \in C^1(\overline{\Omega})$.
Hence, (\ref{ell_example4_eq3}) becomes
\[
\int_\Omega \left( \graD u \cdot \graD v) + u\, v
 - f\, v\right) \dx{\VEC{x}} = 0
\]
for all $\displaystyle v \in C^1(\overline{\Omega})$.
Since $\displaystyle C^1(\overline{\Omega}) \supset \DD(\Omega)$ is
dense in $\displaystyle H^{1,2}(\Omega)$, we
have that (\ref{ell_example4_eq2}) is true.

\subI{Existence and uniqueness of the weak solution}
The bilinear form
\begin{align*}
B:H^{1,2}(\Omega)\times H^{1,2}(\Omega) &\rightarrow \RR \\
(u,v) &\mapsto \int_\Omega \left( \graD u \cdot \graD v + u\, v\right)
\dx{\VEC{x}}
\end{align*}
is coercive, symmetric and bounded.  The proof is identical to the
proof in Example~\ref{ell_example2} that the bilinear form
$\displaystyle B:H^{1,2}_0(\Omega)\times H^{1,2}_0(\Omega) \rightarrow \RR$
is coercive, symmetric and bounded.  Moreover, we also have that the
functional on $\displaystyle H^{1,2}(\Omega)$ defined by 
\[
\tilde{f}(u) = \ps{u}{f}_2 = \int_\Omega f\, u\dx{\VEC{x}}
\]
for all $\displaystyle u \in H^{1,2}(\Omega)$ is a bounded mapping.
We may use Lax-Milgram Theorem to
(\ref{ell_example4_eq2}) to conclude that there exists a unique weak
solution $\displaystyle u \in H^{1,2}(\Omega)$ given by
\[
\frac{1}{2} \int_\Omega \left( \graD u \cdot \graD u 
+u\, u \right) \dx{\VEC{x}} -\int_\Omega f\, u \dx{\VEC{x}}
= \min_{v\in H^{1,2}(\Omega)} \left\{ \frac{1}{2}
\int_\Omega \left( \graD v \cdot \graD v 
+v\, v \right) \dx{\VEC{x}} -\int_\Omega f\, v \dx{\VEC{x}} \right\} \ .
\]

\subI{Regularity of the solution}
As a consequence of Theorem~\ref{ell_regular}, $\displaystyle u\in H^2(\Omega)$.

\subI{Classical solution}
If the weak solution $u$ is in $\displaystyle C^2(\overline{\Omega})$,
we get from (\ref{ell_example4_eq2}) and Green's identity (\ref{laplace_green1})
that
\begin{equation} \label{ell_GW_classic}
\int_{\partial \Omega} v\,\pdydx{u}{\nu} \dss{S}{x}
+ \int_\Omega \left( - v\, \Delta u + u\, v - f\, v \right) \dx{\VEC{x}} = 0
\end{equation}
for all $\displaystyle v \in C^1(\overline{\Omega})$.

Since $\displaystyle C^1_c(\Omega) \subset C^1(\overline{\Omega})$, we get from
(\ref{ell_GW_classic}) that
\[
\int_\Omega \left( - v\, \Delta u + u\, v - f\, v \right) \dx{\VEC{x}} = 0
\]
for all $\displaystyle v \in C^1_c(\Omega)$ 
because $v=0$ on $\partial \Omega$ (we assume that $v$ is extended to
$\overline{\Omega}$).  Since $\displaystyle C^1_c(\Omega)
\supset \DD(\Omega)$ is dense in
$\displaystyle L^2(\Omega)$, we get that
$\displaystyle -\Delta u + u - f = 0$ almost
everywhere.  If $f \in C(\Omega)$, we have
$\displaystyle -\Delta u + u -f = 0$ everywhere on $\Omega$ since
$\displaystyle u\in C^2(\Omega)$.  Back to (\ref{ell_GW_classic}),
this implies that
\[
\int_{\partial \Omega} v\,\pdydx{u}{\nu} \dss{S}{x} = 0
\]
for all $\displaystyle v \in C^1(\overline{\Omega})$.
Since $\displaystyle C^1(\overline{\Omega})\big|_{\partial \Omega}
\supset \DD(\partial \Omega)$ is dense in $L^2(\partial \Omega)$,
we get $\displaystyle \pdydx{u}{\nu} =0$ almost everywhere on
$\partial \Omega$ (using the measure along $\partial \Omega$).  Since
$\displaystyle u\in C^2(\overline{\Omega})$, this implies that
$\displaystyle \pdydx{u}{\nu}(\VEC{x}) =0$ for all
$\VEC{x} \in \partial \Omega$.
\end{egg}

For the next example, we will use a slightly different procedure than
the procedure introduce in Remark~\ref{ellRmkNhom}.  Both procedures
are equivalent.

\begin{egg}
Let $\Omega$ be a bounded subset of $\displaystyle \RR^n$.  Find the
weak solution of the Dirichlet Problem
\begin{equation} \label{ell_example3_eq1}
\begin{split}
-\Delta u + u &= f \quad \text{in} \quad \Omega \\
u&= g \quad \text{on} \quad \partial \Omega
\end{split}
\end{equation}
where $\displaystyle f \in L^2(\Omega)$ and $g\in C^\infty(\partial \Omega)$.  

\subI{Variational Problem}
Choose
$\displaystyle \tilde{g} \in H^{1,2}(\Omega) \cap C(\overline{\Omega})$
such that
$\tilde{g}=g$ on $\partial \Omega$.  Let
\[
K = \left\{ v \in H^{1,2}(\Omega) : v - \tilde{g} \in H^{1,2}_0(\Omega)
\right\} \ .
\]
$K$ is a closed and convex subset of $\displaystyle H^{1,2}(\Omega)$.
That $K$ be independent of the choice of $\tilde{g}$ is a consequence of
Proposition~\ref{sob_w0_trad2}.  $K$ depends only of $g$.

The weak solution is the solution
$\displaystyle u\in K \subset H^{1,2}(\Omega)$ of the equation
\begin{equation} \label{ell_example3_eq2}
B(u,v) \equiv \int_\Omega \left( \graD u \cdot \graD v
+ u\, v - f\, v \right) \dx{\VEC{x}} = 0
\end{equation}
for all $\displaystyle v \in H^{1,2}(\Omega)$.

We prove that a classical solution is a weak solution.  As usual,
if $\displaystyle u\in C^2(\overline{\Omega})$ is a classical solution of
(\ref{ell_example3_eq1}), then integrating (\ref{ell_example3_eq1})
multiplied by $\displaystyle v \in C^1_c(\Omega)$ over $\Omega$ gives
\[
\int_\Omega \left( \graD u \cdot \graD v
+ u\, v - f\, v \right) \dx{\VEC{x}} = 0
\]
for all $\displaystyle v \in C^1_c(\Omega)$.
As in the previous example, Green's identity (\ref{laplace_green1})
may be used to compute the integral.  Since
$\displaystyle C^1_c(\Omega) \supset \DD(\Omega)$ is dense in
$\displaystyle H^{1,2}_0(\Omega)$, we have that
(\ref{ell_example3_eq2}) is true.  Moreover, it follows from 
Proposition~\ref{sob_w0_trad2} that
$\displaystyle u- \tilde{g} \in H^{1,2}_0(\Omega)$ because
$\displaystyle u - \tilde{g} \in H^{1,2}(\Omega)\cap C(\overline{\Omega})$ and
$u - \tilde{g}=0$ on $\partial \Omega$.  Thus $u\in K$.

\subI{Existence and uniqueness of the weak solution}
As we have proved in previous examples, the bilinear form
\begin{align*}
B:H^{1,2}(\Omega)\times H^{1,2}(\Omega) &\rightarrow \RR \\
(u,v) &\mapsto \int_\Omega \left( \graD u \cdot \graD v
+ u\, v \right) \dx{\VEC{x}}
\end{align*}
is coercive, symmetric and bounded.  Moreover,
\[
\tilde{f}(u) = \ps{u}{f}_2 = \int_\Omega f(x)u(x)\dx{\VEC{x}}
\]
for all $\displaystyle u \in H^{1,2}(\Omega)$
is a bounded linear functional.

Before using Stampacchia Theorem, Theorem~\ref{fu_an_stamp}, we need
to reformat the variational equation.  We have that $u\in K$ is a weak
solution if and only if
\begin{equation} \label{ell_example3_eq3}
B(v-u,u) = \int_\Omega \left( \graD (v-u) \cdot \graD u
+ u \,(v-u) \right) \dx{\VEC{x}}
\geq \int_\Omega f\,(v-u) \dx{\VEC{x}} = \ps{v-u}{f}_2
\end{equation}
for all $v \in K$.
If $u$ is a weak solution, we get from (\ref{ell_example3_eq2})
that (\ref{ell_example3_eq3}) is true because
$\displaystyle u-v\in H^1_0(\Omega)$ for all $v\in K$.  In fact we have equality
in (\ref{ell_example3_eq3}).  Conversely, if $u\in K$ satisfies
(\ref{ell_example3_eq3}), we substitute $v=u\pm w$ with
$\displaystyle w\in H^1_0(\Omega)$ in (\ref{ell_example3_eq3}) to get
respectively
\begin{align*}
B(w,u) = \int_\Omega \left( \graD w \cdot \graD u
+ w\,u \right) \dx{\VEC{x}} &\geq \int_\Omega f\,w \dx{\VEC{x}} = \ps{w}{f}_2
\intertext{and}
-B(w,u) = - \int_\Omega \left( \graD w \cdot \graD u + w\,u \right) \dx{\VEC{x}}
&\geq -\int_\Omega f\,w \dx{\VEC{x}} = - \ps{w}{f}_2
\end{align*}
for all $\displaystyle w \in H^1_0(\Omega)$.  Thus
(\ref{ell_example3_eq2}) is true.

We may now apply Stampacchia Theorem with $\displaystyle H=H^{1,2}(\Omega)$ to
conclude that there exists a unique $u\in K$ satisfying
(\ref{ell_example3_eq3}).  Therefore, there exists a unique weak
solution.  This solution $u$ is given by
\[
\frac{1}{2} \int_\Omega \left( \graD u \cdot \graD u 
+u\, u \right) \dx{\VEC{x}} -\int_\Omega f \, u \dx{\VEC{x}}
= \min_{v\in K} \left\{ \frac{1}{2}
\int_\Omega \left( \graD v \cdot \graD v 
+v\, v \right) \dx{\VEC{x}} -\int_\Omega f\,v \dx{\VEC{x}} \right\} \ .
\]

\subI{Regularity of the solution}
As a consequence of Theorem~\ref{ell_regular}, $u\in H^2(\Omega)$.

\subI{Classical solution}
If $f\in C(\Omega)$ and the weak solution $u$ is in
$\displaystyle C^2(\overline{\Omega})$, we can proved as we did many
times before that $\displaystyle -\Delta u + u -f = 0$ everywhere on $\Omega$.
Moreover, it follows from Proposition~\ref{sob_w0_trad2} that
$u - \tilde{g}=0$ on $\partial \Omega$
because $\displaystyle u \in K \cap C^2(\overline{\Omega})$ implies
that
$\displaystyle u- \tilde{g} \in H^{1,2}_0(\Omega)\cap C(\overline{\Omega})$.
\end{egg}

If the sequilinear form in (\ref{ell_exist_1}) is weakly coercive and
symmetric, we get the following additional properties.

\begin{theorem} \label{ell_exist_th3}
Let $\Omega$ be an open subset of $\displaystyle \RR^n$ and let $X$ be a closed
subspace of $\displaystyle H^{k,2}(\Omega)$ containing
$\displaystyle H^{k,2}_0(\Omega)$.
Consider the variational formulation (\ref{ell_exist_1}) of
(\ref{ell_exist_2}), where $B$ is coercive or weakly coercive, and symmetric.
\begin{enumerate}
\item There exists an orthonormal basis $\displaystyle \{u_j\}_{j=1}^\infty$ of
$\displaystyle L^2(\Omega)$ consisting of eigenvectors of the sequilinear form
$B$ defined in (\ref{ell_exist_1}).  We define
$\displaystyle u_j \in L^2(\Omega)$ as
an eigenvector associated to the eigenvalue $\lambda_j$ if $u_j\neq 0$
and $B(v,u_j)= \ps{v}{\lambda_j u_j}_2$ for all $v\in X$.
\item If $\lambda_j$ is an eigenvalue of $B$, then $\lambda_j$ is real
and $\lambda_j > -\lambda$, where $\lambda$ is the constant in the
definition of a weakly coercive sequilinear form if $B$ is weakly
coercive, or $\lambda =0$ if $B$ is coercive.
\item The eigenvalues $\displaystyle \{\lambda_j\}_{j=1}^\infty$ can
be arranged in an increasing sequence converging to $+\infty$.
\item If the coefficients $b_{\VEC{\alpha},\VEC{\beta}}$ of
$L(\VEC{x},\diff)$ are all of class $\displaystyle C^\infty(\Omega)$,
then the eigenvectors $u_j$ are all in $\displaystyle C^\infty(\Omega)$.
\end{enumerate}
\end{theorem}

\begin{proof}
The following proof is valid for $B$ weakly coercive (case $\lambda>0$
below) or $B$ coercive (case $\lambda =0$).  The restriction
$\lambda>0$ in the proof of Theorem~\ref{ell_exist_th2} is not needed
here because we do not have to play with $\displaystyle \lambda^{-1}$.

Let $\tilde{B}$ be the sequilinear form defined in (\ref{ell_Bhat_sql})
of Theorem~\ref{ell_exist_th2}.  Let $K_1$ and $K_2$ be the compact operators
defined in the paragraph preceding Theorem~\ref{ell_exist_th2}) with
$B$ replaced by $\tilde{B}$ and $\displaystyle B^\ast$ replaced by
$\displaystyle \tilde{B}^\ast$ respectively.

Since $B$ is symmetric, $\tilde{B}$ is symmetric.  It follows from its
definition that $K_1$ is a symmetric operator because
$F_{\tilde{B}} = F_{\tilde{B}^\ast}$ in the definition of $K_1$ and $K_2$
given in the paragraph preceding Theorem~\ref{ell_exist_th2}.

$K_1$ is also one-to-one since it is the composition $i \circ F_{\tilde{B}}$,
where $\displaystyle F_{\tilde{B}}:L^2(\Omega)\rightarrow X$ given by
Theorem~\ref{ell_exist_th1} is one-to-one and
$\displaystyle i:X\rightarrow L^2(\Omega)$ is the inclusion.

If follows from Theorem~\ref{fu_an_HStheorem} (and the two paragraphs
preceding this theorem) that there exists an orthonormal basis of
eigenvectors $\displaystyle \{u_j\}_{j=1}^\infty \subset L^2(\Omega)$
of $K_1$ such that the eigenvalue $\beta_j$ associated to $u_j$ is real and the
sequence $\displaystyle \{|\beta_j|\}_{j=1}^\infty$ is a decreasing sequence
converging to $0$ if there are infinitely many distinct eigenvalues.
The $\beta_j$ may have the same value for some values of $j$ if
$\dim \KE( K_1 - \beta_j\Id) > 1$; namely, if the eigenspace
associated to $\beta_j$ has dimension larger than $1$.

The eigenvalues are positive numbers because
\begin{align*}
\beta_j \|u_j\|_2^2 &= \ps{\beta_j u_j}{u_j}_2
= \ps{K_1(u_j)}{u_j}_2 = \tilde{B}(F_{\tilde{B}}(u_j),F_{\tilde{B}}(u_j)) \\
&= \RE \tilde{B}(F_{\tilde{B}}(u_j),F_{\tilde{B}}(u_j))
\geq C \|F_{\tilde{B}}(u_j)\|_{k,2}^2 > 0 \ ,
\end{align*}
where we have used $\tilde{B}(v,v) \in \RR$ for all $v \in X$
because $B$ is a symmetric operator, and where $C$ is the constant from the
definition of a coercive or weakly coercive operator satisfied by $B$.

From
\[
\tilde{B}(v, \beta_j u_j) = \tilde{B}(v, (i \circ K_1)(u_j))
= \tilde{B}(v, F_{\tilde{B}}(u_j)) = \ps{v}{u_j}_2
\]
for all $v \in X$, we get
$\displaystyle \tilde{B}(v, u_j) = \ps{v}{\beta_j^{-1}u_j}_2$ for all
$v \in X$.

Let $\lambda_j = \beta_j^{-1}-\lambda$ for $j>0$.  Then 
\begin{align*}
B(v,u_j) &= \tilde{B}(v,u_j) - \lambda \ps{v}{u_j}_2
=\ps{v}{\beta_j^{-1}u_j}_2 - \lambda \ps{v}{u_j}_2 \\
&= \ps{v}{(\beta_j^{-1}-\lambda)u_j}_2 = \ps{v}{\lambda_j u_j}_2
\end{align*}
for all $v \in X$.  Thus, $\lambda_j > -\lambda$ for all $j$ and
$\displaystyle \lim_{j\rightarrow \infty} \lambda_j = \infty$.
Note that there are infinitely many distinct eigenvalues because the
eigenspaces $\KE(K_1-\beta_j \Id)$ for $j>0$ are
finite dimensional and $\displaystyle L^2(\Omega)$ is infinite dimensional.

Finally, to prove item (4), we note that an
eigenvector $u_j$ is a solution in the sense of distribution of the
partial differential equation
$\displaystyle \left(L(\VEC{x},\diff) - \lambda_j \Id \right)u_j = 0$
because
\[
\ps{v}{L(\VEC{x},\diff)u_j - \lambda_j u_j}_2
= B(v,u_j) - \ps{v}{\lambda_j u_j}_2 = 0
\]
for all
$\displaystyle v \in \DD(\Omega) \subset H^{k,2}_0(\Omega) \subset$.  Since
$L(\VEC{x},\diff) - \lambda_j \Id$ is an elliptic operator with
coefficients of class $\displaystyle C^\infty$ on $\Omega$, it follows from
Corollary~\ref{ell_Hak_Cinfty} that we will see in the next section
that $\displaystyle u_j \in C^\infty(\Omega)$.
\end{proof}

We now apply our results to the original Dirichlet and Neumann
problems.

\begin{theorem}  \label{ell_exist_th4}
Let $\Omega$ be a bounded open subset of $\displaystyle \RR^n$ with a
boundary of class $\displaystyle C^2$.  If the operator
$L(\VEC{x},\diff)$ in the Dirichlet problem
(\ref{ell_PDE1}) is strongly elliptic on $\overline{\Omega}$, and
$\displaystyle L(\VEC{x},\diff)=L^\ast(\VEC{x},\diff)$, we get the
following results.
\begin{enumerate}
\item There exists an orthonormal basis $\displaystyle \{u_j\}_{j=1}^\infty$ of
$\displaystyle L^2(\Omega)$ consisting of eigenvectors of $L(\VEC{x},\diff)$.
\item $\displaystyle u_j \in C^\infty(\overline{\Omega})$ for all $j$.
\item If $\nu(\VEC{x})$ denotes the outward unit normal to
$\partial \Omega$ at $\VEC{x} \in \partial \Omega$, then
$\displaystyle \pdydxn{u_j}{\nu}{i}=0$ on $\partial \Omega$
for $0 \leq i < k$.
\item The eigenvalues are real, isolated and can be arranged in a
increasing sequence in absolute value that converges to $+\infty$.
\end{enumerate}
Moreover, if the sequilinear form $B$ of the variational formulation
(\ref{ell_PDE2}) of the Dirichlet problem (\ref{ell_PDE1}) is
also coercive, we get the following result.
\begin{enumerate}
\setcounter{enumi}{4}
\item The eigenvalues of $L(\VEC{x},\diff)$ are positive.
\end{enumerate}
\end{theorem}

\begin{proof}
Let $\displaystyle B(v,u) = \ps{v}{f}_2$ for all
$\displaystyle u, v \in H^{k,2}_0(\Omega)$
be a variational formulation of the Dirichlet problem (\ref{ell_PDE1})
with $\displaystyle f\in L^2(\Omega)$, where $B$ is defined in
(\ref{ell_exist_1}) with $\displaystyle X=H^{k,2}_0(\Omega)$.  A
variational formulation of the adjoint equation
$\displaystyle L^\ast(\VEC{x},\diff)v = f$ is
\begin{equation} \label{ell_th4_lbl1}
B^\ast(u,v) = \ps{u}{f}_2 \quad , \quad u,v \in H^{k,2}_0(\Omega) \ ,
\end{equation}
where $\displaystyle B^\ast$ is defined in (\ref{ell_var_adj2}) with
$\displaystyle X=H^{k,2}_0(\Omega)$. Since
$\displaystyle L^\ast(\VEC{x},\diff)=L(\VEC{x},\diff)$,
(\ref{ell_th4_lbl1}) is another variational formulation of the
Dirichlet problem (\ref{ell_PDE1}).  Let
\[
D(u,v) = \frac{1}{2} \left( B(u,v) + B^\ast(u,v) \right)
\quad , \quad u,v \in H^{k,2}_0(\Omega) \ .
\]
Then
\[
D(v,u) = \ps{v}{f}_2 \quad , \quad u,v \in H^{k,2}_0(\Omega) \ ,
\]
is still another variational formulation of the Dirichlet problem
(\ref{ell_PDE1}).  The sequilinear form $D$ is symmetric.  Moreover, 
it follows from G\r{a}rding's inequality,
Theorem~\ref{ell_garding1}, that $D$ is weakly coercive on
$\displaystyle H^{k,2}_0(\Omega)$.

The conclusions of the theorem are direct consequences of the
conclusions of Theorem~\ref{ell_exist_th3} except for items
(2) and (3).  For item (2), we have from
Theorem~\ref{ell_exist_th3} that
$\displaystyle u_j \in C^\infty(\Omega)$ for all $j$.
For all $j$, the continuous extension of $u_j$ and its partial
derivatives to $\overline{\Omega}$ is proved in
Section~\ref{ell_sect_RB} later.  As for item (3), this result
follows from Proposition~\ref{sob_w0_trad2}.
Since $\displaystyle u_j \in H^{k,2}_0(\Omega) \cap C^\infty(\overline{\Omega})$,
Proposition~\ref{sob_w0_trad2} implies that
$\displaystyle \diff^{\VEC{\alpha}} u_j(\VEC{x}) = 0$ for all
$\VEC{x} \in \partial \Omega$ and $|\VEC{\alpha}|<k$. 
\end{proof}

\begin{cor}
Let $\Omega$ be a bounded open subset of $\displaystyle \RR^n$ with a
boundary of class $\displaystyle C^2$.  There exists an
orthonormal basis $\displaystyle \{u_j\}_{j=1}^\infty$ of
$\displaystyle L^2(\Omega)$ consisting
of eigenvectors of $-\Delta$ such that
$\displaystyle u_j \in C^\infty(\overline{\Omega})$ and
$u_j=0$ on $\partial \Omega$ for all $j$.  The eigenvalues are
positive and can be arranged to form an increasing sequence in
absolute value that converges to $+\infty$.
\end{cor}

\begin{proof}
The operator $L(\VEC{x},\diff) \equiv - \Delta$ is strongly elliptic
and $\displaystyle L^\ast(\VEC{x},\diff) = L(\VEC{x},\diff)$.  We may
then apply the previous theorem.  Note that the variational formulation of
$\displaystyle L(\VEC{x},\diff)u = - \Delta u = f \in L^2(\Omega)$ is
\[
B(v,u) = \sum_{j=1}^n \ps{\pdydx{v}{x_j}}{\pdydx{u}{x_j}}_2 = \ps{v}{f}_2 \quad,
\quad u, v \in H^{1,2}_0(\Omega) \ .
\]
The sequilinear form $B$ is symmetric and coercive on
$\displaystyle H^{1,2}_0(\Omega)$.  
\end{proof}

\begin{egg}
The conclusions of this example can all be directly deduced from
Theorem~\ref{ell_exist_th4}, however we deduce the conclusions using
basic results in functional analysis.  These results are those used
to proof Theorem~\ref{ell_exist_th4}.  We hope that this example may
provide a better understanding of Theorem~\ref{ell_exist_th4} and its
proof.

Let $I = ]0,1[$.  Suppose that $\displaystyle p \in C^1(\overline{I})$ and that
there exists $\alpha>0$ such that $p(x) \geq \alpha$ for all
$x \in I$.  Moreover, suppose that $q \in C(\overline{I})$.

Consider the problem
\begin{equation} \label{ell_eigA}
\begin{split}
-(p u')' + q u &= \lambda u \quad \text{on} \quad I \\
u(0) &= u(1) = 0
\end{split}
\end{equation}
We show that there exist a sequence
$\displaystyle \left\{\lambda_j\right\}_{j=1}^\infty$ of real
numbers and a sequence $\displaystyle \left\{u_j\right\}_{j=1}^\infty$
of functions in $\displaystyle L^2(I)$ such that
\begin{enumerate}
\item $u_j$ is a weak solution of (\ref{ell_eigA}) when
$\lambda = \lambda_j$.
\item $\displaystyle \left\{u_j\right\}_{j=1}^\infty$ is a basis of
$\displaystyle L^2(I)$.
\item $\lambda_j \leq \lambda_{j+1}$ for all $j$ and
$\lambda_j \to \infty$ as $j \to \infty$.
\item $\displaystyle u_j \in C^2(\overline{I})$ for all $j$,
and the $u_j$'s are
classical solutions of (\ref{ell_eigA}) for $\lambda = \lambda_j$.
\end{enumerate}

Note that we may have $\lambda_{j_1} = \lambda_{j_2}$ with $j_1 < j_2$
if there are $j_2 -j_1 + 1$ linearly independent weak solutions of
(\ref{ell_eigA}) for $\lambda = \lambda_{j_1} = \lambda_{j_2}$.

The variational problem associated to (\ref{ell_eigA}) is
\begin{equation} \label{ell_eigB}
\int_I \left( p\, u'\, v'+ q\, u\, v \right) \dx{x} =
\lambda \int_I u\, v \dx{x}
\end{equation}
for all $\displaystyle u,v \in H_0^1(I)$.
This is a motivation to consider the variational problem
\begin{equation} \label{ell_eigC}
\int_I \left( p\, u'\, v' + q\, u\, v \right) \dx{x} = \int_I f\, v \dx{x}
\end{equation}
for all $\displaystyle u,v \in H_0^1(I)$.
We have seen in Example~\ref{ell_exampleSL} that for each
$\displaystyle f \in L^2(I)$ there
exists a unique solution $\displaystyle u \in H^2(I) \cap H^1_0(I)$ of this
variational problem.  We define an operator $\displaystyle T:L^2(I) \to L^2(I)$
by $T(f) = u$, where $\displaystyle u \in H^2(I) \cap H^1_0(I)$ is the
weak solution of the variational problem (\ref{ell_eigC}).  In this context,
$1/\lambda_j$ is an eigenvalue of $T$ and $u_j$ is an eigenvector
associated to $1/\lambda_j$ because $u_j$ and $\lambda_j$ satisfy
$T(\lambda_j u_j) = u_j$.

Without lost of generality, we may assume that $q(x)\geq 0$ for all
$x \in I$.  If it is not so, let
$\displaystyle C = \min_{x\in\overline{I}} q(x)$
and replace $q$ by $q + C$.  The values $\lambda_j$ are then all
shifted to $\lambda_j + C$.

We now prove that $T$ is a self-adjoint, compact operator.  We first
prove that $\displaystyle T:L^2(I) \to H_0^1(I)$ is continuous.  Suppose that
$T(f) =u$.  From (\ref{ell_eigC}), we have
\begin{equation} \label{ell_px_evpEq1}
\alpha \|u'\|_2^2 = \int_I \alpha (u')^2 \dx{x}
\leq \int_I p\, (u')^2 \dx{x} \leq \int_I f \, u \dx{x}
\leq \|f\|_2 \|u\|_2
\end{equation}
because $p(x)\geq \alpha$ and $q(x) \geq 0$ for all $x$.
It follows from Poincar\'e's Inequality, Theorem~\ref{sob_pt_carre},
that there exists $C>0$ independent of $u$ such that
\begin{equation} \label{ell_px_evp}
\| u \|_{1,2} \leq C \|u'\|_2 \ .
\end{equation}
We may assume that $C>1$.  Thus,
$\displaystyle \|u\|_2^2 + \|u'\|_2^2 \leq C^2 \|u'\|_2^2$ yields
$\displaystyle \|u \|_2 \leq \sqrt{C^2-1} \|u'\|_2$.
We then have from (\ref{ell_px_evpEq1}) that
$\displaystyle \alpha \|u'\|_2^2 \leq \sqrt{C^2-1} \|u'\|_2 \|f\|_2$
and therefore
$\displaystyle \|u'\|_2 \leq \frac{\sqrt{C^2-1}}{\alpha} \|f\|_2$.
We finally get from (\ref{ell_px_evp}) that
\[
\|T(f)\|_{1,2} = \|u\|_{1,2} \leq \frac{C\sqrt{C^2-1}}{\alpha}
\|f\|_2 \ .
\]
Since the injection of $\displaystyle H^1(I)$ into
$C(\overline{I}) \subset L^2(I)$
is compact according to Corollary~\ref{sob_cor_sob_lem} to the Sobolev Lemma,
we get that $\displaystyle T:L^2(I) \to L^2(I)$ is compact because the
composition of a continuous operator with a compact operator is a
compact operator.

To prove that $T$ is self-adjoint, we need to show that
\begin{equation} \label{ell_eigD}
\int_I T(f) \, g \dx{x} = \int_I f\, T(g)\dx{x}
\end{equation}
for all $\displaystyle f,g \in L^2(I)$.  Let $u=T(f)$ and $v=T(g)$.
We then have that
\[
\int_I f \, v \dx{x}
= \int_I \left( p\, u'\, v' + q\, u\, v \right) \dx{x}
= \int_I g \, u \dx{x} \ .
\]
This is (\ref{ell_eigD}).

Since $T$ is a self-adjoint compact operator, the spectrum of $T$
contains only real eigenvalues, there is at most a countable number of
eigenvalues, and the eigenspace associate to an eigenvalue is of finite
dimension.  Since $\displaystyle L^2(I)$ is of infinite dimension, there is
therefore an infinite number of distinct eigenvalues.  Moreover, we
may arrange the real eigenvalues
$\displaystyle \left\{ \mu_j \right\}_{j=1}^\infty$
such that $|\mu_j| \geq |\mu_{j+1}|$, and we have $|\mu_j|\to 0$ as
$j \to \infty$.

If we take a basis of eigenfunctions for each eigenspace and combine
them, we get a basis of eigenfunctions
$\displaystyle \left\{u_j\right\}_{j=1}^\infty$ for $\displaystyle L^2(I)$
(see Theorem~\ref{fu_an_HStheorem} and the paragraphs before this theorem).
We assume that $\mu_{j} = \mu_{j+1} = \ldots = \mu_{j+k}$ if
the eigenfunctions $u_j$, $u_{j+1}$, \ldots, $\mu_{j+k}$ are $k+1$ linearly
independent eigenfunctions associated to $\mu_j$.

Since $\KE(T) = \left\{ 0 \right\}$, the unique solution of 
(\ref{ell_eigC}) with $f=0$ is $u=0$; namely, $0$ is not an eignevalue.  Thus
$\mu_j \neq 0$ for all $j$.  Moreover, we have
\[
\int_I T(f)\, f \dx{x} = \int_I u \,f\dx{x} =
\int_I \left( p \,(u')^2 + q\, u^2 \right) \dx{x} \geq 0
\]
for all $\displaystyle f \in L^2(I)$.  Thus
\[
\mu_j \int_I u_j^2 \dx{x}
= \int_I T(u_j)\, u_j \dx{x} \geq 0
\]
implies that $\mu_j\geq 0$ for all $j$.  Hence, we have
$\mu_j \geq \mu_{j+1} > 0$ for all $j$ and $\mu_j \to 0$ as
$j \to \infty$.

Except for the last item about the differentiability of the $u_j$'s,
we have proved all the other claims of the example with
$\lambda_j = 1/\mu_j$ for all $j$.

It reminds to prove that $\displaystyle u_j \in C^2(\overline{I})$ for
all $j$, and they are classical solutions of (\ref{ell_eigA}) when
$\lambda = \lambda_j$.

We have shown in Example~\ref{ell_exampleSL} that
$\displaystyle u = T(f) \in C^1(\overline{I})$ if
$\displaystyle f \in L^2(I)$.  For the
variational problem (\ref{ell_eigB}), we then have that
$\displaystyle u_j \in C^1(\overline{I})$ and
thus $\displaystyle f = \lambda_j u_j \in C^1(\overline{I})$.  Again in
Example~\ref{ell_exampleSL}, we have show that if $f \in C(I)$, then
$\displaystyle u \in C^2(\overline{I})$.  Because
$\displaystyle f = \lambda_j u_j \in C^1(\overline{I})$ in the variational
problem (\ref{ell_eigB}), we finally obtain that
$\displaystyle u_j \in C^2(\overline{I})$.

The justification given at the end of Example~\ref{ell_exampleSL}
proves that the weak solution
$\displaystyle u_j \in C^2(\overline{I}) \cap H^1_0(I)$ of (\ref{ell_eigA}) is
in fact a classical solution of (\ref{ell_eigA}) with
$\lambda = \lambda_j$.
\end{egg}

To study the Neumann's problem (\ref{ell_PDE4}), we need the following lemma.

\begin{lemma} \label{ell_stell_coerc}
Suppose that the coefficients in the differential operator
$L(\VEC{x},\diff)$ defined in (\ref{ell_PDE4}) are bounded functions on
$\Omega$ and, moreover, the coefficients $b_{i,j}$ are all real-valued
functions.  Let
$\displaystyle B:H^{1,2}(\Omega)\times H^{1,2}(\Omega) \rightarrow \RR$
be the sequilinear form defined in (\ref{ell_PDE6}) and associated to
(\ref{ell_PDE4}).  If $L(\VEC{x},\diff)$ is strongly elliptic on $\Omega$,
then $B$ is weakly coercive over $\displaystyle H^{1,2}(\Omega)$, and
over any closed subspaces of $\displaystyle H^{1,2}(\Omega)$.
\end{lemma}

\begin{proof}
Let $\displaystyle c_{i,j} = (b_{i,j}+ b_{j,i})/2$.  Then, for
each $\VEC{x} \in \Omega$,
\[
R(\VEC{x},\VEC{\xi}) = \sum_{i,j=1}^n c_{i,j}(\VEC{x}) \xi_i,\xi_j
\]
for all $\displaystyle \VEC{\xi} \in \RR^n$
is a symmetric bilinear form.  Moreover, from the definition
of a strongly elliptic operator, there exists a constant
$C_1$ independent of $\VEC{x} \in \Omega$ such that
\[
R(\VEC{x}, \VEC{\xi}) = Q(\VEC{x}, \VEC{\xi})
\equiv \sum_{i,j=1}^n b_{i,j}(\VEC{x}) \xi_i\,\xi_j \geq C_1 \|\VEC{\xi}\|^2
\]
for all $\displaystyle \VEC{\xi} \in \RR^n$.
Recall that $Q$ is the principal symbol associated to (\ref{ell_PDE4}).
The matrix $C$ with entries $c_{i,j}$ for $1\leq i,j \leq n$ is
therefore a symmetric positive definite \nn matrix.  Hence,
\begin{equation} \label{ell_stell_coerc_eq}
\RE \sum_{i,j=1}^n b_{i,j}(\VEC{x}) \xi_i\,\overline{\xi_j} =
\sum_{i,j=1}^n c_{i,j}(\VEC{x}) \xi_i\,\overline{\xi_j} \geq C_1
\|\VEC{\xi}\|^2
\end{equation}
for all $\displaystyle \VEC{\xi} \in \CC^n$.
If $\VEC{\xi} = \graD u(\VEC{x})$, then
\[
\RE \sum_{i,j=1}^n b_{i,j}(\VEC{x})\, \diff_{x_i}u(\VEC{x})
 \,\diff_{x_j}\overline{u}(\VEC{x}) \geq C_1 \|\graD u(\VEC{x})\|^2
= C_1 \sum_{j=1}^n \left| \diff_{x_j} u(\VEC{x}) \right|^2
\]
for all $\VEC{x} \in \Omega$.  It follows that
\begin{equation} \label{ell_NlemEq1}
\RE \sum_{i,j=1}^n \int_\Omega b_{i,j}(\VEC{x}) \diff_{x_i}u(\VEC{x})
 \,\diff_{x_j}\overline{u}(\VEC{x}) \dx{\VEC{x}} \geq C_1 
\sum_{j=1}^n \int_\Omega \left|\diff_{x_j} u(\VEC{x})\right|^2 \dx{\VEC{x}}
= C_1 \left( \|u\|_{1,2,\Omega}^2 - \|u\|_{2,\Omega}^2 \right) \  .
\end{equation}
Moreover, let $C_2$ be a constant such that $|b_i(\VEC{x})| \leq C_2$ for all
$\VEC{x} \in \Omega$ and $0\leq i \leq n$.  Using Schwarz inequality,
we find that
\begin{equation} \label{ell_NlemEq2}
\left| \ps{u}{b_i \diff_{x_i} u}_2 \right| \leq
\|u\|_{2,\Omega} \,\| b_i \diff_{x_i} u \|_{2,\Omega}
\leq C_2 \|u\|_{2,\Omega} \,\|u\|_{1,2,\Omega}
\end{equation}
for $0 <i \leq n$ and
\begin{equation} \label{ell_NlemEq3}
\left| \ps{u}{b_0 u}_2 \right| \leq 
\|u\|_{2,\Omega} \, \|b_0 u \|_{2,\Omega}
\leq C_2 \|u\|_{2,\Omega} \, \|u\|_{1,2,\Omega} \ .
\end{equation}
Hence, it follows from (\ref{ell_NlemEq1}), (\ref{ell_NlemEq2}) and
(\ref{ell_NlemEq3}) that
\[
\RE B(u,u) \geq C_1 \left( \|u\|_{1,2,\Omega}^2 - \|u\|_{2,\Omega}^2 \right) -
(n+1)C_2 \|u\|_{2,\Omega} \, \|u\|_{1,2,\Omega}
\]
for all $\displaystyle u \in H^{1,2}(\Omega)$.  From
$\displaystyle ab \leq (a^2+b^2)/2$ with
$\displaystyle a = \sqrt{C_1} \|u\|_{1,2,\Omega}$ and
$\displaystyle b = \frac{(n+1)C_2}{\sqrt{C_1}} \|u\|_{2,\Omega}$, we get
\[
(n+1)C_2 \|u\|_{2,\Omega} \, \|u\|_{1,2,\Omega}
\leq \frac{1}{2} \left(
C_1 \|u\|_{1,2,\Omega}^2 + \frac{(n+1)^2C_2^2}{C_1} \, \|u\|_{2,\Omega}^2 \right)
\]
for all $\displaystyle u \in H^{1,2}(\Omega)$.  Therefore,
\begin{align*}
\RE B(u,u) &\geq C_1 \left( \|u\|_{1,2,\Omega}^2 - \|u\|_{2,\Omega}^2 \right) -
\frac{1}{2} \left(
C_1 \|u\|_{1,2,\Omega}^2 + \frac{(n+1)^2C_2^2}{C_1} \,
\|u\|_{2,\Omega}^2 \right) \\
&= \frac{C_1}{2} \|u\|_{1,2,\Omega}^2 -
\left( C_1 + \frac{(n+1)^2C_2^2}{2C_1}\right) \|u\|_{2,\Omega}^2
\geq C \|u\|_{1,2,\Omega}^2 - \lambda \|u\|_{2,\Omega}^2
\end{align*}
for all $\displaystyle u \in H^{1,2}(\Omega)$, where
$\displaystyle C= \frac{C_1}{2}$ and
$\displaystyle \lambda = C_1 + \frac{(n+1)^2C_2^2}{2C_1}$.
This proves that $B$ is weakly coercive on $\displaystyle H^{1,2}(\Omega)$.
\end{proof}

Before stating our existence result for the Neumann problem, we need to
recall that the adjoint of the operator
\[
L(\VEC{x}, \diff)u = -\sum_{i,j=1}^n
\pdfdx{\left( b_{i,j}(\VEC{x}) \pdydx{u}{x_j} \right)}{x_i} + b_0(\VEC{x})u
\]
given in (\ref{ell_PDE4}) is the operator
\[
L^\ast(\VEC{x}, \diff)u = -\sum_{i,j=1}^n
\pdfdx{\left( b_{i,j}(\VEC{x}) \pdydx{u}{x_i} \right)}{x_j} + b_0(\VEC{x})u
\]
where we have assumed that the functions $b_{i,j}$ were real-valued
functions of class $\displaystyle C^\infty(\Omega)$.  If the
matrix $B$ with entries $b_{i,j}$ for $1\leq i,j\leq n$ is symmetric, then
\[
\sum_{i,j=1}^n \pdfdx{\left( b_{i,j}\,\pdydx{u}{x_j}\right)}{x_i}
= \sum_{i,j=1}^n \pdfdx{\left( b_{j,i}\,\pdydx{u}{x_j}\right)}{x_i}
= \sum_{i,j=1}^n \pdfdx{\left( b_{i,j}\,\pdydx{u}{x_i}\right)}{x_j}
\]
for $\displaystyle u \in C^2(\Omega)$, where we have use the symmetry
of $b_{i,j}$ for the first equality and interchanged $i$ and $j$ for
the second equality.  We get that
$\displaystyle L(\VEC{x},\diff) = L^\ast(\VEC{x},\diff)$.

\begin{rmk}
If we expand
\[
L(\VEC{x}, \diff)u = -\sum_{i,j=1}^n
\pdfdx{\left( b_{i,j}(\VEC{x}) \pdydx{u}{x_j} \right)}{x_i} + b_0(\VEC{x})u \ ,
\]
we get
\[
L(\VEC{x}, \diff)u = -\sum_{i,j=1}^n b_{i,j}(\VEC{x})
\pdydxnm{u}{x_i}{x_j}{2}{}{}
- \sum_{i=1}^n \left( \sum_{j=1}^n \pdydx{b_{j,i}}{x_j}(\VEC{x}) \right)
\pdydx{u}{x_i} + b_0(\VEC{x})u \ .
\]
If we compare with
\[
L(\VEC{x}, \diff)u = -\sum_{i,j=1}^n a_{i,j}(\VEC{x})
\pdydxnm{u}{x_i}{x_j}{2}{}{}
+ \sum_{i=1}^n a_i(\VEC{x}) \pdydx{u}{x_i} + a_0(\VEC{x})u \ ,
\]
we get $a_{i,j} = b_{i,j}$, 
$\displaystyle a_i = - \sum_{j=1}^n \pdydx{b_{j,i}}{x_j}$ and
$b_0 = a_0$.  The condition
$\displaystyle a_i = - \sum_{j=1}^n \pdydx{a_{j,i}}{x_j}$ for
$1 \leq i \leq n$ is the condition for $L(\VEC{x}, \diff)u$ in
standard form to be equal to its adjoint
\[
L^\ast(\VEC{x}, \diff)u = -\sum_{i,j=1}^n a_{i,j}(\VEC{x})
\pdydxnm{u}{x_i}{x_j}{2}{}{}
- \sum_{i=1}^n \pdfdx{\left( a_i(\VEC{x}) u\right)}{x_i} +
a_0(\VEC{x})u
\]
if we assume that the matrix with entries $a_{i,j}$ is symmetric.  We
leave it to the reader to verify this statement.
\end{rmk}

\begin{theorem} \label{ell_exist_th5}
Let $\Omega$ be a bounded open subset of $\displaystyle \RR^n$ with a boundary
of class $\displaystyle C^2$.
Suppose that the operator $L(\VEC{x},\diff)$ in the Newmann problem
(\ref{ell_PDE4}) is elliptic on $\overline{\Omega}$, that
$\displaystyle L(\VEC{x},\diff)=L^\ast(\VEC{x},\diff)$, that the
coefficients $b_{i,j}$ of $L(\VEC{x},\diff)$ are real-valued
functions of class $C^\infty(\Omega)$, and that the \nn matrix $B$ with entries
$b_{i,j}$ for $1\leq i,j\leq n$ is symmetric.  We have the following results.
\begin{enumerate}
\item There exists an orthonormal basis $\displaystyle \{u_j\}_{j=1}^\infty$ of
$\displaystyle L^2(\Omega)$ consisting of eigenvectors of $L(\VEC{x},\diff)$.
\item $\displaystyle u_j \in C^\infty(\overline{\Omega})$ for all $j$.
\item If $\nu(\VEC{x})$ denotes the outward unit normal to
$\partial \Omega$ at $\VEC{x} \in \partial \Omega$, then
\[
\sum_{i,j=1}^n b_{i,j}(\VEC{x}) \pdydx{u_k}{x_j}(\VEC{x})\,\nu_i(\VEC{x})= 0
\]
on $\partial \Omega$ for all $k$.
\item The eigenvalues are real, isolated and can be arranged to form an
increasing sequence in absolute value that converges to $+\infty$.
\item If $b_0\geq 0$ on $\Omega$, the eigenvalues are non-negative.
If in addition $b_0>0$ on an open subset of $\Omega$, the eigenvalues
are positive.
\end{enumerate}
\end{theorem}

\begin{proof}
Recall from Remark~\ref{ell_implies_strong_ell} that
$L(\VEC{x},\diff)$ elliptic on a compact set implies that
$L(\VEC{x},\diff)$ is strongly elliptic on that set.
We then have from Lemma~\ref{ell_stell_coerc} that the bilinear form $B$
defined in (\ref{ell_PDE6}) is weakly coercive over
$\displaystyle H^{1,2}(\Omega)$.

Hence, items (1) and (4) comes from
Theorem~\ref{ell_exist_th3}.  Item (2) is a
regularity results and will be proved in Section~\ref{ell_sect_RB}.

Item (3) comes from (\ref{ell_rm_bdr_Neum}),
where $u$ is replaced by $u_k$ and $f$ by $\lambda_k u_k$ with
$\lambda_k$ the eigenvalue associated to $u_k$.

To prove item (5), we refer to
(\ref{ell_stell_coerc_eq}) in the proof of
Lemma~\ref{ell_stell_coerc}.  Since the matrix $B$ is
symmetric, $c_{i,j} = b_{i,j}$ in the proof of
Lemma~\ref{ell_stell_coerc}.  Thus (\ref{ell_stell_coerc_eq}) becomes
\[
\sum_{i,j=1}^n b_{i,j}(\VEC{x}) \xi_i\,\overline{\xi_j} \geq C_1
\|\VEC{\xi}\|^2 
\]
for all $\displaystyle \VEC{\xi} \in \CC^n$ for some constant $C_1$.
As we have proved before the statement of the theorem, we have
\begin{align*}
B(u,u) &= \sum_{i,j=1}^n \int_\Omega b_{i,j}(\VEC{x}) \,\pdydx{u}{x_i}
\pdydx{\overline{u}}{x_j}\dx{\VEC{x}} + \int_\Omega b_0 |u|^2 \dx{\VEC{x}} \\
&\geq C_1 \int_\Omega \sum_{i=1}^n \left| \diff_{x_i} u \right|^2 \dx{\VEC{x}}
+ \int_\Omega b_0 u^2 \dx{\VEC{x}} \geq \int_\Omega b_0 |u|^2 \dx{\VEC{x}}
\end{align*}
for $\displaystyle u \in H^{1,2}(\Omega)$.
Hence, if $\lambda_j$ is an eigenvalue associated to the eigenvector
$u_j$, then $\lambda_j \in \RR$ by item (4) and
\[
\lambda_j \|u_j\|_2^2 = \ps{u_j}{\lambda_j u_j}_2
= \ps{u_j}{F_B(u_j)}_2 = B(u_j,u_j)
\geq \int_\Omega b_0 |u_j|^2 \dx{\VEC{x}} \ .
\]
Since $u_j$ and $b_0$ are at least continuous function on $\Omega$, we
have that $\lambda_j \geq 0$ if $b_0\geq 0$ on $\Omega$, and $\lambda_j>0$
if in addition of $b_0\geq 0$ on $\Omega$ we have that $b_0> 0$ on an
open subset of $\Omega$.
\end{proof}

\subsection{Eigenvalue Problems for the Laplacian Operator
(continued)} \label{subsectEigLaplace}

We can now look back at the eigenvalue problems for the Laplacian
operator that we studied in Section~\ref{sectEigLaplace} and provide a
complete justification for the eigenvalues and eigenfunctions that we
found.

As before, let $\Omega$ be an open and bounded subset of
$\displaystyle \RR^n$ with a sufficiently smooth boundary.
We seek $\lambda$ such that there exists a non-trivial
function $\displaystyle u \in H^1_0(\Omega)$ satisfying
$-\Delta u = \lambda u$ on $\Omega$.

The {\bfseries variational formulation}%
\index{Laplace Equation!Variational Formulation} of the eigenvalue problem
$-\Delta u = \lambda u$ is to find
$\displaystyle u \in H^1_0(\Omega)$ such that
\begin{equation} \label{laplace_vf}
B(u,v) \equiv \int_{\Omega} \nabla u(\VEC{x})\cdot \nabla v(\VEC{x})
\dx{\VEC{x}}
= \lambda \int_{\Omega} u(\VEC{x}) v(\VEC{x}) \dx{\VEC{x}}
\end{equation}
for all $\displaystyle v \in H^1_0(\Omega)$.

To justify this formulation, suppose that
$\displaystyle u \in C(\overline{\Omega})\cap C^1(\Omega)$ is a
non-trivial classical solution of $-\Delta u = \lambda u$ for a given $\lambda$.
From the Green's identity (\ref{laplace_green1}), we have
\begin{align*}
\int_\Omega \nabla u(\VEC{x}) \cdot \nabla v(\VEC{x}) \dx{\VEC{x}}
&= \int_{\partial \Omega} v(\VEC{x})
\pdydx{u}{\nu}(\VEC{x}) \dss{S}{x} -
\int_{\Omega} v(\VEC{x}) \Delta u(\VEC{x}) \dx{\VEC{x}} \\
&= - \int_{\Omega} v(\VEC{x}) \Delta u(\VEC{x}) \dx{\VEC{x}}
= \lambda \int_{\Omega} v(\VEC{x}) u(\VEC{x}) \dx{\VEC{x}}
\end{align*}
for all functions $v \in \DD(\Omega)$.  Thus,
(\ref{laplace_vf}) follows by density of $\DD(\Omega)$ in
$\displaystyle H_0^1(\Omega)$.  Recall that
$\displaystyle \pdydx{u}{\nu}(\VEC{x})$ is the directional
derivative of $u$ at $\VEC{x}\in \partial \Omega$ in the direction of
the outward unit normal $\nu(\VEC{x})$ to the surface
$\partial \Omega$ at $\VEC{x}$.

Since $-\Delta$ is strongly elliptic and the sequilinear form $B$ of
the variational formulation (\ref{laplace_vf}) is also coercive, we
may use Theorem~\ref{ell_exist_th4} to get a complete description of
the eigenvalues and eigenfunctions of $-\Delta$ with their properties.

It is interesting to prove the orthogonality of eigenfunctions
associated to different eigenvalues.  Suppose that $\lambda_1$ and
$\lambda_2$ are two distinct eigenvalues of $-\Delta$,
$\displaystyle u_1 \in L^2(\Omega)$ is an eigenfunction associate to 
$\lambda_1$ and $\displaystyle u_2 \in L^2(\Omega)$ is an
eigenfunction associate to $\lambda_2$.  Then, it follows from
(\ref{laplace_vf}) that
\[
\lambda_2 \int_{\Omega} u_1(\VEC{x}) u_2(\VEC{x}) \dx{\VEC{x}} =
\int_{\Omega} \nabla u_1(\VEC{x}) \cdot \nabla u_2(\VEC{x}) \dx{\VEC{x}}
= \lambda_1 \int_{\Omega} u_1(\VEC{x}) u_2(\VEC{x}) \dx{\VEC{x}} \ .
\]
This gives
\[
\left( \lambda_2 - \lambda_1 \right) \int_{\Omega} u_1(\VEC{x})
u_2(\VEC{x}) \dx{\VEC{x}} = 0 \ .
\]
Since $\lambda_2 - \lambda_1 \neq 0$, this shows that $u_1$ and $u_2$
are orthogonal with respect to the scalar product
$\displaystyle \ps{f}{g} = \int_{\Omega} f\, g \dx{\VEC{x}}$
in $\displaystyle L^2(\Omega)$.

Let $\displaystyle \left\{ u_j \right\}_{j=1}^\infty$ be an orthogonal basis of
eigenfunctions for $-\Delta$.  We assume that $u_j$ is an
eigenfunction associated to the eigenvalue $\lambda_j$ and that this
eigenvalue is repeated $k$ times in
$\lambda_1 \leq \lambda_2 \leq \ldots$ if the dimension of the
eigenspace is $k$.

\begin{egg}
Fourier series can be used to solve the Poisson equation
$-\Delta u = f$ with $u =0$ on $\partial \Omega$
if $\displaystyle f \in L^2(\Omega)$.  Suppose that
$\displaystyle f = \sum_{j=1}^\infty a_j u_j$ is the Fourier series
expansion of $f$ with
$\displaystyle a_j = \frac{\ps{f}{u_j}_2}{\ps{u_j}{u_j}_2}$.  If 
$\displaystyle u = \sum_{j=1}^\infty b_j u_j$ is the solution
the Poisson equation, then a substitution of these two series in
$-\Delta u = f$ yields $\lambda_j b_j = a_j$ for all $j$.
Thus $b_j = a_j/\lambda_j$ for $j>0$.   Recall that the
series above converges with respect to the $\displaystyle L^2$-norm on
$\displaystyle L^2(\Omega)$ and that $-\Delta u = f$ in the variational sense.
\end{egg}

\begin{egg}
We may express the Green function on $\Omega$ in terms of the basis
of eigenfunctions.  Suppose that
$G:\Omega\times\overline{\Omega} \rightarrow \RR$ is the Green
function on $\Omega$.  According to
Theorem~\ref{laplace_dirichlet1}, the solution $u_j$ of
$\Delta u = -\lambda_j u$ with $u=0$ on $\partial \Omega$ satisfies
\[
u_j(\VEC{x}) = -\lambda_j \int_{\Omega} G(\VEC{x},\VEC{y})\, u_j(\VEC{y})
\dx{\VEC{y}}
\]
for all $\VEC{x} \in \Omega$.  We have assumed that
$\displaystyle u_j \in L^1(\Omega)$.

The following discussion could be justified rigorously.
For $\VEC{x} \in \Omega$ fixed, the Fourier series of
the function $\VEC{y} \mapsto G(\VEC{x},\VEC{y})$ for
$\VEC{y} \in \Omega$ is
\[
G(\VEC{x}, \cdot) = \sum_{j=1}^\infty a_j(\VEC{x}) u_j(\cdot) \ ,
\]
where this series converges with respect to the
standard $\displaystyle L^2$-norm on $\displaystyle L^2(\Omega)$ and
\[
a_j(\VEC{x})
= \left( \int_{\Omega} \left|u_j(\VEC{y})\right|^2 \dx{\VEC{y}} \right)^{-1}
\int_{\Omega} G(\VEC{x},\VEC{y}) u_j(\VEC{y}) \dx{\VEC{y}}
= - \frac{u_j(\VEC{x})}{\lambda_j \ps{u_j}{u_j}} \ .
\]
Therefore,
\[
G(\VEC{x},\VEC{y}) = - \sum_{j=1}^\infty \frac{u_j(\VEC{y})\,u_j(\VEC{x})}
{\lambda_j \ps{u_j}{u_j}} \ ,
\]
where this series converges with respect to the standard
$\displaystyle L^2$-norm on $\displaystyle L^2(\Omega\times\Omega)$.
\end{egg}

\section{Regularity of Weak Solutions} \label{ell_reg_of_sols}

Let $\Omega$ be an open subset of $\displaystyle \RR^n$.  In this section, we
consider the elliptic operator 
\begin{equation} \label{ell_reg_ellOp}
L(\VEC{x},\diff) = \sum_{|\VEC{\alpha}|\leq k}
a_{\VEC{\alpha}}(\VEC{x}) \diff^{\VEC{\alpha}}
\end{equation}
with $\displaystyle a_{\VEC{\alpha}} \in C^\infty(\Omega)$ for all
$\VEC{\alpha}$.  Some additional constraints may be imposed on the
$a_{\VEC{\alpha}}$ in some occasions. 

The goal of this section is to determine in which space the solution
of an elliptic partial differential equation belongs to.
The analysis is divided in two sections, the regularity of the
solution in the interior of $\Omega$ (local regularity) and the
regularity of the solution at the boundary of $\Omega$ (including on
the boundary).

Before talking about the regularity of solutions, we need to present
some results and tools that will be useful for this purpose.

\begin{lemma} \label{ell_reg_lemma1}
Given $s\in \RR$ and $t>n/2$, there exists a constant $C$ such that
\[
  \left\| \phi \, f \right\|_{s,\rho} \leq
(2\pi)^{n/2} \sup_{\VEC{x} \in \RR^n} |\phi(\VEC{x})|\, \|f\|_{s,\rho}
+ C \|\phi\|_{|s-1|+1+t,\rho} \, \|f \|_{s-1,\rho}
\]
for all $\phi \in \SS(\RR^n)$ and $f \in H^{s,2}(\RR^n)$.
\end{lemma}

\begin{proof}
\stage{i} We first prove that there exists $C$ such that
\begin{equation} \label{RWS_lem1E1}
\begin{split}
&\left( \int_{\RR^n} \left|
\int_{\RR^n} \left( (1+\|\VEC{x}\|_2^2)^{s/2} - (1+\|\VEC{y}\|_2^2)^{s/2}\right)
\, \hat{\phi}(\VEC{x}-\VEC{y}) \,
\hat{f}(\VEC{y}) \dx{\VEC{y}} \right|^2 \dx{\VEC{x}} \right)^{1/2}\\
&\qquad \qquad \leq C \|\phi\|_{|s-1|+1+t,\rho}\|f \|_{s-1,\rho}
\end{split}
\end{equation}
for all $\displaystyle \phi \in \SS(\RR^n)$ and
$\displaystyle f \in H^{s-1,2}(\RR^n)$.
Choose $\displaystyle v \in L^2(\RR^n)$ such that
$\displaystyle \hat{v}(\VEC{y}) = (1+\|\VEC{y}\|_2^2)^{(s-1)/2} \hat{f}(\VEC{y})$
for $\displaystyle \VEC{y} \in \RR^n$.  Such a $v$ exists
according to Plancherel theorem, Theorem~\ref{distr_plancherel},
because $\displaystyle f \in H^{s-1,2}(\RR^n)$ implies that
$\displaystyle (1 + \|\VEC{y}\|_2^2)^{(s-1)/2} \hat{f} \in L^2(\RR^n)$
by definition.  We may restate (\ref{RWS_lem1E1}) as
\begin{equation} \label{RWS_lem1E2}
\begin{split}
&\left( \int_{\RR^n} \left|
\int_{\RR^n} \left( (1+\|\VEC{x}\|_2^2)^{s/2} - (1+\|\VEC{y}\|_2^2)^{s/2}\right)
\left( 1 + \|\VEC{y}\|_2^2\right)^{(1-s)/2} \hat{\phi}(\VEC{x}-\VEC{y}) \,
\hat{v}(\VEC{y}) \dx{\VEC{y}} \right|^2 \dx{\VEC{x}} \right)^{1/2} \\
&\qquad \qquad \leq C \|\phi\|_{|s-1|+1+t,\rho}\|v \|_{0,\rho}
\end{split}
\end{equation}
for all $\displaystyle \phi \in \SS(\RR^n)$ and
$\displaystyle v \in H^{0,2}(\RR^n) = L^2(\RR^n)$.

We have that
\[
\int_{\RR^n} \left( (1+\|\VEC{x}\|_2^2)^{s/2} - (1+\|\VEC{y}\|_2^2)^{s/2}\right)
\left( 1 + \|\VEC{y}\|_2^2\right)^{(1-s)/2} \hat{\phi}(\VEC{x}-\VEC{y}) \,
\hat{v}(\VEC{y}) \dx{\VEC{y}}
= \int_{\RR^n} K(\VEC{x},\VEC{y}) \hat{v}(\VEC{y}) \dx{\VEC{y}}
\]
with
\[
K(\VEC{x},\VEC{y}) =
\left( (1+\|\VEC{x}\|_2^2)^{s/2} - (1+\|\VEC{y}\|_2^2)^{s/2}\right)
\left( 1 + \|\VEC{y}\|_2^2\right)^{(1-s)/2}
\hat{\phi}(\VEC{x}-\VEC{y}) \ .
\]
Our goal is to apply the Generalized Young's Inequality,
Theorem~\ref{distr_GyoungI}, to show that
\[
\left\| \int_{\RR^n} K(\VEC{x},\VEC{y}) \hat{v}(\VEC{y}) \dx{\VEC{y}}
\right\|_2 \leq C_1 \|\hat{v}\|_2 = C_1 \|v\|_{s-1,\rho}
\]
for a constant $C_1$ such that
$\displaystyle
\sup_{\VEC{x}\in \RR^n} \int_{\RR^n} \left|K(\VEC{x},\VEC{y})\right| \dx{\VEC{y}}
\leq C_1$
and
$\displaystyle
\sup_{\VEC{y}\in \RR^n} \int_{\RR^n} \left|K(\VEC{x},\VEC{y})\right|
\dx{\VEC{x}} \leq C_1$.
This will give us (\ref{RWS_lem1E2}) after we show that we can take
$C_1 = C \|\phi\|_{|s-1|+1+t,\rho}$ for some $C$.

Since
$\displaystyle \pdfdx{(1+w^2)^{s/2}}{w} = sw(1+w^2)^{(s-2)/2}$, we get
from the Mean Value Theorem that
\[
(1+\|\VEC{x}\|_2^2)^{s/2} - (1+\|\VEC{y}\|_2^2)^{s/2}
= sw(1+w^2)^{(s-2)/2} \, \left( \|\VEC{x}\|_2 - \|\VEC{y}\|_2 \right)
\]
for some $w$ between $\|\VEC{x}\|_2$ and $\|\VEC{y}\|_2$.  Hence,
\begin{align*}
&\left| (1+\|\VEC{x}\|_2^2)^{s/2} - (1+\|\VEC{y}\|_2^2)^{s/2} \right|
\leq |s| \sup_{w} \left| w(1+w^2)^{(s-2)/2}\right|
\, \big| \|\VEC{x}\|_2 - \|\VEC{y}\|_2 \big| \\
&\qquad \leq |s| \sup_{w} \left| (1 + w^2)^{1/2} (1+w^2)^{(s-2)/2}\right|
\, \big| \|\VEC{x}\|_2 - \|\VEC{y}\|_2 \big|
\leq |s| \sup_{w} \left|(1+w^2)^{(s-1)/2}\right|
\, \|\VEC{x} - \VEC{y}\|_2 \ ,
\end{align*}
where the supremum is taken over all $w$ between $\|\VEC{x}\|_2$ and
$\|\VEC{y}\|_2$.  Since $w \to (1+w^2)^{(s-1)/2}$ is increasing on
$[0,\infty[$ for $s > 1$ or decreasing on $[0,\infty[$ for $s<1$, we
get that
\[
\left| (1+\|\VEC{x}\|_2^2)^{s/2} - (1+\|\VEC{y}\|_2^2)^{s/2} \right|
\leq |s| \left( (1+\|\VEC{x}\|_2^2)^{(s-1)/2} + (1+\|\VEC{y}\|_2^2)^{(s-1)/2}
\right) \, \|\VEC{x} - \VEC{y}\|_2 \ ,
\]
It follows from this relation and Lemma~\ref{sob_tri_equ} that
\begin{align*}
\left| K(\VEC{x},\VEC{y}) \right|
&\leq |s| \left(
\frac{(1+\|\VEC{x}\|_2^2)^{(s-1)/2}}{(1+\|\VEC{y}\|_2^2)^{(s-1)/2}} + 1 \right)
\, \|\VEC{x} - \VEC{y}\|_2
\left|\hat{\phi}(\VEC{x}-\VEC{y})\right| \\
&\leq  |s| \left( 2^{|s-1|/2} (1+\|\VEC{x}-\VEC{y}\|_2^2)^{|s-1|/2} + 1 \right)
\, \|\VEC{x} - \VEC{y}\|_2
\left|\hat{\phi}(\VEC{x}-\VEC{y})\right| \\
&\leq 2^{1+|s-1|/2} |s|\, (1+\|\VEC{x}-\VEC{y}\|_2^2)^{|s-1|/2}
\, \left( 1 + \|\VEC{x} - \VEC{y}\|_2^2\right)^{1/2}
\left|\hat{\phi}(\VEC{x}-\VEC{y})\right| \\
&= 2^{1+|s-1|/2} |s|\, (1+\|\VEC{x}-\VEC{y}\|_2^2)^{(|s-1|+1)/2}
\left|\hat{\phi}(\VEC{x}-\VEC{y})\right| \ .
\end{align*}
Hence, from Schwarz inequality, we get
\begin{align*}
\int_{\RR^n} \left|K(\VEC{x},\VEC{y})\right| \dx{\VEC{y}}
&\leq 2^{1+|s-1|/2} |s|\, \int_{\RR^n} (1+\|\VEC{x}-\VEC{y}\|_2^2)^{(|s-1|+1)/2}
\left|\hat{\phi}(\VEC{x}-\VEC{y})\right| \dx{\VEC{y}} \\
&\leq  2^{1+|s-1|/2} |s|\, \left(\int_{\RR^n}
\left((1+\|\VEC{x}-\VEC{y}\|_2^2)^{(|s-1|+1+t)/2}
\left|\hat{\phi}(\VEC{x}-\VEC{y})\right|\right)^2 \dx{\VEC{y}} \right)^{1/2} \\
&\qquad \left( \int_{\RR^n} \left( (1+\|\VEC{x}-\VEC{y}\|_2^2)^{-t/2} \right)^2 
\dx{\VEC{y}} \right)^{1/2}
= C \|\phi\|_{|s-1|+1+t,\rho}
\end{align*}
for all $\displaystyle \VEC{x} \in \RR^n$ with
$\displaystyle C = 2^{1+|s-1|/2} |s|\, 
\left( \int_{\RR^n} (1+\|\VEC{x}-\VEC{y}\|_2^2)^{-t}\dx{\VEC{y}} \right)^{1/2}$.

A very similar reasoning shows that
$\displaystyle
\sup_{\VEC{y}\in \RR^n} \int_{\RR^n} \left|K(\VEC{x},\VEC{y})\right|
\dx{\VEC{x}} \leq C \|\phi\|_{|s-1|+1+t,\rho}$.

\stage{ii}  Using triangle inequality on $L^2(\RR^n)$, we find that
\begin{align*}
\|\phi\, f \|_{s,\rho}
&= \left( \int_{\RR^n} \left|(\phi\,f)^\wedge(\VEC{x})\right|^2
(1 + \|\VEC{x}\|_2^2)^s \dx{\VEC{x}} \right)^{1/2} \\
&= \left( \int_{\RR^n} \left( \int_{\RR^n} \hat{\phi}(\VEC{x}-\VEC{y})
    \hat{f}(\VEC{y}) \dx{\VEC{y}} \right)^2
(1 + \|\VEC{x}\|_2^2)^s \dx{\VEC{x}} \right)^{1/2} \\
&= \left( \int_{\RR^n} \left( \int_{\RR^n}
\left( (1 + \|\VEC{x}\|_2^2)^{s/2} - (1 + \|\VEC{y}\|_2^2)^{s/2}\right)
\hat{\phi}(\VEC{x}-\VEC{y}) \hat{f}(\VEC{y}) \dx{\VEC{y}} \right. \right. \\
&\qquad \left. \left. 
+ \int_{\RR^n} (1 + \|\VEC{y}\|_2^2)^{s/2} \hat{\phi}(\VEC{x}-\VEC{y})
\hat{f}(\VEC{y}) \dx{\VEC{y}} \right)^2 \dx{\VEC{x}} \right)^{1/2} \\
&\leq \left( \int_{\RR^n} \left( \int_{\RR^n}
\left( (1 + \|\VEC{x}\|_2^2)^{s/2} - (1 + \|\VEC{y}\|_2^2)^{s/2}\right)
\hat{\phi}(\VEC{x}-\VEC{y}) \hat{f}(\VEC{y}) \dx{\VEC{y}} \right)^2
  \dx{\VEC{x}} \right)^{1/2} \\
&\qquad + \left( \int_{\RR^n} \left(
\int_{\RR^n} (1 + \|\VEC{y}\|_2^2)^{s/2} \hat{\phi}(\VEC{x}-\VEC{y})
\hat{f}(\VEC{y}) \dx{\VEC{y}} \right)^2 \dx{\VEC{x}} \right)^{1/2} \ .
\end{align*}
From (i), we have that the first integral on the right hand side is
bounded by\\
$C \|\phi\|_{|s-1|+1+t,\rho}\|f \|_{s-1,\rho}$.

For the second integral, let $\displaystyle g \in L^2(\RR^n)$ be such that
$\displaystyle \hat{g} = (1 + \|\VEC{y}\|_2^2)^{s/2} \hat{f}$.  Such a
$g$ exists according to Plancherel theorem, Theorem~\ref{distr_plancherel},
because $\displaystyle f \in H^{s,2}(\RR^n)$ implies that
$\displaystyle (1 + \|\VEC{y}\|_2^2)^{s/2} \hat{f} \in L^2(\RR^n)$ by
definition.  We have that
\begin{align*}
&\left( \int_{\RR^n} \left(
\int_{\RR^n} (1 + \|\VEC{y}\|_2^2)^{s/2} \hat{\phi}(\VEC{x}-\VEC{y})
\hat{f}(\VEC{y}) \dx{\VEC{y}} \right)^2 \dx{\VEC{x}} \right)^{1/2}
= \left( \int_{\RR^n} \left(
\int_{\RR^n} \hat{\phi}(\VEC{x}-\VEC{y}) \hat{g}(\VEC{y})
\dx{\VEC{y}} \right)^2 \dx{\VEC{x}} \right)^{1/2} \\
&\qquad = (2\pi)^{n/2} \left(\int_{\RR^n}
\left|\hat{\phi} \ast \hat{g} \right|^2 \dx{\VEC{x}} \right)^{1/2}
= (2\pi)^{n/2}
\left(\int_{\RR^n} \left|(\phi\, g)^\wedge \right|^2 \dx{\VEC{x}} \right)^{1/2}  
= (2\pi)^{n/2} \big\| \phi\,g\|_{0,\rho} \\
&\qquad
= (2\pi)^{n/2} \big\| \phi\,g\|_2
\leq (2\pi)^{n/2} \sup_{\VEC{x}\in\RR^n} |\phi(\VEC{x})| \|g\|_2
= (2\pi)^{n/2} \sup_{\VEC{x}\in\RR^n} |\phi(\VEC{x})|\, \|g\|_{0,\rho} \\
&\qquad = (2\pi)^{n/2} \sup_{\VEC{x}\in\RR^n} |\phi(\VEC{x})|
\left(\int_{\RR^n} \left|\hat{g} \right|^2 \dx{\VEC{x}} \right)^{1/2}
= (2\pi)^{n/2} \sup_{\VEC{x}\in\RR^n} |\phi(\VEC{x})|
\left(\int_{\RR^n}
\left|(1 + \|\VEC{y}\|_2^2)^{s/2} \hat{f} \right|^2 \dx{\VEC{x}} \right)^{1/2} \\
&\qquad = (2\pi)^{n/2} \sup_{\VEC{x}\in\RR^n} |\phi(\VEC{x})|\, \|f\|_{s,\rho} \ ,
\end{align*}
where we have used Plancherel theorem for the fifth and sixth equality.
\end{proof}

The proofs of regularity of solutions require the
{\bfseries method of translations}\index{Method of Translations}, also
known as the
{\bfseries method of difference quotient}\index{Method of Difference Quotient},
that we present before turning all our attention to the goal of this section.

For $\displaystyle f\in L^2(\RR^n)$ and $h \in \RR\setminus \{0\}$, let
\begin{equation}  \label{ell_diffQuot}
\left(\Delta_{h\VEC{e}_j} f\right)(\VEC{x}) =
\frac{1}{h} \left( f(\VEC{x} + h\VEC{e}_j) -f(\VEC{x}) \right)
\end{equation}
and, recursively,
$\displaystyle \Delta_{h\VEC{e}_j}^q f =
\Delta_{h\VEC{e}_j} \left(\Delta_{h\VEC{e}_j}^{q-1} f \right)$
for $q>1$.
We also define $\Delta^0_{h\VEC{e}_j}$ as
$\displaystyle \Delta^0_{h\VEC{e}_j} f = f$ for $h \in \RR$.
For $\VEC{h} \in \RR^n$ and $\VEC{\alpha} \in \NN^n$ with $h_j \neq 0$ for
$\alpha_j \neq 0$, let
\[
\Delta_{\VEC{h}}^{\VEC{\alpha}} f = \Delta_{h_n\VEC{e}_n}^{\alpha_n} 
\Delta_{h_{n-1}\VEC{e}_{n-1}}^{\alpha_{n-1}} \ldots
\Delta_{h_1\VEC{e}_1}^{\alpha_1} f \ .
\]
$\displaystyle \Delta_{\VEC{h}}^{\VEC{\alpha}}$ is independent of the
order of the difference quotients
$\displaystyle \Delta_{h_i\VEC{e}_i}^{\alpha_i}$ chosen.
We also have that
$\displaystyle \Delta_{\VEC{h}}^{\VEC{\alpha}}
\left( \Delta_{\VEC{h}}^{\VEC{\beta}} f\right)
= \Delta_{\VEC{h}}^{\VEC{\alpha}} \left( \Delta_{\VEC{h}}^{\VEC{\beta}} f\right)$
for all multi-indices $\VEC{\alpha}$ and $\VEC{\beta}$ in $\displaystyle \NN^n$.
If $\displaystyle f \in H^k(\RR^n)$, it is useful to note that
$\displaystyle \diff^{\VEC{\alpha}}\left( \Delta_{\VEC{h}}^{\VEC{\beta}} f \right)
= \Delta_{\VEC{h}}^{\VEC{\beta}} \left(\diff^{\VEC{\alpha}} f \right)$ for all
multi-indices $\VEC{\alpha}$ and $\VEC{\beta}$ in $\displaystyle \NN^n$ with
$|\VEC{\alpha}|\leq k$, where the
derivative is in the sense of distribution.  The proofs of these
statements is left to the reader.

The {\bfseries adjoint} of $\displaystyle \Delta_{\VEC{h}}^{\VEC{\alpha}}$ is
$\displaystyle (-1)^{|\VEC{\alpha}|}\Delta_{-\VEC{h}}^{\VEC{\alpha}}$.  The
corollary to the next result justifies this definition.  The nest
result will occasionally be useful later.

\begin{prop} \label{DiffQuotientAdjoint}
Suppose that $k \in \NN$ and that $\displaystyle f,g \in L^2(\Omega)$,
where $\Omega$ is an open subset of $\displaystyle \RR^n$.  Moreover,
suppose that there exists $\delta$ such that
$\supp f + B_{k\delta}(\VEC{0}) \subset \Omega$ and
$\supp g + B_{k\delta}(\VEC{0}) \subset \Omega$, then
\[
\int_{\Omega} \left(\Delta^{\VEC{\alpha}}_{\VEC{h}} f\right) g \dx{\VEC{x}}
= (-1)^{|\VEC{\alpha}|}
\int_{\Omega} f\,\left(\Delta^{\VEC{\alpha}}_{-\VEC{h}} g\right) \dx{\VEC{x}}
\]
for all multi-indices $\VEC{\alpha}$ such that $|\VEC{\alpha}|\leq k$, and all
$\displaystyle \VEC{h} \in \RR^n$ such that $\|\VEC{h}\|<\delta$.
\end{prop}

\begin{proof}
\stage{i} We first consider $\VEC{\alpha} = \alpha_j \VEC{e}_j$ with
$\alpha_j \leq k$ and $\|\VEC{y}\|< \delta$.  We have that
\begin{align*}
\int_{\Omega} \left(\Delta^{\VEC{\alpha}}_{\VEC{h}} f (\VEC{x})\right)
g(\VEC{x}) \dx{\VEC{x}}
&= \int_{\Omega} \left(\Delta^{\alpha_j}_{h_j \VEC{e}_j} f(\VEC{x})\right)
g(\VEC{x}) \dx{\VEC{x}} \\
&= \int_{\Omega} \frac{1}{h_j^{\alpha_j}}
\left( \sum_{i=0}^{\alpha_1} (-i)^i \binom{\alpha_j}{i}
f\big(\VEC{x} + (\alpha_j-i) h_j \VEC{e}_j\big) \right)
g(\VEC{x})\dx{\VEC{x}} \\
&= \frac{1}{h_j^{\alpha_j}} \sum_{i=0}^{\alpha_1} (-i)^i \binom{\alpha_j}{i}
\int_{\Omega} f\big(\VEC{x} + (\alpha_j-i) h_j \VEC{e}_j\big)
g(\VEC{x})\dx{\VEC{x}} \ .
\end{align*}
Using the substitution $\VEC{y} = \VEC{x} + (\alpha_j-i)h_j\VEC{e}_j$,
we get
\begin{align*}
\int_{\Omega} \left(\Delta^{\VEC{\alpha}}_{\VEC{h}} f (\VEC{x})\right)
g(\VEC{x}) \dx{\VEC{x}}
&= \frac{1}{h_j^{\alpha_j}} \sum_{i=0}^{\alpha_1} (-i)^i \binom{\alpha_j}{i}
\int_{\Omega} f(\VEC{y}) g\big(\VEC{y} - (\alpha_j-i) h_j \VEC{e}_j\big)
\dx{\VEC{y}} \\
&= (-1)^{\alpha_j} \int_{\Omega} f(\VEC{y}) \left( \frac{1}{(-h_j)^{\alpha_j}}
\sum_{i=0}^{\alpha_1} (-i)^i \binom{\alpha_j}{i}
g\big(\VEC{y} - (\alpha_j-i) h_j \VEC{e}_j\big) \right) \dx{\VEC{y}} \\
&= (-1)^{\alpha_j} \int_{\Omega} f(\VEC{y})
\left(\Delta^{\alpha_j}_{-h_j \VEC{e}_j} g
(\VEC{y})\right)  \dx{\VEC{y}} \ .
\end{align*}
Note that the previous change of variable is justified by the fact that
$\supp f - (\alpha_j -i)h_j\VEC{e}_j \subset \Omega$
and
$\supp g + (\alpha_j -i)h_j\VEC{e}_j \subset \Omega$
for all $i$.

\stage{ii} For the general case, we proceed recursively using (i).
\begin{align*}
\int_{\Omega} \left(\Delta^{\VEC{\alpha}}_{\VEC{h}} f (\VEC{x})\right) g(\VEC{x})
\dx{\VEC{x}}
&= \int_{\Omega} \left(
\Delta_{h_n\VEC{e}_n}^{\alpha_n} \Delta_{h_{n-1}\VEC{e}_{n-1}}^{\alpha_{n-1}} \ldots
\Delta_{h_1\VEC{e}_1}^{\alpha_1} f (\VEC{x})\right) g(\VEC{x}) \dx{\VEC{x}} \\
&= (-1)^{\alpha_n} \int_{\Omega} \left(
\Delta_{h_{n-1}\VEC{e}_{n-1}}^{\alpha_{n-1}} \ldots
\Delta_{h_1\VEC{e}_1}^{\alpha_1} f (\VEC{x})\right)
\left( \Delta_{-h_n\VEC{e}_n}^{\alpha_n}  g(\VEC{x}) \right) \dx{\VEC{x}} \\
& \ldots \\
&= (-1)^{\alpha_n + \alpha_{n-1} + \ldots + \alpha_1}
\int_{\Omega} f(\VEC{x}) \left( \Delta_{-h_1\VEC{e}_1}^{\alpha_1} \ldots
\Delta_{-h_{n-1}\VEC{e}_{n-1}}^{\alpha_{n-1}} \Delta_{-h_n\VEC{e}_n}^{\alpha_n}
g(\VEC{x})\right) \dx{\VEC{x}} \\
&= (-1)^{|\VEC{\alpha}|} \int_{\Omega} f(\VEC{x})
\left(\Delta^{\VEC{\alpha}}_{-\VEC{h}} g (\VEC{x})\right) \dx{\VEC{x}} \ .
\qedhere
\end{align*}
\end{proof}

\begin{cor}
Suppose that $\displaystyle f,g \in L^2(\RR^n)$.  Then
\[
\int_{\RR^n} \left(\Delta^{\VEC{\alpha}}_{\VEC{h}} f\right) g \dx{\VEC{x}}
= (-1)^{|\VEC{\alpha}|}
\int_{\RR^n} f\,\left(\Delta^{\VEC{\alpha}}_{-\VEC{h}} g\right) \dx{\VEC{x}}
\]
for all multi-indices $\VEC{\alpha}$ and all $\displaystyle \VEC{h} \in \RR^n$.
\end{cor}

\begin{proof}
The conditions $\displaystyle \supp f + B_{k\delta}(\VEC{0}) \subset \RR^n$ and
$\displaystyle \supp g + B_{k\delta}(\VEC{0}) \subset \RR^n$ of the previous
proposition are satisfied for all $\delta>0$ and $k \in \NN$.
\end{proof}

\begin{prop} \label{ell_reg_prop3}
Given $\displaystyle f\in H^{s,2}(\RR^n)$ and
$\displaystyle \VEC{\alpha} \in \NN^n$, we have that
\begin{equation} \label{ellRegProp3Eq1}
\big\|\diff^{\VEC{\alpha}} f\big\|_{s,\rho} = \limsup_{\VEC{h}\rightarrow \VEC{0}}
\big\|\Delta^{\VEC{\alpha}}_{\VEC{h}} f\big\|_{s,\rho} \ ,
\end{equation}
where the possibility of both sides to be infinite is not ruled out.
It follows that $\displaystyle \diff^{\VEC{\alpha}} f \in H^{s,2}(\RR^n)$
if and only if $\displaystyle \limsup_{\VEC{h}\rightarrow \VEC{0}}
\big\|\Delta^{\VEC{\alpha}}_{\VEC{h}} f\big\|_{s,\rho}$ is finite.

Moreover,
$\displaystyle \left\| \Delta^{\VEC{\alpha}}_{\VEC{h}} f  \right\|_{s,\rho}
\leq \left\| \diff^{\VEC{\alpha}} f\right|_{s,\rho}$ for all
$\displaystyle f\in H^{s,2}(\RR^n)$, $\VEC{h} \in \RR^n$ and
$\displaystyle \VEC{\alpha} \in \NN^n$ with $h_j \neq 0$ for $\alpha_j \neq 0$.
\end{prop}

\begin{proof}
We first recall a definition and a result that will be quite useful in
this proof.  As we have defined in Section~\ref{sectConvolution},
$f_{\VEC{r}}(\VEC{x}) = f(\VEC{x}- \VEC{r})$ for all
$\displaystyle \VEC{x},\VEC{r} \in \RR^n$.
From Proposition~\ref{FTLofPs}, we have that
$\displaystyle (f_{\VEC{r}})^\wedge(\VEC{y}) =
e^{-i \VEC{r}\cdot\VEC{y}} \hat{f}(\VEC{y})$.

We prove by induction that
\begin{equation} \label{ellRegProp3Eq2}
\left(\Delta^{\VEC{\alpha}}_{\VEC{h}} f\right)^\wedge(\VEC{y})
= \prod_{j=1}^m \left(\frac{(2i)^{\alpha_j}}
{h_j^{\alpha_j}} e^{i h_j y_j \alpha_j/2}
\sin^{\alpha_j}\left(\frac{h_j y_j}{2}\right) \right)\hat{f}(\VEC{y})
\end{equation}
for $\VEC{h} = (h_1,\ldots,h_m,0,\ldots,0)$ and
$\VEC{\alpha} = (\alpha_1,\ldots,\alpha_m,0,\ldots,0)$.

For $\VEC{h} = (h_1,0,\ldots,0$ and $\VEC{\alpha} = (\alpha_1,0,\ldots,0)$,
we have that
\begin{align*}
\left(\Delta^{\VEC{\alpha}}_{\VEC{h}} f \right)^\wedge(\VEC{y})
&= \left(\Delta^{\alpha_1}_{h_1 \VEC{e}_1} f \right)^\wedge(\VEC{y})
= \frac{1}{h_1^{\alpha_1}} \sum_{j_1=0}^{\alpha_1} (-1)^{j_1}
\binom{\alpha_1}{j_1}
\left(f_{-(\alpha_1-j_1)h_1 \VEC{e}_1}\right)^\wedge(\VEC{y}) \\
&= \frac{1}{h_1^{\alpha_1}} \sum_{j_1=0}^{\alpha_1} (-1)^{j_1}
\binom{\alpha_1}{j_1} e^{i(\alpha_1-j_1) h_1 y_1} \hat{f}(\VEC{y})
= \frac{1}{h_1^{\alpha_1}} \left(e^{i h_1 y_1} - 1\right)^{\alpha_1}
\hat{f}(\VEC{y}) \\
&= \frac{(2i)^{\alpha_1}}{h_1^{\alpha_1}} e^{i h_1 y_1 \alpha_1/2}
\sin^{\alpha_1}\left(\frac{h_1 y_1}{2}\right) \hat{f}(\VEC{y}) \ .
\end{align*}
So (\ref{ellRegProp3Eq2}) is true for $m=1$.

Suppose that (\ref{ellRegProp3Eq2}) is true for $m < n$.  If
$\VEC{h} = (h_1,\ldots,h_{m+1},0,\ldots,0)$ and\\
$\VEC{\alpha} = (\alpha_1,\ldots,\alpha_{m+1},0,\ldots,0)$, let
$\tilde{\VEC{h}} = (h_1,\ldots,h_m,0,\ldots,0)$ and
$\tilde{\VEC{\alpha}} = (\alpha_1,\ldots,\alpha_m,0,\ldots,0)$.  Then,
\begin{align*}
&\left(\Delta^{\VEC{\alpha}}_{\VEC{h}} f \right)^\wedge(\VEC{y})
= \left( \Delta^{\tilde{\VEC{\alpha}}}_{\tilde{\VEC{h}}}
\left( \Delta^{\alpha_{m+1}}_{h_{m+1}\VEC{e}_{m+1}} f \right)\right)^\wedge
(\VEC{y})
= \prod_{j=1}^m \left(\frac{(2i)^{\alpha_j}}{h_j^{\alpha_j}}
e^{i h_j y_j \alpha_j/2}
\sin^{\alpha_j}\left(\frac{h_j y_j}{2}\right) \right)
\left( \Delta^{\alpha_{m+1}}_{h_{m+1}\VEC{e}_{m+1}} f \right)^\wedge(\VEC{y}) \\
&\qquad
= \prod_{j=1}^m \left(\frac{(2i)^{\alpha_j}}{h_j^{\alpha_j}}
e^{i h_j y_j \alpha_j/2}
\sin^{\alpha_j}\left(\frac{h_j y_j}{2}\right) \right)
\left( \frac{(2i)^{\alpha_{m+1}}}{h_{m+1}^{\alpha_{m+1}}}
e^{i h_{m+1} y_{m+1} \alpha_{m+1}/2}
\sin^{\alpha_{m+1}}\left(\frac{h_{m+1} y_{m+1}}{2}\right)
\hat{f}(\VEC{y}) \right) \\
&\qquad
=\prod_{j=1}^{m+1} \left(\frac{(2i)^{\alpha_j}}{h_j^{\alpha_j}}
e^{i h_j y_j \alpha_1/2}
\sin^{\alpha_j}\left(\frac{h_j y_j}{2}\right) \right)\hat{f}(\VEC{y}) \ ,
\end{align*}
where we have used the hypothesis of induction for the second equality.
This is (\ref{ellRegProp3Eq2}) with $m$ replaced by $m+1$.
Thus (\ref{ellRegProp3Eq2}) is true for all $m \leq n$.

Since $\displaystyle \left| \frac{2}{h_j}
\sin\left(\frac{h_j y_j}{2}\right)\right| \leq |y_j|$ for all $j$, we
get from (\ref{ellRegProp3Eq2}) that
\begin{equation} \label{ellRegProp3Eq3}
\left| \left(\Delta^{\VEC{\alpha}}_{\VEC{h}} f\right)^\wedge(\VEC{y}) \right|
\leq \prod_{j=1}^n \left| y_j \right|^{\alpha_j}
\,\left| \hat{f}(\VEC{y})\right|
= \left| \prod_{j=1}^n y_j^{\alpha_j} \,\hat{f}(\VEC{y})\right|
= \left| \VEC{y}^{\VEC{\alpha}} \,\hat{f}(\VEC{y})\right|
\end{equation}
for all $\displaystyle \VEC{y} \in \RR^n$.  Moreover, since
$\displaystyle \lim_{h_j \to 0} \frac{2}{h_j} \sin\left(\frac{h_j y_j}{2}\right)
= y_j$ for all $j$, we get from (\ref{ellRegProp3Eq2}) that
\begin{equation} \label{ellRegProp3Eq4}
\lim_{\VEC{h}\to \VEC{0}}  \left(\Delta^{\VEC{\alpha}}_{\VEC{h}}
f\right)^\wedge(\VEC{y})
= i^{|\VEC{\alpha}|} \prod_{j=1}^n y_j^{\alpha_j} \,\hat{f}(\VEC{y})
= (i\VEC{y})^{\VEC{\alpha}} \,\hat{f}(\VEC{y})
\end{equation}
for all $\displaystyle \VEC{y} \in \RR^n$.

\stage{i} If $\displaystyle \diff^{\VEC{\alpha}} f \in H^{s,2}(\RR^n)$, then
it follows from (\ref{ellRegProp3Eq3}) that
\begin{equation} \label{ellRegProp3Eq5}
\begin{split}
(1+\|\VEC{y}\|_2^2)^s
\left| (\Delta^{\VEC{\alpha}}_{\VEC{h}} f)^\wedge (\VEC{y}) \right|^2
&\leq (1+\|\VEC{y}\|_2^2)^s
\left| \VEC{y}^{\VEC{\alpha}} \,\hat{f}(\VEC{y})\right|^2 \\
&= (1+\|\VEC{y}\|_2^2)^s \left| (\diff^{\VEC{\alpha}} f)^\wedge(\VEC{y})\right|^2
\in L(\RR^n)
\end{split}
\end{equation}
for all $\displaystyle \VEC{y} \in \RR^n$, and from (\ref{ellRegProp3Eq4}) that
\[
\lim_{\VEC{h}\to \VEC{0}} \, (1+\|\VEC{y}\|_2^2)^s
\left|(\Delta^{\VEC{\alpha}}_{\VEC{h}} f)^\wedge(\VEC{y})\right|^2
= (1+\|\VEC{y}\|_2^2)^s \left|(i \VEC{y})^{\VEC{\alpha}}
\,\hat{f}(\VEC{y})\right|^2
= (1+\|\VEC{y}\|_2^2)^s \left| (\diff^{\VEC{\alpha}} f)^\wedge(\VEC{y})\right|^2
\]
for all $\displaystyle \VEC{y} \in \RR^n$.  Hence, we get from the Lebesgue
Dominate Convergence Theorem that
\[
\lim_{\VEC{h}\rightarrow \VEC{0}}
\big\|\Delta^{\VEC{\alpha}}_{\VEC{h}} f\big\|_{s,\rho}
= \big\|\diff^{\VEC{\alpha}} f\big\|_{s,\rho} \ .
\]

\stage{ii} Let us suppose that
$\displaystyle \limsup_{\VEC{h}\rightarrow \VEC{0}}
\big\|\Delta^{\VEC{\alpha}}_{\VEC{h}} f\big\|_{s,\rho}$ is finite but
$\big\|\diff^{\VEC{\alpha}} f\big\|_{s,\rho} = \infty$; namely,\\
$\displaystyle \int_{\RR^n} \left| \VEC{y}^{\VEC{\alpha}}
\hat{f}(\VEC{y})\right|^2 (1 + \|\VEC{y}\|_2^2)^s \dx{\VEC{y}} = \infty$
(see item (1) in Remark~\ref{ell_reg_prop3_rmk} below).

Therefore, given $M>0$, there exists $R>0$ such that
\[
\int_{\|\VEC{y}\|\leq R} \left| \VEC{y}^{\VEC{\alpha}}\hat{f}(\VEC{y})\right|^2
(1+\|\VEC{y}\|_2^2)^s \dx{\VEC{y}} > M \ .
\]

Since $\displaystyle \lim_{p \to 0}\left|\frac{\sin(p)}{p}\right| = 1$, 
there exists $P>0$ such that
$\displaystyle \left|\frac{\sin(p)}{p}\right| \geq \frac{1}{2}$
for $|p|<P$.  Therefore,
$\displaystyle \left| \frac{2}{h_jy_j} \sin\left(\frac{h_j y_j}{2}\right)\right|
\geq \frac{1}{2}$ for $\displaystyle \left|\frac{h_jy_j}{2}\right|<P$.
It follows that
$\displaystyle \left| \frac{2}{h_jy_j} \sin\left(\frac{h_j y_j}{2}\right)\right|
\geq \frac{1}{2}$ for all $y_j$ such that $|y_j| \leq R$ if
$\displaystyle |h_j| < 2P/R$.

Therefore, we have for $\displaystyle \|h\|_2 < 2P/R$ that
\begin{align*}
&\left\| \Delta^{\VEC{\alpha}}_{\VEC{h}} f \right\|_{s,\rho}^2
= \int_{\RR^n} \prod_{j=1}^n \left| \frac{(2i)^{\alpha_j}}{h_j^{\alpha_j}}
e^{i h_j y_j \alpha_j/2} \sin^{\alpha_j}\left(\frac{h_j y_j}{2}\right) \right|^2
\,\left| \hat{f}(\VEC{y})\right|^2 (1+\|\VEC{y}\|_2^2)^s \dx{\VEC{y}} \\
&\qquad = \int_{\RR^n}
\prod_{j=1}^n \left| \frac{2^{\alpha_j}}{h_j^{\alpha_j}}
\sin^{\alpha_j}\left(\frac{h_j y_j}{2}\right) \right|^2
\,\left| \hat{f}(\VEC{y})\right|^2 (1+\|\VEC{y}\|_2^2)^s \dx{\VEC{y}} \\
&\qquad \geq \int_{\|\VEC{y}\|\leq R} 
\prod_{j=1}^n \left| \frac{2^{\alpha_j}}{h_j^{\alpha_j}}
\sin^{\alpha_j}\left(\frac{h_j y_j}{2}\right) \right|^2
\,\left| \hat{f}(\VEC{y})\right|^2 (1+\|\VEC{y}\|_2^2)^s \dx{\VEC{y}} \\
&\qquad \geq \int_{\|\VEC{y}\|\leq R}
\prod_{j=1}^n \left( \frac{|y_j|^{\alpha_j}}{2^{\alpha_j}} \right)^2
\,\left| \hat{f}(\VEC{y})\right|^2 (1+\|\VEC{y}\|_2^2)^s \dx{\VEC{y}} \\
&\qquad = \frac{1}{2^{2|\VEC{\alpha}|}} \int_{\|\VEC{y}\|\leq R}
\left| \VEC{y}^{\VEC{\alpha}} \, \hat{f}(\VEC{y}) \right|^2
(1+\|\VEC{y}\|_2^2)^s \dx{\VEC{y}}
> \frac{M}{2^{2|\VEC{\alpha}|}} \ .
\end{align*}
Since this is true for all $M>0$, we get that
$\displaystyle \limsup_{\VEC{h}\rightarrow \VEC{0}}
\big\|\Delta^{\VEC{\alpha}}_{\VEC{h}} f\big\|_{s,\rho} = \infty$.
This contradicts our hypothesis, therefore we must have that
$\displaystyle
\int_{\RR^n} \left| \VEC{y}^{\VEC{\alpha}} \hat{f}(\VEC{y})\right|^2
(1 + \|\VEC{y}\|_2^2)^s \dx{\VEC{y}} < \infty$
and so $\displaystyle \diff^{\VEC{\alpha}} f \in H^{s,2}(\RR^n)$ (see
item (2) in Remark~\ref{ell_reg_prop3_rmk} below).

\stage{iii}  The last conclusion of the proposition follows
from (\ref{ellRegProp3Eq5}).
\end{proof}

\begin{rmkList} \label{ell_reg_prop3_rmk}
\begin{enumerate}
\item If $\displaystyle f \in H^{s,2}(\RR^n)$, then $f$ is a temperate
distribution.  It follows from Proposition~\ref{distr_frr_tempD}
that $\displaystyle \left(\diff^{\VEC{\alpha}} f\right)^\wedge =
(i\VEC{y})^{\VEC{\alpha}} \hat{f}$ in the sense of distribution.  So
$\displaystyle \left(\diff^{\VEC{\alpha}} f\right)^\wedge$ is also represented
by a measurable function.  However, there is no guarantee that
$\displaystyle (1+\|\VEC{y}\|_2^2)^{s/2} \left(\diff^{\VEC{\alpha}}
f\right)^\wedge$ is in $L^2(\RR^n)$; namely, that
$\displaystyle \left|\diff^{\VEC{\alpha}} f \right\|_{s,\rho}$
is finite.
\item If $\displaystyle
\int_{\RR^n} \left| \VEC{y}^{\VEC{\alpha}} \hat{f}(\VEC{y})\right|^2
(1 + \|\VEC{y}\|_2^2)^s \dx{\VEC{y}} < \infty$ and $s \geq 0$, then
$\displaystyle \left(\diff^{\VEC{\alpha}} f\right)^\wedge
= (i\VEC{y})^{\VEC{\alpha}} \hat{f} \in L^2(\RR^n)$ in the sense of
distributions.
It follows from Plancherel theorem, Theorem~\ref{distr_plancherel},
that there exists $\displaystyle g \in L^2(\RR^n)$
such that $\displaystyle (\diff^{\VEC{\alpha}} f)^\wedge = \hat{g}$ in
the sense of distributions.  As we have shown in the proof of
Theorem~\ref{sob_wk2_wk2}, this implies that
$\displaystyle \diff^{\VEC{\alpha}} f = g \in L^2(\RR^n)$ in the sense of
distributions.  
\item  From the proof of the previous proposition, we have that
$\displaystyle \limsup_{\VEC{h}\rightarrow \VEC{0}}
\|\Delta^{\VEC{\alpha}}_{\VEC{h}} f\|_{s,\rho} < \infty$ implies that
$\displaystyle \diff^{\VEC{\alpha}} f \in H^{s,2}(\RR^n)$, and that we
have in fact $\displaystyle \lim_{\VEC{h}\rightarrow \VEC{0}}
\|\Delta^{\VEC{\alpha}}_{\VEC{h}} f\|_{s,\rho}
= \|\diff^{\VEC{\alpha}} f\|_{s,\rho}$.
\end{enumerate}
\end{rmkList}

Given $\displaystyle \phi \in \SS(\RR^n)$, we define the linear operator
$\displaystyle [\Delta^{\VEC{\alpha}}_{\VEC{h}}, \phi]$ on
$\displaystyle H^{s,2}(\RR^n)$ as
\[
[\Delta^{\VEC{\alpha}}_{\VEC{h}}, \phi]f
= \Delta^{\VEC{\alpha}}_{\VEC{h}} \left( \phi f \right) -
\phi \Delta^{\VEC{\alpha}}_\VEC{h} f
\]
for $\displaystyle f \in H^{s,2}(\RR^n)$.

We will also refer below to the following order relation between multi-indices.
Two multi-indices $\VEC{\alpha}$ and $\VEC{\beta}$ satisfy
$\VEC{\alpha} \leq \VEC{\beta}$ if $\alpha_j \leq \beta_j$ for all $j$,
and $\VEC{\alpha} < \VEC{\beta}$ if $\alpha_j \leq \beta_j$ for all
$j$ and there exists at least one $j$ such that $\alpha_j < \beta_j$.

\begin{prop} \label{ell_reg_prop1}
Suppose that $\displaystyle \phi \in \SS(\RR^n)$,
$\displaystyle \VEC{\alpha} \in \NN^n$, $\displaystyle \VEC{h}\in \RR^n$
and $s \in \RR$.  Then,
\begin{align*}
[\Delta^{\VEC{\alpha}}_{\VEC{h}}, \phi]: H^{s,2}(\RR^n) &\rightarrow 
H^{s-|\VEC{\alpha}|+1,2}(\RR^n) \\
f &\mapsto \Delta^{\VEC{\alpha}}_{\VEC{h}} \left( \phi f \right) -
\phi \Delta^{\VEC{\alpha}}_\VEC{h} f
\end{align*}
is a bounded operator and the bound is independent of $\VEC{h}$.
\end{prop}

\begin{proof}
The case $|\VEC{\alpha}|=0$ is trivial.
  
\stage{i} We consider the case $|\VEC{\alpha}|=1$.  We have that
\begin{align*}
[\Delta_{h_j\VEC{e}_j},\phi] f
&= \Delta_{h_j\VEC{e}_j} \left( \phi\, f \right) -
\phi \left( \Delta_{h_j\VEC{e}_j} f \right)
= \frac{1}{h_j} \left( \phi_{-h_j\VEC{e}_j} f_{-h_j\VEC{e}_j}
- \phi\, f \right)
- \frac{1}{h_j} \phi \,\left( f_{-h_j\VEC{e}_j} - f \right) \\
&= \frac{1}{h_j} \left( \phi_{-h_j\VEC{e}_j} f_{-h_j\VEC{e}_j}
- \phi \, f_{-h_j\VEC{e}_j} \right)
= \left( \Delta_{h_j\VEC{e}_j} \phi \right) f_{-h_j\VEC{e}_j} \ .
\end{align*}
Moreover, $\displaystyle f \to f_{-h_j\VEC{e}_j}$ is an isometry from
$\displaystyle H^{s,2}(\RR^n)$ onto itself. We leave the proof of this
claim as an exercise for the reader.  Recall that
$\displaystyle \left(f_{-h_j\VEC{e}_j}\right)^\wedge(\VEC{y}) = e^{i h_j\,y_j}
\hat{f}(\VEC{y})$.

Since $\displaystyle \Delta_{h_j\VEC{e}_j} \phi \in \SS(\RR^n)$, we have
from Proposition~\ref{sob_T_dd_wk2} that
\[
\left\| \left( \Delta_{h_j\VEC{e}_j} \phi \right) f_{-h_j\VEC{e}_j}
\right\|_{s,\rho}
\leq C_{j,h_j} \left\| f_{-h_j\VEC{e}_j} \right\|_{s,\rho}
= C_{j,h_j} \left\| f \right\|_{s,\rho}
\]
for a constant $C_{j,h_j}$.  We claim that the constant $C_{j,h_j}$ can be taken
to be independent of $j$ and $h_j$.  Recall that, in the proof of
Proposition~\ref{sob_T_dd_wk2}, $C_{j,h_j}$ satisfies 
\begin{equation} \label{DeltaBoundOpEq1}
\sup_{\VEC{z}\in\RR^n}
\int_{\RR^n} |K_{j,h_j}(\VEC{y},\VEC{z})|\dx{\rho(\VEC{y})} < C_{j,h_i} \quad
\text{and} \quad
\sup_{\VEC{y}\in\RR^n}
\int_{\RR^n} |K_{j,h_j}(\VEC{y},\VEC{z})|\dx{\rho(\VEC{z})} < C_{j,h_j} \ ,
\end{equation}
where
\[
K_{j,h_j}(\VEC{y},\VEC{z}) =(2\pi)^{-n/2} \left(1+\|\VEC{z}\|^2\right)^{-s}
\left(\Delta_{h_j\VEC{e}_j} \phi\right)^\wedge(\VEC{y} - \VEC{z}) \ .
\]
As we have shown in the proof of Proposition~\ref{ell_reg_prop3}, 
\begin{align*}
&\left| \left(\Delta_{h_j\VEC{e}_j} \phi\right)^\wedge(\VEC{z}) \right|
= \left| \frac{1}{h_j} \left( (\phi_{-h_j\VEC{e}_j})^\wedge(\VEC{z})
- \hat{\phi}(\VEC{z}) \right)\right|
= \left|\frac{1}{h_j} \left( e^{ih_jz_j} \hat{\phi}(\VEC{z})
- \hat{\phi}(\VEC{z}) \right)\right| \\
&\qquad = \left|\frac{1}{h_j} \left( e^{ih_jz_j} - 1 \right)\right|\,
\left| \hat{\phi}(\VEC{z})\right|
= \left| \frac{2i}{h_j} e^{ih_jz_j/2} \sin\left(\frac{h_jz_j}{2}\right)\right|\,
\left|\hat{\phi}(\VEC{z})\right|
\leq |z_j|\, \left|\hat{\phi}(\VEC{z})\right|
\leq |\VEC{z}| \, \left|\hat{\phi}(\VEC{z})\right|
\end{align*}
for all $\displaystyle \VEC{z} \in \RR^n$, $h_j \in \RR$ and $j$.
Following the steps of the proof of Proposition~\ref{sob_T_dd_wk2}, we
have that
\begin{align*}
\int_{\RR^n}|K_{j,h_j}(\VEC{y},\VEC{z})| \dx{\rho(\VEC{y})}
&= (2\pi)^{-n/2} \int_{\RR^n} \left(1+\|\VEC{z}\|^2\right)^{-s}
\left(1+\|\VEC{y}\|^2\right)^{s}
\left| \left(\Delta_{h_j\VEC{e}_j} \phi\right)^\wedge
(\VEC{y} - \VEC{z})\right|\dx{\VEC{y}} \\
&\leq 2^s (2\pi)^{-n/2} \int_{\RR^n} \left(1+\|\VEC{y}-\VEC{z}\|^2\right)^{s}
\left|\left(\Delta_{h_j\VEC{e}_j} \phi\right)^\wedge
(\VEC{y} - \VEC{z})\right|\dx{\VEC{y}} \\
&\leq 2^s (2\pi)^{-n/2} \int_{\RR^n} \left(1+\|\VEC{y}-\VEC{z}\|^2\right)^{s}
\,|\VEC{y}-\VEC{z}|\,
\left|\hat{\phi}(\VEC{y} -\VEC{z})\right| \dx{\VEC{y}} \\
&\leq 2^s (2\pi)^{-n/2} \int_{\RR^n} \left(1+\|\VEC{w}\|^2\right)^{s}
\, |\VEC{w}| \, \left|\hat{\phi}(\VEC{w})\right|\dx{\VEC{w}}
\equiv R_{s,1} < \infty
\end{align*}
for all $\displaystyle \VEC{z}\in \RR^n$, $h_j \in \RR$ and $j$
because $\displaystyle \hat{\phi} \in \SS(\RR^n)$ by
Proposition~\ref{distr_frr_tempT}.  Likewise,
\begin{align*}
&\int_{\RR^n} |K_{j,h_j}(\VEC{y}, \VEC{z})| \dx{\rho(\VEC{z})}
= (2\pi)^{-n/2}\int_{\RR^n} \left| \left(\Delta_{h_j\VEC{e}_j} \phi\right)^\wedge
(\VEC{y} - \VEC{z})\right|\dx{\VEC{z}} \\
&\qquad \leq (2\pi)^{-n/2} \int_{\RR^n} |\VEC{y}-\VEC{z}|\,
\left|\hat{\phi}(\VEC{y} -\VEC{z}) \right|\dx{\VEC{z}}
\leq  (2\pi)^{-n/2}\int_{\RR^n} |\VEC{w}|\,
\left|\hat{\phi}(\VEC{w})\right|\dx{\VEC{w}} \equiv R_{s,2} < \infty
\end{align*}
for all $\displaystyle \VEC{y}\in \RR^n$, $h_j \in \RR$ and $j$  because
$\displaystyle \hat{\phi} \in \SS(\RR^n)$ by Proposition~\ref{distr_frr_tempT}.

We have that (\ref{DeltaBoundOpEq1}) is satisfied for all $h_j$ and
$j$ with $C_{j,h_j}$ replaced by\\
$R_s = \max\{R_{s,1},R_{s,2}\}$.  Hence
\begin{equation} \label{DeltaBoundOpEq2}
\left\| \left[ \Delta_{h_j\VEC{e}_j} ,\phi \right] f \right\|_{s,\rho}
\leq R_s \left\| f \right\|_{s,\rho}
\end{equation}
for all $h_j \in \RR$ and $j$.  This proves the proposition for
$|\VEC{\alpha}|=1$.

\stage{ii}  We now consider the case where $|\VEC{\alpha}|>1$.
We have that $\displaystyle [\Delta^{\VEC{\alpha}}_{\VEC{h}}, \phi] f$
is a finite sum of terms of the form
$\displaystyle \Delta^{\VEC{\mu}}_{\VEC{h}} \left[
\Delta_{h_j \VEC{e}_j}, \phi\right] \Delta^{\VEC{\eta}}_{\VEC{h}}f$,
where $|\VEC{\mu}|+1+|\VEC{\eta}| = |\VEC{\alpha}|$; to be precise,
$\VEC{\eta}+\VEC{\mu} + \VEC{e}_j = \VEC{\alpha}$ and
$1 \leq j \leq n$.  For instance,
\begin{align}
\left[\Delta^2_{h_1\VEC{e}_1+h_2\VEC{e}_2} ,\phi\right] f
&=\left[\Delta_{h_2\VEC{e}_2} \Delta_{h_1\VEC{e}_1} ,\phi\right] f
= \Delta_{h_2\VEC{e}_2} \Delta_{h_1\VEC{e}_1} (\phi \, f)
- \phi\, \Delta_{h_2\VEC{e}_2} \Delta_{h_1\VEC{e}_1} f \nonumber \\
&= \Delta_{h_2\VEC{e}_2} \Delta_{h_1\VEC{e}_1} (\phi \, f)
- \Delta_{h_2\VEC{e}_2} \left( \phi \, \Delta_{h_1\VEC{e}_1} f\right)
+ \Delta_{h_2\VEC{e}_2} \left( \phi \, \Delta_{h_1\VEC{e}_1} f\right)
- \phi\, \Delta_{h_2\VEC{e}_2} \Delta_{h_1\VEC{e}_1} f \nonumber \\
&= \Delta_{h_2\VEC{e}_2} \left[ \Delta_{h_1\VEC{e}_1} ,\phi \right] f
+ \left[ \Delta_{h_2\VEC{e}_2}, \phi\right]
\left( \Delta_{h_1\VEC{e}_1} f\right) \ .
\label{DeltaBoundOpEq3}
\end{align}
This in fact proves the claim for $|\VEC{\alpha}|=2$.  We leave it to the
reader to prove by induction on $|\VEC{\alpha}|$ that the claim is true for
all $\VEC{\alpha}$.  Note that (\ref{DeltaBoundOpEq3}) can be generalized to
\[
\left[\Delta_{\VEC{h}}^{\VEC{\beta}} \Delta_{\VEC{h}}^{\VEC{\nu}} ,\phi\right] f
= \Delta_{\VEC{h}}^{\VEC{\beta}} \left[ \Delta_{\VEC{h}}^{\VEC{\nu}}
,\phi \right] f
+ \left[ \Delta_{\VEC{h}}^{\VEC{\beta}}, \phi\right]
\left( \Delta_{\VEC{h}}^{\VEC{\nu}} f\right)
\]
for any multi-indices $\VEC{\beta}$ and $\VEC{\nu}$.

We also have from Propositions~\ref{sob_DinDm} and \ref{ell_reg_prop3}
that
\begin{equation} \label{DeltaBoundOpEq4}
\left\| \Delta^{\VEC{\beta}}_{\VEC{h}} g \right\|_{t,\rho}
\leq \left\| \diff^{\VEC{\beta}} g \right\|_{t,\rho}
\leq \left\| g \right\|_{t+|\VEC{\beta}|,\rho}
\end{equation}
for all $t\in \RR$, $\displaystyle g \in H^{t,2}(\RR^n)$,
$\displaystyle \VEC{h} \in \RR^n$ and
$\displaystyle \VEC{\beta} \in \NN^n$ with $h_j \neq 0$ for
$\beta_j \neq 0$.  Hence,
\begin{align*}
\left\| \Delta^{\VEC{\mu}}_{\VEC{h}} \left[
\Delta_{h_j \VEC{e}_j}, \phi\right] \Delta^{\VEC{\eta}}_{\VEC{h}} f
\right\|_{s-|\VEC{\alpha}|+1,\rho}
&\leq \left\| \left[ \Delta_{h_j \VEC{e}_j}, \phi\right]
\Delta^{\VEC{\eta}}_{\VEC{h}}
f\right\|_{s-|\VEC{\alpha}|+|\VEC{\mu}|+1,\rho}
\leq R_{s-|\VEC{\alpha}|+|\VEC{\mu}|+1}
\left\| \Delta^{\VEC{\eta}}_{\VEC{h}} f
\right\|_{s-|\VEC{\alpha}|+|\VEC{\mu}|+1,\rho} \\
& \leq R_{s-|\VEC{\alpha}|+|\VEC{\mu}|+1}
\left\| f \right\|_{s-|\VEC{\alpha}|+|\VEC{\mu}|+|\VEC{\eta}|+1,\rho}
= R_{s-|\VEC{\alpha}|+|\VEC{\mu}|+1} \|f\|_{s,\rho} \ .
\end{align*}
We finally get that
$\displaystyle 
\left\| [\Delta^{\VEC{\alpha}}_{\VEC{h}}, \phi] f \right\|_{s-|\VEC{\alpha}|+1,\rho}
\leq C \|f\|_{s,\rho}$, where
$\displaystyle C = \sum_{|\VEC{\mu}|+|\VEC{\nu}|+1 = |\VEC{\alpha}|}
R_{s-|\VEC{\alpha}|+|\VEC{\mu}|+1}$.
\end{proof}

\begin{cor} \label{ell_reg_cor1}
Suppose that $L(\VEC{x},\diff)$ is the elliptic operator defined
in (\ref{ell_reg_ellOp}) with $\displaystyle a_{\VEC{\alpha}} \in \SS(\RR^n)$
for all $\VEC{\alpha}$.  Moreover, suppose that
$\displaystyle \VEC{\beta} \in \NN^n$, $\displaystyle \VEC{h}\in \RR^n$ and
$s \in \RR$.  Then the operator
\[
[\Delta^{\VEC{\beta}}_{\VEC{h}}, L(\VEC{x},\diff)]
= \Delta^{\VEC{\beta}}_{\VEC{h}} \circ L(\VEC{x},\diff) - L(\VEC{x},\diff)
\circ \Delta^{\VEC{\beta}}_{\VEC{h}}
\]
is a bounded operator from $\displaystyle H^{s,2}(\RR^n)$ into
$\displaystyle H^{s-k-|\VEC{\beta}|+1,2}(\RR^n)$ with the bound
independent of $h$.
\end{cor}

\begin{proof}
Since $\displaystyle \diff^{\VEC{\alpha}}$ commutes with
$\displaystyle \Delta_{\VEC{h}}^{\VEC{\beta}}$, we get that
\begin{align*}
[\Delta^{\VEC{\beta}}_{\VEC{h}}, L] f(\VEC{x}) &=
\Delta^{\VEC{\beta}}_{\VEC{h}}\bigg(\sum_{|\VEC{\alpha}|\leq k}
a_{\VEC{\alpha}}(\VEC{x}) \diff^{\VEC{\alpha}} f(\VEC{x}) \bigg) -
\sum_{|\VEC{\alpha}|\leq k} a_{\VEC{\alpha}}(\VEC{x}) \diff^{\VEC{\alpha}}
\left(\Delta^{\VEC{\beta}}_{\VEC{h}} f(\VEC{x}) \right) \\
&= \sum_{|\VEC{\alpha}|\leq k} \left( \Delta^{\VEC{\beta}}_{\VEC{h}}
\left( a_{\VEC{\alpha}}(\VEC{x}) \diff^{\VEC{\alpha}} f(\VEC{x}) \right)
- a_{\VEC{\alpha}}(\VEC{x}) \Delta^{\VEC{\beta}}_{\VEC{h}}
\left( \diff^{\VEC{\alpha}} f(\VEC{x}) \right) \right)
= \sum_{|\VEC{\alpha}|\leq k} \left[ \Delta^{\VEC{\beta}}_{\VEC{h}},
a_{\VEC{\alpha}} \right]) \diff^{\VEC{\alpha}} f(\VEC{x})
\end{align*}
for all $\displaystyle f \in H^{s,2}(\RR^n)$.  It follows from
Propositions~\ref{sob_DinDm} and \ref{ell_reg_prop1} that
\begin{align*}
\left\| [\Delta^{\VEC{\beta}}_{\VEC{h}}, L] f \right\|_{s-k-|\VEC{\beta}|+1,\rho}
&\leq \sum_{|\VEC{\alpha}|\leq k} \left\|\left[ \Delta^{\VEC{\beta}}_{\VEC{h}},
a_{\VEC{\alpha}} \right])
\diff^{\VEC{\alpha}} f \right\|_{s-k-|\VEC{\beta}|+1,\rho}
\leq \sum_{|\VEC{\alpha}|\leq k} C_{a_{\VEC{\alpha}},\VEC{\beta}}
\left\| \diff^{\VEC{\alpha}} f \right\|_{s-k,\rho} \\
&\leq \sum_{|\VEC{\alpha}|\leq k} C_{a_{\VEC{\alpha}},\VEC{\beta}}
\left\| f \right\|_{s-k+|\VEC{\alpha}|,\rho}
\leq \bigg( \sum_{|\VEC{\alpha}|\leq k} C_{a_{\VEC{\alpha}},\VEC{\beta}}\bigg)
\left\| f \right\|_{s,\rho} \ ,
\end{align*}
where the constants $C_{a_\VEC{\alpha},\VEC{\beta}}$ are independent
of $\VEC{h}$.
\end{proof}

Using Theorem~\ref{sob_Rp_ext}, we may prove the following corollary
to Proposition~\ref{ell_reg_prop3}.  Recall that
$\displaystyle \RR^n_+ = \{ (x_1,x_2,\ldots,x_n) \in \RR^n : x_n>0 \}$
and $\displaystyle \RR^n_- = \{ (x_1,x_2,\ldots,x_n) \in \RR^n : x_n<0 \}$.

\begin{cor} \label{ell_reg_cor3}
Let $\displaystyle \Omega_r = \RR^n_+ \cap B_r(\VEC{0})$ for some $r>0$.
If the support of $\displaystyle f\in H^{k,2}(\RR^n)$ is a subset of
$\displaystyle \Omega_{r,0} = (\RR^n \setminus \RR^n_-) \cap
B_r(\VEC{0})$ (Figure~\ref{ellOmegaR}) and
$\VEC{\alpha} \in \NN^n$ is a multi-index with $\alpha_n=0$, then
$\displaystyle \diff^{\VEC{\alpha}} f \in H^{k,2}(\Omega_r)$ if and only if
\[
A_{f,\VEC{\alpha}} \equiv \limsup_{\VEC{h}\rightarrow \VEC{0}}
\|\Delta^{\VEC{\alpha}}_{\VEC{h}} f\|_{k,\Omega_r} < \infty \ .
\]
In particular, if $\displaystyle \diff^{\VEC{\alpha}} f \in H^{k,2}(\Omega_r)$,
then there exists a constant $C$, independent of $f$ and $r$, such that 
$\displaystyle \left\| \diff^{\VEC{\alpha}} f \right\|_{k,\Omega_r} \leq
C A_{f,\VEC{\alpha}}$.
\end{cor}

\begin{proof}
According to Theorem~\ref{sob_Rp_ext}, there exists a strong
$k$-extension operator $E$ on $\displaystyle \RR^n_+ $ and a strong
$(k-|\VEC{\alpha}|)$-extension operator $E_{\VEC{\alpha}}$ on
$\displaystyle \RR^n_+$ such that
$\displaystyle \diff^{\VEC{\alpha}} (E(g)) =
E_{\VEC{\alpha}}(\diff^{\VEC{\alpha}} g)$
for all $\displaystyle g \in H^{k,2}(\RR^n_+)$.
Since $\alpha_n=0$, a quick look at the proof of
Theorem~\ref{sob_Rp_ext} shows that $E_{\VEC{\alpha}} = E$.
Moreover, we also see that
$\displaystyle E\left( \Delta^{\VEC{\alpha}}_{\VEC{h}} g\right) =
\Delta^{\VEC{\alpha}}_{\VEC{h}} E(g)$
for all $\displaystyle g \in H^{k,2}(\RR^n_+)$ because $\alpha_n=0$.

We have that $\displaystyle E:H^{k,2}(\RR^n_+) \to H^{k,2}(\RR^n)$ is a
bounded operator and $E(g) = g$ on $\displaystyle \RR^n_+$ for all
$\displaystyle g \in H^{k,2}(\RR^n_+)$.  So, there exists a
positive constant $C_1$ such that
$\displaystyle \left\| E(g) \right\|_{k,2,\RR^n} \leq C_1
\left\| g \right\|_{k,2,\RR^n_+}$ for all $\displaystyle g \in H^{k,2}(\RR^n_+)$.

Since the norms $\displaystyle \left\|\cdot\right\|_{k,2,\RR^n}$ and 
$\displaystyle \left\|\cdot\right\|_{k,\rho}$ are equivalent, there
exist positive constants $C_2$ and $C_3$ such that
\[
C_3 \left\|g \right\|_{k,2,\RR^n} \leq \left\|g \right\|_{k,\rho}
\leq C_2 \left\|g \right\|_{k,2,\RR^n}
\]
for all $\displaystyle g \in H^{k,2}(\RR^n)$.

\stage{i} Suppose that $\displaystyle \diff^{\VEC{\alpha}} f
\in H^{k,2}(\Omega_r)$.
Since $\supp f \subset \Omega_{r,0}$, we have that\\
$\displaystyle \left\|\diff^{\VEC{\alpha}} f\right\|_{k,2,\RR^n_+}
= \displaystyle \left\|\diff^{\VEC{\alpha}} f \right\|_{k,2,\Omega_r} < \infty$.
Hence,
\[
\left\| \diff^{\VEC{\alpha}}a E(f) \right\|_{k,\rho} 
= \left\| E(\diff^{\VEC{\alpha}} f)\right\|_{k,\rho}
\leq C_2 \left\| E(\diff^{\VEC{\alpha}} f)\right\|_{k,2,\RR^n} \leq C_1 C_2
\left\| \diff^{\VEC{\alpha}} f\right\|_{k,2,\RR^n_+} < \infty \ .
\]
Thus, according to Proposition~\ref{ell_reg_prop3},
\[
\limsup_{\VEC{h}\rightarrow \VEC{0}} \big\|\Delta^{\VEC{\alpha}}_{\VEC{h}} E(f)
\big\|_{k,\rho} = \big\|\diff^{\VEC{\alpha}} E(f)\big\|_{k,\rho} < \infty \ .
\]
It follows that
\begin{align*}
A_{f,\VEC{\alpha}} &\equiv \limsup_{\VEC{h}\rightarrow \VEC{0}}
\left\|\Delta^{\VEC{\alpha}}_{\VEC{h}} f\right\|_{k,2,\Omega_r}
= \limsup_{\VEC{h}\rightarrow \VEC{0}}
\left\| E(\Delta^{\VEC{\alpha}}_{\VEC{h}} f)\right \|_{k,2,\Omega_r}
\leq \limsup_{\VEC{h}\rightarrow \VEC{0}}
\left\| E(\Delta^{\VEC{\alpha}}_{\VEC{h}} f)\right \|_{k,2,\RR^n} \\
&\leq C_3^{-1} \limsup_{\VEC{h}\rightarrow \VEC{0}}
\left\| E(\Delta^{\VEC{\alpha}}_{\VEC{h}} f)\right \|_{k,\rho}
< \infty \ .
\end{align*}

\stage{ii} Suppose $A_{f,\VEC{\alpha}} < \infty$.
Since $\supp f$ is a compact subset of the bounded set $\Omega_{r,0}$,
there exists $\delta >0$ such that
$|\VEC{\alpha}| \VEC{h} + \supp f \subset \Omega_{r,0}$
for $\|\VEC{h}\| < \delta$ with $h_n = 0$.  Thus,
$\displaystyle \supp \Delta^{\VEC{\alpha}}_{\VEC{h}} f \subset \Omega_{r,0}$ for
$\|\VEC{h}\| < \delta$ without constraints on $h_n$ since we assume
that $\alpha_n = 0$.  Hence,
\begin{align}
&\limsup_{\VEC{h}\rightarrow \VEC{0}}
  \left\| \Delta^{\VEC{\alpha}}_{\VEC{h}} E(f)\right \|_{k,\rho}
= \limsup_{\VEC{h}\rightarrow \VEC{0}}
  \left\| E(\Delta^{\VEC{\alpha}}_{\VEC{h}} f)\right \|_{k,\rho}
\leq C_2 \limsup_{\VEC{h}\rightarrow \VEC{0}}
\left\| E(\Delta^{\VEC{\alpha}}_{\VEC{h}} f)\right \|_{k,2,\RR^n} \nonumber \\
&\qquad \leq C_1 C_2 \limsup_{\VEC{h}\rightarrow \VEC{0}}
\left\| \Delta^{\VEC{\alpha}}_{\VEC{h}} f\right \|_{k,2,\RR^n_+}
= C_1 C_2 \limsup_{\VEC{h}\rightarrow \VEC{0}}
\left\| \Delta^{\VEC{\alpha}}_{\VEC{h}} f\right \|_{k,2,\Omega_r}
= C_1 C_2 A_{f,\VEC{\alpha}} < \infty \ ,
\label{ellRegCor3Eq1}
\end{align}
where we have used the fact that
$\displaystyle \supp \Delta^{\VEC{\alpha}}_{\VEC{h}} f \subset \Omega_{r,0}$
for $\|\VEC{h}\| < \delta$ for the second to last equality in the
previous expression.
It follows from Proposition~\ref{ell_reg_prop3} that 
$\displaystyle \diff^{\VEC{\alpha}} E(f) \in H^{k,2}(\RR^n)$.
Since $\displaystyle \Omega_r \subset \RR^n$, we have that
$\displaystyle \diff^{\VEC{\alpha}} E(f) \in H^{k,2}(\Omega_r)$.  Since
$E(f) = f$ on $\displaystyle \RR^n_+ \supset \Omega_r$, we have that
$\displaystyle \diff^{\VEC{\alpha}} f \in H^{k,2}(\Omega_r)$.

\stage{iii} It also follows from Proposition~\ref{ell_reg_prop3}
that $\displaystyle \left\|\diff^{\VEC{\alpha}} E(f)\right\|_{k,\rho}
= \limsup_{\VEC{h}\rightarrow \VEC{0}}
\left\| \Delta^{\VEC{\alpha}}_{\VEC{h}} E(f)\right \|_{k,\rho}$.
Hence, we get from (\ref{ellRegCor3Eq1}) that
\begin{align*}
\left\|\diff^{\VEC{\alpha}} f\right\|_{k,2,\Omega_r}
&= \left\| \diff^{\VEC{\alpha}} E(f) \right\|_{k,2,\RR^n_+}
\leq \left\|\diff^{\VEC{\alpha}} E(f)\right\|_{k,2,\RR^n}
\leq C_3^{-1} \left\|\diff^{\VEC{\alpha}} E(f)\right\|_{k,\rho} \\
&= C_3^{-1} \limsup_{\VEC{h}\rightarrow \VEC{0}}
\left\| \Delta^{\VEC{\alpha}}_{\VEC{h}} E(f)\right \|_{k,\rho}
\leq C_1 C_2 C_3^{-1} A_{f,\VEC{\alpha}}
= C A_{f,\VEC{\alpha}} \ ,
\end{align*}
where $C= C_1 C_2 C_3^{-1}$.  Note that $C$ is independent of $f$ and $r$
because $C_1$ depends only on $E$, and $C_2$ and $C_3$ depend only on
the equivalence between the norms
$\displaystyle \left\|\cdot\right\|_{k,2,\RR^n}$ and 
$\displaystyle \left\|\cdot\right\|_{k,\rho}$.
\end{proof}

\pdfF{elliptic/ell_omegaR}{Support of $f$ in Corollary~\ref{ell_reg_cor3}}{
Support of $f$ in Corollary~\ref{ell_reg_cor3}.  We have set
$\displaystyle \breve{\VEC{x}} = (x_1,x_2,\ldots,x_{n-1})$.}{ellOmegaR}

The corollary to the next proposition will be needed to study the
regularity of solutions at the boundary in Section~\ref{ell_sect_RB}.

\begin{prop} \label{ell_reg_prop4}
Suppose that $\displaystyle \phi \in \SS(\RR^n)$,
$\displaystyle \VEC{\alpha} \in \NN^n$, $\displaystyle \VEC{h}\in \RR^n$
and $s \in \RR$.  Then,
there exists a positive constant $C$ independent of $\VEC{h}$ such that
\begin{equation} \label{ellRegProp4Eq1}
\| [\Delta^{\VEC{\alpha}}_{\VEC{h}}, \phi] f \|_{s,\rho} \leq C
\sum_{\VEC{\beta}<\VEC{\alpha}} \| \diff^{\VEC{\beta}} f\|_{s,\rho}
\end{equation}
for all $\displaystyle f \in H^{s,2}(\RR^n)$.
The possibility that the right hand side be $\infty$ is not ruled out.
\end{prop}

\begin{proof}
The case $|\VEC{\alpha}|=0$ is trivial.
The case $|\VEC{\alpha}|=1$ follows word for word from (i) in the proof of
Proposition~\ref{ell_reg_prop1}.  Namely, we get that
\[
\left\| \left[ \Delta_{h_j\VEC{e}_j} ,\phi \right] f \right\|_{s,2,\rho}
\leq R_{\phi,s} \left\| f \right\|_{s,2,\rho}
\]
for all $h_j \in \RR$ and $j$, where $R_{\phi,s}$ is a positive constant that
depends on $s\in \RR$ and $\displaystyle \phi \in \DD(\RR^n)$ only.  We get
(\ref{ellRegProp4Eq1}) for $|\VEC{\alpha}|=1$ if $C=R_{\phi,s}$.

We now consider the case $|\VEC{\alpha}|>1$.  As was mentioned
in (ii) of the prove of Proposition~\ref{ell_reg_prop1}, we have that
$\displaystyle [\Delta^{\VEC{\alpha}}_{\VEC{h}}, \phi] f$
is a finite sum of terms of the form
$\displaystyle \Delta^{\VEC{\eta}}_{\VEC{h}} \left[
\Delta_{h_j \VEC{e}_j}, \phi\right] \Delta^{\VEC{\mu}}_{\VEC{h}} f$,
where $\VEC{\eta}+\VEC{\mu}+\VEC{e}_j = \VEC{\alpha}$ and $1\leq j \leq n$.

We have from Proposition~\ref{ell_reg_prop3} that
$\displaystyle \left\| \Delta^{\VEC{\beta}}_{\VEC{h}} g \right\|_{t,\rho}
\leq \left\| \diff^{\VEC{\beta}} g \right\|_{t,\rho}$
for all $t\in \RR$, $\displaystyle g \in H^{t,2}(\RR^n)$,
$\displaystyle \VEC{h} \in \RR^n$ and $\displaystyle \VEC{\beta} \in \NN^n$
with $h_j \neq 0$ for $\beta_j \neq 0$.

We also have that
\begin{align*}
\diff^{\VEC{\eta}}\left[ \Delta^{\VEC{\beta}}_{\VEC{y}}, \phi\right| g
&= \diff^{\VEC{\eta}} \left( \Delta^{\VEC{\beta}}_{\VEC{y}} (\phi g) - \phi
\Delta^{\VEC{\beta}}_{\VEC{y}} u \right)
= \Delta^{\VEC{\beta}}_{\VEC{y}} \diff^{\VEC{\eta}}(\phi g)
- \diff^{\VEC{\eta}} \left(\phi
\Delta^{\VEC{\beta}}_{\VEC{y}} u \right) \\
&= \Delta^{\VEC{\beta}}_{\VEC{y}} \left(\sum_{\VEC{\zeta}+\VEC{\xi} = \VEC{\eta}}
\binom{\VEC{\eta}}{\VEC{\xi}}
\diff^{\VEC{\zeta}} \phi \, \diff^{\VEC{\xi}} g \right)
- \left(\sum_{\VEC{\zeta}+\VEC{\xi} = \VEC{\eta}}
\binom{\VEC{\eta}}{\VEC{\xi}} \diff^{\VEC{\zeta}} \phi \,
\diff^{\VEC{\xi}} \left( \Delta^{\VEC{\beta}}_{\VEC{y}} g \right) \right) \\
&= \sum_{\VEC{\zeta}+\VEC{\xi} = \VEC{\eta}} \binom{\VEC{\eta}}{\VEC{\xi}} \Big(
\Delta^{\VEC{\beta}}_{\VEC{y}} \left( \diff^{\VEC{\zeta}} \phi \,
\diff^{\VEC{\xi}} g \right)
- \diff^{\VEC{\zeta}} \phi \, \left( \Delta^{\VEC{\beta}}_{\VEC{y}}
\diff^{\VEC{\xi}} g \right) \Big)
= \sum_{\VEC{\zeta}+\VEC{\xi} = \VEC{\eta}} \binom{\VEC{\eta}}{\VEC{\xi}}
\left[ \Delta^{\VEC{\beta}}_{\VEC{y}} , \diff^{\VEC{\zeta}} \phi \right]
\diff^{\VEC{\xi}} g
\end{align*}
for all multi-indices $\VEC{\beta}$ and $\VEC{\eta}$ in $\NN^n$,
and $\displaystyle g \in H^{|\VEC{\eta}|}(\RR^n)$. The previous
equality is in the sense of distributions.

The inequality in (\ref{ellRegProp4Eq1}) is obviously true if
$\displaystyle \diff^{\VEC{\beta}} f \not\in H^{s,2}(\RR^n)$ for some
$\VEC{\beta}$.
The right hand side of (\ref{ellRegProp4Eq1}) is then $\infty$.  So, we may
assume that $\displaystyle \diff^{\VEC{\beta}} f \in H^{s,2}(\RR^n)$ for all
$\VEC{\beta} < \VEC{\alpha}$.

Hence,
\begin{align*}
&\left\| \Delta^{\VEC{\eta}}_{\VEC{h}} \left[
\Delta_{h_j \VEC{e}_j}, \phi\right] \Delta^{\VEC{\mu}}_{\VEC{h}} f
\right\|_{s,\rho}
\leq \left\| \diff^{\VEC{\eta}} \left[ \Delta_{h_j \VEC{e}_j}, \phi\right]
\Delta^{\VEC{\mu}}_{\VEC{h}} f\right\|_{s,\rho} \\
&\qquad \leq \sum_{\VEC{\xi}+\VEC{\zeta} = \VEC{\eta}}
\binom{\VEC{\eta}}{\VEC{\xi}}
\Big\| \left[ \Delta_{h_j\VEC{e}_j} , \diff^{\VEC{\zeta}} \phi \right]
\diff^{\VEC{\xi}} \left(\Delta^{\VEC{\mu}}_{\VEC{h}} f\right) \Big\|_{s,\rho}
\leq \sum_{\VEC{\xi}+\VEC{\zeta} = \VEC{\eta}} \binom{\VEC{\eta}}{\VEC{\xi}}
R_{\diff^{\VEC{\zeta}} \phi,s}
\Big\| \diff^{\VEC{\xi}} \left(\Delta^{\VEC{\mu}}_{\VEC{h}} f\right)
\Big\|_{s,\rho} \\
&\qquad =\sum_{\VEC{\xi}+\VEC{\zeta} = \VEC{\eta}}
\binom{\VEC{\eta}}{\VEC{\xi}} R_{\diff^{\VEC{\zeta}} \phi,s}
\Big\| \Delta^{\VEC{\mu}}_{\VEC{h}}
\left(\diff^{\VEC{\xi}} f\right) \Big\|_{s,\rho}
\leq \sum_{\VEC{\xi}+\VEC{\zeta} = \VEC{\eta}} \binom{\VEC{\eta}}{\VEC{\xi}}
R_{\diff^{\VEC{\zeta}} \phi,s}
\Big\| \left(\diff^{\VEC{\xi}+\VEC{\mu}} f\right) \Big\|_{s,\rho} \ ,
\end{align*}
where the last inequality comes from Proposition~\ref{ell_reg_prop3}.
We then get that
\begin{align*}
\left\| \left[ \Delta_{h_j\VEC{e}_j} ,\psi \right] f \right\|_{s,\rho}
&\leq \sum_{j=1}^n \, \sum_{\VEC{\eta}+\VEC{\mu}+\VEC{e}_j \leq \VEC{\alpha}}
\left\| \Delta^{\VEC{\eta}}_{\VEC{h}} \left[ \Delta_{h_j \VEC{e}_j},
\phi\right] \Delta^{\VEC{\mu}}_{\VEC{h}} f \right\|_{s,\rho} \\
&\leq \sum_{j=1}^n \, \sum_{\VEC{\eta}+\VEC{\mu}+\VEC{e}_j \leq \VEC{\alpha}} \,
\sum_{\VEC{\xi}+\VEC{\zeta} = \VEC{\eta}} \binom{\VEC{\eta}}{\VEC{\xi}}
R_{\diff^{\VEC{\zeta}} \phi,s}
\Big\| \left(\diff^{\VEC{\xi}+\VEC{\mu}} f\right) \Big\|_{s,\rho} \\
&= \sum_{j=1}^n \, \sum_{\substack{\VEC{\mu} \leq \VEC{\alpha} - \VEC{e}_j \\
\text{ if } \alpha_j >0}} \,
\sum_{\VEC{\xi} \leq \VEC{\eta} = \VEC{\alpha} - \VEC{e}_j - \VEC{\mu}}
\binom{\VEC{\eta}}{\VEC{\xi}} R_{\diff^{\VEC{\eta}- \VEC{\xi}} \phi,s}
\Big\| \left(\diff^{\VEC{\xi}+\VEC{\mu}} f\right) \Big\|_{s,\rho} \ .
\end{align*}

We have that $\VEC{\xi} + \VEC{\mu} \leq \VEC{\alpha} - \VEC{e}_j <
\VEC{\alpha}$.  Thus, we get (\ref{ellRegProp4Eq1}) if $C$ is the maximum of\\
$\displaystyle \binom{\VEC{\eta}}{\VEC{\xi}}
R_{\diff^{\VEC{\eta} - \VEC{\xi}} \phi,s}$ for
$\VEC{\mu} \leq \VEC{\alpha} - \VEC{e}_j$ if $\alpha_j >0$ and
$\VEC{\xi} \leq \VEC{\eta} = \VEC{\alpha} - \VEC{e}_j - \VEC{\mu}$.
\end{proof}

\begin{cor} \label{ell_reg_cor2}
Let $\displaystyle \Omega_r = \RR^n_+ \cap B_r(\VEC{0})$ for some $r>0$.
Suppose that $\displaystyle \phi \in \SS(\RR^n)$
and that $\VEC{\alpha}$ is a multi-index with $k = |\VEC{\alpha}|$ and
$\alpha_n=0$.  There exists a constant $C$ such that, for all $0 < \delta <1$,
\begin{equation} \label{ellRegCor2Eq1}
\| [\Delta^{\VEC{\alpha}}_{\VEC{h}}, \phi] f \|_{k,2,\Omega_r} \leq C
\sum_{\VEC{\beta}<\VEC{\alpha}} \| \diff^{\VEC{\beta}} f\|_{k,2,\Omega_r}
\end{equation}
for $\displaystyle f \in H^{k,2}(\Omega_r)$ with
$\displaystyle \supp f \subset \Omega_{\delta r,0}
= (\RR^n \setminus \RR^n_-) \cap B_{\delta r}(\VEC{0})$ and
$\|\VEC{h}\|<(1-\delta)r$.
\end{cor}

\begin{proof}
According to Theorem~\ref{sob_Rp_ext}, there exists a strong
$k$-extension operator $E$ on $\displaystyle \RR^n_+ $ and a strong
$(k-|\VEC{\beta}|)$-extension operator $E_{\VEC{\beta}}$ on
$\displaystyle \RR^n_+$ such that
$\displaystyle \diff^{\VEC{\beta}} (E(g)) = E_{\VEC{\beta}}(\diff^{\VEC{\beta}} g)$
for all $\displaystyle g \in H^{k,2}(\RR^n_+)$, where
$|\VEC{\beta}|\leq k$.  Since $\alpha_n=0$, a quick look at the proof of
Theorem~\ref{sob_Rp_ext} shows that $E_{\VEC{\beta}} = E$.
Moreover, we also see that 
$\displaystyle E\left( \Delta^{\VEC{\alpha}}_{\VEC{h}} f\right) =
\Delta^{\VEC{\alpha}}_{\VEC{h}} E(f)$ because $\alpha_n=0$.

Since $E$ is a strong $k$-extension operator $E$ on
$\displaystyle \RR^n_+ $, we
have that $\displaystyle E:H^{k,2}(\RR^n_+) \to H^{k,2}(\RR^n)$ is a
bounded operator and $E(g) = g$ on $\displaystyle \RR^n_+$ for all
$\displaystyle g \in H^{k,2}(\RR^n_+)$.  So, there exists a
positive constant $C_1$ such that
$\displaystyle \left\| E(g) \right\|_{k,2,\RR^n} \leq C_1
\left\| g \right\|_{k,2,\RR^n_+}$ for all $\displaystyle g \in H^{k,2}(\RR^n_+)$.

As we have done several times in previous proof,
we use the equivalence of the norms
$\displaystyle \left\|\cdot\right\|_{k,2,\RR^n}$ and 
$\displaystyle \left\|\cdot\right\|_{k,\rho}$.  Suppose that
\[
C_3 \left\|g \right\|_{k,2,\RR^n} \leq \left\|g \right\|_{k,\rho}
\leq C_2 \left\|g \right\|_{k,2,\RR^n}
\]
for all $\displaystyle g \in H^{k,2}(\RR^n)$ and some positive
constants $C_2$ and $C_3$.

From Proposition~\ref{ell_reg_prop4}, there exists a positive constant $C_0$
independent of $\VEC{h}$ such that
\[
\| [\Delta^{\VEC{\alpha}}_{\VEC{h}}, \phi] g \|_{k,\rho} \leq C_0
\sum_{\VEC{\beta}<\VEC{\alpha}} \| \diff^{\VEC{\beta}} g\|_{k,\rho}
\]
for all $\displaystyle g \in H^{k,2}(\RR^n)$.

The inequality in (\ref{ellRegCor2Eq1}) is obviously true if
$\displaystyle \diff^{\VEC{\beta}} f \not\in H^{k,2}(\Omega_r)$ for some
$\VEC{\beta}$.  The right hand side of (\ref{ellRegCor2Eq1}) is then
$\infty$.  So, we may assume that
$\displaystyle \diff^{\VEC{\beta}} f \in H^{k,2}(\Omega_r)$ for all
$\VEC{\beta} < \VEC{\alpha}$.
Since $\supp f$ is a compact subset of the bounded set
$\Omega_{\delta r,0}$,
we have that
\[
\left\|E(\diff^{\VEC{\beta}} f)\right\|_{k,\rho}
\leq C_2 \left\|E(\diff^{\VEC{\beta}} f)\right\|_{k,2,\RR^n}
\leq C_1 C_2 \left\|\diff^{\VEC{\beta}} f\right\|_{k,2,\RR^n_+}
= C_1 C_2 \left\|\diff^{\VEC{\beta}} f\right\|_{k,2,\Omega_r}
< \infty
\]
for all $\VEC{\beta} < \VEC{\alpha}$.  Thus,
$\displaystyle E(\diff^{\VEC{\beta}} f) \in H^{k,2}(\RR^n)$
for all $\VEC{\beta} < \VEC{\alpha}$.

Moreover, since $\supp f$ is a compact subset of the bounded set
$\Omega_{\delta r,0}$, we have that
$|\VEC{\alpha}| \VEC{h} + \supp f \subset \Omega_{r,0}$
for $\|\VEC{h}\| < (1-\delta)r$ with $h_n = 0$.    Thus,
$\displaystyle \supp \Delta^{\VEC{\alpha}}_{\VEC{h}} f \subset \Omega_{r,0}$ for
$\|\VEC{h}\| < (1-\delta)r$ without constraints on $h_n$ since we assume
that $\alpha_n = 0$.  Hence,
$\displaystyle \supp [\Delta^{\VEC{\alpha}}_{\VEC{h}}, \phi] f
\subset \Omega_{r,0}$.

Hence, for $\|\VEC{h}\| < (1-\delta)r$, we have that
\begin{align*}
&\| [\Delta^{\VEC{\alpha}}_{\VEC{h}}, \phi] f \|_{k,2,\Omega_r}
= \| [\Delta^{\VEC{\alpha}}_{\VEC{h}}, \phi] E(f) \|_{k,2,\RR^n_+}
\leq \| [\Delta^{\VEC{\alpha}}_{\VEC{h}}, \phi] E(f) \|_{k,2,\RR^n}
\leq C_3^{-1} \| [\Delta^{\VEC{\alpha}}_{\VEC{h}}, \phi] E(f) \|_{k,\rho} \\
&\qquad \leq C_0 C_3^{-1} \sum_{\VEC{\beta}<\VEC{\alpha}}
\| \diff^{\VEC{\beta}} E(f) \|_{k,\rho}
\leq C_0 C_2 C_3^{-1} \sum_{\VEC{\beta}<\VEC{\alpha}} \|
\diff^{\VEC{\beta}} E(f)  \|_{k,2,\RR^n}
= C_0 C_2 C_3^{-1} \sum_{\VEC{\beta}<\VEC{\alpha}} \
E(\diff^{\VEC{\beta}} f) \|_{k,2,\RR^n}\\
&\qquad \leq C_0 C_1 C_2 C_3^{-1}
\sum_{\VEC{\beta}<\VEC{\alpha}} \| \diff^{\VEC{\beta}} f \|_{k,2,\RR^n_+}
=  C_0 C_1 C_2 C_3^{-1} \sum_{\VEC{\beta}<\VEC{\alpha}}
\| \diff^{\VEC{\beta}} f \|_{k,2,\Omega_r} \ .
\end{align*}
This gives (\ref{ellRegCor2Eq1}) with $C = C_0 C_1 C_2 C_3^{-1}$.
\end{proof}

\begin{rmk}
The proof of Corollary~\ref{ell_reg_cor2} can be slightly modified
to show that there exists a constant $C$ such that, for all $0 < \delta <1$, 
\[
  \| [\Delta^{\VEC{\alpha}}_{\VEC{h}}, \phi] f \|_{k,\Omega_r}
\leq C \| f \|_{k+|\VEC{\alpha}|-1,\Omega_r}
\]
for $f \in H^{k+|\VEC{\alpha}|-1,2}(\Omega_r)$ with
$\displaystyle \supp f \subset \Omega_{\delta r,0}
= (\RR^n \setminus \RR^n_-) \cap B_{\delta r}(\VEC{0})$ and
$\|\VEC{h}\|<(1-\delta)r$.
Note that this conclusion is not as strong as the conclusion of
Corollary~\ref{ell_reg_cor2} because $\VEC{\beta} < \VEC{\alpha}$ is more
restrictive than $|\VEC{\beta}| < |\VEC{\alpha}|$ as we have in the
definition of $\displaystyle \| \cdot \|_{k+|\VEC{\alpha}|-1,2,\Omega_r}$.

According to Theorem~\ref{sob_Rp_Totalext}, there exists a total
extension operator $E$ on $\displaystyle \RR^n_+ $.  From a quick look
at the definition of $E$ given in the sketch of the proof of
Theorem~\ref{sob_Rp_Totalext}, we see
that $\displaystyle E\left( \Delta^{\VEC{\alpha}}_{\VEC{h}} f\right) =
\Delta^{\VEC{\alpha}}_{\VEC{h}} E(f)$ because $\alpha_n=0$.

Since $E$ is a total extension operator $E$ on $\displaystyle \RR^n_+ $, we
have for every $q \in \NN$ that
$\displaystyle E:H^{q,2}(\RR^n_+) \to H^{q,2}(\RR^n)$ is a
bounded operator and $E(g) = g$ on $\RR^n_+$ for all
$\displaystyle g \in H^{q,2}(\RR^n_+)$.  It can be
shown that there exists a positive constant $C_1$ independent of $q$ such that
$\displaystyle \left\| E(g) \right\|_{q,2,\RR^n} \leq C_1
\left\| g \right\|_{q,2,\RR^n_+}$ for all $\displaystyle g \in H^{q,2}(\RR^n_+)$.

To prove our claim, we will need the equivalence of the norms
$\displaystyle \left\|\cdot\right\|_{q,2,\RR^n}$ and 
$\displaystyle \left\|\cdot\right\|_{q,\rho}$ for $q=k$ and
$q=k+|\VEC{\alpha}|-1$.
Namely,
\[
C_{q,3} \left\|g \right\|_{q,2,\RR^n} \leq \left\|g \right\|_{k,\rho}
\leq C_{q,2} \left\|g \right\|_{q,2,\RR^n}
\]
for all $\displaystyle g \in H^{q,2}(\RR^n)$ and some positive
constants $C_{q,2}$ and $C_{q,3}$.

From Proposition~\ref{ell_reg_prop1}, there exists a positive constant $C_0$
independent of $\VEC{h}$ such that
\[
\left\| [\Delta^{\VEC{\alpha}}_{\VEC{h}}, \phi] g \right\|_{k,\rho}
\leq C_0 \| g \|_{k+|\VEC{\alpha}|-1,\rho}
\]
for all $g \in H^{k+|\VEC{\alpha}|-1,2}(\RR^n)$.

As we stated in the proof of Corollary~\ref{ell_reg_cor2},
we have that $|\VEC{\alpha}| \VEC{h} + \supp f \subset \Omega_{r,0}$
for $\|\VEC{h}\| < (1-\delta)r$ with $h_n = 0$.    Thus,
$\displaystyle \supp \Delta^{\VEC{\alpha}}_{\VEC{h}} f \subset \Omega_{r,0}$ for
$\|\VEC{h}\| < \delta$ without constraints on $h_n$ since we assume
that $\alpha_n = 0$.  Hence,
$\displaystyle \supp [\Delta^{\VEC{\alpha}}_{\VEC{h}}, \phi] f
\subset \Omega_{r,0}$
and
\begin{align*}
&\| [\Delta^{\VEC{\alpha}}_{\VEC{h}}, \phi] f \|_{k,2,\Omega_r}
\leq \| [\Delta^{\VEC{\alpha}}_{\VEC{h}}, \phi] E(f) \|_{k,2,\RR^n}
\leq C_{k,3}^{-1} \| [\Delta^{\VEC{\alpha}}_{\VEC{h}}, \phi] E(f) \|_{k,\rho}
\leq C_0 C_{k,3}^{-1} \| E(f) \|_{k+|\VEC{\alpha}|-1,\rho} \\
&\qquad \leq C_0 C_1 C_{k+|\VEC{\alpha}|-1,3}^{-1}
\| f \|_{k+|\VEC{\alpha}|-1,2,\RR^n_+}
= C_0 C_1 C_{k+|\VEC{\alpha}|-1,3}^{-1} \| f \|_{k+|\VEC{\alpha}|-1,2,\Omega_r}
= C \| f \|_{k+|\VEC{\alpha}|-1,2,\Omega_r} \ ,
\end{align*}
where $C = C_0 C_1 C_{k+|\VEC{\alpha}|-1,,3}^{-1}$.
\end{rmk}

\subsection{Local Regularity} \index{Local Regularity}

This subject is also called {\bfseries internal
regularity}\index{Internal Regularity} by some authors.  However,
local regularity seems more appropriate because the results
in this section address the regularity of weak solutions of elliptic
equations in local part of the domain $\Omega$ of the equations.

To proof the main result of this section, we need a few preliminary results.

\begin{prop} \label{ell_reg_in}
Suppose that $\displaystyle \Omega \subset \RR^n$ is a bounded open
set and that $L(\VEC{x},\diff)$ defined in (\ref{ell_reg_ellOp}) is
elliptic in a open neighbourhood of $\overline{\Omega}$. 
For every $s\in \RR$, there exists a constant $C>0$ such that
\begin{equation} \label{ell_reg_in_eqt}
\big\|\underline{u} \big\|_{s,\rho} \leq C \left(
\big\| L(\VEC{x},\diff) \underline{u} \big\|_{s-k,\rho}
+ \big\|\underline{u}\big\|_{s-1,\rho} \right)
\end{equation}
for $\displaystyle u \in H^{s,2}_0(\Omega)$, where as usual
\[
\underline{u}(\VEC{x}) = \begin{cases}
u(\VEC{x}) & \quad \ \text{if} \ \VEC{x} \in \Omega \\
0 & \quad \ \text{if} \ \VEC{x} \in \RR^n\setminus \Omega  
\end{cases}
\]
In particular, if $s \in \NN$, there exists a constant $\tilde{C}$ such that
\begin{equation} \label{ell_reg_in_eqtV2}
\big\|u \big\|_{s,2,\Omega} \leq \tilde{C} \left(
\big\| L(\VEC{x},\diff) u \big\|_{s-k,2,\Omega}
+ \big\|u\big\|_{s-1,2,\Omega} \right)
\end{equation}
for $\displaystyle u \in H^{s,2}_0(\Omega)$.
\end{prop}

\begin{proof}
\stage{i} We first consider the elliptic operator
$\displaystyle L_1(\VEC{x},\diff) = \sum_{|\VEC{\alpha}|= k}
a_{\VEC{\alpha}} \diff^{\VEC{\alpha}}$,
where we assume that the coefficients $a_{\VEC{\alpha}}$ are constant.  We
have that
\[
 \left(L_1(\VEC{x},\diff) u \right)^\wedge(\VEC{y})
= (-i)^k \sum_{|\VEC{\alpha}|= k} a_{\VEC{\alpha}} \VEC{y}^{\VEC{\alpha}} \,
\hat{u}(\VEC{y})
\]
for all $\displaystyle u \in H^{s,2}(\RR^n)$.  Since
$L_1(\VEC{x},\diff)$ is elliptic, there exists a constant $C_0>0$ such that
\[
\bigg|\sum_{|\VEC{\alpha}|= k} a_{\VEC{\alpha}}
\VEC{y}^{\VEC{\alpha}}\bigg| \geq  C_0 \|\VEC{y}\|^k
\]
for all $\VEC{y} \in \RR^n$.  thus,
\[
 \Big| \left(L_1(\VEC{x},\diff) u \right)^\wedge(\VEC{y}) \Big|
= \bigg|\sum_{|\VEC{\alpha}|= k} a_{\VEC{\alpha}} \VEC{y}^{\VEC{\alpha}}\bigg| \,
\left|\hat{u}(\VEC{y})\right|
\geq C_0 \|\VEC{y}\|^k \, \left|\hat{u}(\VEC{y})\right|
\]
and so
\begin{align*}
\left(1 + \|\VEC{y}\|_2^2\right)^s \left| \hat{u}(\VEC{y})\right|^2
&\leq \left(1 + \|\VEC{y}\|_2^2\right)^{s-k}
\left(1 + \|\VEC{y}\|_2^2\right)^k \left| \hat{u}(\VEC{y})\right|^2
\leq 2^k \left(1 + \|\VEC{y}\|_2^2\right)^{s-k}
\left(1 + \|\VEC{y}\|_2^{2k}\right) \left| \hat{u}(\VEC{y})\right|^2 \\
&\leq 2^k \left(1 + \|\VEC{y}\|_2^2\right)^{s-k}
\left| \hat{u}(\VEC{y})\right|^2
+ \frac{2^k}{C_0^2} \left(1 + \|\VEC{y}\|_2^2\right)^{s-k}
\left| \left(L_1(\VEC{x},\diff) u\right)^\wedge(\VEC{y})\right|^2 \ .
\end{align*}
We get that
\begin{align*}
\|u\|_{s,\rho}^2
&= \int_{\RR^n}\left(1 + \|\VEC{y}\|_2^2\right)^s
\left| \hat{u}(\VEC{y})\right|^2 \dx{\VEC{u}} \\
&\leq 2^k \int_{\RR^n}
\left(1 + \|\VEC{y}\|_2^2\right)^{s-k}
\left| \hat{u}(\VEC{y})\right|^2 \dx{\VEC{y}}
+ \frac{2^k}{C_0^2} \int_{\RR^n} \left(1 + \|\VEC{y}\|_2^2\right)^{s-k}
\left| \left(L_1(\VEC{x},\diff) u\right)^\wedge(\VEC{y})\right|^2
\dx{\VEC{y}} \\
&= 2^k \|u\|_{s-k,\rho}^2
+ \frac{2^k}{C_0^2} \big\|L_1(\VEC{x},\diff)u\big\|_{s-k,\rho}^2
\leq 2^k \|u\|_{s-1,\rho}^2
+ \frac{2^k}{C_0^2} \big\|L_1(\VEC{x},\diff)u\big\|_{s-k,\rho}^2 \\
&\leq C_1^2 \left( \|u\|_{s-1,\rho} +
\big\|L_1(\VEC{x},\diff)u\big\|_{s-k,\rho}\right)^2 \ ,
\end{align*}
where $\displaystyle C_1^2 = 2^k \max\{1,1/C_0^2\}$.  We get
\begin{equation} \label{ellRegInEq1}
\|u\|_{s,\rho} \leq C_1 \left( \|u\|_{s-1,\rho} +
\big\|L_1(\VEC{x},\diff)u\big\|_{s-k,\rho}\right) \ .
\end{equation}

\stage{ii} We now consider the elliptic operator
$\displaystyle L_2(\VEC{x},\diff)
= \sum_{|\VEC{\alpha}|= k} a_{\VEC{\alpha}}(\VEC{x}) \diff^{\VEC{\alpha}}$,
where we assume that $\displaystyle a_{\VEC{\alpha}} \in \DD(\RR^n)$.
We may make this assumption because we assume that the $a_{\VEC{\alpha}}$ are
of class $\displaystyle C^\infty$ in an open neighbourhood $U$ of
$\overline{\Omega}$.
Therefore, we may choose $\phi \in \DD(\RR^n)$ such that
$\supp \phi \subset U$ and $\phi(\VEC{x}) = 1$ for all
$\VEC{x} \in \overline{\Omega}$, and set
\[
\tilde{a}_{\VEC{\alpha}}(\VEC{x})
= \begin{cases}
\phi(\VEC{x}) a_{\VEC{\alpha}}(\VEC{x}) & \quad \text{if} \ \VEC{x} \in U \\
0 & \quad \text{if} \ \VEC{x} \in \RR^n \setminus U
\end{cases}
\]
to get functions $\displaystyle \tilde{a}_{\VEC{\alpha}} \in \DD(\RR^n)$
such that $\tilde{a}_{\VEC{\alpha}} = a_{\VEC{\alpha}}$ on $\Omega$.
Therefore, the elliptic partial differential equation on $\Omega$ is
not changed if we replace the coefficients $a_{\VEC{\alpha}}$ by the
coefficients $\tilde{a}_\alpha$.
This is what we do.  From now on, we drop the tilde to simplify the notation.

Since $\graD a_{\VEC{\alpha}}$ is continuous and has a compact support,
we have that
\[
K = \sup_{\VEC{x} \in \RR^n} \|\graD a_{\VEC{\alpha}} (\VEC{x}) \|_2 < \infty \ .
\]
Therefore, from the Mean Value Theorem, we get that
\begin{equation} \label{ellRegInEq2}
\left| a_{\VEC{\alpha}}(\VEC{x}_1) - a_{\VEC{\alpha}}(\VEC{x}_2) \right| \leq K
\|\VEC{x}_1 - \VEC{x}_2 \|_2
\end{equation}
for all $\displaystyle \VEC{x}_1, \VEC{x}_2 \in \RR^n$.  Since there
is a finite number of $a_{\VEC{\alpha}}$, we may assume that
(\ref{ellRegInEq2}) is true for all $\VEC{\alpha}$ with $|\VEC{\alpha}|=k$
and all $\displaystyle \VEC{x}_1, \VEC{x}_2 \in \RR^n$.

Given $\VEC{x}_0 \in \overline{\Omega}$ and $\delta >0$,
choose $\displaystyle \psi \in \DD(\RR^n)$ such that
$\supp \psi \subset B_{2\delta}(\VEC{0})$, $|\psi(\VEC{x})|\leq 1$ for
all $\VEC{x}$, and $\psi(\VEC{x}) = 1$ for all $\VEC{x} \in B_\delta(\VEC{0})$.
Suppose that $u \in H^{s,2}(\RR^n)$ is such that
$\displaystyle \supp u \subset B_\delta(\VEC{x}_0)$.
Then it follows
from Lemma~\ref{ell_reg_lemma1} that there exists a constant $C_3$ such that
\begin{align*}
\left\| \big(a_{\VEC{\alpha}} - a_{\VEC{\alpha}}(\VEC{x}_0)\big)
\diff^{\VEC{\alpha}} u \right\|_{s-k,\rho}
&= \left\| \big(a_{\VEC{\alpha}} - a_{\VEC{\alpha}}(\VEC{x}_0)\big)
\,\psi_{\VEC{x}_0}\, \diff^{\VEC{\alpha}} u \right\|_{s-k,\rho} \\
&\leq (2\pi)^{n/2}
\sup_{\VEC{x} \in \RR^n} \big| (a_{\VEC{\alpha}}(\VEC{x}) -
a_{\VEC{\alpha}}(\VEC{x}_0)) \psi_{\VEC{x}_0} \big|\,
\big\|\diff^{\VEC{\alpha}}  u\big\|_{s-k,\rho} \\
&\qquad
+ C_3 \left\|(a_{\VEC{\alpha}} - a_{\VEC{\alpha}}(\VEC{x}_0))
\psi_{\VEC{x}_0}\right\|_{|s-k-1|+1+t,\rho}
\, \big\|\diff^{\VEC{\alpha}} u \big\|_{s-k-1,\rho} \\
&\leq C_2 \delta \, \big\|\diff^{\VEC{\alpha}} u\big\|_{s-k,\rho}
+ C_4 \, \big\|\diff^{\VEC{\alpha}} u \big\|_{s-k-1,\rho} 
\leq C_2 \delta \, \|u\|_{s,\rho}
+ C_4 \, \|u \|_{s-1,\rho}  \ ,
\end{align*}
where $t > n/2$ is arbitrary but fixed,
$\displaystyle C_2 = 2 (2\pi)^{n/2}K$ and \\
$\displaystyle C_4 \geq C_3 \max_{\VEC{x}_0 \in \overline{\Omega}}
\|(a_{\VEC{\alpha}} - a_{\VEC{\alpha}}(\VEC{x}_0))
\psi_{\VEC{x}_0}\|_{|s-k-1|+1+t,\rho}$.
Such a maximum exists because $\overline{\Omega}$ is compact
and $\VEC{x}_0 \mapsto (a_{\VEC{\alpha}} -
a_{\VEC{\alpha}}(\VEC{x}_0))\psi_{\VEC{x}_0}$
is a continuous mapping from $\overline{\Omega}$ into
$\displaystyle H^{|s-k-1|+1+t}(\RR^n)$.  Recall that
$\psi_{\VEC{x}_0}(\VEC{x}) = \psi(\VEC{x} - \VEC{x}_0)$ for all
$\displaystyle \VEC{x} \in \RR^n$.
Hence, $C_4$ is independent of $\VEC{x}_0 \in \overline{\Omega}$.

We then have that
\[
\left\| L_2(\VEC{x},\diff) u - L_2(\VEC{x}_0,\diff) u\right\|_{s-k,\rho}
\leq \sum_{|\VEC{\alpha}|=k} \left\| \big(a_{\VEC{\alpha}}
- a_{\VEC{\alpha}}(\VEC{x}_0)\big)
\diff^{\VEC{\alpha}} u \right\|_{s-k,\rho}
\leq C_2 \delta N \| u \|_{s,\rho}
+C_4 N \| u \|_{s-1,\rho} \ ,
\]
where $\displaystyle N = \sum_{|\VEC{\alpha}|=k} 1$.  From
(\ref{ellRegInEq1}) with $L_1(\VEC{x},\diff)$ replaced by
$L_2(\VEC{x}_0,\diff)$ which has constant coefficients, we get
\begin{align*}
\|u\|_{s,\rho}
&\leq C_1 \left( \|u\|_{s-1,\rho} +
\big\|L_2(\VEC{x}_0,\diff)u\big\|_{s-k,\rho}\right) \\
&\leq C_1 \left( \|u\|_{s-1,\rho} +
\big\|L_2(\VEC{x}_0,\diff)u - L_2(\VEC{x},\diff)u\big\|_{s-k,\rho}
+\big\|L_2(\VEC{x},\diff)u\big\|_{s-k,\rho}\right) \\
&\leq C_1 \left( \|u\|_{s-1,\rho} + C_2 \delta N \| u \|_{s,\rho}
+C_4 N \left\| u \right\|_{s-1,\rho} 
+\big\|L_2(\VEC{x},\diff)u\big\|_{s-k,\rho}\right) \\
& = C_1 (1+C_4 N) \|u\|_{s-1,\rho} +
C_1 \big\|L_2(\VEC{x},\diff)u\big\|_{s-k,\rho}
+ C_1 C_2 \delta N \| u \|_{s,\rho} \ .
\end{align*}
If we choose $0<\delta < 1/(2C_1C_2 N)$, we get
\[
\|u\|_{s,\rho}
\leq C_1 (1+C_4 N) \|u\|_{s-1,\rho} +
C_1 \big\|L_2(\VEC{x},\diff)u\big\|_{s-k,\rho}
+ \frac{1}{2} \left\| u \right\|_{s,\rho} \ .
\]
Therefore,
\begin{equation} \label{ellRegInEq3}
\begin{split}
\|u\|_{s,\rho}
&\leq 2 C_1 (1+C_4 N) \|u\|_{s-1,\rho} +
2C_1 \big\|L_2(\VEC{x},\diff)u\big\|_{s-k,\rho} \\
&\leq C_5 \left( \|u\|_{s-1,\rho} +
\big\|L_2(\VEC{x},\diff)u\big\|_{s-k,\rho} \right) \ ,
\end{split}
\end{equation}
where $\displaystyle C_5 = 2C_1(1+C_4N)$.

\stage{iii} As in (ii), we still consider the elliptic operator
$\displaystyle L_2(\VEC{x},\diff)
= \sum_{|\alpha|= k} a_{\VEC{\alpha}}(\VEC{x}) \diff^{\VEC{\alpha}}$,
where we assume that $a_{\VEC{\alpha}} \in \DD(\RR^n)$.
Since $\overline{\Omega}$ is compact, there exists a finite subset 
$\displaystyle \{x_i\}_{i=1}^I$ of $\overline{\Omega}$ such that
$\displaystyle \left\{B_\delta(x_i)\right\}_{i=1}^I$ is an open cover
of $\overline{\Omega}$ with $\delta$ selected in (ii).  Let
$\displaystyle \{\phi_i\}_{i=1}^I$ be a
partition of unity of $\overline{\Omega}$ subordinate to
$\displaystyle \left\{B_\delta(x_i)\right\}_{i=1}^I$.

Given $\displaystyle u \in H^{s,2}_0(\Omega)$, then
$\displaystyle \underline{u} \in H^{s,2}(\RR^n)$ according to
Proposition~\ref{sob_expand_WKP}
and $\supp \phi_i\,\underline{u} \subset B_\delta(\VEC{x}_i)$.  Therefore,
\begin{align}
\big\|\underline{u}\big\|_{s,2}
& \leq \sum_{i=1}^I \big\|\phi_i\,\underline{u}\big\|_{s,2}
\leq C_5 \sum_{i=1}^I \left( \big\|\phi_i\,\underline{u}\big\|_{s-1,\rho} +
\big\|L_2(\VEC{x},\diff)(\phi_i\,\underline{u})\big\|_{s-k,\rho} \right)
\nonumber \\
&\leq C_5 \sum_{i=1}^I \left( \big\|\phi_i\,\underline{u}\big\|_{s-1,\rho}
+ \big\|\phi_i \, L_2(\VEC{x},\diff)\underline{u} \big\|_{s-k,\rho}
\right. \nonumber \\
&\qquad \qquad \left. + \big\|L_2(\VEC{x},\diff)(\phi_i\,\underline{u})
- \phi_i\, L_2(\VEC{x},\diff)\underline{u} \big\|_{s-k,\rho} \right) \ .
\label{ellRegInEq4}
\end{align}
We have from Proposition~\ref{sob_T_dd_wk2} that there exists
a constant $C_6>0$ such that
\begin{equation} \label{ellRegInEq5}
\big\|\phi_i\,\underline{u}\big\|_{s-1,\rho}
\leq C_6 \big\|\underline{u}\big\|_{s-1,\rho} \ .
\end{equation}
Since there are a finite number of $\phi_i$, we may assume that $C_6$
is large enough for (\ref{ellRegInEq5}) to be true for all $i$.
Similarly, from Proposition~\ref{sob_T_dd_wk2}, we have that there exists
a constant $C_7>0$ such that
\begin{equation} \label{ellRegInEq6}
\big\|\phi_i\, L_2(\VEC{x},\diff)\underline{u}\big\|_{s-k,\rho}
\leq C_7 \big\|L_2(\VEC{x},\diff)\underline{u}\big\|_{s-k,\rho} \ .
\end{equation}
Again, since there are a finite number of $\phi_i$, we may assume that $C_7$
is large enough for (\ref{ellRegInEq6}) to be true for all $i$.
Since
\[
L_2(\VEC{x},\diff)(\phi_i\,\underline{u})
- \left(L_2(\VEC{x},\diff)\phi_i\right) \,\underline{u}
= \sum_{|\alpha|\leq k-1} \psi_{\VEC{\alpha}} \diff^{\VEC{\alpha}} \underline{u} \ ,
\]
where $\displaystyle \psi_{\VEC{\alpha}} \in \DD(\RR^n)$, we get from
Proposition~\ref{sob_DinDm} and Proposition~\ref{sob_T_dd_wk2} that
\begin{align}
&\big\|L_2(\VEC{x},\diff)(\phi_i\,\underline{u})
- \phi_i\, L_2(\VEC{x},\diff)\underline{u} \big\|_{s-k,\rho}
= \Big\| \sum_{|\VEC{\alpha}|\leq k-1} \psi_{\alpha}
\diff^{\VEC{\alpha}} \underline{u} \Big\|_{s-k,\rho}
\leq \sum_{|\VEC{\alpha}|\leq k-1} \big\| \psi_{\VEC{\alpha}}
\diff^{\VEC{\alpha}} \underline{u} \big\|_{s-k,\rho} \nonumber \\
&\quad = \sum_{|\VEC{\alpha}|\leq k-1} R_{\VEC{\alpha},s-k}
\big\|\diff^{\VEC{\alpha}} \underline{u} \big\|_{s-k,\rho}
\leq \sum_{|\VEC{\alpha}|\leq k-1} R_{\VEC{\alpha},s-k}
\big\| \underline{u} \big\|_{s-k+|\VEC{\alpha}|,\rho}
\leq \bigg( \sum_{|\VEC{\alpha}|\leq k-1} R_{\VEC{\alpha},s-k} \bigg)
\big\| \underline{u} \big\|_{s-1,\rho} \label{ellRegInEq7}
\end{align}
for some constants $R_{\VEC{\alpha},s-k}>0$.
As we did before, since there are a finite number of $\phi_i$, we may
assume that the $R_{\VEC{\alpha},s-k}>0$ are large
enough for (\ref{ellRegInEq7}) to be true for all $i$.
If $\displaystyle C_8 = \sum_{|\VEC{\alpha}|\leq k-1} R_{\VEC{\alpha},s-k}$,
we get that
\begin{equation} \label{ellRegInEq8}
\big\|L_2(\VEC{x},\diff)(\phi_i\,\underline{u})
- \phi_i\, L_2(\VEC{x},\diff)\underline{u} \big\|_{s-k,\rho}
\leq C_8 \big\| \underline{u} \big\|_{s-1,\rho}
\end{equation}
for all $i$.  Combining (\ref{ellRegInEq4}), (\ref{ellRegInEq5}),
(\ref{ellRegInEq6}) and (\ref{ellRegInEq8}), we obtain
\begin{align}
\big\|\underline{u}\big\|_{s,\rho}
&\leq C_5 \sum_{i=1}^I  
\left( C_6 \big\|\underline{u}\big\|_{s-1,\rho} 
+ C_7 \big\|L_2(\VEC{x},\diff)\underline{u}\big\|_{s-k,\rho}
+ C_8 \big\| \underline{u} \big\|_{s-1,\rho} \right) \nonumber \\
&\leq C_9 \left( \big\|\underline{u}\big\|_{s-1,\rho}
+ \big\|L_2(\VEC{x},\diff)\underline{u}\big\|_{s-k,\rho} \right) \ ,
\label{ellRegInEq9}
\end{align}
where $C_9 = I C_5 \max\{ C_6 + C_8, C_7\}$.

\stage{iv} We now consider the full elliptic operator
$L(\VEC{x},\diff)$ given in (\ref{ell_reg_in_eqt}).  We have that
$L(\VEC{x},\diff) = L_2(\VEC{x},\diff) + L_3(\VEC{x},\diff)$, where
$\displaystyle L_3(\VEC{x},\diff)
= \sum_{|\alpha|< k} a_{\VEC{\alpha}}(\VEC{x}) \diff^{\VEC{\alpha}}$.
We have that
\begin{equation} \label{ellRegInEq10}
\big\|\underline{u}\big\|_{s,\rho} \leq C_9
\left( \big\|\underline{u}\big\|_{s-1,\rho} +
\big\|L_2(\VEC{x},\diff)\underline{u}\big\|_{s-k,\rho} \right)
\end{equation}
from (\ref{ellRegInEq9}).  Proceeding as we just did in (iii) for
$\displaystyle \big\|L_2(\VEC{x},\diff)(\phi_i\,\underline{u})
- \phi_i\, L_2(\VEC{x},\diff)\underline{u} \big\|_{s-k,\rho}$, we find
that there exists a constant $C_{10}>0$ such that
\[
 \big\| L_3(\VEC{x},\diff)(\phi_i\,\underline{u}) \big\|_{s-k,\rho}
\leq C_{10} \big\|\underline{u}\big\|_{s-1,\rho}
\]
for all $i$.  Thus,
\begin{equation} \label{ellRegInEq11}
\big\| L_3(\VEC{x},\diff)(\underline{u}) \big\|_{s-k,\rho}
\leq \sum_{i=1}^I \big\| L_3(\VEC{x},\diff)(\phi_i \underline{u})
\big\|_{s-k,\rho} \leq C_{11} \big\|\underline{u}\big\|_{s-1,\rho} \ ,
\end{equation}
where $C_{11} = I C_{10}$.  It follows from (\ref{ellRegInEq10}) and
(\ref{ellRegInEq11}) that
\begin{align*}
&\big\|\underline{u}\big\|_{s,\rho}
\leq C_9 \left( \big\|\underline{u}\big\|_{s-1,\rho} +
\big\|L_2(\VEC{x},\diff)\underline{u}\big\|_{s-k,\rho} \right)
= C_9 \left( \big\|\underline{u}\big\|_{s-1,\rho} +
\big\|L(\VEC{x},\diff)\underline{u}
- L_3(\VEC{x},\diff)\underline{u}\big\|_{s-k,\rho} \right) \\
&\qquad \leq C_9 \left( \big\|\underline{u}\big\|_{s-1,\rho} +
\big\|L(\VEC{x},\diff)\underline{u}\big\|_{s-k,\rho} +
\big\|L_3(\VEC{x},\diff)\underline{u}\big\|_{s-k,\rho} \right) \\
&\qquad \leq C_9 \left( \big\|\underline{u}\big\|_{s-1,\rho} +
\big\|L(\VEC{x},\diff)\underline{u}\big\|_{s-k,\rho} +
C_{11} \big\|\underline{u}\big\|_{s-1,\rho} \right)
\leq C \left( \big\|\underline{u}\big\|_{s-1,\rho} +
\big\|L(\VEC{x},\diff)\underline{u}\big\|_{s-k,\rho} \right)
\end{align*}
for $u \in H^{s,2}_0(\Omega)$, where $C = C_9 \max\{1, 1+C_{11}\}$.

\stage{v} (\ref{ell_reg_in_eqtV2}) is a consequence of
(\ref{ell_reg_in_eqt}) if we use the equivalence between the norms
$\|\cdot\|_{s,\rho}$ and $\|\cdot\|_{s,2}$ for $s\in \NN$, and the
fact that $\|f\|_{k,2,\Omega} = \|\underline{f}\|_{k,2,\RR^n}$ for all
$\displaystyle f \in H^k_0(\Omega)$ and $k \in \NN$.
\end{proof}

\begin{cor} \label{ell_cor_convex}
For every non-negative integers $t$ and $s$ such that
$t<s-1$, there exists a positive constant $C_t$ such that
$\displaystyle \|u \|_{s,2} \leq C_t \left( \| L(\VEC{x},\diff) u \|_{s-k,2}
+ \|u\|_{t,2} \right)$ for all $\displaystyle u \in H^{s,2}_0(\Omega)$.
\end{cor}

\begin{proof}
From Lemma~\ref{ell_garding4} with
$\displaystyle \epsilon = \left(\frac{1}{2C}\right)^2$, where $C$ is the
constant in (\ref{ell_reg_in_eqt}), there exists $K>0$ such that
\begin{equation} \label{ellCorConvexEq1}
\|u \|_{s-1,2,\Omega} \leq \frac{1}{2C} \| u \|_{s,2,\Omega} + K \|u\|_{t,2,\Omega}
\end{equation}
for all $\displaystyle u \in H^{s,2}_0(\Omega)$.
To be more precise, Lemma~\ref{ell_garding4} gives a constant
$\displaystyle K^2$ such that
\[
\big\|\underline{u} \big\|_{s-1,2,\RR^n}^2 \leq
\left(\frac{1}{2C}\right)^2 \big\|\underline{u}\big\|_{s,2,\RR^n}^2
+ K^2 \big\|\underline{u}\big\|_{t,2,\RR^n}^2
\leq \left( \frac{1}{2C} \big\| \underline{u} \big\|_{s,2,\RR^n}
+ K \big\|\underline{u}\big\|_{t,2,\RR^n} \right)^2
\]
for all $\displaystyle u \in H^{q,2}_0(\Omega)$, where
\[
\underline{u}(\VEC{x}) = \begin{cases}
u(\VEC{x}) & \quad \text{if} \ \VEC{x} \in \Omega \\
  0 & \quad \text{if} \ \VEC{x} \in \RR^n \setminus \Omega
\end{cases}
\]
Then we use the fact that
$\big\|\underline{f}\big\|_{k,2,\RR^n} = \big\|f\big\|_{k,2,\Omega}$
for all $\displaystyle f \in H^{k,2}_0(\Omega)$
to get (\ref{ellCorConvexEq1}).

If we substitute in (\ref{ell_reg_in_eqtV2}), we get
\begin{align*}
\|u \|_{s,2} &\leq C \left( \| L(\VEC{x},\diff)u \|_{s-k,2}
+ \|u\|_{s-1,2} \right)
\leq C \left( \| L(\VEC{x},\diff)u \|_{s-k,2} + \frac{1}{2C} \| u \|_{s,2} + K
\|u\|_{t,2} \right) \\
&= \frac{1}{2} \| u \|_{s,2} + C \left( \| L(\VEC{x},\diff)u \|_{s-k,2}
+ K \|u\|_{t,2} \right)
\end{align*}
for all $\displaystyle u \in H^{q,2}_0(\Omega)$.  Thus
\[
\|u \|_{s,2} \leq 2C \left( \| L(\VEC{x},\diff)u \|_{s-k,2} + K \|u\|_{t,2}
\right) \leq C_t \left( \| L(\VEC{x},\diff)u \|_{s-k,2} + \|u\|_{t,2} \right)
\]
for all $\displaystyle u \in H^{q,2}_0(\Omega)$,
where $C_t = 2C \max \{1, K \}$.  Do not forget that $K$ depends on
$C$ and $t$.
\end{proof}

\begin{lemma} \label{ell_reg_in_lem1}
We consider the elliptic operator $L(\VEC{x},\diff)$ defined in
(\ref{ell_reg_ellOp}).  Suppose that $\displaystyle \phi \in \DD(\RR^n)$
and $s\in \RR$.
The operator $\left[ L(\VEC{x},\diff), \phi\right]$ defined by
\[
\left[ L(\VEC{x},\diff), \phi\right](v)
= L(\VEC{x},\diff) (\phi\, v) - \phi\, L(\VEC{x},\diff) v
\]
for $\displaystyle v \in \displaystyle H^{s,2}(\RR^n)$ is a bounded mapping from
$\displaystyle H^{s,2}(\RR^n)$ into $\displaystyle H^{s-k+1,2}(\RR^n)$
\footnotemark.
\end{lemma}

\footnotetext{The reader may wonder why we did not regroup all the
$[A,B]$ operators that we have introduced so far, and presented them in
a beautiful unified theory.  We just did not want to introduce more
abstraction than we have already introduce.}

\begin{proof}
We have that
\begin{align*}
&\left[ L(\VEC{x},\diff), \phi\right] v
= L(\VEC{x},\diff) (\phi\, v) - \phi\, L(\VEC{x},\diff) v
= \sum_{|\VEC{\alpha}|\leq k} a_{\VEC{\alpha}}(\VEC{x})
\diff^{\VEC{\alpha}} (\phi\, v)
- \phi\, \sum_{|\VEC{\alpha}|\leq k} a_{\VEC{\alpha}}(\VEC{x})
\diff^{\VEC{\alpha}} v \\
&\qquad = \sum_{|\VEC{\alpha}|\leq k} a_{\VEC{\alpha}}(\VEC{x})
\left( \sum_{|\VEC{\nu}|+|\VEC{\eta}|=|\VEC{\alpha}|}
b_{\VEC{\nu},\VEC{\eta}}
\big(\diff^{\VEC{\nu}} \phi\big)\big(\diff^{\VEC{\eta}} v\big) \right)
- \phi\, \sum_{|\VEC{\alpha}|\leq k} a_{\VEC{\alpha}}(\VEC{x})
\diff^{\VEC{\alpha}} v \\
&\qquad = \sum_{|\VEC{\eta}|\leq k} \left(
\sum_{|\VEC{\nu}|\leq k-|\VEC{\eta}|} 
a_{\VEC{\nu}+\VEC{\eta}}(\VEC{x})\, b_{\VEC{\nu},\VEC{\eta}}
\big(\diff^{\VEC{\nu}} \phi\big) \right) \diff^{\VEC{\eta}} v
- \phi\, \sum_{|\VEC{\alpha}|\leq k} a_{\VEC{\alpha}}(\VEC{x})
\diff^{\VEC{\alpha}} v \\
&\qquad =\sum_{|\VEC{\eta}|\leq k} \left(
\sum_{|\VEC{\nu}|\leq k-|\VEC{\eta}|} 
a_{\VEC{\nu}+\VEC{\eta}}(\VEC{x})\, b_{\VEC{\nu},\VEC{\eta}}
\big(\diff^{\VEC{\nu}} \phi\big) - a_{\VEC{\eta}}(\VEC{x})\,\phi \right)
\diff^{\VEC{\eta}} v
= \sum_{|\VEC{\eta}|< k} A_{\VEC{\eta}}(\VEC{x}) \diff^{\VEC{\eta}} v
\end{align*}
for $\displaystyle v \in H^{s,2}(\RR^n)$,
where the coefficients $A_{\VEC{\eta}}$ are in $\displaystyle \DD(\RR^n)$.
Note that $\displaystyle a_{\VEC{\nu}+\VEC{\eta}}(\VEC{x})\,
b_{\VEC{\nu},\VEC{\eta}}
\big(\diff^{\VEC{\nu}} \phi\big) - a_{\VEC{\eta}}(\VEC{x})\phi = 0$ for
$|\VEC{\eta}| = k$ because $|\VEC{\nu}|=0$ and
$b_{\VEC{\nu},\VEC{\eta}} = 1$.
It follows from Propositions Propositions~\ref{sob_DinDm} and
\ref{sob_T_dd_wk2} that there exist
constants $C_{\VEC{\eta}}$ such that
\begin{equation} \label{ellRegInLem1Eq1}
\left\| A_{\VEC{\eta}}(\VEC{x}) \diff^{\VEC{\eta}} v \right\|_{s-k+1,\rho}
\leq C_{\VEC{\eta}} \left\| \diff^{\VEC{\eta}} v \right\|_{s-k+1,\rho}
\leq C_{\VEC{\eta}} \left\| v \right\|_{s-k+|\VEC{\eta}|+1,\rho}
\leq C_{\VEC{\eta}} \left\| v \right\|_{s,\rho}
\end{equation}
for $\displaystyle v \in H^{s+1,2}(\RR^n)$ and $|\VEC{\eta}|<k$.  Thus
$\left[ L(\VEC{x},\diff), \phi\right]$ is a bounded mapping from
$\displaystyle H^{s,2}(\RR^n)$ into $\displaystyle H^{s-k+1,2}(\RR^n)$.
\end{proof}

\begin{lemma} \label{ell_reg_in_lem2}
We consider the elliptic operator $L(\VEC{x},\diff)$ defined in
(\ref{ell_reg_ellOp}).  Assume that $\Omega$ is a bounded open subsets of
$\displaystyle \RR^n$ and $s\in \RR$.  If $\displaystyle u\in H^{s,2}(\RR^n)$
and $\displaystyle L(\VEC{x},\diff) u \in H^{s-k+1,2}_{loc}(\Omega)$, then
$\displaystyle u \in H^{s+1,2}_{loc}(\Omega)$.
\end{lemma}

\begin{proof}
According to Proposition~\ref{sob_Hsloc_cond}, to prove that
$\displaystyle u \in H^{s+1,2}_{loc}(\Omega)$, we only need to prove that
$\displaystyle \phi u \in H^{s+1,2}(\RR^n)$ for all $\phi\in \DD(\Omega)$,
where as usual $\phi \in \DD(\Omega)$ is identified to
$\displaystyle \phi \in \DD(\RR^n)$ with
support in $\Omega$.  Recall that $\phi  u \in \DD'(\RR^n)$ is
defined by $(\phi u)(\psi) = u(\phi \psi)$ for all
$\displaystyle \psi \in \DD(\RR^n)$.

Given $\phi \in \DD(\Omega)$, we have from
Proposition~\ref{sob_T_dd_wk2} that $\displaystyle \phi u \in H^{s,2}(\RR^n)$
and from Proposition~\ref{sob_Hsloc_cond} that
$\displaystyle \phi L(\VEC{x},\diff) u \in H^{s-k+1,2}(\RR^n)$.
Moreover, we have from Lemma~\ref{ell_reg_in_lem1}
that $\left[ L(\VEC{x},\diff), \phi\right]$ is a bounded mapping from
$\displaystyle H^{s,2}(\RR^n)$ into $\displaystyle H^{s-k+1,2}(\RR^n)$.
Thus,
$\displaystyle \left[ L(\VEC{x},\diff), \phi\right]u \in H^{s-k+1,2}(\RR^n)$.
Therefore,
$\displaystyle L(\VEC{x},\diff) \big(\phi u\big) = \phi L(\VEC{x}, \diff) u +
\left[ L(\VEC{x},\diff), \phi\right] u \in H^{s-k+1,2}(\RR^n)$.

Let $K = \supp \phi$.  Since $K$ is compact and $\Omega$ is open, there
exists $\delta >0$ such that
$\displaystyle Q = \{ \VEC{x} :
\sup_{\VEC{y} \in K} \|\VEC{x} -\VEC{y}\|\leq \delta \}$
is a compact subset of $\Omega$.  If $\|\VEC{h}\| \leq \delta$, then
$\Delta_{h_j\VEC{e}_j} (\phi u) \in H^{s,2}(\Omega)$
with $\supp \Delta_{h_j\VEC{e}_j} (\phi u) \subset Q
\subset \Omega$ for all $j$.  Hence
$\Delta_{h_j\VEC{e}_j} (\phi u) \in H^{s,2}_0(\Omega)$
for all $j$ according to Lemma~\ref{sob_w0_trad1}.  Moreover,
$\underline{\Delta_{h_j\VEC{e}_j} (\phi u)}
= \Delta_{h_j\VEC{e}_j} (\phi u)$ because
$\supp \Delta_{h_j\VEC{e}_j} (\phi u)\subset \Omega$.

Hence, from Proposition~\ref{ell_reg_in}, we have for $\|\VEC{h}\| \leq \delta$
that
\begin{align*}
&\left\| \Delta_{h_j\VEC{e}_j} (\phi u) \right\|_{s,\rho}
\leq C \left( \left\| L(\VEC{x},\diff) \left(\Delta_{h_j\VEC{e}_j}
(\phi u)\right) \right\|_{s-k,\rho}
+ \left\|\Delta_{h_j\VEC{e}_j} (\phi u)\right\|_{s-1,\rho} \right) \\
&\qquad = C \left( \left\| \Delta_{h_j\VEC{e}_j}\left(L(\VEC{x},\diff)
(\phi u)\right) - \left[\Delta_{h_j\VEC{e}_j} , L(\VEC{x},\diff)\right]
(\phi u) \right\|_{s-k,\rho}
+ \left\|\Delta_{h_j\VEC{e}_j} (\phi u)\right\|_{s-1,\rho} \right) \\
&\qquad \leq C \left( \left\| \Delta_{h_j\VEC{e}_j}\left(L(\VEC{x},\diff)
(\phi u)\right)\right\|_{s-k,\rho}
+ \left\| \left[\Delta_{h_j\VEC{e}_j} , L(\VEC{x},\diff)\right]
(\phi u) \right\|_{s-k,\rho}
+ \left\|\Delta_{h_j\VEC{e}_j} (\phi u)\right\|_{s-1,\rho} \right) \\
&\qquad \leq C \left( \left\| \Delta_{h_j\VEC{e}_j}\left(L(\VEC{x},\diff)
(\phi u)\right)\right\|_{s-k,\rho} + C_1\left\| \phi u \right\|_{s,\rho}
+ \left\|\Delta_{h_j\VEC{e}_j} (\phi u )\right\|_{s-1,\rho} \right) \\
&\qquad \leq C \left( \left\| L(\VEC{x},\diff) (\phi u) \right\|_{s-k+1,\rho}
+ C_1\left\| \phi u \right\|_{s,\rho}
+ \left\|\phi u\right\|_{s-1,\rho} \right) \ ,
\end{align*}
where we have used Corollary~\ref{ell_reg_cor1} to obtain the second
to last inequality with the constant $C_1$, and
Proposition~\ref{sob_DinDm} and Proposition~\ref{ell_reg_prop3}
to obtain the last inequality.

We have therefore proved that
$\displaystyle \limsup_{\VEC{h}\rightarrow \VEC{0}} \left\| \Delta_{h_j\VEC{e}_j}
(\phi u)\right\|_{s,\rho} < \infty$
for $1\leq j \leq n$.  It follows from Proposition~\ref{ell_reg_prop3} that
$\displaystyle \diff_{x_j} (\phi u) \in H^{s,2}(\RR^n)$
for $1\leq j \leq n$.  Finally, Proposition~\ref{sob_DinDm} implies that
$\displaystyle \phi u \in H^{s+1,2}(\RR^n)$.  Since $\phi \in \DD(\Omega)$
was arbitrary, we have that $\displaystyle u \in H^{s+1,2}_{loc}(\Omega)$.
\end{proof}

The mean result of this section is the following theorem.

\begin{theorem}[Elliptic Regularity Theorem] \label{ell_int_regTH}
We consider the elliptic operator $L(\VEC{x},\diff)$ defined in
(\ref{ell_reg_ellOp}) and assume that $s \in \RR$.  If
$\displaystyle f \in H^{s,2}_{loc}(\Omega)$ and $u \in \DD'(\Omega)$ satisfies
$L(\VEC{x},\diff) u=f$ in the sense of distribution,
then $\displaystyle u\in H^{s+k,2}_{loc}(\Omega)$.
\index{Elliptic Regularity Theorem}
\end{theorem}

\begin{proof}
Given $\phi \in \DD(\Omega)$, choose an open set $U_0$ such that
$K = \supp \phi \subset U_0 \subset \overline{U_0} \subset \Omega$.
This can be done since $K$ is compact and $\Omega$ is open.
Choose $\psi_0 \in \DD(\Omega)$ such that $\psi_0(\VEC{x}) = 1$
for all $\VEC{x} \in U_0$.

Since $\supp (\psi_0 u) \subset \supp \psi_0$ is compact, we have that
$\displaystyle \psi_0 u \in \EE'(\RR^n)$; namely, $\psi_0 u $ is a distribution
with a compact support \footnote{Recall that $\psi_0 u$ is defined by
$(\psi_0 u)(\mu) = u (\psi_0\mu)$ for all $\mu \in \DD(\RR^n)$.  Its
support is a subset of $\supp \psi_0$ because
$(\psi_0 u)(\mu) = 0$ for all $\mu \in \DD(\RR^n)$ with
$\supp \mu \cap \supp \psi_0 = \emptyset$.}.  We then have from
Corollary~\ref{sob_TheSobLemma_Cor1} that
$\displaystyle \psi_0 \, u \in H^{t,2}(\RR^n)$
for some $t \in \RR$.  By decreasing $t$ if necessary, we may assume that
$m = s + k - t$ is a positive integer.  We still have that
$\displaystyle \psi_0 \, u \in H^{t,2}(\RR^n)$ because we decrease $t$.

Choose an open set $U_1$ such that
$K \subset U_1 \subset \overline{U_1} \subset U_0$
and $\psi_1 \in \DD(\Omega)$ such that $\supp \psi_1 \subset U_0$ and
$\psi_1(\VEC{x}) = 1$ for all $\VEC{x} \in U_1$.  Proceeding
inductively, we can create a sequence
$\displaystyle \{ \psi_j, U_j\}_{j=0}^{m-1}$ such that
$K \subset U_j \subset \overline{U_j} \subset U_{j-1}$,
$\supp \psi_j \subset U_{j-1}$ and $\psi_j(\VEC{x}) = 1$ for all
$\VEC{x} \in U_j$.  Finally, set $\psi_m = \phi$.

We prove by induction on $j$ that $\displaystyle \psi_j u \in H^{t+j,2}(\RR^n)$
for $0 \leq j \leq m$.  The result is true for $j=0$ by construction.  
suppose that $\displaystyle \psi_j u \in H^{t+j,2}(\RR^n)$ for $j <m$.
We have for all $\displaystyle \mu \in \DD(\RR^n)$ that
\begin{align*}
&\left(\psi_{j+1} L(\VEC{x},\diff) (\psi_j u)\right)(\mu)
= \left(L(\VEC{x},\diff) (\psi_j u)\right) (\psi_{j+1} \mu)
= \sum_{|\VEC{\alpha}|\leq k} a_{\VEC{\alpha}}(\VEC{x})
\diff^{\VEC{\alpha}}(\psi_j u)(\psi_{j+1} \mu) \\
&\qquad = \sum_{|\VEC{\alpha}|\leq k} (-1)^{|\VEC{\alpha}|}
a_{\VEC{\alpha}}(\VEC{x})
\left(\psi_j u\right) \left(\diff^{\VEC{\alpha}}(\psi_{j+1} \mu)\right)
= \sum_{|\VEC{\alpha}|\leq k} (-1)^{|\VEC{\alpha}|} a_{\VEC{\alpha}}(\VEC{x}) u
\left(\diff^{\VEC{\alpha}}(\psi_{j+1} \mu)\right) \\
&\qquad = \sum_{|\VEC{\alpha}|\leq k} a_{\VEC{\alpha}}(\VEC{x})
\diff^{\VEC{\alpha}} u (\psi_{j+1} \mu)
= L(\VEC{x},\diff) (\psi_{j+1} \mu)
= \psi_{j+1} L(\VEC{x},\diff) (\mu)
= (\psi_{j+1} f)(\mu) \ ,
\end{align*}
where the fourth equality comes from
$\displaystyle \left(\psi_j u\right)(\eta) = u(\psi_j \eta) = u(\eta)$
for all test functions $\eta \in \DD(\Omega)$ with
$\supp \eta \subset U_j$ because
$\psi_j(\VEC{x}) = 1$ for all $\VEC{x} \in U_j$.  Moreover, 
$\displaystyle \supp \diff^{\VEC{\alpha}}(\psi_{j+1} \xi) \subset U_j$.
Hence,
\begin{align*}
L(\VEC{x},\diff)(\psi_{j+1} u)
&= L(\VEC{x},\diff)(\psi_{j+1} \psi_j u)
= \psi_{j+1} L(\VEC{x},\diff)(\psi_j u)
+ \left[ L(\VEC{x},\diff),\psi_{j+1}\right](\psi_j u) \\
&= \psi_{j+1} f + \left[ L(\VEC{x},\diff),\psi_{j+1}\right](\psi_j u) \ ,
\end{align*}
where $\displaystyle \psi_{j+1} f \in H^{s,2}(\RR^n)$ because
$\displaystyle f \in H^{s,2}_{loc}(\RR^n)$ and $\psi_{j+1} \in \DD(\Omega)$,
and $\displaystyle \left[ L(\VEC{x},\diff),\psi_{j+1}\right](\psi_j u)
\in H^{t+j-k+1,2}(\RR^n)$ according to 
Lemma~\ref{ell_reg_in_lem1} because
$\displaystyle \psi_j u \in H^{t+j,2}(\RR^n)$.
Therefore, \\
$\displaystyle L(\VEC{x},\diff)(\psi_{j+1} u) \in H^{t+j-k+1,2}(\RR^n)$.
Moreover,
$\displaystyle \psi_{j+1} u = \psi_{j+1}\psi_j u \in H^{t+j,2}(\RR^n)$
according to Proposition~\ref{sob_T_dd_wk2} because
$\displaystyle \psi_j u \in H^{t+j,2}(\RR^n)$ by our hypothesis of induction.
It then follows from Lemma~\ref{ell_reg_in_lem2} that
$\displaystyle \psi_{j+1} u \in H^{t+j+1,2}_{loc}(\Omega)$.

Given $\eta \in \DD(\Omega)$ with $\eta(\VEC{x}) = 1$ for all
$\VEC{x} \in V_j$, we have that
$\displaystyle \eta \psi_{j+1} u \in H^{t+j+1,2}(\RR^n)$
according to Proposition~\ref{sob_Hsloc_cond}.  However,
$\eta \psi_{j+1} u = \psi_{j+1} u$ on $\RR^n$ because
\[
(\eta \psi_{j+1} u)(\mu) = u\left(\eta \psi_{j+1} \mu\right) =
u\left(\psi_{j+1} \mu\right) = (\psi_{j+1} u)(\mu)
\]
for all $\mu \in \DD(\RR^n)$ because $\supp \psi_{j+1} \subset V_j$ and
so $\eta \psi_{j+1} \xi = \psi_{j+1} \xi$.
Therefore, $\psi_{j+1} u \in H^{t+j+1,2}(\RR^n)$.
This complete the proof by induction.

For $j=m$, we get $\displaystyle \phi u = \psi_m u \in H^{t+m,2}(\RR^n)
= H^{s+k,2}(\RR^n)$.  Since $\phi$ is arbitrary, we get that
$\displaystyle u \in H^{s+k,2}_{loc}(\RR^n)$ according to
Proposition~\ref{sob_Hsloc_cond}.
\end{proof}

It is true that the previous theorem is a very important theorem for
the internal regularity of elliptic operators.  However, its corollary
given below is probably the most famous result.

\begin{cor} \label{ell_Hak_Cinfty}
We consider the elliptic operator $L(\VEC{x},\diff)$ defined in
(\ref{ell_reg_ellOp}).  If
$\displaystyle L(\VEC{x},\diff)u = f \in C^\infty(\Omega)$ in
the sense of distributions, then $\displaystyle u \in C^\infty(\Omega)$.
Namely, $L(\VEC{x},\diff)$ is hypoelliptic.
\end{cor}

\begin{proof}
Since $\displaystyle f \in C^\infty(\Omega)$, we have that
$\displaystyle f \in H^{s,2}_{loc}(\Omega)$
for all $s$ according to Proposition~\ref{sob_Hsloc_cond}.
Therefore, $\displaystyle u \in H^{s+2,2}_{loc}(\Omega)$ for all $s$
according to Theorem~\ref{ell_int_regTH}.

By the Sobolev Lemma, Theorem~\ref{sob_TheSobLemma}, and the
definition of the localized Sobolev spaces, this implies that
$\displaystyle u \in C^\infty(\Omega)$.  To be more precise, let $V$ be an open
bounded subset such that $\overline{V} \subset \Omega$ and let $k$ be
a positive integer.  Choose $s > k + n/2$.
By definition of the localized Sobolev spaces, there exists
$\displaystyle v \in H^{s,2}(\RR^n)$ such that $v=u$ on $V$ in the
sense of distributions.  By the Sobolev Lemma,
$\displaystyle v \in C^k(\RR^n)$.  Thus
$\displaystyle u \in C^k(V)$.  Since $V$ and $k$
are arbitrary, this implies that $\displaystyle u\in C^k(\Omega)$ for all
non-negative integer $k$.
\end{proof}

\subsection{Regularity at the Boundary} \label{ell_sect_RB}

The title of this section is a little bit misleading but it is
traditionally used for the topic of this section.
The results in this section address the regularity of weak solutions
of elliptic partial differential equations on the entire bounded
domain $\Omega$ of the equations.  It is however true that the
boundary plays a significant role.

Let $\Omega$ be a bounded open subset of $\displaystyle \RR^n$.  In
this section, we consider the elliptic operator 
\begin{equation} \label{ell_reg_LDBound}
L(\VEC{x},\diff) = \sum_{|\VEC{\alpha}|\leq 2}
a_{\VEC{\alpha}}(\VEC{x}) \diff^{\VEC{\alpha}}
\end{equation}
on $\overline{\Omega}$ with $\displaystyle a_{\VEC{\alpha}} \in \SS(\RR^n)$
for all $\VEC{\alpha}$.  Our main result is the next theorem.
For this theorem, we go back to the divergence form of the
elliptic operator $L(\VEC{x},\diff)$ presented in
Section~\ref{CSWSsection}.

\begin{theorem} \label{ell_regular}
Let $\Omega$ be a bounded open subset of $\displaystyle \RR^n$
(or $\displaystyle \RR^n_+$)
with a boundary $\partial \Omega$ is of class $\displaystyle C^{q+2}$.  Let
$\displaystyle X = H^{1,2}(\Omega)$ or
$\displaystyle X = H^{1,2}_0(\Omega)$, and suppose that
$u\in X$ is a solution of the variational problem
\[
B(v,u) = \sum_{|\VEC{\alpha}|,|\VEC{\beta}|\leq 1}
\int_\Omega \overline{b_{\VEC{\alpha},\VEC{\beta}}(\VEC{x})
\diff^{\VEC{\beta}} u}\, \diff^{\VEC{\alpha}} v
\dx{\VEC{x}} = \int_\Omega \overline{f}\,v \dx{\VEC{x}}
\]
for $\displaystyle v \in X$,
obtained from the elliptic operator (\ref{ell_reg_LDBound}) on
$\overline{\Omega}$.  If $B$ is weakly coercive and
$\displaystyle f \in H^{q,2}(\Omega)$, then
$\displaystyle u\in H^{q+2,2}(\Omega)$.  Moreover, there exists a constant $C$
depending only on $\Omega$ such that
\begin{equation} \label{ell_norm_reg}
\|u\|_{q+2,2,\Omega} \leq C \left( \|f\|_{q,2,\Omega}
+ \|u\|_{2,\Omega} \right) \ .
\end{equation}
\end{theorem}

The proof of Theorem~\ref{ell_regular} will be a consequence of the
next proposition.  Moreover, we will only prove the theorem for
elliptic operators of order two; namely, $m=2$ and $k=1$.

We need a little lemma first.

\begin{lemma} \label{ellRegProp3Omega}
Let $\Omega$ be an open subset of $\displaystyle \RR^n$, and $k$ and $q$ be two
integers.  There exists a constant $C$ such that
$\displaystyle \left\| \Delta^{\VEC{\gamma}}_{\VEC{h}} f  \right\|_{k,2,\Omega}
\leq C \left\| \diff^{\VEC{\gamma}} f\right|_{k,2,\Omega}$ for all
$\displaystyle f\in H^{k,2}(\Omega)$ with compact support in $\Omega$,
multi-indices $\VEC{\gamma}$ such that $|\VEC{\gamma}|\leq q$, and
$\displaystyle \VEC{h} \in \RR^n$ such that
$\displaystyle \supp f + B_{q\|\VEC{h}\|}(\VEC{0}) \subset \Omega$.
\end{lemma}

\begin{proof}
We have from Lemma~\ref{sob_denselem1} that
$\displaystyle \underline{f} \in H^{k,2}(\RR^n)$.  It follows from
Proposition~\ref{ell_reg_prop3} that
\[
\left\| \Delta^{\VEC{\gamma}}_{\VEC{h}}\, \underline{f}  \right\|_{k,2,\rho}
\leq \left\| \diff^{\VEC{\gamma}} \underline{f}\right|_{k,2,\rho} \ .
\]

Since the norms $\|\cdot\|_{k,2,\RR^n}$ and $\|\cdot\|_{k,\rho}$ are
equivalent, there exist positive constants $C_1$ and $C_2$ such
\[
  C_1 \|g\|_{k,2,\RR^n} \leq \|g\|_{k,\rho} \leq C_2 \|g\|_{k,2,\RR^n}
\]
for all $\displaystyle g \in H^{k,2}(\RR^n)$.

Since $\displaystyle \supp \diff^{\VEC{\alpha}} f \subset \diff f
\subset \Omega$ and
$\displaystyle \supp \Delta^{\VEC{\gamma}}_{\VEC{h}} f \subset
\supp f + B_{q\|\VEC{h}\|}(\VEC{0}) \subset \Omega$, we have that
$\displaystyle \diff^{\VEC{\alpha}} \underline{f}
= \underline{\diff^{\VEC{\alpha}} f}$
and $\displaystyle \Delta^{\VEC{\gamma}}_{\VEC{h}} \underline{f}
= \underline{\Delta^{\VEC{\gamma}}_{\VEC{h}} f}$.  Thus,
\begin{align*}
\left\| \Delta^{\VEC{\gamma}}_{\VEC{h}} f  \right\|_{k,2,\Omega}
&= \left\| \Delta^{\VEC{\gamma}}_{\VEC{h}}\, \underline{f}  \right\|_{k,2,\RR^n}
\leq C_1^{-1} \left\| \Delta^{\VEC{\gamma}}_{\VEC{h}}\, \underline{f}
\right\|_{k,\rho} 
\leq C_1^{-1} \left\| \diff^{\VEC{\gamma}} \underline{f}  \right\|_{k,\rho} \\
&\leq C_1^{-1} C_2 \left\| \diff^{\VEC{\gamma}} \underline{f} \right|_{k,2,\RR^n}
= C \left\| \diff^{\VEC{\gamma}} f \right|_{k,2,\Omega}
\end{align*}
with $C = C_1^{-1} C_2$.  Note that $C$ does not depend of $f$ or $\VEC{h}$.
\end{proof}

We begin by assuming that $\displaystyle \Omega = \RR^n_+$.

\begin{prop} \label{ell_regular_Rnp}
Let $\displaystyle \Omega = \RR^n_+$ and
$\displaystyle X = H^{1,2}(\Omega)$ or
$\displaystyle X = H^{1,2}_0(\Omega)$.  For $r>0$, let
$\displaystyle \Omega_r = \Omega \cap B_r(\VEC{0})$
and
$\displaystyle X_r = \left\{ u \in X : \supp u \subset \Omega_\rho \
\text{for some} \ \rho <r \right\}$.
Suppose that $\displaystyle f \in H^{q,2}(\Omega_r)$ for some $r>0$.
If $u\in X$ is a solution of the variational problem
\begin{equation} \label{ell_var_reg}
B(v,u) =
\sum_{|\VEC{\alpha}|,|\VEC{\beta}|\leq 1}
\int_{\Omega_r} \overline{b_{\VEC{\alpha},\VEC{\beta}}(\VEC{x})
\diff^{\VEC{\beta}} u}\, \diff^{\VEC{\alpha}} v
\dx{\VEC{x}} = \int_{\Omega_r} \overline{f}\,v \dx{\VEC{x}}
\end{equation}
for $\displaystyle v \in X_r$, where $B$ is weakly coercive, then
$\displaystyle u\in H^{q+2,2}(\Omega_\rho)$ for
$\rho<r$ and there exists a constant $C = C(q,\rho)$ such that
\[
\|u\|_{q+2.2,\Omega_\rho} \leq C \left( \|f\|_{q,2,\Omega_r} +
\|u\|_{1,2,\Omega_r} \right) \ .
\]
\end{prop}

\begin{proof}
\subI{Claim A} If $\rho<r$ and $j<q+2$, then
$\displaystyle \diff^{\VEC{\gamma}} u \in H^{1,2}(\Omega_{\rho})$ for all
multi-indices $\VEC{\gamma}$ such that $|\VEC{\gamma}|\leq j$ and
$\gamma_n =0$.
Moreover, there exists a positive constant $C_1 = C_1(j,\rho)$ such that
\begin{equation} \label{ell_reg_B1}
\|\diff^{\VEC{\gamma}} u\|_{1.2,\Omega_\rho} \leq C_1
\left( \left\|f\right\|_{q,2,\Omega_r}
+ \left\|u\right\|_{1,2,\Omega_r} \right)
\end{equation}
for all multi-indices $\VEC{\gamma}$ such that $|\VEC{\gamma}| \leq j$
and $\gamma_n =0$.

The proof is by induction on $j$.  The claim is obviously true for
$j=0$.  Suppose that the claim is true for $0\leq j < q+1$.  Let
$\displaystyle r_1 = \frac{2\rho+r}{3}$ and
$\displaystyle r_2 = \frac{\rho+2r}{3}$.  We have $\rho <r_1<r_2< r$.

The induction hypothesis applied to the domain $\Omega_{r_2}$ instead
of $\Omega_\rho$ says that
$\displaystyle \diff^{\VEC{\gamma}} u \in H^{1,2}(\Omega_{r_2})$ for
all multi-indices $\VEC{\gamma}$ with $|\VEC{\gamma}|\leq j$ and
$\gamma_n = 0$, and
there exists a positive constant $C_2 = C_2(j,r_2)$ such that
\begin{equation} \label{ell_reg_B2}
\|\diff^{\VEC{\gamma}} u\|_{1,2,\Omega_{r_2}} \leq C_2 \left( \|f\|_{q,2,\Omega_r}
+ \|u\|_{1,2,\Omega_r} \right)
\end{equation}
for all multi-indices $\VEC{\gamma}$ such that $|\VEC{\gamma}|\leq j$
and $\gamma_n =0$.

Choose $\displaystyle \psi \in \DD(\RR^n)$ such that
$\supp \psi \subset B_{r_1}(\VEC{0})$ and $\psi = 1$ on $B_{\rho}(\VEC{0})$.
Let $\gamma$ be a multi-index such that $|\gamma|=j+1$ with $\gamma_n=0$.
If $\displaystyle \|\VEC{h}\| < (r_2-r_1)/(j+1)$,
then $\displaystyle \Delta_{\VEC{h}}^{\VEC{\gamma}} (\psi u) \in X_{r_2}$ because
$\supp \psi \subset B_{r_1}(\VEC{0})$, and
$\displaystyle \Delta_{\VEC{h}}^{\VEC{\gamma}} v \in X_{r_2}$ for
$v \in X_{r_1}$.  In
the sequel, we always assume that $\|\VEC{h}\|$ satisfies
$\displaystyle \|\VEC{h}\| < (r_2-r_1)/(j+1)$.

\stage{i} We prove that there exists a constant
$C_3= C_3(j,r_2,\psi)$ such that
\begin{equation} \label{ell_reg_B3}
\left| B\left(v,\Delta_{\VEC{h}}^{\VEC{\gamma}}(\psi u)\right) \right|
\leq C_3 \|v\|_{1,2,\Omega_{r_2}} \left( \|f\|_{q,2,\Omega_r} +
\|u\|_{1,2,\Omega_r} \right)
\end{equation}
for all $v \in X_{r_2}$.

Since the difference quotient operator (\ref{ell_diffQuot}) commutes
with the derivative operator, we get
\begin{align*}
&B\left(v,\Delta_{\VEC{h}}^{\VEC{\gamma}}(\psi u)\right) =
\sum_{|\VEC{\alpha}|,|\VEC{\beta}|\leq 1}
\int_{\Omega_r} \overline{b_{\VEC{\alpha},\VEC{\beta}}(\VEC{x})
\diff^{\VEC{\beta}} \Delta_{\VEC{h}}^{\VEC{\gamma}}(\psi u)}\,
\diff^{\VEC{\alpha}} v \dx{\VEC{x}} \\
&\qquad = \sum_{|\VEC{\alpha}|,|\VEC{\beta}|\leq 1}
\int_{\Omega_r} \overline{b_{\VEC{\alpha},\VEC{\beta}}(\VEC{x})
\Delta_{\VEC{h}}^{\VEC{\gamma}} \diff^{\VEC{\beta}} (\psi u)}\,
\diff^{\VEC{\alpha}} v \dx{\VEC{x}}
= \sum_{|\VEC{\alpha}|,|\VEC{\beta}|\leq 1}
\int_{\Omega_r} \overline{
\Delta_{\VEC{h}}^{\VEC{\gamma}} \left(b_{\VEC{\alpha},\VEC{\beta}}(\VEC{x})
\diff^{\VEC{\beta}} (\psi u)\right)} \, \diff^{\VEC{\alpha}} v \dx{\VEC{x}} + E_1
\end{align*}
for all $v \in X_{r_2}$, where
\begin{align*}
E_1(v,u) &= \sum_{|\VEC{\alpha}|,|\VEC{\beta}|\leq 1}
\int_{\Omega_r} \overline{\bigg( b_{\VEC{\alpha},\VEC{\beta}}(\VEC{x})
\Delta_{\VEC{h}}^{\VEC{\gamma}} \diff^{\VEC{\beta}} (\psi u) - 
\Delta_{\VEC{h}}^{\VEC{\gamma}} \left(b_{\VEC{\alpha},\VEC{\beta}}(\VEC{x})
\diff^{\VEC{\beta}} (\psi u)\right)\bigg)} \,
\diff^{\VEC{\alpha}} v \dx{\VEC{x}} \\
&= \sum_{|\VEC{\alpha}|,|\VEC{\beta}|\leq 1}
\int_{\Omega_r} \overline{-[\Delta_{\VEC{h}}^{\VEC{\gamma}},
b_{\VEC{\alpha},\VEC{\beta}}(\VEC{x})]
\diff^{\VEC{\beta}} (\psi u) }\, \diff^{\VEC{\alpha}} v \dx{\VEC{x}}
\end{align*}
for all $v \in X_{r_2}$.
Furthermore, we may write
\begin{equation} \label{ell_RB_foot}
B\left(v,\Delta_{\VEC{h}}^{\VEC{\gamma}}(\psi u)\right)
= \sum_{|\VEC{\alpha}|,|\VEC{\beta}|\leq 1}
\int_{\Omega_r} \overline{
\Delta_{\VEC{h}}^{\VEC{\gamma}} \left(b_{\VEC{\alpha},\VEC{\beta}}(\VEC{x})
\left( \psi \diff^{\VEC{\beta}} u\right) \right)} \,
\diff^{\VEC{\alpha}} v \dx{\VEC{x}} + E_1 + E_2
\end{equation}
for all $v \in X_{r_2}$,
where
\[
E_2(v,u) = \sum_{|\VEC{\alpha}|,|\VEC{\beta}|\leq 1}
\int_{\Omega_r} \overline{
\Delta_{\VEC{h}}^{\VEC{\gamma}} \left(b_{\VEC{\alpha},\VEC{\beta}}(\VEC{x})
\left(\diff^{\VEC{\beta}}( \psi u) - \psi \diff^{\VEC{\beta}} u\right) \right)}
\, \diff^{\VEC{\alpha}} v \dx{\VEC{x}}
\]
for all $v \in X_{r_2}$.  It follows from Proposition~\ref{DiffQuotientAdjoint}
that
\[
B\left(v,\Delta_{\VEC{h}}^{\VEC{\gamma}}(\psi u)\right) =
(-1)^{j+1} \sum_{|\VEC{\alpha}|,|\VEC{\beta}|\leq 1}
\int_{\Omega_r} \overline{ b_{\VEC{\alpha},\VEC{\beta}}(\VEC{x})
\left(\psi \diff^{\VEC{\beta}} u\right)} \,
\Delta_{-\VEC{h}}^{\VEC{\gamma}} \diff^{\VEC{\alpha}} v \dx{\VEC{x}} + E_1 + E_2
\]
for all $v \in X_{r_2}$.
Again, since the difference quotient operator commutes with the
derivative operator, we get  
\begin{align*}
B\left(v,\Delta_{\VEC{h}}^{\VEC{\gamma}}(\psi u)\right)
&= (-1)^{j+1} \sum_{|\VEC{\alpha}|,|\VEC{\beta}|\leq 1}
\int_{\Omega_r}  \overline{ b_{\VEC{\alpha},\VEC{\beta}}(\VEC{x})
\left(\psi\, \diff^{\VEC{\beta}} u\right)}
\, \diff^{\VEC{\alpha}} \Delta_{-\VEC{h}}^{\VEC{\gamma}} v\dx{\VEC{x}}
+ E_1 + E_2 \\
&= (-1)^{j+1} \sum_{|\VEC{\alpha}|,|\VEC{\beta}|\leq 1}
\int_{\Omega_r} \left(\overline{b_{\VEC{\alpha},\VEC{\beta}}(\VEC{x})
\diff^{\VEC{\beta}} u} \right) \, \diff^{\VEC{\alpha}} \left( \overline{\psi}
\Delta_{-\VEC{h}}^{\VEC{\gamma}} v\right) \dx{\VEC{x}} + E_1 + E_2 + E_3
\end{align*}
for all $v \in X_{r_2}$, where
\[
E_3(v,u) = (-1)^{j+1} \sum_{|\VEC{\alpha}|,|\VEC{\beta}|\leq 1}
\int_{\Omega_r} \left(\overline{b_{\VEC{\alpha},\VEC{\beta}}(\VEC{x})
\diff^{\VEC{\beta}} u}\right) \,
\left(\overline{\psi} \diff^{\VEC{\alpha}} \Delta_{-\VEC{h}}^{\VEC{\gamma}} v -
\diff^{\VEC{\alpha}} \left( \overline{\psi}
\Delta_{-\VEC{h}}^{\VEC{\gamma}} v\right)\right) \dx{\VEC{x}}
\]
for all $v \in X_{r_2}$.  Finally, since
$\displaystyle \overline{\psi} \Delta_{-\VEC{h}}^{\VEC{\gamma}} v
\in X_{r_2} \subset X$, we have from (\ref{ell_var_reg}) that
\begin{align*}
B\left(v,\Delta_{\VEC{h}}^{\VEC{\gamma}}(\psi u)\right)
&= (-1)^{j+1} \int_{\Omega_r}
\left( \overline{\psi} \Delta_{-\VEC{h}}^{\VEC{\gamma}} v\right)\,\overline{f}
\dx{\VEC{x}} +E_1 + E_2 + E_3 = E_1 + E_2 + E_3 + E_4
\end{align*}
for all $v \in X_{r_2}$, where
\begin{align*}
E_4(v,u) &= (-1)^{j+1} \int_{\Omega_r}
\left( \Delta_{-\VEC{h}}^{\VEC{\gamma}} v \right)\,\overline{\psi\, f}
\dx{\VEC{x}}
= (-1)^{j+1} \int_{\Omega_r}
\left( \Delta^{\tilde{\VEC{\gamma}}}_{-\VEC{h}}
\left(\Delta_{-h_i \VEC{e}_i} v \right)
\right) \,\overline{\psi\, f} \dx{\VEC{x}} \\
&= - \int_{\Omega_r} \left(\Delta_{h_i \VEC{e}_i} v \right)
\,\overline{\Delta^{\tilde{\VEC{\gamma}}}_{\VEC{h}} (\psi\, f)} \dx{\VEC{x}}
\end{align*}
for all $v \in X_{r_2}$, where we have selected $1 \leq i < n$ such
that $\gamma_i >0$ and set $\tilde{\VEC{\gamma}} = \VEC{\gamma} - \VEC{e}_i$.
So $|\tilde{\VEC{\gamma}}| = j$ with $\tilde{\gamma}_n = 0$.
We have used Proposition~\ref{DiffQuotientAdjoint} for the last
equality in the expression above.

We now find upperbounds of the form
$\displaystyle C_{E_i} \|v\|_{1,2,\Omega_{r_2}}\, \left( \|f\|_{q,2,\Omega_r}
+ \|u\|_{1,2,\Omega_r} \right)$ for each of the $E_i$, where the $C_{E_i}$
are constants.

We first consider $E_1$.  Since $\supp (\psi u) \subset \Omega_{r_1}$
and $\displaystyle \supp \Delta_{\VEC{h}}^\gamma (\psi u) \subset \Omega_{r_2}$,
we may use Schwarz inequality in $\displaystyle L^2(\Omega_r)$ and
Corollary~\ref{ell_reg_cor2} to get a constant $Q_1$ such that
\begin{align*}
&\left| E_1(u,v) \right| \leq 
\sum_{|\VEC{\alpha}|,|\VEC{\beta}|\leq 1}
\left\| \diff^{\VEC{\alpha}} v\right\|_{2,\Omega_{r_2}} \,
\left\|[\Delta_{\VEC{h}}^{\VEC{\gamma}},
b_{\VEC{\alpha},\VEC{\beta}}(\VEC{x})] \diff^{\VEC{\beta}} (\psi u)
\right\|_{2,\Omega_{r_2}} \\
&\qquad \leq Q_1 \|v\|_{1,2,\Omega_{r_2}}
\sum_{\VEC{\eta}<\VEC{\gamma},\, |\VEC{\beta}|\leq 1}
\| \diff^{\VEC{\eta}} \diff^{\VEC{\beta}} (\psi u) \|_{2,\Omega_{r_2}}
\leq Q_1 \|v\|_{1,2,\Omega_{r_2}}
\sum_{\VEC{\eta}<\VEC{\gamma}} \| \diff^{\VEC{\eta}} u \|_{1,2,\Omega_{r_2}} \\
&\qquad \leq Q_1 \|v\|_{1,2,\Omega_{r_2}}
\sum_{|\VEC{\eta}|\leq j,\, \eta_n=0} \| \diff^{\VEC{\eta}} u \|_{1,2,\Omega_{r_2}}
\end{align*}
for all $v \in X_{r_2}$.  It follows from (\ref{ell_reg_B2}), our
induction hypothesis, that
\[
\left| E_1(v,u) \right| \leq C_{E_1} \|v\|_{1,2,\Omega_{r_2}}
\left( \|f\|_{q,2,\Omega_r} + \|u\|_{1,2,\Omega_r} \right)
\]
for all $v \in X_{r_2}$, where
$\displaystyle C_{E_1} = Q_1 C_2 \sum_{|\VEC{\eta}|\leq j,\,\eta_n=0} 1$.

For $E_2$, we have
\[
\diff^{\VEC{\beta}}( \psi u) - \psi \diff^{\VEC{\beta}} u =
\begin{cases}
(\diff^{\VEC{\beta}} \psi ) u  & \quad \text{if} \ |\VEC{\beta}|=1 \\
0 & \quad \text{if} \ |\VEC{\beta}| = 0
\end{cases}
\]
Hence, we get from Schwarz inequality in $\displaystyle L^2(\Omega_r)$ that
\begin{align*}
&\left| E_2(u,v) \right|
\leq \sum_{|\VEC{\beta}|= 1,\,|\VEC{\alpha}|\leq 1}
\left\| \diff^{\VEC{\alpha}} v\right\|_{2,\Omega_{r_2}} \,
\left\| \Delta_{\VEC{h}}^{\VEC{\gamma}} \left( b_{\VEC{\alpha},\VEC{\beta}}
(\diff^{\VEC{\beta}} \psi) u \right)\right\|_{2,\Omega_{r_2}} \\
&\leq \left\|v\right\|_{1,2,\Omega_{r_2}}
\sum_{|\VEC{\beta}|= 1,\,|\VEC{\alpha}|\leq 1} \left\| 
\Delta_{\VEC{h}}^{\VEC{\gamma}} \left( b_{\VEC{\alpha},\VEC{\beta}}
(\diff^{\VEC{\beta}} \psi) u \right) \right\|_{2,\Omega_{r_2}}
\leq P_1 \left\|v\right\|_{1,2,\Omega_{r_2}}
\sum_{|\VEC{\beta}|= 1,\,|\VEC{\alpha}|\leq 1} \left\| 
\diff^{\VEC{\gamma}} \left( b_{\VEC{\alpha},\VEC{\beta}}
(\diff^{\VEC{\beta}} \psi) u \right) \right\|_{2,\Omega_{r_2}} \\
&\leq P_1 \left\|v\right\|_{1,2,\Omega_{r_2}}
\sum_{|\VEC{\beta}|=1,\,|\VEC{\alpha}|\leq 1} \left(
\sum_{\VEC{\eta} \leq \VEC{\gamma}} Q_{\VEC{\alpha},\VEC{\beta},\VEC{\eta}}
\| \diff^{\VEC{\eta}} u \|_{2,\Omega_{r_2}} \right)
= P_1 \left\|v\right\|_{1,2,\Omega_{r_2}}
\sum_{\VEC{\eta} \leq \VEC{\gamma}} \Big(
\sum_{|\VEC{\beta}|=1,\,|\VEC{\alpha}|\leq 1}
Q_{\VEC{\alpha},\VEC{\beta},\VEC{\eta}} \Big)
\left\| \diff^{\VEC{\eta}} u \right\|_{2,\Omega_{r_2}} \\
&\leq P_1 Q_2 \left\|v\right\|_{1,2,\Omega_{r_2}}
\sum_{\VEC{\eta} \leq \VEC{\gamma}}
\| \diff^{\VEC{\eta}} u \|_{2,\Omega_{r_2}}
\leq P_1 Q_2 \left\|v\right\|_{1,2,\Omega_{r_2}}
\sum_{|\VEC{\eta}|\leq j,\,\eta_n=0}
\left\| \diff^{\VEC{\eta}} u \right\|_{1,2,\Omega_{r_2}}
\end{align*}
for all $v \in X_{r_2}$, where $P_1$ is a constant given by
Lemma~\ref{ellRegProp3Omega} with $\Omega = \Omega_{r_2}$ and $k=0$.
Moreover, the constants $Q_{\VEC{\alpha},\VEC{\beta},\VEC{\eta}}$ are
summations of terms of the form
$\displaystyle \sup_{\VEC{x} \in \overline{\Omega}} \left|\diff^{\VEC{\mu}}
b_{\VEC{\alpha},\VEC{\beta}}(\VEC{x})\diff^{\VEC{\beta}+\VEC{\nu}}
\psi(\VEC{x}) \right|$
for $\VEC{\mu}+\VEC{\nu} \leq \VEC{\gamma} - \VEC{\eta}$, and
$\displaystyle Q_2 = \max_{\VEC{\eta} \leq \VEC{\gamma}}
\sum_{|\VEC{\beta}|=1,\,|\VEC{\alpha}|\leq 1}
Q_{\VEC{\alpha},\VEC{\beta},\VEC{\eta}}$.

As for $E_1$, we can use the induction
hypothesis to find a constant $C_{E_2}$ such that
\[
\left| E_2(v,u) \right| \leq C_{E_2} \|v\|_{1,2,\Omega_{r_2}}
\left( \|f\|_{q,2,\Omega_r} + \|u\|_{1,2,\Omega_r} \right)
\]
for all $v \in X_{r_2}$.

For $E_3$, we have
\[
\overline{\psi} \diff^{\VEC{\alpha}} \Delta_{-\VEC{h}}^{\VEC{\gamma}} v -
\diff^{\VEC{\alpha}} \left( \overline{\psi}
\Delta_{-\VEC{h}}^{\VEC{\gamma}} v \right)
= \begin{cases}
- \left( \diff^{\VEC{\alpha}} \overline{\psi} \right) \,
\Delta_{-\VEC{h}}^{\VEC{\gamma}} v & \quad \text{if} \ |\VEC{\alpha}|=1 \\
0 & \quad \text{if} \ |\VEC{\alpha}| = 0
\end{cases}
\]
Hence,
\[
\left| E_3(v,u) \right| = \left| \sum_{|\VEC{\alpha}|=1,\,|\VEC{\beta}|\leq 1}
\int_{\Omega_r} \overline{b_{\VEC{\alpha},\VEC{\beta}}(\VEC{x})
\diff^{\VEC{\beta}} u} \,
\left( \left( \diff^{\VEC{\alpha}} \overline{\psi} \right) \,
\Delta_{-\VEC{h}}^{\VEC{\gamma}} v \right) \dx{\VEC{x}} \right|
\]
for all $v \in X_{r_2}$.  Choose $i$ such that $\gamma_i \neq 0$ and set
$\tilde{\VEC{\gamma}} = \VEC{\gamma} - \VEC{e}_i$.  Using
proposition~\ref{DiffQuotientAdjoint}, we get
\begin{align*}
\left| E_3(v,u) \right|
&= \left| \sum_{|\VEC{\alpha}|=1,\,|\VEC{\beta}|\leq 1}
\int_{\Omega_r} \overline{b_{\VEC{\alpha},\VEC{\beta}}(\VEC{x})
\diff^{\VEC{\beta}} u \,
\diff^{\VEC{\alpha}} \psi} \, \left(\Delta^{\tilde{\VEC{\gamma}}}_{-\VEC{h}}
\left(\Delta_{-h_i \VEC{e}_i} v \right) \right) \dx{\VEC{x}} \right| \\
&= \left| (-1)^{|\tilde{\VEC{\gamma}}|}
\sum_{|\VEC{\alpha}|= 1,\,|\VEC{\beta}|\leq 1}
\int_{\Omega_r} \overline{ \Delta_{\VEC{h}}^{\tilde{\VEC{\gamma}}} \left(
b_{\VEC{\alpha},\VEC{\beta}}(\VEC{x}) \diff^{\VEC{\beta}} u \,
\diff^{\VEC{\alpha}} \psi \right)} \,
\Delta_{-h_i\VEC{e}_i} v \dx{\VEC{x}} \right|
\end{align*}
for all $v \in X_{r_2}$.

It follows from Schwarz inequality in $L^2(\Omega_r)$ that
\begin{align*}
&\left| E_3(v,u) \right| \leq \sum_{|\VEC{\alpha}|= 1,\,|\VEC{\beta}|\leq 1}
\left\| \Delta_{-h_i \VEC{e}_i} v \right\|_{2,\Omega_{r_2}}
\left\| \Delta_{\VEC{h}}^{\tilde{\VEC{\gamma}}} \left(
b_{\VEC{\alpha},\VEC{\beta}}(\VEC{x}) \diff^{\VEC{\beta}} u
\, \diff^{\VEC{\alpha}} \psi \right) \right\|_{2,\Omega_{r_2}} \\
&\qquad \leq P_1^2 \sum_{|\VEC{\alpha}|= 1,\,|\VEC{\beta}|\leq 1}
\left\| \diff_{x_i} v \right\|_{2,\Omega_{r_2}}
\left\| \diff^{\tilde{\VEC{\gamma}}} \left(
b_{\VEC{\alpha},\VEC{\beta}}(\VEC{x}) \diff^{\VEC{\beta}} u \,
\diff^{\VEC{\alpha}} \psi \right) \right\|_{2,\Omega_{r_2}} \\
&\qquad \leq P_1^2
\left\| v \right\|_{1,2,\Omega_{r_2}}
\sum_{|\VEC{\alpha}|= 1 ,|\VEC{\beta}|\leq 1} \left(
\sum_{\VEC{\eta} \leq \tilde{\VEC{\gamma}}}
\tilde{Q}_{\VEC{\alpha},\VEC{\beta},\VEC{\eta} }
\| \diff^{\VEC{\beta}+\VEC{\eta}} u \|_{2,\Omega_{r_2}} \right) \\
&\qquad = P_1^2 \left\|v\right\|_{1,2,\Omega_{r_2}}
\sum_{\VEC{\eta} \leq \tilde{\VEC{\gamma}}} \Big(
\sum_{|\VEC{\alpha}|=1,\,|\VEC{\beta}|\leq 1}
\tilde{Q}_{\VEC{\alpha},\VEC{\beta},\VEC{\eta}} \Big)
\left\| \diff^{\VEC{\eta}} u \right\|_{1,2,\Omega_{r_2}} \\
&\qquad \leq Q_3 \left\| v \right\|_{1,2,\Omega_{r_2}}
\sum_{\VEC{\eta} \leq \tilde{\VEC{\gamma}}} \left\|
\diff^{\VEC{\eta}} u \right\|_{1,2,\Omega_{r_2}}
\leq Q_3 \left\| v \right\|_{1,2,\Omega_r}
\sum_{|\VEC{\eta}| \leq j,\, \eta_n=0} \left\| u \right\|_{1,2,\Omega_{r_2}}
\end{align*}
for all $v \in X_{r_2}$, where $P_1$ is defined above.  Moreover,
the constants $\tilde{Q}_{\VEC{\alpha},\VEC{\beta},\VEC{\eta}}$ are
summations of terms of the form
$\displaystyle \sup_{\VEC{x} \in \overline{\Omega_r}} \left|\diff^{\VEC{\mu}}
b_{\VEC{\alpha},\VEC{\beta}}(\VEC{x})\diff^{\VEC{\alpha}+\VEC{\nu}}
\psi(\VEC{x}) \right|$
for $\VEC{\mu}+\VEC{\nu} \leq \tilde{\VEC{\gamma}} - \VEC{\eta}$, and
$\displaystyle Q_3 = P_1^2 \max_{\VEC{\eta} \leq \tilde{\VEC{\gamma}}}
\sum_{|\VEC{\beta}|=1,\,|\VEC{\alpha}|\leq 1}
\tilde{Q}_{\VEC{\alpha},\VEC{\beta},\VEC{\eta}}$.

As for $E_1$ and $E_2$, we can use the induction
hypothesis to find a constant $C_{E_3}$ such that
\[
\left| E_3(v,u) \right| \leq C_{E_3} \|v\|_{1,2,\Omega_{r_2}}
\left( \|f\|_{q,2,\Omega_r} + \|u\|_{1,2,\Omega_r} \right)
\]
for all $v \in X_r$.

For $E_4$, we first choose an index $i$ such that $\gamma_i \neq 0$.
Using Schwarz inequality in $\displaystyle L^2(\Omega_r)$, we get
\begin{align*}
\left| E_4(v,u) \right| &\leq
\left\| \Delta_{-h_i \VEC{e}_i}^{\VEC{e}_i} v \right\|_{2,\Omega_r} \,
\left\| \Delta_{\VEC{h}}^{\tilde{\VEC{\gamma}}}\,\left( \psi f \right)
\right\|_{2,\Omega_r}
\leq P_1^2 \|\diff_{x_i} v \|_{2,\Omega_r} \,
\left\| \diff^{\tilde{\VEC{\gamma}}}\,\left( \psi f \right)
\right\|_{2,\Omega_r} \\
&\leq P_1^2 \|v \|_{1,2,\Omega_r} \,
\left\| \diff^{\tilde{\VEC{\gamma}}}\,\left( \psi f \right) \right\|_{2,\Omega_r}
\leq C_{E_4} \|v \|_{1,2,\Omega_{r_2}} \, \| f \|_{q,2,\Omega_r}
\end{align*}
for all $v \in X_r$, where $P_1$ is defined above and $C_{E_4}$ is
defined as it follows.  The last inequality comes from
$|\tilde{\VEC{\gamma}}| \leq j \leq k$ since 
we assume that $j < k+1$, and the relation
\[
\left\| \diff^{\tilde{\VEC{\gamma}}}\,\left( \psi f \right) \right\|_{2,\Omega_r}
\leq \sum_{\VEC{\mu}+\VEC{\nu}=\tilde{\VEC{\gamma}}}
\binom{\tilde{\VEC{\gamma}}}{\VEC{\nu}}
\left\| \diff^{\VEC{\mu}} f \right\|_{2,\Omega_r}\,
\left\|\diff^{\VEC{\nu}} \psi \right\|_{2,\Omega_r}
\leq \left( \sum_{\VEC{\nu} \leq\tilde{\VEC{\gamma}}}
\binom{\tilde{\VEC{\gamma}}}{\VEC{\nu}}
\left\|\diff^{\VEC{\nu}} \psi \right\|_{2,\Omega_r} \right)
\left\| f \right\|_{q,2,\Omega_r} \ .
\]
We therefore have that $\displaystyle C_{E_4} = P_1^2 
\sum_{\VEC{\nu} \leq\tilde{\VEC{\gamma}}} \binom{\tilde{\VEC{\gamma}}}{\VEC{\nu}}
\left\|\diff^{\VEC{\nu}} \psi \right\|_{2,\Omega_r}$.  Hence,
\[
\left| E_4(v,u) \right| \leq C_{E_4} \|v\|_{1,2,\Omega_{r_2}}
\left( \|f\|_{q,2,\Omega_r} + \|u\|_{1,2,\Omega_r} \right)
\]
for all $v \in X_r$.

We have proved that (\ref{ell_reg_B3}) with
$C_3 = C_{E_1} + C_{E_2} + C_{E_3} + C_{E_4}$.  Note that $C_3$
depends indirectly of $\rho$ because $C_3$ depends of $r_2$ and
$r_2 = (\rho + 2r)/3$.

\stage{ii} Let $C_0$ and $\lambda$ be the two constants in the
definition of a weakly coercive form applied to $B$; namely,
\begin{equation} \label{ell_wcREG}
\RE B(v,v) \geq C_0 \|v\|_{1,2}^2 - \lambda \|v\|_2^2
\end{equation}
for all $v \in X_r$.  If we substitute
$\displaystyle v = \Delta^{\VEC{\alpha}}_{\VEC{h}} (\psi u)$ in
(\ref{ell_wcREG}) and use (\ref{ell_reg_B3}), we get that
\begin{align*}
\left\| \Delta^{\VEC{\gamma}}_{\VEC{h}} (\psi u) \right\|_{1,2,\Omega_{r_2}}^2
&\leq \frac{1}{C_0} \left( \left| B\left(\Delta^{\VEC{\gamma}}_{\VEC{h}} (\psi u),
\Delta^{\VEC{\gamma}}_{\VEC{h}} (\psi u)\right)\right| + \lambda
\left\|\Delta^{\VEC{\gamma}}_{\VEC{h}}
(\psi u)\right\|_{2,\Omega_{r_2}}^2 \right)\\
&\leq \frac{1}{C_0} \left(
C_3 \left\|\Delta^{\VEC{\gamma}}_{\VEC{h}} (\psi u)\right\|_{1,2,\Omega_{r_2}}
\left( \|f\|_{q,2,\Omega_r} + \|u\|_{1,2,\Omega_r}\right) + \lambda
\left\|\Delta^{\VEC{\gamma}}_{\VEC{h}}
(\psi u)\right\|_{2,\Omega_{r_2}}^2 \right) \\
&\leq \frac{1}{C_0} \left(
C_3 \left\|\Delta^{\VEC{\gamma}}_{\VEC{h}} (\psi u)\right\|_{1,2,\Omega_{r_2}}
\left( \|f\|_{q,2,\Omega_r} + \|u\|_{1,2,\Omega_r}\right) \right. \\
&\qquad \left. + \lambda
\left\|\Delta^{\VEC{\gamma}}_{\VEC{h}} (\psi u)\right\|_{2,\Omega_{r_2}}\,
\left\|\Delta^{\VEC{\gamma}}_{\VEC{h}}
(\psi u)\right\|_{1,2,\Omega_{r_2}} \right) \ .
\end{align*}
Hence,
\[
\left\| \Delta^{\VEC{\gamma}}_{\VEC{h}} (\psi u) \right\|_{1,2,\Omega_{r_2}}
\leq \frac{1}{C_0} \left(
C_3 \left( \|f\|_{q,2,\Omega_r} + \|u\|_{1,2,\Omega_r}\right) + \lambda
\left\|\Delta^{\VEC{\gamma}}_{\VEC{h}} (\psi u)\right\|_{2,\Omega_{r_2}}\,
\right) \ .
\]
Choose $i$ such that $\gamma_i \neq 0$ and set
$\tilde{\VEC{\gamma}} = \VEC{\alpha} - \VEC{e}_i$.
From (\ref{ell_reg_B2}), that we can use because
$|\tilde{\VEC{\gamma}}|\leq j$,
we find that
\begin{align*}
\left\| \Delta^{\VEC{\gamma}}_{\VEC{h}} (\psi u) \right|_{2,\Omega_{r_2}}
&= \left\| \Delta_{h_i\VEC{e}_1}^{\VEC{e}_i} \Delta^{\tilde{\VEC{\gamma}}}_{\VEC{h}}
(\psi u) \right\|_{2,\Omega_{r_2}}
\leq P_1 \left\| \diff^{\VEC{e}_i} \left( \Delta^{\tilde{\VEC{\gamma}}}_{\VEC{h}}
(\psi u) \right) \right\|_{2,\Omega_{r_2}}
\leq P_1 \left\| \Delta^{\tilde{\VEC{\gamma}}}_{\VEC{h}}
(\psi u) \right\|_{1,2,\Omega_{r_2}} \\
&\leq P_1 P_2 \left\| \diff^{\tilde{\VEC{\gamma}}} (\psi u)
\right\|_{1,2,\Omega_{r_2}}
\leq C_4 \left( \|f\|_{q,2,\Omega_r} + \|u\|_{1,2,\Omega_r} \right)
\end{align*}
for $\|\VEC{h}\| < (r_2-r_1)/(j+1)$, where $P_1$ is defined above,
$P_2$ is a constant given by Lemma~\ref{ellRegProp3Omega} with
$\Omega = \Omega_{r_2}$ and $k=1$, and $C_4 = P_1 P_2 C_2$.  Hence,
\begin{align*}
\left\| \Delta^{\VEC{\gamma}}_{\VEC{h}} (\psi u) \right\|_{1,2,\Omega_{r_2}}
&\leq \frac{1}{C_0} \big(
C_3 \left( \|f\|_{q,2,\Omega_r} + \|u\|_{1,2,\Omega_r}\right) + \lambda
C_4 \left( \|f\|_{q,2,\Omega_r} + \|u\|_{1,2,\Omega_r} \right) \big) \\
&\leq \frac{1}{C_0} \left( C_3 + \lambda C_4 \right)
\left( \|f\|_{q,2,\Omega_r} + \|u\|_{1,2,\Omega_r}\right) \ .
\end{align*}
With $\displaystyle C_5 = \max \left\{ C_2 ,
\frac{1}{C_0} \left( C_3 + \lambda C_4 \right) \right\}$, we get
\begin{equation} \label{ell_reg_fn1}
\left\| \Delta^{\VEC{\gamma}}_{\VEC{h}} (\psi u) \right\|_{1,2,\Omega_{r_2}}
\leq C_5 \left( \|f\|_{q,2,\Omega_r} + \|u\|_{1,2,\Omega_r}\right)
\end{equation}
for all multi-indices $\VEC{\gamma}$ such that $|\VEC{\gamma}|\leq j+1$ and
$\gamma_n=0$, and all $\VEC{h}$ such that $\|\VEC{h}\|< (r_2-r_1)/(j+1)$.

We have from Corollary~\ref{ell_reg_cor3} that
$\displaystyle \diff^{\VEC{\gamma}} (\psi u) \in H^{1,2}(\Omega_{r_2})$
because
$\displaystyle \limsup_{\VEC{h}\rightarrow \VEC{0}}
\|\Delta^{\VEC{\gamma}}_{\VEC{h}} f\|_{q,2,\Omega_r}
\leq C_5 \left( \|f\|_{q,2,\Omega_r} + \|u\|_{1,2,\Omega_r}\right) < \infty$,
and there exists a constant $C_6$ such that
\begin{equation} \label{ell_reg_fn2}
\left\| \diff^{\VEC{\gamma}} (\psi u) \right\|_{1,2,\Omega_{r_2}} \leq C_6
\left\| \Delta^{\VEC{\gamma}}_{\VEC{h}} (\psi u) \right\|_{1,2,\Omega_{r_2}}
\end{equation}
for some (in fact infinitely many) values of $\VEC{h}$ such that
$\|\VEC{h}\| < (r_2-r_1)/(j+1)$.  Since $\psi = 1$ on $\Omega_\rho$, we
have that
\begin{equation} \label{ell_reg_fn3}
\left\| \diff^{\VEC{\gamma}} u \right\|_{1,2,\Omega_{\rho}} \leq
\left\| \diff^{\VEC{\gamma}} (\psi u) \right\|_{1,2,\Omega_{r_2}} \ .
\end{equation}
It follows from (\ref{ell_reg_fn1}), (\ref{ell_reg_fn2}) and
(\ref{ell_reg_fn3}) that
\[
\left\| \diff^{\VEC{\gamma}} u \right\|_{1,2,\Omega_{\rho}} \leq C_1
\left( \|f\|_{q,2,\Omega_r} + \|u\|_{1,2,\Omega_r}\right)
\]
for all multi-indices $\VEC{\gamma}$ such that $|\VEC{\gamma}|\leq j+1$ and
$\gamma_n =0$, where $C_1 = C_5 C_6$.  This complete the proof of
claim A.

\subI{Claim B} Assume that $\rho<r$, $j\leq q+2$ and
$\VEC{\gamma}$ is a multi-index such that $|\VEC{\gamma}|\leq q+2$ and
$\gamma_n \leq j$.  Then,
$\displaystyle \diff^{\VEC{\gamma}} u \in L^2(\Omega_\rho)$ and there
exists a constant $C_7=C_7(j,\rho)$ such that
\begin{equation} \label{ell_claim2a}
\left\|\diff^{\VEC{\gamma}} u \right\|_{2,\Omega_\rho} \leq C_7 \left(
\left\|f\right\|_{q,2,\Omega_r} + \left\|u\right\|_{1,2,\Omega_r} \right) \ .
\end{equation}

The proof is by induction on $j$.  Claim A shows
that this new claim is true for $j=0$ and $j=1$.  If $\gamma_n=0$, let
$\tilde{\VEC{\gamma}} = \VEC{\gamma} - \VEC{e}_i$ for one index $0\leq i <n$
such that $\alpha_i>0$.  If $\gamma_n=1$, let
$\tilde{\VEC{\gamma}} = \VEC{\gamma} - \VEC{e}_n$.
Then $|\tilde{\VEC{\gamma}}|<q+2$ with $\tilde{\gamma}_n =0$.  Hence, from
claim A, we have that
\[
\left\|\diff^{\VEC{\gamma}} u \right\|_{2,\Omega_\rho}
\leq \left\|\diff^{\tilde{\VEC{\gamma}}} u \right\|_{1,2,\Omega_\rho}
\leq C_1(q+1,\rho) \left( \left\|f\right\|_{q,2,\Omega_r}
+ \left\|u\right\|_{1,2,\Omega_r} \right) \ .
\]
So $C_7(j,\rho) = C_1(q+1,\rho)$ for $j=0$ and $j=1$ is acceptable. 

Suppose that $j\geq 2$ and that the claim is true for $\gamma_n<j$.
Consider $\VEC{\gamma}$ such that $|\VEC{\gamma}| \leq q+2$ and
$\gamma_n = j$.  Let $\tilde{\VEC{\gamma}} = \VEC{\gamma} - 2 \VEC{e}_n$.
Since the operator
$\displaystyle
L(\VEC{x},\diff)u = \sum_{|\VEC{\alpha}|\leq 2} a_{\VEC{\alpha}}(\VEC{x})
\diff^{\VEC{\alpha}} u$,
the initial formulation of the variational problem
(\ref{ell_var_reg}), is elliptic on $\overline{\Omega}$, we must have
for $\VEC{\beta} = (0,0,\ldots,0,2)$ that
$a_{\VEC{\beta}}(\VEC{x}) \neq 0$ for all $\VEC{x} \in \overline{\Omega}$.
We may therefore rewrite $L(\VEC{x},\diff)u = f$ as
\[
\diff^{\VEC{\beta}} u = \frac{1}{a_{\VEC{\beta}}} \left( f -
\sum_{|\VEC{\alpha}|\leq 2,\, \alpha_n\neq 2} a_{\VEC{\alpha}}
\diff^{\VEC{\alpha}} u \right)
\]
on $\Omega$.  Thus,
\begin{align*}
\diff^{\VEC{\gamma}} u &= \diff^{\tilde{\VEC{\gamma}}} \diff^{\VEC{\beta}} u
= \diff^{\tilde{\VEC{\gamma}}}\left(
\frac{1}{a_{\VEC{\beta}}} f \right) - \diff^{\tilde{\VEC{\gamma}}}
\left( \frac{1}{a_{\VEC{\beta}}}
\sum_{|\VEC{\alpha}|\leq 2,\, \alpha_n\neq 2} a_{\VEC{\alpha}}
\diff^{\VEC{\alpha}} u \right) \\
&= \sum_{|\VEC{\alpha}|\leq q} g_{\VEC{\alpha}} \diff^{\VEC{\alpha}} f
- \sum_{|\VEC{\alpha}|\leq q+2,\, \alpha_n\leq j-1} h_{\VEC{\alpha}}
\diff^{\VEC{\alpha}} u \ ,
\end{align*}
where the $\displaystyle g_{\VEC{\alpha}}, h_{\VEC{\alpha}}
\in C^\infty(\overline{\Omega})$
for all $\VEC{\alpha}$.  The constraints on the indices for the second sum come
from $\tilde{\gamma}_n \leq j-2$ and $\alpha_n <2$.  Hence,
\begin{align*}
\left\| \diff^{\VEC{\gamma}} u \right\|_{2,\Omega_\rho}
&\leq \sum_{|\VEC{\alpha}|\leq q} \left(
\sup_{\VEC{x}\in \overline{\Omega}_r} \left| g_{\VEC{\alpha}}(\VEC{x}) \right|
\left\| \diff^{\VEC{\alpha}} f \right\|_{2,\Omega_\rho}\right)
+ \sum_{|\VEC{\alpha}|\leq q+2,\, \alpha_n\leq j-1} \left(
\sup_{\VEC{x}\in \overline{\Omega}_r} \left| h_{\VEC{\alpha}}(\VEC{x}) \right|
\left\| \diff^{\VEC{\alpha}}u \right\|_{2,\Omega_\rho} \right) \ .
\end{align*}
Since
\[
\left\| \diff^{\VEC{\alpha}} f \right\|_{2,\Omega_\rho} \leq
\left\| f \right\|_{q,2,\Omega_\rho} \leq
\left\| f \right\|_{q,2,\Omega_r}
\]
for $|\VEC{\alpha}|\leq q$ and
\[
\left\| \diff^{\VEC{\alpha}} u \right\|_{2,\Omega_\rho} \leq
C_7(j-1,\rho) \left(
\left\|f\right\|_{q,2,\Omega_r} + \left\|u\right\|_{1,2,\Omega_r} \right)
\]
for $|\VEC{\alpha}|\leq q+2$ with $\alpha_n\leq j-1$ by induction, we get
\[
\left\| \diff^{\VEC{\gamma}} u \right\|_{2,\Omega_\rho}
\leq A \left\| f \right\|_{q,2,\Omega_r} +
B \left( \left\|f\right\|_{q,2,\Omega_r} +
\left\|u\right\|_{1,2,\Omega_r} \right) \ ,
\]
where
$\displaystyle A = \sum_{|\VEC{\alpha}|\leq q}
\sup_{\VEC{x}\in \overline{\Omega}_r} \left| g_{\VEC{\alpha}}(\VEC{x}) \right|$
and
$\displaystyle B = C_7(j-1,\rho) \sum_{|\VEC{\alpha}|\leq q+2,\, \alpha_n\leq j-1}
\sup_{\VEC{x}\in \overline{\Omega}_r} \left| h_{\VEC{\alpha}}(\VEC{x}) \right|$.
Hence,
\[
\left\|\diff^{\VEC{\gamma}} u \right\|_{2,\Omega_\rho} \leq C_7(j,\rho) \left(
\left\|f\right\|_{q,2,\Omega_r} + \left\|u\right\|_{1,2,\Omega_r}
\right)
\]
with $C_7(j,\rho) = A+B$.  This complete the proof of Claim B.

\stage{C} To complete the proof, let
$\displaystyle \tilde{C} = \max_{0\leq j \leq q+2} C_7(j.\rho)$, where the
$C_7(j,\rho)$ are given by claim B.  Thus
$\displaystyle \diff^{\VEC{\alpha}} u \in L^2(\Omega_\rho)$ and
\[
\left\|\diff^{\VEC{\alpha}} u \right\|_{2,\Omega_\rho} \leq \tilde{C} \left(
\left\|f\right\|_{q,2,\Omega_r} + \left\|u\right\|_{1,2,\Omega_r}
\right)
\]
for all multi-indices $\VEC{\alpha}$ such that
$|\VEC{\alpha}|\leq q+2$.  It follows
that $u \in H^{k+2,2}(\Omega_\rho)$ and
\[
\left\|u \right\|_{q+2,2,\Omega_\rho} \leq C \left(
\left\|f\right\|_{q,2,\Omega_r} + \left\|u\right\|_{1,2,\Omega_r}
\right)
\]
for $\displaystyle C = \tilde{C}
\Big(\sum_{|\VEC{\alpha}|\leq q+2} 1\Big)^{1/2}$.
\end{proof}

Before giving the proof of Theorem~\ref{ell_regular}, we
need to provide a lemma like Lemma~\ref{ell_reg_in_lem1} but for
$\Omega$ a bounded open subset of $\RR^n$.

\begin{lemma} \label{ell_reg_out_lem1}
We consider the elliptic operator $L(\VEC{x},\diff)$ defined in
(\ref{ell_reg_ellOp}).  Suppose that $\Omega$ is a bounded open subset
of $\displaystyle \RR^n$, $\phi \in \DD(\Omega)$ and $q\in \NN$.
The operator $\left[ L(\VEC{x},\diff), \phi\right]$ defined by
\[
\left[ L(\VEC{x},\diff), \phi\right](v)
= L(\VEC{x},\diff) (\phi\, v) - \phi\, L(\VEC{x},\diff) v
\]
for $\displaystyle v \in \displaystyle H^{q,2}(\Omega)$ is a bounded
mapping from $\displaystyle H^{q,2}(\Omega)$ into
$\displaystyle H^{q-k+1,2}(\Omega)$.
\end{lemma}

\begin{proof}
The proof is identical to the proof of Lemma~\ref{ell_reg_in_lem1}
with $\displaystyle \RR^n$ replaced by $\Omega$ except that
(\ref{ellRegInLem1Eq1}) becomes
\[
\left\| A_{\VEC{\eta}}(\VEC{x}) \diff^{\VEC{\eta}} v \right\|_{q-k+1,2,\Omega}
\leq C_{\VEC{\eta}} \left\| \diff^{\VEC{\eta}} v \right\|_{q-k+1,2,\Omega}
\leq C_{\VEC{\eta}} \left\| v \right\|_{q-k+|\VEC{\eta}|+1,2,\Omega}
\leq C_{\VEC{\eta}} \left\| v \right\|_{q,2,\Omega} \ ,
\]
where Proposition~\ref{sob_T_dd_wkp} is used to get the constants
$C_{\VEC{\eta}}$ and the definition of the norm $\|\cdot\|_{m,2,\Omega}$
with $m\in \NN$ is used to get the second inequality.
\end{proof}

\pdfF{elliptic/reg_boundary}{Regularity at the boundary}{
Representation local of the boundary of $\Omega$ as the image of the
hyperplane $\displaystyle \RR^n_0$ near the origin.}{REG_BDR}

\begin{proof}[Proof (Theorem~\ref{ell_regular}).]
Because of the smoothness of the boundary of $\Omega$, for
each $\VEC{z} \in \partial \Omega$, there exist an open ball
$B_{\eta_{\VEC{z}}}(\VEC{z})$ and a $\displaystyle C^{q+2}$-diffeomorphisms
$\displaystyle \psi_{\VEC{z}}:B_{\eta_{\VEC{z}}}(\VEC{z}) \rightarrow \RR^n$
such that
\begin{enumerate}
\item $\psi_\VEC{z}(\VEC{z})=\VEC{0}$,
\item $\displaystyle \psi_\VEC{z}(B_{\eta_\VEC{z}}(\VEC{z})\cap \Omega)
\subset \RR^n_+$, and
\item $\displaystyle \psi_\VEC{z}(B_{\eta_\VEC{z}}(\VEC{z})\cap
\partial \Omega) \subset \RR^n_0=\{\VEC{x} \in \RR^n : x_n=0 \}$.
\end{enumerate}
Choose $r_\VEC{z}$ small enough such that
$\displaystyle \Omega_{r_\VEC{z}} = B_{r_{\VEC{z}}}(\VEC{0}) \cap \RR^n_+
\subset \psi_\VEC{z}(B_{\eta_\VEC{z}}(\VEC{z})\cap \Omega)$,
and $\rho_{\VEC{z}} < r_\VEC{z}$ (Figure~\ref{REG_BDR}).

Since $\overline{\partial \Omega}$ is compact because $\partial \Omega$
is bounded,, we can cover $\partial \Omega$ with a finite
number of open sets of the form
$\displaystyle \psi_{\VEC{z}}^{-1}(B_{\rho_\VEC{z}}(\VEC{0}))$
as described above.  Let $V_1$, $V_2$, \ldots, $V_s$ be those open sets.

Since $\displaystyle K = \overline{\Omega} \setminus \bigcup_{j=1}^s V_j$
is a compact subset of $\Omega$, there exist an open set $V_0$ such
that $K \subset V_0 \subset \overline{V_0} \subset \Omega$.
Then, $\displaystyle \{ V_j \}_{j=0}^s$ is an open cover of $\Omega$.

\stage{i}
We have that $\displaystyle u \in H^{q+2,2}(V_0)$ from
Theorem~\ref{ell_int_regTH}.  To get the inequality
\[
\|u\|_{q+2,2,V_0} \leq C_0 \left( \|f\|_{q,2,\Omega} + \|u\|_{2,\Omega} \right)
\]
for some constant $C_0$, we choose a sequence
$\displaystyle \{ W_j\}_{j=0}^{q+1}$ of open subsets of $\Omega$ such that
\[
V_0 = W_0 \subset \overline{W_0} \subset W_1 \subset \overline{W_1}
\subset W_2 \subset \ldots \subset \overline{W_q} \subset
W_{q+1} = \Omega \  .
\]
For each $0\leq j \leq q$, choose $\psi_j \in \DD(\Omega)$ such
that $\supp \psi_j \subset W_{j+1}$ and $\psi = 1$ on $W_j$.
From Corollary~\ref{ell_cor_convex} with $t=0$, there exists
for each $0\leq j \leq q$ a constant $C_{1,j}$ such that
\begin{align*}
&\| u \|_{q+2-j,2,W_j} \leq \| \psi_j u \|_{q+2-j,2,W_{j+1}}
\leq C_{1,j} \left( \|L(\VEC{x},\diff) (\psi_j u)\|_{q-j,2,W_{j+1}} +
\|\psi_j u \|_{2,W_{j+1}} \right) \\
&\quad \leq C_{1,j} \left( \| \psi_j L(\VEC{x},\diff) u\|_{q-j,2,W_{j+1}}
+ \| [L(\VEC{x},\diff),\psi_j]  u \|_{q-j,2,W_{j+1}}
 + \|\psi_j u \|_{2,W_{j+1}} \right) \ .
\end{align*}
The reason why we could use Corollary~\ref{ell_cor_convex} was that
that $\supp (\psi_j u) \subset W_{j+1}$ and so 
$\displaystyle \psi_j u \in H^{q+2-j,2}_0(W_{j+1})$ according to
Lemma~\ref{sob_w0_trad1}.

We have from Lemma~\ref{ell_reg_out_lem1} that
\[
\| [ L(\VEC{x},\diff) , \psi_j] u\|_{q-j,2,W_{j+1}}
\leq C_{2,j} \|u\|_{q+1-j,2,W_{j+1}}
\]
for some constant $C_{2,j}$.  We end up with
\begin{align}
\| u \|_{q+2-j,2,W_j} &\leq C_{1,j} \left( \| L(\VEC{x},\diff) u\|_{q-j,2,W_{j+1}}
+ C_{2,j} \|u\|_{q+1-j,2,W_{j+1}} + \| u \|_{2,W_{j+1}} \right) \nonumber \\
&\leq C_{1,j} \left( \|f\|_{q,2,\Omega}
+ C_{2,j} \|u\|_{q+1-j,2,W_{j+1}} + \| u \|_{2,\Omega} \right) \nonumber \\
&\leq C_{1,j} \|f\|_{q,2,\Omega}
+ C_{1,j} C_{2,j} \|u\|_{q+1-j,2,W_{j+1}} + C_{1,j} \| u \|_{2,\Omega}
\label{ell_ind_RT1}
\end{align}
for $0\leq j \leq q$.

We prove by induction that, for each $0\leq j \leq q$, there
exists a constant $K_j$ such
\begin{align} \label{ell_ind_RT2}
\|u\|_{q+2-j,2,W_j} \leq K_j \left( \|f\|_{q,2,\Omega} + \|u\|_{1,2,\Omega} \right)
\end{align}
The result is true for $j=q$ since (\ref{ell_ind_RT1}) with $j=q$ yields
\begin{align*}
\| u \|_{2,2,W_q} &\leq C_{1,q} \|f\|_{q,2,\Omega}
+ C_{1,q} C_{2,q} \|u\|_{1,2,W_{q+1}} + C_{1,q} \| u \|_{2,\Omega} \\
&\leq C_{1,q} \|f\|_{q,2,\Omega} + \left( C_{1,q} C_{2,q} + C_{1,q}\right)
\| u \|_{1,2,\Omega}
\leq K_q \left( \|f\|_{q,2,\Omega} + \|u\|_{1,2,\Omega} \right) \ ,
\end{align*}
where $K_q = C_{1,q} C_{2,q} + C_{1,q}$.  Suppose that
(\ref{ell_ind_RT2}) is true for $j$ between $1$ and $q$.  From
(\ref{ell_ind_RT1}) with $j$ replaced by $j-1$, we get
\begin{align*}
\| u \|_{q+3-j,2,W_{j-1}} &\leq C_{1,j-1} \|f\|_{q,2,\Omega}
+ C_{1,j-1} C_{2,j-1} \|u\|_{q+2-j,2,W_j} + C_{1,j-1} \| u \|_{2,\Omega} \\
&\leq C_{1,j-1} \|f\|_{q,2,\Omega}
+ C_{1,j-1} C_{2,j-1} K_j \left( \|f\|_{q,2,\Omega} + \|u\|_{1,2,\Omega} \right)
+ C_{1,j-1} \| u \|_{2,\Omega} \\
&\leq \left( C_{1,j-1} + C_{1,j-1} C_{2,j-1} K_j \right) \|f\|_{q,2,\Omega}
+ \left( C_{1,j-1} C_{2,j-1} K_j + C_{1,j-1} \right)\| u \|_{1,2,\Omega} \ ,
\end{align*}
where we have used the hypothesis of induction for the second
inequality.  If
$\displaystyle K_{j-1} = C_{1,j-1} + C_{1,j-1} C_{2,j-1} K_j$,
we get (\ref{ell_ind_RT2}) for $i$ replaced by $i-1$.  This complete
the proof by induction.
Hence, with $j=0$ in (\ref{ell_ind_RT2}), we get
\begin{equation} \label{ell_ind_RT3}
\|u\|_{q+2,2,V_0} = \|u\|_{q+2,2,W_0} \leq K_0
\left( \|f\|_{q,2,\Omega} + \|u\|_{1,2,\Omega} \right) \ .
\end{equation}

\stage{ii}
Suppose that $V_j = \psi_{\VEC{z}}^{-1}(B_{\rho_\VEC{z}}(\VEC{0}))$
for some $\VEC{z} \in \partial \Omega$ as described at the beginning
of the proof and where $0<j \leq s$.  For this step, we sketch the
reasoning and leave the many details to the reader.  According to
Proposition~\ref{sob_cv}, we may use the isomorphism
\begin{align*}
T_{\VEC{z}}:  H^{q,2}\big(\Omega_{r_{\VEC{z}}}\big) & \to 
H^{q,2}\big(\psi_{\VEC{z}}^{-1}(\Omega_{r_{\VEC{z}}})\big) \\
g &\mapsto g \circ \psi_{\VEC{z}}
\end{align*}
to express (\ref{ell_var_reg}) locally as the variational problem
\[
\tilde{B}(\tilde{v},\tilde{u})
= B(T_{\VEC{z}}(\tilde{v}),T_{\VEC{z}}(\tilde{u}))
= \int_{\Omega_{r_\VEC{z}}} \overline{f(\psi^{-1}(\VEC{x}))}\,\tilde{v}(\VEC{x})
\left|\det \diff \psi_{\VEC{z}}^{-1}(\VEC{x})\right| \dx{\VEC{x}} \ ,
\]
where $\tilde{B}$ is still weekly coercive and
$\tilde{v} \in \displaystyle X_r = \left\{ \tilde{v} \in X : \supp u \subset
B_\rho(0)\cap \RR^n_+ \ \text{for some} \ \rho < r_{\VEC{z}} \right\}$
with $\displaystyle X = H^{q,2}(\RR^n_+)$ or
$\displaystyle X = H^{q,2}_0(\RR^n_+)$.
We can then conclude from Proposition~\ref{ell_regular_Rnp} that
$\displaystyle \tilde{u}\in H^{q+2,2}\big(\Omega_{\rho_\VEC{z}}\big)$ and there
exists a constant $\tilde{K}_j = C(j,q,\rho_{\VEC{z}})$ such that
\[
\|\tilde{u}\|_{q+2.2,\Omega_{\rho_\VEC{z}}} \leq \tilde{K}_j
\left( \|f\circ \psi^{-1}\|_{q,2,\Omega_{r_\VEC{z}}} +
\|\tilde{u}\|_{1,2,\Omega_{r_\VEC{z}}} \right) \ .
\]
We then have that
$\displaystyle u = T_{\VEC{z}}(\tilde{u}) \in H^{q+2,2}
\big(\psi_{\VEC{z}}^{-1}(\Omega_{\rho_\VEC{z}})\big)
= H^{q+2,2} \big(V_j \cap \Omega\big)$.

Since $T_{\VEC{z}}$ and $T_{\VEC{z}}^{-1}$ are bounded mappings, there
exist constants $C_1$ and $C_2$ such that
$\displaystyle
\|T_{\VEC{z}}(\tilde{v})\|_{q,2,\psi^{-1}_{\VEC{z}}(\Omega_{r_\VEC{z}})}
\leq C_1 \|\tilde{v}\|_{q,2,\Omega_{r_\VEC{z}}}$
for $\displaystyle \tilde{v} \in H^{q,2}\big(\Omega_{r_{\VEC{z}}}\big)$
and
$\displaystyle
\|T_{\VEC{z}}^{-1}(v)\|_{q,2,\Omega_{r_\VEC{z}}}
\leq C_2 \|v\|_{q,2,\psi^{-1}_{\VEC{z}}(\Omega_{r_\VEC{z}})}$ and
for $\displaystyle v \in
H^{q,2}\big(\psi_{\VEC{z}}^{-1}\big(\Omega_{r_{\VEC{z}}})\big)$.
We then get that
\[
  \frac{1}{C_2} \|\tilde{v}\|_{q,2,\Omega_{r_\VEC{z}}}
\leq \|T_{\VEC{z}}(\tilde{v})\|_{q,2,\psi^{-1}_{\VEC{z}}(\Omega_{r_\VEC{z}})}
\leq C_1 \|\tilde{v}\|_{q,2,\Omega_{r_\VEC{z}}}
\]
for $\displaystyle \tilde{v} \in H^{q,2}\big(\Omega_{r_{\VEC{z}}}\big)$.
We can use this equivalence to find a constant $K_j$ such that
\[
\|u\|_{q+2.2,V_j\cap \Omega}
= \|u\|_{q+2.2,\psi^{-1}_{\VEC{z}}(\Omega_{r_\VEC{z}})}
\leq K_j \left( \|f\|_{q,2,\psi^{-1}_{\VEC{z}}(\Omega_{r_\VEC{z}})} +
\|u\|_{1,2,\psi_\VEC{z}^{-1}(\Omega_{r_\VEC{z}})} \right)
\leq K_j \left( \|f\|_{q,2,\Omega} + \|u\|_{1,2,\Omega} \right) \ .
\]

\stage{iii} Let $\displaystyle \{\phi_j\}_{j=1}^s$ be a partition of unity of
$\Omega$ subordinated to $\displaystyle \{ V_j \}_{j=0}^s$.
We have that $\displaystyle u = \sum_{j=0}^s \phi_j u$, where
$\supp (\phi_j u) \subset V_j \cap \Omega$ for all $j$.  Hence,
from (i) and (ii), we have that $\displaystyle u\in H^{q+2,2}(\Omega)$ and
\begin{align}
\|u\|_{q+2.2,\Omega} &\leq \sum_{j=0}^s \| \phi_j u\|_{q+2,2,\Omega}
= \sum_{j=0}^s \| \phi_j u\|_{q+2,2,\Omega \cap V_j}
\leq \sum_{j=0}^s C_j \|u\|_{q+2,2,\Omega \cap V_j} \nonumber \\
&\leq \sum_{j=0}^s C_j K_j \left( \|f\|_{q,2,\Omega} + \|u\|_{1,2,\Omega} \right)
= K \left( \|f\|_{q,2,\Omega} + \|u\|_{1,2,\Omega} \right)  \label{ell_ind_RT5}
\end{align}
where $\displaystyle K = \sum_{j=0}^s C_j K_j$.  We have used
Proposition~\ref{sob_T_dd_wkp} to obtain the constants $C_j$.

\stage{iv} Finally, It follows from the definition of weekly coercive
applied to the sequilinear form $B$, that there exist two constants
$C_0$ and $\lambda$ such that
\[
\|u \|_{1,2,\Omega}^2
\leq \frac{1}{C_0} \left( B(u,u) + \lambda \|u\|_{2,\Omega}^2\right) \ .
\]
Recall that we prove the theorem only for second order elliptic
operators, thus $k=1$.  Using $B(v,u) = \ps{v}{f}_2$ for $v\in X$ and
Schwarz inequality, we then get
\begin{align*}
\|u \|_{1,2,\Omega}^2 &\leq
\frac{1}{C_0} \left( \int_{\Omega} f(\VEC{x}) u(\VEC{x}) \dx{\VEC{x}}
+ \lambda \|u\|_{2,\Omega}^2\right)
\leq \frac{1}{C_0} \left( \|f\|_{2,\Omega} \, \| u \|_{2,\Omega}
+\lambda \|u\|_{2,\Omega}^2\right)\\
&= \frac{1}{C_0} \|u\|_2 \left( \|f\|_{2,\Omega} +\lambda \|u\|_{2,\Omega}\right)
\leq \frac{1}{C_0} \|u\|_{1,2} \left( \|f\|_{2,\Omega}
+\lambda \|u\|_{2,\Omega}\right) \ .
\end{align*}
Hence,
\[
\|u \|_{1,2,\Omega} \leq \frac{1}{C_0} \left( \|f\|_{2,\Omega}
  +\lambda \|u\|_{2,\Omega} \right)
\leq \frac{1}{C_0} \left( \|f\|_{q,2,\Omega} +\lambda \|u\|_{2,\Omega} \right) \ .
\]
If we combine this inequality with (\ref{ell_ind_RT5}), we get
\begin{align*}
\|u\|_{q+2.2,\Omega} &\leq 
K \left( \|f\|_{q,2,\Omega} + \|u\|_{1,2,\Omega} \right)
\leq K \left( \|f\|_{q,2,\Omega} + \frac{1}{C_0} \left( \|f\|_{q,2,\Omega}
+\lambda \|u\|_{2,\Omega}\right) \right) \\
&= K \left(1+ \frac{1}{C_0} \right) \|f\|_{q,2,\Omega}
+ K \lambda \|u\|_{2,\Omega}
\leq C \left( \|f\|_{q,2,\Omega} + \|u\|_{2,\Omega}\right) \ ,
\end{align*}
where $\displaystyle C = \max\left\{K\lambda,
K \left(1+ \frac{1}{C_0}\right)\right\}$.
This complete the proof of Theorem~\ref{ell_regular}.
\end{proof}

\begin{cor} \label{ell_fCi_uCi}
In Theorem~\ref{ell_regular}, if $f \in C^\infty(\overline{\Omega})$
and $\partial \Omega$ is a $C^\infty$-manifold, then
$u\in C^\infty(\overline{\Omega})$.
\end{cor}

\begin{proof}
Since $\displaystyle f \in H^{k,2}(\Omega)$ for all $k$
because all the derivatives of $f$ are bounded on $\overline{\Omega}$, 
we get from
Theorem~\ref{ell_regular} that $\displaystyle u \in H^{k,2}(\Omega)$
for all $k$.  It then follows from Corollary~\ref{sob_cor_sob_lem} that
$\displaystyle u \in C^k(\overline{\Omega})$ for all $k$.
\end{proof}

It follows from this corollary that eigenvectors of the strongly
elliptic operator $L(\VEC{x},\diff)$ are of class $\displaystyle C^\infty$ on
$\overline{\Omega}$.

\begin{cor}
Referring to Theorem~\ref{ell_regular}, if $u \in X$ is an eigenvector
of $B$ associated to the eigenvalue $\lambda$ (i.e.
$B(v,u) = \ps{v}{\lambda u}$ for all $v \in X$), then
$\displaystyle u \in C^\infty(\overline{\Omega})$.
\end{cor}

\begin{proof}
The conclusion of this corollary is a consequence of the previous
corollary applied to the variational problem
$B_1(v,u) = B(v,u) - \overline{\lambda}\ps{v}{u} = 0$ for $u,v \in X$.
\end{proof}

\section{Existence of Green's Functions} \label{ell_GreenF}

In this section, we finally prove Theorems~\ref{laplace_GreenExist}
and \ref{laplace_dirichlet2}.

\begin{prop} \label{ell_GreenF_prop}
Let $\Omega$ be a bounded open subset of $\displaystyle \RR^n$ with a boundary
$\partial \Omega$ of class $\displaystyle C^\infty$.  If
$\displaystyle f \in C^\infty(\overline{\Omega})$,
the solution of the Dirichlet problem $\Delta u = 0$ on $\Omega$ with
$u=f$ on $\partial \Omega$ is in $\displaystyle C^\infty(\overline{\Omega})$.
\end{prop}

\begin{proof}
The existence and uniqueness of the solution has been proved in
Theorem~\ref{pot_exist_uniqu_TH}.  Namely, there exists a unique
$u \in C(\overline{\Omega})$ such that $\Delta u = 0$ on $\Omega$ in
the sense of distributions and $u\big|_{\partial \Omega} = f$.

Let $\displaystyle w \in H^{1,2}_0(\Omega)$ be the weak solution of
$-\Delta w = \Delta f$ on $\Omega$ with $w=0$ on $\partial \Omega$.
This solution exists and is unique according to
Theorem~\ref{ell_exist_th1} because the variational formulation of the
Dirichlet problem $-\Delta w = \Delta f$ on $\Omega$ with $w=0$
on $\partial \Omega$ is
\[
B(w,v) = \int_\Omega \graD w(\VEC{x}) \cdot \graD v(\VEC{x})
\dx{\VEC{x}} = \int_\Omega v(\VEC{x}) \, \Delta f(\VEC{x})
\dx{\VEC{x}}
\]
for all $\displaystyle v \in H^{1,2}_0(\Omega)$,
where $B$ is a bounded, symmetric and coercive bilinear form on
$\displaystyle H^{1,2}_0(\Omega)$.
From Corollary~\ref{ell_fCi_uCi}, we have that
$\displaystyle w \in C^\infty(\overline{\Omega})$.  Moreover, $w=0$ on
$\partial \Omega$ according to Proposition~\ref{sob_w0_trad2}.

Since $\Delta (w+f) = 0$ on $\Omega$ and $w+f =f$ on
$\partial \Omega$, we have by uniqueness of solutions that
$\displaystyle u = w+f \in C^\infty(\overline{\Omega})$.
\end{proof}

\begin{proof}[Proof of Theorem~\ref{laplace_GreenExist}.]
Let $N$ be the fundamental solution of $\Delta$ given in
Theorem~\ref{laplace_fund_sol}.

\stage{i} For each $\VEC{x} \in \Omega$, choose
$\displaystyle \phi \in \DD(\RR^n)$ such that $\phi(\VEC{x})=0$ and
$\phi(\VEC{y}) = 1$ for all $\VEC{y} \in \partial \Omega$.  Let
\[
M(\VEC{y}) =
\begin{cases}
\phi(\VEC{y})N(\VEC{x}-\VEC{y}) & \quad \text{if} \ \VEC{y} \in
\RR^n\setminus \{\VEC{x}\} \\
0 & \quad \text{if} \ \VEC{y} = \VEC{x}
\end{cases}
\]
Let $u_{\VEC{x}}$ be the solution of the Dirichlet problem
$\Delta_{\VEC{y}} u = 0$ on $\Omega$ with $u = M$ on $\partial \Omega$.
According to Proposition~\ref{ell_GreenF_prop},
$\displaystyle u_{\VEC{x}} \in C^\infty(\overline{\Omega})$ since
$\displaystyle M \in C^\infty(\overline{\Omega})$.

We have shown that, for each $\VEC{x} \in \Omega$, the solution
$u_{\VEC{x}}$ of $\Delta_{\VEC{y}} u = 0$ on $\Omega$ with
$u(\VEC{y}) = N(\VEC{x}-\VEC{y})$
for $\VEC{y} \in \partial \Omega$ is
$\displaystyle C^\infty(\overline{\Omega})$.

\stage{ii} The Green function for $\Omega$ is
$\displaystyle G(\VEC{x},\VEC{y}) = N(\VEC{x}-\VEC{y})
- u_{\VEC{x}}(\VEC{y})$ for $\VEC{x} \in \Omega$ and
$\VEC{y} \in \overline{\Omega}$.
It is clear that
$\VEC{y} \mapsto G(\VEC{x},\VEC{y}) - N(\VEC{x}-\VEC{y}) =
u_{\VEC{x}}(\VEC{y})$ is harmonic on $\Omega$ and continuous on
$\overline{\Omega}$; it is in fact in
$\displaystyle C^\infty(\overline{\Omega})$.
Moreover,
$G(\VEC{x},\VEC{y}) = N(\VEC{x}-\VEC{y}) - u_{\VEC{x}}(\VEC{y}) = 0$
for $\VEC{y} \in \partial \Omega$ by construction.
\end{proof}

\begin{proof}[Proof of Theorem~\ref{laplace_dirichlet2}.]
According to Theorem~\ref{pot_exist_uniqu_TH}, there exists a unique
solution $u \in C(\overline{\Omega})$ of the Dirichlet problem
$\Delta u = 0$ on $\Omega$ in the sense of distributions with $u=g$
on $\partial \Omega$.
Moreover, from Corollary~\ref{ell_Hak_Cinfty},
$\displaystyle u \in C^\infty(\Omega)$.

We only have to show that
\[
u(\VEC{x}) = w(\VEC{x}) \equiv \int_{\partial \Omega} g(\VEC{y})\,
\pdydx{G_{\VEC{x}}}{\nu}(\VEC{y})\dss{S}{y}
\]
for all $\VEC{x} \in \Omega$.

From Weierstrass approximation theorem, we can construct a sequence
$\displaystyle \{ g_j\}_{j=1}^\infty$ of polynomials such that
$\displaystyle \sup_{\VEC{x} \in \partial \Omega}
|g_j(\VEC{x}) - g(\VEC{x})| \leq 1/j$ for all $j$.
Let $u_j$ be the solution of the Dirichlet problem
$\Delta u = 0$ in $\Omega$ with $u=g_j$ on $\partial \Omega$.

From Proposition~\ref{ell_GreenF_prop}, we have that
$\displaystyle u_j \in C^\infty(\overline{\Omega})$.  Hence, as it was
proved after Theorem~\ref{laplace_dirichlet2}, we have that
\begin{equation} \label{ell_THG2_equ}
u_j(\VEC{x}) = \int_{\partial \Omega} u_j(\VEC{y})\,
\pdydx{G_{\VEC{x}}}{\nu}(\VEC{y})\dss{S}{y}
= \int_{\partial \Omega} g_j(\VEC{y})\,
\pdydx{G_{\VEC{x}}}{\nu}(\VEC{y})\dss{S}{y}
\end{equation}
for all $\displaystyle \VEC{x} \in \Omega$.
If $j \rightarrow \infty$, the right side of (\ref{ell_THG2_equ})
converges to
\[
\int_{\partial \Omega} g(\VEC{y})\,
\pdydx{G_{\VEC{x}}}{\nu}(\VEC{y})\dss{S}{y}
\]
because $\displaystyle g_j(\VEC{y})\,\pdydx{G_{\VEC{x}}}{\nu}(\VEC{y})$
converges uniformly to $\displaystyle g(\VEC{y})\,
\pdydx{G_{\VEC{x}}}{\nu}(\VEC{y})$ on $\partial \Omega$.
For the left side of (\ref{ell_THG2_equ}), we apply the maximum
principle, Corollary~\ref{laplace_cHMP}, to $u_j-u$ to conclude that
\[
\sup_{\VEC{x}\in \overline{\Omega}} |u_j(\VEC{x}) - u(\VEC{x})| \leq
\sup_{\VEC{x}\in \partial \Omega} |g_j(\VEC{x}) - g(\VEC{x})|
\rightarrow 0
\quad \text{as} \quad j \rightarrow \infty \ .
\]
Thus, the left hand side of (\ref{ell_THG2_equ}) converges to
$u(\VEC{x})$.  This prove the Theorem.
\end{proof}

\section{Maximum Principle}

We have already seen in Theorem~\ref{laplace_HMP} a maximum principle
for the Laplace equation.  Our goal in this section is to generalize
if possible this theorem to a general family of elliptic partial differential
equations.  As we will see, the maximum principles that we obtain will
not always be as strong as the maximum principle for the Laplace
equation.  

We begin by presenting the strong maximum principle for a class of
elliptic partial differential equations on a domain $\Omega$ that
admit solutions in $\displaystyle C^2(\Omega) \cap C(\overline{\Omega})$.
After that, we present a slightly weaker maximum principle that
applies to a more general class of strongly elliptic partial
differential equations.   Though weaker, this principle is still very useful.

A good reference about maximum principles for partial differential
equations is \cite{ProWei}.  We follow the presentation in \cite{Smo}
for the strong maximum principle.

\subsection{Strong Maximum Principle}

All the theorems in this subsection form what is usually called the
strong maximum principle.  Several equivalent formulations are
presented in the literature.  Be aware that most of the formulations
of the strong maximum principle are formulated in terms of the
standard form of the linear elliptic differential operator
$L(\VEC{x},\diff)$, not the divergence form. Therefore, some of the
inequalities in the hypothesis of the next theorem may be in the other
direction than usually found.   It is sometime amazing what an extra
minus sign can do.

Before proving any theorem, we need a little result from linear
algebra.

\begin{prop}
Suppose that $A$ and $B$ are two \nn matrices, and $A$ is symmetric
and {\bfseries positive semi-definite}\index{Positive Semi-Definite};
namely, $A$ is symmetric and $\displaystyle \VEC{x}^\top A\VEC{x} \geq 0$ for
all $\displaystyle \VEC{x} \in \RR^n$.
If $B$ is positive semi-definite, then $\tr(AB) \geq 0$.
If $B$ is negative semi-definite, then $\tr(AB) \leq 0$.
\end{prop}

\begin{proof}
We prove the result for $B$ negative semi-definite and leave the
proof of the other result to the reader.

Since $A$ is symmetric, there exists an orthogonal matrix $P$ such that
$\displaystyle Q = P^\top A P$ is a diagonal matrix.  Since $A$ is positive
semi-definite, the elements $q_{i,i}$ on the diagonal of $Q$ (which
are the eigenvalues of $A$) are greater or equal to zero.

Let $\displaystyle C = P^\top B P$.  We have that $C$ is negative
semi-definite because 
$\displaystyle \VEC{x}^\top C \VEC{x} = (P\VEC{x})^\top B (P \VEC{x}) \leq 0$
for all $\displaystyle \VEC{x} \in \RR^n$.  Therefore
$\displaystyle c_{i,i} = \VEC{e}_i^\top C \VEC{e}_i \leq 0$
for $1\leq i \leq n$.

Hence,
\[
  \tr(AB) = \tr(P^\top AB P) = \tr\big( (P^\top A P) (P^\top B P)\big)
  = \tr(Q C) = \sum_{i=1}^n q_{i,i} c_{i,i} \leq 0 \ .
\]
The fact that $\displaystyle \tr(AB) = \tr(P^\top AB P)$ comes from
the fact that the trace of a \nn matrix is the sum of its eigenvalues,
and conjugated matrices have the same eigenvalues.
\end{proof}

The first theorem is not the must general result but it is the crux of
the proof of all the other results in this subsection.

\begin{theorem} \label{ellStrongMaxA}
Let $\Omega$ be a connected open subset of $\displaystyle \RR^n$ with a boundary
$\partial \Omega$ of class $\displaystyle C^2$.  Suppose that
\[
L(\VEC{x},\diff)u = - \sum_{i,j=1}^n
\diff_{x_i} \left( b_{i,j}(\VEC{x}) \diff_{x_j} u \right)
\]
is an elliptic operator, where
$\displaystyle b_{i,j} \in C^1(\overline{\Omega})$
for $1 \leq i,j\leq n$.
Let $\displaystyle u \in C^2(\Omega) \cap C(\overline{\Omega})$ be the
solution of $L(\VEC{x},\diff)u = f$, where $f \in C(\overline{\Omega})$.
\begin{enumerate}
\item If $f(\VEC{x}) \geq 0$ for all $\VEC{x} \in \overline{\Omega}$
and $\displaystyle \min_{\VEC{x}\in \overline{\Omega}} u(\VEC{x}) = u(\VEC{p})$
for some $\VEC{p} \in \Omega$, or
\item if $f(\VEC{x}) \leq 0$ for all $\VEC{x} \in \overline{\Omega}$
and $\displaystyle \max_{\VEC{x}\in \overline{\Omega}} u(\VEC{x}) = u(\VEC{p})$
for some $\VEC{p} \in \Omega$,
\end{enumerate}
then $u(\VEC{x})= u(\VEC{p})$ for all $\VEC{x} \in \Omega$.  In
particular, a non-constant function $u$ may only reach its minimum in
(1) or its maximum in (2) on $\partial \Omega$.
\end{theorem}

\begin{proof}
We have that
\begin{equation} \label{strongMaxAEq3}
L(\VEC{x},\diff)u = - \sum_{i,j=1}^n
\diff_{x_i} \left( b_{i,j}(\VEC{x}) \diff_{x_j} u \right)
= - \sum_{i,j=1}^n b_{i,j}(\VEC{x}) \diff_{x_i x_j} u
- \sum_{j=1}^n b_j(\VEC{x}) \diff_{x_j} u \ ,
\end{equation}
where $\displaystyle b_j = \sum_{i=1}^n \diff_{x_i} b_{i,j}$ for
$1 \leq i \leq n$.

We prove that (2) implies that $u(\VEC{x})= u(\VEC{p})$ for all
$\VEC{x} \in \Omega$ and leave the proof of the other implication to
the reader.  The proof of the other implication is similar.

Let $M = \left\{ \VEC{x} \in \Omega : u(\VEC{x}) = u(\VEC{p}) \right\}$.
If the conclusion is false, then $M$ is a proper subset of $\Omega$.
Moreover, $M$ is non-empty because $\VEC{p} \in M$.

The discussion of the next two paragraphs is summarized in
Figure~\ref{strongMax1}.

Choose $\VEC{q} \in \Omega \setminus M$.  Since $\Omega$ is connected,
we can select a continuous path $\gamma:[0,1] \to \Omega$ from
$\VEC{q}$ to $\VEC{p}$; in particular, $\gamma(0) = \VEC{q}$ and
$\gamma(1) = \VEC{p}$. 
Since $\gamma([0,1])$ is a compact subset of $\Omega$, we have that
\[
  \delta = \dist{\partial \Omega}{\gamma([0,1])}
= \sup\left\{ \|\VEC{x} - \gamma(t)\| : \VEC{x} \in \partial \Omega \
  \text{and} \ 0 \leq t \leq 1 \right\} > 0  \ .
\]
Since $u(\VEC{q}) < u(\VEC{p})$, there exists $0 < \delta_1 < \delta/2$
such that $u(\VEC{x}) < u(\VEC{p})$ for all $\VEC{x} \in B_{\delta_1}(\VEC{q})$.
Let $t_1$ be the smallest value of $t \in ]0,1]$ such that
$M \cap \partial B_{\delta_1}(\gamma(t_1)) \neq \emptyset$.
Let $\VEC{q}_1 = \gamma(t_1)$ and $B_1 = B_{\delta_1}(\VEC{q}_1)$.
Then, $\overline{B}_1 \subset \Omega$ with $B_1 \cap M = \emptyset$ and
$M \cap \partial B_1 = \{ \VEC{q}_2 \}$ for some
$\VEC{q}_2 \in \gamma([0,1])$.
Choose $\VEC{r} \in B_1$ and $\delta_2$ such that
$B_{\delta_2}(\VEC{r}) \subset B_1$ and
$\partial B_{\delta_2}(\VEC{r}) \cap M = \{ \VEC{q}_2 \}$.
Let $B_2 = B_{\delta_2}(\VEC{r})$.
Finally, choose $\delta_3 < \delta_2$ and set
$B_3 = B_{\delta_3}(\VEC{q}_2)$.

We have that $\partial B_3 = S_1 \cup S_2$, where
$S_1 = \overline{B}_2 \cap \partial B_3$ and $S_1 \cap S_2 = \emptyset$.
Since $u \in C(\Omega)$, $u(\VEC{x}) < u(\VEC{q}_2) = u(\VEC{p})$
for all $\VEC{x} \in S_1$ and $S_1$ is a compact set, there exists
$\epsilon >0$ such that $u(\VEC{x}) < u(\VEC{p}) - \epsilon$ for all
$\VEC{x} \in S_1$.

Let
\begin{equation} \label{strongMaxEq4}
  h(\VEC{x}) = e^{-\alpha \|\VEC{x} - \VEC{r}\|^2} - e^{-\alpha \delta_2^2} \ ,
\end{equation}
where $\alpha>0$ will be selected later.  We have that
\begin{equation} \label{strongMaxEq1}
\begin{split}
&e^{-\alpha \|\VEC{x} - \VEC{r}\|^2} L(\VEC{x},\diff) h (\VEC{x}) \\
&\qquad = - 4 \alpha^2 \sum_{i,j=1}^n b_{i,j}(\VEC{x}) (x_i - r_i)(x_j - r_j)
- 2 \alpha \sum_{j=1}^n \big(b_{j,j}(\VEC{x}) + b_j(\VEC{x}) (x_j - r_j)\big) \ .
\end{split}
\end{equation}
Since $\VEC{r} \not\in \overline{B}_3$ because $\delta_3 < \delta_2$, we
have that
$\displaystyle \sum_{i,j=1}^n b_{i,j}(\VEC{x}) (x_i - r_i)(x_j - r_j) > 0$
for all $\VEC{x} \in \overline{B}_3$ because
$L(\VEC{x},\diff)$ is elliptic on $\Omega$ \footnote{The reader should
review the definition of elliptic partial differential operator
that we have adopted for this chapter at the beginning of
Section~\ref{CSWSsection}.}.  It follows from
(\ref{strongMaxEq1}) and the compactness of $\overline{B}_3$ that 
\begin{equation} \label{strongMaxEq2}
  L(\VEC{x},\diff) h (\VEC{x})  < 0
\end{equation}
for all $\VEC{x} \in \overline{B}_3$ if we choose $\alpha$ large
enough.  Suppose that this has been done.

Let $\displaystyle m = \max_{\VEC{x} \in S_1} h(\VEC{x})$.  Note that
$m>0$ because $S_1 \cap B_2 \neq \emptyset$.
Choose $\eta < \epsilon/m$ and let $v = u + \eta h$.
We get that
\[
v(\VEC{x}) = u(\VEC{x}) + \eta h(\VEC{x})
< u(\VEC{p}) - \epsilon + \eta\, m < u(\VEC{p})
\]
for all $\VEC{x} \in S_1$.  Since $S_2 \cap \overline{B}_2 = \emptyset$,
we have that $\|\VEC{x} - \VEC{r}\|>\delta_2$ for all
$\VEC{x} \in S_2$.  Thus, $h(\VEC{x}) < 0$ for all $\VEC{x} \in S_2$ and
we get that
\[
v(\VEC{x}) = u(\VEC{x}) + \eta h(\VEC{x}) < u(\VEC{x}) \leq u(\VEC{p})
\]
for all $\VEC{x} \in S_2$.  Since $v(\VEC{x}) < u(\VEC{p})$ for
$\VEC{x} \in \partial B_3 = S_1 \cap S_2$ and
$v(\VEC{q}_2) = u(\VEC{q}_2) + \eta h(\VEC{q}_2) = u(\VEC{p})$, and
since this is true for any positive $\delta_3$ less than $\delta_2$,
we find that $v$ has a local maximum at $\VEC{q}_2$.  Therefore,
the gradient of $v$ is null and its Hessian matrix is negative
semi-definite at $\VEC{q}_2$; namely, $\nabla v(\VEC{q}_2) = \VEC{0}$ and 
$\displaystyle \sum_{i,j=1}^n \diff_{x_i x_j} v(\VEC{q}_2) y_i y_j \leq 0$
for all $\displaystyle \VEC{y} \in \RR^n$.  Moreover, since
$L(\VEC{x},\diff)$ is elliptic on $\Omega$, we also get
$\displaystyle \sum_{i,j=1}^n b_{i,j}(\VEC{q}_2) y_i y_j \geq 0$
for all $\displaystyle \VEC{y} \in \RR^n$.

Let $A$ and $\displaystyle B^\top$ be the matrices with entries
$b_{i,j}(\VEC{q}_2)$ and $-\diff_{x_i,x_j} v(\VEC{q}_2)$ respectively.
It follows from the previous proposition that
\begin{align*}
0 &\leq \tr(AB)
= - \sum_{i,j=1}^n b_{i,j}(\VEC{q}_2) \diff_{x_i,x_j} v(\VEC{q}_2)
= L(\VEC{x},\diff) v(\VEC{x})\big|_{\VEC{x} = \VEC{q}_2} \\
&\qquad = L(\VEC{x},\diff) u(\VEC{x})\big|_{\VEC{x} = \VEC{q}_2}
+ \eta L(\VEC{x},\diff) h(\VEC{x})\big|_{\VEC{x} = \VEC{q}_2}
< L(\VEC{x},\diff) u(\VEC{x})\big|_{\VEC{x} = \VEC{q}_2} \ ,
\end{align*}
where the second equality comes from $\nabla v(\VEC{q}_2) = \VEC{0}$
and the last inequality comes from (\ref{strongMaxEq2}).  This is a
contradiction of our original assumption that
$L(\VEC{x},\diff) u(\VEC{x}) = f(\VEC{x}) \leq 0$ for all $\VEC{x} \in \Omega$.
\end{proof}

\pdfF{elliptic/strong_max1}{Construction for the proof of
Theorem~\ref{ellStrongMaxA}}{Construction for the proof of
Theorem~\ref{ellStrongMaxA}}{strongMax1}

We now generalize the previous theorem.  However, the conclusion will
not be as strong as for the previous theorem.

\begin{theorem} \label{ellStrongMaxB}
Let $\Omega$ be a connected open subset of $\displaystyle \RR^n$ with
a boundary $\partial \Omega$ of class $\displaystyle C^2$.  Suppose that
\[
L(\VEC{x},\diff)u = - \sum_{i,j=1}^n
\diff_{x_i} \left( b_{i,j}(\VEC{x}) \diff_{x_j} u \right) + b_0(\VEC{x}) u
\]
be an elliptic operator, where
$\displaystyle b_{i,j} \in C^1(\overline{\Omega})$
for $1 \leq i,j\leq n$ and $b_0 \in C(\overline{\Omega})$.
Moreover, we assume that $b_0(\VEC{x}) \geq 0$ for all
$\VEC{x} \in \overline{\Omega}$.
Let $\displaystyle u \in C^2(\Omega) \cap C(\overline{\Omega})$ be the
solution of $L(\VEC{x},\diff)u = f$, where $f \in C(\overline{\Omega})$.
\begin{enumerate}
\item If $f(\VEC{x}) \geq 0$ for all $\VEC{x} \in \overline{\Omega}$ and
$\displaystyle \min_{\VEC{x}\in \overline{\Omega}} u(\VEC{x}) = u(\VEC{p}) < 0$
for some $\VEC{p} \in \Omega$, or
\item if $f(\VEC{x}) \leq 0$ for all $\VEC{x} \in \overline{\Omega}$ and
$\displaystyle \max_{\VEC{x}\in \overline{\Omega}} u(\VEC{x}) = u(\VEC{p}) > 0$)
for some $\VEC{p} \in \Omega$,
\end{enumerate}
then $u(\VEC{x})= u(\VEC{p})$ for all $\VEC{x} \in \Omega$.  In
particular, $u$ may only reach its minimum in (1) or its maximum in
(2) on $\partial \Omega$.
\end{theorem}

\begin{proof}
Let
\begin{equation} \label{strongMaxEq3}
H(\VEC{x},\diff)u = - \sum_{i,j=1}^n
\diff_{x_i} \left( b_{i,j}(\VEC{x}) \diff_{x_j} u \right)
= - \sum_{i,j=1}^n b_{i,j}(\VEC{x}) \diff_{x_i x_j} u
- \sum_{j=1}^n b_j(\VEC{x}) \diff_{x_j} u \ ,
\end{equation}
where $\displaystyle b_j = \sum_{i=1}^n \diff_{x_i} b_{i,j}$ for
$1 \leq i \leq n$.

We prove that (1) implies that $u(\VEC{x})= u(\VEC{p})$ for all
$\VEC{x} \in \Omega$ and leave the proof of the other implication to
the reader.

Let $M = \left\{ \VEC{x} \in \Omega : u(\VEC{x}) = u(\VEC{p}) \right\}$.
If the conclusion is false, then $M$ is a proper subset of $\Omega$.
Moreover, $M$ is non-empty because $\VEC{p} \in M$ and it is
closed in $\Omega$ because $u \in C(\overline{\Omega})$.

Given $\VEC{y} \in M$, since $u(\VEC{y}) = u(\VEC{p}) <0$ and $u$ is
continuous on $\Omega$, there exists
$\rho>0$ such $B_{\rho}(\VEC{y}) \subset \Omega$ and
$0 > u(\VEC{x}) \geq u(\VEC{p})$ for all $\VEC{x} \in B_{\rho}(\VEC{y})$.
Hence,
$H(x,\diff) u(\VEC{x}) = -b_0(\VEC{x}) u(\VEC{x}) + f(\VEC{x}) \geq 0$ 
for all $\VEC{x} \in B_{\rho}(\VEC{y})$.  We get from (1) of
Theorem~\ref{ellStrongMaxA} (with $H(\VEC{x},\diff)$ instead of
$L(\VEC{x},\diff)$ and $B_{\rho}(\VEC{y})$ instead of $\Omega$) that
$u(\VEC{x}) = u(\VEC{p})$ for all $\VEC{x} \in B_{\rho}(\VEC{z})$;
namely, $B_{\rho}(\VEC{y}) \subset M$.  Thus $M \neq \emptyset$ is
open and close in $\Omega$.  Since $\Omega$ is connected, we get that
$M = \Omega$.
\end{proof}

As expected, Theorem~\ref{ellStrongMaxB} can be used to prove the
uniquest of the solution for an elliptic partial differential equation.

\begin{cor}
Consider $\Omega$, $L(x,\diff)$ and $f$ defined in
Theorem~\ref{ellStrongMaxB}.  If
$\displaystyle u_1 , u_2 \in C^2(\Omega) \cap C(\overline{\Omega})$
satisfy $L(\VEC{x},\diff) u = f$ on $\Omega$ and $u_1 = u_2$ on
$\partial \Omega$, then $u_1 = u_2$ on $\Omega$.
\end{cor}

\begin{proof}
Since $\displaystyle u = u_1-u_2 \in C^2(\Omega) \cap C(\overline{\Omega})$
is a solution of $L(\VEC{x},\diff) u = 0$ on $\Omega$ with $u=0$ on
$\partial \Omega$, a simple case elimination using
Theorem~\ref{ellStrongMaxB} yields $u = 0$.
\end{proof}

The next theorems provide useful information for the Neumann problem. 

\begin{theorem} \label{dirDerMaxA}
Consider $\Omega$, $L(x,\diff)$ and $f$ defined in Theorem~\ref{ellStrongMaxA}.
Suppose that $\displaystyle u \in C^2(\Omega) \cap C(\overline{\Omega})$ is a
non-constant solution of $L(\VEC{x},\diff) u = f$ on $\Omega$.
Let $\mu(\VEC{x})$
be any outward-pointing unit vector at $\VEC{x} \in \partial \Omega$.
\begin{enumerate}
\item If $f(\VEC{x}) \geq 0$ for all $\VEC{x} \in \overline{\Omega}$
and $u$ reaches its minimum at $\VEC{p} \in \partial \Omega$,
then $\displaystyle \pdydx{u}{\mu} u(\VEC{p}) < 0$.
\item If $f(\VEC{x}) \leq 0$ for all $\VEC{x} \in \overline{\Omega}$
and $u$ reaches its maximum at $\VEC{p} \in \partial \Omega$,
then $\displaystyle \pdydx{u}{\mu} u(\VEC{p}) > 0$.
\end{enumerate}
\end{theorem}

\begin{proof}
There is a lot of similarity between the proof of this theorem and the
proof of the Theorem~\ref{ellStrongMaxA}.

We proof (2) and leave the proof (1) to the reader.  Since $u$ is not
constant, we get from (2) of Theorem~\ref{ellStrongMaxA}
that $u(\VEC{x}) < u(\VEC{p})$ for all $\VEC{x} \in \Omega$.

Since $\partial \Omega$ is of class $\displaystyle C^2$, we can find
$\VEC{r} \in \Omega$ and $\delta_1>0$ such that
$B_{\delta_1}(\VEC{r}) \subset \Omega$ and
$\partial B_{\delta_1}(\VEC{r}) \cap \partial \Omega = \{ \VEC{p} \}$
(Figure~\ref{strongMax2}).
Let $B_1 = B_{\delta_1}(\VEC{r})$.  Choose $\delta_2 < \delta_1$ and set
$B_2 = B_{\delta_2}(\VEC{p})$.

As in the proof of Theorem~\ref{ellStrongMaxA}, we consider $h$
defined in (\ref{strongMaxEq4}) and select $\alpha$ large enough to have
\begin{equation} \label{dirDerMaxAEq1}
L(\VEC{x},\diff) h (\VEC{x})  < 0
\end{equation}
for all $\VEC{x} \in \overline{B}_2$.

Let $S_1 = \overline{B_1} \cap \partial B_2$, and set
$\displaystyle m_1 = \max_{\VEC{x} \in S_1} h(\VEC{x})$
and $\displaystyle m_2 = \max_{\VEC{x} \in S_1} u(\VEC{x})$.
Since $h$ and $u$ are continuous on the compact set $S_1$, these
maximums are reached at some points of $S_1$.  Since $u(\VEC{x}) < u(\VEC{p})$
for all $\VEC{x} \in S_1 \subset B_1 \setminus \{\VEC{p}\}$, we have
that $m_2 < u(\VEC{p})$.  Moreover, $m_1>0$ because
$S_1 \cap B_1 \neq \emptyset$.
Choose $0< \eta < (u(\VEC{p})- m_2)/m_1$ and let $v = u + \eta h$.
Then, $v(\VEC{x}) = u(\VEC{x}) + \eta h(\VEC{x}) < u(\VEC{p}) = v(\VEC{p})$
for all $\VEC{x} \in S_1$.

Let $S_2 = B_2 \cap \partial B_1$.  Since $h(\VEC{x}) = 0$ for all
$\VEC{x} \in S_2$, we have that
$v(\VEC{x}) = u(\VEC{x}) < u(\VEC{p}) = v(\VEC{p})$
for all $\VEC{x} \in S_2\setminus \{\VEC{p}\}$ because
$u(\VEC{x}) < u(\VEC{p})$ for all $\VEC{x} \in \Omega$.

We have shown that $v(\VEC{x}) \leq v(\VEC{p})$ 
for all $\VEC{x} \in \partial (B_1 \cap B_2) = S_1 \cup S_2$.
Moreover, we get from (\ref{dirDerMaxAEq1}) that
\[
  L(\VEC{x},\diff) v (\VEC{x}) = L(\VEC{x},\diff) u(\VEC{x})
  + \eta L(\VEC{x},\diff) h(\VEC{x}) < 0
\]
for all $\VEC{x} \in B_1 \cap B_2$.  Hence, we get
from (2) of Theorem~\ref{ellStrongMaxA} (with $B_1 \cap B_2$ instead
of $\Omega$) that $v$ has an isolate maximum at $\VEC{p}$ because $v$
is not constant on $B_1 \cap B_2$.  Therefore
\begin{equation} \label{dirDerMaxAEq2}
0 \leq \pdydx{v}{\mu}(\VEC{p}) 
= \pdydx{u}{\mu}(\VEC{p}) + \eta \pdydx{h}{\mu}(\VEC{p}) \ .
\end{equation}
However, if
$\mu(\VEC{p}) = \nu(\VEC{p}) = \|\VEC{p}-\VEC{r}\|^{-1} (\VEC{p} - \VEC{r})$,
the outward unit normal to $\partial (B_1 \cap B_2)$ at $\VEC{p}$, then
\begin{align*}
\pdydx{h}{\mu}(\VEC{p}) &= \nabla h(\VEC{p}) \cdot \nu(\VEC{p})
= \sum_{i=1}^n \left( -2 \alpha (p_i-r_i) e^{-\alpha\|\VEC{p}-\VEC{r}\|^2} \right)
\left( \frac{p_i- r_i}{\|\VEC{p} - \VEC{r}\|} \right) \\
&= \sum_{i=1}^n \left( -2 \alpha \frac{(p_i-r_i)^2}{\|\VEC{p} - \VEC{r}\|}
e^{-\alpha\|\VEC{p}-\VEC{r}\|^2} \right)
= -2 \alpha \|\VEC{p}-\VEC{r}\| e^{-\alpha\|\VEC{p}-\VEC{r}\|^2}
 < 0 \ .
\end{align*}
Therefore, we get $\displaystyle 0 < \pdydx{u}{\mu}(\VEC{p})$ from
(\ref{dirDerMaxAEq2}).

If $\mu(\VEC{p})$ is any outward-pointing unit vector to $\Omega$ at
$\VEC{p} \in \partial \Omega$, then $\mu(\VEC{p}) \cdot \nu(\VEC{p}) > 0$.
Hence, $\sgn \nabla h(\VEC{p}) \cdot \mu(\VEC{p})
= \sgn \nabla h(\VEC{p}) \cdot \nu(\VEC{p}) > 0$.
\end{proof}

\pdfF{elliptic/strong_max2}{Construction for the proof of
Theorem~\ref{dirDerMaxA}}{Construction for the proof of
Theorem~\ref{dirDerMaxA}}{strongMax2}

\begin{theorem}
Consider $\Omega$, $L(x,\diff)$ and $f$ defined in Theorem~\ref{ellStrongMaxB}.
Suppose that $\displaystyle u \in C^2(\Omega) \cap C(\overline{\Omega})$ is a
non-constant solution of $L(\VEC{x},\diff) u = f$ on $\Omega$.
Let $\mu(\VEC{x})$ be any outward-pointing unit vector at
$\VEC{x} \in \partial \Omega$. 
\begin{enumerate}
\item If $f(\VEC{x}) \geq 0$ for all $\VEC{x} \in \overline{\Omega}$
and $u$ reaches it negative minimum at $\VEC{p} \in \partial \Omega$,
then $\displaystyle \pdydx{u}{\mu} u(\VEC{p}) < 0$.
\item If $f(\VEC{x}) \leq 0$ for all $\VEC{x} \in \overline{\Omega}$
and $u$ reaches it positive maximum at $\VEC{p} \in \partial \Omega$,
then $\displaystyle \pdydx{u}{\mu} u(\VEC{p}) > 0$.
\end{enumerate}
\end{theorem}

\begin{proof}
We prove the case (2) and leave the proof of the other case to the reader.

Let
\[
H(\VEC{x},\diff)u = - \sum_{i,j=1}^n
\diff_{x_i} \left( b_{i,j}(\VEC{x}) \diff_{x_j} u \right)
= - \sum_{i,j=1}^n b_{i,j}(\VEC{x}) \diff_{x_i x_j} u
- \sum_{j=1}^n b_j(\VEC{x}) \diff_{x_j} u \ ,
\]
where $\displaystyle b_j = \sum_{i=1}^n \diff_{x_i} b_{i,j}$ for
$1 \leq i \leq n$.

Since $u$ is not constant, we get from (2) of Theorem~\ref{ellStrongMaxB}
that $u(\VEC{x}) < u(\VEC{p})$ for all $\VEC{x} \in \Omega$.

Since $\partial \Omega$ is of class $\displaystyle C^2$, we can select a ball
$B \subset \Omega$ such that $\overline{B} \cap \partial \Omega = \{\VEC{p}\}$.
In particular, $\partial B$ is tangent to $\partial \Omega$ at $\VEC{p}$.

Since $u(\VEC{p}) > 0$ and $u \in C(\overline{\Omega})$, we may assume
that $B$ is small enough to have $u(\VEC{x}) \geq 0$ for all
$\VEC{x} \in B$.  Hence,
\[
H(\VEC{x},\diff) u(\VEC{x}) = -b_0(\VEC{x}) u(\VEC{x}) + f(\VEC{x}) \leq 0
\]
for all $\VEC{x} \in B$.  It follows from Theorem~\ref{dirDerMaxA}
that $\displaystyle \pdydx{u}{\mu}(\VEC{p}) > 0$.
\end{proof}

The fact that $\displaystyle \pdydx{u}{\nu} u(\VEC{x}) \leq 0$ 
at a point $\VEC{p} \in \partial \Omega$ where $u$ reach a minimum or
that $\displaystyle \pdydx{u}{\nu} u(\VEC{x}) \geq 0$ 
at a point $\VEC{p} \in \partial \Omega$ where $u$ reach a maximum is
not surprising and can be deduced from basic calculus.  The strict
inequality is more surprising.

\subsection{General Maximum Principle}

We have a maximum principle for the second order coercive elliptic
partial differential equations if the coefficients are not smooth
functions as we had in the previous subsection.  We begin with maximum
principle for the Dirichlet problem.  As in the previous section, we
consider in this section only real valued functions.

Our maximum principle for the Dirichlet problem will be deduced from
the following proposition.

\begin{prop} \label{ell_max_DP}
Let $\Omega$ be an open subset of $\displaystyle \RR^n$,  Suppose that
$\displaystyle f\in L^2(\Omega)$ and that
$\displaystyle u\in H^{1,2}(\Omega)\cap C(\overline{\Omega})$ is a solution of
\begin{equation} \label{ell_max_eq1}
B(v,u) = \int_\Omega \left( \graD u \cdot \graD v + u\,v \right) \dx{\VEC{x}}
= \int_\Omega f\,v \dx{\VEC{x}}
\end{equation}
for all $\displaystyle v \in H^{1,2}_0(\Omega)$.  Then
\[
\min\left\{ \inf_{\VEC{x}\in \partial \Omega} u(\VEC{x}),
\inf_{\VEC{x}\in \Omega} f(\VEC{x}) \right\} \leq u(\VEC{x}) \leq
\max\left\{ \sup_{\VEC{x}\in \partial \Omega} u(\VEC{x}),
\sup_{\VEC{x}\in \Omega} f(\VEC{x}) \right\}
\]
for all $\VEC{x} \in \Omega$.
\end{prop}

\begin{proof}
\stage{i} We first prove the second inequality.  Let
\[
K = \max\left\{ \sup_{\VEC{x}\in \partial \Omega} u(\VEC{x}),
\sup_{\VEC{x}\in \Omega} f(\VEC{x}) \right\} \ .
\]
If $K=\infty$, we are done.  Suppose that $K<\infty$.
We choose a function $\displaystyle g\in C^1(\RR)$ such that
$0< g'(x) < 1$ for all $x\in ]0,\infty[$ and $g(x)=0$ for all
$x\in]-\infty,0]$.  Let
\begin{align*}
v: \overline{\Omega} & \rightarrow \RR \\
\VEC{x} &\mapsto g(u(\VEC{x})-K)
\end{align*}

\stage{i.a}
Suppose that $\Omega$ is bounded.  Then
$\displaystyle v \in L^2(\Omega) \cap L^1(\Omega)$ because
$v$ is continuous on the compact set $\overline{\Omega}$.  Moreover,
for any multi-index $\VEC{\alpha}$ with $|\VEC{\alpha}|=1$, we have
$\displaystyle \diff^{\VEC{\alpha}} v = g'(u-K) \diff^{\VEC{\alpha}} u$
in the sense of distributions with $|g'|\leq 1$ and
$\displaystyle \diff^{\VEC{\alpha}} u\in L^2(\Omega)$
because $\displaystyle u \in H^{1,2}(\Omega)$.  Thus,
$\displaystyle \diff^{\VEC{\alpha}} v \in L^2(\Omega)$.
Therefore, $\displaystyle v\in H^{1,2}(\Omega)$.

Since $\displaystyle v\in H^{1,2}(\Omega) \cap C(\overline{\Omega})$ and
$v(\VEC{x}) = g(u(\VEC{x})-K) = 0$ for $\VEC{x}\in \partial \Omega$ by
construction, we get from Proposition~\ref{sob_w0_trad2} that
$\displaystyle v\in H^{1,2}_0(\Omega)$.

With this choice of $v$ in (\ref{ell_max_eq1}), we get
\[
\int_\Omega \left( (\graD u \cdot \graD u ) g'(u-K) 
+ u \, g(u-K) \right) \dx{\VEC{x}} = \int_\Omega f\,g(u-K)
\dx{\VEC{x}} \ .
\]
Since $\displaystyle v \in L^1(\Omega)$, we may subtract
$\displaystyle K \int_{\Omega} g(u-K) \dx{\VEC{x}}$ from each side of
the previous equation to get
\[
\int_\Omega \left( \|\graD u \|_2^2\, g'(u-K) 
+ (u-K) \, g(u-K) \right) \dx{\VEC{x}} = \int_\Omega (f-K)\,g(u-K)
\dx{\VEC{x}} \ .
\]
Since $\displaystyle \|\graD u \|_2^2\, g'(u-K)\geq 0$ and
$(f-K) \, g(u-K)\leq 0$ almost everywhere on $\Omega$, we get
\[
\int_\Omega (u-K) \, g(u-K) \dx{\VEC{x}} \leq 0 \ .
\]
However $(u(\VEC{x})-K) \, g(u(\VEC{x})-K) > 0$ for
$\VEC{x} \in \Omega$ such that $u(\VEC{x})>K$ and
$(u(\VEC{x})-K) \, g(u(\VEC{x})-K) = 0$ for
$\VEC{x} \in \Omega$ such that $u(\VEC{x})\leq K$.  Therefore,
$(u(\VEC{x})-K) \, g(u(\VEC{x})-K) = 0$ for all $\VEC{x}\in \Omega$
because the function defined by
$\VEC{x} \mapsto (u(\VEC{x}) - K) g(u(\VEC{x}) -K)$ is
continuous on $\Omega$.
Thus $u(\VEC{x}) \leq K$ for all $\VEC{x}\in \Omega$.

\stage{i.b}
If $\Omega$ is not bounded, we must have $K\geq 0$ because
$f(\VEC{x}) \leq K<0$ for almost all $\VEC{x}$ and
$\Omega$ not bounded implies that $\displaystyle f \not\in L^2(\Omega)$.
Choose $K_0 > K$ and let
\begin{align*}
v: \overline{\Omega} & \rightarrow \RR \\
\VEC{x} &\mapsto g(u(\VEC{x})-K_0)
\end{align*}
for $g$ defined above.  To prove that $\displaystyle v\in H^{1,2}_0(\Omega)$, we
note that for every $x \in \RR$, there exists $y$ between $x-K_0$ and
$-K_0$ such that
\[
\left|g(x-K_0)\right| = \left|g(x-K_0) - g(-K_0)\right| =
|g'(y)|\,|x| \leq |x| \ .
\]
Hence $|g(u-K_0)| \leq |u|$.  Since $\displaystyle u \in L^2(\Omega)$,
we have that $\displaystyle v = g(u-K_0) \in L^2(\Omega)$.  The proof that
$\displaystyle \diff^{\VEC{\alpha}} v \in L^2(\Omega)$ for the multi-index
$\VEC{\alpha}$ such that $|\VEC{\alpha}|=1$ is identical to the proof
given in (i.a).
As in (i.a), $\displaystyle v\in H^{1,2}(\Omega) \cap C(\overline{\Omega})$ and
$v(\VEC{x}) = g(u(\VEC{x})-K_0) = 0$ for $\VEC{x}\in \partial \Omega$
implies that $\displaystyle v\in H^{1,2}_0(\Omega)$.

With this choice of $v$ in (\ref{ell_max_eq1}), we get
\begin{equation} \label{ell_max_eq2}
\int_\Omega \left( (\graD u \cdot \graD u ) g'(u-K_0) 
+ u \, g(u-K_0) \right) \dx{\VEC{x}} = \int_\Omega f\,g(u-K_0)
\dx{\VEC{x}} \ .
\end{equation}
Moreover, it follows from $|g(u-K_0)| \leq |u|$ that
$v \in L^1(\Omega)$ as we now prove.  Let
$\displaystyle S = \left\{ \VEC{x} : u(\VEC{x}) > K_0 \right\}$.  We
have that
\[
\int_\Omega \left|g(u-K_0)\right| \dx{\VEC{x}}
= \int_S \left|g(u-K_0)\right| \dx{\VEC{x}}
\leq \int_S \left|u\right| \dx{\VEC{x}}
\leq \int_S \frac{1}{K_0} \left|u\right|^2 \dx{\VEC{x}}
\leq \frac{1}{K_0} \int_\Omega \left|u\right|^2 \dx{\VEC{x}} < \infty
\]
because $u \in H^{1,2}(\Omega)$ and
$u^2(\VEC{x})\geq K_0 \left|u(\VEC{x})\right|$ for all $\VEC{x} \in S$.
It is here that we need $K_0>K\geq 0$ to get $K_0 \neq 0$.

Since $\displaystyle K_0 \int_\Omega g(u-K_0) \dx{\VEC{x}} \in \RR$, we may
subtract it from each side of the equality in (\ref{ell_max_eq2}) to get
\[
\int_\Omega \left( \|\graD u\|_2^2 g'(u-K_0) 
+ (u-K_0) \, g(u-K_0) \right) \dx{\VEC{x}} = \int_\Omega (f-K_0)\,g(u-K_0)
\dx{\VEC{x}} \ .
\]
As we have done in (i-a), we may use
$(u(\VEC{x})-K_0) \, g(u(\VEC{x})-K_0) > 0$ for
$\VEC{x} \in \Omega$ such that $u(\VEC{x})>K_0$ and
$(u(\VEC{x})-K_0) \, g(u(\VEC{x})-K_0) = 0$ for
$\VEC{x} \in \Omega$ such that $u(\VEC{x})\leq K_0$ to conclude that
$u(\VEC{x}) \leq K_0$ for all $\VEC{x}\in \Omega$.  Since $K_0>K$ is
arbitrary, we get $u(\VEC{x}) \leq K$ for all $\VEC{x}\in \Omega$

\stage{ii} To prove the first inequality in the conclusion of the
theorem, it suffices to consider $-u$.
\end{proof}

The next theorem is similar to Theorem~\ref{ellStrongMaxB} but it is
not as strong.  In particular, it does not rule out the possibility
that the maximum be also reached inside the set $\Omega$.

\begin{theorem}
Let $\Omega$ be an open subset of $\RR^n$ and
\[
L(\VEC{x},\diff)u = - \sum_{i,j=1}^n
\diff_{x_i} \left( b_{i,j}(\VEC{x}) \diff_{x_j} u \right)
+ b_0(\VEC{x}) u
\]
be a strongly elliptic operator.  We assume that
$\displaystyle b_{i,j} \in H^{1,\infty}(\Omega)$ for all $i$ and $j$,
that $\displaystyle b_j \equiv \sum_{i=1}^n \diff_{x_i} b_{i,j} = 0$
almost everywhere on $\Omega$ for all $j$, and that
$\displaystyle b_0 \in L^\infty(\Omega)$
satisfies $b_0 \geq 0$ almost everywhere on $\Omega$.  Suppose that
$\displaystyle f\in L^2(\Omega)$ and that
$\displaystyle u\in H^{1,2}(\Omega)\cap C(\overline{\Omega})$ is a
weak solution of $L(\VEC{x},\diff)u=f$; namely, a solution of
\begin{equation} \label{ell_max_eq3}
B(v,u) = \sum_{i,j}^n
\int_\Omega b_{i,j} \diff_{x_j} u \, \diff_{x_i} v \dx{\VEC{x}} 
+ \int_\Omega b_0\, u \, v \dx{\VEC{x}} = \int_\Omega f\,v \dx{\VEC{x}}
\end{equation}
for all $\displaystyle v \in H^{1,2}_0(\Omega)$.  Then
\begin{enumerate}
\item $u\geq 0$ on $\partial \Omega$ and $f\geq 0$ almost everywhere on
$\Omega$ implies that $u\geq 0$ on $\Omega$.
\end{enumerate}
Moreover, if $b_0 = 0$ almost everywhere on $\Omega$ and $\Omega$ is
bounded, then
\begin{enumerate}
\setcounter{enumi}{1}
\item $f\geq 0$ almost everywhere on $\Omega$ implies that
$\displaystyle u(\VEC{x}) \geq \inf_{\VEC{y}\in \partial \Omega} u(\VEC{y})$
for all $\VEC{x} \in \Omega$, and
\item $f\leq 0$ almost everywhere on $\Omega$ implies that
$\displaystyle u(\VEC{x}) \leq \sup_{\VEC{y}\in \partial \Omega} u(\VEC{y})$
for all $\VEC{x} \in \Omega$, and
\item $f= 0$ almost everywhere on $\Omega$ implies that
$\displaystyle \inf_{\VEC{y}\in \partial \Omega} u(\VEC{y}) \leq u(\VEC{x}) \leq
\sup_{\VEC{y}\in \partial \Omega} u(\VEC{y})$ for all $\VEC{x} \in
\Omega$.
\end{enumerate}
\end{theorem}

\begin{proof}
The variational formulation of our elliptic partial differential
equation is given by (\ref{ell_max_eq3}) because we assume that
$\displaystyle b_j \equiv \sum_{i=1}^n \diff_{x_i} b_{i,j} = 0$
almost everywhere on $\Omega$ for all $j$.  According to \cite{Br},
there is version of the present theorem with
$b_j$ not almost everywhere $0$ but the prove is much harder
than the present proof.

\stage{1}  Let $\displaystyle g\in C^1(\RR)$ be the function defined
in the proof of Proposition~\ref{ell_max_DP}.  We define
$\displaystyle h\in C^1(\RR)$ by $h(x) = -g(-x)$
for $x\in \RR$.  As in the proof of Proposition~\ref{ell_max_DP}, we have
that $\displaystyle v=h(u) \in H^{1,2}_0(\Omega)$.  If we substitute
$v=h(u)$ in (\ref{ell_max_eq3}), we get
\[
\sum_{i,j=1}^n
\int_\Omega b_{i,j} \, \diff_{x_j} u \, \diff_{x_i} u \,
h'(u) \dx{\VEC{x}} + \int_\Omega b_0 \, u \, h(u) \dx{\VEC{x}} 
= \int_\Omega f\,h(u) \dx{\VEC{x}} \ .
\]
Note that $f\,h(u) \leq 0$ and
$b_0 \, u \, h(u)\geq 0$ almost everywhere in $\Omega$ because
$h(u(\VEC{x})) = 0$ for $\VEC{x}$ such that $u(\VEC{x})>0$ and
$h(u(\VEC{x})) < 0$ for $\VEC{x}$ such that $u(\VEC{x}) < 0$ by
construction.  Thus
\[
\sum_{i,j=1}^n
\int_\Omega b_{i,j} \, \diff_{x_j} u \, \diff_{x_i} u \,
h'(u) \dx{\VEC{x}} \leq 0 \ .
\]
Since the partial differential equation is strongly elliptic, and
$h'(x) = g'(-x)>0$ for $x<0$ and $h'(x) = 0$ for $x \geq 0$, there
exists $\theta >0$ such that
\[
\sum_{i,j=1}^n
\int_\Omega b_{i,j} \, \diff_{x_j} u \, \diff_{x_i} u \, h'(u) \dx{\VEC{x}}
\geq \theta \int_\Omega \|\graD u\|_2^2\, h'(u) \dx{\VEC{x}} \  .
\]
Hence,
\[
\int_\Omega \|\graD u\|_2^2\, h'(u) \dx{\VEC{x}} \leq 0 \ .
\]
We get $\|\graD u\|_2^2\, h'(u) =0$ almost everywhere on $\Omega$.

Let
\begin{equation} \label{ell_max_thPr}
E(t) = \int_0^t \sqrt{h'(s)} \dx{s} \ .
\end{equation}
$E\circ u \in C(\overline{\Omega})$ because $u\in C(\overline{\Omega})$.
Moreover, $E(u(\VEC{x})) = 0$ for $\VEC{x} \in \partial \Omega$ because
$u(\VEC{x}) \geq 0$ for $\VEC{x} \in \partial \Omega$ and
$h'(s)=0$ for $s>0$.  Since
$\displaystyle \|\graD E(u(\VEC{x})\|_2^2 =
\|\graD u(\VEC{x})\|_2^2\, h'(u(\VEC{x})) = 0$ almost everywhere
on $\Omega$, we have that $E\circ u$ is constant on $\overline{\Omega}$.
Thus, $E\circ u$ constant on $\overline{\Omega}$ and null on
$\partial \Omega$ imply that
$E(u(\VEC{x})) =0$ for all $\VEC{x} \in \overline{\Omega}$.  This implies
that $u(\VEC{x}) \geq 0$ for all $\VEC{x}\in \Omega$ because
$u(\VEC{x})<0$ for some $\VEC{x}$ implies that
$E(u(\VEC{x})) >0$ since $h'(s) > 0$ for $s<0$.

\stage{2} Let $\displaystyle K = \inf_{\VEC{x}\in \partial \Omega} u(\VEC{x})$ and
consider the function $u-K$.  Since $\Omega$ is bounded, we have that
$\displaystyle u-K \in H^{1,2}(\Omega)$.  Moreover, since $b_0 = 0$ almost
everywhere on $\Omega$, we have that $u-K$ satisfies
(\ref{ell_max_eq3}).  Since $u-K\geq 0$ on $\partial \Omega$, we may
used the first part of the theorem to conclude that $u-K\geq 0$ on
$\Omega$.

\stage{3} The proof is very similar to the proof (2). Let
$\displaystyle K = \max_{\VEC{x}\in \partial \Omega} u(\VEC{x})$ and
consider the function $K-u$.  Since $\Omega$ is bounded, we have that
$\displaystyle K-u \in H^{1,2}(\Omega)$.  Moreover, since $b_0 = 0$ almost
everywhere on $\Omega$, we have that $K-u$ satisfies
(\ref{ell_max_eq3}) with $f$ replaced by $-f$.  Since
$K-u\geq 0$ on $\partial \Omega$, we may
used the first part of the theorem (with $f$ replaced by $-f$ and $u$
by $K-u$) to conclude that $K-u\geq 0$ on $\Omega$.

\stage{4} This follows from (2) and (3).
\end{proof}

We have the following maximum principle for the Neumann problem.

\begin{prop} \label{ell_max_NP}
Let $\Omega$ be an open subset of $\displaystyle \RR^n$,  Suppose that
$\displaystyle f\in L^2(\Omega)$ and that $\displaystyle u\in H^{1,2}(\Omega)$
is a solution of 
\[
B(v,u) =
\int_\Omega \left( \graD u \cdot \graD v +
u\,v \right) \dx{\VEC{x}} = \int_\Omega f\,v \dx{\VEC{x}}
\]
for all $\displaystyle v \in H^{1,2}(\Omega)$.  Then
\[
\inf_{\VEC{x}\in \Omega} f(\VEC{x}) \leq u \leq
\sup_{\VEC{x}\in \Omega} f(\VEC{x})
\]
almost everywhere in $\Omega$.
\end{prop}

\begin{proof}
The proof is similar to the proof of Proposition~\ref{ell_max_DP}.
\end{proof}

\section{Addendum}

We have present some examples of the variational method to solve
elliptic partial differential equations in the previous sections.
More examples can be found in \cite{Br,Sal}.  The presentation of the
variational method in \cite{Sal} is slightly different the our
presentation.  In particular, they do not use Stampacchia's theorem,
Theorem~\ref{fu_an_stamp}.  Nevertheless, they have many interesting
examples with their physical motivation and interpretation.  This is
an excellent reference for a reader interested in the applications of
partial differential equations.

\section{Exercises}

Suggested exercises:

\begin{itemize}
\item In \cite{Br}: number 8.23 to 8.42 in Chapter 8; problems 45 to
51 in the chapter Problems.
\item In \cite{McO}: number 4 in Section 8.2;
number 2 in Section 8.3. 
\end{itemize}

%%% Local Variables: 
%%% mode: latex
%%% TeX-master: "notes"
%%% End: 
